\documentclass[12pt]{article}\usepackage[]{graphicx}\usepackage[]{color}
%% maxwidth is the original width if it is less than linewidth
%% otherwise use linewidth (to make sure the graphics do not exceed the margin)
\makeatletter
\def\maxwidth{ %
  \ifdim\Gin@nat@width>\linewidth
    \linewidth
  \else
    \Gin@nat@width
  \fi
}
\makeatother

\definecolor{fgcolor}{rgb}{0.345, 0.345, 0.345}
\newcommand{\hlnum}[1]{\textcolor[rgb]{0.686,0.059,0.569}{#1}}%
\newcommand{\hlstr}[1]{\textcolor[rgb]{0.192,0.494,0.8}{#1}}%
\newcommand{\hlcom}[1]{\textcolor[rgb]{0.678,0.584,0.686}{\textit{#1}}}%
\newcommand{\hlopt}[1]{\textcolor[rgb]{0,0,0}{#1}}%
\newcommand{\hlstd}[1]{\textcolor[rgb]{0.345,0.345,0.345}{#1}}%
\newcommand{\hlkwa}[1]{\textcolor[rgb]{0.161,0.373,0.58}{\textbf{#1}}}%
\newcommand{\hlkwb}[1]{\textcolor[rgb]{0.69,0.353,0.396}{#1}}%
\newcommand{\hlkwc}[1]{\textcolor[rgb]{0.333,0.667,0.333}{#1}}%
\newcommand{\hlkwd}[1]{\textcolor[rgb]{0.737,0.353,0.396}{\textbf{#1}}}%
\let\hlipl\hlkwb

\usepackage{framed}
\makeatletter
\newenvironment{kframe}{%
 \def\at@end@of@kframe{}%
 \ifinner\ifhmode%
  \def\at@end@of@kframe{\end{minipage}}%
  \begin{minipage}{\columnwidth}%
 \fi\fi%
 \def\FrameCommand##1{\hskip\@totalleftmargin \hskip-\fboxsep
 \colorbox{shadecolor}{##1}\hskip-\fboxsep
     % There is no \\@totalrightmargin, so:
     \hskip-\linewidth \hskip-\@totalleftmargin \hskip\columnwidth}%
 \MakeFramed {\advance\hsize-\width
   \@totalleftmargin\z@ \linewidth\hsize
   \@setminipage}}%
 {\par\unskip\endMakeFramed%
 \at@end@of@kframe}
\makeatother

\definecolor{shadecolor}{rgb}{.97, .97, .97}
\definecolor{messagecolor}{rgb}{0, 0, 0}
\definecolor{warningcolor}{rgb}{1, 0, 1}
\definecolor{errorcolor}{rgb}{1, 0, 0}
\newenvironment{knitrout}{}{} % an empty environment to be redefined in TeX

\usepackage{alltt}  
% \usepackage[sc]{mathpazo}
% \usepackage[T1]{fontenc}
\usepackage{pifont} 
\usepackage{url}
\usepackage{breakurl}
\usepackage[colorlinks = true,
            linkcolor = blue,
            urlcolor  = red!80!blue,
            citecolor = blue!80!black,
            anchorcolor = blue]{hyperref}
\usepackage{soul}
\usepackage{paralist}
\usepackage{bm}
\usepackage[round]{natbib}
\usepackage{graphicx}
\usepackage{amsmath, wrapfig,amssymb,multirow}
\usepackage[margin=1in]{geometry}
%%%Gives a continuous ordering of figures.  Comment out to get
%within section numbering of figures
\usepackage{chngcntr}
\counterwithout{figure}{section}
\usepackage[parfill]{parskip}
\usepackage{epsfig, subfigure}
\usepackage{amsfonts}
\usepackage{authblk}
\renewcommand\Affilfont{\footnotesize}
\usepackage{booktabs}
\usepackage{bigstrut}
\usepackage{tabularx}
\usepackage{threeparttable} 
\usepackage[format=hang,labelfont=bf]{caption}
\usepackage{float}
\floatstyle{boxed}
\usepackage{footnote}
\usepackage{booktabs}
\usepackage{longtable}
\usepackage{array}
\usepackage{multirow}
\usepackage[table]{xcolor}
\usepackage{wrapfig}
\usepackage{float}
\usepackage{colortbl}
\usepackage{pdflscape}
\usepackage{tabu}
\usepackage{threeparttable}
\makesavenoteenv{tabular}
\makesavenoteenv{table}
%%----------------------------------------- customized fonts
\newcommand\code{\bgroup\@makeother\_\@makeother\~\@makeother\$\@codex}
\def\@codex#1{{\normalfont\ttfamily\hyphenchar\font=-1 #1}\egroup}
\let\code=\texttt
\let\proglang=\textsf
\newcommand{\pkg}[1]{{\fontseries{b}\selectfont #1}}
\newcommand{\tm}[1]{\textcolor{blue}{\textit{Tyler: #1}}}
\newcommand{\sjc}[1]{\textcolor{red}{\textit{Sam: #1}}}
\definecolor{light-gray}{gray}{0.75}
\definecolor{orange}{RGB}{255,127,0}
\newcommand{\zl}[1]{\textcolor{orange}{\textit{Richard: #1}}}
\newcommand{\todo}[1]{\textbf{To-do list:} #1}
\newcommand{\blue}[1]{\textcolor{blue}{#1}}
\newcommand{\orange}[1]{\textcolor{orange}{#1}}
\newcommand{\ok}{\nonumber}
\usepackage{xspace} 
\newcommand{\vapkg}{\pkg{openVA}\xspace}
\newcommand{\tmark}{\text{\ding{51}}}
\newcommand{\cmark}{\text{\ding{55}}}
%%----------------------------------------- Increase the separation
\let\tempone\itemize
\let\temptwo\enditemize
\renewenvironment{itemize}{\tempone\addtolength{\itemsep}{0.5\baselineskip}}{\temptwo}


 
\usepackage{tabularx}
\IfFileExists{upquote.sty}{\usepackage{upquote}}{}
\begin{document}

%%----------------------------------------- 


% 
% Need to replace all 
% Cote\_dIvoire
% by
% C\^{o}te d'Ivoire
% 

\title{Supplement Material: Full results}
\author{authors}

\sloppy
\maketitle
\tableofcontents
\addtocontents{toc}{\protect\setcounter{tocdepth}{2}}
\subsection{Partitioning of variability}
Table~\ref{tab:var} presents the partitioning of variability among the random effect terms in the space-time model.

% latex table generated in R 3.4.3 by xtable 1.8-2 package
% Mon Mar 19 16:02:34 2018
\small
\begin{longtable}{lrrrrr}
\caption{Variance component proportions for each country.} \label{tab:var} \\
  \toprule
country & RW2 ($\sigma^2_{\gamma_t}$)& ICAR($\sigma^2_{\phi_i}$) & IID space ($\sigma^2_{\theta_i}$) & IID time ($\sigma^2_{\alpha_t}$) & IID space time ($\sigma^2_{\delta_{it}}$) \\ 
  \midrule
  \endfirsthead
  \toprule
country & RW2 ($\sigma^2_{\gamma_t}$)& ICAR($\sigma^2_{\phi_i}$) & IID space ($\sigma^2_{\theta_i}$) & IID time ($\sigma^2_{\alpha_t}$) & space time ($\sigma^2_{\delta_{it}}$) \\ 
  \midrule
  \endhead
\multicolumn{3}{l}{{\footnotesize Continued on next page $\dots$}}
\endfoot
\endlastfoot
 Angola & 6\% & 56.1\% & 1.4\% & 0.3\% & 36.3\% \\ 
  Benin & 65.5\% & 27.4\% & 2.7\% & 0.5\% & 3.8\% \\ 
  Burkina Faso & 56.9\% & 30.9\% & 3\% & 0.6\% & 8.7\% \\ 
  Burundi & 47.5\% & 35.6\% & 1.8\% & 0.4\% & 14.7\% \\ 
  Cameroon & 29.2\% & 65.3\% & 2.4\% & 0.4\% & 2.7\% \\ 
  Chad & 36.9\% & 41.4\% & 3.3\% & 0.6\% & 17.7\% \\ 
  Comoros & 64.9\% & 16\% & 1.6\% & 0.5\% & 17\% \\ 
  Congo & 72\% & 13.5\% & 2.4\% & 0.6\% & 11.4\% \\ 
  C\^{o}te d'Ivoire & 25.7\% & 54.8\% & 2.5\% & 0.4\% & 16.7\% \\ 
  DRC & 53.9\% & 28.1\% & 2.1\% & 0.4\% & 15.4\% \\ 
  Egypt & 80.2\% & 14.9\% & 0.9\% & 0.2\% & 3.7\% \\ 
  Ethiopia & 70.8\% & 24.6\% & 1\% & 0.2\% & 3.4\% \\ 
  Gabon & 51.4\% & 30.4\% & 3.7\% & 0.8\% & 13.6\% \\ 
  Gambia & 66.2\% & 18.6\% & 0.9\% & 0.2\% & 14\% \\ 
  Ghana & 56.5\% & 34.5\% & 2.6\% & 0.5\% & 5.9\% \\ 
  Guinea & 62.7\% & 31.5\% & 1.1\% & 0.2\% & 4.6\% \\ 
  Kenya & 31.5\% & 48.1\% & 1.7\% & 0.3\% & 18.4\% \\ 
  Lesotho & 28\% & 28.2\% & 4.6\% & 1.1\% & 38.1\% \\ 
  Liberia & 84.2\% & 6.9\% & 1.3\% & 0.3\% & 7.3\% \\ 
  Madagascar & 72.7\% & 16.2\% & 1.4\% & 0.3\% & 9.4\% \\ 
  Malawi & 87\% & 11.1\% & 1\% & 0.2\% & 0.6\% \\ 
  Mali & 42.8\% & 50.4\% & 1.2\% & 0.2\% & 5.4\% \\ 
  Morocco & 83\% & 8.8\% & 1\% & 0.2\% & 7\% \\ 
  Mozambique & 65.2\% & 21.8\% & 1\% & 0.2\% & 11.8\% \\ 
  Namibia & 44.5\% & 32.5\% & 2.2\% & 0.6\% & 20.2\% \\ 
  Niger & 57.1\% & 30.9\% & 1.4\% & 0.3\% & 10.3\% \\ 
  Nigeria & 26.7\% & 65.4\% & 1.9\% & 0.3\% & 5.7\% \\ 
  Rwanda & 83.7\% & 12.5\% & 1\% & 0.2\% & 2.5\% \\ 
  Senegal & 72.5\% & 23\% & 1.2\% & 0.2\% & 3\% \\ 
  Sierra Leone & 59.4\% & 24.7\% & 2.3\% & 0.5\% & 13.1\% \\ 
  Tanzania & 75.8\% & 17.5\% & 1.2\% & 0.2\% & 5.3\% \\ 
  Togo & 53.4\% & 39.9\% & 2.5\% & 0.5\% & 3.6\% \\ 
  Uganda & 87.1\% & 8.7\% & 1.3\% & 0.3\% & 2.7\% \\ 
  Zambia & 75\% & 19.2\% & 1.6\% & 0.3\% & 3.8\% \\ 
  Zimbabwe & 45.2\% & 44.3\% & 2.3\% & 0.5\% & 7.7\% \\ 
  \bottomrule
\end{longtable}


\subsection{Summary of MDG goals}
Table~\ref{tab:mdg} summarizes the MDG4 status by country.
% latex table generated in R 3.4.3 by xtable 1.8-2 package
% Mon Mar 19 13:58:40 2018
% Mon Mar 19 13:58:40 2018
\begin{longtable}{lcrrl}
\caption{MDG4 goal achievement status for subnational regions.} \label{tab:mdg} \\
\toprule
Country & MDG4 achieved & Percent achieved & Median reduction & [Min, Max] \\ 
\midrule
\endfirsthead
\toprule
Country & MDG4 achieved & Percent achieved & Median reduction & [Min, Max] \\ 
\midrule
\endhead
%\bottomrule
{\footnotesize Continued on next page $\dots$}
\endfoot
\endlastfoot
Angola & 1/18 & 5.56\% & 0.098 & [-0.63,0.88] \\ 
  Benin & 0/6 & 0\% & 0.465 & [ 0.38,0.57] \\ 
  Burkina Faso & 0/4 & 0\% & 0.599 & [ 0.41,0.65] \\ 
  Burundi & 1/5 & 20\% & 0.637 & [ 0.44,0.69] \\ 
  Cameroon & 0/5 & 0\% & 0.367 & [ 0.22,0.43] \\ 
  Chad & 0/8 & 0\% & 0.353 & [ 0.08,0.53] \\ 
  Comoros & 0/3 & 0\% & 0.335 & [ 0.10,0.41] \\ 
  Congo & 1/4 & 25\% & 0.621 & [ 0.39,0.69] \\ 
  C\^{o}te d'Ivoire & 0/11 & 0\% & 0.323 & [-0.01,0.59] \\ 
  DRC & 1/11 & 9.09\% & 0.501 & [ 0.27,0.78] \\ 
  Egypt & 2/4 & 50\% & 0.608 & [ 0.51,0.74] \\ 
  Ethiopia & 10/11 & 90.91\% & 0.711 & [ 0.58,0.80] \\ 
  Gabon & 0/5 & 0\% & 0.364 & [ 0.14,0.52] \\ 
  Gambia & 4/6 & 66.67\% & 0.673 & [ 0.36,0.82] \\ 
  Ghana & 0/8 & 0\% & 0.568 & [ 0.35,0.61] \\ 
  Guinea & 3/5 & 60\% & 0.699 & [ 0.40,0.73] \\ 
  Kenya & 3/8 & 37.5\% & 0.509 & [ 0.04,0.75] \\ 
  Lesotho & 0/10 & 0\% & 0.195 & [-0.44,0.46] \\ 
  Liberia & 4/5 & 80\% & 0.748 & [ 0.45,0.78] \\ 
  Madagascar & 6/6 & 100\% & 0.804 & [ 0.69,0.89] \\ 
  Malawi & 3/3 & 100\% & 0.717 & [ 0.71,0.73] \\ 
  Mali & 1/4 & 25\% & 0.617 & [ 0.46,0.74] \\ 
  Morocco & 4/7 & 57.14\% & 0.714 & [ 0.56,0.81] \\ 
  Mozambique & 6/11 & 54.55\% & 0.679 & [ 0.25,0.80] \\ 
  Namibia & 1/13 & 7.69\% & 0.443 & [ 0.03,0.68] \\ 
  Niger & 3/6 & 50\% & 0.706 & [ 0.47,0.84] \\ 
  Nigeria & 0/6 & 0\% & 0.466 & [ 0.22,0.58] \\ 
  Rwanda & 5/5 & 100\% & 0.789 & [ 0.71,0.80] \\ 
  Senegal & 6/11 & 54.55\% & 0.704 & [ 0.59,0.76] \\ 
  Sierra Leone & 0/4 & 0\% & 0.554 & [ 0.33,0.66] \\ 
  Tanzania & 16/20 & 80\% & 0.755 & [ 0.55,0.85] \\ 
  Togo & 0/6 & 0\% & 0.449 & [ 0.35,0.55] \\ 
  Uganda & 3/4 & 75\% & 0.711 & [ 0.67,0.74] \\ 
  Zambia & 4/9 & 44.44\% & 0.658 & [ 0.60,0.76] \\ 
  Zimbabwe & 0/10 & 0\% & 0.032 & [-0.14,0.34] \\ 
   \bottomrule
\end{longtable}


Figures \ref{fig:c1-1} to \ref{fig:c1-4} show the projected U5MR and the projected reduction of U5MR at year of 2015 and the time period of 2015-2019 compared to that of 1990 respectively. In addition to the subnational model results, we include the comparison to the RW2 only model fitted to the combined national data, i.e., without subnational spatial smoothing, after benchmarking with UN estimates.  We also compare our results with UN (B3) estimates described in You et al. (2015) and IHME estimates based on GBD 2015 Child Mortality Collaborators (2016) for the comparisons with 2015 estimates.  

\begin{figure}[htb]
\includegraphics[width = .49\textwidth]{../Main/Figures/Africa_reduction_2015.jpeg}
\includegraphics[width = .49\textwidth]{../Main/Figures/Africa_national_reduction_2015.jpeg}
\includegraphics[width = .49\textwidth]{../Main/Figures/Africa_UN_reduction_2015.jpeg}
\includegraphics[width = .49\textwidth]{../Main/Figures/Africa_IHME_reduction_2015.jpeg}
\caption{Deduction of U5MR from 1990 to 2015 estimated by different methods. Upper left: Subnational model. Upper right: National model. Lower left: UN B-3 estimates. Lower right: IHME GBD estimates}
\label{fig:c1-1}
\end{figure}

\begin{figure}[htb]
\includegraphics[width = .49\textwidth]{../Main/Figures/Africa_u5mr_2015.jpeg}
\includegraphics[width = .49\textwidth]{../Main/Figures/Africa_national_u5mr_2015.jpeg}
\includegraphics[width = .49\textwidth]{../Main/Figures/Africa_UN_u5mr_2015.jpeg}
\includegraphics[width = .49\textwidth]{../Main/Figures/Africa_IHME_u5mr_2015.jpeg}
\caption{Projection of U5MR for 2015 by different methods. Upper left: Subnational model. Upper right: National model. Lower left: UN B-3 estimates. Lower right: IHME GBD estimates}
\label{fig:c1-2}
\end{figure}


\begin{figure}[htb]
\includegraphics[width = .49\textwidth]{../Main/Figures/Africa_reduction_15-19.jpeg}
\includegraphics[width = .49\textwidth]{../Main/Figures/Africa_national_reduction_15-19.jpeg}
% \includegraphics[width = .49\textwidth]{../Main/Figures/Africa_UN_reduction_15-19.jpeg}
% \includegraphics[width = .49\textwidth]{../Main/Figures/Africa_IHME_reduction_15-19.jpeg}
\caption{Deduction of U5MR from 1990 to 2015-2019 period estimated by different methods. Left: Subnational model. Right: National model.}
\label{fig:c1-3}
\end{figure}

\begin{figure}[htb]
\includegraphics[width = .49\textwidth]{../Main/Figures/Africa_u5mr_15-19.jpeg}
\includegraphics[width = .49\textwidth]{../Main/Figures/Africa_national_u5mr_15-19.jpeg}
% \includegraphics[width = .49\textwidth]{../Main/Figures/Africa_UN_u5mr_15-19.jpeg}
% \includegraphics[width = .49\textwidth]{../Main/Figures/Africa_IHME_u5mr_15-19.jpeg}
\caption{Projection of U5MR for 2015-2019 period by different methods. Left: Subnational model. Right: National model.}
\label{fig:c1-4}
\end{figure}

\clearpage
\subsection{Cross validation summary}
Figure~\ref{fig:var} shows the distribution of the cross validation bias combined for all the regions in the study. Table~\ref{tab:cv} summarizes the cross validation results for each country. In general, the scaled bias measure behaves as we would expect if the model is correct (approximately like a standard normal). The plots of the rescaled bias against time for individual countries also do not show systematic patterns, which would be evidence of missing trends. Finally, the coverage of our 95\% interval estimates is generally good (averaging 94\%).


\begin{figure}[htb]
\centering
\includegraphics[width = .8\textwidth]{../Main/Figures/CVbias.pdf}
\caption{Histogram and QQ-plot of the rescaled difference between the smoothed estimates and the direct estimates. The differences between the two estimates are scaled by the square root of the total variance of the two estimates.}
\label{fig:var}
\end{figure}


% latex table generated in R 3.4.3 by xtable 1.8-2 package
% Tue Mar 20 11:11:39 2018
\begin{table}[ht]
\centering
\caption{Coverage of the $95\%$ posterior credible interval for the logit of the direct estimates, mean and standard deviation of the rescaled bias under two cross validation schemes. The biases are scaled by the estimated standard deviation of the difference between the smoothed estimates and the direct estimates.}
\label{tab:cv}
\begin{tabular}{lrrr}
  \toprule
Country & coverage & Average bias & $sd(\mbox{bias})$ \\ 
  \midrule
Angola & 0.94 & 0.06 & 1.11 \\ 
  Benin & 0.94 & 0.01 & 1.01 \\ 
  Burkina Faso & 1.00 & 0.00 & 0.75 \\ 
  Burundi & 0.93 & 0.00 & 1.17 \\ 
  Cameroon & 1.00 & 0.01 & 0.81 \\ 
  Chad & 0.96 & 0.02 & 0.99 \\ 
  Comoros & 1.00 & -0.01 & 0.72 \\ 
  Congo & 0.86 & -0.00 & 1.33 \\ 
   C\^{o}te d'Ivoire & 0.96 & 0.05 & 0.97 \\ 
  DRC & 0.91 & 0.01 & 1.04 \\ 
  Egypt & 0.96 & -0.00 & 0.99 \\ 
  Ethiopia & 0.92 & 0.02 & 1.06 \\ 
  Gabon & 0.94 & -0.00 & 1.11 \\ 
  Gambia & 0.98 & 0.01 & 0.99 \\ 
  Ghana & 0.94 & 0.02 & 1.08 \\ 
  Guinea & 0.97 & 0.02 & 0.88 \\ 
  Kenya & 0.95 & 0.00 & 1.06 \\ 
  Lesotho & 0.99 & 0.03 & 0.87 \\ 
  Liberia & 1.00 & 0.02 & 0.79 \\ 
  Madagascar & 0.89 & -0.01 & 1.18 \\ 
  Malawi & 1.00 & 0.00 & 1.04 \\ 
  Mali & 0.96 & 0.01 & 0.94 \\ 
  Morocco & 0.97 & -0.00 & 1.02 \\ 
  Mozambique & 0.95 & 0.03 & 1.06 \\ 
  Namibia & 0.90 & 0.06 & 1.09 \\ 
  Niger & 0.89 & -0.00 & 1.19 \\ 
  Nigeria & 0.98 & 0.00 & 0.95 \\ 
  Rwanda & 0.69 & -0.02 & 1.58 \\ 
  Senegal & 0.91 & 0.02 & 1.14 \\ 
  Sierra Leone & 0.96 & -0.01 & 0.97 \\ 
  Tanzania & 0.90 & 0.03 & 1.24 \\ 
  Togo & 0.95 & 0.01 & 1.00 \\ 
  Uganda & 0.96 & -0.01 & 1.06 \\ 
  Zambia & 0.89 & 0.02 & 1.14 \\ 
  Zimbabwe & 0.90 & 0.03 & 1.12 \\ 
  \hline
  Average & 0.94 & 0.01 & 1.04 \\ 
   \bottomrule
\end{tabular}
\end{table}


\subsection{Benchmarking Summary}
 The final results were obtained with an additional adjustment step to account for the difference between the smoothed estimates and the estimates from the B3 model \cite{alkema:new:14} on the national-level. Table~\ref{tab:benchmark} summarizes the adjustment factor $\hat r_{t}$, the ratio of the unadjusted direct estimates divided by the direct estimates after benchmarking, in each country and five year periods. 

\begin{table}[htbp]
\centering
\caption{Ratio of the posterior median U5MR from the national model to median U5MR from B3 model for each 5-year period in each country before benchmarking. Values greater than $1$ indicate the direct estimates are adjusted downwards after benchmarking, and vice versa.}
\label{tab:benchmark}
\begin{tabular}{lrrrrrrrr}
  \hline
 Country & 80-84 & 85-89 & 90-94 & 95-99 & 00-04 & 05-09 & 10-14 & Average \\ 
  \toprule
  Benin & 1.05 & 1.02 & 1.00 & 1.03 & 1.00 & 0.88 &  & 1.00 \\ 
  Angola & 1.58 & 1.08 & 0.98 & 0.94 & 0.76 & 0.54 & 0.39 & 0.90 \\ 
  Burkina Faso & 1.02 & 0.98 & 1.00 & 1.01 & 0.99 & 0.95 &  & 0.99 \\ 
  Burundi & 1.07 & 1.05 & 1.12 & 1.27 & 1.26 & 0.89 &  & 1.11 \\ 
  Cameroon & 0.94 & 0.95 & 1.01 & 0.97 & 1.02 & 1.09 & 1.06 & 1.01 \\ 
  Chad & 0.93 & 0.86 & 0.90 & 1.00 & 0.97 & 0.96 & 0.86 & 0.92 \\ 
  Comoros & 0.94 & 0.86 & 0.83 & 0.69 & 0.53 & 0.50 & 0.66 & 0.72 \\ 
  Congo & 0.95 & 0.98 & 0.99 & 1.08 & 0.98 & 0.90 & 1.22 & 1.01 \\ 
  C\^{o}te d'Ivoire & 0.83 & 0.92 & 0.91 & 0.94 & 0.98 & 0.98 & 0.94 & 0.93 \\ 
  DRC & 0.81 & 0.83 & 0.98 & 1.01 & 0.95 & 0.94 & 0.96 & 0.93 \\ 
  Egypt & 0.98 & 0.95 & 1.00 & 0.96 & 0.94 & 0.97 &  & 0.97 \\ 
  Ethiopia & 1.01 & 0.99 & 1.04 & 0.99 & 1.00 & 1.03 & 1.13 & 1.03 \\ 
  Gabon & 0.95 & 0.86 & 0.96 & 0.88 & 0.74 & 0.78 & 1.38 & 0.94 \\ 
  Gambia & 0.77 & 0.67 & 0.67 & 0.70 & 0.76 & 0.73 & 0.67 & 0.71 \\ 
  Ghana & 0.98 & 0.96 & 0.95 & 0.95 & 0.96 & 0.98 & 0.88 & 0.95 \\ 
  Guinea & 0.98 & 0.94 & 0.98 & 0.97 & 1.01 & 1.10 & 1.02 & 1.00 \\ 
  Kenya & 0.97 & 0.94 & 0.94 & 0.88 & 0.88 & 0.83 & 0.92 & 0.91 \\ 
  Lesotho & 0.82 & 0.86 & 0.97 & 0.83 & 0.95 & 0.96 & 0.91 & 0.90 \\ 
  Liberia & 0.95 & 0.96 & 1.03 & 0.97 & 0.96 & 0.94 & 1.13 & 0.99 \\ 
  Madagascar & 1.04 & 1.02 & 0.97 & 0.96 & 0.94 & 0.93 &  & 0.98 \\ 
  Malawi & 0.98 & 0.96 & 0.95 & 0.95 & 0.97 & 0.97 & 0.86 & 0.95 \\ 
  Mali & 0.96 & 0.99 & 0.99 & 1.04 & 1.00 & 1.08 &  & 1.01 \\ 
  Morocco & 0.95 & 0.95 & 0.97 & 1.02 & 0.97 &  &  & 0.97 \\ 
  Mozambique & 0.98 & 0.94 & 0.94 & 0.95 & 0.92 & 0.85 & 0.94 & 0.93 \\ 
  Namibia & 0.88 & 0.87 & 0.92 & 0.85 & 0.97 & 1.02 & 1.06 & 0.94 \\ 
  Niger & 0.98 & 0.97 & 0.95 & 0.96 & 1.00 & 0.97 &  & 0.97 \\ 
  Nigeria & 1.00 & 1.01 & 0.99 & 1.02 & 1.02 & 1.03 & 0.97 & 1.01 \\ 
  Rwanda & 0.98 & 1.01 & 1.16 & 0.96 & 0.99 & 0.94 & 0.92 & 1.00 \\ 
  Senegal & 0.99 & 0.96 & 0.97 & 0.98 & 0.93 & 0.93 & 1.00 & 0.97 \\ 
  Sierra Leone & 1.01 & 1.06 & 0.95 & 0.91 & 0.98 & 1.02 & 0.92 & 0.98 \\ 
  Tanzania & 0.95 & 0.95 & 1.00 & 1.00 & 0.99 & 1.10 & 1.24 & 1.03 \\ 
  Togo & 1.04 & 1.01 & 1.06 & 1.03 & 1.00 & 0.99 & 1.01 & 1.02 \\ 
  Uganda & 0.96 & 0.90 & 0.90 & 0.95 & 1.01 & 1.11 & 1.34 & 1.03 \\ 
  Zambia & 0.99 & 0.97 & 0.96 & 0.90 & 0.91 & 0.85 & 0.96 & 0.93 \\ 
  Zimbabwe & 0.97 & 0.92 & 0.84 & 0.74 & 0.72 & 0.86 &  & 0.84 \\ 
   \hline
  Average & 0.98 & 0.95 & 0.97 & 0.95 & 0.94 & 0.93 & 0.98 & 0.96 \\ 
   \bottomrule
\end{tabular}
\end{table}


% %%%%%%%%%%%%%%%%%%%%%%%%%%%%%%%%%
% %%%%%%%%%%%%%%%%%%%%%%%%%%%%%%%%%
% \subsubsection{Within country relative risk by time}

% Figures \ref{fig:c3-1} and \ref{fig:c3-2}.

% \begin{figure}[htb]
% \includegraphics[width = \textwidth]{../Main/Figures/rr_Africa.jpeg}
% \caption{Africa: relative risk within each country by time period, i.e., $p_{ict} / min_i(p_{ict})$ for region $i$ in country $c$ and time $t$.}
% \label{fig:c3-1}
% \end{figure}

% \begin{figure}[htb]
% \includegraphics[width = \textwidth]{../Main/Figures/rr_Asia.jpeg}
% \caption{Asia: relative risk within each country by time period, i.e., $p_{ict} / min_i(p_{ict})$ for region $i$ in country $c$ and time $t$.}
% \label{fig:c3-2}
% \end{figure}

\clearpage
\subsection{All Results by Country} \label{sec:resultsByCountry}
% %%%%%%%%%%%%%%%%%%%%%%%%%%%%%%%%%%%%%%%%%%%%%%%%%%%%%%%%%%%%%%%%%%%%%%%%%%%%%%%%%%%%%%%%%%%%%%%%%%
% \subsection{Bangladesh}
% <<Bangladesh, echo=FALSE, results='hide'>>=
% countryname <- "Bangladesh"
% @
% <<Sexpr{paste0(countryname, "-run")}, messages=FALSE, child='single-country-combined.rnw'>>=
% @

% %%%%%%%%%%%%%%%%%%%%%%%%%%%%%%%%%%%%%%%%%%%%%%%%%%%%%%%%%%%%%%%%%%%%%%%%%%%%%%%%%%%%%%%%%%%%%%%%%%
% \clearpage
% \subsection{Cambodia}
% <<Cambodia, echo=FALSE, results='hide'>>=
% countryname <- "Cambodia"
% country_count <- country_count + 1
% @
% <<Sexpr{paste0(countryname, "-run")}, messages=FALSE, child='single-country-combined.rnw'>>=
% @


% %%%%%%%%%%%%%%%%%%%%%%%%%%%%%%%%%%%%%%%%%%%%%%%%%%%%%%%%%%%%%%%%%%%%%%%%%%%%%%%%%%%%%%%%%%%%%%%%%%
% \clearpage
% \subsection{Indonesia}
% <<Indonesia, echo=FALSE, results='hide'>>=
% countryname <- "Indonesia"
% country_count <- country_count + 1
% @
% <<Sexpr{paste0(countryname, "-run")}, messages=FALSE, child='single-country-combined.rnw'>>=
% @

% %%%%%%%%%%%%%%%%%%%%%%%%%%%%%%%%%%%%%%%%%%%%%%%%%%%%%%%%%%%%%%%%%%%%%%%%%%%%%%%%%%%%%%%%%%%%%%%%%%
% \clearpage
% \subsection{Philippines}
% <<Philippines, echo=FALSE, results='hide'>>=
% countryname <- "Philippines"
% country_count <- country_count + 1
% @
% <<Sexpr{paste0(countryname, "-run")}, messages=FALSE, child='single-country-combined.rnw'>>=
% @
%%%%%%%%%%%%%%%%%%%%%%%%%%%%%%%%%%%%%%%%%%%%%%%%%%%%%%%%%%%%%%%%%%%%%%%%%%%%%%%%%%%%%%%%%%%%%%%%%%
\subsubsection{Angola}


% \subsubsection{Summary of DHS surveys}

%%%%%%%%%%%%%%%%%%%%%%%%%%% Summary 


DHS surveys were conducted in Angola in 2015.
% years.out[1:(length(years.out)-1)], and years.out[length(years.out)]. 

We fit both the RW2 only model to the combined national data, and compare the time trend at national level with the estimates produced by the UN and IHME in Figure~\ref{fig:unnamed-chunk-2}. We then adjusted the combined national data to the UN estimates of U5MR, and refit the models on the benchmarked data. 

%%%%%%%%%%%%%%%%%%%%%%%%%% Plot5 
\begin{knitrout}
\definecolor{shadecolor}{rgb}{0.969, 0.969, 0.969}\color{fgcolor}\begin{figure}[bht]

{\centering \includegraphics[width=.9\textwidth]{../Main/Figures/Yearly_national_Angola} 

}

\caption[Angola]{Angola: Temporal national trends along with UN (B3) estimates described in You et al. (2015) and IHME estimates based on GBD 2015 Child Mortality Collaborators (2016). RW2 represents the smoothed national estimates using the original data before benchmarking with UN estimates. RW2-adj represents the smoothed national estimates using the benchmarked data.}\label{fig:unnamed-chunk-2}
\end{figure}


\end{knitrout}
 

We fit the RW2 model to the benchmarked data in each area. 
% The proportions of the explained variation is summarized in Table~\ref{tab:paste0(countryname, "-var")}. 
We compare the results in Figure~\ref{fig:unnamed-chunk-3} to \ref{fig:unnamed-chunk-7}.
Figure~\ref{fig:unnamed-chunk-3} compares the smoothed estimates against the direct estimates. Figure~\ref{fig:unnamed-chunk-4} and Figure~\ref{fig:unnamed-chunk-5} show the posterior median estimates of U5MR in each region over time and the reductions from 1990 period respectively.
Figure~\ref{fig:unnamed-chunk-6} shows the smoothed estimates by region over time and Figure~\ref{fig:unnamed-chunk-7} compares the smoothed estimates with direct estimates from each survey for each region over time.


% %%%%%%%%%%%%%%%%%%%%%%%%%%% Table1 
% <<echo=FALSE, results='asis'>>=
% load("rda/variance_tables.rda")
% countryname2 <- gsub(" ", "", countryname)
% variance <- tables.all[[countryname]]

% table_count <- table_count + 1

% names <- c("RW2 ($\\sigma^2_{\\gamma_{t}}$)", "ICAR ($\\sigma^2_{\\phi_{i}}$)", "IID space ($\\sigma^2_{\\theta_{i}}$)", "IID time ($\\sigma^2_{\\alpha_{t}}$)", "IID space time ($\\sigma^2_{\\delta_{it}}$)")

% variance$Proportion <- round(variance$Proportion*100, digits = 2)
% row.names(variance) <- names
% tab <- xtable(variance, digits = c(1, 3, 2),align = "l|ll",
%        label = paste0("tab:", countryname, "-var"),
%        caption = paste(country, ": summary of the variance components in the RW2 model", sep = ''))
% print(tab, comment = FALSE,sanitize.text.function = function(x) {x})
% @

%%%%%%%%%%%%%%%%%%%%%%%%%%% Plot1 
\begin{knitrout}
\definecolor{shadecolor}{rgb}{0.969, 0.969, 0.969}\color{fgcolor}\begin{figure}[bht]

{\centering \includegraphics[width=.9\textwidth]{../Main/Figures/SmoothvDirectAngola_meta} 

}

\caption[Angola]{Angola: Smooth versus direct Admin 1 estimates. Left: Combined (meta-analysis) survey estimate against combined direct estimates. Right: Combined (meta-analysis) survey estimate against direct estimates from each survey.}\label{fig:unnamed-chunk-3}
\end{figure}


\end{knitrout}

%%%%%%%%%%%%%%%%%%%%%%%%%%% Plot2 
\begin{knitrout}
\definecolor{shadecolor}{rgb}{0.969, 0.969, 0.969}\color{fgcolor}\begin{figure}[bht]

{\centering \includegraphics[width=.9\textwidth]{../Main/Figures/SmoothMedianAngola} 

}

\caption[Angola]{Angola: Maps of posterior medians over time.}\label{fig:unnamed-chunk-4}
\end{figure}


\end{knitrout}
%%%%%%%%%%%%%%%%%%%%%%%%%%% Plot2a
\begin{knitrout}
\definecolor{shadecolor}{rgb}{0.969, 0.969, 0.969}\color{fgcolor}\begin{figure}[bht]

{\centering \includegraphics[width=.9\textwidth]{../Main/Figures/ReductionMedianAngola} 

}

\caption[Angola]{Angola: Maps of reduction of posterior median U5MR in each five-year period compared to 1990 over time.}\label{fig:unnamed-chunk-5}
\end{figure}


\end{knitrout}
%%%%%%%%%%%%%%%%%%%%%%%%%%% Plot3 
\begin{knitrout}
\definecolor{shadecolor}{rgb}{0.969, 0.969, 0.969}\color{fgcolor}\begin{figure}[bht]

{\centering \includegraphics[width=.95\textwidth]{../Main/Figures/Yearly_v_Periods_Angola} 

}

\caption[Angola]{Angola: Smoothed regional estimates over time. The line indicates yearly posterior median estimates and error bars indicate 95 \% posterior credible interval at each time period.}\label{fig:unnamed-chunk-6}
\end{figure}


\end{knitrout}

%%%%%%%%%%%%%%%%%%%%%%%%%%% Plot4 
\begin{knitrout}
\definecolor{shadecolor}{rgb}{0.969, 0.969, 0.969}\color{fgcolor}\begin{figure}[bht]

{\centering \includegraphics[width=.9\textwidth]{../Main/Figures/LineSubMedianAngola} 

}

\caption[Angola]{Angola: Smoothed regional estimates over time compared to the direct estimates from each surveys. Direct estimates are not benchmarked with UN estimates. The line indicates posterior median and error bars indicate 95\% posterior credible interval.}\label{fig:unnamed-chunk-7}
\end{figure}


\end{knitrout}
% \subsubsection{National model results}
We further assess the RW2 model by holding out some observations, and compare the projections to the direct estimates in these holdout observations. Figure~\ref{fig:unnamed-chunk-8} compares the predicted estimates for the out-of-sample observations  with the direct estimates by holding out observations from each area in each time period.  Figure~\ref{fig:unnamed-chunk-9} compares the histogram of the bias rescaled by the total variance in the cross validation studies. Figure~\ref{fig:unnamed-chunk-10} compares the rescaled bias by region and time periods.



% %%%%%%%%%%%%%%%%%%%%%%%%%%% Plot6
% << echo=FALSE, out.width = ".9\\textwidth", fig.width = 12, fig.height = 6, fig.cap = "Out-of-sample predictions along with direct estimates in the cross validation study where all data from each time period is held out and predicted using the rest of the data.">>=
% fig_count <- fig_count + 1
% knitr::include_graphics(paste0("../Main/Figures/CV_byYear_withError_", countryname2, ".pdf")) 
% @
 
%%%%%%%%%%%%%%%%%%%%%%%%%%% Plot7
\begin{knitrout}
\definecolor{shadecolor}{rgb}{0.969, 0.969, 0.969}\color{fgcolor}\begin{figure}[bht]

{\centering \includegraphics[width=.9\textwidth]{../Main/Figures/CV_byYearRegion_withError_Angola} 

}

\caption[Angola]{Angola: Out-of-sample predictions along with direct estimates in the cross validation study where data from one region in each time period is held out and predicted using the rest of the data.}\label{fig:unnamed-chunk-8}
\end{figure}


\end{knitrout}

%%%%%%%%%%%%%%%%%%%%%%%%%%% Plot8
\begin{knitrout}
\definecolor{shadecolor}{rgb}{0.969, 0.969, 0.969}\color{fgcolor}\begin{figure}[bht]

{\centering \includegraphics[width=.9\textwidth]{../Main/Figures/CVbiasAngola} 

}

\caption[Angola]{Angola: Histogram and QQ-plot of the rescaled difference between the smoothed estimates and the direct estimates in the cross validation study. The differences between the two estimates are rescaled by the square root of the total variance of the two estimates.}\label{fig:unnamed-chunk-9}
\end{figure}


\end{knitrout}

%%%%%%%%%%%%%%%%%%%%%%%%%%% Plot9
\begin{knitrout}
\definecolor{shadecolor}{rgb}{0.969, 0.969, 0.969}\color{fgcolor}\begin{figure}[bht]

{\centering \includegraphics[width=.7\textwidth]{../Main/Figures/CVbiasbyRegionAngola} 

}

\caption[Angola]{Angola: Line plot of the difference between smoothed estimates and the direct estimates in the cross validation study. The differences between the two estimates are rescaled by the square root of the total variance of the two estimates.}\label{fig:unnamed-chunk-10}
\end{figure}


\end{knitrout}


%%%%%%%%%%%%%%%%%%%%%%%%%%%%%%%%%%%%%%%%%%%%%%%%%%%%%%%%%%%%%%%%%%%%%%%%%%%%%%%%%%%%%%%%%%%%%%%%%%
\clearpage
\subsubsection{Benin}


% \subsubsection{Summary of DHS surveys}

%%%%%%%%%%%%%%%%%%%%%%%%%%% Summary 


DHS surveys were conducted in Benin in 1996, 2001, and 2006.
% years.out[1:(length(years.out)-1)], and years.out[length(years.out)]. 

We fit both the RW2 only model to the combined national data, and compare the time trend at national level with the estimates produced by the UN and IHME in Figure~\ref{fig:unnamed-chunk-12}. We then adjusted the combined national data to the UN estimates of U5MR, and refit the models on the benchmarked data. 

%%%%%%%%%%%%%%%%%%%%%%%%%% Plot5 
\begin{knitrout}
\definecolor{shadecolor}{rgb}{0.969, 0.969, 0.969}\color{fgcolor}\begin{figure}[bht]

{\centering \includegraphics[width=.9\textwidth]{../Main/Figures/Yearly_national_Benin} 

}

\caption[Benin]{Benin: Temporal national trends along with UN (B3) estimates described in You et al. (2015) and IHME estimates based on GBD 2015 Child Mortality Collaborators (2016). RW2 represents the smoothed national estimates using the original data before benchmarking with UN estimates. RW2-adj represents the smoothed national estimates using the benchmarked data.}\label{fig:unnamed-chunk-12}
\end{figure}


\end{knitrout}
 

We fit the RW2 model to the benchmarked data in each area. 
% The proportions of the explained variation is summarized in Table~\ref{tab:paste0(countryname, "-var")}. 
We compare the results in Figure~\ref{fig:unnamed-chunk-13} to \ref{fig:unnamed-chunk-17}.
Figure~\ref{fig:unnamed-chunk-13} compares the smoothed estimates against the direct estimates. Figure~\ref{fig:unnamed-chunk-14} and Figure~\ref{fig:unnamed-chunk-15} show the posterior median estimates of U5MR in each region over time and the reductions from 1990 period respectively.
Figure~\ref{fig:unnamed-chunk-16} shows the smoothed estimates by region over time and Figure~\ref{fig:unnamed-chunk-17} compares the smoothed estimates with direct estimates from each survey for each region over time.


% %%%%%%%%%%%%%%%%%%%%%%%%%%% Table1 
% <<echo=FALSE, results='asis'>>=
% load("rda/variance_tables.rda")
% countryname2 <- gsub(" ", "", countryname)
% variance <- tables.all[[countryname]]

% table_count <- table_count + 1

% names <- c("RW2 ($\\sigma^2_{\\gamma_{t}}$)", "ICAR ($\\sigma^2_{\\phi_{i}}$)", "IID space ($\\sigma^2_{\\theta_{i}}$)", "IID time ($\\sigma^2_{\\alpha_{t}}$)", "IID space time ($\\sigma^2_{\\delta_{it}}$)")

% variance$Proportion <- round(variance$Proportion*100, digits = 2)
% row.names(variance) <- names
% tab <- xtable(variance, digits = c(1, 3, 2),align = "l|ll",
%        label = paste0("tab:", countryname, "-var"),
%        caption = paste(country, ": summary of the variance components in the RW2 model", sep = ''))
% print(tab, comment = FALSE,sanitize.text.function = function(x) {x})
% @

%%%%%%%%%%%%%%%%%%%%%%%%%%% Plot1 
\begin{knitrout}
\definecolor{shadecolor}{rgb}{0.969, 0.969, 0.969}\color{fgcolor}\begin{figure}[bht]

{\centering \includegraphics[width=.9\textwidth]{../Main/Figures/SmoothvDirectBenin_meta} 

}

\caption[Benin]{Benin: Smooth versus direct Admin 1 estimates. Left: Combined (meta-analysis) survey estimate against combined direct estimates. Right: Combined (meta-analysis) survey estimate against direct estimates from each survey.}\label{fig:unnamed-chunk-13}
\end{figure}


\end{knitrout}

%%%%%%%%%%%%%%%%%%%%%%%%%%% Plot2 
\begin{knitrout}
\definecolor{shadecolor}{rgb}{0.969, 0.969, 0.969}\color{fgcolor}\begin{figure}[bht]

{\centering \includegraphics[width=.9\textwidth]{../Main/Figures/SmoothMedianBenin} 

}

\caption[Benin]{Benin: Maps of posterior medians over time.}\label{fig:unnamed-chunk-14}
\end{figure}


\end{knitrout}
%%%%%%%%%%%%%%%%%%%%%%%%%%% Plot2a
\begin{knitrout}
\definecolor{shadecolor}{rgb}{0.969, 0.969, 0.969}\color{fgcolor}\begin{figure}[bht]

{\centering \includegraphics[width=.9\textwidth]{../Main/Figures/ReductionMedianBenin} 

}

\caption[Benin]{Benin: Maps of reduction of posterior median U5MR in each five-year period compared to 1990 over time.}\label{fig:unnamed-chunk-15}
\end{figure}


\end{knitrout}
%%%%%%%%%%%%%%%%%%%%%%%%%%% Plot3 
\begin{knitrout}
\definecolor{shadecolor}{rgb}{0.969, 0.969, 0.969}\color{fgcolor}\begin{figure}[bht]

{\centering \includegraphics[width=.95\textwidth]{../Main/Figures/Yearly_v_Periods_Benin} 

}

\caption[Benin]{Benin: Smoothed regional estimates over time. The line indicates yearly posterior median estimates and error bars indicate 95 \% posterior credible interval at each time period.}\label{fig:unnamed-chunk-16}
\end{figure}


\end{knitrout}

%%%%%%%%%%%%%%%%%%%%%%%%%%% Plot4 
\begin{knitrout}
\definecolor{shadecolor}{rgb}{0.969, 0.969, 0.969}\color{fgcolor}\begin{figure}[bht]

{\centering \includegraphics[width=.9\textwidth]{../Main/Figures/LineSubMedianBenin} 

}

\caption[Benin]{Benin: Smoothed regional estimates over time compared to the direct estimates from each surveys. Direct estimates are not benchmarked with UN estimates. The line indicates posterior median and error bars indicate 95\% posterior credible interval.}\label{fig:unnamed-chunk-17}
\end{figure}


\end{knitrout}
% \subsubsection{National model results}
We further assess the RW2 model by holding out some observations, and compare the projections to the direct estimates in these holdout observations. Figure~\ref{fig:unnamed-chunk-18} compares the predicted estimates for the out-of-sample observations  with the direct estimates by holding out observations from each area in each time period.  Figure~\ref{fig:unnamed-chunk-19} compares the histogram of the bias rescaled by the total variance in the cross validation studies. Figure~\ref{fig:unnamed-chunk-20} compares the rescaled bias by region and time periods.



% %%%%%%%%%%%%%%%%%%%%%%%%%%% Plot6
% << echo=FALSE, out.width = ".9\\textwidth", fig.width = 12, fig.height = 6, fig.cap = "Out-of-sample predictions along with direct estimates in the cross validation study where all data from each time period is held out and predicted using the rest of the data.">>=
% fig_count <- fig_count + 1
% knitr::include_graphics(paste0("../Main/Figures/CV_byYear_withError_", countryname2, ".pdf")) 
% @
 
%%%%%%%%%%%%%%%%%%%%%%%%%%% Plot7
\begin{knitrout}
\definecolor{shadecolor}{rgb}{0.969, 0.969, 0.969}\color{fgcolor}\begin{figure}[bht]

{\centering \includegraphics[width=.9\textwidth]{../Main/Figures/CV_byYearRegion_withError_Benin} 

}

\caption[Benin]{Benin: Out-of-sample predictions along with direct estimates in the cross validation study where data from one region in each time period is held out and predicted using the rest of the data.}\label{fig:unnamed-chunk-18}
\end{figure}


\end{knitrout}

%%%%%%%%%%%%%%%%%%%%%%%%%%% Plot8
\begin{knitrout}
\definecolor{shadecolor}{rgb}{0.969, 0.969, 0.969}\color{fgcolor}\begin{figure}[bht]

{\centering \includegraphics[width=.9\textwidth]{../Main/Figures/CVbiasBenin} 

}

\caption[Benin]{Benin: Histogram and QQ-plot of the rescaled difference between the smoothed estimates and the direct estimates in the cross validation study. The differences between the two estimates are rescaled by the square root of the total variance of the two estimates.}\label{fig:unnamed-chunk-19}
\end{figure}


\end{knitrout}

%%%%%%%%%%%%%%%%%%%%%%%%%%% Plot9
\begin{knitrout}
\definecolor{shadecolor}{rgb}{0.969, 0.969, 0.969}\color{fgcolor}\begin{figure}[bht]

{\centering \includegraphics[width=.7\textwidth]{../Main/Figures/CVbiasbyRegionBenin} 

}

\caption[Benin]{Benin: Line plot of the difference between smoothed estimates and the direct estimates in the cross validation study. The differences between the two estimates are rescaled by the square root of the total variance of the two estimates.}\label{fig:unnamed-chunk-20}
\end{figure}


\end{knitrout}

%%%%%%%%%%%%%%%%%%%%%%%%%%%%%%%%%%%%%%%%%%%%%%%%%%%%%%%%%%%%%%%%%%%%%%%%%%%%%%%%%%%%%%%%%%%%%%%%%%
\clearpage
\subsubsection{Burkina Faso}


% \subsubsection{Summary of DHS surveys}

%%%%%%%%%%%%%%%%%%%%%%%%%%% Summary 


DHS surveys were conducted in Burkina Faso in 1993, 1999, 2003, and 2010.
% years.out[1:(length(years.out)-1)], and years.out[length(years.out)]. 

We fit both the RW2 only model to the combined national data, and compare the time trend at national level with the estimates produced by the UN and IHME in Figure~\ref{fig:unnamed-chunk-22}. We then adjusted the combined national data to the UN estimates of U5MR, and refit the models on the benchmarked data. 

%%%%%%%%%%%%%%%%%%%%%%%%%% Plot5 
\begin{knitrout}
\definecolor{shadecolor}{rgb}{0.969, 0.969, 0.969}\color{fgcolor}\begin{figure}[bht]

{\centering \includegraphics[width=.9\textwidth]{../Main/Figures/Yearly_national_BurkinaFaso} 

}

\caption[Burkina Faso]{Burkina Faso: Temporal national trends along with UN (B3) estimates described in You et al. (2015) and IHME estimates based on GBD 2015 Child Mortality Collaborators (2016). RW2 represents the smoothed national estimates using the original data before benchmarking with UN estimates. RW2-adj represents the smoothed national estimates using the benchmarked data.}\label{fig:unnamed-chunk-22}
\end{figure}


\end{knitrout}
 

We fit the RW2 model to the benchmarked data in each area. 
% The proportions of the explained variation is summarized in Table~\ref{tab:paste0(countryname, "-var")}. 
We compare the results in Figure~\ref{fig:unnamed-chunk-23} to \ref{fig:unnamed-chunk-27}.
Figure~\ref{fig:unnamed-chunk-23} compares the smoothed estimates against the direct estimates. Figure~\ref{fig:unnamed-chunk-24} and Figure~\ref{fig:unnamed-chunk-25} show the posterior median estimates of U5MR in each region over time and the reductions from 1990 period respectively.
Figure~\ref{fig:unnamed-chunk-26} shows the smoothed estimates by region over time and Figure~\ref{fig:unnamed-chunk-27} compares the smoothed estimates with direct estimates from each survey for each region over time.


% %%%%%%%%%%%%%%%%%%%%%%%%%%% Table1 
% <<echo=FALSE, results='asis'>>=
% load("rda/variance_tables.rda")
% countryname2 <- gsub(" ", "", countryname)
% variance <- tables.all[[countryname]]

% table_count <- table_count + 1

% names <- c("RW2 ($\\sigma^2_{\\gamma_{t}}$)", "ICAR ($\\sigma^2_{\\phi_{i}}$)", "IID space ($\\sigma^2_{\\theta_{i}}$)", "IID time ($\\sigma^2_{\\alpha_{t}}$)", "IID space time ($\\sigma^2_{\\delta_{it}}$)")

% variance$Proportion <- round(variance$Proportion*100, digits = 2)
% row.names(variance) <- names
% tab <- xtable(variance, digits = c(1, 3, 2),align = "l|ll",
%        label = paste0("tab:", countryname, "-var"),
%        caption = paste(country, ": summary of the variance components in the RW2 model", sep = ''))
% print(tab, comment = FALSE,sanitize.text.function = function(x) {x})
% @

%%%%%%%%%%%%%%%%%%%%%%%%%%% Plot1 
\begin{knitrout}
\definecolor{shadecolor}{rgb}{0.969, 0.969, 0.969}\color{fgcolor}\begin{figure}[bht]

{\centering \includegraphics[width=.9\textwidth]{../Main/Figures/SmoothvDirectBurkinaFaso_meta} 

}

\caption[Burkina Faso]{Burkina Faso: Smooth versus direct Admin 1 estimates. Left: Combined (meta-analysis) survey estimate against combined direct estimates. Right: Combined (meta-analysis) survey estimate against direct estimates from each survey.}\label{fig:unnamed-chunk-23}
\end{figure}


\end{knitrout}

%%%%%%%%%%%%%%%%%%%%%%%%%%% Plot2 
\begin{knitrout}
\definecolor{shadecolor}{rgb}{0.969, 0.969, 0.969}\color{fgcolor}\begin{figure}[bht]

{\centering \includegraphics[width=.9\textwidth]{../Main/Figures/SmoothMedianBurkinaFaso} 

}

\caption[Burkina Faso]{Burkina Faso: Maps of posterior medians over time.}\label{fig:unnamed-chunk-24}
\end{figure}


\end{knitrout}
%%%%%%%%%%%%%%%%%%%%%%%%%%% Plot2a
\begin{knitrout}
\definecolor{shadecolor}{rgb}{0.969, 0.969, 0.969}\color{fgcolor}\begin{figure}[bht]

{\centering \includegraphics[width=.9\textwidth]{../Main/Figures/ReductionMedianBurkinaFaso} 

}

\caption[Burkina Faso]{Burkina Faso: Maps of reduction of posterior median U5MR in each five-year period compared to 1990 over time.}\label{fig:unnamed-chunk-25}
\end{figure}


\end{knitrout}
%%%%%%%%%%%%%%%%%%%%%%%%%%% Plot3 
\begin{knitrout}
\definecolor{shadecolor}{rgb}{0.969, 0.969, 0.969}\color{fgcolor}\begin{figure}[bht]

{\centering \includegraphics[width=.95\textwidth]{../Main/Figures/Yearly_v_Periods_BurkinaFaso} 

}

\caption[Burkina Faso]{Burkina Faso: Smoothed regional estimates over time. The line indicates yearly posterior median estimates and error bars indicate 95 \% posterior credible interval at each time period.}\label{fig:unnamed-chunk-26}
\end{figure}


\end{knitrout}

%%%%%%%%%%%%%%%%%%%%%%%%%%% Plot4 
\begin{knitrout}
\definecolor{shadecolor}{rgb}{0.969, 0.969, 0.969}\color{fgcolor}\begin{figure}[bht]

{\centering \includegraphics[width=.9\textwidth]{../Main/Figures/LineSubMedianBurkinaFaso} 

}

\caption[Burkina Faso]{Burkina Faso: Smoothed regional estimates over time compared to the direct estimates from each surveys. Direct estimates are not benchmarked with UN estimates. The line indicates posterior median and error bars indicate 95\% posterior credible interval.}\label{fig:unnamed-chunk-27}
\end{figure}


\end{knitrout}
% \subsubsection{National model results}
We further assess the RW2 model by holding out some observations, and compare the projections to the direct estimates in these holdout observations. Figure~\ref{fig:unnamed-chunk-28} compares the predicted estimates for the out-of-sample observations  with the direct estimates by holding out observations from each area in each time period.  Figure~\ref{fig:unnamed-chunk-29} compares the histogram of the bias rescaled by the total variance in the cross validation studies. Figure~\ref{fig:unnamed-chunk-30} compares the rescaled bias by region and time periods.



% %%%%%%%%%%%%%%%%%%%%%%%%%%% Plot6
% << echo=FALSE, out.width = ".9\\textwidth", fig.width = 12, fig.height = 6, fig.cap = "Out-of-sample predictions along with direct estimates in the cross validation study where all data from each time period is held out and predicted using the rest of the data.">>=
% fig_count <- fig_count + 1
% knitr::include_graphics(paste0("../Main/Figures/CV_byYear_withError_", countryname2, ".pdf")) 
% @
 
%%%%%%%%%%%%%%%%%%%%%%%%%%% Plot7
\begin{knitrout}
\definecolor{shadecolor}{rgb}{0.969, 0.969, 0.969}\color{fgcolor}\begin{figure}[bht]

{\centering \includegraphics[width=.9\textwidth]{../Main/Figures/CV_byYearRegion_withError_BurkinaFaso} 

}

\caption[Burkina Faso]{Burkina Faso: Out-of-sample predictions along with direct estimates in the cross validation study where data from one region in each time period is held out and predicted using the rest of the data.}\label{fig:unnamed-chunk-28}
\end{figure}


\end{knitrout}

%%%%%%%%%%%%%%%%%%%%%%%%%%% Plot8
\begin{knitrout}
\definecolor{shadecolor}{rgb}{0.969, 0.969, 0.969}\color{fgcolor}\begin{figure}[bht]

{\centering \includegraphics[width=.9\textwidth]{../Main/Figures/CVbiasBurkinaFaso} 

}

\caption[Burkina Faso]{Burkina Faso: Histogram and QQ-plot of the rescaled difference between the smoothed estimates and the direct estimates in the cross validation study. The differences between the two estimates are rescaled by the square root of the total variance of the two estimates.}\label{fig:unnamed-chunk-29}
\end{figure}


\end{knitrout}

%%%%%%%%%%%%%%%%%%%%%%%%%%% Plot9
\begin{knitrout}
\definecolor{shadecolor}{rgb}{0.969, 0.969, 0.969}\color{fgcolor}\begin{figure}[bht]

{\centering \includegraphics[width=.7\textwidth]{../Main/Figures/CVbiasbyRegionBurkinaFaso} 

}

\caption[Burkina Faso]{Burkina Faso: Line plot of the difference between smoothed estimates and the direct estimates in the cross validation study. The differences between the two estimates are rescaled by the square root of the total variance of the two estimates.}\label{fig:unnamed-chunk-30}
\end{figure}


\end{knitrout}

%%%%%%%%%%%%%%%%%%%%%%%%%%%%%%%%%%%%%%%%%%%%%%%%%%%%%%%%%%%%%%%%%%%%%%%%%%%%%%%%%%%%%%%%%%%%%%%%%%
\clearpage
\subsubsection{Burundi}


% \subsubsection{Summary of DHS surveys}

%%%%%%%%%%%%%%%%%%%%%%%%%%% Summary 


DHS surveys were conducted in Burundi in 2010.
% years.out[1:(length(years.out)-1)], and years.out[length(years.out)]. 

We fit both the RW2 only model to the combined national data, and compare the time trend at national level with the estimates produced by the UN and IHME in Figure~\ref{fig:unnamed-chunk-32}. We then adjusted the combined national data to the UN estimates of U5MR, and refit the models on the benchmarked data. 

%%%%%%%%%%%%%%%%%%%%%%%%%% Plot5 
\begin{knitrout}
\definecolor{shadecolor}{rgb}{0.969, 0.969, 0.969}\color{fgcolor}\begin{figure}[bht]

{\centering \includegraphics[width=.9\textwidth]{../Main/Figures/Yearly_national_Burundi} 

}

\caption[Burundi]{Burundi: Temporal national trends along with UN (B3) estimates described in You et al. (2015) and IHME estimates based on GBD 2015 Child Mortality Collaborators (2016). RW2 represents the smoothed national estimates using the original data before benchmarking with UN estimates. RW2-adj represents the smoothed national estimates using the benchmarked data.}\label{fig:unnamed-chunk-32}
\end{figure}


\end{knitrout}
 

We fit the RW2 model to the benchmarked data in each area. 
% The proportions of the explained variation is summarized in Table~\ref{tab:paste0(countryname, "-var")}. 
We compare the results in Figure~\ref{fig:unnamed-chunk-33} to \ref{fig:unnamed-chunk-37}.
Figure~\ref{fig:unnamed-chunk-33} compares the smoothed estimates against the direct estimates. Figure~\ref{fig:unnamed-chunk-34} and Figure~\ref{fig:unnamed-chunk-35} show the posterior median estimates of U5MR in each region over time and the reductions from 1990 period respectively.
Figure~\ref{fig:unnamed-chunk-36} shows the smoothed estimates by region over time and Figure~\ref{fig:unnamed-chunk-37} compares the smoothed estimates with direct estimates from each survey for each region over time.


% %%%%%%%%%%%%%%%%%%%%%%%%%%% Table1 
% <<echo=FALSE, results='asis'>>=
% load("rda/variance_tables.rda")
% countryname2 <- gsub(" ", "", countryname)
% variance <- tables.all[[countryname]]

% table_count <- table_count + 1

% names <- c("RW2 ($\\sigma^2_{\\gamma_{t}}$)", "ICAR ($\\sigma^2_{\\phi_{i}}$)", "IID space ($\\sigma^2_{\\theta_{i}}$)", "IID time ($\\sigma^2_{\\alpha_{t}}$)", "IID space time ($\\sigma^2_{\\delta_{it}}$)")

% variance$Proportion <- round(variance$Proportion*100, digits = 2)
% row.names(variance) <- names
% tab <- xtable(variance, digits = c(1, 3, 2),align = "l|ll",
%        label = paste0("tab:", countryname, "-var"),
%        caption = paste(country, ": summary of the variance components in the RW2 model", sep = ''))
% print(tab, comment = FALSE,sanitize.text.function = function(x) {x})
% @

%%%%%%%%%%%%%%%%%%%%%%%%%%% Plot1 
\begin{knitrout}
\definecolor{shadecolor}{rgb}{0.969, 0.969, 0.969}\color{fgcolor}\begin{figure}[bht]

{\centering \includegraphics[width=.9\textwidth]{../Main/Figures/SmoothvDirectBurundi_meta} 

}

\caption[Burundi]{Burundi: Smooth versus direct Admin 1 estimates. Left: Combined (meta-analysis) survey estimate against combined direct estimates. Right: Combined (meta-analysis) survey estimate against direct estimates from each survey.}\label{fig:unnamed-chunk-33}
\end{figure}


\end{knitrout}

%%%%%%%%%%%%%%%%%%%%%%%%%%% Plot2 
\begin{knitrout}
\definecolor{shadecolor}{rgb}{0.969, 0.969, 0.969}\color{fgcolor}\begin{figure}[bht]

{\centering \includegraphics[width=.9\textwidth]{../Main/Figures/SmoothMedianBurundi} 

}

\caption[Burundi]{Burundi: Maps of posterior medians over time.}\label{fig:unnamed-chunk-34}
\end{figure}


\end{knitrout}
%%%%%%%%%%%%%%%%%%%%%%%%%%% Plot2a
\begin{knitrout}
\definecolor{shadecolor}{rgb}{0.969, 0.969, 0.969}\color{fgcolor}\begin{figure}[bht]

{\centering \includegraphics[width=.9\textwidth]{../Main/Figures/ReductionMedianBurundi} 

}

\caption[Burundi]{Burundi: Maps of reduction of posterior median U5MR in each five-year period compared to 1990 over time.}\label{fig:unnamed-chunk-35}
\end{figure}


\end{knitrout}
%%%%%%%%%%%%%%%%%%%%%%%%%%% Plot3 
\begin{knitrout}
\definecolor{shadecolor}{rgb}{0.969, 0.969, 0.969}\color{fgcolor}\begin{figure}[bht]

{\centering \includegraphics[width=.95\textwidth]{../Main/Figures/Yearly_v_Periods_Burundi} 

}

\caption[Burundi]{Burundi: Smoothed regional estimates over time. The line indicates yearly posterior median estimates and error bars indicate 95 \% posterior credible interval at each time period.}\label{fig:unnamed-chunk-36}
\end{figure}


\end{knitrout}

%%%%%%%%%%%%%%%%%%%%%%%%%%% Plot4 
\begin{knitrout}
\definecolor{shadecolor}{rgb}{0.969, 0.969, 0.969}\color{fgcolor}\begin{figure}[bht]

{\centering \includegraphics[width=.9\textwidth]{../Main/Figures/LineSubMedianBurundi} 

}

\caption[Burundi]{Burundi: Smoothed regional estimates over time compared to the direct estimates from each surveys. Direct estimates are not benchmarked with UN estimates. The line indicates posterior median and error bars indicate 95\% posterior credible interval.}\label{fig:unnamed-chunk-37}
\end{figure}


\end{knitrout}
% \subsubsection{National model results}
We further assess the RW2 model by holding out some observations, and compare the projections to the direct estimates in these holdout observations. Figure~\ref{fig:unnamed-chunk-38} compares the predicted estimates for the out-of-sample observations  with the direct estimates by holding out observations from each area in each time period.  Figure~\ref{fig:unnamed-chunk-39} compares the histogram of the bias rescaled by the total variance in the cross validation studies. Figure~\ref{fig:unnamed-chunk-40} compares the rescaled bias by region and time periods.



% %%%%%%%%%%%%%%%%%%%%%%%%%%% Plot6
% << echo=FALSE, out.width = ".9\\textwidth", fig.width = 12, fig.height = 6, fig.cap = "Out-of-sample predictions along with direct estimates in the cross validation study where all data from each time period is held out and predicted using the rest of the data.">>=
% fig_count <- fig_count + 1
% knitr::include_graphics(paste0("../Main/Figures/CV_byYear_withError_", countryname2, ".pdf")) 
% @
 
%%%%%%%%%%%%%%%%%%%%%%%%%%% Plot7
\begin{knitrout}
\definecolor{shadecolor}{rgb}{0.969, 0.969, 0.969}\color{fgcolor}\begin{figure}[bht]

{\centering \includegraphics[width=.9\textwidth]{../Main/Figures/CV_byYearRegion_withError_Burundi} 

}

\caption[Burundi]{Burundi: Out-of-sample predictions along with direct estimates in the cross validation study where data from one region in each time period is held out and predicted using the rest of the data.}\label{fig:unnamed-chunk-38}
\end{figure}


\end{knitrout}

%%%%%%%%%%%%%%%%%%%%%%%%%%% Plot8
\begin{knitrout}
\definecolor{shadecolor}{rgb}{0.969, 0.969, 0.969}\color{fgcolor}\begin{figure}[bht]

{\centering \includegraphics[width=.9\textwidth]{../Main/Figures/CVbiasBurundi} 

}

\caption[Burundi]{Burundi: Histogram and QQ-plot of the rescaled difference between the smoothed estimates and the direct estimates in the cross validation study. The differences between the two estimates are rescaled by the square root of the total variance of the two estimates.}\label{fig:unnamed-chunk-39}
\end{figure}


\end{knitrout}

%%%%%%%%%%%%%%%%%%%%%%%%%%% Plot9
\begin{knitrout}
\definecolor{shadecolor}{rgb}{0.969, 0.969, 0.969}\color{fgcolor}\begin{figure}[bht]

{\centering \includegraphics[width=.7\textwidth]{../Main/Figures/CVbiasbyRegionBurundi} 

}

\caption[Burundi]{Burundi: Line plot of the difference between smoothed estimates and the direct estimates in the cross validation study. The differences between the two estimates are rescaled by the square root of the total variance of the two estimates.}\label{fig:unnamed-chunk-40}
\end{figure}


\end{knitrout}

%%%%%%%%%%%%%%%%%%%%%%%%%%%%%%%%%%%%%%%%%%%%%%%%%%%%%%%%%%%%%%%%%%%%%%%%%%%%%%%%%%%%%%%%%%%%%%%%%
\clearpage
\subsubsection{Cameroon}


% \subsubsection{Summary of DHS surveys}

%%%%%%%%%%%%%%%%%%%%%%%%%%% Summary 


DHS surveys were conducted in Cameroon in 1998, 2004, and 2011.
% years.out[1:(length(years.out)-1)], and years.out[length(years.out)]. 

We fit both the RW2 only model to the combined national data, and compare the time trend at national level with the estimates produced by the UN and IHME in Figure~\ref{fig:unnamed-chunk-42}. We then adjusted the combined national data to the UN estimates of U5MR, and refit the models on the benchmarked data. 

%%%%%%%%%%%%%%%%%%%%%%%%%% Plot5 
\begin{knitrout}
\definecolor{shadecolor}{rgb}{0.969, 0.969, 0.969}\color{fgcolor}\begin{figure}[bht]

{\centering \includegraphics[width=.9\textwidth]{../Main/Figures/Yearly_national_Cameroon} 

}

\caption[Cameroon]{Cameroon: Temporal national trends along with UN (B3) estimates described in You et al. (2015) and IHME estimates based on GBD 2015 Child Mortality Collaborators (2016). RW2 represents the smoothed national estimates using the original data before benchmarking with UN estimates. RW2-adj represents the smoothed national estimates using the benchmarked data.}\label{fig:unnamed-chunk-42}
\end{figure}


\end{knitrout}
 

We fit the RW2 model to the benchmarked data in each area. 
% The proportions of the explained variation is summarized in Table~\ref{tab:paste0(countryname, "-var")}. 
We compare the results in Figure~\ref{fig:unnamed-chunk-43} to \ref{fig:unnamed-chunk-47}.
Figure~\ref{fig:unnamed-chunk-43} compares the smoothed estimates against the direct estimates. Figure~\ref{fig:unnamed-chunk-44} and Figure~\ref{fig:unnamed-chunk-45} show the posterior median estimates of U5MR in each region over time and the reductions from 1990 period respectively.
Figure~\ref{fig:unnamed-chunk-46} shows the smoothed estimates by region over time and Figure~\ref{fig:unnamed-chunk-47} compares the smoothed estimates with direct estimates from each survey for each region over time.


% %%%%%%%%%%%%%%%%%%%%%%%%%%% Table1 
% <<echo=FALSE, results='asis'>>=
% load("rda/variance_tables.rda")
% countryname2 <- gsub(" ", "", countryname)
% variance <- tables.all[[countryname]]

% table_count <- table_count + 1

% names <- c("RW2 ($\\sigma^2_{\\gamma_{t}}$)", "ICAR ($\\sigma^2_{\\phi_{i}}$)", "IID space ($\\sigma^2_{\\theta_{i}}$)", "IID time ($\\sigma^2_{\\alpha_{t}}$)", "IID space time ($\\sigma^2_{\\delta_{it}}$)")

% variance$Proportion <- round(variance$Proportion*100, digits = 2)
% row.names(variance) <- names
% tab <- xtable(variance, digits = c(1, 3, 2),align = "l|ll",
%        label = paste0("tab:", countryname, "-var"),
%        caption = paste(country, ": summary of the variance components in the RW2 model", sep = ''))
% print(tab, comment = FALSE,sanitize.text.function = function(x) {x})
% @

%%%%%%%%%%%%%%%%%%%%%%%%%%% Plot1 
\begin{knitrout}
\definecolor{shadecolor}{rgb}{0.969, 0.969, 0.969}\color{fgcolor}\begin{figure}[bht]

{\centering \includegraphics[width=.9\textwidth]{../Main/Figures/SmoothvDirectCameroon_meta} 

}

\caption[Cameroon]{Cameroon: Smooth versus direct Admin 1 estimates. Left: Combined (meta-analysis) survey estimate against combined direct estimates. Right: Combined (meta-analysis) survey estimate against direct estimates from each survey.}\label{fig:unnamed-chunk-43}
\end{figure}


\end{knitrout}

%%%%%%%%%%%%%%%%%%%%%%%%%%% Plot2 
\begin{knitrout}
\definecolor{shadecolor}{rgb}{0.969, 0.969, 0.969}\color{fgcolor}\begin{figure}[bht]

{\centering \includegraphics[width=.9\textwidth]{../Main/Figures/SmoothMedianCameroon} 

}

\caption[Cameroon]{Cameroon: Maps of posterior medians over time.}\label{fig:unnamed-chunk-44}
\end{figure}


\end{knitrout}
%%%%%%%%%%%%%%%%%%%%%%%%%%% Plot2a
\begin{knitrout}
\definecolor{shadecolor}{rgb}{0.969, 0.969, 0.969}\color{fgcolor}\begin{figure}[bht]

{\centering \includegraphics[width=.9\textwidth]{../Main/Figures/ReductionMedianCameroon} 

}

\caption[Cameroon]{Cameroon: Maps of reduction of posterior median U5MR in each five-year period compared to 1990 over time.}\label{fig:unnamed-chunk-45}
\end{figure}


\end{knitrout}
%%%%%%%%%%%%%%%%%%%%%%%%%%% Plot3 
\begin{knitrout}
\definecolor{shadecolor}{rgb}{0.969, 0.969, 0.969}\color{fgcolor}\begin{figure}[bht]

{\centering \includegraphics[width=.95\textwidth]{../Main/Figures/Yearly_v_Periods_Cameroon} 

}

\caption[Cameroon]{Cameroon: Smoothed regional estimates over time. The line indicates yearly posterior median estimates and error bars indicate 95 \% posterior credible interval at each time period.}\label{fig:unnamed-chunk-46}
\end{figure}


\end{knitrout}

%%%%%%%%%%%%%%%%%%%%%%%%%%% Plot4 
\begin{knitrout}
\definecolor{shadecolor}{rgb}{0.969, 0.969, 0.969}\color{fgcolor}\begin{figure}[bht]

{\centering \includegraphics[width=.9\textwidth]{../Main/Figures/LineSubMedianCameroon} 

}

\caption[Cameroon]{Cameroon: Smoothed regional estimates over time compared to the direct estimates from each surveys. Direct estimates are not benchmarked with UN estimates. The line indicates posterior median and error bars indicate 95\% posterior credible interval.}\label{fig:unnamed-chunk-47}
\end{figure}


\end{knitrout}
% \subsubsection{National model results}
We further assess the RW2 model by holding out some observations, and compare the projections to the direct estimates in these holdout observations. Figure~\ref{fig:unnamed-chunk-48} compares the predicted estimates for the out-of-sample observations  with the direct estimates by holding out observations from each area in each time period.  Figure~\ref{fig:unnamed-chunk-49} compares the histogram of the bias rescaled by the total variance in the cross validation studies. Figure~\ref{fig:unnamed-chunk-50} compares the rescaled bias by region and time periods.



% %%%%%%%%%%%%%%%%%%%%%%%%%%% Plot6
% << echo=FALSE, out.width = ".9\\textwidth", fig.width = 12, fig.height = 6, fig.cap = "Out-of-sample predictions along with direct estimates in the cross validation study where all data from each time period is held out and predicted using the rest of the data.">>=
% fig_count <- fig_count + 1
% knitr::include_graphics(paste0("../Main/Figures/CV_byYear_withError_", countryname2, ".pdf")) 
% @
 
%%%%%%%%%%%%%%%%%%%%%%%%%%% Plot7
\begin{knitrout}
\definecolor{shadecolor}{rgb}{0.969, 0.969, 0.969}\color{fgcolor}\begin{figure}[bht]

{\centering \includegraphics[width=.9\textwidth]{../Main/Figures/CV_byYearRegion_withError_Cameroon} 

}

\caption[Cameroon]{Cameroon: Out-of-sample predictions along with direct estimates in the cross validation study where data from one region in each time period is held out and predicted using the rest of the data.}\label{fig:unnamed-chunk-48}
\end{figure}


\end{knitrout}

%%%%%%%%%%%%%%%%%%%%%%%%%%% Plot8
\begin{knitrout}
\definecolor{shadecolor}{rgb}{0.969, 0.969, 0.969}\color{fgcolor}\begin{figure}[bht]

{\centering \includegraphics[width=.9\textwidth]{../Main/Figures/CVbiasCameroon} 

}

\caption[Cameroon]{Cameroon: Histogram and QQ-plot of the rescaled difference between the smoothed estimates and the direct estimates in the cross validation study. The differences between the two estimates are rescaled by the square root of the total variance of the two estimates.}\label{fig:unnamed-chunk-49}
\end{figure}


\end{knitrout}

%%%%%%%%%%%%%%%%%%%%%%%%%%% Plot9
\begin{knitrout}
\definecolor{shadecolor}{rgb}{0.969, 0.969, 0.969}\color{fgcolor}\begin{figure}[bht]

{\centering \includegraphics[width=.7\textwidth]{../Main/Figures/CVbiasbyRegionCameroon} 

}

\caption[Cameroon]{Cameroon: Line plot of the difference between smoothed estimates and the direct estimates in the cross validation study. The differences between the two estimates are rescaled by the square root of the total variance of the two estimates.}\label{fig:unnamed-chunk-50}
\end{figure}


\end{knitrout}


%%%%%%%%%%%%%%%%%%%%%%%%%%%%%%%%%%%%%%%%%%%%%%%%%%%%%%%%%%%%%%%%%%%%%%%%%%%%%%%%%%%%%%%%%%%%%%%%%%
\clearpage
\subsubsection{Chad}


% \subsubsection{Summary of DHS surveys}

%%%%%%%%%%%%%%%%%%%%%%%%%%% Summary 


DHS surveys were conducted in Chad in 2004, and 2015.
% years.out[1:(length(years.out)-1)], and years.out[length(years.out)]. 

We fit both the RW2 only model to the combined national data, and compare the time trend at national level with the estimates produced by the UN and IHME in Figure~\ref{fig:unnamed-chunk-52}. We then adjusted the combined national data to the UN estimates of U5MR, and refit the models on the benchmarked data. 

%%%%%%%%%%%%%%%%%%%%%%%%%% Plot5 
\begin{knitrout}
\definecolor{shadecolor}{rgb}{0.969, 0.969, 0.969}\color{fgcolor}\begin{figure}[bht]

{\centering \includegraphics[width=.9\textwidth]{../Main/Figures/Yearly_national_Chad} 

}

\caption[Chad]{Chad: Temporal national trends along with UN (B3) estimates described in You et al. (2015) and IHME estimates based on GBD 2015 Child Mortality Collaborators (2016). RW2 represents the smoothed national estimates using the original data before benchmarking with UN estimates. RW2-adj represents the smoothed national estimates using the benchmarked data.}\label{fig:unnamed-chunk-52}
\end{figure}


\end{knitrout}
 

We fit the RW2 model to the benchmarked data in each area. 
% The proportions of the explained variation is summarized in Table~\ref{tab:paste0(countryname, "-var")}. 
We compare the results in Figure~\ref{fig:unnamed-chunk-53} to \ref{fig:unnamed-chunk-57}.
Figure~\ref{fig:unnamed-chunk-53} compares the smoothed estimates against the direct estimates. Figure~\ref{fig:unnamed-chunk-54} and Figure~\ref{fig:unnamed-chunk-55} show the posterior median estimates of U5MR in each region over time and the reductions from 1990 period respectively.
Figure~\ref{fig:unnamed-chunk-56} shows the smoothed estimates by region over time and Figure~\ref{fig:unnamed-chunk-57} compares the smoothed estimates with direct estimates from each survey for each region over time.


% %%%%%%%%%%%%%%%%%%%%%%%%%%% Table1 
% <<echo=FALSE, results='asis'>>=
% load("rda/variance_tables.rda")
% countryname2 <- gsub(" ", "", countryname)
% variance <- tables.all[[countryname]]

% table_count <- table_count + 1

% names <- c("RW2 ($\\sigma^2_{\\gamma_{t}}$)", "ICAR ($\\sigma^2_{\\phi_{i}}$)", "IID space ($\\sigma^2_{\\theta_{i}}$)", "IID time ($\\sigma^2_{\\alpha_{t}}$)", "IID space time ($\\sigma^2_{\\delta_{it}}$)")

% variance$Proportion <- round(variance$Proportion*100, digits = 2)
% row.names(variance) <- names
% tab <- xtable(variance, digits = c(1, 3, 2),align = "l|ll",
%        label = paste0("tab:", countryname, "-var"),
%        caption = paste(country, ": summary of the variance components in the RW2 model", sep = ''))
% print(tab, comment = FALSE,sanitize.text.function = function(x) {x})
% @

%%%%%%%%%%%%%%%%%%%%%%%%%%% Plot1 
\begin{knitrout}
\definecolor{shadecolor}{rgb}{0.969, 0.969, 0.969}\color{fgcolor}\begin{figure}[bht]

{\centering \includegraphics[width=.9\textwidth]{../Main/Figures/SmoothvDirectChad_meta} 

}

\caption[Chad]{Chad: Smooth versus direct Admin 1 estimates. Left: Combined (meta-analysis) survey estimate against combined direct estimates. Right: Combined (meta-analysis) survey estimate against direct estimates from each survey.}\label{fig:unnamed-chunk-53}
\end{figure}


\end{knitrout}

%%%%%%%%%%%%%%%%%%%%%%%%%%% Plot2 
\begin{knitrout}
\definecolor{shadecolor}{rgb}{0.969, 0.969, 0.969}\color{fgcolor}\begin{figure}[bht]

{\centering \includegraphics[width=.9\textwidth]{../Main/Figures/SmoothMedianChad} 

}

\caption[Chad]{Chad: Maps of posterior medians over time.}\label{fig:unnamed-chunk-54}
\end{figure}


\end{knitrout}
%%%%%%%%%%%%%%%%%%%%%%%%%%% Plot2a
\begin{knitrout}
\definecolor{shadecolor}{rgb}{0.969, 0.969, 0.969}\color{fgcolor}\begin{figure}[bht]

{\centering \includegraphics[width=.9\textwidth]{../Main/Figures/ReductionMedianChad} 

}

\caption[Chad]{Chad: Maps of reduction of posterior median U5MR in each five-year period compared to 1990 over time.}\label{fig:unnamed-chunk-55}
\end{figure}


\end{knitrout}
%%%%%%%%%%%%%%%%%%%%%%%%%%% Plot3 
\begin{knitrout}
\definecolor{shadecolor}{rgb}{0.969, 0.969, 0.969}\color{fgcolor}\begin{figure}[bht]

{\centering \includegraphics[width=.95\textwidth]{../Main/Figures/Yearly_v_Periods_Chad} 

}

\caption[Chad]{Chad: Smoothed regional estimates over time. The line indicates yearly posterior median estimates and error bars indicate 95 \% posterior credible interval at each time period.}\label{fig:unnamed-chunk-56}
\end{figure}


\end{knitrout}

%%%%%%%%%%%%%%%%%%%%%%%%%%% Plot4 
\begin{knitrout}
\definecolor{shadecolor}{rgb}{0.969, 0.969, 0.969}\color{fgcolor}\begin{figure}[bht]

{\centering \includegraphics[width=.9\textwidth]{../Main/Figures/LineSubMedianChad} 

}

\caption[Chad]{Chad: Smoothed regional estimates over time compared to the direct estimates from each surveys. Direct estimates are not benchmarked with UN estimates. The line indicates posterior median and error bars indicate 95\% posterior credible interval.}\label{fig:unnamed-chunk-57}
\end{figure}


\end{knitrout}
% \subsubsection{National model results}
We further assess the RW2 model by holding out some observations, and compare the projections to the direct estimates in these holdout observations. Figure~\ref{fig:unnamed-chunk-58} compares the predicted estimates for the out-of-sample observations  with the direct estimates by holding out observations from each area in each time period.  Figure~\ref{fig:unnamed-chunk-59} compares the histogram of the bias rescaled by the total variance in the cross validation studies. Figure~\ref{fig:unnamed-chunk-60} compares the rescaled bias by region and time periods.



% %%%%%%%%%%%%%%%%%%%%%%%%%%% Plot6
% << echo=FALSE, out.width = ".9\\textwidth", fig.width = 12, fig.height = 6, fig.cap = "Out-of-sample predictions along with direct estimates in the cross validation study where all data from each time period is held out and predicted using the rest of the data.">>=
% fig_count <- fig_count + 1
% knitr::include_graphics(paste0("../Main/Figures/CV_byYear_withError_", countryname2, ".pdf")) 
% @
 
%%%%%%%%%%%%%%%%%%%%%%%%%%% Plot7
\begin{knitrout}
\definecolor{shadecolor}{rgb}{0.969, 0.969, 0.969}\color{fgcolor}\begin{figure}[bht]

{\centering \includegraphics[width=.9\textwidth]{../Main/Figures/CV_byYearRegion_withError_Chad} 

}

\caption[Chad]{Chad: Out-of-sample predictions along with direct estimates in the cross validation study where data from one region in each time period is held out and predicted using the rest of the data.}\label{fig:unnamed-chunk-58}
\end{figure}


\end{knitrout}

%%%%%%%%%%%%%%%%%%%%%%%%%%% Plot8
\begin{knitrout}
\definecolor{shadecolor}{rgb}{0.969, 0.969, 0.969}\color{fgcolor}\begin{figure}[bht]

{\centering \includegraphics[width=.9\textwidth]{../Main/Figures/CVbiasChad} 

}

\caption[Chad]{Chad: Histogram and QQ-plot of the rescaled difference between the smoothed estimates and the direct estimates in the cross validation study. The differences between the two estimates are rescaled by the square root of the total variance of the two estimates.}\label{fig:unnamed-chunk-59}
\end{figure}


\end{knitrout}

%%%%%%%%%%%%%%%%%%%%%%%%%%% Plot9
\begin{knitrout}
\definecolor{shadecolor}{rgb}{0.969, 0.969, 0.969}\color{fgcolor}\begin{figure}[bht]

{\centering \includegraphics[width=.7\textwidth]{../Main/Figures/CVbiasbyRegionChad} 

}

\caption[Chad]{Chad: Line plot of the difference between smoothed estimates and the direct estimates in the cross validation study. The differences between the two estimates are rescaled by the square root of the total variance of the two estimates.}\label{fig:unnamed-chunk-60}
\end{figure}


\end{knitrout}

%%%%%%%%%%%%%%%%%%%%%%%%%%%%%%%%%%%%%%%%%%%%%%%%%%%%%%%%%%%%%%%%%%%%%%%%%%%%%%%%%%%%%%%%%%%%%%%%%%
\clearpage
\subsubsection{Comoros}


% \subsubsection{Summary of DHS surveys}

%%%%%%%%%%%%%%%%%%%%%%%%%%% Summary 


DHS surveys were conducted in Comoros in 1996, and 2012.
% years.out[1:(length(years.out)-1)], and years.out[length(years.out)]. 

We fit both the RW2 only model to the combined national data, and compare the time trend at national level with the estimates produced by the UN and IHME in Figure~\ref{fig:unnamed-chunk-62}. We then adjusted the combined national data to the UN estimates of U5MR, and refit the models on the benchmarked data. 

%%%%%%%%%%%%%%%%%%%%%%%%%% Plot5 
\begin{knitrout}
\definecolor{shadecolor}{rgb}{0.969, 0.969, 0.969}\color{fgcolor}\begin{figure}[bht]

{\centering \includegraphics[width=.9\textwidth]{../Main/Figures/Yearly_national_Comoros} 

}

\caption[Comoros]{Comoros: Temporal national trends along with UN (B3) estimates described in You et al. (2015) and IHME estimates based on GBD 2015 Child Mortality Collaborators (2016). RW2 represents the smoothed national estimates using the original data before benchmarking with UN estimates. RW2-adj represents the smoothed national estimates using the benchmarked data.}\label{fig:unnamed-chunk-62}
\end{figure}


\end{knitrout}
 

We fit the RW2 model to the benchmarked data in each area. 
% The proportions of the explained variation is summarized in Table~\ref{tab:paste0(countryname, "-var")}. 
We compare the results in Figure~\ref{fig:unnamed-chunk-63} to \ref{fig:unnamed-chunk-67}.
Figure~\ref{fig:unnamed-chunk-63} compares the smoothed estimates against the direct estimates. Figure~\ref{fig:unnamed-chunk-64} and Figure~\ref{fig:unnamed-chunk-65} show the posterior median estimates of U5MR in each region over time and the reductions from 1990 period respectively.
Figure~\ref{fig:unnamed-chunk-66} shows the smoothed estimates by region over time and Figure~\ref{fig:unnamed-chunk-67} compares the smoothed estimates with direct estimates from each survey for each region over time.


% %%%%%%%%%%%%%%%%%%%%%%%%%%% Table1 
% <<echo=FALSE, results='asis'>>=
% load("rda/variance_tables.rda")
% countryname2 <- gsub(" ", "", countryname)
% variance <- tables.all[[countryname]]

% table_count <- table_count + 1

% names <- c("RW2 ($\\sigma^2_{\\gamma_{t}}$)", "ICAR ($\\sigma^2_{\\phi_{i}}$)", "IID space ($\\sigma^2_{\\theta_{i}}$)", "IID time ($\\sigma^2_{\\alpha_{t}}$)", "IID space time ($\\sigma^2_{\\delta_{it}}$)")

% variance$Proportion <- round(variance$Proportion*100, digits = 2)
% row.names(variance) <- names
% tab <- xtable(variance, digits = c(1, 3, 2),align = "l|ll",
%        label = paste0("tab:", countryname, "-var"),
%        caption = paste(country, ": summary of the variance components in the RW2 model", sep = ''))
% print(tab, comment = FALSE,sanitize.text.function = function(x) {x})
% @

%%%%%%%%%%%%%%%%%%%%%%%%%%% Plot1 
\begin{knitrout}
\definecolor{shadecolor}{rgb}{0.969, 0.969, 0.969}\color{fgcolor}\begin{figure}[bht]

{\centering \includegraphics[width=.9\textwidth]{../Main/Figures/SmoothvDirectComoros_meta} 

}

\caption[Comoros]{Comoros: Smooth versus direct Admin 1 estimates. Left: Combined (meta-analysis) survey estimate against combined direct estimates. Right: Combined (meta-analysis) survey estimate against direct estimates from each survey.}\label{fig:unnamed-chunk-63}
\end{figure}


\end{knitrout}

%%%%%%%%%%%%%%%%%%%%%%%%%%% Plot2 
\begin{knitrout}
\definecolor{shadecolor}{rgb}{0.969, 0.969, 0.969}\color{fgcolor}\begin{figure}[bht]

{\centering \includegraphics[width=.9\textwidth]{../Main/Figures/SmoothMedianComoros} 

}

\caption[Comoros]{Comoros: Maps of posterior medians over time.}\label{fig:unnamed-chunk-64}
\end{figure}


\end{knitrout}
%%%%%%%%%%%%%%%%%%%%%%%%%%% Plot2a
\begin{knitrout}
\definecolor{shadecolor}{rgb}{0.969, 0.969, 0.969}\color{fgcolor}\begin{figure}[bht]

{\centering \includegraphics[width=.9\textwidth]{../Main/Figures/ReductionMedianComoros} 

}

\caption[Comoros]{Comoros: Maps of reduction of posterior median U5MR in each five-year period compared to 1990 over time.}\label{fig:unnamed-chunk-65}
\end{figure}


\end{knitrout}
%%%%%%%%%%%%%%%%%%%%%%%%%%% Plot3 
\begin{knitrout}
\definecolor{shadecolor}{rgb}{0.969, 0.969, 0.969}\color{fgcolor}\begin{figure}[bht]

{\centering \includegraphics[width=.95\textwidth]{../Main/Figures/Yearly_v_Periods_Comoros} 

}

\caption[Comoros]{Comoros: Smoothed regional estimates over time. The line indicates yearly posterior median estimates and error bars indicate 95 \% posterior credible interval at each time period.}\label{fig:unnamed-chunk-66}
\end{figure}


\end{knitrout}

%%%%%%%%%%%%%%%%%%%%%%%%%%% Plot4 
\begin{knitrout}
\definecolor{shadecolor}{rgb}{0.969, 0.969, 0.969}\color{fgcolor}\begin{figure}[bht]

{\centering \includegraphics[width=.9\textwidth]{../Main/Figures/LineSubMedianComoros} 

}

\caption[Comoros]{Comoros: Smoothed regional estimates over time compared to the direct estimates from each surveys. Direct estimates are not benchmarked with UN estimates. The line indicates posterior median and error bars indicate 95\% posterior credible interval.}\label{fig:unnamed-chunk-67}
\end{figure}


\end{knitrout}
% \subsubsection{National model results}
We further assess the RW2 model by holding out some observations, and compare the projections to the direct estimates in these holdout observations. Figure~\ref{fig:unnamed-chunk-68} compares the predicted estimates for the out-of-sample observations  with the direct estimates by holding out observations from each area in each time period.  Figure~\ref{fig:unnamed-chunk-69} compares the histogram of the bias rescaled by the total variance in the cross validation studies. Figure~\ref{fig:unnamed-chunk-70} compares the rescaled bias by region and time periods.



% %%%%%%%%%%%%%%%%%%%%%%%%%%% Plot6
% << echo=FALSE, out.width = ".9\\textwidth", fig.width = 12, fig.height = 6, fig.cap = "Out-of-sample predictions along with direct estimates in the cross validation study where all data from each time period is held out and predicted using the rest of the data.">>=
% fig_count <- fig_count + 1
% knitr::include_graphics(paste0("../Main/Figures/CV_byYear_withError_", countryname2, ".pdf")) 
% @
 
%%%%%%%%%%%%%%%%%%%%%%%%%%% Plot7
\begin{knitrout}
\definecolor{shadecolor}{rgb}{0.969, 0.969, 0.969}\color{fgcolor}\begin{figure}[bht]

{\centering \includegraphics[width=.9\textwidth]{../Main/Figures/CV_byYearRegion_withError_Comoros} 

}

\caption[Comoros]{Comoros: Out-of-sample predictions along with direct estimates in the cross validation study where data from one region in each time period is held out and predicted using the rest of the data.}\label{fig:unnamed-chunk-68}
\end{figure}


\end{knitrout}

%%%%%%%%%%%%%%%%%%%%%%%%%%% Plot8
\begin{knitrout}
\definecolor{shadecolor}{rgb}{0.969, 0.969, 0.969}\color{fgcolor}\begin{figure}[bht]

{\centering \includegraphics[width=.9\textwidth]{../Main/Figures/CVbiasComoros} 

}

\caption[Comoros]{Comoros: Histogram and QQ-plot of the rescaled difference between the smoothed estimates and the direct estimates in the cross validation study. The differences between the two estimates are rescaled by the square root of the total variance of the two estimates.}\label{fig:unnamed-chunk-69}
\end{figure}


\end{knitrout}

%%%%%%%%%%%%%%%%%%%%%%%%%%% Plot9
\begin{knitrout}
\definecolor{shadecolor}{rgb}{0.969, 0.969, 0.969}\color{fgcolor}\begin{figure}[bht]

{\centering \includegraphics[width=.7\textwidth]{../Main/Figures/CVbiasbyRegionComoros} 

}

\caption[Comoros]{Comoros: Line plot of the difference between smoothed estimates and the direct estimates in the cross validation study. The differences between the two estimates are rescaled by the square root of the total variance of the two estimates.}\label{fig:unnamed-chunk-70}
\end{figure}


\end{knitrout}

%%%%%%%%%%%%%%%%%%%%%%%%%%%%%%%%%%%%%%%%%%%%%%%%%%%%%%%%%%%%%%%%%%%%%%%%%%%%%%%%%%%%%%%%%%%%%%%%%%
\clearpage
\subsubsection{Congo}


% \subsubsection{Summary of DHS surveys}

%%%%%%%%%%%%%%%%%%%%%%%%%%% Summary 


DHS surveys were conducted in Congo in 2005, and 2012.
% years.out[1:(length(years.out)-1)], and years.out[length(years.out)]. 

We fit both the RW2 only model to the combined national data, and compare the time trend at national level with the estimates produced by the UN and IHME in Figure~\ref{fig:unnamed-chunk-72}. We then adjusted the combined national data to the UN estimates of U5MR, and refit the models on the benchmarked data. 

%%%%%%%%%%%%%%%%%%%%%%%%%% Plot5 
\begin{knitrout}
\definecolor{shadecolor}{rgb}{0.969, 0.969, 0.969}\color{fgcolor}\begin{figure}[bht]

{\centering \includegraphics[width=.9\textwidth]{../Main/Figures/Yearly_national_Congo} 

}

\caption[Congo]{Congo: Temporal national trends along with UN (B3) estimates described in You et al. (2015) and IHME estimates based on GBD 2015 Child Mortality Collaborators (2016). RW2 represents the smoothed national estimates using the original data before benchmarking with UN estimates. RW2-adj represents the smoothed national estimates using the benchmarked data.}\label{fig:unnamed-chunk-72}
\end{figure}


\end{knitrout}
 

We fit the RW2 model to the benchmarked data in each area. 
% The proportions of the explained variation is summarized in Table~\ref{tab:paste0(countryname, "-var")}. 
We compare the results in Figure~\ref{fig:unnamed-chunk-73} to \ref{fig:unnamed-chunk-77}.
Figure~\ref{fig:unnamed-chunk-73} compares the smoothed estimates against the direct estimates. Figure~\ref{fig:unnamed-chunk-74} and Figure~\ref{fig:unnamed-chunk-75} show the posterior median estimates of U5MR in each region over time and the reductions from 1990 period respectively.
Figure~\ref{fig:unnamed-chunk-76} shows the smoothed estimates by region over time and Figure~\ref{fig:unnamed-chunk-77} compares the smoothed estimates with direct estimates from each survey for each region over time.


% %%%%%%%%%%%%%%%%%%%%%%%%%%% Table1 
% <<echo=FALSE, results='asis'>>=
% load("rda/variance_tables.rda")
% countryname2 <- gsub(" ", "", countryname)
% variance <- tables.all[[countryname]]

% table_count <- table_count + 1

% names <- c("RW2 ($\\sigma^2_{\\gamma_{t}}$)", "ICAR ($\\sigma^2_{\\phi_{i}}$)", "IID space ($\\sigma^2_{\\theta_{i}}$)", "IID time ($\\sigma^2_{\\alpha_{t}}$)", "IID space time ($\\sigma^2_{\\delta_{it}}$)")

% variance$Proportion <- round(variance$Proportion*100, digits = 2)
% row.names(variance) <- names
% tab <- xtable(variance, digits = c(1, 3, 2),align = "l|ll",
%        label = paste0("tab:", countryname, "-var"),
%        caption = paste(country, ": summary of the variance components in the RW2 model", sep = ''))
% print(tab, comment = FALSE,sanitize.text.function = function(x) {x})
% @

%%%%%%%%%%%%%%%%%%%%%%%%%%% Plot1 
\begin{knitrout}
\definecolor{shadecolor}{rgb}{0.969, 0.969, 0.969}\color{fgcolor}\begin{figure}[bht]

{\centering \includegraphics[width=.9\textwidth]{../Main/Figures/SmoothvDirectCongo_meta} 

}

\caption[Congo]{Congo: Smooth versus direct Admin 1 estimates. Left: Combined (meta-analysis) survey estimate against combined direct estimates. Right: Combined (meta-analysis) survey estimate against direct estimates from each survey.}\label{fig:unnamed-chunk-73}
\end{figure}


\end{knitrout}

%%%%%%%%%%%%%%%%%%%%%%%%%%% Plot2 
\begin{knitrout}
\definecolor{shadecolor}{rgb}{0.969, 0.969, 0.969}\color{fgcolor}\begin{figure}[bht]

{\centering \includegraphics[width=.9\textwidth]{../Main/Figures/SmoothMedianCongo} 

}

\caption[Congo]{Congo: Maps of posterior medians over time.}\label{fig:unnamed-chunk-74}
\end{figure}


\end{knitrout}
%%%%%%%%%%%%%%%%%%%%%%%%%%% Plot2a
\begin{knitrout}
\definecolor{shadecolor}{rgb}{0.969, 0.969, 0.969}\color{fgcolor}\begin{figure}[bht]

{\centering \includegraphics[width=.9\textwidth]{../Main/Figures/ReductionMedianCongo} 

}

\caption[Congo]{Congo: Maps of reduction of posterior median U5MR in each five-year period compared to 1990 over time.}\label{fig:unnamed-chunk-75}
\end{figure}


\end{knitrout}
%%%%%%%%%%%%%%%%%%%%%%%%%%% Plot3 
\begin{knitrout}
\definecolor{shadecolor}{rgb}{0.969, 0.969, 0.969}\color{fgcolor}\begin{figure}[bht]

{\centering \includegraphics[width=.95\textwidth]{../Main/Figures/Yearly_v_Periods_Congo} 

}

\caption[Congo]{Congo: Smoothed regional estimates over time. The line indicates yearly posterior median estimates and error bars indicate 95 \% posterior credible interval at each time period.}\label{fig:unnamed-chunk-76}
\end{figure}


\end{knitrout}

%%%%%%%%%%%%%%%%%%%%%%%%%%% Plot4 
\begin{knitrout}
\definecolor{shadecolor}{rgb}{0.969, 0.969, 0.969}\color{fgcolor}\begin{figure}[bht]

{\centering \includegraphics[width=.9\textwidth]{../Main/Figures/LineSubMedianCongo} 

}

\caption[Congo]{Congo: Smoothed regional estimates over time compared to the direct estimates from each surveys. Direct estimates are not benchmarked with UN estimates. The line indicates posterior median and error bars indicate 95\% posterior credible interval.}\label{fig:unnamed-chunk-77}
\end{figure}


\end{knitrout}
% \subsubsection{National model results}
We further assess the RW2 model by holding out some observations, and compare the projections to the direct estimates in these holdout observations. Figure~\ref{fig:unnamed-chunk-78} compares the predicted estimates for the out-of-sample observations  with the direct estimates by holding out observations from each area in each time period.  Figure~\ref{fig:unnamed-chunk-79} compares the histogram of the bias rescaled by the total variance in the cross validation studies. Figure~\ref{fig:unnamed-chunk-80} compares the rescaled bias by region and time periods.



% %%%%%%%%%%%%%%%%%%%%%%%%%%% Plot6
% << echo=FALSE, out.width = ".9\\textwidth", fig.width = 12, fig.height = 6, fig.cap = "Out-of-sample predictions along with direct estimates in the cross validation study where all data from each time period is held out and predicted using the rest of the data.">>=
% fig_count <- fig_count + 1
% knitr::include_graphics(paste0("../Main/Figures/CV_byYear_withError_", countryname2, ".pdf")) 
% @
 
%%%%%%%%%%%%%%%%%%%%%%%%%%% Plot7
\begin{knitrout}
\definecolor{shadecolor}{rgb}{0.969, 0.969, 0.969}\color{fgcolor}\begin{figure}[bht]

{\centering \includegraphics[width=.9\textwidth]{../Main/Figures/CV_byYearRegion_withError_Congo} 

}

\caption[Congo]{Congo: Out-of-sample predictions along with direct estimates in the cross validation study where data from one region in each time period is held out and predicted using the rest of the data.}\label{fig:unnamed-chunk-78}
\end{figure}


\end{knitrout}

%%%%%%%%%%%%%%%%%%%%%%%%%%% Plot8
\begin{knitrout}
\definecolor{shadecolor}{rgb}{0.969, 0.969, 0.969}\color{fgcolor}\begin{figure}[bht]

{\centering \includegraphics[width=.9\textwidth]{../Main/Figures/CVbiasCongo} 

}

\caption[Congo]{Congo: Histogram and QQ-plot of the rescaled difference between the smoothed estimates and the direct estimates in the cross validation study. The differences between the two estimates are rescaled by the square root of the total variance of the two estimates.}\label{fig:unnamed-chunk-79}
\end{figure}


\end{knitrout}

%%%%%%%%%%%%%%%%%%%%%%%%%%% Plot9
\begin{knitrout}
\definecolor{shadecolor}{rgb}{0.969, 0.969, 0.969}\color{fgcolor}\begin{figure}[bht]

{\centering \includegraphics[width=.7\textwidth]{../Main/Figures/CVbiasbyRegionCongo} 

}

\caption[Congo]{Congo: Line plot of the difference between smoothed estimates and the direct estimates in the cross validation study. The differences between the two estimates are rescaled by the square root of the total variance of the two estimates.}\label{fig:unnamed-chunk-80}
\end{figure}


\end{knitrout}

%%%%%%%%%%%%%%%%%%%%%%%%%%%%%%%%%%%%%%%%%%%%%%%%%%%%%%%%%%%%%%%%%%%%%%%%%%%%%%%%%%%%%%%%%%%%%%%%%%
\clearpage
\subsubsection{C\^{o}te d'Ivoire}


% \subsubsection{Summary of DHS surveys}

%%%%%%%%%%%%%%%%%%%%%%%%%%% Summary 


DHS surveys were conducted in C\^{o}te d'Ivoire in 2011.
% years.out[1:(length(years.out)-1)], and years.out[length(years.out)]. 

We fit both the RW2 only model to the combined national data, and compare the time trend at national level with the estimates produced by the UN and IHME in Figure~\ref{fig:unnamed-chunk-82}. We then adjusted the combined national data to the UN estimates of U5MR, and refit the models on the benchmarked data. 

%%%%%%%%%%%%%%%%%%%%%%%%%% Plot5 
\begin{knitrout}
\definecolor{shadecolor}{rgb}{0.969, 0.969, 0.969}\color{fgcolor}\begin{figure}[bht]

{\centering \includegraphics[width=.9\textwidth]{../Main/Figures/Yearly_national_Cote_dIvoire} 

}

\caption[C\^{o}te d'Ivoire]{C\^{o}te d'Ivoire: Temporal national trends along with UN (B3) estimates described in You et al. (2015) and IHME estimates based on GBD 2015 Child Mortality Collaborators (2016). RW2 represents the smoothed national estimates using the original data before benchmarking with UN estimates. RW2-adj represents the smoothed national estimates using the benchmarked data.}\label{fig:unnamed-chunk-82}
\end{figure}


\end{knitrout}
 

We fit the RW2 model to the benchmarked data in each area. 
% The proportions of the explained variation is summarized in Table~\ref{tab:paste0(countryname, "-var")}. 
We compare the results in Figure~\ref{fig:unnamed-chunk-83} to \ref{fig:unnamed-chunk-87}.
Figure~\ref{fig:unnamed-chunk-83} compares the smoothed estimates against the direct estimates. Figure~\ref{fig:unnamed-chunk-84} and Figure~\ref{fig:unnamed-chunk-85} show the posterior median estimates of U5MR in each region over time and the reductions from 1990 period respectively.
Figure~\ref{fig:unnamed-chunk-86} shows the smoothed estimates by region over time and Figure~\ref{fig:unnamed-chunk-87} compares the smoothed estimates with direct estimates from each survey for each region over time.


% %%%%%%%%%%%%%%%%%%%%%%%%%%% Table1 
% <<echo=FALSE, results='asis'>>=
% load("rda/variance_tables.rda")
% countryname2 <- gsub(" ", "", countryname)
% variance <- tables.all[[countryname]]

% table_count <- table_count + 1

% names <- c("RW2 ($\\sigma^2_{\\gamma_{t}}$)", "ICAR ($\\sigma^2_{\\phi_{i}}$)", "IID space ($\\sigma^2_{\\theta_{i}}$)", "IID time ($\\sigma^2_{\\alpha_{t}}$)", "IID space time ($\\sigma^2_{\\delta_{it}}$)")

% variance$Proportion <- round(variance$Proportion*100, digits = 2)
% row.names(variance) <- names
% tab <- xtable(variance, digits = c(1, 3, 2),align = "l|ll",
%        label = paste0("tab:", countryname, "-var"),
%        caption = paste(country, ": summary of the variance components in the RW2 model", sep = ''))
% print(tab, comment = FALSE,sanitize.text.function = function(x) {x})
% @

%%%%%%%%%%%%%%%%%%%%%%%%%%% Plot1 
\begin{knitrout}
\definecolor{shadecolor}{rgb}{0.969, 0.969, 0.969}\color{fgcolor}\begin{figure}[bht]

{\centering \includegraphics[width=.9\textwidth]{../Main/Figures/SmoothvDirectCote_dIvoire_meta} 

}

\caption[C\^{o}te d'Ivoire]{C\^{o}te d'Ivoire: Smooth versus direct Admin 1 estimates. Left: Combined (meta-analysis) survey estimate against combined direct estimates. Right: Combined (meta-analysis) survey estimate against direct estimates from each survey.}\label{fig:unnamed-chunk-83}
\end{figure}


\end{knitrout}

%%%%%%%%%%%%%%%%%%%%%%%%%%% Plot2 
\begin{knitrout}
\definecolor{shadecolor}{rgb}{0.969, 0.969, 0.969}\color{fgcolor}\begin{figure}[bht]

{\centering \includegraphics[width=.9\textwidth]{../Main/Figures/SmoothMedianCote_dIvoire} 

}

\caption[C\^{o}te d'Ivoire]{C\^{o}te d'Ivoire: Maps of posterior medians over time.}\label{fig:unnamed-chunk-84}
\end{figure}


\end{knitrout}
%%%%%%%%%%%%%%%%%%%%%%%%%%% Plot2a
\begin{knitrout}
\definecolor{shadecolor}{rgb}{0.969, 0.969, 0.969}\color{fgcolor}\begin{figure}[bht]

{\centering \includegraphics[width=.9\textwidth]{../Main/Figures/ReductionMedianCote_dIvoire} 

}

\caption[C\^{o}te d'Ivoire]{C\^{o}te d'Ivoire: Maps of reduction of posterior median U5MR in each five-year period compared to 1990 over time.}\label{fig:unnamed-chunk-85}
\end{figure}


\end{knitrout}
%%%%%%%%%%%%%%%%%%%%%%%%%%% Plot3 
\begin{knitrout}
\definecolor{shadecolor}{rgb}{0.969, 0.969, 0.969}\color{fgcolor}\begin{figure}[bht]

{\centering \includegraphics[width=.95\textwidth]{../Main/Figures/Yearly_v_Periods_Cote_dIvoire} 

}

\caption[C\^{o}te d'Ivoire]{C\^{o}te d'Ivoire: Smoothed regional estimates over time. The line indicates yearly posterior median estimates and error bars indicate 95 \% posterior credible interval at each time period.}\label{fig:unnamed-chunk-86}
\end{figure}


\end{knitrout}

%%%%%%%%%%%%%%%%%%%%%%%%%%% Plot4 
\begin{knitrout}
\definecolor{shadecolor}{rgb}{0.969, 0.969, 0.969}\color{fgcolor}\begin{figure}[bht]

{\centering \includegraphics[width=.9\textwidth]{../Main/Figures/LineSubMedianCote_dIvoire} 

}

\caption[C\^{o}te d'Ivoire]{C\^{o}te d'Ivoire: Smoothed regional estimates over time compared to the direct estimates from each surveys. Direct estimates are not benchmarked with UN estimates. The line indicates posterior median and error bars indicate 95\% posterior credible interval.}\label{fig:unnamed-chunk-87}
\end{figure}


\end{knitrout}
% \subsubsection{National model results}
We further assess the RW2 model by holding out some observations, and compare the projections to the direct estimates in these holdout observations. Figure~\ref{fig:unnamed-chunk-88} compares the predicted estimates for the out-of-sample observations  with the direct estimates by holding out observations from each area in each time period.  Figure~\ref{fig:unnamed-chunk-89} compares the histogram of the bias rescaled by the total variance in the cross validation studies. Figure~\ref{fig:unnamed-chunk-90} compares the rescaled bias by region and time periods.



% %%%%%%%%%%%%%%%%%%%%%%%%%%% Plot6
% << echo=FALSE, out.width = ".9\\textwidth", fig.width = 12, fig.height = 6, fig.cap = "Out-of-sample predictions along with direct estimates in the cross validation study where all data from each time period is held out and predicted using the rest of the data.">>=
% fig_count <- fig_count + 1
% knitr::include_graphics(paste0("../Main/Figures/CV_byYear_withError_", countryname2, ".pdf")) 
% @
 
%%%%%%%%%%%%%%%%%%%%%%%%%%% Plot7
\begin{knitrout}
\definecolor{shadecolor}{rgb}{0.969, 0.969, 0.969}\color{fgcolor}\begin{figure}[bht]

{\centering \includegraphics[width=.9\textwidth]{../Main/Figures/CV_byYearRegion_withError_Cote_dIvoire} 

}

\caption[C\^{o}te d'Ivoire]{C\^{o}te d'Ivoire: Out-of-sample predictions along with direct estimates in the cross validation study where data from one region in each time period is held out and predicted using the rest of the data.}\label{fig:unnamed-chunk-88}
\end{figure}


\end{knitrout}

%%%%%%%%%%%%%%%%%%%%%%%%%%% Plot8
\begin{knitrout}
\definecolor{shadecolor}{rgb}{0.969, 0.969, 0.969}\color{fgcolor}\begin{figure}[bht]

{\centering \includegraphics[width=.9\textwidth]{../Main/Figures/CVbiasCote_dIvoire} 

}

\caption[C\^{o}te d'Ivoire]{C\^{o}te d'Ivoire: Histogram and QQ-plot of the rescaled difference between the smoothed estimates and the direct estimates in the cross validation study. The differences between the two estimates are rescaled by the square root of the total variance of the two estimates.}\label{fig:unnamed-chunk-89}
\end{figure}


\end{knitrout}

%%%%%%%%%%%%%%%%%%%%%%%%%%% Plot9
\begin{knitrout}
\definecolor{shadecolor}{rgb}{0.969, 0.969, 0.969}\color{fgcolor}\begin{figure}[bht]

{\centering \includegraphics[width=.7\textwidth]{../Main/Figures/CVbiasbyRegionCote_dIvoire} 

}

\caption[C\^{o}te d'Ivoire]{C\^{o}te d'Ivoire: Line plot of the difference between smoothed estimates and the direct estimates in the cross validation study. The differences between the two estimates are rescaled by the square root of the total variance of the two estimates.}\label{fig:unnamed-chunk-90}
\end{figure}


\end{knitrout}



%%%%%%%%%%%%%%%%%%%%%%%%%%%%%%%%%%%%%%%%%%%%%%%%%%%%%%%%%%%%%%%%%%%%%%%%%%%%%%%%%%%%%%%%%%%%%%%%%%
\clearpage
\subsubsection{DRC}


% \subsubsection{Summary of DHS surveys}

%%%%%%%%%%%%%%%%%%%%%%%%%%% Summary 


DHS surveys were conducted in DRC in 2007, and 2014.
% years.out[1:(length(years.out)-1)], and years.out[length(years.out)]. 

We fit both the RW2 only model to the combined national data, and compare the time trend at national level with the estimates produced by the UN and IHME in Figure~\ref{fig:unnamed-chunk-92}. We then adjusted the combined national data to the UN estimates of U5MR, and refit the models on the benchmarked data. 

%%%%%%%%%%%%%%%%%%%%%%%%%% Plot5 
\begin{knitrout}
\definecolor{shadecolor}{rgb}{0.969, 0.969, 0.969}\color{fgcolor}\begin{figure}[bht]

{\centering \includegraphics[width=.9\textwidth]{../Main/Figures/Yearly_national_DRC} 

}

\caption[DRC]{DRC: Temporal national trends along with UN (B3) estimates described in You et al. (2015) and IHME estimates based on GBD 2015 Child Mortality Collaborators (2016). RW2 represents the smoothed national estimates using the original data before benchmarking with UN estimates. RW2-adj represents the smoothed national estimates using the benchmarked data.}\label{fig:unnamed-chunk-92}
\end{figure}


\end{knitrout}
 

We fit the RW2 model to the benchmarked data in each area. 
% The proportions of the explained variation is summarized in Table~\ref{tab:paste0(countryname, "-var")}. 
We compare the results in Figure~\ref{fig:unnamed-chunk-93} to \ref{fig:unnamed-chunk-97}.
Figure~\ref{fig:unnamed-chunk-93} compares the smoothed estimates against the direct estimates. Figure~\ref{fig:unnamed-chunk-94} and Figure~\ref{fig:unnamed-chunk-95} show the posterior median estimates of U5MR in each region over time and the reductions from 1990 period respectively.
Figure~\ref{fig:unnamed-chunk-96} shows the smoothed estimates by region over time and Figure~\ref{fig:unnamed-chunk-97} compares the smoothed estimates with direct estimates from each survey for each region over time.


% %%%%%%%%%%%%%%%%%%%%%%%%%%% Table1 
% <<echo=FALSE, results='asis'>>=
% load("rda/variance_tables.rda")
% countryname2 <- gsub(" ", "", countryname)
% variance <- tables.all[[countryname]]

% table_count <- table_count + 1

% names <- c("RW2 ($\\sigma^2_{\\gamma_{t}}$)", "ICAR ($\\sigma^2_{\\phi_{i}}$)", "IID space ($\\sigma^2_{\\theta_{i}}$)", "IID time ($\\sigma^2_{\\alpha_{t}}$)", "IID space time ($\\sigma^2_{\\delta_{it}}$)")

% variance$Proportion <- round(variance$Proportion*100, digits = 2)
% row.names(variance) <- names
% tab <- xtable(variance, digits = c(1, 3, 2),align = "l|ll",
%        label = paste0("tab:", countryname, "-var"),
%        caption = paste(country, ": summary of the variance components in the RW2 model", sep = ''))
% print(tab, comment = FALSE,sanitize.text.function = function(x) {x})
% @

%%%%%%%%%%%%%%%%%%%%%%%%%%% Plot1 
\begin{knitrout}
\definecolor{shadecolor}{rgb}{0.969, 0.969, 0.969}\color{fgcolor}\begin{figure}[bht]

{\centering \includegraphics[width=.9\textwidth]{../Main/Figures/SmoothvDirectDRC_meta} 

}

\caption[DRC]{DRC: Smooth versus direct Admin 1 estimates. Left: Combined (meta-analysis) survey estimate against combined direct estimates. Right: Combined (meta-analysis) survey estimate against direct estimates from each survey.}\label{fig:unnamed-chunk-93}
\end{figure}


\end{knitrout}

%%%%%%%%%%%%%%%%%%%%%%%%%%% Plot2 
\begin{knitrout}
\definecolor{shadecolor}{rgb}{0.969, 0.969, 0.969}\color{fgcolor}\begin{figure}[bht]

{\centering \includegraphics[width=.9\textwidth]{../Main/Figures/SmoothMedianDRC} 

}

\caption[DRC]{DRC: Maps of posterior medians over time.}\label{fig:unnamed-chunk-94}
\end{figure}


\end{knitrout}
%%%%%%%%%%%%%%%%%%%%%%%%%%% Plot2a
\begin{knitrout}
\definecolor{shadecolor}{rgb}{0.969, 0.969, 0.969}\color{fgcolor}\begin{figure}[bht]

{\centering \includegraphics[width=.9\textwidth]{../Main/Figures/ReductionMedianDRC} 

}

\caption[DRC]{DRC: Maps of reduction of posterior median U5MR in each five-year period compared to 1990 over time.}\label{fig:unnamed-chunk-95}
\end{figure}


\end{knitrout}
%%%%%%%%%%%%%%%%%%%%%%%%%%% Plot3 
\begin{knitrout}
\definecolor{shadecolor}{rgb}{0.969, 0.969, 0.969}\color{fgcolor}\begin{figure}[bht]

{\centering \includegraphics[width=.95\textwidth]{../Main/Figures/Yearly_v_Periods_DRC} 

}

\caption[DRC]{DRC: Smoothed regional estimates over time. The line indicates yearly posterior median estimates and error bars indicate 95 \% posterior credible interval at each time period.}\label{fig:unnamed-chunk-96}
\end{figure}


\end{knitrout}

%%%%%%%%%%%%%%%%%%%%%%%%%%% Plot4 
\begin{knitrout}
\definecolor{shadecolor}{rgb}{0.969, 0.969, 0.969}\color{fgcolor}\begin{figure}[bht]

{\centering \includegraphics[width=.9\textwidth]{../Main/Figures/LineSubMedianDRC} 

}

\caption[DRC]{DRC: Smoothed regional estimates over time compared to the direct estimates from each surveys. Direct estimates are not benchmarked with UN estimates. The line indicates posterior median and error bars indicate 95\% posterior credible interval.}\label{fig:unnamed-chunk-97}
\end{figure}


\end{knitrout}
% \subsubsection{National model results}
We further assess the RW2 model by holding out some observations, and compare the projections to the direct estimates in these holdout observations. Figure~\ref{fig:unnamed-chunk-98} compares the predicted estimates for the out-of-sample observations  with the direct estimates by holding out observations from each area in each time period.  Figure~\ref{fig:unnamed-chunk-99} compares the histogram of the bias rescaled by the total variance in the cross validation studies. Figure~\ref{fig:unnamed-chunk-100} compares the rescaled bias by region and time periods.



% %%%%%%%%%%%%%%%%%%%%%%%%%%% Plot6
% << echo=FALSE, out.width = ".9\\textwidth", fig.width = 12, fig.height = 6, fig.cap = "Out-of-sample predictions along with direct estimates in the cross validation study where all data from each time period is held out and predicted using the rest of the data.">>=
% fig_count <- fig_count + 1
% knitr::include_graphics(paste0("../Main/Figures/CV_byYear_withError_", countryname2, ".pdf")) 
% @
 
%%%%%%%%%%%%%%%%%%%%%%%%%%% Plot7
\begin{knitrout}
\definecolor{shadecolor}{rgb}{0.969, 0.969, 0.969}\color{fgcolor}\begin{figure}[bht]

{\centering \includegraphics[width=.9\textwidth]{../Main/Figures/CV_byYearRegion_withError_DRC} 

}

\caption[DRC]{DRC: Out-of-sample predictions along with direct estimates in the cross validation study where data from one region in each time period is held out and predicted using the rest of the data.}\label{fig:unnamed-chunk-98}
\end{figure}


\end{knitrout}

%%%%%%%%%%%%%%%%%%%%%%%%%%% Plot8
\begin{knitrout}
\definecolor{shadecolor}{rgb}{0.969, 0.969, 0.969}\color{fgcolor}\begin{figure}[bht]

{\centering \includegraphics[width=.9\textwidth]{../Main/Figures/CVbiasDRC} 

}

\caption[DRC]{DRC: Histogram and QQ-plot of the rescaled difference between the smoothed estimates and the direct estimates in the cross validation study. The differences between the two estimates are rescaled by the square root of the total variance of the two estimates.}\label{fig:unnamed-chunk-99}
\end{figure}


\end{knitrout}

%%%%%%%%%%%%%%%%%%%%%%%%%%% Plot9
\begin{knitrout}
\definecolor{shadecolor}{rgb}{0.969, 0.969, 0.969}\color{fgcolor}\begin{figure}[bht]

{\centering \includegraphics[width=.7\textwidth]{../Main/Figures/CVbiasbyRegionDRC} 

}

\caption[DRC]{DRC: Line plot of the difference between smoothed estimates and the direct estimates in the cross validation study. The differences between the two estimates are rescaled by the square root of the total variance of the two estimates.}\label{fig:unnamed-chunk-100}
\end{figure}


\end{knitrout}


%%%%%%%%%%%%%%%%%%%%%%%%%%%%%%%%%%%%%%%%%%%%%%%%%%%%%%%%%%%%%%%%%%%%%%%%%%%%%%%%%%%%%%%%%%%%%%%%%%
\clearpage
\subsubsection{Egypt}


% \subsubsection{Summary of DHS surveys}

%%%%%%%%%%%%%%%%%%%%%%%%%%% Summary 


DHS surveys were conducted in Egypt in 1988, 1992, 1995, 2000, 2003, 2005, 2008, and 2014.
% years.out[1:(length(years.out)-1)], and years.out[length(years.out)]. 

We fit both the RW2 only model to the combined national data, and compare the time trend at national level with the estimates produced by the UN and IHME in Figure~\ref{fig:unnamed-chunk-102}. We then adjusted the combined national data to the UN estimates of U5MR, and refit the models on the benchmarked data. 

%%%%%%%%%%%%%%%%%%%%%%%%%% Plot5 
\begin{knitrout}
\definecolor{shadecolor}{rgb}{0.969, 0.969, 0.969}\color{fgcolor}\begin{figure}[bht]

{\centering \includegraphics[width=.9\textwidth]{../Main/Figures/Yearly_national_Egypt} 

}

\caption[Egypt]{Egypt: Temporal national trends along with UN (B3) estimates described in You et al. (2015) and IHME estimates based on GBD 2015 Child Mortality Collaborators (2016). RW2 represents the smoothed national estimates using the original data before benchmarking with UN estimates. RW2-adj represents the smoothed national estimates using the benchmarked data.}\label{fig:unnamed-chunk-102}
\end{figure}


\end{knitrout}
 

We fit the RW2 model to the benchmarked data in each area. 
% The proportions of the explained variation is summarized in Table~\ref{tab:paste0(countryname, "-var")}. 
We compare the results in Figure~\ref{fig:unnamed-chunk-103} to \ref{fig:unnamed-chunk-107}.
Figure~\ref{fig:unnamed-chunk-103} compares the smoothed estimates against the direct estimates. Figure~\ref{fig:unnamed-chunk-104} and Figure~\ref{fig:unnamed-chunk-105} show the posterior median estimates of U5MR in each region over time and the reductions from 1990 period respectively.
Figure~\ref{fig:unnamed-chunk-106} shows the smoothed estimates by region over time and Figure~\ref{fig:unnamed-chunk-107} compares the smoothed estimates with direct estimates from each survey for each region over time.


% %%%%%%%%%%%%%%%%%%%%%%%%%%% Table1 
% <<echo=FALSE, results='asis'>>=
% load("rda/variance_tables.rda")
% countryname2 <- gsub(" ", "", countryname)
% variance <- tables.all[[countryname]]

% table_count <- table_count + 1

% names <- c("RW2 ($\\sigma^2_{\\gamma_{t}}$)", "ICAR ($\\sigma^2_{\\phi_{i}}$)", "IID space ($\\sigma^2_{\\theta_{i}}$)", "IID time ($\\sigma^2_{\\alpha_{t}}$)", "IID space time ($\\sigma^2_{\\delta_{it}}$)")

% variance$Proportion <- round(variance$Proportion*100, digits = 2)
% row.names(variance) <- names
% tab <- xtable(variance, digits = c(1, 3, 2),align = "l|ll",
%        label = paste0("tab:", countryname, "-var"),
%        caption = paste(country, ": summary of the variance components in the RW2 model", sep = ''))
% print(tab, comment = FALSE,sanitize.text.function = function(x) {x})
% @

%%%%%%%%%%%%%%%%%%%%%%%%%%% Plot1 
\begin{knitrout}
\definecolor{shadecolor}{rgb}{0.969, 0.969, 0.969}\color{fgcolor}\begin{figure}[bht]

{\centering \includegraphics[width=.9\textwidth]{../Main/Figures/SmoothvDirectEgypt_meta} 

}

\caption[Egypt]{Egypt: Smooth versus direct Admin 1 estimates. Left: Combined (meta-analysis) survey estimate against combined direct estimates. Right: Combined (meta-analysis) survey estimate against direct estimates from each survey.}\label{fig:unnamed-chunk-103}
\end{figure}


\end{knitrout}

%%%%%%%%%%%%%%%%%%%%%%%%%%% Plot2 
\begin{knitrout}
\definecolor{shadecolor}{rgb}{0.969, 0.969, 0.969}\color{fgcolor}\begin{figure}[bht]

{\centering \includegraphics[width=.9\textwidth]{../Main/Figures/SmoothMedianEgypt} 

}

\caption[Egypt]{Egypt: Maps of posterior medians over time.}\label{fig:unnamed-chunk-104}
\end{figure}


\end{knitrout}
%%%%%%%%%%%%%%%%%%%%%%%%%%% Plot2a
\begin{knitrout}
\definecolor{shadecolor}{rgb}{0.969, 0.969, 0.969}\color{fgcolor}\begin{figure}[bht]

{\centering \includegraphics[width=.9\textwidth]{../Main/Figures/ReductionMedianEgypt} 

}

\caption[Egypt]{Egypt: Maps of reduction of posterior median U5MR in each five-year period compared to 1990 over time.}\label{fig:unnamed-chunk-105}
\end{figure}


\end{knitrout}
%%%%%%%%%%%%%%%%%%%%%%%%%%% Plot3 
\begin{knitrout}
\definecolor{shadecolor}{rgb}{0.969, 0.969, 0.969}\color{fgcolor}\begin{figure}[bht]

{\centering \includegraphics[width=.95\textwidth]{../Main/Figures/Yearly_v_Periods_Egypt} 

}

\caption[Egypt]{Egypt: Smoothed regional estimates over time. The line indicates yearly posterior median estimates and error bars indicate 95 \% posterior credible interval at each time period.}\label{fig:unnamed-chunk-106}
\end{figure}


\end{knitrout}

%%%%%%%%%%%%%%%%%%%%%%%%%%% Plot4 
\begin{knitrout}
\definecolor{shadecolor}{rgb}{0.969, 0.969, 0.969}\color{fgcolor}\begin{figure}[bht]

{\centering \includegraphics[width=.9\textwidth]{../Main/Figures/LineSubMedianEgypt} 

}

\caption[Egypt]{Egypt: Smoothed regional estimates over time compared to the direct estimates from each surveys. Direct estimates are not benchmarked with UN estimates. The line indicates posterior median and error bars indicate 95\% posterior credible interval.}\label{fig:unnamed-chunk-107}
\end{figure}


\end{knitrout}
% \subsubsection{National model results}
We further assess the RW2 model by holding out some observations, and compare the projections to the direct estimates in these holdout observations. Figure~\ref{fig:unnamed-chunk-108} compares the predicted estimates for the out-of-sample observations  with the direct estimates by holding out observations from each area in each time period.  Figure~\ref{fig:unnamed-chunk-109} compares the histogram of the bias rescaled by the total variance in the cross validation studies. Figure~\ref{fig:unnamed-chunk-110} compares the rescaled bias by region and time periods.



% %%%%%%%%%%%%%%%%%%%%%%%%%%% Plot6
% << echo=FALSE, out.width = ".9\\textwidth", fig.width = 12, fig.height = 6, fig.cap = "Out-of-sample predictions along with direct estimates in the cross validation study where all data from each time period is held out and predicted using the rest of the data.">>=
% fig_count <- fig_count + 1
% knitr::include_graphics(paste0("../Main/Figures/CV_byYear_withError_", countryname2, ".pdf")) 
% @
 
%%%%%%%%%%%%%%%%%%%%%%%%%%% Plot7
\begin{knitrout}
\definecolor{shadecolor}{rgb}{0.969, 0.969, 0.969}\color{fgcolor}\begin{figure}[bht]

{\centering \includegraphics[width=.9\textwidth]{../Main/Figures/CV_byYearRegion_withError_Egypt} 

}

\caption[Egypt]{Egypt: Out-of-sample predictions along with direct estimates in the cross validation study where data from one region in each time period is held out and predicted using the rest of the data.}\label{fig:unnamed-chunk-108}
\end{figure}


\end{knitrout}

%%%%%%%%%%%%%%%%%%%%%%%%%%% Plot8
\begin{knitrout}
\definecolor{shadecolor}{rgb}{0.969, 0.969, 0.969}\color{fgcolor}\begin{figure}[bht]

{\centering \includegraphics[width=.9\textwidth]{../Main/Figures/CVbiasEgypt} 

}

\caption[Egypt]{Egypt: Histogram and QQ-plot of the rescaled difference between the smoothed estimates and the direct estimates in the cross validation study. The differences between the two estimates are rescaled by the square root of the total variance of the two estimates.}\label{fig:unnamed-chunk-109}
\end{figure}


\end{knitrout}

%%%%%%%%%%%%%%%%%%%%%%%%%%% Plot9
\begin{knitrout}
\definecolor{shadecolor}{rgb}{0.969, 0.969, 0.969}\color{fgcolor}\begin{figure}[bht]

{\centering \includegraphics[width=.7\textwidth]{../Main/Figures/CVbiasbyRegionEgypt} 

}

\caption[Egypt]{Egypt: Line plot of the difference between smoothed estimates and the direct estimates in the cross validation study. The differences between the two estimates are rescaled by the square root of the total variance of the two estimates.}\label{fig:unnamed-chunk-110}
\end{figure}


\end{knitrout}


%%%%%%%%%%%%%%%%%%%%%%%%%%%%%%%%%%%%%%%%%%%%%%%%%%%%%%%%%%%%%%%%%%%%%%%%%%%%%%%%%%%%%%%%%%%%%%%%%%
\clearpage
\subsubsection{Ethiopia}


% \subsubsection{Summary of DHS surveys}

%%%%%%%%%%%%%%%%%%%%%%%%%%% Summary 


DHS surveys were conducted in Ethiopia in 2000, 2005, 2011, and 2016.
% years.out[1:(length(years.out)-1)], and years.out[length(years.out)]. 

We fit both the RW2 only model to the combined national data, and compare the time trend at national level with the estimates produced by the UN and IHME in Figure~\ref{fig:unnamed-chunk-112}. We then adjusted the combined national data to the UN estimates of U5MR, and refit the models on the benchmarked data. 

%%%%%%%%%%%%%%%%%%%%%%%%%% Plot5 
\begin{knitrout}
\definecolor{shadecolor}{rgb}{0.969, 0.969, 0.969}\color{fgcolor}\begin{figure}[bht]

{\centering \includegraphics[width=.9\textwidth]{../Main/Figures/Yearly_national_Ethiopia} 

}

\caption[Ethiopia]{Ethiopia: Temporal national trends along with UN (B3) estimates described in You et al. (2015) and IHME estimates based on GBD 2015 Child Mortality Collaborators (2016). RW2 represents the smoothed national estimates using the original data before benchmarking with UN estimates. RW2-adj represents the smoothed national estimates using the benchmarked data.}\label{fig:unnamed-chunk-112}
\end{figure}


\end{knitrout}
 

We fit the RW2 model to the benchmarked data in each area. 
% The proportions of the explained variation is summarized in Table~\ref{tab:paste0(countryname, "-var")}. 
We compare the results in Figure~\ref{fig:unnamed-chunk-113} to \ref{fig:unnamed-chunk-117}.
Figure~\ref{fig:unnamed-chunk-113} compares the smoothed estimates against the direct estimates. Figure~\ref{fig:unnamed-chunk-114} and Figure~\ref{fig:unnamed-chunk-115} show the posterior median estimates of U5MR in each region over time and the reductions from 1990 period respectively.
Figure~\ref{fig:unnamed-chunk-116} shows the smoothed estimates by region over time and Figure~\ref{fig:unnamed-chunk-117} compares the smoothed estimates with direct estimates from each survey for each region over time.


% %%%%%%%%%%%%%%%%%%%%%%%%%%% Table1 
% <<echo=FALSE, results='asis'>>=
% load("rda/variance_tables.rda")
% countryname2 <- gsub(" ", "", countryname)
% variance <- tables.all[[countryname]]

% table_count <- table_count + 1

% names <- c("RW2 ($\\sigma^2_{\\gamma_{t}}$)", "ICAR ($\\sigma^2_{\\phi_{i}}$)", "IID space ($\\sigma^2_{\\theta_{i}}$)", "IID time ($\\sigma^2_{\\alpha_{t}}$)", "IID space time ($\\sigma^2_{\\delta_{it}}$)")

% variance$Proportion <- round(variance$Proportion*100, digits = 2)
% row.names(variance) <- names
% tab <- xtable(variance, digits = c(1, 3, 2),align = "l|ll",
%        label = paste0("tab:", countryname, "-var"),
%        caption = paste(country, ": summary of the variance components in the RW2 model", sep = ''))
% print(tab, comment = FALSE,sanitize.text.function = function(x) {x})
% @

%%%%%%%%%%%%%%%%%%%%%%%%%%% Plot1 
\begin{knitrout}
\definecolor{shadecolor}{rgb}{0.969, 0.969, 0.969}\color{fgcolor}\begin{figure}[bht]

{\centering \includegraphics[width=.9\textwidth]{../Main/Figures/SmoothvDirectEthiopia_meta} 

}

\caption[Ethiopia]{Ethiopia: Smooth versus direct Admin 1 estimates. Left: Combined (meta-analysis) survey estimate against combined direct estimates. Right: Combined (meta-analysis) survey estimate against direct estimates from each survey.}\label{fig:unnamed-chunk-113}
\end{figure}


\end{knitrout}

%%%%%%%%%%%%%%%%%%%%%%%%%%% Plot2 
\begin{knitrout}
\definecolor{shadecolor}{rgb}{0.969, 0.969, 0.969}\color{fgcolor}\begin{figure}[bht]

{\centering \includegraphics[width=.9\textwidth]{../Main/Figures/SmoothMedianEthiopia} 

}

\caption[Ethiopia]{Ethiopia: Maps of posterior medians over time.}\label{fig:unnamed-chunk-114}
\end{figure}


\end{knitrout}
%%%%%%%%%%%%%%%%%%%%%%%%%%% Plot2a
\begin{knitrout}
\definecolor{shadecolor}{rgb}{0.969, 0.969, 0.969}\color{fgcolor}\begin{figure}[bht]

{\centering \includegraphics[width=.9\textwidth]{../Main/Figures/ReductionMedianEthiopia} 

}

\caption[Ethiopia]{Ethiopia: Maps of reduction of posterior median U5MR in each five-year period compared to 1990 over time.}\label{fig:unnamed-chunk-115}
\end{figure}


\end{knitrout}
%%%%%%%%%%%%%%%%%%%%%%%%%%% Plot3 
\begin{knitrout}
\definecolor{shadecolor}{rgb}{0.969, 0.969, 0.969}\color{fgcolor}\begin{figure}[bht]

{\centering \includegraphics[width=.95\textwidth]{../Main/Figures/Yearly_v_Periods_Ethiopia} 

}

\caption[Ethiopia]{Ethiopia: Smoothed regional estimates over time. The line indicates yearly posterior median estimates and error bars indicate 95 \% posterior credible interval at each time period.}\label{fig:unnamed-chunk-116}
\end{figure}


\end{knitrout}

%%%%%%%%%%%%%%%%%%%%%%%%%%% Plot4 
\begin{knitrout}
\definecolor{shadecolor}{rgb}{0.969, 0.969, 0.969}\color{fgcolor}\begin{figure}[bht]

{\centering \includegraphics[width=.9\textwidth]{../Main/Figures/LineSubMedianEthiopia} 

}

\caption[Ethiopia]{Ethiopia: Smoothed regional estimates over time compared to the direct estimates from each surveys. Direct estimates are not benchmarked with UN estimates. The line indicates posterior median and error bars indicate 95\% posterior credible interval.}\label{fig:unnamed-chunk-117}
\end{figure}


\end{knitrout}
% \subsubsection{National model results}
We further assess the RW2 model by holding out some observations, and compare the projections to the direct estimates in these holdout observations. Figure~\ref{fig:unnamed-chunk-118} compares the predicted estimates for the out-of-sample observations  with the direct estimates by holding out observations from each area in each time period.  Figure~\ref{fig:unnamed-chunk-119} compares the histogram of the bias rescaled by the total variance in the cross validation studies. Figure~\ref{fig:unnamed-chunk-120} compares the rescaled bias by region and time periods.



% %%%%%%%%%%%%%%%%%%%%%%%%%%% Plot6
% << echo=FALSE, out.width = ".9\\textwidth", fig.width = 12, fig.height = 6, fig.cap = "Out-of-sample predictions along with direct estimates in the cross validation study where all data from each time period is held out and predicted using the rest of the data.">>=
% fig_count <- fig_count + 1
% knitr::include_graphics(paste0("../Main/Figures/CV_byYear_withError_", countryname2, ".pdf")) 
% @
 
%%%%%%%%%%%%%%%%%%%%%%%%%%% Plot7
\begin{knitrout}
\definecolor{shadecolor}{rgb}{0.969, 0.969, 0.969}\color{fgcolor}\begin{figure}[bht]

{\centering \includegraphics[width=.9\textwidth]{../Main/Figures/CV_byYearRegion_withError_Ethiopia} 

}

\caption[Ethiopia]{Ethiopia: Out-of-sample predictions along with direct estimates in the cross validation study where data from one region in each time period is held out and predicted using the rest of the data.}\label{fig:unnamed-chunk-118}
\end{figure}


\end{knitrout}

%%%%%%%%%%%%%%%%%%%%%%%%%%% Plot8
\begin{knitrout}
\definecolor{shadecolor}{rgb}{0.969, 0.969, 0.969}\color{fgcolor}\begin{figure}[bht]

{\centering \includegraphics[width=.9\textwidth]{../Main/Figures/CVbiasEthiopia} 

}

\caption[Ethiopia]{Ethiopia: Histogram and QQ-plot of the rescaled difference between the smoothed estimates and the direct estimates in the cross validation study. The differences between the two estimates are rescaled by the square root of the total variance of the two estimates.}\label{fig:unnamed-chunk-119}
\end{figure}


\end{knitrout}

%%%%%%%%%%%%%%%%%%%%%%%%%%% Plot9
\begin{knitrout}
\definecolor{shadecolor}{rgb}{0.969, 0.969, 0.969}\color{fgcolor}\begin{figure}[bht]

{\centering \includegraphics[width=.7\textwidth]{../Main/Figures/CVbiasbyRegionEthiopia} 

}

\caption[Ethiopia]{Ethiopia: Line plot of the difference between smoothed estimates and the direct estimates in the cross validation study. The differences between the two estimates are rescaled by the square root of the total variance of the two estimates.}\label{fig:unnamed-chunk-120}
\end{figure}


\end{knitrout}


%%%%%%%%%%%%%%%%%%%%%%%%%%%%%%%%%%%%%%%%%%%%%%%%%%%%%%%%%%%%%%%%%%%%%%%%%%%%%%%%%%%%%%%%%%%%%%%%%%
\clearpage
\subsubsection{Gabon}


% \subsubsection{Summary of DHS surveys}

%%%%%%%%%%%%%%%%%%%%%%%%%%% Summary 


DHS surveys were conducted in Gabon in 2000, and 2012.
% years.out[1:(length(years.out)-1)], and years.out[length(years.out)]. 

We fit both the RW2 only model to the combined national data, and compare the time trend at national level with the estimates produced by the UN and IHME in Figure~\ref{fig:unnamed-chunk-122}. We then adjusted the combined national data to the UN estimates of U5MR, and refit the models on the benchmarked data. 

%%%%%%%%%%%%%%%%%%%%%%%%%% Plot5 
\begin{knitrout}
\definecolor{shadecolor}{rgb}{0.969, 0.969, 0.969}\color{fgcolor}\begin{figure}[bht]

{\centering \includegraphics[width=.9\textwidth]{../Main/Figures/Yearly_national_Gabon} 

}

\caption[Gabon]{Gabon: Temporal national trends along with UN (B3) estimates described in You et al. (2015) and IHME estimates based on GBD 2015 Child Mortality Collaborators (2016). RW2 represents the smoothed national estimates using the original data before benchmarking with UN estimates. RW2-adj represents the smoothed national estimates using the benchmarked data.}\label{fig:unnamed-chunk-122}
\end{figure}


\end{knitrout}
 

We fit the RW2 model to the benchmarked data in each area. 
% The proportions of the explained variation is summarized in Table~\ref{tab:paste0(countryname, "-var")}. 
We compare the results in Figure~\ref{fig:unnamed-chunk-123} to \ref{fig:unnamed-chunk-127}.
Figure~\ref{fig:unnamed-chunk-123} compares the smoothed estimates against the direct estimates. Figure~\ref{fig:unnamed-chunk-124} and Figure~\ref{fig:unnamed-chunk-125} show the posterior median estimates of U5MR in each region over time and the reductions from 1990 period respectively.
Figure~\ref{fig:unnamed-chunk-126} shows the smoothed estimates by region over time and Figure~\ref{fig:unnamed-chunk-127} compares the smoothed estimates with direct estimates from each survey for each region over time.


% %%%%%%%%%%%%%%%%%%%%%%%%%%% Table1 
% <<echo=FALSE, results='asis'>>=
% load("rda/variance_tables.rda")
% countryname2 <- gsub(" ", "", countryname)
% variance <- tables.all[[countryname]]

% table_count <- table_count + 1

% names <- c("RW2 ($\\sigma^2_{\\gamma_{t}}$)", "ICAR ($\\sigma^2_{\\phi_{i}}$)", "IID space ($\\sigma^2_{\\theta_{i}}$)", "IID time ($\\sigma^2_{\\alpha_{t}}$)", "IID space time ($\\sigma^2_{\\delta_{it}}$)")

% variance$Proportion <- round(variance$Proportion*100, digits = 2)
% row.names(variance) <- names
% tab <- xtable(variance, digits = c(1, 3, 2),align = "l|ll",
%        label = paste0("tab:", countryname, "-var"),
%        caption = paste(country, ": summary of the variance components in the RW2 model", sep = ''))
% print(tab, comment = FALSE,sanitize.text.function = function(x) {x})
% @

%%%%%%%%%%%%%%%%%%%%%%%%%%% Plot1 
\begin{knitrout}
\definecolor{shadecolor}{rgb}{0.969, 0.969, 0.969}\color{fgcolor}\begin{figure}[bht]

{\centering \includegraphics[width=.9\textwidth]{../Main/Figures/SmoothvDirectGabon_meta} 

}

\caption[Gabon]{Gabon: Smooth versus direct Admin 1 estimates. Left: Combined (meta-analysis) survey estimate against combined direct estimates. Right: Combined (meta-analysis) survey estimate against direct estimates from each survey.}\label{fig:unnamed-chunk-123}
\end{figure}


\end{knitrout}

%%%%%%%%%%%%%%%%%%%%%%%%%%% Plot2 
\begin{knitrout}
\definecolor{shadecolor}{rgb}{0.969, 0.969, 0.969}\color{fgcolor}\begin{figure}[bht]

{\centering \includegraphics[width=.9\textwidth]{../Main/Figures/SmoothMedianGabon} 

}

\caption[Gabon]{Gabon: Maps of posterior medians over time.}\label{fig:unnamed-chunk-124}
\end{figure}


\end{knitrout}
%%%%%%%%%%%%%%%%%%%%%%%%%%% Plot2a
\begin{knitrout}
\definecolor{shadecolor}{rgb}{0.969, 0.969, 0.969}\color{fgcolor}\begin{figure}[bht]

{\centering \includegraphics[width=.9\textwidth]{../Main/Figures/ReductionMedianGabon} 

}

\caption[Gabon]{Gabon: Maps of reduction of posterior median U5MR in each five-year period compared to 1990 over time.}\label{fig:unnamed-chunk-125}
\end{figure}


\end{knitrout}
%%%%%%%%%%%%%%%%%%%%%%%%%%% Plot3 
\begin{knitrout}
\definecolor{shadecolor}{rgb}{0.969, 0.969, 0.969}\color{fgcolor}\begin{figure}[bht]

{\centering \includegraphics[width=.95\textwidth]{../Main/Figures/Yearly_v_Periods_Gabon} 

}

\caption[Gabon]{Gabon: Smoothed regional estimates over time. The line indicates yearly posterior median estimates and error bars indicate 95 \% posterior credible interval at each time period.}\label{fig:unnamed-chunk-126}
\end{figure}


\end{knitrout}

%%%%%%%%%%%%%%%%%%%%%%%%%%% Plot4 
\begin{knitrout}
\definecolor{shadecolor}{rgb}{0.969, 0.969, 0.969}\color{fgcolor}\begin{figure}[bht]

{\centering \includegraphics[width=.9\textwidth]{../Main/Figures/LineSubMedianGabon} 

}

\caption[Gabon]{Gabon: Smoothed regional estimates over time compared to the direct estimates from each surveys. Direct estimates are not benchmarked with UN estimates. The line indicates posterior median and error bars indicate 95\% posterior credible interval.}\label{fig:unnamed-chunk-127}
\end{figure}


\end{knitrout}
% \subsubsection{National model results}
We further assess the RW2 model by holding out some observations, and compare the projections to the direct estimates in these holdout observations. Figure~\ref{fig:unnamed-chunk-128} compares the predicted estimates for the out-of-sample observations  with the direct estimates by holding out observations from each area in each time period.  Figure~\ref{fig:unnamed-chunk-129} compares the histogram of the bias rescaled by the total variance in the cross validation studies. Figure~\ref{fig:unnamed-chunk-130} compares the rescaled bias by region and time periods.



% %%%%%%%%%%%%%%%%%%%%%%%%%%% Plot6
% << echo=FALSE, out.width = ".9\\textwidth", fig.width = 12, fig.height = 6, fig.cap = "Out-of-sample predictions along with direct estimates in the cross validation study where all data from each time period is held out and predicted using the rest of the data.">>=
% fig_count <- fig_count + 1
% knitr::include_graphics(paste0("../Main/Figures/CV_byYear_withError_", countryname2, ".pdf")) 
% @
 
%%%%%%%%%%%%%%%%%%%%%%%%%%% Plot7
\begin{knitrout}
\definecolor{shadecolor}{rgb}{0.969, 0.969, 0.969}\color{fgcolor}\begin{figure}[bht]

{\centering \includegraphics[width=.9\textwidth]{../Main/Figures/CV_byYearRegion_withError_Gabon} 

}

\caption[Gabon]{Gabon: Out-of-sample predictions along with direct estimates in the cross validation study where data from one region in each time period is held out and predicted using the rest of the data.}\label{fig:unnamed-chunk-128}
\end{figure}


\end{knitrout}

%%%%%%%%%%%%%%%%%%%%%%%%%%% Plot8
\begin{knitrout}
\definecolor{shadecolor}{rgb}{0.969, 0.969, 0.969}\color{fgcolor}\begin{figure}[bht]

{\centering \includegraphics[width=.9\textwidth]{../Main/Figures/CVbiasGabon} 

}

\caption[Gabon]{Gabon: Histogram and QQ-plot of the rescaled difference between the smoothed estimates and the direct estimates in the cross validation study. The differences between the two estimates are rescaled by the square root of the total variance of the two estimates.}\label{fig:unnamed-chunk-129}
\end{figure}


\end{knitrout}

%%%%%%%%%%%%%%%%%%%%%%%%%%% Plot9
\begin{knitrout}
\definecolor{shadecolor}{rgb}{0.969, 0.969, 0.969}\color{fgcolor}\begin{figure}[bht]

{\centering \includegraphics[width=.7\textwidth]{../Main/Figures/CVbiasbyRegionGabon} 

}

\caption[Gabon]{Gabon: Line plot of the difference between smoothed estimates and the direct estimates in the cross validation study. The differences between the two estimates are rescaled by the square root of the total variance of the two estimates.}\label{fig:unnamed-chunk-130}
\end{figure}


\end{knitrout}

%%%%%%%%%%%%%%%%%%%%%%%%%%%%%%%%%%%%%%%%%%%%%%%%%%%%%%%%%%%%%%%%%%%%%%%%%%%%%%%%%%%%%%%%%%%%%%%%%%
\clearpage
\subsubsection{Gambia}


% \subsubsection{Summary of DHS surveys}

%%%%%%%%%%%%%%%%%%%%%%%%%%% Summary 


DHS surveys were conducted in Gambia in 2013.
% years.out[1:(length(years.out)-1)], and years.out[length(years.out)]. 

We fit both the RW2 only model to the combined national data, and compare the time trend at national level with the estimates produced by the UN and IHME in Figure~\ref{fig:unnamed-chunk-132}. We then adjusted the combined national data to the UN estimates of U5MR, and refit the models on the benchmarked data. 

%%%%%%%%%%%%%%%%%%%%%%%%%% Plot5 
\begin{knitrout}
\definecolor{shadecolor}{rgb}{0.969, 0.969, 0.969}\color{fgcolor}\begin{figure}[bht]

{\centering \includegraphics[width=.9\textwidth]{../Main/Figures/Yearly_national_Gambia} 

}

\caption[Gambia]{Gambia: Temporal national trends along with UN (B3) estimates described in You et al. (2015) and IHME estimates based on GBD 2015 Child Mortality Collaborators (2016). RW2 represents the smoothed national estimates using the original data before benchmarking with UN estimates. RW2-adj represents the smoothed national estimates using the benchmarked data.}\label{fig:unnamed-chunk-132}
\end{figure}


\end{knitrout}
 

We fit the RW2 model to the benchmarked data in each area. 
% The proportions of the explained variation is summarized in Table~\ref{tab:paste0(countryname, "-var")}. 
We compare the results in Figure~\ref{fig:unnamed-chunk-133} to \ref{fig:unnamed-chunk-137}.
Figure~\ref{fig:unnamed-chunk-133} compares the smoothed estimates against the direct estimates. Figure~\ref{fig:unnamed-chunk-134} and Figure~\ref{fig:unnamed-chunk-135} show the posterior median estimates of U5MR in each region over time and the reductions from 1990 period respectively.
Figure~\ref{fig:unnamed-chunk-136} shows the smoothed estimates by region over time and Figure~\ref{fig:unnamed-chunk-137} compares the smoothed estimates with direct estimates from each survey for each region over time.


% %%%%%%%%%%%%%%%%%%%%%%%%%%% Table1 
% <<echo=FALSE, results='asis'>>=
% load("rda/variance_tables.rda")
% countryname2 <- gsub(" ", "", countryname)
% variance <- tables.all[[countryname]]

% table_count <- table_count + 1

% names <- c("RW2 ($\\sigma^2_{\\gamma_{t}}$)", "ICAR ($\\sigma^2_{\\phi_{i}}$)", "IID space ($\\sigma^2_{\\theta_{i}}$)", "IID time ($\\sigma^2_{\\alpha_{t}}$)", "IID space time ($\\sigma^2_{\\delta_{it}}$)")

% variance$Proportion <- round(variance$Proportion*100, digits = 2)
% row.names(variance) <- names
% tab <- xtable(variance, digits = c(1, 3, 2),align = "l|ll",
%        label = paste0("tab:", countryname, "-var"),
%        caption = paste(country, ": summary of the variance components in the RW2 model", sep = ''))
% print(tab, comment = FALSE,sanitize.text.function = function(x) {x})
% @

%%%%%%%%%%%%%%%%%%%%%%%%%%% Plot1 
\begin{knitrout}
\definecolor{shadecolor}{rgb}{0.969, 0.969, 0.969}\color{fgcolor}\begin{figure}[bht]

{\centering \includegraphics[width=.9\textwidth]{../Main/Figures/SmoothvDirectGambia_meta} 

}

\caption[Gambia]{Gambia: Smooth versus direct Admin 1 estimates. Left: Combined (meta-analysis) survey estimate against combined direct estimates. Right: Combined (meta-analysis) survey estimate against direct estimates from each survey.}\label{fig:unnamed-chunk-133}
\end{figure}


\end{knitrout}

%%%%%%%%%%%%%%%%%%%%%%%%%%% Plot2 
\begin{knitrout}
\definecolor{shadecolor}{rgb}{0.969, 0.969, 0.969}\color{fgcolor}\begin{figure}[bht]

{\centering \includegraphics[width=.9\textwidth]{../Main/Figures/SmoothMedianGambia} 

}

\caption[Gambia]{Gambia: Maps of posterior medians over time.}\label{fig:unnamed-chunk-134}
\end{figure}


\end{knitrout}
%%%%%%%%%%%%%%%%%%%%%%%%%%% Plot2a
\begin{knitrout}
\definecolor{shadecolor}{rgb}{0.969, 0.969, 0.969}\color{fgcolor}\begin{figure}[bht]

{\centering \includegraphics[width=.9\textwidth]{../Main/Figures/ReductionMedianGambia} 

}

\caption[Gambia]{Gambia: Maps of reduction of posterior median U5MR in each five-year period compared to 1990 over time.}\label{fig:unnamed-chunk-135}
\end{figure}


\end{knitrout}
%%%%%%%%%%%%%%%%%%%%%%%%%%% Plot3 
\begin{knitrout}
\definecolor{shadecolor}{rgb}{0.969, 0.969, 0.969}\color{fgcolor}\begin{figure}[bht]

{\centering \includegraphics[width=.95\textwidth]{../Main/Figures/Yearly_v_Periods_Gambia} 

}

\caption[Gambia]{Gambia: Smoothed regional estimates over time. The line indicates yearly posterior median estimates and error bars indicate 95 \% posterior credible interval at each time period.}\label{fig:unnamed-chunk-136}
\end{figure}


\end{knitrout}

%%%%%%%%%%%%%%%%%%%%%%%%%%% Plot4 
\begin{knitrout}
\definecolor{shadecolor}{rgb}{0.969, 0.969, 0.969}\color{fgcolor}\begin{figure}[bht]

{\centering \includegraphics[width=.9\textwidth]{../Main/Figures/LineSubMedianGambia} 

}

\caption[Gambia]{Gambia: Smoothed regional estimates over time compared to the direct estimates from each surveys. Direct estimates are not benchmarked with UN estimates. The line indicates posterior median and error bars indicate 95\% posterior credible interval.}\label{fig:unnamed-chunk-137}
\end{figure}


\end{knitrout}
% \subsubsection{National model results}
We further assess the RW2 model by holding out some observations, and compare the projections to the direct estimates in these holdout observations. Figure~\ref{fig:unnamed-chunk-138} compares the predicted estimates for the out-of-sample observations  with the direct estimates by holding out observations from each area in each time period.  Figure~\ref{fig:unnamed-chunk-139} compares the histogram of the bias rescaled by the total variance in the cross validation studies. Figure~\ref{fig:unnamed-chunk-140} compares the rescaled bias by region and time periods.



% %%%%%%%%%%%%%%%%%%%%%%%%%%% Plot6
% << echo=FALSE, out.width = ".9\\textwidth", fig.width = 12, fig.height = 6, fig.cap = "Out-of-sample predictions along with direct estimates in the cross validation study where all data from each time period is held out and predicted using the rest of the data.">>=
% fig_count <- fig_count + 1
% knitr::include_graphics(paste0("../Main/Figures/CV_byYear_withError_", countryname2, ".pdf")) 
% @
 
%%%%%%%%%%%%%%%%%%%%%%%%%%% Plot7
\begin{knitrout}
\definecolor{shadecolor}{rgb}{0.969, 0.969, 0.969}\color{fgcolor}\begin{figure}[bht]

{\centering \includegraphics[width=.9\textwidth]{../Main/Figures/CV_byYearRegion_withError_Gambia} 

}

\caption[Gambia]{Gambia: Out-of-sample predictions along with direct estimates in the cross validation study where data from one region in each time period is held out and predicted using the rest of the data.}\label{fig:unnamed-chunk-138}
\end{figure}


\end{knitrout}

%%%%%%%%%%%%%%%%%%%%%%%%%%% Plot8
\begin{knitrout}
\definecolor{shadecolor}{rgb}{0.969, 0.969, 0.969}\color{fgcolor}\begin{figure}[bht]

{\centering \includegraphics[width=.9\textwidth]{../Main/Figures/CVbiasGambia} 

}

\caption[Gambia]{Gambia: Histogram and QQ-plot of the rescaled difference between the smoothed estimates and the direct estimates in the cross validation study. The differences between the two estimates are rescaled by the square root of the total variance of the two estimates.}\label{fig:unnamed-chunk-139}
\end{figure}


\end{knitrout}

%%%%%%%%%%%%%%%%%%%%%%%%%%% Plot9
\begin{knitrout}
\definecolor{shadecolor}{rgb}{0.969, 0.969, 0.969}\color{fgcolor}\begin{figure}[bht]

{\centering \includegraphics[width=.7\textwidth]{../Main/Figures/CVbiasbyRegionGambia} 

}

\caption[Gambia]{Gambia: Line plot of the difference between smoothed estimates and the direct estimates in the cross validation study. The differences between the two estimates are rescaled by the square root of the total variance of the two estimates.}\label{fig:unnamed-chunk-140}
\end{figure}


\end{knitrout}


%%%%%%%%%%%%%%%%%%%%%%%%%%%%%%%%%%%%%%%%%%%%%%%%%%%%%%%%%%%%%%%%%%%%%%%%%%%%%%%%%%%%%%%%%%%%%%%%%%
\clearpage
\subsubsection{Ghana}


% \subsubsection{Summary of DHS surveys}

%%%%%%%%%%%%%%%%%%%%%%%%%%% Summary 


DHS surveys were conducted in Ghana in 1989, 1993, 1998, 2003, 2008, and 2014.
% years.out[1:(length(years.out)-1)], and years.out[length(years.out)]. 

We fit both the RW2 only model to the combined national data, and compare the time trend at national level with the estimates produced by the UN and IHME in Figure~\ref{fig:unnamed-chunk-142}. We then adjusted the combined national data to the UN estimates of U5MR, and refit the models on the benchmarked data. 

%%%%%%%%%%%%%%%%%%%%%%%%%% Plot5 
\begin{knitrout}
\definecolor{shadecolor}{rgb}{0.969, 0.969, 0.969}\color{fgcolor}\begin{figure}[bht]

{\centering \includegraphics[width=.9\textwidth]{../Main/Figures/Yearly_national_Ghana} 

}

\caption[Ghana]{Ghana: Temporal national trends along with UN (B3) estimates described in You et al. (2015) and IHME estimates based on GBD 2015 Child Mortality Collaborators (2016). RW2 represents the smoothed national estimates using the original data before benchmarking with UN estimates. RW2-adj represents the smoothed national estimates using the benchmarked data.}\label{fig:unnamed-chunk-142}
\end{figure}


\end{knitrout}
 

We fit the RW2 model to the benchmarked data in each area. 
% The proportions of the explained variation is summarized in Table~\ref{tab:paste0(countryname, "-var")}. 
We compare the results in Figure~\ref{fig:unnamed-chunk-143} to \ref{fig:unnamed-chunk-147}.
Figure~\ref{fig:unnamed-chunk-143} compares the smoothed estimates against the direct estimates. Figure~\ref{fig:unnamed-chunk-144} and Figure~\ref{fig:unnamed-chunk-145} show the posterior median estimates of U5MR in each region over time and the reductions from 1990 period respectively.
Figure~\ref{fig:unnamed-chunk-146} shows the smoothed estimates by region over time and Figure~\ref{fig:unnamed-chunk-147} compares the smoothed estimates with direct estimates from each survey for each region over time.


% %%%%%%%%%%%%%%%%%%%%%%%%%%% Table1 
% <<echo=FALSE, results='asis'>>=
% load("rda/variance_tables.rda")
% countryname2 <- gsub(" ", "", countryname)
% variance <- tables.all[[countryname]]

% table_count <- table_count + 1

% names <- c("RW2 ($\\sigma^2_{\\gamma_{t}}$)", "ICAR ($\\sigma^2_{\\phi_{i}}$)", "IID space ($\\sigma^2_{\\theta_{i}}$)", "IID time ($\\sigma^2_{\\alpha_{t}}$)", "IID space time ($\\sigma^2_{\\delta_{it}}$)")

% variance$Proportion <- round(variance$Proportion*100, digits = 2)
% row.names(variance) <- names
% tab <- xtable(variance, digits = c(1, 3, 2),align = "l|ll",
%        label = paste0("tab:", countryname, "-var"),
%        caption = paste(country, ": summary of the variance components in the RW2 model", sep = ''))
% print(tab, comment = FALSE,sanitize.text.function = function(x) {x})
% @

%%%%%%%%%%%%%%%%%%%%%%%%%%% Plot1 
\begin{knitrout}
\definecolor{shadecolor}{rgb}{0.969, 0.969, 0.969}\color{fgcolor}\begin{figure}[bht]

{\centering \includegraphics[width=.9\textwidth]{../Main/Figures/SmoothvDirectGhana_meta} 

}

\caption[Ghana]{Ghana: Smooth versus direct Admin 1 estimates. Left: Combined (meta-analysis) survey estimate against combined direct estimates. Right: Combined (meta-analysis) survey estimate against direct estimates from each survey.}\label{fig:unnamed-chunk-143}
\end{figure}


\end{knitrout}

%%%%%%%%%%%%%%%%%%%%%%%%%%% Plot2 
\begin{knitrout}
\definecolor{shadecolor}{rgb}{0.969, 0.969, 0.969}\color{fgcolor}\begin{figure}[bht]

{\centering \includegraphics[width=.9\textwidth]{../Main/Figures/SmoothMedianGhana} 

}

\caption[Ghana]{Ghana: Maps of posterior medians over time.}\label{fig:unnamed-chunk-144}
\end{figure}


\end{knitrout}
%%%%%%%%%%%%%%%%%%%%%%%%%%% Plot2a
\begin{knitrout}
\definecolor{shadecolor}{rgb}{0.969, 0.969, 0.969}\color{fgcolor}\begin{figure}[bht]

{\centering \includegraphics[width=.9\textwidth]{../Main/Figures/ReductionMedianGhana} 

}

\caption[Ghana]{Ghana: Maps of reduction of posterior median U5MR in each five-year period compared to 1990 over time.}\label{fig:unnamed-chunk-145}
\end{figure}


\end{knitrout}
%%%%%%%%%%%%%%%%%%%%%%%%%%% Plot3 
\begin{knitrout}
\definecolor{shadecolor}{rgb}{0.969, 0.969, 0.969}\color{fgcolor}\begin{figure}[bht]

{\centering \includegraphics[width=.95\textwidth]{../Main/Figures/Yearly_v_Periods_Ghana} 

}

\caption[Ghana]{Ghana: Smoothed regional estimates over time. The line indicates yearly posterior median estimates and error bars indicate 95 \% posterior credible interval at each time period.}\label{fig:unnamed-chunk-146}
\end{figure}


\end{knitrout}

%%%%%%%%%%%%%%%%%%%%%%%%%%% Plot4 
\begin{knitrout}
\definecolor{shadecolor}{rgb}{0.969, 0.969, 0.969}\color{fgcolor}\begin{figure}[bht]

{\centering \includegraphics[width=.9\textwidth]{../Main/Figures/LineSubMedianGhana} 

}

\caption[Ghana]{Ghana: Smoothed regional estimates over time compared to the direct estimates from each surveys. Direct estimates are not benchmarked with UN estimates. The line indicates posterior median and error bars indicate 95\% posterior credible interval.}\label{fig:unnamed-chunk-147}
\end{figure}


\end{knitrout}
% \subsubsection{National model results}
We further assess the RW2 model by holding out some observations, and compare the projections to the direct estimates in these holdout observations. Figure~\ref{fig:unnamed-chunk-148} compares the predicted estimates for the out-of-sample observations  with the direct estimates by holding out observations from each area in each time period.  Figure~\ref{fig:unnamed-chunk-149} compares the histogram of the bias rescaled by the total variance in the cross validation studies. Figure~\ref{fig:unnamed-chunk-150} compares the rescaled bias by region and time periods.



% %%%%%%%%%%%%%%%%%%%%%%%%%%% Plot6
% << echo=FALSE, out.width = ".9\\textwidth", fig.width = 12, fig.height = 6, fig.cap = "Out-of-sample predictions along with direct estimates in the cross validation study where all data from each time period is held out and predicted using the rest of the data.">>=
% fig_count <- fig_count + 1
% knitr::include_graphics(paste0("../Main/Figures/CV_byYear_withError_", countryname2, ".pdf")) 
% @
 
%%%%%%%%%%%%%%%%%%%%%%%%%%% Plot7
\begin{knitrout}
\definecolor{shadecolor}{rgb}{0.969, 0.969, 0.969}\color{fgcolor}\begin{figure}[bht]

{\centering \includegraphics[width=.9\textwidth]{../Main/Figures/CV_byYearRegion_withError_Ghana} 

}

\caption[Ghana]{Ghana: Out-of-sample predictions along with direct estimates in the cross validation study where data from one region in each time period is held out and predicted using the rest of the data.}\label{fig:unnamed-chunk-148}
\end{figure}


\end{knitrout}

%%%%%%%%%%%%%%%%%%%%%%%%%%% Plot8
\begin{knitrout}
\definecolor{shadecolor}{rgb}{0.969, 0.969, 0.969}\color{fgcolor}\begin{figure}[bht]

{\centering \includegraphics[width=.9\textwidth]{../Main/Figures/CVbiasGhana} 

}

\caption[Ghana]{Ghana: Histogram and QQ-plot of the rescaled difference between the smoothed estimates and the direct estimates in the cross validation study. The differences between the two estimates are rescaled by the square root of the total variance of the two estimates.}\label{fig:unnamed-chunk-149}
\end{figure}


\end{knitrout}

%%%%%%%%%%%%%%%%%%%%%%%%%%% Plot9
\begin{knitrout}
\definecolor{shadecolor}{rgb}{0.969, 0.969, 0.969}\color{fgcolor}\begin{figure}[bht]

{\centering \includegraphics[width=.7\textwidth]{../Main/Figures/CVbiasbyRegionGhana} 

}

\caption[Ghana]{Ghana: Line plot of the difference between smoothed estimates and the direct estimates in the cross validation study. The differences between the two estimates are rescaled by the square root of the total variance of the two estimates.}\label{fig:unnamed-chunk-150}
\end{figure}


\end{knitrout}


%%%%%%%%%%%%%%%%%%%%%%%%%%%%%%%%%%%%%%%%%%%%%%%%%%%%%%%%%%%%%%%%%%%%%%%%%%%%%%%%%%%%%%%%%%%%%%%%%%
\clearpage
\subsubsection{Guinea}


% \subsubsection{Summary of DHS surveys}

%%%%%%%%%%%%%%%%%%%%%%%%%%% Summary 


DHS surveys were conducted in Guinea in 1999, 2005, and 2012.
% years.out[1:(length(years.out)-1)], and years.out[length(years.out)]. 

We fit both the RW2 only model to the combined national data, and compare the time trend at national level with the estimates produced by the UN and IHME in Figure~\ref{fig:unnamed-chunk-152}. We then adjusted the combined national data to the UN estimates of U5MR, and refit the models on the benchmarked data. 

%%%%%%%%%%%%%%%%%%%%%%%%%% Plot5 
\begin{knitrout}
\definecolor{shadecolor}{rgb}{0.969, 0.969, 0.969}\color{fgcolor}\begin{figure}[bht]

{\centering \includegraphics[width=.9\textwidth]{../Main/Figures/Yearly_national_Guinea} 

}

\caption[Guinea]{Guinea: Temporal national trends along with UN (B3) estimates described in You et al. (2015) and IHME estimates based on GBD 2015 Child Mortality Collaborators (2016). RW2 represents the smoothed national estimates using the original data before benchmarking with UN estimates. RW2-adj represents the smoothed national estimates using the benchmarked data.}\label{fig:unnamed-chunk-152}
\end{figure}


\end{knitrout}
 

We fit the RW2 model to the benchmarked data in each area. 
% The proportions of the explained variation is summarized in Table~\ref{tab:paste0(countryname, "-var")}. 
We compare the results in Figure~\ref{fig:unnamed-chunk-153} to \ref{fig:unnamed-chunk-157}.
Figure~\ref{fig:unnamed-chunk-153} compares the smoothed estimates against the direct estimates. Figure~\ref{fig:unnamed-chunk-154} and Figure~\ref{fig:unnamed-chunk-155} show the posterior median estimates of U5MR in each region over time and the reductions from 1990 period respectively.
Figure~\ref{fig:unnamed-chunk-156} shows the smoothed estimates by region over time and Figure~\ref{fig:unnamed-chunk-157} compares the smoothed estimates with direct estimates from each survey for each region over time.


% %%%%%%%%%%%%%%%%%%%%%%%%%%% Table1 
% <<echo=FALSE, results='asis'>>=
% load("rda/variance_tables.rda")
% countryname2 <- gsub(" ", "", countryname)
% variance <- tables.all[[countryname]]

% table_count <- table_count + 1

% names <- c("RW2 ($\\sigma^2_{\\gamma_{t}}$)", "ICAR ($\\sigma^2_{\\phi_{i}}$)", "IID space ($\\sigma^2_{\\theta_{i}}$)", "IID time ($\\sigma^2_{\\alpha_{t}}$)", "IID space time ($\\sigma^2_{\\delta_{it}}$)")

% variance$Proportion <- round(variance$Proportion*100, digits = 2)
% row.names(variance) <- names
% tab <- xtable(variance, digits = c(1, 3, 2),align = "l|ll",
%        label = paste0("tab:", countryname, "-var"),
%        caption = paste(country, ": summary of the variance components in the RW2 model", sep = ''))
% print(tab, comment = FALSE,sanitize.text.function = function(x) {x})
% @

%%%%%%%%%%%%%%%%%%%%%%%%%%% Plot1 
\begin{knitrout}
\definecolor{shadecolor}{rgb}{0.969, 0.969, 0.969}\color{fgcolor}\begin{figure}[bht]

{\centering \includegraphics[width=.9\textwidth]{../Main/Figures/SmoothvDirectGuinea_meta} 

}

\caption[Guinea]{Guinea: Smooth versus direct Admin 1 estimates. Left: Combined (meta-analysis) survey estimate against combined direct estimates. Right: Combined (meta-analysis) survey estimate against direct estimates from each survey.}\label{fig:unnamed-chunk-153}
\end{figure}


\end{knitrout}

%%%%%%%%%%%%%%%%%%%%%%%%%%% Plot2 
\begin{knitrout}
\definecolor{shadecolor}{rgb}{0.969, 0.969, 0.969}\color{fgcolor}\begin{figure}[bht]

{\centering \includegraphics[width=.9\textwidth]{../Main/Figures/SmoothMedianGuinea} 

}

\caption[Guinea]{Guinea: Maps of posterior medians over time.}\label{fig:unnamed-chunk-154}
\end{figure}


\end{knitrout}
%%%%%%%%%%%%%%%%%%%%%%%%%%% Plot2a
\begin{knitrout}
\definecolor{shadecolor}{rgb}{0.969, 0.969, 0.969}\color{fgcolor}\begin{figure}[bht]

{\centering \includegraphics[width=.9\textwidth]{../Main/Figures/ReductionMedianGuinea} 

}

\caption[Guinea]{Guinea: Maps of reduction of posterior median U5MR in each five-year period compared to 1990 over time.}\label{fig:unnamed-chunk-155}
\end{figure}


\end{knitrout}
%%%%%%%%%%%%%%%%%%%%%%%%%%% Plot3 
\begin{knitrout}
\definecolor{shadecolor}{rgb}{0.969, 0.969, 0.969}\color{fgcolor}\begin{figure}[bht]

{\centering \includegraphics[width=.95\textwidth]{../Main/Figures/Yearly_v_Periods_Guinea} 

}

\caption[Guinea]{Guinea: Smoothed regional estimates over time. The line indicates yearly posterior median estimates and error bars indicate 95 \% posterior credible interval at each time period.}\label{fig:unnamed-chunk-156}
\end{figure}


\end{knitrout}

%%%%%%%%%%%%%%%%%%%%%%%%%%% Plot4 
\begin{knitrout}
\definecolor{shadecolor}{rgb}{0.969, 0.969, 0.969}\color{fgcolor}\begin{figure}[bht]

{\centering \includegraphics[width=.9\textwidth]{../Main/Figures/LineSubMedianGuinea} 

}

\caption[Guinea]{Guinea: Smoothed regional estimates over time compared to the direct estimates from each surveys. Direct estimates are not benchmarked with UN estimates. The line indicates posterior median and error bars indicate 95\% posterior credible interval.}\label{fig:unnamed-chunk-157}
\end{figure}


\end{knitrout}
% \subsubsection{National model results}
We further assess the RW2 model by holding out some observations, and compare the projections to the direct estimates in these holdout observations. Figure~\ref{fig:unnamed-chunk-158} compares the predicted estimates for the out-of-sample observations  with the direct estimates by holding out observations from each area in each time period.  Figure~\ref{fig:unnamed-chunk-159} compares the histogram of the bias rescaled by the total variance in the cross validation studies. Figure~\ref{fig:unnamed-chunk-160} compares the rescaled bias by region and time periods.



% %%%%%%%%%%%%%%%%%%%%%%%%%%% Plot6
% << echo=FALSE, out.width = ".9\\textwidth", fig.width = 12, fig.height = 6, fig.cap = "Out-of-sample predictions along with direct estimates in the cross validation study where all data from each time period is held out and predicted using the rest of the data.">>=
% fig_count <- fig_count + 1
% knitr::include_graphics(paste0("../Main/Figures/CV_byYear_withError_", countryname2, ".pdf")) 
% @
 
%%%%%%%%%%%%%%%%%%%%%%%%%%% Plot7
\begin{knitrout}
\definecolor{shadecolor}{rgb}{0.969, 0.969, 0.969}\color{fgcolor}\begin{figure}[bht]

{\centering \includegraphics[width=.9\textwidth]{../Main/Figures/CV_byYearRegion_withError_Guinea} 

}

\caption[Guinea]{Guinea: Out-of-sample predictions along with direct estimates in the cross validation study where data from one region in each time period is held out and predicted using the rest of the data.}\label{fig:unnamed-chunk-158}
\end{figure}


\end{knitrout}

%%%%%%%%%%%%%%%%%%%%%%%%%%% Plot8
\begin{knitrout}
\definecolor{shadecolor}{rgb}{0.969, 0.969, 0.969}\color{fgcolor}\begin{figure}[bht]

{\centering \includegraphics[width=.9\textwidth]{../Main/Figures/CVbiasGuinea} 

}

\caption[Guinea]{Guinea: Histogram and QQ-plot of the rescaled difference between the smoothed estimates and the direct estimates in the cross validation study. The differences between the two estimates are rescaled by the square root of the total variance of the two estimates.}\label{fig:unnamed-chunk-159}
\end{figure}


\end{knitrout}

%%%%%%%%%%%%%%%%%%%%%%%%%%% Plot9
\begin{knitrout}
\definecolor{shadecolor}{rgb}{0.969, 0.969, 0.969}\color{fgcolor}\begin{figure}[bht]

{\centering \includegraphics[width=.7\textwidth]{../Main/Figures/CVbiasbyRegionGuinea} 

}

\caption[Guinea]{Guinea: Line plot of the difference between smoothed estimates and the direct estimates in the cross validation study. The differences between the two estimates are rescaled by the square root of the total variance of the two estimates.}\label{fig:unnamed-chunk-160}
\end{figure}


\end{knitrout}


%%%%%%%%%%%%%%%%%%%%%%%%%%%%%%%%%%%%%%%%%%%%%%%%%%%%%%%%%%%%%%%%%%%%%%%%%%%%%%%%%%%%%%%%%%%%%%%%%%
\clearpage
\subsubsection{Kenya}


% \subsubsection{Summary of DHS surveys}

%%%%%%%%%%%%%%%%%%%%%%%%%%% Summary 


DHS surveys were conducted in Kenya in 1993, 1998, 2003, 2008, and 2014.
% years.out[1:(length(years.out)-1)], and years.out[length(years.out)]. 

We fit both the RW2 only model to the combined national data, and compare the time trend at national level with the estimates produced by the UN and IHME in Figure~\ref{fig:unnamed-chunk-162}. We then adjusted the combined national data to the UN estimates of U5MR, and refit the models on the benchmarked data. 

%%%%%%%%%%%%%%%%%%%%%%%%%% Plot5 
\begin{knitrout}
\definecolor{shadecolor}{rgb}{0.969, 0.969, 0.969}\color{fgcolor}\begin{figure}[bht]

{\centering \includegraphics[width=.9\textwidth]{../Main/Figures/Yearly_national_Kenya} 

}

\caption[Kenya]{Kenya: Temporal national trends along with UN (B3) estimates described in You et al. (2015) and IHME estimates based on GBD 2015 Child Mortality Collaborators (2016). RW2 represents the smoothed national estimates using the original data before benchmarking with UN estimates. RW2-adj represents the smoothed national estimates using the benchmarked data.}\label{fig:unnamed-chunk-162}
\end{figure}


\end{knitrout}
 

We fit the RW2 model to the benchmarked data in each area. 
% The proportions of the explained variation is summarized in Table~\ref{tab:paste0(countryname, "-var")}. 
We compare the results in Figure~\ref{fig:unnamed-chunk-163} to \ref{fig:unnamed-chunk-167}.
Figure~\ref{fig:unnamed-chunk-163} compares the smoothed estimates against the direct estimates. Figure~\ref{fig:unnamed-chunk-164} and Figure~\ref{fig:unnamed-chunk-165} show the posterior median estimates of U5MR in each region over time and the reductions from 1990 period respectively.
Figure~\ref{fig:unnamed-chunk-166} shows the smoothed estimates by region over time and Figure~\ref{fig:unnamed-chunk-167} compares the smoothed estimates with direct estimates from each survey for each region over time.


% %%%%%%%%%%%%%%%%%%%%%%%%%%% Table1 
% <<echo=FALSE, results='asis'>>=
% load("rda/variance_tables.rda")
% countryname2 <- gsub(" ", "", countryname)
% variance <- tables.all[[countryname]]

% table_count <- table_count + 1

% names <- c("RW2 ($\\sigma^2_{\\gamma_{t}}$)", "ICAR ($\\sigma^2_{\\phi_{i}}$)", "IID space ($\\sigma^2_{\\theta_{i}}$)", "IID time ($\\sigma^2_{\\alpha_{t}}$)", "IID space time ($\\sigma^2_{\\delta_{it}}$)")

% variance$Proportion <- round(variance$Proportion*100, digits = 2)
% row.names(variance) <- names
% tab <- xtable(variance, digits = c(1, 3, 2),align = "l|ll",
%        label = paste0("tab:", countryname, "-var"),
%        caption = paste(country, ": summary of the variance components in the RW2 model", sep = ''))
% print(tab, comment = FALSE,sanitize.text.function = function(x) {x})
% @

%%%%%%%%%%%%%%%%%%%%%%%%%%% Plot1 
\begin{knitrout}
\definecolor{shadecolor}{rgb}{0.969, 0.969, 0.969}\color{fgcolor}\begin{figure}[bht]

{\centering \includegraphics[width=.9\textwidth]{../Main/Figures/SmoothvDirectKenya_meta} 

}

\caption[Kenya]{Kenya: Smooth versus direct Admin 1 estimates. Left: Combined (meta-analysis) survey estimate against combined direct estimates. Right: Combined (meta-analysis) survey estimate against direct estimates from each survey.}\label{fig:unnamed-chunk-163}
\end{figure}


\end{knitrout}

%%%%%%%%%%%%%%%%%%%%%%%%%%% Plot2 
\begin{knitrout}
\definecolor{shadecolor}{rgb}{0.969, 0.969, 0.969}\color{fgcolor}\begin{figure}[bht]

{\centering \includegraphics[width=.9\textwidth]{../Main/Figures/SmoothMedianKenya} 

}

\caption[Kenya]{Kenya: Maps of posterior medians over time.}\label{fig:unnamed-chunk-164}
\end{figure}


\end{knitrout}
%%%%%%%%%%%%%%%%%%%%%%%%%%% Plot2a
\begin{knitrout}
\definecolor{shadecolor}{rgb}{0.969, 0.969, 0.969}\color{fgcolor}\begin{figure}[bht]

{\centering \includegraphics[width=.9\textwidth]{../Main/Figures/ReductionMedianKenya} 

}

\caption[Kenya]{Kenya: Maps of reduction of posterior median U5MR in each five-year period compared to 1990 over time.}\label{fig:unnamed-chunk-165}
\end{figure}


\end{knitrout}
%%%%%%%%%%%%%%%%%%%%%%%%%%% Plot3 
\begin{knitrout}
\definecolor{shadecolor}{rgb}{0.969, 0.969, 0.969}\color{fgcolor}\begin{figure}[bht]

{\centering \includegraphics[width=.95\textwidth]{../Main/Figures/Yearly_v_Periods_Kenya} 

}

\caption[Kenya]{Kenya: Smoothed regional estimates over time. The line indicates yearly posterior median estimates and error bars indicate 95 \% posterior credible interval at each time period.}\label{fig:unnamed-chunk-166}
\end{figure}


\end{knitrout}

%%%%%%%%%%%%%%%%%%%%%%%%%%% Plot4 
\begin{knitrout}
\definecolor{shadecolor}{rgb}{0.969, 0.969, 0.969}\color{fgcolor}\begin{figure}[bht]

{\centering \includegraphics[width=.9\textwidth]{../Main/Figures/LineSubMedianKenya} 

}

\caption[Kenya]{Kenya: Smoothed regional estimates over time compared to the direct estimates from each surveys. Direct estimates are not benchmarked with UN estimates. The line indicates posterior median and error bars indicate 95\% posterior credible interval.}\label{fig:unnamed-chunk-167}
\end{figure}


\end{knitrout}
% \subsubsection{National model results}
We further assess the RW2 model by holding out some observations, and compare the projections to the direct estimates in these holdout observations. Figure~\ref{fig:unnamed-chunk-168} compares the predicted estimates for the out-of-sample observations  with the direct estimates by holding out observations from each area in each time period.  Figure~\ref{fig:unnamed-chunk-169} compares the histogram of the bias rescaled by the total variance in the cross validation studies. Figure~\ref{fig:unnamed-chunk-170} compares the rescaled bias by region and time periods.



% %%%%%%%%%%%%%%%%%%%%%%%%%%% Plot6
% << echo=FALSE, out.width = ".9\\textwidth", fig.width = 12, fig.height = 6, fig.cap = "Out-of-sample predictions along with direct estimates in the cross validation study where all data from each time period is held out and predicted using the rest of the data.">>=
% fig_count <- fig_count + 1
% knitr::include_graphics(paste0("../Main/Figures/CV_byYear_withError_", countryname2, ".pdf")) 
% @
 
%%%%%%%%%%%%%%%%%%%%%%%%%%% Plot7
\begin{knitrout}
\definecolor{shadecolor}{rgb}{0.969, 0.969, 0.969}\color{fgcolor}\begin{figure}[bht]

{\centering \includegraphics[width=.9\textwidth]{../Main/Figures/CV_byYearRegion_withError_Kenya} 

}

\caption[Kenya]{Kenya: Out-of-sample predictions along with direct estimates in the cross validation study where data from one region in each time period is held out and predicted using the rest of the data.}\label{fig:unnamed-chunk-168}
\end{figure}


\end{knitrout}

%%%%%%%%%%%%%%%%%%%%%%%%%%% Plot8
\begin{knitrout}
\definecolor{shadecolor}{rgb}{0.969, 0.969, 0.969}\color{fgcolor}\begin{figure}[bht]

{\centering \includegraphics[width=.9\textwidth]{../Main/Figures/CVbiasKenya} 

}

\caption[Kenya]{Kenya: Histogram and QQ-plot of the rescaled difference between the smoothed estimates and the direct estimates in the cross validation study. The differences between the two estimates are rescaled by the square root of the total variance of the two estimates.}\label{fig:unnamed-chunk-169}
\end{figure}


\end{knitrout}

%%%%%%%%%%%%%%%%%%%%%%%%%%% Plot9
\begin{knitrout}
\definecolor{shadecolor}{rgb}{0.969, 0.969, 0.969}\color{fgcolor}\begin{figure}[bht]

{\centering \includegraphics[width=.7\textwidth]{../Main/Figures/CVbiasbyRegionKenya} 

}

\caption[Kenya]{Kenya: Line plot of the difference between smoothed estimates and the direct estimates in the cross validation study. The differences between the two estimates are rescaled by the square root of the total variance of the two estimates.}\label{fig:unnamed-chunk-170}
\end{figure}


\end{knitrout}


%%%%%%%%%%%%%%%%%%%%%%%%%%%%%%%%%%%%%%%%%%%%%%%%%%%%%%%%%%%%%%%%%%%%%%%%%%%%%%%%%%%%%%%%%%%%%%%%%%
\clearpage
\subsubsection{Lesotho}


% \subsubsection{Summary of DHS surveys}

%%%%%%%%%%%%%%%%%%%%%%%%%%% Summary 


DHS surveys were conducted in Lesotho in 2005, 2010, and 2014.
% years.out[1:(length(years.out)-1)], and years.out[length(years.out)]. 

We fit both the RW2 only model to the combined national data, and compare the time trend at national level with the estimates produced by the UN and IHME in Figure~\ref{fig:unnamed-chunk-172}. We then adjusted the combined national data to the UN estimates of U5MR, and refit the models on the benchmarked data. 

%%%%%%%%%%%%%%%%%%%%%%%%%% Plot5 
\begin{knitrout}
\definecolor{shadecolor}{rgb}{0.969, 0.969, 0.969}\color{fgcolor}\begin{figure}[bht]

{\centering \includegraphics[width=.9\textwidth]{../Main/Figures/Yearly_national_Lesotho} 

}

\caption[Lesotho]{Lesotho: Temporal national trends along with UN (B3) estimates described in You et al. (2015) and IHME estimates based on GBD 2015 Child Mortality Collaborators (2016). RW2 represents the smoothed national estimates using the original data before benchmarking with UN estimates. RW2-adj represents the smoothed national estimates using the benchmarked data.}\label{fig:unnamed-chunk-172}
\end{figure}


\end{knitrout}
 

We fit the RW2 model to the benchmarked data in each area. 
% The proportions of the explained variation is summarized in Table~\ref{tab:paste0(countryname, "-var")}. 
We compare the results in Figure~\ref{fig:unnamed-chunk-173} to \ref{fig:unnamed-chunk-177}.
Figure~\ref{fig:unnamed-chunk-173} compares the smoothed estimates against the direct estimates. Figure~\ref{fig:unnamed-chunk-174} and Figure~\ref{fig:unnamed-chunk-175} show the posterior median estimates of U5MR in each region over time and the reductions from 1990 period respectively.
Figure~\ref{fig:unnamed-chunk-176} shows the smoothed estimates by region over time and Figure~\ref{fig:unnamed-chunk-177} compares the smoothed estimates with direct estimates from each survey for each region over time.


% %%%%%%%%%%%%%%%%%%%%%%%%%%% Table1 
% <<echo=FALSE, results='asis'>>=
% load("rda/variance_tables.rda")
% countryname2 <- gsub(" ", "", countryname)
% variance <- tables.all[[countryname]]

% table_count <- table_count + 1

% names <- c("RW2 ($\\sigma^2_{\\gamma_{t}}$)", "ICAR ($\\sigma^2_{\\phi_{i}}$)", "IID space ($\\sigma^2_{\\theta_{i}}$)", "IID time ($\\sigma^2_{\\alpha_{t}}$)", "IID space time ($\\sigma^2_{\\delta_{it}}$)")

% variance$Proportion <- round(variance$Proportion*100, digits = 2)
% row.names(variance) <- names
% tab <- xtable(variance, digits = c(1, 3, 2),align = "l|ll",
%        label = paste0("tab:", countryname, "-var"),
%        caption = paste(country, ": summary of the variance components in the RW2 model", sep = ''))
% print(tab, comment = FALSE,sanitize.text.function = function(x) {x})
% @

%%%%%%%%%%%%%%%%%%%%%%%%%%% Plot1 
\begin{knitrout}
\definecolor{shadecolor}{rgb}{0.969, 0.969, 0.969}\color{fgcolor}\begin{figure}[bht]

{\centering \includegraphics[width=.9\textwidth]{../Main/Figures/SmoothvDirectLesotho_meta} 

}

\caption[Lesotho]{Lesotho: Smooth versus direct Admin 1 estimates. Left: Combined (meta-analysis) survey estimate against combined direct estimates. Right: Combined (meta-analysis) survey estimate against direct estimates from each survey.}\label{fig:unnamed-chunk-173}
\end{figure}


\end{knitrout}

%%%%%%%%%%%%%%%%%%%%%%%%%%% Plot2 
\begin{knitrout}
\definecolor{shadecolor}{rgb}{0.969, 0.969, 0.969}\color{fgcolor}\begin{figure}[bht]

{\centering \includegraphics[width=.9\textwidth]{../Main/Figures/SmoothMedianLesotho} 

}

\caption[Lesotho]{Lesotho: Maps of posterior medians over time.}\label{fig:unnamed-chunk-174}
\end{figure}


\end{knitrout}
%%%%%%%%%%%%%%%%%%%%%%%%%%% Plot2a
\begin{knitrout}
\definecolor{shadecolor}{rgb}{0.969, 0.969, 0.969}\color{fgcolor}\begin{figure}[bht]

{\centering \includegraphics[width=.9\textwidth]{../Main/Figures/ReductionMedianLesotho} 

}

\caption[Lesotho]{Lesotho: Maps of reduction of posterior median U5MR in each five-year period compared to 1990 over time.}\label{fig:unnamed-chunk-175}
\end{figure}


\end{knitrout}
%%%%%%%%%%%%%%%%%%%%%%%%%%% Plot3 
\begin{knitrout}
\definecolor{shadecolor}{rgb}{0.969, 0.969, 0.969}\color{fgcolor}\begin{figure}[bht]

{\centering \includegraphics[width=.95\textwidth]{../Main/Figures/Yearly_v_Periods_Lesotho} 

}

\caption[Lesotho]{Lesotho: Smoothed regional estimates over time. The line indicates yearly posterior median estimates and error bars indicate 95 \% posterior credible interval at each time period.}\label{fig:unnamed-chunk-176}
\end{figure}


\end{knitrout}

%%%%%%%%%%%%%%%%%%%%%%%%%%% Plot4 
\begin{knitrout}
\definecolor{shadecolor}{rgb}{0.969, 0.969, 0.969}\color{fgcolor}\begin{figure}[bht]

{\centering \includegraphics[width=.9\textwidth]{../Main/Figures/LineSubMedianLesotho} 

}

\caption[Lesotho]{Lesotho: Smoothed regional estimates over time compared to the direct estimates from each surveys. Direct estimates are not benchmarked with UN estimates. The line indicates posterior median and error bars indicate 95\% posterior credible interval.}\label{fig:unnamed-chunk-177}
\end{figure}


\end{knitrout}
% \subsubsection{National model results}
We further assess the RW2 model by holding out some observations, and compare the projections to the direct estimates in these holdout observations. Figure~\ref{fig:unnamed-chunk-178} compares the predicted estimates for the out-of-sample observations  with the direct estimates by holding out observations from each area in each time period.  Figure~\ref{fig:unnamed-chunk-179} compares the histogram of the bias rescaled by the total variance in the cross validation studies. Figure~\ref{fig:unnamed-chunk-180} compares the rescaled bias by region and time periods.



% %%%%%%%%%%%%%%%%%%%%%%%%%%% Plot6
% << echo=FALSE, out.width = ".9\\textwidth", fig.width = 12, fig.height = 6, fig.cap = "Out-of-sample predictions along with direct estimates in the cross validation study where all data from each time period is held out and predicted using the rest of the data.">>=
% fig_count <- fig_count + 1
% knitr::include_graphics(paste0("../Main/Figures/CV_byYear_withError_", countryname2, ".pdf")) 
% @
 
%%%%%%%%%%%%%%%%%%%%%%%%%%% Plot7
\begin{knitrout}
\definecolor{shadecolor}{rgb}{0.969, 0.969, 0.969}\color{fgcolor}\begin{figure}[bht]

{\centering \includegraphics[width=.9\textwidth]{../Main/Figures/CV_byYearRegion_withError_Lesotho} 

}

\caption[Lesotho]{Lesotho: Out-of-sample predictions along with direct estimates in the cross validation study where data from one region in each time period is held out and predicted using the rest of the data.}\label{fig:unnamed-chunk-178}
\end{figure}


\end{knitrout}

%%%%%%%%%%%%%%%%%%%%%%%%%%% Plot8
\begin{knitrout}
\definecolor{shadecolor}{rgb}{0.969, 0.969, 0.969}\color{fgcolor}\begin{figure}[bht]

{\centering \includegraphics[width=.9\textwidth]{../Main/Figures/CVbiasLesotho} 

}

\caption[Lesotho]{Lesotho: Histogram and QQ-plot of the rescaled difference between the smoothed estimates and the direct estimates in the cross validation study. The differences between the two estimates are rescaled by the square root of the total variance of the two estimates.}\label{fig:unnamed-chunk-179}
\end{figure}


\end{knitrout}

%%%%%%%%%%%%%%%%%%%%%%%%%%% Plot9
\begin{knitrout}
\definecolor{shadecolor}{rgb}{0.969, 0.969, 0.969}\color{fgcolor}\begin{figure}[bht]

{\centering \includegraphics[width=.7\textwidth]{../Main/Figures/CVbiasbyRegionLesotho} 

}

\caption[Lesotho]{Lesotho: Line plot of the difference between smoothed estimates and the direct estimates in the cross validation study. The differences between the two estimates are rescaled by the square root of the total variance of the two estimates.}\label{fig:unnamed-chunk-180}
\end{figure}


\end{knitrout}


%%%%%%%%%%%%%%%%%%%%%%%%%%%%%%%%%%%%%%%%%%%%%%%%%%%%%%%%%%%%%%%%%%%%%%%%%%%%%%%%%%%%%%%%%%%%%%%%%%
\clearpage
\subsubsection{Liberia}


% \subsubsection{Summary of DHS surveys}

%%%%%%%%%%%%%%%%%%%%%%%%%%% Summary 


DHS surveys were conducted in Liberia in 2007, and 2013.
% years.out[1:(length(years.out)-1)], and years.out[length(years.out)]. 

We fit both the RW2 only model to the combined national data, and compare the time trend at national level with the estimates produced by the UN and IHME in Figure~\ref{fig:unnamed-chunk-182}. We then adjusted the combined national data to the UN estimates of U5MR, and refit the models on the benchmarked data. 

%%%%%%%%%%%%%%%%%%%%%%%%%% Plot5 
\begin{knitrout}
\definecolor{shadecolor}{rgb}{0.969, 0.969, 0.969}\color{fgcolor}\begin{figure}[bht]

{\centering \includegraphics[width=.9\textwidth]{../Main/Figures/Yearly_national_Liberia} 

}

\caption[Liberia]{Liberia: Temporal national trends along with UN (B3) estimates described in You et al. (2015) and IHME estimates based on GBD 2015 Child Mortality Collaborators (2016). RW2 represents the smoothed national estimates using the original data before benchmarking with UN estimates. RW2-adj represents the smoothed national estimates using the benchmarked data.}\label{fig:unnamed-chunk-182}
\end{figure}


\end{knitrout}
 

We fit the RW2 model to the benchmarked data in each area. 
% The proportions of the explained variation is summarized in Table~\ref{tab:paste0(countryname, "-var")}. 
We compare the results in Figure~\ref{fig:unnamed-chunk-183} to \ref{fig:unnamed-chunk-187}.
Figure~\ref{fig:unnamed-chunk-183} compares the smoothed estimates against the direct estimates. Figure~\ref{fig:unnamed-chunk-184} and Figure~\ref{fig:unnamed-chunk-185} show the posterior median estimates of U5MR in each region over time and the reductions from 1990 period respectively.
Figure~\ref{fig:unnamed-chunk-186} shows the smoothed estimates by region over time and Figure~\ref{fig:unnamed-chunk-187} compares the smoothed estimates with direct estimates from each survey for each region over time.


% %%%%%%%%%%%%%%%%%%%%%%%%%%% Table1 
% <<echo=FALSE, results='asis'>>=
% load("rda/variance_tables.rda")
% countryname2 <- gsub(" ", "", countryname)
% variance <- tables.all[[countryname]]

% table_count <- table_count + 1

% names <- c("RW2 ($\\sigma^2_{\\gamma_{t}}$)", "ICAR ($\\sigma^2_{\\phi_{i}}$)", "IID space ($\\sigma^2_{\\theta_{i}}$)", "IID time ($\\sigma^2_{\\alpha_{t}}$)", "IID space time ($\\sigma^2_{\\delta_{it}}$)")

% variance$Proportion <- round(variance$Proportion*100, digits = 2)
% row.names(variance) <- names
% tab <- xtable(variance, digits = c(1, 3, 2),align = "l|ll",
%        label = paste0("tab:", countryname, "-var"),
%        caption = paste(country, ": summary of the variance components in the RW2 model", sep = ''))
% print(tab, comment = FALSE,sanitize.text.function = function(x) {x})
% @

%%%%%%%%%%%%%%%%%%%%%%%%%%% Plot1 
\begin{knitrout}
\definecolor{shadecolor}{rgb}{0.969, 0.969, 0.969}\color{fgcolor}\begin{figure}[bht]

{\centering \includegraphics[width=.9\textwidth]{../Main/Figures/SmoothvDirectLiberia_meta} 

}

\caption[Liberia]{Liberia: Smooth versus direct Admin 1 estimates. Left: Combined (meta-analysis) survey estimate against combined direct estimates. Right: Combined (meta-analysis) survey estimate against direct estimates from each survey.}\label{fig:unnamed-chunk-183}
\end{figure}


\end{knitrout}

%%%%%%%%%%%%%%%%%%%%%%%%%%% Plot2 
\begin{knitrout}
\definecolor{shadecolor}{rgb}{0.969, 0.969, 0.969}\color{fgcolor}\begin{figure}[bht]

{\centering \includegraphics[width=.9\textwidth]{../Main/Figures/SmoothMedianLiberia} 

}

\caption[Liberia]{Liberia: Maps of posterior medians over time.}\label{fig:unnamed-chunk-184}
\end{figure}


\end{knitrout}
%%%%%%%%%%%%%%%%%%%%%%%%%%% Plot2a
\begin{knitrout}
\definecolor{shadecolor}{rgb}{0.969, 0.969, 0.969}\color{fgcolor}\begin{figure}[bht]

{\centering \includegraphics[width=.9\textwidth]{../Main/Figures/ReductionMedianLiberia} 

}

\caption[Liberia]{Liberia: Maps of reduction of posterior median U5MR in each five-year period compared to 1990 over time.}\label{fig:unnamed-chunk-185}
\end{figure}


\end{knitrout}
%%%%%%%%%%%%%%%%%%%%%%%%%%% Plot3 
\begin{knitrout}
\definecolor{shadecolor}{rgb}{0.969, 0.969, 0.969}\color{fgcolor}\begin{figure}[bht]

{\centering \includegraphics[width=.95\textwidth]{../Main/Figures/Yearly_v_Periods_Liberia} 

}

\caption[Liberia]{Liberia: Smoothed regional estimates over time. The line indicates yearly posterior median estimates and error bars indicate 95 \% posterior credible interval at each time period.}\label{fig:unnamed-chunk-186}
\end{figure}


\end{knitrout}

%%%%%%%%%%%%%%%%%%%%%%%%%%% Plot4 
\begin{knitrout}
\definecolor{shadecolor}{rgb}{0.969, 0.969, 0.969}\color{fgcolor}\begin{figure}[bht]

{\centering \includegraphics[width=.9\textwidth]{../Main/Figures/LineSubMedianLiberia} 

}

\caption[Liberia]{Liberia: Smoothed regional estimates over time compared to the direct estimates from each surveys. Direct estimates are not benchmarked with UN estimates. The line indicates posterior median and error bars indicate 95\% posterior credible interval.}\label{fig:unnamed-chunk-187}
\end{figure}


\end{knitrout}
% \subsubsection{National model results}
We further assess the RW2 model by holding out some observations, and compare the projections to the direct estimates in these holdout observations. Figure~\ref{fig:unnamed-chunk-188} compares the predicted estimates for the out-of-sample observations  with the direct estimates by holding out observations from each area in each time period.  Figure~\ref{fig:unnamed-chunk-189} compares the histogram of the bias rescaled by the total variance in the cross validation studies. Figure~\ref{fig:unnamed-chunk-190} compares the rescaled bias by region and time periods.



% %%%%%%%%%%%%%%%%%%%%%%%%%%% Plot6
% << echo=FALSE, out.width = ".9\\textwidth", fig.width = 12, fig.height = 6, fig.cap = "Out-of-sample predictions along with direct estimates in the cross validation study where all data from each time period is held out and predicted using the rest of the data.">>=
% fig_count <- fig_count + 1
% knitr::include_graphics(paste0("../Main/Figures/CV_byYear_withError_", countryname2, ".pdf")) 
% @
 
%%%%%%%%%%%%%%%%%%%%%%%%%%% Plot7
\begin{knitrout}
\definecolor{shadecolor}{rgb}{0.969, 0.969, 0.969}\color{fgcolor}\begin{figure}[bht]

{\centering \includegraphics[width=.9\textwidth]{../Main/Figures/CV_byYearRegion_withError_Liberia} 

}

\caption[Liberia]{Liberia: Out-of-sample predictions along with direct estimates in the cross validation study where data from one region in each time period is held out and predicted using the rest of the data.}\label{fig:unnamed-chunk-188}
\end{figure}


\end{knitrout}

%%%%%%%%%%%%%%%%%%%%%%%%%%% Plot8
\begin{knitrout}
\definecolor{shadecolor}{rgb}{0.969, 0.969, 0.969}\color{fgcolor}\begin{figure}[bht]

{\centering \includegraphics[width=.9\textwidth]{../Main/Figures/CVbiasLiberia} 

}

\caption[Liberia]{Liberia: Histogram and QQ-plot of the rescaled difference between the smoothed estimates and the direct estimates in the cross validation study. The differences between the two estimates are rescaled by the square root of the total variance of the two estimates.}\label{fig:unnamed-chunk-189}
\end{figure}


\end{knitrout}

%%%%%%%%%%%%%%%%%%%%%%%%%%% Plot9
\begin{knitrout}
\definecolor{shadecolor}{rgb}{0.969, 0.969, 0.969}\color{fgcolor}\begin{figure}[bht]

{\centering \includegraphics[width=.7\textwidth]{../Main/Figures/CVbiasbyRegionLiberia} 

}

\caption[Liberia]{Liberia: Line plot of the difference between smoothed estimates and the direct estimates in the cross validation study. The differences between the two estimates are rescaled by the square root of the total variance of the two estimates.}\label{fig:unnamed-chunk-190}
\end{figure}


\end{knitrout}


%%%%%%%%%%%%%%%%%%%%%%%%%%%%%%%%%%%%%%%%%%%%%%%%%%%%%%%%%%%%%%%%%%%%%%%%%%%%%%%%%%%%%%%%%%%%%%%%%%
\clearpage
\subsubsection{Madagascar}


% \subsubsection{Summary of DHS surveys}

%%%%%%%%%%%%%%%%%%%%%%%%%%% Summary 


DHS surveys were conducted in Madagascar in 1992, 1997, 2004, and 2009.
% years.out[1:(length(years.out)-1)], and years.out[length(years.out)]. 

We fit both the RW2 only model to the combined national data, and compare the time trend at national level with the estimates produced by the UN and IHME in Figure~\ref{fig:unnamed-chunk-192}. We then adjusted the combined national data to the UN estimates of U5MR, and refit the models on the benchmarked data. 

%%%%%%%%%%%%%%%%%%%%%%%%%% Plot5 
\begin{knitrout}
\definecolor{shadecolor}{rgb}{0.969, 0.969, 0.969}\color{fgcolor}\begin{figure}[bht]

{\centering \includegraphics[width=.9\textwidth]{../Main/Figures/Yearly_national_Madagascar} 

}

\caption[Madagascar]{Madagascar: Temporal national trends along with UN (B3) estimates described in You et al. (2015) and IHME estimates based on GBD 2015 Child Mortality Collaborators (2016). RW2 represents the smoothed national estimates using the original data before benchmarking with UN estimates. RW2-adj represents the smoothed national estimates using the benchmarked data.}\label{fig:unnamed-chunk-192}
\end{figure}


\end{knitrout}
 

We fit the RW2 model to the benchmarked data in each area. 
% The proportions of the explained variation is summarized in Table~\ref{tab:paste0(countryname, "-var")}. 
We compare the results in Figure~\ref{fig:unnamed-chunk-193} to \ref{fig:unnamed-chunk-197}.
Figure~\ref{fig:unnamed-chunk-193} compares the smoothed estimates against the direct estimates. Figure~\ref{fig:unnamed-chunk-194} and Figure~\ref{fig:unnamed-chunk-195} show the posterior median estimates of U5MR in each region over time and the reductions from 1990 period respectively.
Figure~\ref{fig:unnamed-chunk-196} shows the smoothed estimates by region over time and Figure~\ref{fig:unnamed-chunk-197} compares the smoothed estimates with direct estimates from each survey for each region over time.


% %%%%%%%%%%%%%%%%%%%%%%%%%%% Table1 
% <<echo=FALSE, results='asis'>>=
% load("rda/variance_tables.rda")
% countryname2 <- gsub(" ", "", countryname)
% variance <- tables.all[[countryname]]

% table_count <- table_count + 1

% names <- c("RW2 ($\\sigma^2_{\\gamma_{t}}$)", "ICAR ($\\sigma^2_{\\phi_{i}}$)", "IID space ($\\sigma^2_{\\theta_{i}}$)", "IID time ($\\sigma^2_{\\alpha_{t}}$)", "IID space time ($\\sigma^2_{\\delta_{it}}$)")

% variance$Proportion <- round(variance$Proportion*100, digits = 2)
% row.names(variance) <- names
% tab <- xtable(variance, digits = c(1, 3, 2),align = "l|ll",
%        label = paste0("tab:", countryname, "-var"),
%        caption = paste(country, ": summary of the variance components in the RW2 model", sep = ''))
% print(tab, comment = FALSE,sanitize.text.function = function(x) {x})
% @

%%%%%%%%%%%%%%%%%%%%%%%%%%% Plot1 
\begin{knitrout}
\definecolor{shadecolor}{rgb}{0.969, 0.969, 0.969}\color{fgcolor}\begin{figure}[bht]

{\centering \includegraphics[width=.9\textwidth]{../Main/Figures/SmoothvDirectMadagascar_meta} 

}

\caption[Madagascar]{Madagascar: Smooth versus direct Admin 1 estimates. Left: Combined (meta-analysis) survey estimate against combined direct estimates. Right: Combined (meta-analysis) survey estimate against direct estimates from each survey.}\label{fig:unnamed-chunk-193}
\end{figure}


\end{knitrout}

%%%%%%%%%%%%%%%%%%%%%%%%%%% Plot2 
\begin{knitrout}
\definecolor{shadecolor}{rgb}{0.969, 0.969, 0.969}\color{fgcolor}\begin{figure}[bht]

{\centering \includegraphics[width=.9\textwidth]{../Main/Figures/SmoothMedianMadagascar} 

}

\caption[Madagascar]{Madagascar: Maps of posterior medians over time.}\label{fig:unnamed-chunk-194}
\end{figure}


\end{knitrout}
%%%%%%%%%%%%%%%%%%%%%%%%%%% Plot2a
\begin{knitrout}
\definecolor{shadecolor}{rgb}{0.969, 0.969, 0.969}\color{fgcolor}\begin{figure}[bht]

{\centering \includegraphics[width=.9\textwidth]{../Main/Figures/ReductionMedianMadagascar} 

}

\caption[Madagascar]{Madagascar: Maps of reduction of posterior median U5MR in each five-year period compared to 1990 over time.}\label{fig:unnamed-chunk-195}
\end{figure}


\end{knitrout}
%%%%%%%%%%%%%%%%%%%%%%%%%%% Plot3 
\begin{knitrout}
\definecolor{shadecolor}{rgb}{0.969, 0.969, 0.969}\color{fgcolor}\begin{figure}[bht]

{\centering \includegraphics[width=.95\textwidth]{../Main/Figures/Yearly_v_Periods_Madagascar} 

}

\caption[Madagascar]{Madagascar: Smoothed regional estimates over time. The line indicates yearly posterior median estimates and error bars indicate 95 \% posterior credible interval at each time period.}\label{fig:unnamed-chunk-196}
\end{figure}


\end{knitrout}

%%%%%%%%%%%%%%%%%%%%%%%%%%% Plot4 
\begin{knitrout}
\definecolor{shadecolor}{rgb}{0.969, 0.969, 0.969}\color{fgcolor}\begin{figure}[bht]

{\centering \includegraphics[width=.9\textwidth]{../Main/Figures/LineSubMedianMadagascar} 

}

\caption[Madagascar]{Madagascar: Smoothed regional estimates over time compared to the direct estimates from each surveys. Direct estimates are not benchmarked with UN estimates. The line indicates posterior median and error bars indicate 95\% posterior credible interval.}\label{fig:unnamed-chunk-197}
\end{figure}


\end{knitrout}
% \subsubsection{National model results}
We further assess the RW2 model by holding out some observations, and compare the projections to the direct estimates in these holdout observations. Figure~\ref{fig:unnamed-chunk-198} compares the predicted estimates for the out-of-sample observations  with the direct estimates by holding out observations from each area in each time period.  Figure~\ref{fig:unnamed-chunk-199} compares the histogram of the bias rescaled by the total variance in the cross validation studies. Figure~\ref{fig:unnamed-chunk-200} compares the rescaled bias by region and time periods.



% %%%%%%%%%%%%%%%%%%%%%%%%%%% Plot6
% << echo=FALSE, out.width = ".9\\textwidth", fig.width = 12, fig.height = 6, fig.cap = "Out-of-sample predictions along with direct estimates in the cross validation study where all data from each time period is held out and predicted using the rest of the data.">>=
% fig_count <- fig_count + 1
% knitr::include_graphics(paste0("../Main/Figures/CV_byYear_withError_", countryname2, ".pdf")) 
% @
 
%%%%%%%%%%%%%%%%%%%%%%%%%%% Plot7
\begin{knitrout}
\definecolor{shadecolor}{rgb}{0.969, 0.969, 0.969}\color{fgcolor}\begin{figure}[bht]

{\centering \includegraphics[width=.9\textwidth]{../Main/Figures/CV_byYearRegion_withError_Madagascar} 

}

\caption[Madagascar]{Madagascar: Out-of-sample predictions along with direct estimates in the cross validation study where data from one region in each time period is held out and predicted using the rest of the data.}\label{fig:unnamed-chunk-198}
\end{figure}


\end{knitrout}

%%%%%%%%%%%%%%%%%%%%%%%%%%% Plot8
\begin{knitrout}
\definecolor{shadecolor}{rgb}{0.969, 0.969, 0.969}\color{fgcolor}\begin{figure}[bht]

{\centering \includegraphics[width=.9\textwidth]{../Main/Figures/CVbiasMadagascar} 

}

\caption[Madagascar]{Madagascar: Histogram and QQ-plot of the rescaled difference between the smoothed estimates and the direct estimates in the cross validation study. The differences between the two estimates are rescaled by the square root of the total variance of the two estimates.}\label{fig:unnamed-chunk-199}
\end{figure}


\end{knitrout}

%%%%%%%%%%%%%%%%%%%%%%%%%%% Plot9
\begin{knitrout}
\definecolor{shadecolor}{rgb}{0.969, 0.969, 0.969}\color{fgcolor}\begin{figure}[bht]

{\centering \includegraphics[width=.7\textwidth]{../Main/Figures/CVbiasbyRegionMadagascar} 

}

\caption[Madagascar]{Madagascar: Line plot of the difference between smoothed estimates and the direct estimates in the cross validation study. The differences between the two estimates are rescaled by the square root of the total variance of the two estimates.}\label{fig:unnamed-chunk-200}
\end{figure}


\end{knitrout}


%%%%%%%%%%%%%%%%%%%%%%%%%%%%%%%%%%%%%%%%%%%%%%%%%%%%%%%%%%%%%%%%%%%%%%%%%%%%%%%%%%%%%%%%%%%%%%%%%%
\clearpage
\subsubsection{Malawi}


% \subsubsection{Summary of DHS surveys}

%%%%%%%%%%%%%%%%%%%%%%%%%%% Summary 


DHS surveys were conducted in Malawi in 1992, 2000, 2004, 2010, and 2015.
% years.out[1:(length(years.out)-1)], and years.out[length(years.out)]. 

We fit both the RW2 only model to the combined national data, and compare the time trend at national level with the estimates produced by the UN and IHME in Figure~\ref{fig:unnamed-chunk-202}. We then adjusted the combined national data to the UN estimates of U5MR, and refit the models on the benchmarked data. 

%%%%%%%%%%%%%%%%%%%%%%%%%% Plot5 
\begin{knitrout}
\definecolor{shadecolor}{rgb}{0.969, 0.969, 0.969}\color{fgcolor}\begin{figure}[bht]

{\centering \includegraphics[width=.9\textwidth]{../Main/Figures/Yearly_national_Malawi} 

}

\caption[Malawi]{Malawi: Temporal national trends along with UN (B3) estimates described in You et al. (2015) and IHME estimates based on GBD 2015 Child Mortality Collaborators (2016). RW2 represents the smoothed national estimates using the original data before benchmarking with UN estimates. RW2-adj represents the smoothed national estimates using the benchmarked data.}\label{fig:unnamed-chunk-202}
\end{figure}


\end{knitrout}
 

We fit the RW2 model to the benchmarked data in each area. 
% The proportions of the explained variation is summarized in Table~\ref{tab:paste0(countryname, "-var")}. 
We compare the results in Figure~\ref{fig:unnamed-chunk-203} to \ref{fig:unnamed-chunk-207}.
Figure~\ref{fig:unnamed-chunk-203} compares the smoothed estimates against the direct estimates. Figure~\ref{fig:unnamed-chunk-204} and Figure~\ref{fig:unnamed-chunk-205} show the posterior median estimates of U5MR in each region over time and the reductions from 1990 period respectively.
Figure~\ref{fig:unnamed-chunk-206} shows the smoothed estimates by region over time and Figure~\ref{fig:unnamed-chunk-207} compares the smoothed estimates with direct estimates from each survey for each region over time.


% %%%%%%%%%%%%%%%%%%%%%%%%%%% Table1 
% <<echo=FALSE, results='asis'>>=
% load("rda/variance_tables.rda")
% countryname2 <- gsub(" ", "", countryname)
% variance <- tables.all[[countryname]]

% table_count <- table_count + 1

% names <- c("RW2 ($\\sigma^2_{\\gamma_{t}}$)", "ICAR ($\\sigma^2_{\\phi_{i}}$)", "IID space ($\\sigma^2_{\\theta_{i}}$)", "IID time ($\\sigma^2_{\\alpha_{t}}$)", "IID space time ($\\sigma^2_{\\delta_{it}}$)")

% variance$Proportion <- round(variance$Proportion*100, digits = 2)
% row.names(variance) <- names
% tab <- xtable(variance, digits = c(1, 3, 2),align = "l|ll",
%        label = paste0("tab:", countryname, "-var"),
%        caption = paste(country, ": summary of the variance components in the RW2 model", sep = ''))
% print(tab, comment = FALSE,sanitize.text.function = function(x) {x})
% @

%%%%%%%%%%%%%%%%%%%%%%%%%%% Plot1 
\begin{knitrout}
\definecolor{shadecolor}{rgb}{0.969, 0.969, 0.969}\color{fgcolor}\begin{figure}[bht]

{\centering \includegraphics[width=.9\textwidth]{../Main/Figures/SmoothvDirectMalawi_meta} 

}

\caption[Malawi]{Malawi: Smooth versus direct Admin 1 estimates. Left: Combined (meta-analysis) survey estimate against combined direct estimates. Right: Combined (meta-analysis) survey estimate against direct estimates from each survey.}\label{fig:unnamed-chunk-203}
\end{figure}


\end{knitrout}

%%%%%%%%%%%%%%%%%%%%%%%%%%% Plot2 
\begin{knitrout}
\definecolor{shadecolor}{rgb}{0.969, 0.969, 0.969}\color{fgcolor}\begin{figure}[bht]

{\centering \includegraphics[width=.9\textwidth]{../Main/Figures/SmoothMedianMalawi} 

}

\caption[Malawi]{Malawi: Maps of posterior medians over time.}\label{fig:unnamed-chunk-204}
\end{figure}


\end{knitrout}
%%%%%%%%%%%%%%%%%%%%%%%%%%% Plot2a
\begin{knitrout}
\definecolor{shadecolor}{rgb}{0.969, 0.969, 0.969}\color{fgcolor}\begin{figure}[bht]

{\centering \includegraphics[width=.9\textwidth]{../Main/Figures/ReductionMedianMalawi} 

}

\caption[Malawi]{Malawi: Maps of reduction of posterior median U5MR in each five-year period compared to 1990 over time.}\label{fig:unnamed-chunk-205}
\end{figure}


\end{knitrout}
%%%%%%%%%%%%%%%%%%%%%%%%%%% Plot3 
\begin{knitrout}
\definecolor{shadecolor}{rgb}{0.969, 0.969, 0.969}\color{fgcolor}\begin{figure}[bht]

{\centering \includegraphics[width=.95\textwidth]{../Main/Figures/Yearly_v_Periods_Malawi} 

}

\caption[Malawi]{Malawi: Smoothed regional estimates over time. The line indicates yearly posterior median estimates and error bars indicate 95 \% posterior credible interval at each time period.}\label{fig:unnamed-chunk-206}
\end{figure}


\end{knitrout}

%%%%%%%%%%%%%%%%%%%%%%%%%%% Plot4 
\begin{knitrout}
\definecolor{shadecolor}{rgb}{0.969, 0.969, 0.969}\color{fgcolor}\begin{figure}[bht]

{\centering \includegraphics[width=.9\textwidth]{../Main/Figures/LineSubMedianMalawi} 

}

\caption[Malawi]{Malawi: Smoothed regional estimates over time compared to the direct estimates from each surveys. Direct estimates are not benchmarked with UN estimates. The line indicates posterior median and error bars indicate 95\% posterior credible interval.}\label{fig:unnamed-chunk-207}
\end{figure}


\end{knitrout}
% \subsubsection{National model results}
We further assess the RW2 model by holding out some observations, and compare the projections to the direct estimates in these holdout observations. Figure~\ref{fig:unnamed-chunk-208} compares the predicted estimates for the out-of-sample observations  with the direct estimates by holding out observations from each area in each time period.  Figure~\ref{fig:unnamed-chunk-209} compares the histogram of the bias rescaled by the total variance in the cross validation studies. Figure~\ref{fig:unnamed-chunk-210} compares the rescaled bias by region and time periods.



% %%%%%%%%%%%%%%%%%%%%%%%%%%% Plot6
% << echo=FALSE, out.width = ".9\\textwidth", fig.width = 12, fig.height = 6, fig.cap = "Out-of-sample predictions along with direct estimates in the cross validation study where all data from each time period is held out and predicted using the rest of the data.">>=
% fig_count <- fig_count + 1
% knitr::include_graphics(paste0("../Main/Figures/CV_byYear_withError_", countryname2, ".pdf")) 
% @
 
%%%%%%%%%%%%%%%%%%%%%%%%%%% Plot7
\begin{knitrout}
\definecolor{shadecolor}{rgb}{0.969, 0.969, 0.969}\color{fgcolor}\begin{figure}[bht]

{\centering \includegraphics[width=.9\textwidth]{../Main/Figures/CV_byYearRegion_withError_Malawi} 

}

\caption[Malawi]{Malawi: Out-of-sample predictions along with direct estimates in the cross validation study where data from one region in each time period is held out and predicted using the rest of the data.}\label{fig:unnamed-chunk-208}
\end{figure}


\end{knitrout}

%%%%%%%%%%%%%%%%%%%%%%%%%%% Plot8
\begin{knitrout}
\definecolor{shadecolor}{rgb}{0.969, 0.969, 0.969}\color{fgcolor}\begin{figure}[bht]

{\centering \includegraphics[width=.9\textwidth]{../Main/Figures/CVbiasMalawi} 

}

\caption[Malawi]{Malawi: Histogram and QQ-plot of the rescaled difference between the smoothed estimates and the direct estimates in the cross validation study. The differences between the two estimates are rescaled by the square root of the total variance of the two estimates.}\label{fig:unnamed-chunk-209}
\end{figure}


\end{knitrout}

%%%%%%%%%%%%%%%%%%%%%%%%%%% Plot9
\begin{knitrout}
\definecolor{shadecolor}{rgb}{0.969, 0.969, 0.969}\color{fgcolor}\begin{figure}[bht]

{\centering \includegraphics[width=.7\textwidth]{../Main/Figures/CVbiasbyRegionMalawi} 

}

\caption[Malawi]{Malawi: Line plot of the difference between smoothed estimates and the direct estimates in the cross validation study. The differences between the two estimates are rescaled by the square root of the total variance of the two estimates.}\label{fig:unnamed-chunk-210}
\end{figure}


\end{knitrout}


%%%%%%%%%%%%%%%%%%%%%%%%%%%%%%%%%%%%%%%%%%%%%%%%%%%%%%%%%%%%%%%%%%%%%%%%%%%%%%%%%%%%%%%%%%%%%%%%%%
\clearpage
\subsubsection{Mali}


% \subsubsection{Summary of DHS surveys}

%%%%%%%%%%%%%%%%%%%%%%%%%%% Summary 


DHS surveys were conducted in Mali in 1987, 1995, 2001, and 2006.
% years.out[1:(length(years.out)-1)], and years.out[length(years.out)]. 

We fit both the RW2 only model to the combined national data, and compare the time trend at national level with the estimates produced by the UN and IHME in Figure~\ref{fig:unnamed-chunk-212}. We then adjusted the combined national data to the UN estimates of U5MR, and refit the models on the benchmarked data. 

%%%%%%%%%%%%%%%%%%%%%%%%%% Plot5 
\begin{knitrout}
\definecolor{shadecolor}{rgb}{0.969, 0.969, 0.969}\color{fgcolor}\begin{figure}[bht]

{\centering \includegraphics[width=.9\textwidth]{../Main/Figures/Yearly_national_Mali} 

}

\caption[Mali]{Mali: Temporal national trends along with UN (B3) estimates described in You et al. (2015) and IHME estimates based on GBD 2015 Child Mortality Collaborators (2016). RW2 represents the smoothed national estimates using the original data before benchmarking with UN estimates. RW2-adj represents the smoothed national estimates using the benchmarked data.}\label{fig:unnamed-chunk-212}
\end{figure}


\end{knitrout}
 

We fit the RW2 model to the benchmarked data in each area. 
% The proportions of the explained variation is summarized in Table~\ref{tab:paste0(countryname, "-var")}. 
We compare the results in Figure~\ref{fig:unnamed-chunk-213} to \ref{fig:unnamed-chunk-217}.
Figure~\ref{fig:unnamed-chunk-213} compares the smoothed estimates against the direct estimates. Figure~\ref{fig:unnamed-chunk-214} and Figure~\ref{fig:unnamed-chunk-215} show the posterior median estimates of U5MR in each region over time and the reductions from 1990 period respectively.
Figure~\ref{fig:unnamed-chunk-216} shows the smoothed estimates by region over time and Figure~\ref{fig:unnamed-chunk-217} compares the smoothed estimates with direct estimates from each survey for each region over time.


% %%%%%%%%%%%%%%%%%%%%%%%%%%% Table1 
% <<echo=FALSE, results='asis'>>=
% load("rda/variance_tables.rda")
% countryname2 <- gsub(" ", "", countryname)
% variance <- tables.all[[countryname]]

% table_count <- table_count + 1

% names <- c("RW2 ($\\sigma^2_{\\gamma_{t}}$)", "ICAR ($\\sigma^2_{\\phi_{i}}$)", "IID space ($\\sigma^2_{\\theta_{i}}$)", "IID time ($\\sigma^2_{\\alpha_{t}}$)", "IID space time ($\\sigma^2_{\\delta_{it}}$)")

% variance$Proportion <- round(variance$Proportion*100, digits = 2)
% row.names(variance) <- names
% tab <- xtable(variance, digits = c(1, 3, 2),align = "l|ll",
%        label = paste0("tab:", countryname, "-var"),
%        caption = paste(country, ": summary of the variance components in the RW2 model", sep = ''))
% print(tab, comment = FALSE,sanitize.text.function = function(x) {x})
% @

%%%%%%%%%%%%%%%%%%%%%%%%%%% Plot1 
\begin{knitrout}
\definecolor{shadecolor}{rgb}{0.969, 0.969, 0.969}\color{fgcolor}\begin{figure}[bht]

{\centering \includegraphics[width=.9\textwidth]{../Main/Figures/SmoothvDirectMali_meta} 

}

\caption[Mali]{Mali: Smooth versus direct Admin 1 estimates. Left: Combined (meta-analysis) survey estimate against combined direct estimates. Right: Combined (meta-analysis) survey estimate against direct estimates from each survey.}\label{fig:unnamed-chunk-213}
\end{figure}


\end{knitrout}

%%%%%%%%%%%%%%%%%%%%%%%%%%% Plot2 
\begin{knitrout}
\definecolor{shadecolor}{rgb}{0.969, 0.969, 0.969}\color{fgcolor}\begin{figure}[bht]

{\centering \includegraphics[width=.9\textwidth]{../Main/Figures/SmoothMedianMali} 

}

\caption[Mali]{Mali: Maps of posterior medians over time.}\label{fig:unnamed-chunk-214}
\end{figure}


\end{knitrout}
%%%%%%%%%%%%%%%%%%%%%%%%%%% Plot2a
\begin{knitrout}
\definecolor{shadecolor}{rgb}{0.969, 0.969, 0.969}\color{fgcolor}\begin{figure}[bht]

{\centering \includegraphics[width=.9\textwidth]{../Main/Figures/ReductionMedianMali} 

}

\caption[Mali]{Mali: Maps of reduction of posterior median U5MR in each five-year period compared to 1990 over time.}\label{fig:unnamed-chunk-215}
\end{figure}


\end{knitrout}
%%%%%%%%%%%%%%%%%%%%%%%%%%% Plot3 
\begin{knitrout}
\definecolor{shadecolor}{rgb}{0.969, 0.969, 0.969}\color{fgcolor}\begin{figure}[bht]

{\centering \includegraphics[width=.95\textwidth]{../Main/Figures/Yearly_v_Periods_Mali} 

}

\caption[Mali]{Mali: Smoothed regional estimates over time. The line indicates yearly posterior median estimates and error bars indicate 95 \% posterior credible interval at each time period.}\label{fig:unnamed-chunk-216}
\end{figure}


\end{knitrout}

%%%%%%%%%%%%%%%%%%%%%%%%%%% Plot4 
\begin{knitrout}
\definecolor{shadecolor}{rgb}{0.969, 0.969, 0.969}\color{fgcolor}\begin{figure}[bht]

{\centering \includegraphics[width=.9\textwidth]{../Main/Figures/LineSubMedianMali} 

}

\caption[Mali]{Mali: Smoothed regional estimates over time compared to the direct estimates from each surveys. Direct estimates are not benchmarked with UN estimates. The line indicates posterior median and error bars indicate 95\% posterior credible interval.}\label{fig:unnamed-chunk-217}
\end{figure}


\end{knitrout}
% \subsubsection{National model results}
We further assess the RW2 model by holding out some observations, and compare the projections to the direct estimates in these holdout observations. Figure~\ref{fig:unnamed-chunk-218} compares the predicted estimates for the out-of-sample observations  with the direct estimates by holding out observations from each area in each time period.  Figure~\ref{fig:unnamed-chunk-219} compares the histogram of the bias rescaled by the total variance in the cross validation studies. Figure~\ref{fig:unnamed-chunk-220} compares the rescaled bias by region and time periods.



% %%%%%%%%%%%%%%%%%%%%%%%%%%% Plot6
% << echo=FALSE, out.width = ".9\\textwidth", fig.width = 12, fig.height = 6, fig.cap = "Out-of-sample predictions along with direct estimates in the cross validation study where all data from each time period is held out and predicted using the rest of the data.">>=
% fig_count <- fig_count + 1
% knitr::include_graphics(paste0("../Main/Figures/CV_byYear_withError_", countryname2, ".pdf")) 
% @
 
%%%%%%%%%%%%%%%%%%%%%%%%%%% Plot7
\begin{knitrout}
\definecolor{shadecolor}{rgb}{0.969, 0.969, 0.969}\color{fgcolor}\begin{figure}[bht]

{\centering \includegraphics[width=.9\textwidth]{../Main/Figures/CV_byYearRegion_withError_Mali} 

}

\caption[Mali]{Mali: Out-of-sample predictions along with direct estimates in the cross validation study where data from one region in each time period is held out and predicted using the rest of the data.}\label{fig:unnamed-chunk-218}
\end{figure}


\end{knitrout}

%%%%%%%%%%%%%%%%%%%%%%%%%%% Plot8
\begin{knitrout}
\definecolor{shadecolor}{rgb}{0.969, 0.969, 0.969}\color{fgcolor}\begin{figure}[bht]

{\centering \includegraphics[width=.9\textwidth]{../Main/Figures/CVbiasMali} 

}

\caption[Mali]{Mali: Histogram and QQ-plot of the rescaled difference between the smoothed estimates and the direct estimates in the cross validation study. The differences between the two estimates are rescaled by the square root of the total variance of the two estimates.}\label{fig:unnamed-chunk-219}
\end{figure}


\end{knitrout}

%%%%%%%%%%%%%%%%%%%%%%%%%%% Plot9
\begin{knitrout}
\definecolor{shadecolor}{rgb}{0.969, 0.969, 0.969}\color{fgcolor}\begin{figure}[bht]

{\centering \includegraphics[width=.7\textwidth]{../Main/Figures/CVbiasbyRegionMali} 

}

\caption[Mali]{Mali: Line plot of the difference between smoothed estimates and the direct estimates in the cross validation study. The differences between the two estimates are rescaled by the square root of the total variance of the two estimates.}\label{fig:unnamed-chunk-220}
\end{figure}


\end{knitrout}


%%%%%%%%%%%%%%%%%%%%%%%%%%%%%%%%%%%%%%%%%%%%%%%%%%%%%%%%%%%%%%%%%%%%%%%%%%%%%%%%%%%%%%%%%%%%%%%%%%
\clearpage
\subsubsection{Morocco}


% \subsubsection{Summary of DHS surveys}

%%%%%%%%%%%%%%%%%%%%%%%%%%% Summary 


DHS surveys were conducted in Morocco in 1987, 1992, and 2003.
% years.out[1:(length(years.out)-1)], and years.out[length(years.out)]. 

We fit both the RW2 only model to the combined national data, and compare the time trend at national level with the estimates produced by the UN and IHME in Figure~\ref{fig:unnamed-chunk-222}. We then adjusted the combined national data to the UN estimates of U5MR, and refit the models on the benchmarked data. 

%%%%%%%%%%%%%%%%%%%%%%%%%% Plot5 
\begin{knitrout}
\definecolor{shadecolor}{rgb}{0.969, 0.969, 0.969}\color{fgcolor}\begin{figure}[bht]

{\centering \includegraphics[width=.9\textwidth]{../Main/Figures/Yearly_national_Morocco} 

}

\caption[Morocco]{Morocco: Temporal national trends along with UN (B3) estimates described in You et al. (2015) and IHME estimates based on GBD 2015 Child Mortality Collaborators (2016). RW2 represents the smoothed national estimates using the original data before benchmarking with UN estimates. RW2-adj represents the smoothed national estimates using the benchmarked data.}\label{fig:unnamed-chunk-222}
\end{figure}


\end{knitrout}
 

We fit the RW2 model to the benchmarked data in each area. 
% The proportions of the explained variation is summarized in Table~\ref{tab:paste0(countryname, "-var")}. 
We compare the results in Figure~\ref{fig:unnamed-chunk-223} to \ref{fig:unnamed-chunk-227}.
Figure~\ref{fig:unnamed-chunk-223} compares the smoothed estimates against the direct estimates. Figure~\ref{fig:unnamed-chunk-224} and Figure~\ref{fig:unnamed-chunk-225} show the posterior median estimates of U5MR in each region over time and the reductions from 1990 period respectively.
Figure~\ref{fig:unnamed-chunk-226} shows the smoothed estimates by region over time and Figure~\ref{fig:unnamed-chunk-227} compares the smoothed estimates with direct estimates from each survey for each region over time.


% %%%%%%%%%%%%%%%%%%%%%%%%%%% Table1 
% <<echo=FALSE, results='asis'>>=
% load("rda/variance_tables.rda")
% countryname2 <- gsub(" ", "", countryname)
% variance <- tables.all[[countryname]]

% table_count <- table_count + 1

% names <- c("RW2 ($\\sigma^2_{\\gamma_{t}}$)", "ICAR ($\\sigma^2_{\\phi_{i}}$)", "IID space ($\\sigma^2_{\\theta_{i}}$)", "IID time ($\\sigma^2_{\\alpha_{t}}$)", "IID space time ($\\sigma^2_{\\delta_{it}}$)")

% variance$Proportion <- round(variance$Proportion*100, digits = 2)
% row.names(variance) <- names
% tab <- xtable(variance, digits = c(1, 3, 2),align = "l|ll",
%        label = paste0("tab:", countryname, "-var"),
%        caption = paste(country, ": summary of the variance components in the RW2 model", sep = ''))
% print(tab, comment = FALSE,sanitize.text.function = function(x) {x})
% @

%%%%%%%%%%%%%%%%%%%%%%%%%%% Plot1 
\begin{knitrout}
\definecolor{shadecolor}{rgb}{0.969, 0.969, 0.969}\color{fgcolor}\begin{figure}[bht]

{\centering \includegraphics[width=.9\textwidth]{../Main/Figures/SmoothvDirectMorocco_meta} 

}

\caption[Morocco]{Morocco: Smooth versus direct Admin 1 estimates. Left: Combined (meta-analysis) survey estimate against combined direct estimates. Right: Combined (meta-analysis) survey estimate against direct estimates from each survey.}\label{fig:unnamed-chunk-223}
\end{figure}


\end{knitrout}

%%%%%%%%%%%%%%%%%%%%%%%%%%% Plot2 
\begin{knitrout}
\definecolor{shadecolor}{rgb}{0.969, 0.969, 0.969}\color{fgcolor}\begin{figure}[bht]

{\centering \includegraphics[width=.9\textwidth]{../Main/Figures/SmoothMedianMorocco} 

}

\caption[Morocco]{Morocco: Maps of posterior medians over time.}\label{fig:unnamed-chunk-224}
\end{figure}


\end{knitrout}
%%%%%%%%%%%%%%%%%%%%%%%%%%% Plot2a
\begin{knitrout}
\definecolor{shadecolor}{rgb}{0.969, 0.969, 0.969}\color{fgcolor}\begin{figure}[bht]

{\centering \includegraphics[width=.9\textwidth]{../Main/Figures/ReductionMedianMorocco} 

}

\caption[Morocco]{Morocco: Maps of reduction of posterior median U5MR in each five-year period compared to 1990 over time.}\label{fig:unnamed-chunk-225}
\end{figure}


\end{knitrout}
%%%%%%%%%%%%%%%%%%%%%%%%%%% Plot3 
\begin{knitrout}
\definecolor{shadecolor}{rgb}{0.969, 0.969, 0.969}\color{fgcolor}\begin{figure}[bht]

{\centering \includegraphics[width=.95\textwidth]{../Main/Figures/Yearly_v_Periods_Morocco} 

}

\caption[Morocco]{Morocco: Smoothed regional estimates over time. The line indicates yearly posterior median estimates and error bars indicate 95 \% posterior credible interval at each time period.}\label{fig:unnamed-chunk-226}
\end{figure}


\end{knitrout}

%%%%%%%%%%%%%%%%%%%%%%%%%%% Plot4 
\begin{knitrout}
\definecolor{shadecolor}{rgb}{0.969, 0.969, 0.969}\color{fgcolor}\begin{figure}[bht]

{\centering \includegraphics[width=.9\textwidth]{../Main/Figures/LineSubMedianMorocco} 

}

\caption[Morocco]{Morocco: Smoothed regional estimates over time compared to the direct estimates from each surveys. Direct estimates are not benchmarked with UN estimates. The line indicates posterior median and error bars indicate 95\% posterior credible interval.}\label{fig:unnamed-chunk-227}
\end{figure}


\end{knitrout}
% \subsubsection{National model results}
We further assess the RW2 model by holding out some observations, and compare the projections to the direct estimates in these holdout observations. Figure~\ref{fig:unnamed-chunk-228} compares the predicted estimates for the out-of-sample observations  with the direct estimates by holding out observations from each area in each time period.  Figure~\ref{fig:unnamed-chunk-229} compares the histogram of the bias rescaled by the total variance in the cross validation studies. Figure~\ref{fig:unnamed-chunk-230} compares the rescaled bias by region and time periods.



% %%%%%%%%%%%%%%%%%%%%%%%%%%% Plot6
% << echo=FALSE, out.width = ".9\\textwidth", fig.width = 12, fig.height = 6, fig.cap = "Out-of-sample predictions along with direct estimates in the cross validation study where all data from each time period is held out and predicted using the rest of the data.">>=
% fig_count <- fig_count + 1
% knitr::include_graphics(paste0("../Main/Figures/CV_byYear_withError_", countryname2, ".pdf")) 
% @
 
%%%%%%%%%%%%%%%%%%%%%%%%%%% Plot7
\begin{knitrout}
\definecolor{shadecolor}{rgb}{0.969, 0.969, 0.969}\color{fgcolor}\begin{figure}[bht]

{\centering \includegraphics[width=.9\textwidth]{../Main/Figures/CV_byYearRegion_withError_Morocco} 

}

\caption[Morocco]{Morocco: Out-of-sample predictions along with direct estimates in the cross validation study where data from one region in each time period is held out and predicted using the rest of the data.}\label{fig:unnamed-chunk-228}
\end{figure}


\end{knitrout}

%%%%%%%%%%%%%%%%%%%%%%%%%%% Plot8
\begin{knitrout}
\definecolor{shadecolor}{rgb}{0.969, 0.969, 0.969}\color{fgcolor}\begin{figure}[bht]

{\centering \includegraphics[width=.9\textwidth]{../Main/Figures/CVbiasMorocco} 

}

\caption[Morocco]{Morocco: Histogram and QQ-plot of the rescaled difference between the smoothed estimates and the direct estimates in the cross validation study. The differences between the two estimates are rescaled by the square root of the total variance of the two estimates.}\label{fig:unnamed-chunk-229}
\end{figure}


\end{knitrout}

%%%%%%%%%%%%%%%%%%%%%%%%%%% Plot9
\begin{knitrout}
\definecolor{shadecolor}{rgb}{0.969, 0.969, 0.969}\color{fgcolor}\begin{figure}[bht]

{\centering \includegraphics[width=.7\textwidth]{../Main/Figures/CVbiasbyRegionMorocco} 

}

\caption[Morocco]{Morocco: Line plot of the difference between smoothed estimates and the direct estimates in the cross validation study. The differences between the two estimates are rescaled by the square root of the total variance of the two estimates.}\label{fig:unnamed-chunk-230}
\end{figure}


\end{knitrout}


%%%%%%%%%%%%%%%%%%%%%%%%%%%%%%%%%%%%%%%%%%%%%%%%%%%%%%%%%%%%%%%%%%%%%%%%%%%%%%%%%%%%%%%%%%%%%%%%%%
\clearpage
\subsubsection{Mozambique}


% \subsubsection{Summary of DHS surveys}

%%%%%%%%%%%%%%%%%%%%%%%%%%% Summary 


DHS surveys were conducted in Mozambique in 2003, and 2011.
% years.out[1:(length(years.out)-1)], and years.out[length(years.out)]. 

We fit both the RW2 only model to the combined national data, and compare the time trend at national level with the estimates produced by the UN and IHME in Figure~\ref{fig:unnamed-chunk-232}. We then adjusted the combined national data to the UN estimates of U5MR, and refit the models on the benchmarked data. 

%%%%%%%%%%%%%%%%%%%%%%%%%% Plot5 
\begin{knitrout}
\definecolor{shadecolor}{rgb}{0.969, 0.969, 0.969}\color{fgcolor}\begin{figure}[bht]

{\centering \includegraphics[width=.9\textwidth]{../Main/Figures/Yearly_national_Mozambique} 

}

\caption[Mozambique]{Mozambique: Temporal national trends along with UN (B3) estimates described in You et al. (2015) and IHME estimates based on GBD 2015 Child Mortality Collaborators (2016). RW2 represents the smoothed national estimates using the original data before benchmarking with UN estimates. RW2-adj represents the smoothed national estimates using the benchmarked data.}\label{fig:unnamed-chunk-232}
\end{figure}


\end{knitrout}
 

We fit the RW2 model to the benchmarked data in each area. 
% The proportions of the explained variation is summarized in Table~\ref{tab:paste0(countryname, "-var")}. 
We compare the results in Figure~\ref{fig:unnamed-chunk-233} to \ref{fig:unnamed-chunk-237}.
Figure~\ref{fig:unnamed-chunk-233} compares the smoothed estimates against the direct estimates. Figure~\ref{fig:unnamed-chunk-234} and Figure~\ref{fig:unnamed-chunk-235} show the posterior median estimates of U5MR in each region over time and the reductions from 1990 period respectively.
Figure~\ref{fig:unnamed-chunk-236} shows the smoothed estimates by region over time and Figure~\ref{fig:unnamed-chunk-237} compares the smoothed estimates with direct estimates from each survey for each region over time.


% %%%%%%%%%%%%%%%%%%%%%%%%%%% Table1 
% <<echo=FALSE, results='asis'>>=
% load("rda/variance_tables.rda")
% countryname2 <- gsub(" ", "", countryname)
% variance <- tables.all[[countryname]]

% table_count <- table_count + 1

% names <- c("RW2 ($\\sigma^2_{\\gamma_{t}}$)", "ICAR ($\\sigma^2_{\\phi_{i}}$)", "IID space ($\\sigma^2_{\\theta_{i}}$)", "IID time ($\\sigma^2_{\\alpha_{t}}$)", "IID space time ($\\sigma^2_{\\delta_{it}}$)")

% variance$Proportion <- round(variance$Proportion*100, digits = 2)
% row.names(variance) <- names
% tab <- xtable(variance, digits = c(1, 3, 2),align = "l|ll",
%        label = paste0("tab:", countryname, "-var"),
%        caption = paste(country, ": summary of the variance components in the RW2 model", sep = ''))
% print(tab, comment = FALSE,sanitize.text.function = function(x) {x})
% @

%%%%%%%%%%%%%%%%%%%%%%%%%%% Plot1 
\begin{knitrout}
\definecolor{shadecolor}{rgb}{0.969, 0.969, 0.969}\color{fgcolor}\begin{figure}[bht]

{\centering \includegraphics[width=.9\textwidth]{../Main/Figures/SmoothvDirectMozambique_meta} 

}

\caption[Mozambique]{Mozambique: Smooth versus direct Admin 1 estimates. Left: Combined (meta-analysis) survey estimate against combined direct estimates. Right: Combined (meta-analysis) survey estimate against direct estimates from each survey.}\label{fig:unnamed-chunk-233}
\end{figure}


\end{knitrout}

%%%%%%%%%%%%%%%%%%%%%%%%%%% Plot2 
\begin{knitrout}
\definecolor{shadecolor}{rgb}{0.969, 0.969, 0.969}\color{fgcolor}\begin{figure}[bht]

{\centering \includegraphics[width=.9\textwidth]{../Main/Figures/SmoothMedianMozambique} 

}

\caption[Mozambique]{Mozambique: Maps of posterior medians over time.}\label{fig:unnamed-chunk-234}
\end{figure}


\end{knitrout}
%%%%%%%%%%%%%%%%%%%%%%%%%%% Plot2a
\begin{knitrout}
\definecolor{shadecolor}{rgb}{0.969, 0.969, 0.969}\color{fgcolor}\begin{figure}[bht]

{\centering \includegraphics[width=.9\textwidth]{../Main/Figures/ReductionMedianMozambique} 

}

\caption[Mozambique]{Mozambique: Maps of reduction of posterior median U5MR in each five-year period compared to 1990 over time.}\label{fig:unnamed-chunk-235}
\end{figure}


\end{knitrout}
%%%%%%%%%%%%%%%%%%%%%%%%%%% Plot3 
\begin{knitrout}
\definecolor{shadecolor}{rgb}{0.969, 0.969, 0.969}\color{fgcolor}\begin{figure}[bht]

{\centering \includegraphics[width=.95\textwidth]{../Main/Figures/Yearly_v_Periods_Mozambique} 

}

\caption[Mozambique]{Mozambique: Smoothed regional estimates over time. The line indicates yearly posterior median estimates and error bars indicate 95 \% posterior credible interval at each time period.}\label{fig:unnamed-chunk-236}
\end{figure}


\end{knitrout}

%%%%%%%%%%%%%%%%%%%%%%%%%%% Plot4 
\begin{knitrout}
\definecolor{shadecolor}{rgb}{0.969, 0.969, 0.969}\color{fgcolor}\begin{figure}[bht]

{\centering \includegraphics[width=.9\textwidth]{../Main/Figures/LineSubMedianMozambique} 

}

\caption[Mozambique]{Mozambique: Smoothed regional estimates over time compared to the direct estimates from each surveys. Direct estimates are not benchmarked with UN estimates. The line indicates posterior median and error bars indicate 95\% posterior credible interval.}\label{fig:unnamed-chunk-237}
\end{figure}


\end{knitrout}
% \subsubsection{National model results}
We further assess the RW2 model by holding out some observations, and compare the projections to the direct estimates in these holdout observations. Figure~\ref{fig:unnamed-chunk-238} compares the predicted estimates for the out-of-sample observations  with the direct estimates by holding out observations from each area in each time period.  Figure~\ref{fig:unnamed-chunk-239} compares the histogram of the bias rescaled by the total variance in the cross validation studies. Figure~\ref{fig:unnamed-chunk-240} compares the rescaled bias by region and time periods.



% %%%%%%%%%%%%%%%%%%%%%%%%%%% Plot6
% << echo=FALSE, out.width = ".9\\textwidth", fig.width = 12, fig.height = 6, fig.cap = "Out-of-sample predictions along with direct estimates in the cross validation study where all data from each time period is held out and predicted using the rest of the data.">>=
% fig_count <- fig_count + 1
% knitr::include_graphics(paste0("../Main/Figures/CV_byYear_withError_", countryname2, ".pdf")) 
% @
 
%%%%%%%%%%%%%%%%%%%%%%%%%%% Plot7
\begin{knitrout}
\definecolor{shadecolor}{rgb}{0.969, 0.969, 0.969}\color{fgcolor}\begin{figure}[bht]

{\centering \includegraphics[width=.9\textwidth]{../Main/Figures/CV_byYearRegion_withError_Mozambique} 

}

\caption[Mozambique]{Mozambique: Out-of-sample predictions along with direct estimates in the cross validation study where data from one region in each time period is held out and predicted using the rest of the data.}\label{fig:unnamed-chunk-238}
\end{figure}


\end{knitrout}

%%%%%%%%%%%%%%%%%%%%%%%%%%% Plot8
\begin{knitrout}
\definecolor{shadecolor}{rgb}{0.969, 0.969, 0.969}\color{fgcolor}\begin{figure}[bht]

{\centering \includegraphics[width=.9\textwidth]{../Main/Figures/CVbiasMozambique} 

}

\caption[Mozambique]{Mozambique: Histogram and QQ-plot of the rescaled difference between the smoothed estimates and the direct estimates in the cross validation study. The differences between the two estimates are rescaled by the square root of the total variance of the two estimates.}\label{fig:unnamed-chunk-239}
\end{figure}


\end{knitrout}

%%%%%%%%%%%%%%%%%%%%%%%%%%% Plot9
\begin{knitrout}
\definecolor{shadecolor}{rgb}{0.969, 0.969, 0.969}\color{fgcolor}\begin{figure}[bht]

{\centering \includegraphics[width=.7\textwidth]{../Main/Figures/CVbiasbyRegionMozambique} 

}

\caption[Mozambique]{Mozambique: Line plot of the difference between smoothed estimates and the direct estimates in the cross validation study. The differences between the two estimates are rescaled by the square root of the total variance of the two estimates.}\label{fig:unnamed-chunk-240}
\end{figure}


\end{knitrout}


%%%%%%%%%%%%%%%%%%%%%%%%%%%%%%%%%%%%%%%%%%%%%%%%%%%%%%%%%%%%%%%%%%%%%%%%%%%%%%%%%%%%%%%%%%%%%%%%%%
\clearpage
\subsubsection{Namibia}


% \subsubsection{Summary of DHS surveys}

%%%%%%%%%%%%%%%%%%%%%%%%%%% Summary 


DHS surveys were conducted in Namibia in 2000, 2007, and 2013.
% years.out[1:(length(years.out)-1)], and years.out[length(years.out)]. 

We fit both the RW2 only model to the combined national data, and compare the time trend at national level with the estimates produced by the UN and IHME in Figure~\ref{fig:unnamed-chunk-242}. We then adjusted the combined national data to the UN estimates of U5MR, and refit the models on the benchmarked data. 

%%%%%%%%%%%%%%%%%%%%%%%%%% Plot5 
\begin{knitrout}
\definecolor{shadecolor}{rgb}{0.969, 0.969, 0.969}\color{fgcolor}\begin{figure}[bht]

{\centering \includegraphics[width=.9\textwidth]{../Main/Figures/Yearly_national_Namibia} 

}

\caption[Namibia]{Namibia: Temporal national trends along with UN (B3) estimates described in You et al. (2015) and IHME estimates based on GBD 2015 Child Mortality Collaborators (2016). RW2 represents the smoothed national estimates using the original data before benchmarking with UN estimates. RW2-adj represents the smoothed national estimates using the benchmarked data.}\label{fig:unnamed-chunk-242}
\end{figure}


\end{knitrout}
 

We fit the RW2 model to the benchmarked data in each area. 
% The proportions of the explained variation is summarized in Table~\ref{tab:paste0(countryname, "-var")}. 
We compare the results in Figure~\ref{fig:unnamed-chunk-243} to \ref{fig:unnamed-chunk-247}.
Figure~\ref{fig:unnamed-chunk-243} compares the smoothed estimates against the direct estimates. Figure~\ref{fig:unnamed-chunk-244} and Figure~\ref{fig:unnamed-chunk-245} show the posterior median estimates of U5MR in each region over time and the reductions from 1990 period respectively.
Figure~\ref{fig:unnamed-chunk-246} shows the smoothed estimates by region over time and Figure~\ref{fig:unnamed-chunk-247} compares the smoothed estimates with direct estimates from each survey for each region over time.


% %%%%%%%%%%%%%%%%%%%%%%%%%%% Table1 
% <<echo=FALSE, results='asis'>>=
% load("rda/variance_tables.rda")
% countryname2 <- gsub(" ", "", countryname)
% variance <- tables.all[[countryname]]

% table_count <- table_count + 1

% names <- c("RW2 ($\\sigma^2_{\\gamma_{t}}$)", "ICAR ($\\sigma^2_{\\phi_{i}}$)", "IID space ($\\sigma^2_{\\theta_{i}}$)", "IID time ($\\sigma^2_{\\alpha_{t}}$)", "IID space time ($\\sigma^2_{\\delta_{it}}$)")

% variance$Proportion <- round(variance$Proportion*100, digits = 2)
% row.names(variance) <- names
% tab <- xtable(variance, digits = c(1, 3, 2),align = "l|ll",
%        label = paste0("tab:", countryname, "-var"),
%        caption = paste(country, ": summary of the variance components in the RW2 model", sep = ''))
% print(tab, comment = FALSE,sanitize.text.function = function(x) {x})
% @

%%%%%%%%%%%%%%%%%%%%%%%%%%% Plot1 
\begin{knitrout}
\definecolor{shadecolor}{rgb}{0.969, 0.969, 0.969}\color{fgcolor}\begin{figure}[bht]

{\centering \includegraphics[width=.9\textwidth]{../Main/Figures/SmoothvDirectNamibia_meta} 

}

\caption[Namibia]{Namibia: Smooth versus direct Admin 1 estimates. Left: Combined (meta-analysis) survey estimate against combined direct estimates. Right: Combined (meta-analysis) survey estimate against direct estimates from each survey.}\label{fig:unnamed-chunk-243}
\end{figure}


\end{knitrout}

%%%%%%%%%%%%%%%%%%%%%%%%%%% Plot2 
\begin{knitrout}
\definecolor{shadecolor}{rgb}{0.969, 0.969, 0.969}\color{fgcolor}\begin{figure}[bht]

{\centering \includegraphics[width=.9\textwidth]{../Main/Figures/SmoothMedianNamibia} 

}

\caption[Namibia]{Namibia: Maps of posterior medians over time.}\label{fig:unnamed-chunk-244}
\end{figure}


\end{knitrout}
%%%%%%%%%%%%%%%%%%%%%%%%%%% Plot2a
\begin{knitrout}
\definecolor{shadecolor}{rgb}{0.969, 0.969, 0.969}\color{fgcolor}\begin{figure}[bht]

{\centering \includegraphics[width=.9\textwidth]{../Main/Figures/ReductionMedianNamibia} 

}

\caption[Namibia]{Namibia: Maps of reduction of posterior median U5MR in each five-year period compared to 1990 over time.}\label{fig:unnamed-chunk-245}
\end{figure}


\end{knitrout}
%%%%%%%%%%%%%%%%%%%%%%%%%%% Plot3 
\begin{knitrout}
\definecolor{shadecolor}{rgb}{0.969, 0.969, 0.969}\color{fgcolor}\begin{figure}[bht]

{\centering \includegraphics[width=.95\textwidth]{../Main/Figures/Yearly_v_Periods_Namibia} 

}

\caption[Namibia]{Namibia: Smoothed regional estimates over time. The line indicates yearly posterior median estimates and error bars indicate 95 \% posterior credible interval at each time period.}\label{fig:unnamed-chunk-246}
\end{figure}


\end{knitrout}

%%%%%%%%%%%%%%%%%%%%%%%%%%% Plot4 
\begin{knitrout}
\definecolor{shadecolor}{rgb}{0.969, 0.969, 0.969}\color{fgcolor}\begin{figure}[bht]

{\centering \includegraphics[width=.9\textwidth]{../Main/Figures/LineSubMedianNamibia} 

}

\caption[Namibia]{Namibia: Smoothed regional estimates over time compared to the direct estimates from each surveys. Direct estimates are not benchmarked with UN estimates. The line indicates posterior median and error bars indicate 95\% posterior credible interval.}\label{fig:unnamed-chunk-247}
\end{figure}


\end{knitrout}
% \subsubsection{National model results}
We further assess the RW2 model by holding out some observations, and compare the projections to the direct estimates in these holdout observations. Figure~\ref{fig:unnamed-chunk-248} compares the predicted estimates for the out-of-sample observations  with the direct estimates by holding out observations from each area in each time period.  Figure~\ref{fig:unnamed-chunk-249} compares the histogram of the bias rescaled by the total variance in the cross validation studies. Figure~\ref{fig:unnamed-chunk-250} compares the rescaled bias by region and time periods.



% %%%%%%%%%%%%%%%%%%%%%%%%%%% Plot6
% << echo=FALSE, out.width = ".9\\textwidth", fig.width = 12, fig.height = 6, fig.cap = "Out-of-sample predictions along with direct estimates in the cross validation study where all data from each time period is held out and predicted using the rest of the data.">>=
% fig_count <- fig_count + 1
% knitr::include_graphics(paste0("../Main/Figures/CV_byYear_withError_", countryname2, ".pdf")) 
% @
 
%%%%%%%%%%%%%%%%%%%%%%%%%%% Plot7
\begin{knitrout}
\definecolor{shadecolor}{rgb}{0.969, 0.969, 0.969}\color{fgcolor}\begin{figure}[bht]

{\centering \includegraphics[width=.9\textwidth]{../Main/Figures/CV_byYearRegion_withError_Namibia} 

}

\caption[Namibia]{Namibia: Out-of-sample predictions along with direct estimates in the cross validation study where data from one region in each time period is held out and predicted using the rest of the data.}\label{fig:unnamed-chunk-248}
\end{figure}


\end{knitrout}

%%%%%%%%%%%%%%%%%%%%%%%%%%% Plot8
\begin{knitrout}
\definecolor{shadecolor}{rgb}{0.969, 0.969, 0.969}\color{fgcolor}\begin{figure}[bht]

{\centering \includegraphics[width=.9\textwidth]{../Main/Figures/CVbiasNamibia} 

}

\caption[Namibia]{Namibia: Histogram and QQ-plot of the rescaled difference between the smoothed estimates and the direct estimates in the cross validation study. The differences between the two estimates are rescaled by the square root of the total variance of the two estimates.}\label{fig:unnamed-chunk-249}
\end{figure}


\end{knitrout}

%%%%%%%%%%%%%%%%%%%%%%%%%%% Plot9
\begin{knitrout}
\definecolor{shadecolor}{rgb}{0.969, 0.969, 0.969}\color{fgcolor}\begin{figure}[bht]

{\centering \includegraphics[width=.7\textwidth]{../Main/Figures/CVbiasbyRegionNamibia} 

}

\caption[Namibia]{Namibia: Line plot of the difference between smoothed estimates and the direct estimates in the cross validation study. The differences between the two estimates are rescaled by the square root of the total variance of the two estimates.}\label{fig:unnamed-chunk-250}
\end{figure}


\end{knitrout}


%%%%%%%%%%%%%%%%%%%%%%%%%%%%%%%%%%%%%%%%%%%%%%%%%%%%%%%%%%%%%%%%%%%%%%%%%%%%%%%%%%%%%%%%%%%%%%%%%%
\clearpage
\subsubsection{Niger}


% \subsubsection{Summary of DHS surveys}

%%%%%%%%%%%%%%%%%%%%%%%%%%% Summary 


DHS surveys were conducted in Niger in 1992, 1998, 2006, and 2012.
% years.out[1:(length(years.out)-1)], and years.out[length(years.out)]. 

We fit both the RW2 only model to the combined national data, and compare the time trend at national level with the estimates produced by the UN and IHME in Figure~\ref{fig:unnamed-chunk-252}. We then adjusted the combined national data to the UN estimates of U5MR, and refit the models on the benchmarked data. 

%%%%%%%%%%%%%%%%%%%%%%%%%% Plot5 
\begin{knitrout}
\definecolor{shadecolor}{rgb}{0.969, 0.969, 0.969}\color{fgcolor}\begin{figure}[bht]

{\centering \includegraphics[width=.9\textwidth]{../Main/Figures/Yearly_national_Niger} 

}

\caption[Niger]{Niger: Temporal national trends along with UN (B3) estimates described in You et al. (2015) and IHME estimates based on GBD 2015 Child Mortality Collaborators (2016). RW2 represents the smoothed national estimates using the original data before benchmarking with UN estimates. RW2-adj represents the smoothed national estimates using the benchmarked data.}\label{fig:unnamed-chunk-252}
\end{figure}


\end{knitrout}
 

We fit the RW2 model to the benchmarked data in each area. 
% The proportions of the explained variation is summarized in Table~\ref{tab:paste0(countryname, "-var")}. 
We compare the results in Figure~\ref{fig:unnamed-chunk-253} to \ref{fig:unnamed-chunk-257}.
Figure~\ref{fig:unnamed-chunk-253} compares the smoothed estimates against the direct estimates. Figure~\ref{fig:unnamed-chunk-254} and Figure~\ref{fig:unnamed-chunk-255} show the posterior median estimates of U5MR in each region over time and the reductions from 1990 period respectively.
Figure~\ref{fig:unnamed-chunk-256} shows the smoothed estimates by region over time and Figure~\ref{fig:unnamed-chunk-257} compares the smoothed estimates with direct estimates from each survey for each region over time.


% %%%%%%%%%%%%%%%%%%%%%%%%%%% Table1 
% <<echo=FALSE, results='asis'>>=
% load("rda/variance_tables.rda")
% countryname2 <- gsub(" ", "", countryname)
% variance <- tables.all[[countryname]]

% table_count <- table_count + 1

% names <- c("RW2 ($\\sigma^2_{\\gamma_{t}}$)", "ICAR ($\\sigma^2_{\\phi_{i}}$)", "IID space ($\\sigma^2_{\\theta_{i}}$)", "IID time ($\\sigma^2_{\\alpha_{t}}$)", "IID space time ($\\sigma^2_{\\delta_{it}}$)")

% variance$Proportion <- round(variance$Proportion*100, digits = 2)
% row.names(variance) <- names
% tab <- xtable(variance, digits = c(1, 3, 2),align = "l|ll",
%        label = paste0("tab:", countryname, "-var"),
%        caption = paste(country, ": summary of the variance components in the RW2 model", sep = ''))
% print(tab, comment = FALSE,sanitize.text.function = function(x) {x})
% @

%%%%%%%%%%%%%%%%%%%%%%%%%%% Plot1 
\begin{knitrout}
\definecolor{shadecolor}{rgb}{0.969, 0.969, 0.969}\color{fgcolor}\begin{figure}[bht]

{\centering \includegraphics[width=.9\textwidth]{../Main/Figures/SmoothvDirectNiger_meta} 

}

\caption[Niger]{Niger: Smooth versus direct Admin 1 estimates. Left: Combined (meta-analysis) survey estimate against combined direct estimates. Right: Combined (meta-analysis) survey estimate against direct estimates from each survey.}\label{fig:unnamed-chunk-253}
\end{figure}


\end{knitrout}

%%%%%%%%%%%%%%%%%%%%%%%%%%% Plot2 
\begin{knitrout}
\definecolor{shadecolor}{rgb}{0.969, 0.969, 0.969}\color{fgcolor}\begin{figure}[bht]

{\centering \includegraphics[width=.9\textwidth]{../Main/Figures/SmoothMedianNiger} 

}

\caption[Niger]{Niger: Maps of posterior medians over time.}\label{fig:unnamed-chunk-254}
\end{figure}


\end{knitrout}
%%%%%%%%%%%%%%%%%%%%%%%%%%% Plot2a
\begin{knitrout}
\definecolor{shadecolor}{rgb}{0.969, 0.969, 0.969}\color{fgcolor}\begin{figure}[bht]

{\centering \includegraphics[width=.9\textwidth]{../Main/Figures/ReductionMedianNiger} 

}

\caption[Niger]{Niger: Maps of reduction of posterior median U5MR in each five-year period compared to 1990 over time.}\label{fig:unnamed-chunk-255}
\end{figure}


\end{knitrout}
%%%%%%%%%%%%%%%%%%%%%%%%%%% Plot3 
\begin{knitrout}
\definecolor{shadecolor}{rgb}{0.969, 0.969, 0.969}\color{fgcolor}\begin{figure}[bht]

{\centering \includegraphics[width=.95\textwidth]{../Main/Figures/Yearly_v_Periods_Niger} 

}

\caption[Niger]{Niger: Smoothed regional estimates over time. The line indicates yearly posterior median estimates and error bars indicate 95 \% posterior credible interval at each time period.}\label{fig:unnamed-chunk-256}
\end{figure}


\end{knitrout}

%%%%%%%%%%%%%%%%%%%%%%%%%%% Plot4 
\begin{knitrout}
\definecolor{shadecolor}{rgb}{0.969, 0.969, 0.969}\color{fgcolor}\begin{figure}[bht]

{\centering \includegraphics[width=.9\textwidth]{../Main/Figures/LineSubMedianNiger} 

}

\caption[Niger]{Niger: Smoothed regional estimates over time compared to the direct estimates from each surveys. Direct estimates are not benchmarked with UN estimates. The line indicates posterior median and error bars indicate 95\% posterior credible interval.}\label{fig:unnamed-chunk-257}
\end{figure}


\end{knitrout}
% \subsubsection{National model results}
We further assess the RW2 model by holding out some observations, and compare the projections to the direct estimates in these holdout observations. Figure~\ref{fig:unnamed-chunk-258} compares the predicted estimates for the out-of-sample observations  with the direct estimates by holding out observations from each area in each time period.  Figure~\ref{fig:unnamed-chunk-259} compares the histogram of the bias rescaled by the total variance in the cross validation studies. Figure~\ref{fig:unnamed-chunk-260} compares the rescaled bias by region and time periods.



% %%%%%%%%%%%%%%%%%%%%%%%%%%% Plot6
% << echo=FALSE, out.width = ".9\\textwidth", fig.width = 12, fig.height = 6, fig.cap = "Out-of-sample predictions along with direct estimates in the cross validation study where all data from each time period is held out and predicted using the rest of the data.">>=
% fig_count <- fig_count + 1
% knitr::include_graphics(paste0("../Main/Figures/CV_byYear_withError_", countryname2, ".pdf")) 
% @
 
%%%%%%%%%%%%%%%%%%%%%%%%%%% Plot7
\begin{knitrout}
\definecolor{shadecolor}{rgb}{0.969, 0.969, 0.969}\color{fgcolor}\begin{figure}[bht]

{\centering \includegraphics[width=.9\textwidth]{../Main/Figures/CV_byYearRegion_withError_Niger} 

}

\caption[Niger]{Niger: Out-of-sample predictions along with direct estimates in the cross validation study where data from one region in each time period is held out and predicted using the rest of the data.}\label{fig:unnamed-chunk-258}
\end{figure}


\end{knitrout}

%%%%%%%%%%%%%%%%%%%%%%%%%%% Plot8
\begin{knitrout}
\definecolor{shadecolor}{rgb}{0.969, 0.969, 0.969}\color{fgcolor}\begin{figure}[bht]

{\centering \includegraphics[width=.9\textwidth]{../Main/Figures/CVbiasNiger} 

}

\caption[Niger]{Niger: Histogram and QQ-plot of the rescaled difference between the smoothed estimates and the direct estimates in the cross validation study. The differences between the two estimates are rescaled by the square root of the total variance of the two estimates.}\label{fig:unnamed-chunk-259}
\end{figure}


\end{knitrout}

%%%%%%%%%%%%%%%%%%%%%%%%%%% Plot9
\begin{knitrout}
\definecolor{shadecolor}{rgb}{0.969, 0.969, 0.969}\color{fgcolor}\begin{figure}[bht]

{\centering \includegraphics[width=.7\textwidth]{../Main/Figures/CVbiasbyRegionNiger} 

}

\caption[Niger]{Niger: Line plot of the difference between smoothed estimates and the direct estimates in the cross validation study. The differences between the two estimates are rescaled by the square root of the total variance of the two estimates.}\label{fig:unnamed-chunk-260}
\end{figure}


\end{knitrout}


%%%%%%%%%%%%%%%%%%%%%%%%%%%%%%%%%%%%%%%%%%%%%%%%%%%%%%%%%%%%%%%%%%%%%%%%%%%%%%%%%%%%%%%%%%%%%%%%%%
\clearpage
\subsubsection{Nigeria}


% \subsubsection{Summary of DHS surveys}

%%%%%%%%%%%%%%%%%%%%%%%%%%% Summary 


DHS surveys were conducted in Nigeria in 1990, 2003, 2008, and 2013.
% years.out[1:(length(years.out)-1)], and years.out[length(years.out)]. 

We fit both the RW2 only model to the combined national data, and compare the time trend at national level with the estimates produced by the UN and IHME in Figure~\ref{fig:unnamed-chunk-262}. We then adjusted the combined national data to the UN estimates of U5MR, and refit the models on the benchmarked data. 

%%%%%%%%%%%%%%%%%%%%%%%%%% Plot5 
\begin{knitrout}
\definecolor{shadecolor}{rgb}{0.969, 0.969, 0.969}\color{fgcolor}\begin{figure}[bht]

{\centering \includegraphics[width=.9\textwidth]{../Main/Figures/Yearly_national_Nigeria} 

}

\caption[Nigeria]{Nigeria: Temporal national trends along with UN (B3) estimates described in You et al. (2015) and IHME estimates based on GBD 2015 Child Mortality Collaborators (2016). RW2 represents the smoothed national estimates using the original data before benchmarking with UN estimates. RW2-adj represents the smoothed national estimates using the benchmarked data.}\label{fig:unnamed-chunk-262}
\end{figure}


\end{knitrout}
 

We fit the RW2 model to the benchmarked data in each area. 
% The proportions of the explained variation is summarized in Table~\ref{tab:paste0(countryname, "-var")}. 
We compare the results in Figure~\ref{fig:unnamed-chunk-263} to \ref{fig:unnamed-chunk-267}.
Figure~\ref{fig:unnamed-chunk-263} compares the smoothed estimates against the direct estimates. Figure~\ref{fig:unnamed-chunk-264} and Figure~\ref{fig:unnamed-chunk-265} show the posterior median estimates of U5MR in each region over time and the reductions from 1990 period respectively.
Figure~\ref{fig:unnamed-chunk-266} shows the smoothed estimates by region over time and Figure~\ref{fig:unnamed-chunk-267} compares the smoothed estimates with direct estimates from each survey for each region over time.


% %%%%%%%%%%%%%%%%%%%%%%%%%%% Table1 
% <<echo=FALSE, results='asis'>>=
% load("rda/variance_tables.rda")
% countryname2 <- gsub(" ", "", countryname)
% variance <- tables.all[[countryname]]

% table_count <- table_count + 1

% names <- c("RW2 ($\\sigma^2_{\\gamma_{t}}$)", "ICAR ($\\sigma^2_{\\phi_{i}}$)", "IID space ($\\sigma^2_{\\theta_{i}}$)", "IID time ($\\sigma^2_{\\alpha_{t}}$)", "IID space time ($\\sigma^2_{\\delta_{it}}$)")

% variance$Proportion <- round(variance$Proportion*100, digits = 2)
% row.names(variance) <- names
% tab <- xtable(variance, digits = c(1, 3, 2),align = "l|ll",
%        label = paste0("tab:", countryname, "-var"),
%        caption = paste(country, ": summary of the variance components in the RW2 model", sep = ''))
% print(tab, comment = FALSE,sanitize.text.function = function(x) {x})
% @

%%%%%%%%%%%%%%%%%%%%%%%%%%% Plot1 
\begin{knitrout}
\definecolor{shadecolor}{rgb}{0.969, 0.969, 0.969}\color{fgcolor}\begin{figure}[bht]

{\centering \includegraphics[width=.9\textwidth]{../Main/Figures/SmoothvDirectNigeria_meta} 

}

\caption[Nigeria]{Nigeria: Smooth versus direct Admin 1 estimates. Left: Combined (meta-analysis) survey estimate against combined direct estimates. Right: Combined (meta-analysis) survey estimate against direct estimates from each survey.}\label{fig:unnamed-chunk-263}
\end{figure}


\end{knitrout}

%%%%%%%%%%%%%%%%%%%%%%%%%%% Plot2 
\begin{knitrout}
\definecolor{shadecolor}{rgb}{0.969, 0.969, 0.969}\color{fgcolor}\begin{figure}[bht]

{\centering \includegraphics[width=.9\textwidth]{../Main/Figures/SmoothMedianNigeria} 

}

\caption[Nigeria]{Nigeria: Maps of posterior medians over time.}\label{fig:unnamed-chunk-264}
\end{figure}


\end{knitrout}
%%%%%%%%%%%%%%%%%%%%%%%%%%% Plot2a
\begin{knitrout}
\definecolor{shadecolor}{rgb}{0.969, 0.969, 0.969}\color{fgcolor}\begin{figure}[bht]

{\centering \includegraphics[width=.9\textwidth]{../Main/Figures/ReductionMedianNigeria} 

}

\caption[Nigeria]{Nigeria: Maps of reduction of posterior median U5MR in each five-year period compared to 1990 over time.}\label{fig:unnamed-chunk-265}
\end{figure}


\end{knitrout}
%%%%%%%%%%%%%%%%%%%%%%%%%%% Plot3 
\begin{knitrout}
\definecolor{shadecolor}{rgb}{0.969, 0.969, 0.969}\color{fgcolor}\begin{figure}[bht]

{\centering \includegraphics[width=.95\textwidth]{../Main/Figures/Yearly_v_Periods_Nigeria} 

}

\caption[Nigeria]{Nigeria: Smoothed regional estimates over time. The line indicates yearly posterior median estimates and error bars indicate 95 \% posterior credible interval at each time period.}\label{fig:unnamed-chunk-266}
\end{figure}


\end{knitrout}

%%%%%%%%%%%%%%%%%%%%%%%%%%% Plot4 
\begin{knitrout}
\definecolor{shadecolor}{rgb}{0.969, 0.969, 0.969}\color{fgcolor}\begin{figure}[bht]

{\centering \includegraphics[width=.9\textwidth]{../Main/Figures/LineSubMedianNigeria} 

}

\caption[Nigeria]{Nigeria: Smoothed regional estimates over time compared to the direct estimates from each surveys. Direct estimates are not benchmarked with UN estimates. The line indicates posterior median and error bars indicate 95\% posterior credible interval.}\label{fig:unnamed-chunk-267}
\end{figure}


\end{knitrout}
% \subsubsection{National model results}
We further assess the RW2 model by holding out some observations, and compare the projections to the direct estimates in these holdout observations. Figure~\ref{fig:unnamed-chunk-268} compares the predicted estimates for the out-of-sample observations  with the direct estimates by holding out observations from each area in each time period.  Figure~\ref{fig:unnamed-chunk-269} compares the histogram of the bias rescaled by the total variance in the cross validation studies. Figure~\ref{fig:unnamed-chunk-270} compares the rescaled bias by region and time periods.



% %%%%%%%%%%%%%%%%%%%%%%%%%%% Plot6
% << echo=FALSE, out.width = ".9\\textwidth", fig.width = 12, fig.height = 6, fig.cap = "Out-of-sample predictions along with direct estimates in the cross validation study where all data from each time period is held out and predicted using the rest of the data.">>=
% fig_count <- fig_count + 1
% knitr::include_graphics(paste0("../Main/Figures/CV_byYear_withError_", countryname2, ".pdf")) 
% @
 
%%%%%%%%%%%%%%%%%%%%%%%%%%% Plot7
\begin{knitrout}
\definecolor{shadecolor}{rgb}{0.969, 0.969, 0.969}\color{fgcolor}\begin{figure}[bht]

{\centering \includegraphics[width=.9\textwidth]{../Main/Figures/CV_byYearRegion_withError_Nigeria} 

}

\caption[Nigeria]{Nigeria: Out-of-sample predictions along with direct estimates in the cross validation study where data from one region in each time period is held out and predicted using the rest of the data.}\label{fig:unnamed-chunk-268}
\end{figure}


\end{knitrout}

%%%%%%%%%%%%%%%%%%%%%%%%%%% Plot8
\begin{knitrout}
\definecolor{shadecolor}{rgb}{0.969, 0.969, 0.969}\color{fgcolor}\begin{figure}[bht]

{\centering \includegraphics[width=.9\textwidth]{../Main/Figures/CVbiasNigeria} 

}

\caption[Nigeria]{Nigeria: Histogram and QQ-plot of the rescaled difference between the smoothed estimates and the direct estimates in the cross validation study. The differences between the two estimates are rescaled by the square root of the total variance of the two estimates.}\label{fig:unnamed-chunk-269}
\end{figure}


\end{knitrout}

%%%%%%%%%%%%%%%%%%%%%%%%%%% Plot9
\begin{knitrout}
\definecolor{shadecolor}{rgb}{0.969, 0.969, 0.969}\color{fgcolor}\begin{figure}[bht]

{\centering \includegraphics[width=.7\textwidth]{../Main/Figures/CVbiasbyRegionNigeria} 

}

\caption[Nigeria]{Nigeria: Line plot of the difference between smoothed estimates and the direct estimates in the cross validation study. The differences between the two estimates are rescaled by the square root of the total variance of the two estimates.}\label{fig:unnamed-chunk-270}
\end{figure}


\end{knitrout}


%%%%%%%%%%%%%%%%%%%%%%%%%%%%%%%%%%%%%%%%%%%%%%%%%%%%%%%%%%%%%%%%%%%%%%%%%%%%%%%%%%%%%%%%%%%%%%%%%%
\clearpage
\subsubsection{Rwanda}


% \subsubsection{Summary of DHS surveys}

%%%%%%%%%%%%%%%%%%%%%%%%%%% Summary 


DHS surveys were conducted in Rwanda in 2000, 2005, 2008, 2010, and 2015.
% years.out[1:(length(years.out)-1)], and years.out[length(years.out)]. 

We fit both the RW2 only model to the combined national data, and compare the time trend at national level with the estimates produced by the UN and IHME in Figure~\ref{fig:unnamed-chunk-272}. We then adjusted the combined national data to the UN estimates of U5MR, and refit the models on the benchmarked data. 

%%%%%%%%%%%%%%%%%%%%%%%%%% Plot5 
\begin{knitrout}
\definecolor{shadecolor}{rgb}{0.969, 0.969, 0.969}\color{fgcolor}\begin{figure}[bht]

{\centering \includegraphics[width=.9\textwidth]{../Main/Figures/Yearly_national_Rwanda} 

}

\caption[Rwanda]{Rwanda: Temporal national trends along with UN (B3) estimates described in You et al. (2015) and IHME estimates based on GBD 2015 Child Mortality Collaborators (2016). RW2 represents the smoothed national estimates using the original data before benchmarking with UN estimates. RW2-adj represents the smoothed national estimates using the benchmarked data.}\label{fig:unnamed-chunk-272}
\end{figure}


\end{knitrout}
 

We fit the RW2 model to the benchmarked data in each area. 
% The proportions of the explained variation is summarized in Table~\ref{tab:paste0(countryname, "-var")}. 
We compare the results in Figure~\ref{fig:unnamed-chunk-273} to \ref{fig:unnamed-chunk-277}.
Figure~\ref{fig:unnamed-chunk-273} compares the smoothed estimates against the direct estimates. Figure~\ref{fig:unnamed-chunk-274} and Figure~\ref{fig:unnamed-chunk-275} show the posterior median estimates of U5MR in each region over time and the reductions from 1990 period respectively.
Figure~\ref{fig:unnamed-chunk-276} shows the smoothed estimates by region over time and Figure~\ref{fig:unnamed-chunk-277} compares the smoothed estimates with direct estimates from each survey for each region over time.


% %%%%%%%%%%%%%%%%%%%%%%%%%%% Table1 
% <<echo=FALSE, results='asis'>>=
% load("rda/variance_tables.rda")
% countryname2 <- gsub(" ", "", countryname)
% variance <- tables.all[[countryname]]

% table_count <- table_count + 1

% names <- c("RW2 ($\\sigma^2_{\\gamma_{t}}$)", "ICAR ($\\sigma^2_{\\phi_{i}}$)", "IID space ($\\sigma^2_{\\theta_{i}}$)", "IID time ($\\sigma^2_{\\alpha_{t}}$)", "IID space time ($\\sigma^2_{\\delta_{it}}$)")

% variance$Proportion <- round(variance$Proportion*100, digits = 2)
% row.names(variance) <- names
% tab <- xtable(variance, digits = c(1, 3, 2),align = "l|ll",
%        label = paste0("tab:", countryname, "-var"),
%        caption = paste(country, ": summary of the variance components in the RW2 model", sep = ''))
% print(tab, comment = FALSE,sanitize.text.function = function(x) {x})
% @

%%%%%%%%%%%%%%%%%%%%%%%%%%% Plot1 
\begin{knitrout}
\definecolor{shadecolor}{rgb}{0.969, 0.969, 0.969}\color{fgcolor}\begin{figure}[bht]

{\centering \includegraphics[width=.9\textwidth]{../Main/Figures/SmoothvDirectRwanda_meta} 

}

\caption[Rwanda]{Rwanda: Smooth versus direct Admin 1 estimates. Left: Combined (meta-analysis) survey estimate against combined direct estimates. Right: Combined (meta-analysis) survey estimate against direct estimates from each survey.}\label{fig:unnamed-chunk-273}
\end{figure}


\end{knitrout}

%%%%%%%%%%%%%%%%%%%%%%%%%%% Plot2 
\begin{knitrout}
\definecolor{shadecolor}{rgb}{0.969, 0.969, 0.969}\color{fgcolor}\begin{figure}[bht]

{\centering \includegraphics[width=.9\textwidth]{../Main/Figures/SmoothMedianRwanda} 

}

\caption[Rwanda]{Rwanda: Maps of posterior medians over time.}\label{fig:unnamed-chunk-274}
\end{figure}


\end{knitrout}
%%%%%%%%%%%%%%%%%%%%%%%%%%% Plot2a
\begin{knitrout}
\definecolor{shadecolor}{rgb}{0.969, 0.969, 0.969}\color{fgcolor}\begin{figure}[bht]

{\centering \includegraphics[width=.9\textwidth]{../Main/Figures/ReductionMedianRwanda} 

}

\caption[Rwanda]{Rwanda: Maps of reduction of posterior median U5MR in each five-year period compared to 1990 over time.}\label{fig:unnamed-chunk-275}
\end{figure}


\end{knitrout}
%%%%%%%%%%%%%%%%%%%%%%%%%%% Plot3 
\begin{knitrout}
\definecolor{shadecolor}{rgb}{0.969, 0.969, 0.969}\color{fgcolor}\begin{figure}[bht]

{\centering \includegraphics[width=.95\textwidth]{../Main/Figures/Yearly_v_Periods_Rwanda} 

}

\caption[Rwanda]{Rwanda: Smoothed regional estimates over time. The line indicates yearly posterior median estimates and error bars indicate 95 \% posterior credible interval at each time period.}\label{fig:unnamed-chunk-276}
\end{figure}


\end{knitrout}

%%%%%%%%%%%%%%%%%%%%%%%%%%% Plot4 
\begin{knitrout}
\definecolor{shadecolor}{rgb}{0.969, 0.969, 0.969}\color{fgcolor}\begin{figure}[bht]

{\centering \includegraphics[width=.9\textwidth]{../Main/Figures/LineSubMedianRwanda} 

}

\caption[Rwanda]{Rwanda: Smoothed regional estimates over time compared to the direct estimates from each surveys. Direct estimates are not benchmarked with UN estimates. The line indicates posterior median and error bars indicate 95\% posterior credible interval.}\label{fig:unnamed-chunk-277}
\end{figure}


\end{knitrout}
% \subsubsection{National model results}
We further assess the RW2 model by holding out some observations, and compare the projections to the direct estimates in these holdout observations. Figure~\ref{fig:unnamed-chunk-278} compares the predicted estimates for the out-of-sample observations  with the direct estimates by holding out observations from each area in each time period.  Figure~\ref{fig:unnamed-chunk-279} compares the histogram of the bias rescaled by the total variance in the cross validation studies. Figure~\ref{fig:unnamed-chunk-280} compares the rescaled bias by region and time periods.



% %%%%%%%%%%%%%%%%%%%%%%%%%%% Plot6
% << echo=FALSE, out.width = ".9\\textwidth", fig.width = 12, fig.height = 6, fig.cap = "Out-of-sample predictions along with direct estimates in the cross validation study where all data from each time period is held out and predicted using the rest of the data.">>=
% fig_count <- fig_count + 1
% knitr::include_graphics(paste0("../Main/Figures/CV_byYear_withError_", countryname2, ".pdf")) 
% @
 
%%%%%%%%%%%%%%%%%%%%%%%%%%% Plot7
\begin{knitrout}
\definecolor{shadecolor}{rgb}{0.969, 0.969, 0.969}\color{fgcolor}\begin{figure}[bht]

{\centering \includegraphics[width=.9\textwidth]{../Main/Figures/CV_byYearRegion_withError_Rwanda} 

}

\caption[Rwanda]{Rwanda: Out-of-sample predictions along with direct estimates in the cross validation study where data from one region in each time period is held out and predicted using the rest of the data.}\label{fig:unnamed-chunk-278}
\end{figure}


\end{knitrout}

%%%%%%%%%%%%%%%%%%%%%%%%%%% Plot8
\begin{knitrout}
\definecolor{shadecolor}{rgb}{0.969, 0.969, 0.969}\color{fgcolor}\begin{figure}[bht]

{\centering \includegraphics[width=.9\textwidth]{../Main/Figures/CVbiasRwanda} 

}

\caption[Rwanda]{Rwanda: Histogram and QQ-plot of the rescaled difference between the smoothed estimates and the direct estimates in the cross validation study. The differences between the two estimates are rescaled by the square root of the total variance of the two estimates.}\label{fig:unnamed-chunk-279}
\end{figure}


\end{knitrout}

%%%%%%%%%%%%%%%%%%%%%%%%%%% Plot9
\begin{knitrout}
\definecolor{shadecolor}{rgb}{0.969, 0.969, 0.969}\color{fgcolor}\begin{figure}[bht]

{\centering \includegraphics[width=.7\textwidth]{../Main/Figures/CVbiasbyRegionRwanda} 

}

\caption[Rwanda]{Rwanda: Line plot of the difference between smoothed estimates and the direct estimates in the cross validation study. The differences between the two estimates are rescaled by the square root of the total variance of the two estimates.}\label{fig:unnamed-chunk-280}
\end{figure}


\end{knitrout}


%%%%%%%%%%%%%%%%%%%%%%%%%%%%%%%%%%%%%%%%%%%%%%%%%%%%%%%%%%%%%%%%%%%%%%%%%%%%%%%%%%%%%%%%%%%%%%%%%%
\clearpage
\subsubsection{Senegal}


% \subsubsection{Summary of DHS surveys}

%%%%%%%%%%%%%%%%%%%%%%%%%%% Summary 


DHS surveys were conducted in Senegal in 1992, 1997, 2005, 2010, 2012, 2014, 2015, and 2016.
% years.out[1:(length(years.out)-1)], and years.out[length(years.out)]. 

We fit both the RW2 only model to the combined national data, and compare the time trend at national level with the estimates produced by the UN and IHME in Figure~\ref{fig:unnamed-chunk-282}. We then adjusted the combined national data to the UN estimates of U5MR, and refit the models on the benchmarked data. 

%%%%%%%%%%%%%%%%%%%%%%%%%% Plot5 
\begin{knitrout}
\definecolor{shadecolor}{rgb}{0.969, 0.969, 0.969}\color{fgcolor}\begin{figure}[bht]

{\centering \includegraphics[width=.9\textwidth]{../Main/Figures/Yearly_national_Senegal} 

}

\caption[Senegal]{Senegal: Temporal national trends along with UN (B3) estimates described in You et al. (2015) and IHME estimates based on GBD 2015 Child Mortality Collaborators (2016). RW2 represents the smoothed national estimates using the original data before benchmarking with UN estimates. RW2-adj represents the smoothed national estimates using the benchmarked data.}\label{fig:unnamed-chunk-282}
\end{figure}


\end{knitrout}
 

We fit the RW2 model to the benchmarked data in each area. 
% The proportions of the explained variation is summarized in Table~\ref{tab:paste0(countryname, "-var")}. 
We compare the results in Figure~\ref{fig:unnamed-chunk-283} to \ref{fig:unnamed-chunk-287}.
Figure~\ref{fig:unnamed-chunk-283} compares the smoothed estimates against the direct estimates. Figure~\ref{fig:unnamed-chunk-284} and Figure~\ref{fig:unnamed-chunk-285} show the posterior median estimates of U5MR in each region over time and the reductions from 1990 period respectively.
Figure~\ref{fig:unnamed-chunk-286} shows the smoothed estimates by region over time and Figure~\ref{fig:unnamed-chunk-287} compares the smoothed estimates with direct estimates from each survey for each region over time.


% %%%%%%%%%%%%%%%%%%%%%%%%%%% Table1 
% <<echo=FALSE, results='asis'>>=
% load("rda/variance_tables.rda")
% countryname2 <- gsub(" ", "", countryname)
% variance <- tables.all[[countryname]]

% table_count <- table_count + 1

% names <- c("RW2 ($\\sigma^2_{\\gamma_{t}}$)", "ICAR ($\\sigma^2_{\\phi_{i}}$)", "IID space ($\\sigma^2_{\\theta_{i}}$)", "IID time ($\\sigma^2_{\\alpha_{t}}$)", "IID space time ($\\sigma^2_{\\delta_{it}}$)")

% variance$Proportion <- round(variance$Proportion*100, digits = 2)
% row.names(variance) <- names
% tab <- xtable(variance, digits = c(1, 3, 2),align = "l|ll",
%        label = paste0("tab:", countryname, "-var"),
%        caption = paste(country, ": summary of the variance components in the RW2 model", sep = ''))
% print(tab, comment = FALSE,sanitize.text.function = function(x) {x})
% @

%%%%%%%%%%%%%%%%%%%%%%%%%%% Plot1 
\begin{knitrout}
\definecolor{shadecolor}{rgb}{0.969, 0.969, 0.969}\color{fgcolor}\begin{figure}[bht]

{\centering \includegraphics[width=.9\textwidth]{../Main/Figures/SmoothvDirectSenegal_meta} 

}

\caption[Senegal]{Senegal: Smooth versus direct Admin 1 estimates. Left: Combined (meta-analysis) survey estimate against combined direct estimates. Right: Combined (meta-analysis) survey estimate against direct estimates from each survey.}\label{fig:unnamed-chunk-283}
\end{figure}


\end{knitrout}

%%%%%%%%%%%%%%%%%%%%%%%%%%% Plot2 
\begin{knitrout}
\definecolor{shadecolor}{rgb}{0.969, 0.969, 0.969}\color{fgcolor}\begin{figure}[bht]

{\centering \includegraphics[width=.9\textwidth]{../Main/Figures/SmoothMedianSenegal} 

}

\caption[Senegal]{Senegal: Maps of posterior medians over time.}\label{fig:unnamed-chunk-284}
\end{figure}


\end{knitrout}
%%%%%%%%%%%%%%%%%%%%%%%%%%% Plot2a
\begin{knitrout}
\definecolor{shadecolor}{rgb}{0.969, 0.969, 0.969}\color{fgcolor}\begin{figure}[bht]

{\centering \includegraphics[width=.9\textwidth]{../Main/Figures/ReductionMedianSenegal} 

}

\caption[Senegal]{Senegal: Maps of reduction of posterior median U5MR in each five-year period compared to 1990 over time.}\label{fig:unnamed-chunk-285}
\end{figure}


\end{knitrout}
%%%%%%%%%%%%%%%%%%%%%%%%%%% Plot3 
\begin{knitrout}
\definecolor{shadecolor}{rgb}{0.969, 0.969, 0.969}\color{fgcolor}\begin{figure}[bht]

{\centering \includegraphics[width=.95\textwidth]{../Main/Figures/Yearly_v_Periods_Senegal} 

}

\caption[Senegal]{Senegal: Smoothed regional estimates over time. The line indicates yearly posterior median estimates and error bars indicate 95 \% posterior credible interval at each time period.}\label{fig:unnamed-chunk-286}
\end{figure}


\end{knitrout}

%%%%%%%%%%%%%%%%%%%%%%%%%%% Plot4 
\begin{knitrout}
\definecolor{shadecolor}{rgb}{0.969, 0.969, 0.969}\color{fgcolor}\begin{figure}[bht]

{\centering \includegraphics[width=.9\textwidth]{../Main/Figures/LineSubMedianSenegal} 

}

\caption[Senegal]{Senegal: Smoothed regional estimates over time compared to the direct estimates from each surveys. Direct estimates are not benchmarked with UN estimates. The line indicates posterior median and error bars indicate 95\% posterior credible interval.}\label{fig:unnamed-chunk-287}
\end{figure}


\end{knitrout}
% \subsubsection{National model results}
We further assess the RW2 model by holding out some observations, and compare the projections to the direct estimates in these holdout observations. Figure~\ref{fig:unnamed-chunk-288} compares the predicted estimates for the out-of-sample observations  with the direct estimates by holding out observations from each area in each time period.  Figure~\ref{fig:unnamed-chunk-289} compares the histogram of the bias rescaled by the total variance in the cross validation studies. Figure~\ref{fig:unnamed-chunk-290} compares the rescaled bias by region and time periods.



% %%%%%%%%%%%%%%%%%%%%%%%%%%% Plot6
% << echo=FALSE, out.width = ".9\\textwidth", fig.width = 12, fig.height = 6, fig.cap = "Out-of-sample predictions along with direct estimates in the cross validation study where all data from each time period is held out and predicted using the rest of the data.">>=
% fig_count <- fig_count + 1
% knitr::include_graphics(paste0("../Main/Figures/CV_byYear_withError_", countryname2, ".pdf")) 
% @
 
%%%%%%%%%%%%%%%%%%%%%%%%%%% Plot7
\begin{knitrout}
\definecolor{shadecolor}{rgb}{0.969, 0.969, 0.969}\color{fgcolor}\begin{figure}[bht]

{\centering \includegraphics[width=.9\textwidth]{../Main/Figures/CV_byYearRegion_withError_Senegal} 

}

\caption[Senegal]{Senegal: Out-of-sample predictions along with direct estimates in the cross validation study where data from one region in each time period is held out and predicted using the rest of the data.}\label{fig:unnamed-chunk-288}
\end{figure}


\end{knitrout}

%%%%%%%%%%%%%%%%%%%%%%%%%%% Plot8
\begin{knitrout}
\definecolor{shadecolor}{rgb}{0.969, 0.969, 0.969}\color{fgcolor}\begin{figure}[bht]

{\centering \includegraphics[width=.9\textwidth]{../Main/Figures/CVbiasSenegal} 

}

\caption[Senegal]{Senegal: Histogram and QQ-plot of the rescaled difference between the smoothed estimates and the direct estimates in the cross validation study. The differences between the two estimates are rescaled by the square root of the total variance of the two estimates.}\label{fig:unnamed-chunk-289}
\end{figure}


\end{knitrout}

%%%%%%%%%%%%%%%%%%%%%%%%%%% Plot9
\begin{knitrout}
\definecolor{shadecolor}{rgb}{0.969, 0.969, 0.969}\color{fgcolor}\begin{figure}[bht]

{\centering \includegraphics[width=.7\textwidth]{../Main/Figures/CVbiasbyRegionSenegal} 

}

\caption[Senegal]{Senegal: Line plot of the difference between smoothed estimates and the direct estimates in the cross validation study. The differences between the two estimates are rescaled by the square root of the total variance of the two estimates.}\label{fig:unnamed-chunk-290}
\end{figure}


\end{knitrout}


%%%%%%%%%%%%%%%%%%%%%%%%%%%%%%%%%%%%%%%%%%%%%%%%%%%%%%%%%%%%%%%%%%%%%%%%%%%%%%%%%%%%%%%%%%%%%%%%%%
\clearpage
\subsubsection{Sierra Leone}


% \subsubsection{Summary of DHS surveys}

%%%%%%%%%%%%%%%%%%%%%%%%%%% Summary 


DHS surveys were conducted in Sierra Leone in 2013.
% years.out[1:(length(years.out)-1)], and years.out[length(years.out)]. 

We fit both the RW2 only model to the combined national data, and compare the time trend at national level with the estimates produced by the UN and IHME in Figure~\ref{fig:unnamed-chunk-292}. We then adjusted the combined national data to the UN estimates of U5MR, and refit the models on the benchmarked data. 

%%%%%%%%%%%%%%%%%%%%%%%%%% Plot5 
\begin{knitrout}
\definecolor{shadecolor}{rgb}{0.969, 0.969, 0.969}\color{fgcolor}\begin{figure}[bht]

{\centering \includegraphics[width=.9\textwidth]{../Main/Figures/Yearly_national_SierraLeone} 

}

\caption[Sierra Leone]{Sierra Leone: Temporal national trends along with UN (B3) estimates described in You et al. (2015) and IHME estimates based on GBD 2015 Child Mortality Collaborators (2016). RW2 represents the smoothed national estimates using the original data before benchmarking with UN estimates. RW2-adj represents the smoothed national estimates using the benchmarked data.}\label{fig:unnamed-chunk-292}
\end{figure}


\end{knitrout}
 

We fit the RW2 model to the benchmarked data in each area. 
% The proportions of the explained variation is summarized in Table~\ref{tab:paste0(countryname, "-var")}. 
We compare the results in Figure~\ref{fig:unnamed-chunk-293} to \ref{fig:unnamed-chunk-297}.
Figure~\ref{fig:unnamed-chunk-293} compares the smoothed estimates against the direct estimates. Figure~\ref{fig:unnamed-chunk-294} and Figure~\ref{fig:unnamed-chunk-295} show the posterior median estimates of U5MR in each region over time and the reductions from 1990 period respectively.
Figure~\ref{fig:unnamed-chunk-296} shows the smoothed estimates by region over time and Figure~\ref{fig:unnamed-chunk-297} compares the smoothed estimates with direct estimates from each survey for each region over time.


% %%%%%%%%%%%%%%%%%%%%%%%%%%% Table1 
% <<echo=FALSE, results='asis'>>=
% load("rda/variance_tables.rda")
% countryname2 <- gsub(" ", "", countryname)
% variance <- tables.all[[countryname]]

% table_count <- table_count + 1

% names <- c("RW2 ($\\sigma^2_{\\gamma_{t}}$)", "ICAR ($\\sigma^2_{\\phi_{i}}$)", "IID space ($\\sigma^2_{\\theta_{i}}$)", "IID time ($\\sigma^2_{\\alpha_{t}}$)", "IID space time ($\\sigma^2_{\\delta_{it}}$)")

% variance$Proportion <- round(variance$Proportion*100, digits = 2)
% row.names(variance) <- names
% tab <- xtable(variance, digits = c(1, 3, 2),align = "l|ll",
%        label = paste0("tab:", countryname, "-var"),
%        caption = paste(country, ": summary of the variance components in the RW2 model", sep = ''))
% print(tab, comment = FALSE,sanitize.text.function = function(x) {x})
% @

%%%%%%%%%%%%%%%%%%%%%%%%%%% Plot1 
\begin{knitrout}
\definecolor{shadecolor}{rgb}{0.969, 0.969, 0.969}\color{fgcolor}\begin{figure}[bht]

{\centering \includegraphics[width=.9\textwidth]{../Main/Figures/SmoothvDirectSierraLeone_meta} 

}

\caption[Sierra Leone]{Sierra Leone: Smooth versus direct Admin 1 estimates. Left: Combined (meta-analysis) survey estimate against combined direct estimates. Right: Combined (meta-analysis) survey estimate against direct estimates from each survey.}\label{fig:unnamed-chunk-293}
\end{figure}


\end{knitrout}

%%%%%%%%%%%%%%%%%%%%%%%%%%% Plot2 
\begin{knitrout}
\definecolor{shadecolor}{rgb}{0.969, 0.969, 0.969}\color{fgcolor}\begin{figure}[bht]

{\centering \includegraphics[width=.9\textwidth]{../Main/Figures/SmoothMedianSierraLeone} 

}

\caption[Sierra Leone]{Sierra Leone: Maps of posterior medians over time.}\label{fig:unnamed-chunk-294}
\end{figure}


\end{knitrout}
%%%%%%%%%%%%%%%%%%%%%%%%%%% Plot2a
\begin{knitrout}
\definecolor{shadecolor}{rgb}{0.969, 0.969, 0.969}\color{fgcolor}\begin{figure}[bht]

{\centering \includegraphics[width=.9\textwidth]{../Main/Figures/ReductionMedianSierraLeone} 

}

\caption[Sierra Leone]{Sierra Leone: Maps of reduction of posterior median U5MR in each five-year period compared to 1990 over time.}\label{fig:unnamed-chunk-295}
\end{figure}


\end{knitrout}
%%%%%%%%%%%%%%%%%%%%%%%%%%% Plot3 
\begin{knitrout}
\definecolor{shadecolor}{rgb}{0.969, 0.969, 0.969}\color{fgcolor}\begin{figure}[bht]

{\centering \includegraphics[width=.95\textwidth]{../Main/Figures/Yearly_v_Periods_SierraLeone} 

}

\caption[Sierra Leone]{Sierra Leone: Smoothed regional estimates over time. The line indicates yearly posterior median estimates and error bars indicate 95 \% posterior credible interval at each time period.}\label{fig:unnamed-chunk-296}
\end{figure}


\end{knitrout}

%%%%%%%%%%%%%%%%%%%%%%%%%%% Plot4 
\begin{knitrout}
\definecolor{shadecolor}{rgb}{0.969, 0.969, 0.969}\color{fgcolor}\begin{figure}[bht]

{\centering \includegraphics[width=.9\textwidth]{../Main/Figures/LineSubMedianSierraLeone} 

}

\caption[Sierra Leone]{Sierra Leone: Smoothed regional estimates over time compared to the direct estimates from each surveys. Direct estimates are not benchmarked with UN estimates. The line indicates posterior median and error bars indicate 95\% posterior credible interval.}\label{fig:unnamed-chunk-297}
\end{figure}


\end{knitrout}
% \subsubsection{National model results}
We further assess the RW2 model by holding out some observations, and compare the projections to the direct estimates in these holdout observations. Figure~\ref{fig:unnamed-chunk-298} compares the predicted estimates for the out-of-sample observations  with the direct estimates by holding out observations from each area in each time period.  Figure~\ref{fig:unnamed-chunk-299} compares the histogram of the bias rescaled by the total variance in the cross validation studies. Figure~\ref{fig:unnamed-chunk-300} compares the rescaled bias by region and time periods.



% %%%%%%%%%%%%%%%%%%%%%%%%%%% Plot6
% << echo=FALSE, out.width = ".9\\textwidth", fig.width = 12, fig.height = 6, fig.cap = "Out-of-sample predictions along with direct estimates in the cross validation study where all data from each time period is held out and predicted using the rest of the data.">>=
% fig_count <- fig_count + 1
% knitr::include_graphics(paste0("../Main/Figures/CV_byYear_withError_", countryname2, ".pdf")) 
% @
 
%%%%%%%%%%%%%%%%%%%%%%%%%%% Plot7
\begin{knitrout}
\definecolor{shadecolor}{rgb}{0.969, 0.969, 0.969}\color{fgcolor}\begin{figure}[bht]

{\centering \includegraphics[width=.9\textwidth]{../Main/Figures/CV_byYearRegion_withError_SierraLeone} 

}

\caption[Sierra Leone]{Sierra Leone: Out-of-sample predictions along with direct estimates in the cross validation study where data from one region in each time period is held out and predicted using the rest of the data.}\label{fig:unnamed-chunk-298}
\end{figure}


\end{knitrout}

%%%%%%%%%%%%%%%%%%%%%%%%%%% Plot8
\begin{knitrout}
\definecolor{shadecolor}{rgb}{0.969, 0.969, 0.969}\color{fgcolor}\begin{figure}[bht]

{\centering \includegraphics[width=.9\textwidth]{../Main/Figures/CVbiasSierraLeone} 

}

\caption[Sierra Leone]{Sierra Leone: Histogram and QQ-plot of the rescaled difference between the smoothed estimates and the direct estimates in the cross validation study. The differences between the two estimates are rescaled by the square root of the total variance of the two estimates.}\label{fig:unnamed-chunk-299}
\end{figure}


\end{knitrout}

%%%%%%%%%%%%%%%%%%%%%%%%%%% Plot9
\begin{knitrout}
\definecolor{shadecolor}{rgb}{0.969, 0.969, 0.969}\color{fgcolor}\begin{figure}[bht]

{\centering \includegraphics[width=.7\textwidth]{../Main/Figures/CVbiasbyRegionSierraLeone} 

}

\caption[Sierra Leone]{Sierra Leone: Line plot of the difference between smoothed estimates and the direct estimates in the cross validation study. The differences between the two estimates are rescaled by the square root of the total variance of the two estimates.}\label{fig:unnamed-chunk-300}
\end{figure}


\end{knitrout}


%%%%%%%%%%%%%%%%%%%%%%%%%%%%%%%%%%%%%%%%%%%%%%%%%%%%%%%%%%%%%%%%%%%%%%%%%%%%%%%%%%%%%%%%%%%%%%%%%%
\clearpage
\subsubsection{Tanzania}


% \subsubsection{Summary of DHS surveys}

%%%%%%%%%%%%%%%%%%%%%%%%%%% Summary 


DHS surveys were conducted in Tanzania in 1996, 1999, 2005, 2010, and 2015.
% years.out[1:(length(years.out)-1)], and years.out[length(years.out)]. 

We fit both the RW2 only model to the combined national data, and compare the time trend at national level with the estimates produced by the UN and IHME in Figure~\ref{fig:unnamed-chunk-302}. We then adjusted the combined national data to the UN estimates of U5MR, and refit the models on the benchmarked data. 

%%%%%%%%%%%%%%%%%%%%%%%%%% Plot5 
\begin{knitrout}
\definecolor{shadecolor}{rgb}{0.969, 0.969, 0.969}\color{fgcolor}\begin{figure}[bht]

{\centering \includegraphics[width=.9\textwidth]{../Main/Figures/Yearly_national_Tanzania} 

}

\caption[Tanzania]{Tanzania: Temporal national trends along with UN (B3) estimates described in You et al. (2015) and IHME estimates based on GBD 2015 Child Mortality Collaborators (2016). RW2 represents the smoothed national estimates using the original data before benchmarking with UN estimates. RW2-adj represents the smoothed national estimates using the benchmarked data.}\label{fig:unnamed-chunk-302}
\end{figure}


\end{knitrout}
 

We fit the RW2 model to the benchmarked data in each area. 
% The proportions of the explained variation is summarized in Table~\ref{tab:paste0(countryname, "-var")}. 
We compare the results in Figure~\ref{fig:unnamed-chunk-303} to \ref{fig:unnamed-chunk-307}.
Figure~\ref{fig:unnamed-chunk-303} compares the smoothed estimates against the direct estimates. Figure~\ref{fig:unnamed-chunk-304} and Figure~\ref{fig:unnamed-chunk-305} show the posterior median estimates of U5MR in each region over time and the reductions from 1990 period respectively.
Figure~\ref{fig:unnamed-chunk-306} shows the smoothed estimates by region over time and Figure~\ref{fig:unnamed-chunk-307} compares the smoothed estimates with direct estimates from each survey for each region over time.


% %%%%%%%%%%%%%%%%%%%%%%%%%%% Table1 
% <<echo=FALSE, results='asis'>>=
% load("rda/variance_tables.rda")
% countryname2 <- gsub(" ", "", countryname)
% variance <- tables.all[[countryname]]

% table_count <- table_count + 1

% names <- c("RW2 ($\\sigma^2_{\\gamma_{t}}$)", "ICAR ($\\sigma^2_{\\phi_{i}}$)", "IID space ($\\sigma^2_{\\theta_{i}}$)", "IID time ($\\sigma^2_{\\alpha_{t}}$)", "IID space time ($\\sigma^2_{\\delta_{it}}$)")

% variance$Proportion <- round(variance$Proportion*100, digits = 2)
% row.names(variance) <- names
% tab <- xtable(variance, digits = c(1, 3, 2),align = "l|ll",
%        label = paste0("tab:", countryname, "-var"),
%        caption = paste(country, ": summary of the variance components in the RW2 model", sep = ''))
% print(tab, comment = FALSE,sanitize.text.function = function(x) {x})
% @

%%%%%%%%%%%%%%%%%%%%%%%%%%% Plot1 
\begin{knitrout}
\definecolor{shadecolor}{rgb}{0.969, 0.969, 0.969}\color{fgcolor}\begin{figure}[bht]

{\centering \includegraphics[width=.9\textwidth]{../Main/Figures/SmoothvDirectTanzania_meta} 

}

\caption[Tanzania]{Tanzania: Smooth versus direct Admin 1 estimates. Left: Combined (meta-analysis) survey estimate against combined direct estimates. Right: Combined (meta-analysis) survey estimate against direct estimates from each survey.}\label{fig:unnamed-chunk-303}
\end{figure}


\end{knitrout}

%%%%%%%%%%%%%%%%%%%%%%%%%%% Plot2 
\begin{knitrout}
\definecolor{shadecolor}{rgb}{0.969, 0.969, 0.969}\color{fgcolor}\begin{figure}[bht]

{\centering \includegraphics[width=.9\textwidth]{../Main/Figures/SmoothMedianTanzania} 

}

\caption[Tanzania]{Tanzania: Maps of posterior medians over time.}\label{fig:unnamed-chunk-304}
\end{figure}


\end{knitrout}
%%%%%%%%%%%%%%%%%%%%%%%%%%% Plot2a
\begin{knitrout}
\definecolor{shadecolor}{rgb}{0.969, 0.969, 0.969}\color{fgcolor}\begin{figure}[bht]

{\centering \includegraphics[width=.9\textwidth]{../Main/Figures/ReductionMedianTanzania} 

}

\caption[Tanzania]{Tanzania: Maps of reduction of posterior median U5MR in each five-year period compared to 1990 over time.}\label{fig:unnamed-chunk-305}
\end{figure}


\end{knitrout}
%%%%%%%%%%%%%%%%%%%%%%%%%%% Plot3 
\begin{knitrout}
\definecolor{shadecolor}{rgb}{0.969, 0.969, 0.969}\color{fgcolor}\begin{figure}[bht]

{\centering \includegraphics[width=.95\textwidth]{../Main/Figures/Yearly_v_Periods_Tanzania} 

}

\caption[Tanzania]{Tanzania: Smoothed regional estimates over time. The line indicates yearly posterior median estimates and error bars indicate 95 \% posterior credible interval at each time period.}\label{fig:unnamed-chunk-306}
\end{figure}


\end{knitrout}

%%%%%%%%%%%%%%%%%%%%%%%%%%% Plot4 
\begin{knitrout}
\definecolor{shadecolor}{rgb}{0.969, 0.969, 0.969}\color{fgcolor}\begin{figure}[bht]

{\centering \includegraphics[width=.9\textwidth]{../Main/Figures/LineSubMedianTanzania} 

}

\caption[Tanzania]{Tanzania: Smoothed regional estimates over time compared to the direct estimates from each surveys. Direct estimates are not benchmarked with UN estimates. The line indicates posterior median and error bars indicate 95\% posterior credible interval.}\label{fig:unnamed-chunk-307}
\end{figure}


\end{knitrout}
% \subsubsection{National model results}
We further assess the RW2 model by holding out some observations, and compare the projections to the direct estimates in these holdout observations. Figure~\ref{fig:unnamed-chunk-308} compares the predicted estimates for the out-of-sample observations  with the direct estimates by holding out observations from each area in each time period.  Figure~\ref{fig:unnamed-chunk-309} compares the histogram of the bias rescaled by the total variance in the cross validation studies. Figure~\ref{fig:unnamed-chunk-310} compares the rescaled bias by region and time periods.



% %%%%%%%%%%%%%%%%%%%%%%%%%%% Plot6
% << echo=FALSE, out.width = ".9\\textwidth", fig.width = 12, fig.height = 6, fig.cap = "Out-of-sample predictions along with direct estimates in the cross validation study where all data from each time period is held out and predicted using the rest of the data.">>=
% fig_count <- fig_count + 1
% knitr::include_graphics(paste0("../Main/Figures/CV_byYear_withError_", countryname2, ".pdf")) 
% @
 
%%%%%%%%%%%%%%%%%%%%%%%%%%% Plot7
\begin{knitrout}
\definecolor{shadecolor}{rgb}{0.969, 0.969, 0.969}\color{fgcolor}\begin{figure}[bht]

{\centering \includegraphics[width=.9\textwidth]{../Main/Figures/CV_byYearRegion_withError_Tanzania} 

}

\caption[Tanzania]{Tanzania: Out-of-sample predictions along with direct estimates in the cross validation study where data from one region in each time period is held out and predicted using the rest of the data.}\label{fig:unnamed-chunk-308}
\end{figure}


\end{knitrout}

%%%%%%%%%%%%%%%%%%%%%%%%%%% Plot8
\begin{knitrout}
\definecolor{shadecolor}{rgb}{0.969, 0.969, 0.969}\color{fgcolor}\begin{figure}[bht]

{\centering \includegraphics[width=.9\textwidth]{../Main/Figures/CVbiasTanzania} 

}

\caption[Tanzania]{Tanzania: Histogram and QQ-plot of the rescaled difference between the smoothed estimates and the direct estimates in the cross validation study. The differences between the two estimates are rescaled by the square root of the total variance of the two estimates.}\label{fig:unnamed-chunk-309}
\end{figure}


\end{knitrout}

%%%%%%%%%%%%%%%%%%%%%%%%%%% Plot9
\begin{knitrout}
\definecolor{shadecolor}{rgb}{0.969, 0.969, 0.969}\color{fgcolor}\begin{figure}[bht]

{\centering \includegraphics[width=.7\textwidth]{../Main/Figures/CVbiasbyRegionTanzania} 

}

\caption[Tanzania]{Tanzania: Line plot of the difference between smoothed estimates and the direct estimates in the cross validation study. The differences between the two estimates are rescaled by the square root of the total variance of the two estimates.}\label{fig:unnamed-chunk-310}
\end{figure}


\end{knitrout}


%%%%%%%%%%%%%%%%%%%%%%%%%%%%%%%%%%%%%%%%%%%%%%%%%%%%%%%%%%%%%%%%%%%%%%%%%%%%%%%%%%%%%%%%%%%%%%%%%%
\clearpage
\subsubsection{Togo}


% \subsubsection{Summary of DHS surveys}

%%%%%%%%%%%%%%%%%%%%%%%%%%% Summary 


DHS surveys were conducted in Togo in 1998, and 2013.
% years.out[1:(length(years.out)-1)], and years.out[length(years.out)]. 

We fit both the RW2 only model to the combined national data, and compare the time trend at national level with the estimates produced by the UN and IHME in Figure~\ref{fig:unnamed-chunk-312}. We then adjusted the combined national data to the UN estimates of U5MR, and refit the models on the benchmarked data. 

%%%%%%%%%%%%%%%%%%%%%%%%%% Plot5 
\begin{knitrout}
\definecolor{shadecolor}{rgb}{0.969, 0.969, 0.969}\color{fgcolor}\begin{figure}[bht]

{\centering \includegraphics[width=.9\textwidth]{../Main/Figures/Yearly_national_Togo} 

}

\caption[Togo]{Togo: Temporal national trends along with UN (B3) estimates described in You et al. (2015) and IHME estimates based on GBD 2015 Child Mortality Collaborators (2016). RW2 represents the smoothed national estimates using the original data before benchmarking with UN estimates. RW2-adj represents the smoothed national estimates using the benchmarked data.}\label{fig:unnamed-chunk-312}
\end{figure}


\end{knitrout}
 

We fit the RW2 model to the benchmarked data in each area. 
% The proportions of the explained variation is summarized in Table~\ref{tab:paste0(countryname, "-var")}. 
We compare the results in Figure~\ref{fig:unnamed-chunk-313} to \ref{fig:unnamed-chunk-317}.
Figure~\ref{fig:unnamed-chunk-313} compares the smoothed estimates against the direct estimates. Figure~\ref{fig:unnamed-chunk-314} and Figure~\ref{fig:unnamed-chunk-315} show the posterior median estimates of U5MR in each region over time and the reductions from 1990 period respectively.
Figure~\ref{fig:unnamed-chunk-316} shows the smoothed estimates by region over time and Figure~\ref{fig:unnamed-chunk-317} compares the smoothed estimates with direct estimates from each survey for each region over time.


% %%%%%%%%%%%%%%%%%%%%%%%%%%% Table1 
% <<echo=FALSE, results='asis'>>=
% load("rda/variance_tables.rda")
% countryname2 <- gsub(" ", "", countryname)
% variance <- tables.all[[countryname]]

% table_count <- table_count + 1

% names <- c("RW2 ($\\sigma^2_{\\gamma_{t}}$)", "ICAR ($\\sigma^2_{\\phi_{i}}$)", "IID space ($\\sigma^2_{\\theta_{i}}$)", "IID time ($\\sigma^2_{\\alpha_{t}}$)", "IID space time ($\\sigma^2_{\\delta_{it}}$)")

% variance$Proportion <- round(variance$Proportion*100, digits = 2)
% row.names(variance) <- names
% tab <- xtable(variance, digits = c(1, 3, 2),align = "l|ll",
%        label = paste0("tab:", countryname, "-var"),
%        caption = paste(country, ": summary of the variance components in the RW2 model", sep = ''))
% print(tab, comment = FALSE,sanitize.text.function = function(x) {x})
% @

%%%%%%%%%%%%%%%%%%%%%%%%%%% Plot1 
\begin{knitrout}
\definecolor{shadecolor}{rgb}{0.969, 0.969, 0.969}\color{fgcolor}\begin{figure}[bht]

{\centering \includegraphics[width=.9\textwidth]{../Main/Figures/SmoothvDirectTogo_meta} 

}

\caption[Togo]{Togo: Smooth versus direct Admin 1 estimates. Left: Combined (meta-analysis) survey estimate against combined direct estimates. Right: Combined (meta-analysis) survey estimate against direct estimates from each survey.}\label{fig:unnamed-chunk-313}
\end{figure}


\end{knitrout}

%%%%%%%%%%%%%%%%%%%%%%%%%%% Plot2 
\begin{knitrout}
\definecolor{shadecolor}{rgb}{0.969, 0.969, 0.969}\color{fgcolor}\begin{figure}[bht]

{\centering \includegraphics[width=.9\textwidth]{../Main/Figures/SmoothMedianTogo} 

}

\caption[Togo]{Togo: Maps of posterior medians over time.}\label{fig:unnamed-chunk-314}
\end{figure}


\end{knitrout}
%%%%%%%%%%%%%%%%%%%%%%%%%%% Plot2a
\begin{knitrout}
\definecolor{shadecolor}{rgb}{0.969, 0.969, 0.969}\color{fgcolor}\begin{figure}[bht]

{\centering \includegraphics[width=.9\textwidth]{../Main/Figures/ReductionMedianTogo} 

}

\caption[Togo]{Togo: Maps of reduction of posterior median U5MR in each five-year period compared to 1990 over time.}\label{fig:unnamed-chunk-315}
\end{figure}


\end{knitrout}
%%%%%%%%%%%%%%%%%%%%%%%%%%% Plot3 
\begin{knitrout}
\definecolor{shadecolor}{rgb}{0.969, 0.969, 0.969}\color{fgcolor}\begin{figure}[bht]

{\centering \includegraphics[width=.95\textwidth]{../Main/Figures/Yearly_v_Periods_Togo} 

}

\caption[Togo]{Togo: Smoothed regional estimates over time. The line indicates yearly posterior median estimates and error bars indicate 95 \% posterior credible interval at each time period.}\label{fig:unnamed-chunk-316}
\end{figure}


\end{knitrout}

%%%%%%%%%%%%%%%%%%%%%%%%%%% Plot4 
\begin{knitrout}
\definecolor{shadecolor}{rgb}{0.969, 0.969, 0.969}\color{fgcolor}\begin{figure}[bht]

{\centering \includegraphics[width=.9\textwidth]{../Main/Figures/LineSubMedianTogo} 

}

\caption[Togo]{Togo: Smoothed regional estimates over time compared to the direct estimates from each surveys. Direct estimates are not benchmarked with UN estimates. The line indicates posterior median and error bars indicate 95\% posterior credible interval.}\label{fig:unnamed-chunk-317}
\end{figure}


\end{knitrout}
% \subsubsection{National model results}
We further assess the RW2 model by holding out some observations, and compare the projections to the direct estimates in these holdout observations. Figure~\ref{fig:unnamed-chunk-318} compares the predicted estimates for the out-of-sample observations  with the direct estimates by holding out observations from each area in each time period.  Figure~\ref{fig:unnamed-chunk-319} compares the histogram of the bias rescaled by the total variance in the cross validation studies. Figure~\ref{fig:unnamed-chunk-320} compares the rescaled bias by region and time periods.



% %%%%%%%%%%%%%%%%%%%%%%%%%%% Plot6
% << echo=FALSE, out.width = ".9\\textwidth", fig.width = 12, fig.height = 6, fig.cap = "Out-of-sample predictions along with direct estimates in the cross validation study where all data from each time period is held out and predicted using the rest of the data.">>=
% fig_count <- fig_count + 1
% knitr::include_graphics(paste0("../Main/Figures/CV_byYear_withError_", countryname2, ".pdf")) 
% @
 
%%%%%%%%%%%%%%%%%%%%%%%%%%% Plot7
\begin{knitrout}
\definecolor{shadecolor}{rgb}{0.969, 0.969, 0.969}\color{fgcolor}\begin{figure}[bht]

{\centering \includegraphics[width=.9\textwidth]{../Main/Figures/CV_byYearRegion_withError_Togo} 

}

\caption[Togo]{Togo: Out-of-sample predictions along with direct estimates in the cross validation study where data from one region in each time period is held out and predicted using the rest of the data.}\label{fig:unnamed-chunk-318}
\end{figure}


\end{knitrout}

%%%%%%%%%%%%%%%%%%%%%%%%%%% Plot8
\begin{knitrout}
\definecolor{shadecolor}{rgb}{0.969, 0.969, 0.969}\color{fgcolor}\begin{figure}[bht]

{\centering \includegraphics[width=.9\textwidth]{../Main/Figures/CVbiasTogo} 

}

\caption[Togo]{Togo: Histogram and QQ-plot of the rescaled difference between the smoothed estimates and the direct estimates in the cross validation study. The differences between the two estimates are rescaled by the square root of the total variance of the two estimates.}\label{fig:unnamed-chunk-319}
\end{figure}


\end{knitrout}

%%%%%%%%%%%%%%%%%%%%%%%%%%% Plot9
\begin{knitrout}
\definecolor{shadecolor}{rgb}{0.969, 0.969, 0.969}\color{fgcolor}\begin{figure}[bht]

{\centering \includegraphics[width=.7\textwidth]{../Main/Figures/CVbiasbyRegionTogo} 

}

\caption[Togo]{Togo: Line plot of the difference between smoothed estimates and the direct estimates in the cross validation study. The differences between the two estimates are rescaled by the square root of the total variance of the two estimates.}\label{fig:unnamed-chunk-320}
\end{figure}


\end{knitrout}


%%%%%%%%%%%%%%%%%%%%%%%%%%%%%%%%%%%%%%%%%%%%%%%%%%%%%%%%%%%%%%%%%%%%%%%%%%%%%%%%%%%%%%%%%%%%%%%%%%
\clearpage
\subsubsection{Uganda}


% \subsubsection{Summary of DHS surveys}

%%%%%%%%%%%%%%%%%%%%%%%%%%% Summary 


DHS surveys were conducted in Uganda in 1989, 1995, 2001, 2006, and 2011.
% years.out[1:(length(years.out)-1)], and years.out[length(years.out)]. 

We fit both the RW2 only model to the combined national data, and compare the time trend at national level with the estimates produced by the UN and IHME in Figure~\ref{fig:unnamed-chunk-322}. We then adjusted the combined national data to the UN estimates of U5MR, and refit the models on the benchmarked data. 

%%%%%%%%%%%%%%%%%%%%%%%%%% Plot5 
\begin{knitrout}
\definecolor{shadecolor}{rgb}{0.969, 0.969, 0.969}\color{fgcolor}\begin{figure}[bht]

{\centering \includegraphics[width=.9\textwidth]{../Main/Figures/Yearly_national_Uganda} 

}

\caption[Uganda]{Uganda: Temporal national trends along with UN (B3) estimates described in You et al. (2015) and IHME estimates based on GBD 2015 Child Mortality Collaborators (2016). RW2 represents the smoothed national estimates using the original data before benchmarking with UN estimates. RW2-adj represents the smoothed national estimates using the benchmarked data.}\label{fig:unnamed-chunk-322}
\end{figure}


\end{knitrout}
 

We fit the RW2 model to the benchmarked data in each area. 
% The proportions of the explained variation is summarized in Table~\ref{tab:paste0(countryname, "-var")}. 
We compare the results in Figure~\ref{fig:unnamed-chunk-323} to \ref{fig:unnamed-chunk-327}.
Figure~\ref{fig:unnamed-chunk-323} compares the smoothed estimates against the direct estimates. Figure~\ref{fig:unnamed-chunk-324} and Figure~\ref{fig:unnamed-chunk-325} show the posterior median estimates of U5MR in each region over time and the reductions from 1990 period respectively.
Figure~\ref{fig:unnamed-chunk-326} shows the smoothed estimates by region over time and Figure~\ref{fig:unnamed-chunk-327} compares the smoothed estimates with direct estimates from each survey for each region over time.


% %%%%%%%%%%%%%%%%%%%%%%%%%%% Table1 
% <<echo=FALSE, results='asis'>>=
% load("rda/variance_tables.rda")
% countryname2 <- gsub(" ", "", countryname)
% variance <- tables.all[[countryname]]

% table_count <- table_count + 1

% names <- c("RW2 ($\\sigma^2_{\\gamma_{t}}$)", "ICAR ($\\sigma^2_{\\phi_{i}}$)", "IID space ($\\sigma^2_{\\theta_{i}}$)", "IID time ($\\sigma^2_{\\alpha_{t}}$)", "IID space time ($\\sigma^2_{\\delta_{it}}$)")

% variance$Proportion <- round(variance$Proportion*100, digits = 2)
% row.names(variance) <- names
% tab <- xtable(variance, digits = c(1, 3, 2),align = "l|ll",
%        label = paste0("tab:", countryname, "-var"),
%        caption = paste(country, ": summary of the variance components in the RW2 model", sep = ''))
% print(tab, comment = FALSE,sanitize.text.function = function(x) {x})
% @

%%%%%%%%%%%%%%%%%%%%%%%%%%% Plot1 
\begin{knitrout}
\definecolor{shadecolor}{rgb}{0.969, 0.969, 0.969}\color{fgcolor}\begin{figure}[bht]

{\centering \includegraphics[width=.9\textwidth]{../Main/Figures/SmoothvDirectUganda_meta} 

}

\caption[Uganda]{Uganda: Smooth versus direct Admin 1 estimates. Left: Combined (meta-analysis) survey estimate against combined direct estimates. Right: Combined (meta-analysis) survey estimate against direct estimates from each survey.}\label{fig:unnamed-chunk-323}
\end{figure}


\end{knitrout}

%%%%%%%%%%%%%%%%%%%%%%%%%%% Plot2 
\begin{knitrout}
\definecolor{shadecolor}{rgb}{0.969, 0.969, 0.969}\color{fgcolor}\begin{figure}[bht]

{\centering \includegraphics[width=.9\textwidth]{../Main/Figures/SmoothMedianUganda} 

}

\caption[Uganda]{Uganda: Maps of posterior medians over time.}\label{fig:unnamed-chunk-324}
\end{figure}


\end{knitrout}
%%%%%%%%%%%%%%%%%%%%%%%%%%% Plot2a
\begin{knitrout}
\definecolor{shadecolor}{rgb}{0.969, 0.969, 0.969}\color{fgcolor}\begin{figure}[bht]

{\centering \includegraphics[width=.9\textwidth]{../Main/Figures/ReductionMedianUganda} 

}

\caption[Uganda]{Uganda: Maps of reduction of posterior median U5MR in each five-year period compared to 1990 over time.}\label{fig:unnamed-chunk-325}
\end{figure}


\end{knitrout}
%%%%%%%%%%%%%%%%%%%%%%%%%%% Plot3 
\begin{knitrout}
\definecolor{shadecolor}{rgb}{0.969, 0.969, 0.969}\color{fgcolor}\begin{figure}[bht]

{\centering \includegraphics[width=.95\textwidth]{../Main/Figures/Yearly_v_Periods_Uganda} 

}

\caption[Uganda]{Uganda: Smoothed regional estimates over time. The line indicates yearly posterior median estimates and error bars indicate 95 \% posterior credible interval at each time period.}\label{fig:unnamed-chunk-326}
\end{figure}


\end{knitrout}

%%%%%%%%%%%%%%%%%%%%%%%%%%% Plot4 
\begin{knitrout}
\definecolor{shadecolor}{rgb}{0.969, 0.969, 0.969}\color{fgcolor}\begin{figure}[bht]

{\centering \includegraphics[width=.9\textwidth]{../Main/Figures/LineSubMedianUganda} 

}

\caption[Uganda]{Uganda: Smoothed regional estimates over time compared to the direct estimates from each surveys. Direct estimates are not benchmarked with UN estimates. The line indicates posterior median and error bars indicate 95\% posterior credible interval.}\label{fig:unnamed-chunk-327}
\end{figure}


\end{knitrout}
% \subsubsection{National model results}
We further assess the RW2 model by holding out some observations, and compare the projections to the direct estimates in these holdout observations. Figure~\ref{fig:unnamed-chunk-328} compares the predicted estimates for the out-of-sample observations  with the direct estimates by holding out observations from each area in each time period.  Figure~\ref{fig:unnamed-chunk-329} compares the histogram of the bias rescaled by the total variance in the cross validation studies. Figure~\ref{fig:unnamed-chunk-330} compares the rescaled bias by region and time periods.



% %%%%%%%%%%%%%%%%%%%%%%%%%%% Plot6
% << echo=FALSE, out.width = ".9\\textwidth", fig.width = 12, fig.height = 6, fig.cap = "Out-of-sample predictions along with direct estimates in the cross validation study where all data from each time period is held out and predicted using the rest of the data.">>=
% fig_count <- fig_count + 1
% knitr::include_graphics(paste0("../Main/Figures/CV_byYear_withError_", countryname2, ".pdf")) 
% @
 
%%%%%%%%%%%%%%%%%%%%%%%%%%% Plot7
\begin{knitrout}
\definecolor{shadecolor}{rgb}{0.969, 0.969, 0.969}\color{fgcolor}\begin{figure}[bht]

{\centering \includegraphics[width=.9\textwidth]{../Main/Figures/CV_byYearRegion_withError_Uganda} 

}

\caption[Uganda]{Uganda: Out-of-sample predictions along with direct estimates in the cross validation study where data from one region in each time period is held out and predicted using the rest of the data.}\label{fig:unnamed-chunk-328}
\end{figure}


\end{knitrout}

%%%%%%%%%%%%%%%%%%%%%%%%%%% Plot8
\begin{knitrout}
\definecolor{shadecolor}{rgb}{0.969, 0.969, 0.969}\color{fgcolor}\begin{figure}[bht]

{\centering \includegraphics[width=.9\textwidth]{../Main/Figures/CVbiasUganda} 

}

\caption[Uganda]{Uganda: Histogram and QQ-plot of the rescaled difference between the smoothed estimates and the direct estimates in the cross validation study. The differences between the two estimates are rescaled by the square root of the total variance of the two estimates.}\label{fig:unnamed-chunk-329}
\end{figure}


\end{knitrout}

%%%%%%%%%%%%%%%%%%%%%%%%%%% Plot9
\begin{knitrout}
\definecolor{shadecolor}{rgb}{0.969, 0.969, 0.969}\color{fgcolor}\begin{figure}[bht]

{\centering \includegraphics[width=.7\textwidth]{../Main/Figures/CVbiasbyRegionUganda} 

}

\caption[Uganda]{Uganda: Line plot of the difference between smoothed estimates and the direct estimates in the cross validation study. The differences between the two estimates are rescaled by the square root of the total variance of the two estimates.}\label{fig:unnamed-chunk-330}
\end{figure}


\end{knitrout}



%%%%%%%%%%%%%%%%%%%%%%%%%%%%%%%%%%%%%%%%%%%%%%%%%%%%%%%%%%%%%%%%%%%%%%%%%%%%%%%%%%%%%%%%%%%%%%%%%%
\clearpage
\subsubsection{Zambia}


% \subsubsection{Summary of DHS surveys}

%%%%%%%%%%%%%%%%%%%%%%%%%%% Summary 


DHS surveys were conducted in Zambia in 1992, 1996, 2007, and 2014.
% years.out[1:(length(years.out)-1)], and years.out[length(years.out)]. 

We fit both the RW2 only model to the combined national data, and compare the time trend at national level with the estimates produced by the UN and IHME in Figure~\ref{fig:unnamed-chunk-332}. We then adjusted the combined national data to the UN estimates of U5MR, and refit the models on the benchmarked data. 

%%%%%%%%%%%%%%%%%%%%%%%%%% Plot5 
\begin{knitrout}
\definecolor{shadecolor}{rgb}{0.969, 0.969, 0.969}\color{fgcolor}\begin{figure}[bht]

{\centering \includegraphics[width=.9\textwidth]{../Main/Figures/Yearly_national_Zambia} 

}

\caption[Zambia]{Zambia: Temporal national trends along with UN (B3) estimates described in You et al. (2015) and IHME estimates based on GBD 2015 Child Mortality Collaborators (2016). RW2 represents the smoothed national estimates using the original data before benchmarking with UN estimates. RW2-adj represents the smoothed national estimates using the benchmarked data.}\label{fig:unnamed-chunk-332}
\end{figure}


\end{knitrout}
 

We fit the RW2 model to the benchmarked data in each area. 
% The proportions of the explained variation is summarized in Table~\ref{tab:paste0(countryname, "-var")}. 
We compare the results in Figure~\ref{fig:unnamed-chunk-333} to \ref{fig:unnamed-chunk-337}.
Figure~\ref{fig:unnamed-chunk-333} compares the smoothed estimates against the direct estimates. Figure~\ref{fig:unnamed-chunk-334} and Figure~\ref{fig:unnamed-chunk-335} show the posterior median estimates of U5MR in each region over time and the reductions from 1990 period respectively.
Figure~\ref{fig:unnamed-chunk-336} shows the smoothed estimates by region over time and Figure~\ref{fig:unnamed-chunk-337} compares the smoothed estimates with direct estimates from each survey for each region over time.


% %%%%%%%%%%%%%%%%%%%%%%%%%%% Table1 
% <<echo=FALSE, results='asis'>>=
% load("rda/variance_tables.rda")
% countryname2 <- gsub(" ", "", countryname)
% variance <- tables.all[[countryname]]

% table_count <- table_count + 1

% names <- c("RW2 ($\\sigma^2_{\\gamma_{t}}$)", "ICAR ($\\sigma^2_{\\phi_{i}}$)", "IID space ($\\sigma^2_{\\theta_{i}}$)", "IID time ($\\sigma^2_{\\alpha_{t}}$)", "IID space time ($\\sigma^2_{\\delta_{it}}$)")

% variance$Proportion <- round(variance$Proportion*100, digits = 2)
% row.names(variance) <- names
% tab <- xtable(variance, digits = c(1, 3, 2),align = "l|ll",
%        label = paste0("tab:", countryname, "-var"),
%        caption = paste(country, ": summary of the variance components in the RW2 model", sep = ''))
% print(tab, comment = FALSE,sanitize.text.function = function(x) {x})
% @

%%%%%%%%%%%%%%%%%%%%%%%%%%% Plot1 
\begin{knitrout}
\definecolor{shadecolor}{rgb}{0.969, 0.969, 0.969}\color{fgcolor}\begin{figure}[bht]

{\centering \includegraphics[width=.9\textwidth]{../Main/Figures/SmoothvDirectZambia_meta} 

}

\caption[Zambia]{Zambia: Smooth versus direct Admin 1 estimates. Left: Combined (meta-analysis) survey estimate against combined direct estimates. Right: Combined (meta-analysis) survey estimate against direct estimates from each survey.}\label{fig:unnamed-chunk-333}
\end{figure}


\end{knitrout}

%%%%%%%%%%%%%%%%%%%%%%%%%%% Plot2 
\begin{knitrout}
\definecolor{shadecolor}{rgb}{0.969, 0.969, 0.969}\color{fgcolor}\begin{figure}[bht]

{\centering \includegraphics[width=.9\textwidth]{../Main/Figures/SmoothMedianZambia} 

}

\caption[Zambia]{Zambia: Maps of posterior medians over time.}\label{fig:unnamed-chunk-334}
\end{figure}


\end{knitrout}
%%%%%%%%%%%%%%%%%%%%%%%%%%% Plot2a
\begin{knitrout}
\definecolor{shadecolor}{rgb}{0.969, 0.969, 0.969}\color{fgcolor}\begin{figure}[bht]

{\centering \includegraphics[width=.9\textwidth]{../Main/Figures/ReductionMedianZambia} 

}

\caption[Zambia]{Zambia: Maps of reduction of posterior median U5MR in each five-year period compared to 1990 over time.}\label{fig:unnamed-chunk-335}
\end{figure}


\end{knitrout}
%%%%%%%%%%%%%%%%%%%%%%%%%%% Plot3 
\begin{knitrout}
\definecolor{shadecolor}{rgb}{0.969, 0.969, 0.969}\color{fgcolor}\begin{figure}[bht]

{\centering \includegraphics[width=.95\textwidth]{../Main/Figures/Yearly_v_Periods_Zambia} 

}

\caption[Zambia]{Zambia: Smoothed regional estimates over time. The line indicates yearly posterior median estimates and error bars indicate 95 \% posterior credible interval at each time period.}\label{fig:unnamed-chunk-336}
\end{figure}


\end{knitrout}

%%%%%%%%%%%%%%%%%%%%%%%%%%% Plot4 
\begin{knitrout}
\definecolor{shadecolor}{rgb}{0.969, 0.969, 0.969}\color{fgcolor}\begin{figure}[bht]

{\centering \includegraphics[width=.9\textwidth]{../Main/Figures/LineSubMedianZambia} 

}

\caption[Zambia]{Zambia: Smoothed regional estimates over time compared to the direct estimates from each surveys. Direct estimates are not benchmarked with UN estimates. The line indicates posterior median and error bars indicate 95\% posterior credible interval.}\label{fig:unnamed-chunk-337}
\end{figure}


\end{knitrout}
% \subsubsection{National model results}
We further assess the RW2 model by holding out some observations, and compare the projections to the direct estimates in these holdout observations. Figure~\ref{fig:unnamed-chunk-338} compares the predicted estimates for the out-of-sample observations  with the direct estimates by holding out observations from each area in each time period.  Figure~\ref{fig:unnamed-chunk-339} compares the histogram of the bias rescaled by the total variance in the cross validation studies. Figure~\ref{fig:unnamed-chunk-340} compares the rescaled bias by region and time periods.



% %%%%%%%%%%%%%%%%%%%%%%%%%%% Plot6
% << echo=FALSE, out.width = ".9\\textwidth", fig.width = 12, fig.height = 6, fig.cap = "Out-of-sample predictions along with direct estimates in the cross validation study where all data from each time period is held out and predicted using the rest of the data.">>=
% fig_count <- fig_count + 1
% knitr::include_graphics(paste0("../Main/Figures/CV_byYear_withError_", countryname2, ".pdf")) 
% @
 
%%%%%%%%%%%%%%%%%%%%%%%%%%% Plot7
\begin{knitrout}
\definecolor{shadecolor}{rgb}{0.969, 0.969, 0.969}\color{fgcolor}\begin{figure}[bht]

{\centering \includegraphics[width=.9\textwidth]{../Main/Figures/CV_byYearRegion_withError_Zambia} 

}

\caption[Zambia]{Zambia: Out-of-sample predictions along with direct estimates in the cross validation study where data from one region in each time period is held out and predicted using the rest of the data.}\label{fig:unnamed-chunk-338}
\end{figure}


\end{knitrout}

%%%%%%%%%%%%%%%%%%%%%%%%%%% Plot8
\begin{knitrout}
\definecolor{shadecolor}{rgb}{0.969, 0.969, 0.969}\color{fgcolor}\begin{figure}[bht]

{\centering \includegraphics[width=.9\textwidth]{../Main/Figures/CVbiasZambia} 

}

\caption[Zambia]{Zambia: Histogram and QQ-plot of the rescaled difference between the smoothed estimates and the direct estimates in the cross validation study. The differences between the two estimates are rescaled by the square root of the total variance of the two estimates.}\label{fig:unnamed-chunk-339}
\end{figure}


\end{knitrout}

%%%%%%%%%%%%%%%%%%%%%%%%%%% Plot9
\begin{knitrout}
\definecolor{shadecolor}{rgb}{0.969, 0.969, 0.969}\color{fgcolor}\begin{figure}[bht]

{\centering \includegraphics[width=.7\textwidth]{../Main/Figures/CVbiasbyRegionZambia} 

}

\caption[Zambia]{Zambia: Line plot of the difference between smoothed estimates and the direct estimates in the cross validation study. The differences between the two estimates are rescaled by the square root of the total variance of the two estimates.}\label{fig:unnamed-chunk-340}
\end{figure}


\end{knitrout}



%%%%%%%%%%%%%%%%%%%%%%%%%%%%%%%%%%%%%%%%%%%%%%%%%%%%%%%%%%%%%%%%%%%%%%%%%%%%%%%%%%%%%%%%%%%%%%%%%%
\clearpage
\subsubsection{Zimbabwe}


% \subsubsection{Summary of DHS surveys}

%%%%%%%%%%%%%%%%%%%%%%%%%%% Summary 


DHS surveys were conducted in Zimbabwe in 1994, 1999, 2006, and 2015.
% years.out[1:(length(years.out)-1)], and years.out[length(years.out)]. 

We fit both the RW2 only model to the combined national data, and compare the time trend at national level with the estimates produced by the UN and IHME in Figure~\ref{fig:unnamed-chunk-342}. We then adjusted the combined national data to the UN estimates of U5MR, and refit the models on the benchmarked data. 

%%%%%%%%%%%%%%%%%%%%%%%%%% Plot5 
\begin{knitrout}
\definecolor{shadecolor}{rgb}{0.969, 0.969, 0.969}\color{fgcolor}\begin{figure}[bht]

{\centering \includegraphics[width=.9\textwidth]{../Main/Figures/Yearly_national_Zimbabwe} 

}

\caption[Zimbabwe]{Zimbabwe: Temporal national trends along with UN (B3) estimates described in You et al. (2015) and IHME estimates based on GBD 2015 Child Mortality Collaborators (2016). RW2 represents the smoothed national estimates using the original data before benchmarking with UN estimates. RW2-adj represents the smoothed national estimates using the benchmarked data.}\label{fig:unnamed-chunk-342}
\end{figure}


\end{knitrout}
 

We fit the RW2 model to the benchmarked data in each area. 
% The proportions of the explained variation is summarized in Table~\ref{tab:paste0(countryname, "-var")}. 
We compare the results in Figure~\ref{fig:unnamed-chunk-343} to \ref{fig:unnamed-chunk-347}.
Figure~\ref{fig:unnamed-chunk-343} compares the smoothed estimates against the direct estimates. Figure~\ref{fig:unnamed-chunk-344} and Figure~\ref{fig:unnamed-chunk-345} show the posterior median estimates of U5MR in each region over time and the reductions from 1990 period respectively.
Figure~\ref{fig:unnamed-chunk-346} shows the smoothed estimates by region over time and Figure~\ref{fig:unnamed-chunk-347} compares the smoothed estimates with direct estimates from each survey for each region over time.


% %%%%%%%%%%%%%%%%%%%%%%%%%%% Table1 
% <<echo=FALSE, results='asis'>>=
% load("rda/variance_tables.rda")
% countryname2 <- gsub(" ", "", countryname)
% variance <- tables.all[[countryname]]

% table_count <- table_count + 1

% names <- c("RW2 ($\\sigma^2_{\\gamma_{t}}$)", "ICAR ($\\sigma^2_{\\phi_{i}}$)", "IID space ($\\sigma^2_{\\theta_{i}}$)", "IID time ($\\sigma^2_{\\alpha_{t}}$)", "IID space time ($\\sigma^2_{\\delta_{it}}$)")

% variance$Proportion <- round(variance$Proportion*100, digits = 2)
% row.names(variance) <- names
% tab <- xtable(variance, digits = c(1, 3, 2),align = "l|ll",
%        label = paste0("tab:", countryname, "-var"),
%        caption = paste(country, ": summary of the variance components in the RW2 model", sep = ''))
% print(tab, comment = FALSE,sanitize.text.function = function(x) {x})
% @

%%%%%%%%%%%%%%%%%%%%%%%%%%% Plot1 
\begin{knitrout}
\definecolor{shadecolor}{rgb}{0.969, 0.969, 0.969}\color{fgcolor}\begin{figure}[bht]

{\centering \includegraphics[width=.9\textwidth]{../Main/Figures/SmoothvDirectZimbabwe_meta} 

}

\caption[Zimbabwe]{Zimbabwe: Smooth versus direct Admin 1 estimates. Left: Combined (meta-analysis) survey estimate against combined direct estimates. Right: Combined (meta-analysis) survey estimate against direct estimates from each survey.}\label{fig:unnamed-chunk-343}
\end{figure}


\end{knitrout}

%%%%%%%%%%%%%%%%%%%%%%%%%%% Plot2 
\begin{knitrout}
\definecolor{shadecolor}{rgb}{0.969, 0.969, 0.969}\color{fgcolor}\begin{figure}[bht]

{\centering \includegraphics[width=.9\textwidth]{../Main/Figures/SmoothMedianZimbabwe} 

}

\caption[Zimbabwe]{Zimbabwe: Maps of posterior medians over time.}\label{fig:unnamed-chunk-344}
\end{figure}


\end{knitrout}
%%%%%%%%%%%%%%%%%%%%%%%%%%% Plot2a
\begin{knitrout}
\definecolor{shadecolor}{rgb}{0.969, 0.969, 0.969}\color{fgcolor}\begin{figure}[bht]

{\centering \includegraphics[width=.9\textwidth]{../Main/Figures/ReductionMedianZimbabwe} 

}

\caption[Zimbabwe]{Zimbabwe: Maps of reduction of posterior median U5MR in each five-year period compared to 1990 over time.}\label{fig:unnamed-chunk-345}
\end{figure}


\end{knitrout}
%%%%%%%%%%%%%%%%%%%%%%%%%%% Plot3 
\begin{knitrout}
\definecolor{shadecolor}{rgb}{0.969, 0.969, 0.969}\color{fgcolor}\begin{figure}[bht]

{\centering \includegraphics[width=.95\textwidth]{../Main/Figures/Yearly_v_Periods_Zimbabwe} 

}

\caption[Zimbabwe]{Zimbabwe: Smoothed regional estimates over time. The line indicates yearly posterior median estimates and error bars indicate 95 \% posterior credible interval at each time period.}\label{fig:unnamed-chunk-346}
\end{figure}


\end{knitrout}

%%%%%%%%%%%%%%%%%%%%%%%%%%% Plot4 
\begin{knitrout}
\definecolor{shadecolor}{rgb}{0.969, 0.969, 0.969}\color{fgcolor}\begin{figure}[bht]

{\centering \includegraphics[width=.9\textwidth]{../Main/Figures/LineSubMedianZimbabwe} 

}

\caption[Zimbabwe]{Zimbabwe: Smoothed regional estimates over time compared to the direct estimates from each surveys. Direct estimates are not benchmarked with UN estimates. The line indicates posterior median and error bars indicate 95\% posterior credible interval.}\label{fig:unnamed-chunk-347}
\end{figure}


\end{knitrout}
% \subsubsection{National model results}
We further assess the RW2 model by holding out some observations, and compare the projections to the direct estimates in these holdout observations. Figure~\ref{fig:unnamed-chunk-348} compares the predicted estimates for the out-of-sample observations  with the direct estimates by holding out observations from each area in each time period.  Figure~\ref{fig:unnamed-chunk-349} compares the histogram of the bias rescaled by the total variance in the cross validation studies. Figure~\ref{fig:unnamed-chunk-350} compares the rescaled bias by region and time periods.



% %%%%%%%%%%%%%%%%%%%%%%%%%%% Plot6
% << echo=FALSE, out.width = ".9\\textwidth", fig.width = 12, fig.height = 6, fig.cap = "Out-of-sample predictions along with direct estimates in the cross validation study where all data from each time period is held out and predicted using the rest of the data.">>=
% fig_count <- fig_count + 1
% knitr::include_graphics(paste0("../Main/Figures/CV_byYear_withError_", countryname2, ".pdf")) 
% @
 
%%%%%%%%%%%%%%%%%%%%%%%%%%% Plot7
\begin{knitrout}
\definecolor{shadecolor}{rgb}{0.969, 0.969, 0.969}\color{fgcolor}\begin{figure}[bht]

{\centering \includegraphics[width=.9\textwidth]{../Main/Figures/CV_byYearRegion_withError_Zimbabwe} 

}

\caption[Zimbabwe]{Zimbabwe: Out-of-sample predictions along with direct estimates in the cross validation study where data from one region in each time period is held out and predicted using the rest of the data.}\label{fig:unnamed-chunk-348}
\end{figure}


\end{knitrout}

%%%%%%%%%%%%%%%%%%%%%%%%%%% Plot8
\begin{knitrout}
\definecolor{shadecolor}{rgb}{0.969, 0.969, 0.969}\color{fgcolor}\begin{figure}[bht]

{\centering \includegraphics[width=.9\textwidth]{../Main/Figures/CVbiasZimbabwe} 

}

\caption[Zimbabwe]{Zimbabwe: Histogram and QQ-plot of the rescaled difference between the smoothed estimates and the direct estimates in the cross validation study. The differences between the two estimates are rescaled by the square root of the total variance of the two estimates.}\label{fig:unnamed-chunk-349}
\end{figure}


\end{knitrout}

%%%%%%%%%%%%%%%%%%%%%%%%%%% Plot9
\begin{knitrout}
\definecolor{shadecolor}{rgb}{0.969, 0.969, 0.969}\color{fgcolor}\begin{figure}[bht]

{\centering \includegraphics[width=.7\textwidth]{../Main/Figures/CVbiasbyRegionZimbabwe} 

}

\caption[Zimbabwe]{Zimbabwe: Line plot of the difference between smoothed estimates and the direct estimates in the cross validation study. The differences between the two estimates are rescaled by the square root of the total variance of the two estimates.}\label{fig:unnamed-chunk-350}
\end{figure}


\end{knitrout}


% %%%%%%%%%%%%%%%%%%%%%%%%%%%%%%%%%%%%%%%%%%%%%%%%%%%%%%%%%%%%%%%%%%%%%%%%%%%%%%%%%%%%%%%%%%%%%%%%%%
% 
\clearpage
\subsection{Table of All Results: 5-year Periods}
{\scriptsize
% latex table generated in R 3.4.3 by xtable 1.8-2 package
% Mon Apr  2 21:00:33 2018
\begin{longtable}{lllrrrl}
  \hline
Country & Region & Year & Median & Lower & Upper & Method \\ 
  \hline 
\endhead 
\hline 
{\footnotesize Continued on next page} 
\endfoot 
\endlastfoot 
Angola & ALL & 80-84 & 189.53 & 179.33 & 200.89 & IHME \\ 
  Angola & ALL & 80-84 & 251.28 & 192.05 & 322.16 & RW2 \\ 
  Angola & ALL & 80-84 & 232.95 & 214.36 & 255.00 & UN \\ 
  Angola & ALL & 85-89 & 187.44 & 179.24 & 196.32 & IHME \\ 
  Angola & ALL & 85-89 & 222.56 & 189.53 & 258.44 & RW2 \\ 
  Angola & ALL & 85-89 & 227.72 & 214.73 & 241.69 & UN \\ 
  Angola & ALL & 90-94 & 182.79 & 175.12 & 191.15 & IHME \\ 
  Angola & ALL & 90-94 & 226.02 & 200.25 & 254.31 & RW2 \\ 
  Angola & ALL & 90-94 & 227.00 & 215.43 & 238.75 & UN \\ 
  Angola & ALL & 95-99 & 161.43 & 154.52 & 168.39 & IHME \\ 
  Angola & ALL & 95-99 & 224.40 & 202.98 & 247.22 & RW2 \\ 
  Angola & ALL & 95-99 & 223.29 & 211.97 & 235.65 & UN \\ 
  Angola & ALL & 00-04 & 144.35 & 138.13 & 151.24 & IHME \\ 
  Angola & ALL & 00-04 & 212.56 & 185.32 & 242.76 & RW2 \\ 
  Angola & ALL & 00-04 & 212.39 & 198.39 & 226.68 & UN \\ 
  Angola & ALL & 05-09 & 113.54 & 107.89 & 119.83 & IHME \\ 
  Angola & ALL & 05-09 & 194.23 & 133.86 & 273.39 & RW2 \\ 
  Angola & ALL & 05-09 & 195.23 & 177.10 & 218.26 & UN \\ 
  Angola & ALL & 10-14 & 91.11 & 83.25 & 99.61 & IHME \\ 
  Angola & ALL & 10-14 & 173.94 & 150.46 & 200.03 & RW2 \\ 
  Angola & ALL & 10-14 & 173.83 & 144.34 & 207.56 & UN \\ 
  Angola & BENGO & 80-84 & 109.42 & 28.82 & 341.04 & RW2 \\ 
  Angola & BENGO & 85-89 & 98.46 & 430.20 & 15.55 & HT-Direct \\ 
  Angola & BENGO & 85-89 & 98.18 & 37.18 & 232.16 & RW2 \\ 
  Angola & BENGO & 90-94 & 37.81 & 145.36 & 9.00 & HT-Direct \\ 
  Angola & BENGO & 90-94 & 107.18 & 54.88 & 196.73 & RW2 \\ 
  Angola & BENGO & 95-99 & 99.96 & 213.76 & 43.40 & HT-Direct \\ 
  Angola & BENGO & 95-99 & 110.33 & 68.70 & 170.86 & RW2 \\ 
  Angola & BENGO & 00-04 & 131.53 & 220.77 & 74.90 & HT-Direct \\ 
  Angola & BENGO & 00-04 & 121.38 & 77.40 & 184.04 & RW2 \\ 
  Angola & BENGO & 05-09 & 66.52 & 113.31 & 38.22 & HT-Direct \\ 
  Angola & BENGO & 05-09 & 120.34 & 61.94 & 221.01 & RW2 \\ 
  Angola & BENGO & 10-14 & 28.58 & 52.50 & 15.39 & HT-Direct \\ 
  Angola & BENGO & 10-14 & 109.96 & 48.26 & 231.45 & RW2 \\ 
  Angola & BENGO & 15-19 & 99.26 & 19.43 & 382.74 & RW2 \\ 
  Angola & BENGUELA & 80-84 & 528.44 & 701.44 & 348.32 & HT-Direct \\ 
  Angola & BENGUELA & 80-84 & 382.03 & 295.80 & 472.34 & RW2 \\ 
  Angola & BENGUELA & 85-89 & 472.84 & 609.98 & 339.67 & HT-Direct \\ 
  Angola & BENGUELA & 85-89 & 371.05 & 308.05 & 437.98 & RW2 \\ 
  Angola & BENGUELA & 90-94 & 382.04 & 456.77 & 312.51 & HT-Direct \\ 
  Angola & BENGUELA & 90-94 & 399.95 & 350.79 & 452.10 & RW2 \\ 
  Angola & BENGUELA & 95-99 & 396.29 & 459.02 & 336.80 & HT-Direct \\ 
  Angola & BENGUELA & 95-99 & 391.60 & 345.92 & 441.97 & RW2 \\ 
  Angola & BENGUELA & 00-04 & 284.90 & 347.64 & 229.50 & HT-Direct \\ 
  Angola & BENGUELA & 00-04 & 388.81 & 331.42 & 450.89 & RW2 \\ 
  Angola & BENGUELA & 05-09 & 173.61 & 214.51 & 139.13 & HT-Direct \\ 
  Angola & BENGUELA & 05-09 & 356.35 & 259.79 & 466.03 & RW2 \\ 
  Angola & BENGUELA & 10-14 & 117.95 & 149.51 & 92.34 & HT-Direct \\ 
  Angola & BENGUELA & 10-14 & 306.23 & 239.98 & 380.60 & RW2 \\ 
  Angola & BENGUELA & 15-19 & 256.79 & 87.07 & 554.63 & RW2 \\ 
  Angola & BIÉ & 80-84 & 530.35 & 747.96 & 300.55 & HT-Direct \\ 
  Angola & BIÉ & 80-84 & 207.78 & 140.72 & 302.01 & RW2 \\ 
  Angola & BIÉ & 85-89 & 104.44 & 232.42 & 42.98 & HT-Direct \\ 
  Angola & BIÉ & 85-89 & 179.37 & 131.64 & 239.28 & RW2 \\ 
  Angola & BIÉ & 90-94 & 145.09 & 247.55 & 80.50 & HT-Direct \\ 
  Angola & BIÉ & 90-94 & 185.19 & 147.45 & 229.35 & RW2 \\ 
  Angola & BIÉ & 95-99 & 150.46 & 183.96 & 122.14 & HT-Direct \\ 
  Angola & BIÉ & 95-99 & 179.43 & 150.51 & 211.60 & RW2 \\ 
  Angola & BIÉ & 00-04 & 171.01 & 217.76 & 132.60 & HT-Direct \\ 
  Angola & BIÉ & 00-04 & 188.75 & 153.86 & 228.77 & RW2 \\ 
  Angola & BIÉ & 05-09 & 88.68 & 116.56 & 66.97 & HT-Direct \\ 
  Angola & BIÉ & 05-09 & 187.08 & 127.22 & 266.30 & RW2 \\ 
  Angola & BIÉ & 10-14 & 76.05 & 98.32 & 58.50 & HT-Direct \\ 
  Angola & BIÉ & 10-14 & 176.26 & 129.69 & 236.48 & RW2 \\ 
  Angola & BIÉ & 15-19 & 164.50 & 51.31 & 419.47 & RW2 \\ 
  Angola & CABINDA & 80-84 & 99.11 & 32.41 & 280.32 & RW2 \\ 
  Angola & CABINDA & 85-89 & 38.54 & 228.39 & 5.40 & HT-Direct \\ 
  Angola & CABINDA & 85-89 & 87.35 & 39.54 & 185.43 & RW2 \\ 
  Angola & CABINDA & 90-94 & 150.85 & 295.21 & 70.06 & HT-Direct \\ 
  Angola & CABINDA & 90-94 & 92.79 & 53.96 & 154.23 & RW2 \\ 
  Angola & CABINDA & 95-99 & 49.46 & 96.83 & 24.64 & HT-Direct \\ 
  Angola & CABINDA & 95-99 & 89.74 & 59.41 & 132.96 & RW2 \\ 
  Angola & CABINDA & 00-04 & 54.12 & 94.76 & 30.32 & HT-Direct \\ 
  Angola & CABINDA & 00-04 & 94.76 & 62.66 & 138.82 & RW2 \\ 
  Angola & CABINDA & 05-09 & 65.27 & 112.63 & 37.00 & HT-Direct \\ 
  Angola & CABINDA & 05-09 & 92.97 & 51.48 & 161.67 & RW2 \\ 
  Angola & CABINDA & 10-14 & 34.72 & 54.72 & 21.85 & HT-Direct \\ 
  Angola & CABINDA & 10-14 & 86.45 & 44.88 & 159.87 & RW2 \\ 
  Angola & CABINDA & 15-19 & 79.89 & 18.84 & 287.17 & RW2 \\ 
  Angola & CUANDO CUBANGO & 80-84 & 117.10 & 34.29 & 333.57 & RW2 \\ 
  Angola & CUANDO CUBANGO & 85-89 & 114.77 & 46.62 & 256.40 & RW2 \\ 
  Angola & CUANDO CUBANGO & 90-94 & 79.09 & 200.42 & 28.58 & HT-Direct \\ 
  Angola & CUANDO CUBANGO & 90-94 & 134.36 & 73.13 & 232.00 & RW2 \\ 
  Angola & CUANDO CUBANGO & 95-99 & 160.05 & 274.70 & 87.48 & HT-Direct \\ 
  Angola & CUANDO CUBANGO & 95-99 & 142.79 & 92.50 & 211.75 & RW2 \\ 
  Angola & CUANDO CUBANGO & 00-04 & 127.93 & 227.75 & 68.00 & HT-Direct \\ 
  Angola & CUANDO CUBANGO & 00-04 & 160.56 & 111.27 & 225.42 & RW2 \\ 
  Angola & CUANDO CUBANGO & 05-09 & 93.82 & 141.57 & 61.03 & HT-Direct \\ 
  Angola & CUANDO CUBANGO & 05-09 & 166.30 & 100.17 & 262.85 & RW2 \\ 
  Angola & CUANDO CUBANGO & 10-14 & 58.76 & 85.86 & 39.84 & HT-Direct \\ 
  Angola & CUANDO CUBANGO & 10-14 & 161.09 & 96.22 & 258.20 & RW2 \\ 
  Angola & CUANDO CUBANGO & 15-19 & 154.35 & 41.23 & 437.49 & RW2 \\ 
  Angola & CUANZA NORTE & 80-84 & 192.10 & 79.26 & 390.69 & RW2 \\ 
  Angola & CUANZA NORTE & 85-89 & 34.74 & 198.72 & 5.20 & HT-Direct \\ 
  Angola & CUANZA NORTE & 85-89 & 178.63 & 97.94 & 303.21 & RW2 \\ 
  Angola & CUANZA NORTE & 90-94 & 208.84 & 329.17 & 124.34 & HT-Direct \\ 
  Angola & CUANZA NORTE & 90-94 & 199.01 & 134.75 & 283.23 & RW2 \\ 
  Angola & CUANZA NORTE & 95-99 & 243.31 & 376.80 & 146.04 & HT-Direct \\ 
  Angola & CUANZA NORTE & 95-99 & 203.92 & 151.92 & 269.34 & RW2 \\ 
  Angola & CUANZA NORTE & 00-04 & 210.17 & 304.41 & 139.26 & HT-Direct \\ 
  Angola & CUANZA NORTE & 00-04 & 218.26 & 161.94 & 290.73 & RW2 \\ 
  Angola & CUANZA NORTE & 05-09 & 115.65 & 147.23 & 90.12 & HT-Direct \\ 
  Angola & CUANZA NORTE & 05-09 & 212.26 & 132.84 & 324.83 & RW2 \\ 
  Angola & CUANZA NORTE & 10-14 & 64.68 & 96.87 & 42.69 & HT-Direct \\ 
  Angola & CUANZA NORTE & 10-14 & 192.49 & 116.23 & 300.44 & RW2 \\ 
  Angola & CUANZA NORTE & 15-19 & 170.96 & 46.61 & 466.61 & RW2 \\ 
  Angola & CUANZA SUL & 80-84 & 238.79 & 689.39 & 42.46 & HT-Direct \\ 
  Angola & CUANZA SUL & 80-84 & 223.42 & 139.32 & 338.52 & RW2 \\ 
  Angola & CUANZA SUL & 85-89 & 263.21 & 392.14 & 165.15 & HT-Direct \\ 
  Angola & CUANZA SUL & 85-89 & 217.14 & 160.52 & 286.99 & RW2 \\ 
  Angola & CUANZA SUL & 90-94 & 268.58 & 376.47 & 182.56 & HT-Direct \\ 
  Angola & CUANZA SUL & 90-94 & 244.80 & 197.98 & 298.81 & RW2 \\ 
  Angola & CUANZA SUL & 95-99 & 225.53 & 285.73 & 174.91 & HT-Direct \\ 
  Angola & CUANZA SUL & 95-99 & 253.65 & 211.66 & 300.30 & RW2 \\ 
  Angola & CUANZA SUL & 00-04 & 201.75 & 284.21 & 138.59 & HT-Direct \\ 
  Angola & CUANZA SUL & 00-04 & 275.07 & 220.81 & 337.17 & RW2 \\ 
  Angola & CUANZA SUL & 05-09 & 173.43 & 207.23 & 144.15 & HT-Direct \\ 
  Angola & CUANZA SUL & 05-09 & 277.19 & 188.37 & 388.35 & RW2 \\ 
  Angola & CUANZA SUL & 10-14 & 111.11 & 154.93 & 78.53 & HT-Direct \\ 
  Angola & CUANZA SUL & 10-14 & 264.54 & 187.38 & 359.78 & RW2 \\ 
  Angola & CUANZA SUL & 15-19 & 248.55 & 80.73 & 557.06 & RW2 \\ 
  Angola & CUNENE & 80-84 & 114.44 & 46.21 & 244.93 & RW2 \\ 
  Angola & CUNENE & 85-89 & 26.46 & 158.80 & 3.90 & HT-Direct \\ 
  Angola & CUNENE & 85-89 & 119.82 & 68.96 & 196.42 & RW2 \\ 
  Angola & CUNENE & 90-94 & 162.50 & 228.99 & 112.50 & HT-Direct \\ 
  Angola & CUNENE & 90-94 & 147.94 & 110.37 & 195.19 & RW2 \\ 
  Angola & CUNENE & 95-99 & 139.97 & 182.02 & 106.36 & HT-Direct \\ 
  Angola & CUNENE & 95-99 & 161.94 & 131.41 & 197.84 & RW2 \\ 
  Angola & CUNENE & 00-04 & 178.73 & 242.58 & 128.83 & HT-Direct \\ 
  Angola & CUNENE & 00-04 & 183.52 & 140.49 & 237.25 & RW2 \\ 
  Angola & CUNENE & 05-09 & 100.54 & 153.70 & 64.36 & HT-Direct \\ 
  Angola & CUNENE & 05-09 & 188.07 & 117.10 & 288.73 & RW2 \\ 
  Angola & CUNENE & 10-14 & 59.55 & 89.76 & 39.07 & HT-Direct \\ 
  Angola & CUNENE & 10-14 & 179.08 & 108.03 & 282.03 & RW2 \\ 
  Angola & CUNENE & 15-19 & 167.24 & 44.99 & 460.88 & RW2 \\ 
  Angola & HUAMBO & 80-84 & 420.65 & 626.69 & 238.98 & HT-Direct \\ 
  Angola & HUAMBO & 80-84 & 301.64 & 221.49 & 394.39 & RW2 \\ 
  Angola & HUAMBO & 85-89 & 297.44 & 430.74 & 191.52 & HT-Direct \\ 
  Angola & HUAMBO & 85-89 & 276.24 & 218.44 & 342.88 & RW2 \\ 
  Angola & HUAMBO & 90-94 & 338.47 & 441.04 & 249.11 & HT-Direct \\ 
  Angola & HUAMBO & 90-94 & 287.52 & 238.45 & 342.94 & RW2 \\ 
  Angola & HUAMBO & 95-99 & 229.27 & 301.63 & 170.05 & HT-Direct \\ 
  Angola & HUAMBO & 95-99 & 270.72 & 225.77 & 322.69 & RW2 \\ 
  Angola & HUAMBO & 00-04 & 225.22 & 296.42 & 167.07 & HT-Direct \\ 
  Angola & HUAMBO & 00-04 & 263.53 & 211.78 & 323.65 & RW2 \\ 
  Angola & HUAMBO & 05-09 & 132.93 & 179.51 & 97.01 & HT-Direct \\ 
  Angola & HUAMBO & 05-09 & 236.06 & 160.68 & 333.28 & RW2 \\ 
  Angola & HUAMBO & 10-14 & 73.48 & 96.21 & 55.78 & HT-Direct \\ 
  Angola & HUAMBO & 10-14 & 197.53 & 143.12 & 265.31 & RW2 \\ 
  Angola & HUAMBO & 15-19 & 161.33 & 49.39 & 412.12 & RW2 \\ 
  Angola & HUÍLA & 80-84 & 489.05 & 924.26 & 69.83 & HT-Direct \\ 
  Angola & HUÍLA & 80-84 & 246.63 & 156.09 & 369.85 & RW2 \\ 
  Angola & HUÍLA & 85-89 & 268.10 & 401.89 & 166.45 & HT-Direct \\ 
  Angola & HUÍLA & 85-89 & 233.35 & 174.04 & 305.37 & RW2 \\ 
  Angola & HUÍLA & 90-94 & 248.32 & 328.09 & 182.68 & HT-Direct \\ 
  Angola & HUÍLA & 90-94 & 255.43 & 209.47 & 307.91 & RW2 \\ 
  Angola & HUÍLA & 95-99 & 247.66 & 322.29 & 185.58 & HT-Direct \\ 
  Angola & HUÍLA & 95-99 & 254.78 & 214.39 & 299.72 & RW2 \\ 
  Angola & HUÍLA & 00-04 & 185.56 & 228.04 & 149.46 & HT-Direct \\ 
  Angola & HUÍLA & 00-04 & 265.88 & 219.65 & 317.35 & RW2 \\ 
  Angola & HUÍLA & 05-09 & 140.70 & 182.53 & 107.20 & HT-Direct \\ 
  Angola & HUÍLA & 05-09 & 257.80 & 179.81 & 356.02 & RW2 \\ 
  Angola & HUÍLA & 10-14 & 99.42 & 129.59 & 75.66 & HT-Direct \\ 
  Angola & HUÍLA & 10-14 & 236.71 & 177.01 & 309.04 & RW2 \\ 
  Angola & HUÍLA & 15-19 & 214.04 & 69.37 & 498.84 & RW2 \\ 
  Angola & LUANDA & 80-84 & 336.21 & 625.39 & 133.20 & HT-Direct \\ 
  Angola & LUANDA & 80-84 & 191.71 & 117.26 & 294.76 & RW2 \\ 
  Angola & LUANDA & 85-89 & 129.35 & 262.95 & 58.27 & HT-Direct \\ 
  Angola & LUANDA & 85-89 & 164.54 & 115.98 & 226.65 & RW2 \\ 
  Angola & LUANDA & 90-94 & 145.78 & 219.95 & 93.62 & HT-Direct \\ 
  Angola & LUANDA & 90-94 & 169.20 & 132.42 & 213.09 & RW2 \\ 
  Angola & LUANDA & 95-99 & 170.36 & 217.74 & 131.55 & HT-Direct \\ 
  Angola & LUANDA & 95-99 & 160.13 & 131.39 & 194.23 & RW2 \\ 
  Angola & LUANDA & 00-04 & 111.93 & 144.81 & 85.76 & HT-Direct \\ 
  Angola & LUANDA & 00-04 & 157.62 & 123.93 & 199.09 & RW2 \\ 
  Angola & LUANDA & 05-09 & 73.93 & 103.88 & 52.11 & HT-Direct \\ 
  Angola & LUANDA & 05-09 & 139.66 & 88.33 & 213.93 & RW2 \\ 
  Angola & LUANDA & 10-14 & 40.39 & 59.59 & 27.20 & HT-Direct \\ 
  Angola & LUANDA & 10-14 & 114.17 & 70.24 & 179.05 & RW2 \\ 
  Angola & LUANDA & 15-19 & 91.90 & 23.73 & 293.55 & RW2 \\ 
  Angola & LUNDA NORTE & 80-84 & 193.46 & 83.79 & 387.91 & RW2 \\ 
  Angola & LUNDA NORTE & 85-89 & 171.61 & 351.28 & 73.44 & HT-Direct \\ 
  Angola & LUNDA NORTE & 85-89 & 173.83 & 103.34 & 278.07 & RW2 \\ 
  Angola & LUNDA NORTE & 90-94 & 211.78 & 348.48 & 118.92 & HT-Direct \\ 
  Angola & LUNDA NORTE & 90-94 & 183.91 & 129.87 & 253.73 & RW2 \\ 
  Angola & LUNDA NORTE & 95-99 & 167.23 & 244.71 & 110.68 & HT-Direct \\ 
  Angola & LUNDA NORTE & 95-99 & 176.81 & 132.62 & 231.84 & RW2 \\ 
  Angola & LUNDA NORTE & 00-04 & 115.43 & 185.27 & 69.66 & HT-Direct \\ 
  Angola & LUNDA NORTE & 00-04 & 180.85 & 129.36 & 245.83 & RW2 \\ 
  Angola & LUNDA NORTE & 05-09 & 62.90 & 107.24 & 36.15 & HT-Direct \\ 
  Angola & LUNDA NORTE & 05-09 & 173.75 & 104.84 & 273.15 & RW2 \\ 
  Angola & LUNDA NORTE & 10-14 & 66.14 & 99.90 & 43.25 & HT-Direct \\ 
  Angola & LUNDA NORTE & 10-14 & 158.73 & 93.56 & 256.28 & RW2 \\ 
  Angola & LUNDA NORTE & 15-19 & 143.19 & 37.82 & 423.32 & RW2 \\ 
  Angola & LUNDA SUL & 80-84 & 109.20 & 353.84 & 26.71 & HT-Direct \\ 
  Angola & LUNDA SUL & 80-84 & 108.90 & 53.92 & 204.54 & RW2 \\ 
  Angola & LUNDA SUL & 85-89 & 15.23 & 105.87 & 2.02 & HT-Direct \\ 
  Angola & LUNDA SUL & 85-89 & 105.99 & 63.36 & 171.73 & RW2 \\ 
  Angola & LUNDA SUL & 90-94 & 193.83 & 321.41 & 108.77 & HT-Direct \\ 
  Angola & LUNDA SUL & 90-94 & 123.01 & 84.78 & 174.77 & RW2 \\ 
  Angola & LUNDA SUL & 95-99 & 143.73 & 241.71 & 81.21 & HT-Direct \\ 
  Angola & LUNDA SUL & 95-99 & 127.56 & 93.35 & 172.49 & RW2 \\ 
  Angola & LUNDA SUL & 00-04 & 82.35 & 123.86 & 53.90 & HT-Direct \\ 
  Angola & LUNDA SUL & 00-04 & 139.75 & 101.44 & 189.40 & RW2 \\ 
  Angola & LUNDA SUL & 05-09 & 46.97 & 70.00 & 31.26 & HT-Direct \\ 
  Angola & LUNDA SUL & 05-09 & 141.73 & 86.66 & 222.70 & RW2 \\ 
  Angola & LUNDA SUL & 10-14 & 48.19 & 70.85 & 32.53 & HT-Direct \\ 
  Angola & LUNDA SUL & 10-14 & 136.08 & 82.95 & 215.43 & RW2 \\ 
  Angola & LUNDA SUL & 15-19 & 129.21 & 34.99 & 379.73 & RW2 \\ 
  Angola & MALANJE & 80-84 & 579.93 & 986.73 & 25.00 & HT-Direct \\ 
  Angola & MALANJE & 80-84 & 264.83 & 131.24 & 468.08 & RW2 \\ 
  Angola & MALANJE & 85-89 & 274.02 & 660.60 & 68.20 & HT-Direct \\ 
  Angola & MALANJE & 85-89 & 216.82 & 129.29 & 342.15 & RW2 \\ 
  Angola & MALANJE & 90-94 & 198.51 & 336.01 & 108.12 & HT-Direct \\ 
  Angola & MALANJE & 90-94 & 208.45 & 144.93 & 289.43 & RW2 \\ 
  Angola & MALANJE & 95-99 & 168.71 & 266.33 & 101.90 & HT-Direct \\ 
  Angola & MALANJE & 95-99 & 185.66 & 139.92 & 241.23 & RW2 \\ 
  Angola & MALANJE & 00-04 & 125.26 & 171.96 & 89.86 & HT-Direct \\ 
  Angola & MALANJE & 00-04 & 177.25 & 135.58 & 227.22 & RW2 \\ 
  Angola & MALANJE & 05-09 & 64.34 & 102.49 & 39.76 & HT-Direct \\ 
  Angola & MALANJE & 05-09 & 157.73 & 101.88 & 236.56 & RW2 \\ 
  Angola & MALANJE & 10-14 & 54.22 & 73.86 & 39.57 & HT-Direct \\ 
  Angola & MALANJE & 10-14 & 132.66 & 88.06 & 195.04 & RW2 \\ 
  Angola & MALANJE & 15-19 & 109.37 & 30.90 & 320.51 & RW2 \\ 
  Angola & MOXICO & 80-84 & 420.82 & 22.26 & 951.40 & RW2 \\ 
  Angola & MOXICO & 85-89 & 284.96 & 25.53 & 835.53 & RW2 \\ 
  Angola & MOXICO & 90-94 & 214.87 & 35.74 & 631.55 & RW2 \\ 
  Angola & MOXICO & 95-99 & 144.37 & 41.52 & 361.23 & RW2 \\ 
  Angola & MOXICO & 00-04 & 8.73 & 56.86 & 1.29 & HT-Direct \\ 
  Angola & MOXICO & 00-04 & 100.98 & 41.93 & 210.40 & RW2 \\ 
  Angola & MOXICO & 05-09 & 1.33 & 9.89 & 0.18 & HT-Direct \\ 
  Angola & MOXICO & 05-09 & 64.80 & 22.69 & 170.38 & RW2 \\ 
  Angola & MOXICO & 10-14 & 18.10 & 44.66 & 7.22 & HT-Direct \\ 
  Angola & MOXICO & 10-14 & 39.12 & 8.51 & 163.55 & RW2 \\ 
  Angola & MOXICO & 15-19 & 23.17 & 1.88 & 228.51 & RW2 \\ 
  Angola & NAMIBE & 80-84 & 291.47 & 586.81 & 106.47 & HT-Direct \\ 
  Angola & NAMIBE & 80-84 & 187.91 & 112.15 & 293.95 & RW2 \\ 
  Angola & NAMIBE & 85-89 & 150.73 & 320.17 & 62.69 & HT-Direct \\ 
  Angola & NAMIBE & 85-89 & 187.73 & 131.24 & 261.06 & RW2 \\ 
  Angola & NAMIBE & 90-94 & 223.44 & 324.81 & 146.82 & HT-Direct \\ 
  Angola & NAMIBE & 90-94 & 217.18 & 168.81 & 275.41 & RW2 \\ 
  Angola & NAMIBE & 95-99 & 218.53 & 294.91 & 157.51 & HT-Direct \\ 
  Angola & NAMIBE & 95-99 & 223.17 & 178.55 & 276.85 & RW2 \\ 
  Angola & NAMIBE & 00-04 & 183.89 & 259.15 & 126.74 & HT-Direct \\ 
  Angola & NAMIBE & 00-04 & 234.44 & 180.41 & 299.39 & RW2 \\ 
  Angola & NAMIBE & 05-09 & 132.11 & 187.89 & 91.03 & HT-Direct \\ 
  Angola & NAMIBE & 05-09 & 222.75 & 145.23 & 325.79 & RW2 \\ 
  Angola & NAMIBE & 10-14 & 71.44 & 100.25 & 50.44 & HT-Direct \\ 
  Angola & NAMIBE & 10-14 & 197.56 & 131.06 & 284.00 & RW2 \\ 
  Angola & NAMIBE & 15-19 & 170.59 & 49.46 & 448.96 & RW2 \\ 
  Angola & UÍGE & 80-84 & 694.11 & 878.49 & 415.95 & HT-Direct \\ 
  Angola & UÍGE & 80-84 & 390.28 & 277.44 & 524.09 & RW2 \\ 
  Angola & UÍGE & 85-89 & 431.39 & 609.64 & 269.31 & HT-Direct \\ 
  Angola & UÍGE & 85-89 & 293.74 & 222.37 & 376.10 & RW2 \\ 
  Angola & UÍGE & 90-94 & 182.55 & 285.37 & 111.02 & HT-Direct \\ 
  Angola & UÍGE & 90-94 & 254.00 & 195.77 & 318.83 & RW2 \\ 
  Angola & UÍGE & 95-99 & 157.17 & 237.29 & 100.54 & HT-Direct \\ 
  Angola & UÍGE & 95-99 & 210.63 & 162.59 & 263.11 & RW2 \\ 
  Angola & UÍGE & 00-04 & 155.05 & 205.94 & 114.91 & HT-Direct \\ 
  Angola & UÍGE & 00-04 & 190.29 & 146.06 & 242.81 & RW2 \\ 
  Angola & UÍGE & 05-09 & 100.91 & 150.54 & 66.36 & HT-Direct \\ 
  Angola & UÍGE & 05-09 & 160.43 & 103.89 & 238.66 & RW2 \\ 
  Angola & UÍGE & 10-14 & 58.50 & 78.62 & 43.29 & HT-Direct \\ 
  Angola & UÍGE & 10-14 & 126.76 & 86.46 & 183.65 & RW2 \\ 
  Angola & UÍGE & 15-19 & 97.68 & 27.61 & 292.41 & RW2 \\ 
  Angola & ZAIRE & 80-84 & 73.36 & 27.47 & 191.95 & RW2 \\ 
  Angola & ZAIRE & 85-89 & 76.72 & 204.35 & 26.18 & HT-Direct \\ 
  Angola & ZAIRE & 85-89 & 70.64 & 37.55 & 130.03 & RW2 \\ 
  Angola & ZAIRE & 90-94 & 67.85 & 134.50 & 32.97 & HT-Direct \\ 
  Angola & ZAIRE & 90-94 & 83.32 & 54.91 & 124.25 & RW2 \\ 
  Angola & ZAIRE & 95-99 & 85.15 & 130.15 & 54.73 & HT-Direct \\ 
  Angola & ZAIRE & 95-99 & 93.36 & 67.63 & 127.21 & RW2 \\ 
  Angola & ZAIRE & 00-04 & 111.82 & 175.54 & 69.29 & HT-Direct \\ 
  Angola & ZAIRE & 00-04 & 113.69 & 80.46 & 159.50 & RW2 \\ 
  Angola & ZAIRE & 05-09 & 78.01 & 128.45 & 46.32 & HT-Direct \\ 
  Angola & ZAIRE & 05-09 & 124.74 & 74.57 & 202.46 & RW2 \\ 
  Angola & ZAIRE & 10-14 & 44.31 & 64.82 & 30.08 & HT-Direct \\ 
  Angola & ZAIRE & 10-14 & 126.81 & 75.20 & 205.41 & RW2 \\ 
  Angola & ZAIRE & 15-19 & 126.76 & 32.44 & 383.78 & RW2 \\ 
  Benin & ALL & 80-84 & 213.87 & 210.96 & 216.86 & IHME \\ 
  Benin & ALL & 80-84 & 210.20 & 200.64 & 220.10 & RW2 \\ 
  Benin & ALL & 80-84 & 210.24 & 204.33 & 216.73 & UN \\ 
  Benin & ALL & 85-89 & 191.79 & 189.49 & 194.29 & IHME \\ 
  Benin & ALL & 85-89 & 192.37 & 184.32 & 200.65 & RW2 \\ 
  Benin & ALL & 85-89 & 192.27 & 186.66 & 198.04 & UN \\ 
  Benin & ALL & 90-94 & 172.46 & 170.20 & 174.63 & IHME \\ 
  Benin & ALL & 90-94 & 170.36 & 163.93 & 177.00 & RW2 \\ 
  Benin & ALL & 90-94 & 170.47 & 165.82 & 175.17 & UN \\ 
  Benin & ALL & 95-99 & 152.56 & 150.34 & 154.78 & IHME \\ 
  Benin & ALL & 95-99 & 153.61 & 147.78 & 159.58 & RW2 \\ 
  Benin & ALL & 95-99 & 153.58 & 149.23 & 158.09 & UN \\ 
  Benin & ALL & 00-04 & 130.49 & 128.21 & 132.69 & IHME \\ 
  Benin & ALL & 00-04 & 136.99 & 130.31 & 143.99 & RW2 \\ 
  Benin & ALL & 00-04 & 136.92 & 132.18 & 141.28 & UN \\ 
  Benin & ALL & 05-09 & 107.75 & 105.25 & 110.37 & IHME \\ 
  Benin & ALL & 05-09 & 118.12 & 103.83 & 134.08 & RW2 \\ 
  Benin & ALL & 05-09 & 118.35 & 113.03 & 124.29 & UN \\ 
  Benin & ALL & 10-14 & 88.79 & 85.44 & 92.52 & IHME \\ 
  Benin & ALL & 10-14 & 100.31 & 38.17 & 235.56 & RW2 \\ 
  Benin & ALL & 10-14 & 106.29 & 98.52 & 115.37 & UN \\ 
  Benin & ATACORA & 80-84 & 290.00 & 315.66 & 265.61 & HT-Direct \\ 
  Benin & ATACORA & 80-84 & 269.43 & 250.07 & 289.97 & RW2 \\ 
  Benin & ATACORA & 85-89 & 237.50 & 256.40 & 219.59 & HT-Direct \\ 
  Benin & ATACORA & 85-89 & 239.27 & 225.75 & 252.98 & RW2 \\ 
  Benin & ATACORA & 90-94 & 205.36 & 222.70 & 189.04 & HT-Direct \\ 
  Benin & ATACORA & 90-94 & 211.46 & 199.56 & 223.69 & RW2 \\ 
  Benin & ATACORA & 95-99 & 202.99 & 219.10 & 187.77 & HT-Direct \\ 
  Benin & ATACORA & 95-99 & 192.08 & 181.34 & 203.35 & RW2 \\ 
  Benin & ATACORA & 00-04 & 170.83 & 190.14 & 153.11 & HT-Direct \\ 
  Benin & ATACORA & 00-04 & 173.59 & 159.91 & 188.37 & RW2 \\ 
  Benin & ATACORA & 05-09 & 139.52 & 173.80 & 111.09 & HT-Direct \\ 
  Benin & ATACORA & 05-09 & 150.79 & 126.74 & 178.95 & RW2 \\ 
  Benin & ATACORA & 10-14 & 128.39 & 54.04 & 274.00 & RW2 \\ 
  Benin & ATACORA & 15-19 & 108.71 & 11.83 & 546.08 & RW2 \\ 
  Benin & ATLANTIQUE & 80-84 & 179.55 & 201.53 & 159.48 & HT-Direct \\ 
  Benin & ATLANTIQUE & 80-84 & 171.63 & 155.85 & 188.66 & RW2 \\ 
  Benin & ATLANTIQUE & 85-89 & 163.21 & 183.00 & 145.19 & HT-Direct \\ 
  Benin & ATLANTIQUE & 85-89 & 153.80 & 142.83 & 165.58 & RW2 \\ 
  Benin & ATLANTIQUE & 90-94 & 126.24 & 139.82 & 113.80 & HT-Direct \\ 
  Benin & ATLANTIQUE & 90-94 & 134.31 & 125.47 & 143.42 & RW2 \\ 
  Benin & ATLANTIQUE & 95-99 & 124.64 & 136.40 & 113.77 & HT-Direct \\ 
  Benin & ATLANTIQUE & 95-99 & 121.24 & 113.57 & 129.31 & RW2 \\ 
  Benin & ATLANTIQUE & 00-04 & 115.02 & 128.93 & 102.43 & HT-Direct \\ 
  Benin & ATLANTIQUE & 00-04 & 111.03 & 101.64 & 121.36 & RW2 \\ 
  Benin & ATLANTIQUE & 05-09 & 80.50 & 104.68 & 61.52 & HT-Direct \\ 
  Benin & ATLANTIQUE & 05-09 & 98.12 & 81.56 & 118.03 & RW2 \\ 
  Benin & ATLANTIQUE & 10-14 & 85.53 & 35.34 & 193.84 & RW2 \\ 
  Benin & ATLANTIQUE & 15-19 & 73.90 & 8.03 & 435.52 & RW2 \\ 
  Benin & BORGOU & 80-84 & 213.43 & 239.38 & 189.59 & HT-Direct \\ 
  Benin & BORGOU & 80-84 & 207.22 & 187.95 & 227.75 & RW2 \\ 
  Benin & BORGOU & 85-89 & 194.27 & 216.02 & 174.23 & HT-Direct \\ 
  Benin & BORGOU & 85-89 & 195.00 & 181.62 & 209.18 & RW2 \\ 
  Benin & BORGOU & 90-94 & 184.89 & 203.83 & 167.33 & HT-Direct \\ 
  Benin & BORGOU & 90-94 & 179.74 & 168.23 & 192.27 & RW2 \\ 
  Benin & BORGOU & 95-99 & 174.96 & 191.97 & 159.17 & HT-Direct \\ 
  Benin & BORGOU & 95-99 & 166.02 & 155.09 & 177.78 & RW2 \\ 
  Benin & BORGOU & 00-04 & 141.96 & 163.68 & 122.70 & HT-Direct \\ 
  Benin & BORGOU & 00-04 & 150.40 & 135.61 & 166.29 & RW2 \\ 
  Benin & BORGOU & 05-09 & 110.61 & 145.48 & 83.29 & HT-Direct \\ 
  Benin & BORGOU & 05-09 & 130.48 & 105.74 & 158.21 & RW2 \\ 
  Benin & BORGOU & 10-14 & 111.02 & 45.50 & 243.60 & RW2 \\ 
  Benin & BORGOU & 15-19 & 93.29 & 10.08 & 505.17 & RW2 \\ 
  Benin & MONO & 80-84 & 204.65 & 231.64 & 180.07 & HT-Direct \\ 
  Benin & MONO & 80-84 & 200.29 & 180.72 & 221.16 & RW2 \\ 
  Benin & MONO & 85-89 & 182.97 & 203.22 & 164.33 & HT-Direct \\ 
  Benin & MONO & 85-89 & 178.87 & 166.30 & 192.17 & RW2 \\ 
  Benin & MONO & 90-94 & 155.61 & 170.84 & 141.52 & HT-Direct \\ 
  Benin & MONO & 90-94 & 154.41 & 144.56 & 165.12 & RW2 \\ 
  Benin & MONO & 95-99 & 141.13 & 157.00 & 126.62 & HT-Direct \\ 
  Benin & MONO & 95-99 & 134.49 & 125.12 & 144.42 & RW2 \\ 
  Benin & MONO & 00-04 & 112.33 & 127.30 & 98.92 & HT-Direct \\ 
  Benin & MONO & 00-04 & 116.72 & 105.78 & 128.66 & RW2 \\ 
  Benin & MONO & 05-09 & 89.04 & 121.94 & 64.37 & HT-Direct \\ 
  Benin & MONO & 05-09 & 97.56 & 78.95 & 119.27 & RW2 \\ 
  Benin & MONO & 10-14 & 79.99 & 32.10 & 185.68 & RW2 \\ 
  Benin & MONO & 15-19 & 65.76 & 6.86 & 410.35 & RW2 \\ 
  Benin & OUEME & 80-84 & 212.06 & 240.44 & 186.20 & HT-Direct \\ 
  Benin & OUEME & 80-84 & 204.85 & 185.25 & 226.06 & RW2 \\ 
  Benin & OUEME & 85-89 & 194.26 & 213.37 & 176.48 & HT-Direct \\ 
  Benin & OUEME & 85-89 & 187.77 & 175.45 & 200.89 & RW2 \\ 
  Benin & OUEME & 90-94 & 165.29 & 179.90 & 151.64 & HT-Direct \\ 
  Benin & OUEME & 90-94 & 167.77 & 157.66 & 178.20 & RW2 \\ 
  Benin & OUEME & 95-99 & 160.48 & 176.84 & 145.37 & HT-Direct \\ 
  Benin & OUEME & 95-99 & 154.00 & 143.99 & 164.48 & RW2 \\ 
  Benin & OUEME & 00-04 & 138.31 & 154.71 & 123.40 & HT-Direct \\ 
  Benin & OUEME & 00-04 & 143.47 & 131.56 & 156.42 & RW2 \\ 
  Benin & OUEME & 05-09 & 130.33 & 162.18 & 103.96 & HT-Direct \\ 
  Benin & OUEME & 05-09 & 130.33 & 109.21 & 155.32 & RW2 \\ 
  Benin & OUEME & 10-14 & 116.77 & 48.78 & 253.84 & RW2 \\ 
  Benin & OUEME & 15-19 & 104.58 & 11.75 & 530.27 & RW2 \\ 
  Benin & ZOU & 80-84 & 233.74 & 260.97 & 208.55 & HT-Direct \\ 
  Benin & ZOU & 80-84 & 224.20 & 205.09 & 244.77 & RW2 \\ 
  Benin & ZOU & 85-89 & 201.71 & 224.15 & 181.00 & HT-Direct \\ 
  Benin & ZOU & 85-89 & 198.81 & 185.78 & 212.65 & RW2 \\ 
  Benin & ZOU & 90-94 & 179.20 & 194.89 & 164.52 & HT-Direct \\ 
  Benin & ZOU & 90-94 & 172.31 & 162.59 & 182.48 & RW2 \\ 
  Benin & ZOU & 95-99 & 144.26 & 156.87 & 132.50 & HT-Direct \\ 
  Benin & ZOU & 95-99 & 151.86 & 143.11 & 160.93 & RW2 \\ 
  Benin & ZOU & 00-04 & 147.49 & 163.21 & 133.04 & HT-Direct \\ 
  Benin & ZOU & 00-04 & 135.44 & 124.76 & 146.93 & RW2 \\ 
  Benin & ZOU & 05-09 & 82.36 & 113.91 & 58.97 & HT-Direct \\ 
  Benin & ZOU & 05-09 & 115.79 & 96.90 & 137.97 & RW2 \\ 
  Benin & ZOU & 10-14 & 97.41 & 40.36 & 216.81 & RW2 \\ 
  Benin & ZOU & 15-19 & 82.14 & 8.90 & 460.26 & RW2 \\ 
  Burkina Faso & ALL & 80-84 & 242.41 & 239.49 & 245.15 & IHME \\ 
  Burkina Faso & ALL & 80-84 & 232.08 & 222.75 & 241.67 & RW2 \\ 
  Burkina Faso & ALL & 80-84 & 232.03 & 224.90 & 238.70 & UN \\ 
  Burkina Faso & ALL & 85-89 & 218.13 & 215.86 & 220.79 & IHME \\ 
  Burkina Faso & ALL & 85-89 & 209.98 & 202.20 & 217.93 & RW2 \\ 
  Burkina Faso & ALL & 85-89 & 210.07 & 204.38 & 216.27 & UN \\ 
  Burkina Faso & ALL & 90-94 & 202.65 & 200.30 & 204.84 & IHME \\ 
  Burkina Faso & ALL & 90-94 & 201.84 & 195.42 & 208.43 & RW2 \\ 
  Burkina Faso & ALL & 90-94 & 201.77 & 196.36 & 207.54 & UN \\ 
  Burkina Faso & ALL & 95-99 & 188.88 & 186.70 & 191.42 & IHME \\ 
  Burkina Faso & ALL & 95-99 & 193.74 & 187.24 & 200.36 & RW2 \\ 
  Burkina Faso & ALL & 95-99 & 193.79 & 188.27 & 200.09 & UN \\ 
  Burkina Faso & ALL & 00-04 & 168.26 & 165.68 & 170.79 & IHME \\ 
  Burkina Faso & ALL & 00-04 & 176.96 & 169.51 & 184.71 & RW2 \\ 
  Burkina Faso & ALL & 00-04 & 176.96 & 170.60 & 183.88 & UN \\ 
  Burkina Faso & ALL & 05-09 & 144.04 & 140.79 & 147.39 & IHME \\ 
  Burkina Faso & ALL & 05-09 & 139.53 & 131.41 & 148.06 & RW2 \\ 
  Burkina Faso & ALL & 05-09 & 139.54 & 133.63 & 145.72 & UN \\ 
  Burkina Faso & ALL & 10-14 & 122.58 & 118.41 & 127.04 & IHME \\ 
  Burkina Faso & ALL & 10-14 & 103.58 & 40.39 & 236.89 & RW2 \\ 
  Burkina Faso & ALL & 10-14 & 102.06 & 93.11 & 111.53 & UN \\ 
  Burkina Faso & CENTRAL/SOUTH & 80-84 & 235.28 & 252.06 & 219.28 & HT-Direct \\ 
  Burkina Faso & CENTRAL/SOUTH & 80-84 & 216.67 & 202.70 & 231.13 & RW2 \\ 
  Burkina Faso & CENTRAL/SOUTH & 85-89 & 187.11 & 199.35 & 175.46 & HT-Direct \\ 
  Burkina Faso & CENTRAL/SOUTH & 85-89 & 203.32 & 193.13 & 214.06 & RW2 \\ 
  Burkina Faso & CENTRAL/SOUTH & 90-94 & 180.67 & 191.09 & 170.69 & HT-Direct \\ 
  Burkina Faso & CENTRAL/SOUTH & 90-94 & 201.46 & 191.83 & 211.50 & RW2 \\ 
  Burkina Faso & CENTRAL/SOUTH & 95-99 & 177.16 & 187.85 & 166.95 & HT-Direct \\ 
  Burkina Faso & CENTRAL/SOUTH & 95-99 & 195.31 & 185.52 & 205.52 & RW2 \\ 
  Burkina Faso & CENTRAL/SOUTH & 00-04 & 145.59 & 156.55 & 135.28 & HT-Direct \\ 
  Burkina Faso & CENTRAL/SOUTH & 00-04 & 179.66 & 168.84 & 191.14 & RW2 \\ 
  Burkina Faso & CENTRAL/SOUTH & 05-09 & 109.25 & 121.11 & 98.43 & HT-Direct \\ 
  Burkina Faso & CENTRAL/SOUTH & 05-09 & 142.91 & 129.38 & 157.46 & RW2 \\ 
  Burkina Faso & CENTRAL/SOUTH & 10-14 & 107.21 & 47.01 & 226.55 & RW2 \\ 
  Burkina Faso & CENTRAL/SOUTH & 15-19 & 78.90 & 8.82 & 443.68 & RW2 \\ 
  Burkina Faso & EAST & 80-84 & 263.51 & 287.21 & 241.11 & HT-Direct \\ 
  Burkina Faso & EAST & 80-84 & 257.92 & 239.19 & 277.88 & RW2 \\ 
  Burkina Faso & EAST & 85-89 & 222.35 & 242.60 & 203.34 & HT-Direct \\ 
  Burkina Faso & EAST & 85-89 & 226.38 & 213.11 & 240.16 & RW2 \\ 
  Burkina Faso & EAST & 90-94 & 213.19 & 228.46 & 198.68 & HT-Direct \\ 
  Burkina Faso & EAST & 90-94 & 210.81 & 200.04 & 221.98 & RW2 \\ 
  Burkina Faso & EAST & 95-99 & 193.41 & 207.86 & 179.75 & HT-Direct \\ 
  Burkina Faso & EAST & 95-99 & 194.16 & 183.89 & 204.83 & RW2 \\ 
  Burkina Faso & EAST & 00-04 & 168.41 & 187.29 & 151.09 & HT-Direct \\ 
  Burkina Faso & EAST & 00-04 & 170.34 & 158.43 & 183.07 & RW2 \\ 
  Burkina Faso & EAST & 05-09 & 125.02 & 142.50 & 109.40 & HT-Direct \\ 
  Burkina Faso & EAST & 05-09 & 129.97 & 115.93 & 145.58 & RW2 \\ 
  Burkina Faso & EAST & 10-14 & 93.32 & 39.90 & 201.19 & RW2 \\ 
  Burkina Faso & EAST & 15-19 & 66.11 & 7.28 & 403.03 & RW2 \\ 
  Burkina Faso & NORTH & 80-84 & 233.39 & 257.81 & 210.63 & HT-Direct \\ 
  Burkina Faso & NORTH & 80-84 & 224.74 & 211.26 & 238.88 & RW2 \\ 
  Burkina Faso & NORTH & 85-89 & 229.12 & 246.29 & 212.81 & HT-Direct \\ 
  Burkina Faso & NORTH & 85-89 & 195.25 & 185.86 & 204.99 & RW2 \\ 
  Burkina Faso & NORTH & 90-94 & 231.48 & 248.44 & 215.35 & HT-Direct \\ 
  Burkina Faso & NORTH & 90-94 & 182.83 & 174.85 & 191.00 & RW2 \\ 
  Burkina Faso & NORTH & 95-99 & 234.46 & 251.92 & 217.86 & HT-Direct \\ 
  Burkina Faso & NORTH & 95-99 & 170.71 & 163.02 & 178.72 & RW2 \\ 
  Burkina Faso & NORTH & 00-04 & 228.79 & 248.25 & 210.42 & HT-Direct \\ 
  Burkina Faso & NORTH & 00-04 & 150.03 & 141.69 & 158.90 & RW2 \\ 
  Burkina Faso & NORTH & 05-09 & 183.59 & 203.35 & 165.35 & HT-Direct \\ 
  Burkina Faso & NORTH & 05-09 & 114.15 & 104.13 & 125.00 & RW2 \\ 
  Burkina Faso & NORTH & 10-14 & 81.85 & 35.12 & 177.19 & RW2 \\ 
  Burkina Faso & NORTH & 15-19 & 57.96 & 6.36 & 367.23 & RW2 \\ 
  Burkina Faso & WEST & 80-84 & 219.42 & 236.24 & 203.49 & HT-Direct \\ 
  Burkina Faso & WEST & 80-84 & 233.50 & 215.16 & 252.69 & RW2 \\ 
  Burkina Faso & WEST & 85-89 & 200.83 & 214.64 & 187.69 & HT-Direct \\ 
  Burkina Faso & WEST & 85-89 & 226.80 & 214.52 & 239.47 & RW2 \\ 
  Burkina Faso & WEST & 90-94 & 205.41 & 218.87 & 192.57 & HT-Direct \\ 
  Burkina Faso & WEST & 90-94 & 232.52 & 221.20 & 244.37 & RW2 \\ 
  Burkina Faso & WEST & 95-99 & 191.78 & 205.57 & 178.70 & HT-Direct \\ 
  Burkina Faso & WEST & 95-99 & 235.74 & 224.02 & 247.81 & RW2 \\ 
  Burkina Faso & WEST & 00-04 & 181.98 & 197.44 & 167.48 & HT-Direct \\ 
  Burkina Faso & WEST & 00-04 & 228.14 & 214.80 & 242.14 & RW2 \\ 
  Burkina Faso & WEST & 05-09 & 133.38 & 148.88 & 119.26 & HT-Direct \\ 
  Burkina Faso & WEST & 05-09 & 193.66 & 176.90 & 211.83 & RW2 \\ 
  Burkina Faso & WEST & 10-14 & 155.58 & 70.09 & 308.32 & RW2 \\ 
  Burkina Faso & WEST & 15-19 & 123.44 & 14.09 & 570.02 & RW2 \\ 
  Burundi & ALL & 80-84 & 202.90 & 196.35 & 208.65 & IHME \\ 
  Burundi & ALL & 80-84 & 202.14 & 163.91 & 246.94 & RW2 \\ 
  Burundi & ALL & 80-84 & 201.57 & 193.32 & 211.06 & UN \\ 
  Burundi & ALL & 85-89 & 174.26 & 169.47 & 179.73 & IHME \\ 
  Burundi & ALL & 85-89 & 172.52 & 152.02 & 194.58 & RW2 \\ 
  Burundi & ALL & 85-89 & 172.35 & 164.06 & 180.82 & UN \\ 
  Burundi & ALL & 90-94 & 174.19 & 169.16 & 179.53 & IHME \\ 
  Burundi & ALL & 90-94 & 173.31 & 158.82 & 188.95 & RW2 \\ 
  Burundi & ALL & 90-94 & 173.74 & 165.42 & 182.23 & UN \\ 
  Burundi & ALL & 95-99 & 172.33 & 167.23 & 177.79 & IHME \\ 
  Burundi & ALL & 95-99 & 164.52 & 153.42 & 176.14 & RW2 \\ 
  Burundi & ALL & 95-99 & 164.77 & 156.21 & 173.93 & UN \\ 
  Burundi & ALL & 00-04 & 149.22 & 144.17 & 154.66 & IHME \\ 
  Burundi & ALL & 00-04 & 143.59 & 135.13 & 152.53 & RW2 \\ 
  Burundi & ALL & 00-04 & 142.94 & 134.94 & 151.10 & UN \\ 
  Burundi & ALL & 05-09 & 99.50 & 95.67 & 103.48 & IHME \\ 
  Burundi & ALL & 05-09 & 114.07 & 103.87 & 125.14 & RW2 \\ 
  Burundi & ALL & 05-09 & 114.34 & 104.29 & 124.91 & UN \\ 
  Burundi & ALL & 10-14 & 82.36 & 74.46 & 90.07 & IHME \\ 
  Burundi & ALL & 10-14 & 87.32 & 31.01 & 218.25 & RW2 \\ 
  Burundi & ALL & 10-14 & 92.23 & 78.38 & 109.08 & UN \\ 
  Burundi & BUJUMBURA & 80-84 & 237.56 & 389.25 & 132.18 & HT-Direct \\ 
  Burundi & BUJUMBURA & 80-84 & 186.14 & 118.57 & 288.34 & RW2 \\ 
  Burundi & BUJUMBURA & 85-89 & 138.40 & 240.16 & 75.47 & HT-Direct \\ 
  Burundi & BUJUMBURA & 85-89 & 133.40 & 96.33 & 181.29 & RW2 \\ 
  Burundi & BUJUMBURA & 90-94 & 101.43 & 154.75 & 65.07 & HT-Direct \\ 
  Burundi & BUJUMBURA & 90-94 & 114.59 & 87.59 & 146.46 & RW2 \\ 
  Burundi & BUJUMBURA & 95-99 & 148.69 & 218.95 & 98.14 & HT-Direct \\ 
  Burundi & BUJUMBURA & 95-99 & 98.61 & 78.39 & 122.21 & RW2 \\ 
  Burundi & BUJUMBURA & 00-04 & 86.48 & 122.71 & 60.22 & HT-Direct \\ 
  Burundi & BUJUMBURA & 00-04 & 77.61 & 62.65 & 95.46 & RW2 \\ 
  Burundi & BUJUMBURA & 05-09 & 56.89 & 77.28 & 41.63 & HT-Direct \\ 
  Burundi & BUJUMBURA & 05-09 & 58.74 & 43.17 & 80.34 & RW2 \\ 
  Burundi & BUJUMBURA & 10-14 & 44.27 & 14.32 & 134.58 & RW2 \\ 
  Burundi & BUJUMBURA & 15-19 & 33.49 & 2.36 & 358.63 & RW2 \\ 
  Burundi & CENTRE-EAST & 80-84 & 313.83 & 457.10 & 199.00 & HT-Direct \\ 
  Burundi & CENTRE-EAST & 80-84 & 203.27 & 146.40 & 285.79 & RW2 \\ 
  Burundi & CENTRE-EAST & 85-89 & 152.09 & 198.98 & 114.67 & HT-Direct \\ 
  Burundi & CENTRE-EAST & 85-89 & 161.60 & 134.30 & 193.81 & RW2 \\ 
  Burundi & CENTRE-EAST & 90-94 & 168.81 & 200.59 & 141.17 & HT-Direct \\ 
  Burundi & CENTRE-EAST & 90-94 & 154.14 & 135.59 & 173.90 & RW2 \\ 
  Burundi & CENTRE-EAST & 95-99 & 180.82 & 209.82 & 155.04 & HT-Direct \\ 
  Burundi & CENTRE-EAST & 95-99 & 148.11 & 133.88 & 163.11 & RW2 \\ 
  Burundi & CENTRE-EAST & 00-04 & 179.53 & 207.30 & 154.75 & HT-Direct \\ 
  Burundi & CENTRE-EAST & 00-04 & 134.06 & 121.55 & 148.22 & RW2 \\ 
  Burundi & CENTRE-EAST & 05-09 & 92.76 & 108.54 & 79.08 & HT-Direct \\ 
  Burundi & CENTRE-EAST & 05-09 & 109.01 & 93.01 & 127.46 & RW2 \\ 
  Burundi & CENTRE-EAST & 10-14 & 84.85 & 32.72 & 202.48 & RW2 \\ 
  Burundi & CENTRE-EAST & 15-19 & 65.02 & 5.63 & 451.28 & RW2 \\ 
  Burundi & NORTH & 80-84 & 207.97 & 314.69 & 130.54 & HT-Direct \\ 
  Burundi & NORTH & 80-84 & 184.52 & 130.46 & 250.53 & RW2 \\ 
  Burundi & NORTH & 85-89 & 160.34 & 213.88 & 118.18 & HT-Direct \\ 
  Burundi & NORTH & 85-89 & 174.22 & 142.68 & 209.40 & RW2 \\ 
  Burundi & NORTH & 90-94 & 223.90 & 265.46 & 187.18 & HT-Direct \\ 
  Burundi & NORTH & 90-94 & 191.07 & 168.67 & 216.47 & RW2 \\ 
  Burundi & NORTH & 95-99 & 250.90 & 295.09 & 211.34 & HT-Direct \\ 
  Burundi & NORTH & 95-99 & 192.42 & 173.37 & 213.87 & RW2 \\ 
  Burundi & NORTH & 00-04 & 210.64 & 239.95 & 184.04 & HT-Direct \\ 
  Burundi & NORTH & 00-04 & 172.62 & 157.31 & 188.95 & RW2 \\ 
  Burundi & NORTH & 05-09 & 130.72 & 148.74 & 114.59 & HT-Direct \\ 
  Burundi & NORTH & 05-09 & 144.22 & 125.62 & 165.16 & RW2 \\ 
  Burundi & NORTH & 10-14 & 116.80 & 45.84 & 265.72 & RW2 \\ 
  Burundi & NORTH & 15-19 & 93.55 & 8.04 & 559.64 & RW2 \\ 
  Burundi & SOUTH & 80-84 & 203.48 & 291.21 & 137.07 & HT-Direct \\ 
  Burundi & SOUTH & 80-84 & 218.88 & 164.37 & 287.38 & RW2 \\ 
  Burundi & SOUTH & 85-89 & 228.74 & 292.34 & 175.55 & HT-Direct \\ 
  Burundi & SOUTH & 85-89 & 173.87 & 143.34 & 210.32 & RW2 \\ 
  Burundi & SOUTH & 90-94 & 151.94 & 198.30 & 114.86 & HT-Direct \\ 
  Burundi & SOUTH & 90-94 & 152.59 & 130.89 & 177.25 & RW2 \\ 
  Burundi & SOUTH & 95-99 & 164.09 & 193.32 & 138.52 & HT-Direct \\ 
  Burundi & SOUTH & 95-99 & 132.11 & 117.45 & 147.28 & RW2 \\ 
  Burundi & SOUTH & 00-04 & 138.80 & 158.69 & 121.04 & HT-Direct \\ 
  Burundi & SOUTH & 00-04 & 110.82 & 100.65 & 121.93 & RW2 \\ 
  Burundi & SOUTH & 05-09 & 81.50 & 97.90 & 67.64 & HT-Direct \\ 
  Burundi & SOUTH & 05-09 & 87.93 & 73.18 & 106.22 & RW2 \\ 
  Burundi & SOUTH & 10-14 & 68.02 & 25.20 & 173.35 & RW2 \\ 
  Burundi & SOUTH & 15-19 & 52.71 & 4.31 & 413.21 & RW2 \\ 
  Burundi & WEST & 80-84 & 166.54 & 258.40 & 102.80 & HT-Direct \\ 
  Burundi & WEST & 80-84 & 195.53 & 135.00 & 266.75 & RW2 \\ 
  Burundi & WEST & 85-89 & 180.68 & 239.49 & 133.78 & HT-Direct \\ 
  Burundi & WEST & 85-89 & 187.12 & 153.64 & 225.21 & RW2 \\ 
  Burundi & WEST & 90-94 & 253.72 & 309.47 & 205.03 & HT-Direct \\ 
  Burundi & WEST & 90-94 & 200.39 & 174.87 & 229.88 & RW2 \\ 
  Burundi & WEST & 95-99 & 245.65 & 292.03 & 204.50 & HT-Direct \\ 
  Burundi & WEST & 95-99 & 193.27 & 172.41 & 217.85 & RW2 \\ 
  Burundi & WEST & 00-04 & 208.71 & 246.05 & 175.72 & HT-Direct \\ 
  Burundi & WEST & 00-04 & 161.80 & 144.84 & 180.47 & RW2 \\ 
  Burundi & WEST & 05-09 & 96.47 & 118.86 & 77.93 & HT-Direct \\ 
  Burundi & WEST & 05-09 & 120.06 & 96.81 & 146.00 & RW2 \\ 
  Burundi & WEST & 10-14 & 84.90 & 31.35 & 205.07 & RW2 \\ 
  Burundi & WEST & 15-19 & 58.55 & 4.90 & 430.22 & RW2 \\ 
  Cameroon & ADAM/NORD/EXT-NORD & 80-84 & 219.24 & 245.15 & 195.36 & HT-Direct \\ 
  Cameroon & ADAM/NORD/EXT-NORD & 80-84 & 242.28 & 218.86 & 267.71 & RW2 \\ 
  Cameroon & ADAM/NORD/EXT-NORD & 85-89 & 197.13 & 214.59 & 180.76 & HT-Direct \\ 
  Cameroon & ADAM/NORD/EXT-NORD & 85-89 & 198.95 & 185.03 & 213.61 & RW2 \\ 
  Cameroon & ADAM/NORD/EXT-NORD & 90-94 & 196.96 & 211.34 & 183.33 & HT-Direct \\ 
  Cameroon & ADAM/NORD/EXT-NORD & 90-94 & 195.57 & 184.65 & 207.00 & RW2 \\ 
  Cameroon & ADAM/NORD/EXT-NORD & 95-99 & 198.76 & 214.58 & 183.83 & HT-Direct \\ 
  Cameroon & ADAM/NORD/EXT-NORD & 95-99 & 202.45 & 190.76 & 214.52 & RW2 \\ 
  Cameroon & ADAM/NORD/EXT-NORD & 00-04 & 182.22 & 195.51 & 169.64 & HT-Direct \\ 
  Cameroon & ADAM/NORD/EXT-NORD & 00-04 & 185.29 & 175.08 & 195.88 & RW2 \\ 
  Cameroon & ADAM/NORD/EXT-NORD & 05-09 & 169.03 & 184.41 & 154.68 & HT-Direct \\ 
  Cameroon & ADAM/NORD/EXT-NORD & 05-09 & 152.49 & 142.35 & 163.26 & RW2 \\ 
  Cameroon & ADAM/NORD/EXT-NORD & 10-14 & 149.39 & 175.90 & 126.26 & HT-Direct \\ 
  Cameroon & ADAM/NORD/EXT-NORD & 10-14 & 134.00 & 117.11 & 153.30 & RW2 \\ 
  Cameroon & ADAM/NORD/EXT-NORD & 15-19 & 121.25 & 53.91 & 254.20 & RW2 \\ 
  Cameroon & ALL & 80-84 & 158.55 & 156.21 & 160.95 & IHME \\ 
  Cameroon & ALL & 80-84 & 169.36 & 155.99 & 183.63 & RW2 \\ 
  Cameroon & ALL & 80-84 & 169.28 & 163.27 & 175.03 & UN \\ 
  Cameroon & ALL & 85-89 & 140.84 & 138.75 & 142.90 & IHME \\ 
  Cameroon & ALL & 85-89 & 141.89 & 133.22 & 150.96 & RW2 \\ 
  Cameroon & ALL & 85-89 & 142.18 & 137.56 & 146.54 & UN \\ 
  Cameroon & ALL & 90-94 & 138.19 & 136.10 & 140.09 & IHME \\ 
  Cameroon & ALL & 90-94 & 143.56 & 136.57 & 150.85 & RW2 \\ 
  Cameroon & ALL & 90-94 & 143.13 & 138.46 & 148.01 & UN \\ 
  Cameroon & ALL & 95-99 & 140.21 & 137.84 & 142.57 & IHME \\ 
  Cameroon & ALL & 95-99 & 153.74 & 145.96 & 161.82 & RW2 \\ 
  Cameroon & ALL & 95-99 & 154.14 & 148.71 & 159.73 & UN \\ 
  Cameroon & ALL & 00-04 & 132.62 & 130.14 & 135.27 & IHME \\ 
  Cameroon & ALL & 00-04 & 139.93 & 133.35 & 146.81 & RW2 \\ 
  Cameroon & ALL & 00-04 & 139.95 & 133.56 & 146.00 & UN \\ 
  Cameroon & ALL & 05-09 & 117.53 & 114.49 & 120.78 & IHME \\ 
  Cameroon & ALL & 05-09 & 116.82 & 110.32 & 123.65 & RW2 \\ 
  Cameroon & ALL & 05-09 & 116.63 & 105.17 & 128.48 & UN \\ 
  Cameroon & ALL & 10-14 & 99.64 & 95.07 & 104.52 & IHME \\ 
  Cameroon & ALL & 10-14 & 98.48 & 87.41 & 110.70 & RW2 \\ 
  Cameroon & ALL & 10-14 & 98.85 & 80.91 & 120.50 & UN \\ 
  Cameroon & CENTRE/SUD/EST & 80-84 & 148.34 & 173.72 & 126.10 & HT-Direct \\ 
  Cameroon & CENTRE/SUD/EST & 80-84 & 159.13 & 138.62 & 181.61 & RW2 \\ 
  Cameroon & CENTRE/SUD/EST & 85-89 & 123.21 & 138.17 & 109.67 & HT-Direct \\ 
  Cameroon & CENTRE/SUD/EST & 85-89 & 131.98 & 120.48 & 144.19 & RW2 \\ 
  Cameroon & CENTRE/SUD/EST & 90-94 & 138.00 & 153.23 & 124.06 & HT-Direct \\ 
  Cameroon & CENTRE/SUD/EST & 90-94 & 134.73 & 125.08 & 145.12 & RW2 \\ 
  Cameroon & CENTRE/SUD/EST & 95-99 & 137.79 & 151.57 & 125.09 & HT-Direct \\ 
  Cameroon & CENTRE/SUD/EST & 95-99 & 144.35 & 134.58 & 154.87 & RW2 \\ 
  Cameroon & CENTRE/SUD/EST & 00-04 & 143.06 & 158.49 & 128.90 & HT-Direct \\ 
  Cameroon & CENTRE/SUD/EST & 00-04 & 132.44 & 123.17 & 142.55 & RW2 \\ 
  Cameroon & CENTRE/SUD/EST & 05-09 & 108.10 & 122.82 & 94.95 & HT-Direct \\ 
  Cameroon & CENTRE/SUD/EST & 05-09 & 103.32 & 94.02 & 113.50 & RW2 \\ 
  Cameroon & CENTRE/SUD/EST & 10-14 & 88.00 & 120.02 & 63.90 & HT-Direct \\ 
  Cameroon & CENTRE/SUD/EST & 10-14 & 84.67 & 69.13 & 102.25 & RW2 \\ 
  Cameroon & CENTRE/SUD/EST & 15-19 & 71.07 & 29.18 & 162.97 & RW2 \\ 
  Cameroon & NORD-OUEST/SUD-OUEST & 80-84 & 121.50 & 145.99 & 100.64 & HT-Direct \\ 
  Cameroon & NORD-OUEST/SUD-OUEST & 80-84 & 122.46 & 104.66 & 142.71 & RW2 \\ 
  Cameroon & NORD-OUEST/SUD-OUEST & 85-89 & 86.39 & 102.10 & 72.89 & HT-Direct \\ 
  Cameroon & NORD-OUEST/SUD-OUEST & 85-89 & 99.98 & 89.33 & 111.77 & RW2 \\ 
  Cameroon & NORD-OUEST/SUD-OUEST & 90-94 & 104.27 & 119.32 & 90.92 & HT-Direct \\ 
  Cameroon & NORD-OUEST/SUD-OUEST & 90-94 & 102.30 & 93.43 & 111.77 & RW2 \\ 
  Cameroon & NORD-OUEST/SUD-OUEST & 95-99 & 107.73 & 122.55 & 94.51 & HT-Direct \\ 
  Cameroon & NORD-OUEST/SUD-OUEST & 95-99 & 110.95 & 101.92 & 120.83 & RW2 \\ 
  Cameroon & NORD-OUEST/SUD-OUEST & 00-04 & 111.34 & 128.37 & 96.32 & HT-Direct \\ 
  Cameroon & NORD-OUEST/SUD-OUEST & 00-04 & 102.91 & 93.94 & 112.88 & RW2 \\ 
  Cameroon & NORD-OUEST/SUD-OUEST & 05-09 & 85.67 & 102.81 & 71.17 & HT-Direct \\ 
  Cameroon & NORD-OUEST/SUD-OUEST & 05-09 & 81.60 & 72.15 & 92.13 & RW2 \\ 
  Cameroon & NORD-OUEST/SUD-OUEST & 10-14 & 67.56 & 98.00 & 46.10 & HT-Direct \\ 
  Cameroon & NORD-OUEST/SUD-OUEST & 10-14 & 68.23 & 54.58 & 84.37 & RW2 \\ 
  Cameroon & NORD-OUEST/SUD-OUEST & 15-19 & 58.52 & 23.98 & 136.32 & RW2 \\ 
  Cameroon & OUEST/LITTORAL & 80-84 & 120.23 & 141.80 & 101.54 & HT-Direct \\ 
  Cameroon & OUEST/LITTORAL & 80-84 & 125.26 & 107.90 & 145.22 & RW2 \\ 
  Cameroon & OUEST/LITTORAL & 85-89 & 97.43 & 115.77 & 81.73 & HT-Direct \\ 
  Cameroon & OUEST/LITTORAL & 85-89 & 104.67 & 93.42 & 117.10 & RW2 \\ 
  Cameroon & OUEST/LITTORAL & 90-94 & 104.84 & 122.48 & 89.49 & HT-Direct \\ 
  Cameroon & OUEST/LITTORAL & 90-94 & 109.46 & 99.64 & 119.99 & RW2 \\ 
  Cameroon & OUEST/LITTORAL & 95-99 & 123.76 & 137.69 & 111.06 & HT-Direct \\ 
  Cameroon & OUEST/LITTORAL & 95-99 & 122.01 & 112.60 & 132.17 & RW2 \\ 
  Cameroon & OUEST/LITTORAL & 00-04 & 116.38 & 133.09 & 101.53 & HT-Direct \\ 
  Cameroon & OUEST/LITTORAL & 00-04 & 116.87 & 106.81 & 127.84 & RW2 \\ 
  Cameroon & OUEST/LITTORAL & 05-09 & 107.68 & 132.79 & 86.85 & HT-Direct \\ 
  Cameroon & OUEST/LITTORAL & 05-09 & 97.29 & 85.49 & 110.54 & RW2 \\ 
  Cameroon & OUEST/LITTORAL & 10-14 & 88.84 & 126.66 & 61.51 & HT-Direct \\ 
  Cameroon & OUEST/LITTORAL & 10-14 & 85.77 & 68.41 & 106.58 & RW2 \\ 
  Cameroon & OUEST/LITTORAL & 15-19 & 77.59 & 32.01 & 177.60 & RW2 \\ 
  Cameroon & YAOUNDE/DOUALA & 80-84 & 119.33 & 156.26 & 90.19 & HT-Direct \\ 
  Cameroon & YAOUNDE/DOUALA & 80-84 & 119.33 & 96.39 & 147.62 & RW2 \\ 
  Cameroon & YAOUNDE/DOUALA & 85-89 & 86.58 & 109.53 & 68.07 & HT-Direct \\ 
  Cameroon & YAOUNDE/DOUALA & 85-89 & 93.92 & 80.71 & 108.93 & RW2 \\ 
  Cameroon & YAOUNDE/DOUALA & 90-94 & 93.75 & 113.08 & 77.44 & HT-Direct \\ 
  Cameroon & YAOUNDE/DOUALA & 90-94 & 92.88 & 82.55 & 104.28 & RW2 \\ 
  Cameroon & YAOUNDE/DOUALA & 95-99 & 89.72 & 106.92 & 75.06 & HT-Direct \\ 
  Cameroon & YAOUNDE/DOUALA & 95-99 & 98.27 & 88.27 & 109.05 & RW2 \\ 
  Cameroon & YAOUNDE/DOUALA & 00-04 & 94.35 & 111.87 & 79.33 & HT-Direct \\ 
  Cameroon & YAOUNDE/DOUALA & 00-04 & 89.53 & 80.58 & 99.55 & RW2 \\ 
  Cameroon & YAOUNDE/DOUALA & 05-09 & 78.76 & 94.65 & 65.34 & HT-Direct \\ 
  Cameroon & YAOUNDE/DOUALA & 05-09 & 69.76 & 60.89 & 79.76 & RW2 \\ 
  Cameroon & YAOUNDE/DOUALA & 10-14 & 48.81 & 84.64 & 27.69 & HT-Direct \\ 
  Cameroon & YAOUNDE/DOUALA & 10-14 & 57.03 & 43.91 & 73.39 & RW2 \\ 
  Cameroon & YAOUNDE/DOUALA & 15-19 & 48.07 & 18.83 & 117.04 & RW2 \\ 
  Chad & ALL & 80-84 & 229.89 & 226.29 & 233.50 & IHME \\ 
  Chad & ALL & 80-84 & 237.13 & 209.02 & 267.75 & RW2 \\ 
  Chad & ALL & 80-84 & 236.47 & 227.15 & 245.73 & UN \\ 
  Chad & ALL & 85-89 & 212.02 & 208.56 & 215.04 & IHME \\ 
  Chad & ALL & 85-89 & 222.45 & 200.68 & 245.80 & RW2 \\ 
  Chad & ALL & 85-89 & 223.02 & 215.48 & 230.64 & UN \\ 
  Chad & ALL & 90-94 & 195.25 & 192.48 & 198.26 & IHME \\ 
  Chad & ALL & 90-94 & 209.21 & 193.84 & 225.43 & RW2 \\ 
  Chad & ALL & 90-94 & 209.68 & 202.88 & 216.80 & UN \\ 
  Chad & ALL & 95-99 & 186.83 & 183.80 & 190.07 & IHME \\ 
  Chad & ALL & 95-99 & 199.03 & 187.71 & 210.78 & RW2 \\ 
  Chad & ALL & 95-99 & 198.22 & 191.97 & 205.09 & UN \\ 
  Chad & ALL & 00-04 & 178.32 & 174.29 & 182.45 & IHME \\ 
  Chad & ALL & 00-04 & 183.50 & 172.37 & 195.20 & RW2 \\ 
  Chad & ALL & 00-04 & 184.23 & 176.95 & 191.97 & UN \\ 
  Chad & ALL & 05-09 & 159.79 & 154.39 & 165.02 & IHME \\ 
  Chad & ALL & 05-09 & 172.08 & 157.73 & 187.54 & RW2 \\ 
  Chad & ALL & 05-09 & 171.26 & 160.47 & 182.07 & UN \\ 
  Chad & ALL & 10-14 & 141.83 & 135.05 & 148.74 & IHME \\ 
  Chad & ALL & 10-14 & 152.93 & 142.70 & 163.68 & RW2 \\ 
  Chad & ALL & 10-14 & 153.04 & 134.67 & 175.78 & UN \\ 
  Chad & ZONE 1 & 80-84 & 167.45 & 224.83 & 122.40 & HT-Direct \\ 
  Chad & ZONE 1 & 80-84 & 190.73 & 148.14 & 242.34 & RW2 \\ 
  Chad & ZONE 1 & 85-89 & 148.35 & 193.06 & 112.54 & HT-Direct \\ 
  Chad & ZONE 1 & 85-89 & 189.30 & 159.13 & 224.28 & RW2 \\ 
  Chad & ZONE 1 & 90-94 & 193.20 & 226.56 & 163.72 & HT-Direct \\ 
  Chad & ZONE 1 & 90-94 & 183.50 & 162.50 & 207.00 & RW2 \\ 
  Chad & ZONE 1 & 95-99 & 164.39 & 187.35 & 143.75 & HT-Direct \\ 
  Chad & ZONE 1 & 95-99 & 171.58 & 156.07 & 188.15 & RW2 \\ 
  Chad & ZONE 1 & 00-04 & 156.96 & 177.17 & 138.67 & HT-Direct \\ 
  Chad & ZONE 1 & 00-04 & 165.34 & 150.90 & 180.71 & RW2 \\ 
  Chad & ZONE 1 & 05-09 & 157.94 & 185.07 & 134.13 & HT-Direct \\ 
  Chad & ZONE 1 & 05-09 & 164.45 & 146.88 & 183.76 & RW2 \\ 
  Chad & ZONE 1 & 10-14 & 141.00 & 171.24 & 115.35 & HT-Direct \\ 
  Chad & ZONE 1 & 10-14 & 158.05 & 131.74 & 189.07 & RW2 \\ 
  Chad & ZONE 1 & 15-19 & 149.13 & 67.40 & 298.47 & RW2 \\ 
  Chad & ZONE 2 & 80-84 & 242.74 & 306.55 & 188.60 & HT-Direct \\ 
  Chad & ZONE 2 & 80-84 & 254.60 & 206.28 & 310.88 & RW2 \\ 
  Chad & ZONE 2 & 85-89 & 184.89 & 234.51 & 143.80 & HT-Direct \\ 
  Chad & ZONE 2 & 85-89 & 222.74 & 188.95 & 260.52 & RW2 \\ 
  Chad & ZONE 2 & 90-94 & 171.45 & 208.28 & 139.98 & HT-Direct \\ 
  Chad & ZONE 2 & 90-94 & 192.84 & 170.04 & 218.15 & RW2 \\ 
  Chad & ZONE 2 & 95-99 & 170.44 & 196.51 & 147.19 & HT-Direct \\ 
  Chad & ZONE 2 & 95-99 & 163.08 & 146.99 & 180.35 & RW2 \\ 
  Chad & ZONE 2 & 00-04 & 125.39 & 148.89 & 105.13 & HT-Direct \\ 
  Chad & ZONE 2 & 00-04 & 138.55 & 124.19 & 154.08 & RW2 \\ 
  Chad & ZONE 2 & 05-09 & 110.69 & 135.88 & 89.69 & HT-Direct \\ 
  Chad & ZONE 2 & 05-09 & 121.67 & 106.58 & 138.26 & RW2 \\ 
  Chad & ZONE 2 & 10-14 & 98.12 & 119.27 & 80.38 & HT-Direct \\ 
  Chad & ZONE 2 & 10-14 & 105.82 & 87.59 & 128.10 & RW2 \\ 
  Chad & ZONE 2 & 15-19 & 91.60 & 40.21 & 195.05 & RW2 \\ 
  Chad & ZONE 3 & 80-84 & 299.22 & 372.66 & 234.84 & HT-Direct \\ 
  Chad & ZONE 3 & 80-84 & 296.11 & 243.13 & 357.48 & RW2 \\ 
  Chad & ZONE 3 & 85-89 & 236.85 & 299.84 & 183.63 & HT-Direct \\ 
  Chad & ZONE 3 & 85-89 & 262.23 & 224.99 & 303.66 & RW2 \\ 
  Chad & ZONE 3 & 90-94 & 191.19 & 222.18 & 163.62 & HT-Direct \\ 
  Chad & ZONE 3 & 90-94 & 230.30 & 206.03 & 256.07 & RW2 \\ 
  Chad & ZONE 3 & 95-99 & 206.20 & 235.94 & 179.34 & HT-Direct \\ 
  Chad & ZONE 3 & 95-99 & 199.03 & 181.51 & 217.60 & RW2 \\ 
  Chad & ZONE 3 & 00-04 & 164.77 & 187.62 & 144.21 & HT-Direct \\ 
  Chad & ZONE 3 & 00-04 & 173.39 & 158.75 & 189.05 & RW2 \\ 
  Chad & ZONE 3 & 05-09 & 146.73 & 167.58 & 128.07 & HT-Direct \\ 
  Chad & ZONE 3 & 05-09 & 154.59 & 139.69 & 170.70 & RW2 \\ 
  Chad & ZONE 3 & 10-14 & 122.50 & 145.93 & 102.38 & HT-Direct \\ 
  Chad & ZONE 3 & 10-14 & 134.96 & 115.01 & 158.55 & RW2 \\ 
  Chad & ZONE 3 & 15-19 & 116.53 & 52.95 & 237.10 & RW2 \\ 
  Chad & ZONE 4 & 80-84 & 203.05 & 272.46 & 147.73 & HT-Direct \\ 
  Chad & ZONE 4 & 80-84 & 211.81 & 162.77 & 271.71 & RW2 \\ 
  Chad & ZONE 4 & 85-89 & 140.08 & 192.51 & 100.16 & HT-Direct \\ 
  Chad & ZONE 4 & 85-89 & 186.80 & 153.48 & 225.67 & RW2 \\ 
  Chad & ZONE 4 & 90-94 & 147.21 & 183.84 & 116.83 & HT-Direct \\ 
  Chad & ZONE 4 & 90-94 & 164.02 & 141.50 & 189.42 & RW2 \\ 
  Chad & ZONE 4 & 95-99 & 145.52 & 173.16 & 121.64 & HT-Direct \\ 
  Chad & ZONE 4 & 95-99 & 140.78 & 125.15 & 158.13 & RW2 \\ 
  Chad & ZONE 4 & 00-04 & 120.58 & 141.55 & 102.35 & HT-Direct \\ 
  Chad & ZONE 4 & 00-04 & 120.64 & 108.09 & 134.36 & RW2 \\ 
  Chad & ZONE 4 & 05-09 & 93.41 & 109.41 & 79.54 & HT-Direct \\ 
  Chad & ZONE 4 & 05-09 & 106.11 & 94.24 & 119.04 & RW2 \\ 
  Chad & ZONE 4 & 10-14 & 85.28 & 102.31 & 70.86 & HT-Direct \\ 
  Chad & ZONE 4 & 10-14 & 93.23 & 78.00 & 111.58 & RW2 \\ 
  Chad & ZONE 4 & 15-19 & 81.80 & 35.57 & 176.12 & RW2 \\ 
  Chad & ZONE 5 & 80-84 & 144.19 & 207.32 & 97.90 & HT-Direct \\ 
  Chad & ZONE 5 & 80-84 & 189.37 & 145.88 & 241.28 & RW2 \\ 
  Chad & ZONE 5 & 85-89 & 189.24 & 248.24 & 141.62 & HT-Direct \\ 
  Chad & ZONE 5 & 85-89 & 184.43 & 152.60 & 221.26 & RW2 \\ 
  Chad & ZONE 5 & 90-94 & 162.12 & 197.39 & 132.12 & HT-Direct \\ 
  Chad & ZONE 5 & 90-94 & 176.27 & 153.12 & 202.03 & RW2 \\ 
  Chad & ZONE 5 & 95-99 & 152.60 & 192.95 & 119.43 & HT-Direct \\ 
  Chad & ZONE 5 & 95-99 & 163.89 & 145.64 & 183.76 & RW2 \\ 
  Chad & ZONE 5 & 00-04 & 156.70 & 184.85 & 132.14 & HT-Direct \\ 
  Chad & ZONE 5 & 00-04 & 153.98 & 138.06 & 171.11 & RW2 \\ 
  Chad & ZONE 5 & 05-09 & 123.34 & 153.27 & 98.58 & HT-Direct \\ 
  Chad & ZONE 5 & 05-09 & 146.67 & 129.40 & 165.87 & RW2 \\ 
  Chad & ZONE 5 & 10-14 & 119.85 & 141.31 & 101.26 & HT-Direct \\ 
  Chad & ZONE 5 & 10-14 & 134.83 & 114.50 & 158.49 & RW2 \\ 
  Chad & ZONE 5 & 15-19 & 121.95 & 56.40 & 244.40 & RW2 \\ 
  Chad & ZONE 6 & 80-84 & 190.20 & 266.92 & 131.57 & HT-Direct \\ 
  Chad & ZONE 6 & 80-84 & 175.28 & 133.38 & 230.10 & RW2 \\ 
  Chad & ZONE 6 & 85-89 & 159.45 & 202.65 & 124.03 & HT-Direct \\ 
  Chad & ZONE 6 & 85-89 & 171.25 & 141.66 & 205.18 & RW2 \\ 
  Chad & ZONE 6 & 90-94 & 153.84 & 190.64 & 123.07 & HT-Direct \\ 
  Chad & ZONE 6 & 90-94 & 165.42 & 143.77 & 189.27 & RW2 \\ 
  Chad & ZONE 6 & 95-99 & 125.35 & 151.68 & 103.04 & HT-Direct \\ 
  Chad & ZONE 6 & 95-99 & 158.68 & 140.89 & 176.98 & RW2 \\ 
  Chad & ZONE 6 & 00-04 & 158.13 & 178.92 & 139.35 & HT-Direct \\ 
  Chad & ZONE 6 & 00-04 & 159.16 & 145.27 & 173.91 & RW2 \\ 
  Chad & ZONE 6 & 05-09 & 165.67 & 186.22 & 146.98 & HT-Direct \\ 
  Chad & ZONE 6 & 05-09 & 163.04 & 148.40 & 178.91 & RW2 \\ 
  Chad & ZONE 6 & 10-14 & 129.62 & 151.71 & 110.33 & HT-Direct \\ 
  Chad & ZONE 6 & 10-14 & 158.47 & 136.73 & 183.30 & RW2 \\ 
  Chad & ZONE 6 & 15-19 & 150.81 & 70.01 & 295.52 & RW2 \\ 
  Chad & ZONE 7 & 80-84 & 210.70 & 276.54 & 157.13 & HT-Direct \\ 
  Chad & ZONE 7 & 80-84 & 238.72 & 187.39 & 295.14 & RW2 \\ 
  Chad & ZONE 7 & 85-89 & 202.04 & 249.50 & 161.66 & HT-Direct \\ 
  Chad & ZONE 7 & 85-89 & 250.18 & 214.16 & 289.37 & RW2 \\ 
  Chad & ZONE 7 & 90-94 & 221.09 & 255.93 & 189.78 & HT-Direct \\ 
  Chad & ZONE 7 & 90-94 & 257.58 & 232.41 & 284.47 & RW2 \\ 
  Chad & ZONE 7 & 95-99 & 268.87 & 294.52 & 244.67 & HT-Direct \\ 
  Chad & ZONE 7 & 95-99 & 255.67 & 237.31 & 275.40 & RW2 \\ 
  Chad & ZONE 7 & 00-04 & 236.21 & 265.94 & 208.86 & HT-Direct \\ 
  Chad & ZONE 7 & 00-04 & 248.96 & 229.45 & 270.85 & RW2 \\ 
  Chad & ZONE 7 & 05-09 & 247.06 & 291.14 & 207.71 & HT-Direct \\ 
  Chad & ZONE 7 & 05-09 & 238.38 & 216.39 & 261.89 & RW2 \\ 
  Chad & ZONE 7 & 10-14 & 175.32 & 195.23 & 157.05 & HT-Direct \\ 
  Chad & ZONE 7 & 10-14 & 214.12 & 189.32 & 240.23 & RW2 \\ 
  Chad & ZONE 7 & 15-19 & 187.90 & 90.63 & 345.34 & RW2 \\ 
  Chad & ZONE 8 & 80-84 & 224.78 & 300.63 & 163.59 & HT-Direct \\ 
  Chad & ZONE 8 & 80-84 & 261.49 & 207.35 & 321.46 & RW2 \\ 
  Chad & ZONE 8 & 85-89 & 226.53 & 271.48 & 187.10 & HT-Direct \\ 
  Chad & ZONE 8 & 85-89 & 253.54 & 217.09 & 293.07 & RW2 \\ 
  Chad & ZONE 8 & 90-94 & 216.01 & 265.49 & 173.56 & HT-Direct \\ 
  Chad & ZONE 8 & 90-94 & 241.89 & 214.23 & 271.82 & RW2 \\ 
  Chad & ZONE 8 & 95-99 & 220.46 & 252.68 & 191.30 & HT-Direct \\ 
  Chad & ZONE 8 & 95-99 & 223.75 & 203.33 & 245.79 & RW2 \\ 
  Chad & ZONE 8 & 00-04 & 202.47 & 235.33 & 173.15 & HT-Direct \\ 
  Chad & ZONE 8 & 00-04 & 205.60 & 187.58 & 225.59 & RW2 \\ 
  Chad & ZONE 8 & 05-09 & 186.19 & 209.14 & 165.24 & HT-Direct \\ 
  Chad & ZONE 8 & 05-09 & 187.45 & 170.95 & 205.41 & RW2 \\ 
  Chad & ZONE 8 & 10-14 & 130.85 & 150.99 & 113.04 & HT-Direct \\ 
  Chad & ZONE 8 & 10-14 & 161.61 & 140.20 & 185.09 & RW2 \\ 
  Chad & ZONE 8 & 15-19 & 135.92 & 62.65 & 269.76 & RW2 \\ 
  Comoros & ALL & 80-84 & 154.72 & 150.37 & 159.08 & IHME \\ 
  Comoros & ALL & 80-84 & 164.34 & 141.77 & 189.72 & RW2 \\ 
  Comoros & ALL & 80-84 & 164.08 & 156.29 & 172.55 & UN \\ 
  Comoros & ALL & 85-89 & 131.45 & 127.70 & 135.50 & IHME \\ 
  Comoros & ALL & 85-89 & 136.56 & 116.60 & 159.15 & RW2 \\ 
  Comoros & ALL & 85-89 & 137.66 & 130.88 & 145.53 & UN \\ 
  Comoros & ALL & 90-94 & 113.33 & 109.47 & 116.99 & IHME \\ 
  Comoros & ALL & 90-94 & 117.71 & 101.57 & 136.09 & RW2 \\ 
  Comoros & ALL & 90-94 & 116.85 & 110.74 & 123.03 & UN \\ 
  Comoros & ALL & 95-99 & 95.92 & 90.91 & 101.18 & IHME \\ 
  Comoros & ALL & 95-99 & 102.93 & 73.27 & 142.05 & RW2 \\ 
  Comoros & ALL & 95-99 & 101.90 & 94.26 & 109.80 & UN \\ 
  Comoros & ALL & 00-04 & 77.83 & 71.47 & 84.91 & IHME \\ 
  Comoros & ALL & 00-04 & 94.29 & 42.54 & 195.48 & RW2 \\ 
  Comoros & ALL & 00-04 & 97.73 & 87.11 & 108.71 & UN \\ 
  Comoros & ALL & 05-09 & 62.98 & 55.76 & 71.17 & IHME \\ 
  Comoros & ALL & 05-09 & 89.76 & 35.41 & 208.83 & RW2 \\ 
  Comoros & ALL & 05-09 & 95.07 & 79.54 & 113.51 & UN \\ 
  Comoros & ALL & 10-14 & 50.62 & 43.93 & 57.81 & IHME \\ 
  Comoros & ALL & 10-14 & 87.45 & 44.68 & 163.48 & RW2 \\ 
  Comoros & ALL & 10-14 & 85.93 & 66.84 & 110.09 & UN \\ 
  Comoros & MOHELI & 80-84 & 164.52 & 248.69 & 104.87 & HT-Direct \\ 
  Comoros & MOHELI & 80-84 & 168.02 & 110.21 & 247.31 & RW2 \\ 
  Comoros & MOHELI & 85-89 & 122.96 & 207.33 & 69.89 & HT-Direct \\ 
  Comoros & MOHELI & 85-89 & 129.48 & 92.23 & 179.12 & RW2 \\ 
  Comoros & MOHELI & 90-94 & 72.72 & 113.18 & 45.97 & HT-Direct \\ 
  Comoros & MOHELI & 90-94 & 107.54 & 75.37 & 150.51 & RW2 \\ 
  Comoros & MOHELI & 95-99 & 66.63 & 98.47 & 44.58 & HT-Direct \\ 
  Comoros & MOHELI & 95-99 & 92.97 & 56.36 & 149.07 & RW2 \\ 
  Comoros & MOHELI & 00-04 & 41.40 & 62.21 & 27.35 & HT-Direct \\ 
  Comoros & MOHELI & 00-04 & 84.28 & 35.11 & 189.88 & RW2 \\ 
  Comoros & MOHELI & 05-09 & 27.35 & 50.73 & 14.58 & HT-Direct \\ 
  Comoros & MOHELI & 05-09 & 79.90 & 26.20 & 219.10 & RW2 \\ 
  Comoros & MOHELI & 10-14 & 70.35 & 146.95 & 32.18 & HT-Direct \\ 
  Comoros & MOHELI & 10-14 & 77.37 & 22.68 & 234.68 & RW2 \\ 
  Comoros & MOHELI & 15-19 & 75.07 & 8.29 & 451.39 & RW2 \\ 
  Comoros & NDZOUANI & 80-84 & 174.50 & 221.34 & 135.85 & HT-Direct \\ 
  Comoros & NDZOUANI & 80-84 & 185.78 & 147.35 & 232.28 & RW2 \\ 
  Comoros & NDZOUANI & 85-89 & 122.95 & 149.11 & 100.83 & HT-Direct \\ 
  Comoros & NDZOUANI & 85-89 & 144.22 & 120.36 & 171.61 & RW2 \\ 
  Comoros & NDZOUANI & 90-94 & 101.84 & 122.39 & 84.41 & HT-Direct \\ 
  Comoros & NDZOUANI & 90-94 & 119.33 & 98.13 & 144.33 & RW2 \\ 
  Comoros & NDZOUANI & 95-99 & 58.10 & 81.23 & 41.27 & HT-Direct \\ 
  Comoros & NDZOUANI & 95-99 & 101.40 & 69.26 & 146.00 & RW2 \\ 
  Comoros & NDZOUANI & 00-04 & 52.66 & 76.39 & 36.01 & HT-Direct \\ 
  Comoros & NDZOUANI & 00-04 & 89.86 & 41.92 & 182.45 & RW2 \\ 
  Comoros & NDZOUANI & 05-09 & 42.27 & 63.31 & 28.01 & HT-Direct \\ 
  Comoros & NDZOUANI & 05-09 & 82.89 & 33.10 & 192.37 & RW2 \\ 
  Comoros & NDZOUANI & 10-14 & 55.60 & 88.31 & 34.55 & HT-Direct \\ 
  Comoros & NDZOUANI & 10-14 & 78.26 & 32.99 & 174.79 & RW2 \\ 
  Comoros & NDZOUANI & 15-19 & 73.86 & 11.22 & 365.63 & RW2 \\ 
  Comoros & NGAZIDJA & 80-84 & 137.42 & 159.67 & 117.84 & HT-Direct \\ 
  Comoros & NGAZIDJA & 80-84 & 146.63 & 125.37 & 170.73 & RW2 \\ 
  Comoros & NGAZIDJA & 85-89 & 105.57 & 129.30 & 85.77 & HT-Direct \\ 
  Comoros & NGAZIDJA & 85-89 & 124.42 & 103.95 & 148.14 & RW2 \\ 
  Comoros & NGAZIDJA & 90-94 & 94.66 & 112.25 & 79.58 & HT-Direct \\ 
  Comoros & NGAZIDJA & 90-94 & 113.53 & 94.68 & 135.93 & RW2 \\ 
  Comoros & NGAZIDJA & 95-99 & 80.42 & 105.43 & 60.93 & HT-Direct \\ 
  Comoros & NGAZIDJA & 95-99 & 106.48 & 74.67 & 149.49 & RW2 \\ 
  Comoros & NGAZIDJA & 00-04 & 51.51 & 69.91 & 37.75 & HT-Direct \\ 
  Comoros & NGAZIDJA & 00-04 & 102.91 & 49.45 & 201.94 & RW2 \\ 
  Comoros & NGAZIDJA & 05-09 & 57.07 & 76.20 & 42.52 & HT-Direct \\ 
  Comoros & NGAZIDJA & 05-09 & 102.84 & 42.91 & 225.84 & RW2 \\ 
  Comoros & NGAZIDJA & 10-14 & 59.22 & 90.94 & 38.10 & HT-Direct \\ 
  Comoros & NGAZIDJA & 10-14 & 104.23 & 46.46 & 215.53 & RW2 \\ 
  Comoros & NGAZIDJA & 15-19 & 105.93 & 16.93 & 452.55 & RW2 \\ 
  Congo & ALL & 80-84 & 112.06 & 107.16 & 117.11 & IHME \\ 
  Congo & ALL & 80-84 & 106.72 & 87.03 & 130.28 & RW2 \\ 
  Congo & ALL & 80-84 & 107.21 & 97.73 & 116.73 & UN \\ 
  Congo & ALL & 85-89 & 96.31 & 93.24 & 99.30 & IHME \\ 
  Congo & ALL & 85-89 & 95.22 & 83.68 & 108.10 & RW2 \\ 
  Congo & ALL & 85-89 & 94.68 & 88.56 & 101.05 & UN \\ 
  Congo & ALL & 90-94 & 95.27 & 92.60 & 97.85 & IHME \\ 
  Congo & ALL & 90-94 & 97.68 & 87.66 & 108.59 & RW2 \\ 
  Congo & ALL & 90-94 & 98.32 & 93.04 & 103.57 & UN \\ 
  Congo & ALL & 95-99 & 108.34 & 104.24 & 112.75 & IHME \\ 
  Congo & ALL & 95-99 & 116.21 & 107.24 & 125.76 & RW2 \\ 
  Congo & ALL & 95-99 & 115.58 & 110.27 & 120.96 & UN \\ 
  Congo & ALL & 00-04 & 103.83 & 100.74 & 106.98 & IHME \\ 
  Congo & ALL & 00-04 & 113.89 & 105.82 & 122.58 & RW2 \\ 
  Congo & ALL & 00-04 & 113.93 & 108.87 & 118.74 & UN \\ 
  Congo & ALL & 05-09 & 79.42 & 76.79 & 81.99 & IHME \\ 
  Congo & ALL & 05-09 & 79.17 & 70.08 & 89.16 & RW2 \\ 
  Congo & ALL & 05-09 & 79.66 & 75.41 & 84.77 & UN \\ 
  Congo & ALL & 10-14 & 62.66 & 58.56 & 67.24 & IHME \\ 
  Congo & ALL & 10-14 & 54.21 & 45.46 & 64.47 & RW2 \\ 
  Congo & ALL & 10-14 & 53.17 & 47.00 & 60.03 & UN \\ 
  Congo & BRAZZAVILLE & 80-84 & 80.37 & 121.51 & 52.32 & HT-Direct \\ 
  Congo & BRAZZAVILLE & 80-84 & 75.19 & 54.51 & 102.75 & RW2 \\ 
  Congo & BRAZZAVILLE & 85-89 & 64.78 & 86.74 & 48.08 & HT-Direct \\ 
  Congo & BRAZZAVILLE & 85-89 & 68.99 & 56.20 & 84.46 & RW2 \\ 
  Congo & BRAZZAVILLE & 90-94 & 61.65 & 79.69 & 47.48 & HT-Direct \\ 
  Congo & BRAZZAVILLE & 90-94 & 74.61 & 63.83 & 86.99 & RW2 \\ 
  Congo & BRAZZAVILLE & 95-99 & 128.71 & 151.23 & 109.11 & HT-Direct \\ 
  Congo & BRAZZAVILLE & 95-99 & 103.59 & 92.41 & 116.20 & RW2 \\ 
  Congo & BRAZZAVILLE & 00-04 & 93.21 & 110.92 & 78.08 & HT-Direct \\ 
  Congo & BRAZZAVILLE & 00-04 & 105.74 & 93.25 & 119.55 & RW2 \\ 
  Congo & BRAZZAVILLE & 05-09 & 68.38 & 90.44 & 51.40 & HT-Direct \\ 
  Congo & BRAZZAVILLE & 05-09 & 83.21 & 68.79 & 99.98 & RW2 \\ 
  Congo & BRAZZAVILLE & 10-14 & 80.14 & 127.51 & 49.37 & HT-Direct \\ 
  Congo & BRAZZAVILLE & 10-14 & 55.01 & 41.03 & 74.01 & RW2 \\ 
  Congo & BRAZZAVILLE & 15-19 & 34.40 & 11.71 & 97.51 & RW2 \\ 
  Congo & NORD & 80-84 & 119.86 & 164.48 & 86.10 & HT-Direct \\ 
  Congo & NORD & 80-84 & 123.68 & 95.56 & 159.28 & RW2 \\ 
  Congo & NORD & 85-89 & 108.94 & 133.69 & 88.30 & HT-Direct \\ 
  Congo & NORD & 85-89 & 104.64 & 89.35 & 122.42 & RW2 \\ 
  Congo & NORD & 90-94 & 94.11 & 111.34 & 79.30 & HT-Direct \\ 
  Congo & NORD & 90-94 & 102.75 & 90.80 & 115.70 & RW2 \\ 
  Congo & NORD & 95-99 & 142.46 & 160.16 & 126.42 & HT-Direct \\ 
  Congo & NORD & 95-99 & 130.48 & 119.58 & 142.10 & RW2 \\ 
  Congo & NORD & 00-04 & 122.69 & 133.67 & 112.49 & HT-Direct \\ 
  Congo & NORD & 00-04 & 126.53 & 117.51 & 136.33 & RW2 \\ 
  Congo & NORD & 05-09 & 92.19 & 104.62 & 81.09 & HT-Direct \\ 
  Congo & NORD & 05-09 & 93.22 & 83.15 & 104.51 & RW2 \\ 
  Congo & NORD & 10-14 & 60.70 & 77.28 & 47.50 & HT-Direct \\ 
  Congo & NORD & 10-14 & 54.59 & 45.70 & 64.95 & RW2 \\ 
  Congo & NORD & 15-19 & 29.42 & 10.58 & 77.40 & RW2 \\ 
  Congo & POINTE NOIRE & 80-84 & 93.52 & 142.88 & 60.02 & HT-Direct \\ 
  Congo & POINTE NOIRE & 80-84 & 110.51 & 80.47 & 151.23 & RW2 \\ 
  Congo & POINTE NOIRE & 85-89 & 105.98 & 142.81 & 77.79 & HT-Direct \\ 
  Congo & POINTE NOIRE & 85-89 & 90.97 & 73.50 & 112.29 & RW2 \\ 
  Congo & POINTE NOIRE & 90-94 & 79.97 & 103.85 & 61.21 & HT-Direct \\ 
  Congo & POINTE NOIRE & 90-94 & 85.81 & 72.95 & 100.87 & RW2 \\ 
  Congo & POINTE NOIRE & 95-99 & 115.66 & 140.18 & 94.96 & HT-Direct \\ 
  Congo & POINTE NOIRE & 95-99 & 103.98 & 91.35 & 117.86 & RW2 \\ 
  Congo & POINTE NOIRE & 00-04 & 88.65 & 108.93 & 71.84 & HT-Direct \\ 
  Congo & POINTE NOIRE & 00-04 & 96.18 & 83.30 & 110.19 & RW2 \\ 
  Congo & POINTE NOIRE & 05-09 & 52.13 & 69.27 & 39.05 & HT-Direct \\ 
  Congo & POINTE NOIRE & 05-09 & 69.52 & 57.36 & 83.91 & RW2 \\ 
  Congo & POINTE NOIRE & 10-14 & 68.53 & 105.75 & 43.78 & HT-Direct \\ 
  Congo & POINTE NOIRE & 10-14 & 42.53 & 31.87 & 58.01 & RW2 \\ 
  Congo & POINTE NOIRE & 15-19 & 24.68 & 8.39 & 72.50 & RW2 \\ 
  Congo & SUD & 80-84 & 131.98 & 175.90 & 97.72 & HT-Direct \\ 
  Congo & SUD & 80-84 & 139.28 & 109.73 & 174.23 & RW2 \\ 
  Congo & SUD & 85-89 & 100.64 & 126.63 & 79.50 & HT-Direct \\ 
  Congo & SUD & 85-89 & 115.62 & 98.43 & 135.04 & RW2 \\ 
  Congo & SUD & 90-94 & 120.06 & 139.19 & 103.25 & HT-Direct \\ 
  Congo & SUD & 90-94 & 110.69 & 98.74 & 123.93 & RW2 \\ 
  Congo & SUD & 95-99 & 133.53 & 152.73 & 116.40 & HT-Direct \\ 
  Congo & SUD & 95-99 & 134.12 & 122.67 & 146.66 & RW2 \\ 
  Congo & SUD & 00-04 & 133.71 & 150.67 & 118.39 & HT-Direct \\ 
  Congo & SUD & 00-04 & 122.96 & 112.37 & 134.60 & RW2 \\ 
  Congo & SUD & 05-09 & 72.61 & 85.61 & 61.46 & HT-Direct \\ 
  Congo & SUD & 05-09 & 85.04 & 74.82 & 96.63 & RW2 \\ 
  Congo & SUD & 10-14 & 55.61 & 71.04 & 43.37 & HT-Direct \\ 
  Congo & SUD & 10-14 & 47.91 & 40.19 & 56.91 & RW2 \\ 
  Congo & SUD & 15-19 & 25.11 & 9.06 & 66.07 & RW2 \\ 
  C\^{o}te d'Ivoire & ALL & 80-84 & 155.44 & 153.04 & 157.92 & IHME \\ 
  C\^{o}te d'Ivoire & ALL & 80-84 & 158.39 & 114.30 & 215.37 & RW2 \\ 
  C\^{o}te d'Ivoire & ALL & 80-84 & 160.36 & 154.83 & 165.77 & UN \\ 
  C\^{o}te d'Ivoire & ALL & 85-89 & 149.63 & 147.56 & 151.83 & IHME \\ 
  C\^{o}te d'Ivoire & ALL & 85-89 & 155.10 & 130.26 & 183.60 & RW2 \\ 
  C\^{o}te d'Ivoire & ALL & 85-89 & 152.92 & 148.31 & 157.65 & UN \\ 
  C\^{o}te d'Ivoire & ALL & 90-94 & 148.94 & 146.74 & 151.09 & IHME \\ 
  C\^{o}te d'Ivoire & ALL & 90-94 & 152.43 & 135.08 & 171.50 & RW2 \\ 
  C\^{o}te d'Ivoire & ALL & 90-94 & 153.01 & 148.26 & 157.66 & UN \\ 
  C\^{o}te d'Ivoire & ALL & 95-99 & 145.95 & 143.33 & 148.31 & IHME \\ 
  C\^{o}te d'Ivoire & ALL & 95-99 & 150.77 & 136.70 & 165.94 & RW2 \\ 
  C\^{o}te d'Ivoire & ALL & 95-99 & 150.89 & 145.62 & 156.27 & UN \\ 
  C\^{o}te d'Ivoire & ALL & 00-04 & 135.89 & 133.26 & 138.59 & IHME \\ 
  C\^{o}te d'Ivoire & ALL & 00-04 & 139.70 & 127.98 & 152.38 & RW2 \\ 
  C\^{o}te d'Ivoire & ALL & 00-04 & 139.57 & 133.91 & 145.15 & UN \\ 
  C\^{o}te d'Ivoire & ALL & 05-09 & 121.06 & 117.85 & 124.21 & IHME \\ 
  C\^{o}te d'Ivoire & ALL & 05-09 & 121.06 & 111.56 & 131.24 & RW2 \\ 
  C\^{o}te d'Ivoire & ALL & 05-09 & 121.06 & 115.65 & 126.46 & UN \\ 
  C\^{o}te d'Ivoire & ALL & 10-14 & 104.54 & 100.31 & 108.75 & IHME \\ 
  C\^{o}te d'Ivoire & ALL & 10-14 & 102.70 & 85.19 & 123.18 & RW2 \\ 
  C\^{o}te d'Ivoire & ALL & 10-14 & 102.89 & 94.95 & 110.57 & UN \\ 
  C\^{o}te d'Ivoire & CENTRE & 80-84 & 160.59 & 265.44 & 91.97 & HT-Direct \\ 
  C\^{o}te d'Ivoire & CENTRE & 80-84 & 168.41 & 106.67 & 257.22 & RW2 \\ 
  C\^{o}te d'Ivoire & CENTRE & 85-89 & 119.37 & 202.67 & 67.42 & HT-Direct \\ 
  C\^{o}te d'Ivoire & CENTRE & 85-89 & 140.18 & 100.26 & 193.04 & RW2 \\ 
  C\^{o}te d'Ivoire & CENTRE & 90-94 & 137.82 & 210.27 & 87.56 & HT-Direct \\ 
  C\^{o}te d'Ivoire & CENTRE & 90-94 & 139.94 & 109.68 & 176.96 & RW2 \\ 
  C\^{o}te d'Ivoire & CENTRE & 95-99 & 125.98 & 163.04 & 96.38 & HT-Direct \\ 
  C\^{o}te d'Ivoire & CENTRE & 95-99 & 148.69 & 125.26 & 175.23 & RW2 \\ 
  C\^{o}te d'Ivoire & CENTRE & 00-04 & 158.78 & 199.53 & 125.06 & HT-Direct \\ 
  C\^{o}te d'Ivoire & CENTRE & 00-04 & 153.45 & 131.88 & 177.89 & RW2 \\ 
  C\^{o}te d'Ivoire & CENTRE & 05-09 & 152.11 & 193.31 & 118.41 & HT-Direct \\ 
  C\^{o}te d'Ivoire & CENTRE & 05-09 & 141.61 & 115.94 & 171.82 & RW2 \\ 
  C\^{o}te d'Ivoire & CENTRE & 10-14 & 61.13 & 139.39 & 25.50 & HT-Direct \\ 
  C\^{o}te d'Ivoire & CENTRE & 10-14 & 131.33 & 92.27 & 183.38 & RW2 \\ 
  C\^{o}te d'Ivoire & CENTRE & 15-19 & 123.63 & 45.03 & 295.00 & RW2 \\ 
  C\^{o}te d'Ivoire & CENTRE-EST & 80-84 & 162.46 & 280.28 & 88.10 & HT-Direct \\ 
  C\^{o}te d'Ivoire & CENTRE-EST & 80-84 & 152.28 & 92.10 & 243.73 & RW2 \\ 
  C\^{o}te d'Ivoire & CENTRE-EST & 85-89 & 106.01 & 220.16 & 47.45 & HT-Direct \\ 
  C\^{o}te d'Ivoire & CENTRE-EST & 85-89 & 125.17 & 87.82 & 175.79 & RW2 \\ 
  C\^{o}te d'Ivoire & CENTRE-EST & 90-94 & 120.90 & 162.24 & 88.97 & HT-Direct \\ 
  C\^{o}te d'Ivoire & CENTRE-EST & 90-94 & 123.85 & 97.65 & 155.72 & RW2 \\ 
  C\^{o}te d'Ivoire & CENTRE-EST & 95-99 & 101.71 & 138.75 & 73.71 & HT-Direct \\ 
  C\^{o}te d'Ivoire & CENTRE-EST & 95-99 & 132.01 & 109.50 & 157.48 & RW2 \\ 
  C\^{o}te d'Ivoire & CENTRE-EST & 00-04 & 130.38 & 169.98 & 98.91 & HT-Direct \\ 
  C\^{o}te d'Ivoire & CENTRE-EST & 00-04 & 138.68 & 116.19 & 164.19 & RW2 \\ 
  C\^{o}te d'Ivoire & CENTRE-EST & 05-09 & 160.01 & 214.63 & 117.21 & HT-Direct \\ 
  C\^{o}te d'Ivoire & CENTRE-EST & 05-09 & 131.66 & 103.54 & 166.37 & RW2 \\ 
  C\^{o}te d'Ivoire & CENTRE-EST & 10-14 & 67.70 & 155.70 & 27.80 & HT-Direct \\ 
  C\^{o}te d'Ivoire & CENTRE-EST & 10-14 & 125.42 & 84.30 & 184.69 & RW2 \\ 
  C\^{o}te d'Ivoire & CENTRE-EST & 15-19 & 120.97 & 42.98 & 301.44 & RW2 \\ 
  C\^{o}te d'Ivoire & CENTRE-NORD & 80-84 & 127.06 & 189.30 & 83.19 & HT-Direct \\ 
  C\^{o}te d'Ivoire & CENTRE-NORD & 80-84 & 162.55 & 109.68 & 233.27 & RW2 \\ 
  C\^{o}te d'Ivoire & CENTRE-NORD & 85-89 & 116.88 & 176.47 & 75.57 & HT-Direct \\ 
  C\^{o}te d'Ivoire & CENTRE-NORD & 85-89 & 122.43 & 89.54 & 163.77 & RW2 \\ 
  C\^{o}te d'Ivoire & CENTRE-NORD & 90-94 & 82.78 & 137.49 & 48.62 & HT-Direct \\ 
  C\^{o}te d'Ivoire & CENTRE-NORD & 90-94 & 110.27 & 85.08 & 141.54 & RW2 \\ 
  C\^{o}te d'Ivoire & CENTRE-NORD & 95-99 & 132.57 & 218.07 & 77.27 & HT-Direct \\ 
  C\^{o}te d'Ivoire & CENTRE-NORD & 95-99 & 105.61 & 84.23 & 131.41 & RW2 \\ 
  C\^{o}te d'Ivoire & CENTRE-NORD & 00-04 & 71.74 & 109.00 & 46.55 & HT-Direct \\ 
  C\^{o}te d'Ivoire & CENTRE-NORD & 00-04 & 97.29 & 77.87 & 120.85 & RW2 \\ 
  C\^{o}te d'Ivoire & CENTRE-NORD & 05-09 & 81.70 & 117.29 & 56.22 & HT-Direct \\ 
  C\^{o}te d'Ivoire & CENTRE-NORD & 05-09 & 79.81 & 60.68 & 104.11 & RW2 \\ 
  C\^{o}te d'Ivoire & CENTRE-NORD & 10-14 & 57.26 & 112.87 & 28.17 & HT-Direct \\ 
  C\^{o}te d'Ivoire & CENTRE-NORD & 10-14 & 66.34 & 44.38 & 98.12 & RW2 \\ 
  C\^{o}te d'Ivoire & CENTRE-NORD & 15-19 & 55.93 & 19.71 & 151.53 & RW2 \\ 
  C\^{o}te d'Ivoire & CENTRE-OUEST & 80-84 & 53.46 & 189.78 & 13.44 & HT-Direct \\ 
  C\^{o}te d'Ivoire & CENTRE-OUEST & 80-84 & 174.79 & 99.11 & 288.65 & RW2 \\ 
  C\^{o}te d'Ivoire & CENTRE-OUEST & 85-89 & 95.55 & 203.46 & 41.86 & HT-Direct \\ 
  C\^{o}te d'Ivoire & CENTRE-OUEST & 85-89 & 135.56 & 91.25 & 197.01 & RW2 \\ 
  C\^{o}te d'Ivoire & CENTRE-OUEST & 90-94 & 143.56 & 197.48 & 102.48 & HT-Direct \\ 
  C\^{o}te d'Ivoire & CENTRE-OUEST & 90-94 & 124.87 & 95.59 & 161.67 & RW2 \\ 
  C\^{o}te d'Ivoire & CENTRE-OUEST & 95-99 & 92.43 & 129.04 & 65.42 & HT-Direct \\ 
  C\^{o}te d'Ivoire & CENTRE-OUEST & 95-99 & 120.62 & 99.04 & 146.20 & RW2 \\ 
  C\^{o}te d'Ivoire & CENTRE-OUEST & 00-04 & 119.58 & 168.48 & 83.46 & HT-Direct \\ 
  C\^{o}te d'Ivoire & CENTRE-OUEST & 00-04 & 111.23 & 93.04 & 132.57 & RW2 \\ 
  C\^{o}te d'Ivoire & CENTRE-OUEST & 05-09 & 84.60 & 110.52 & 64.33 & HT-Direct \\ 
  C\^{o}te d'Ivoire & CENTRE-OUEST & 05-09 & 90.84 & 72.46 & 113.01 & RW2 \\ 
  C\^{o}te d'Ivoire & CENTRE-OUEST & 10-14 & 70.92 & 139.11 & 34.81 & HT-Direct \\ 
  C\^{o}te d'Ivoire & CENTRE-OUEST & 10-14 & 75.03 & 51.53 & 107.52 & RW2 \\ 
  C\^{o}te d'Ivoire & CENTRE-OUEST & 15-19 & 63.23 & 22.53 & 165.02 & RW2 \\ 
  C\^{o}te d'Ivoire & NORD & 80-84 & 234.48 & 394.16 & 126.03 & HT-Direct \\ 
  C\^{o}te d'Ivoire & NORD & 80-84 & 247.92 & 158.70 & 364.52 & RW2 \\ 
  C\^{o}te d'Ivoire & NORD & 85-89 & 185.93 & 296.60 & 110.09 & HT-Direct \\ 
  C\^{o}te d'Ivoire & NORD & 85-89 & 213.44 & 156.04 & 283.80 & RW2 \\ 
  C\^{o}te d'Ivoire & NORD & 90-94 & 192.67 & 277.88 & 128.93 & HT-Direct \\ 
  C\^{o}te d'Ivoire & NORD & 90-94 & 217.05 & 174.60 & 266.40 & RW2 \\ 
  C\^{o}te d'Ivoire & NORD & 95-99 & 223.74 & 273.84 & 180.52 & HT-Direct \\ 
  C\^{o}te d'Ivoire & NORD & 95-99 & 232.72 & 200.91 & 267.63 & RW2 \\ 
  C\^{o}te d'Ivoire & NORD & 00-04 & 244.98 & 289.95 & 204.97 & HT-Direct \\ 
  C\^{o}te d'Ivoire & NORD & 00-04 & 240.39 & 214.03 & 268.99 & RW2 \\ 
  C\^{o}te d'Ivoire & NORD & 05-09 & 211.03 & 246.95 & 179.10 & HT-Direct \\ 
  C\^{o}te d'Ivoire & NORD & 05-09 & 223.92 & 197.19 & 253.48 & RW2 \\ 
  C\^{o}te d'Ivoire & NORD & 10-14 & 211.81 & 274.95 & 159.96 & HT-Direct \\ 
  C\^{o}te d'Ivoire & NORD & 10-14 & 211.57 & 170.01 & 260.34 & RW2 \\ 
  C\^{o}te d'Ivoire & NORD & 15-19 & 202.53 & 85.46 & 413.40 & RW2 \\ 
  C\^{o}te d'Ivoire & NORD-EST & 80-84 & 274.70 & 461.90 & 143.19 & HT-Direct \\ 
  C\^{o}te d'Ivoire & NORD-EST & 80-84 & 216.17 & 129.74 & 340.78 & RW2 \\ 
  C\^{o}te d'Ivoire & NORD-EST & 85-89 & 117.48 & 224.25 & 57.76 & HT-Direct \\ 
  C\^{o}te d'Ivoire & NORD-EST & 85-89 & 167.35 & 115.35 & 236.99 & RW2 \\ 
  C\^{o}te d'Ivoire & NORD-EST & 90-94 & 182.93 & 269.00 & 119.89 & HT-Direct \\ 
  C\^{o}te d'Ivoire & NORD-EST & 90-94 & 153.76 & 118.15 & 197.33 & RW2 \\ 
  C\^{o}te d'Ivoire & NORD-EST & 95-99 & 118.32 & 157.40 & 87.93 & HT-Direct \\ 
  C\^{o}te d'Ivoire & NORD-EST & 95-99 & 151.15 & 124.25 & 182.14 & RW2 \\ 
  C\^{o}te d'Ivoire & NORD-EST & 00-04 & 145.72 & 199.23 & 104.71 & HT-Direct \\ 
  C\^{o}te d'Ivoire & NORD-EST & 00-04 & 145.94 & 120.96 & 175.15 & RW2 \\ 
  C\^{o}te d'Ivoire & NORD-EST & 05-09 & 129.59 & 191.22 & 85.72 & HT-Direct \\ 
  C\^{o}te d'Ivoire & NORD-EST & 05-09 & 127.42 & 99.84 & 160.99 & RW2 \\ 
  C\^{o}te d'Ivoire & NORD-EST & 10-14 & 116.29 & 179.33 & 73.42 & HT-Direct \\ 
  C\^{o}te d'Ivoire & NORD-EST & 10-14 & 112.60 & 78.96 & 158.41 & RW2 \\ 
  C\^{o}te d'Ivoire & NORD-EST & 15-19 & 100.67 & 37.67 & 246.80 & RW2 \\ 
  C\^{o}te d'Ivoire & NORD-OUEST & 80-84 & 224.51 & 321.45 & 150.33 & HT-Direct \\ 
  C\^{o}te d'Ivoire & NORD-OUEST & 80-84 & 226.71 & 158.30 & 311.51 & RW2 \\ 
  C\^{o}te d'Ivoire & NORD-OUEST & 85-89 & 168.60 & 235.20 & 117.95 & HT-Direct \\ 
  C\^{o}te d'Ivoire & NORD-OUEST & 85-89 & 195.99 & 153.51 & 246.58 & RW2 \\ 
  C\^{o}te d'Ivoire & NORD-OUEST & 90-94 & 144.36 & 194.88 & 105.22 & HT-Direct \\ 
  C\^{o}te d'Ivoire & NORD-OUEST & 90-94 & 200.70 & 167.86 & 238.36 & RW2 \\ 
  C\^{o}te d'Ivoire & NORD-OUEST & 95-99 & 226.21 & 272.04 & 186.12 & HT-Direct \\ 
  C\^{o}te d'Ivoire & NORD-OUEST & 95-99 & 215.87 & 190.17 & 244.69 & RW2 \\ 
  C\^{o}te d'Ivoire & NORD-OUEST & 00-04 & 223.28 & 254.77 & 194.67 & HT-Direct \\ 
  C\^{o}te d'Ivoire & NORD-OUEST & 00-04 & 220.26 & 198.75 & 243.46 & RW2 \\ 
  C\^{o}te d'Ivoire & NORD-OUEST & 05-09 & 184.22 & 219.80 & 153.28 & HT-Direct \\ 
  C\^{o}te d'Ivoire & NORD-OUEST & 05-09 & 200.06 & 174.52 & 228.26 & RW2 \\ 
  C\^{o}te d'Ivoire & NORD-OUEST & 10-14 & 183.99 & 260.85 & 125.91 & HT-Direct \\ 
  C\^{o}te d'Ivoire & NORD-OUEST & 10-14 & 184.08 & 142.92 & 232.85 & RW2 \\ 
  C\^{o}te d'Ivoire & NORD-OUEST & 15-19 & 171.47 & 69.78 & 366.07 & RW2 \\ 
  C\^{o}te d'Ivoire & OUEST & 80-84 & 135.65 & 295.00 & 55.59 & HT-Direct \\ 
  C\^{o}te d'Ivoire & OUEST & 80-84 & 319.33 & 201.30 & 448.68 & RW2 \\ 
  C\^{o}te d'Ivoire & OUEST & 85-89 & 226.42 & 334.19 & 145.79 & HT-Direct \\ 
  C\^{o}te d'Ivoire & OUEST & 85-89 & 253.74 & 191.66 & 323.72 & RW2 \\ 
  C\^{o}te d'Ivoire & OUEST & 90-94 & 207.04 & 257.76 & 164.09 & HT-Direct \\ 
  C\^{o}te d'Ivoire & OUEST & 90-94 & 231.59 & 195.31 & 272.27 & RW2 \\ 
  C\^{o}te d'Ivoire & OUEST & 95-99 & 220.44 & 256.47 & 188.19 & HT-Direct \\ 
  C\^{o}te d'Ivoire & OUEST & 95-99 & 217.31 & 192.95 & 244.81 & RW2 \\ 
  C\^{o}te d'Ivoire & OUEST & 00-04 & 193.03 & 224.33 & 165.17 & HT-Direct \\ 
  C\^{o}te d'Ivoire & OUEST & 00-04 & 190.30 & 170.03 & 212.67 & RW2 \\ 
  C\^{o}te d'Ivoire & OUEST & 05-09 & 123.79 & 153.40 & 99.22 & HT-Direct \\ 
  C\^{o}te d'Ivoire & OUEST & 05-09 & 146.19 & 123.99 & 171.27 & RW2 \\ 
  C\^{o}te d'Ivoire & OUEST & 10-14 & 132.87 & 208.99 & 81.61 & HT-Direct \\ 
  C\^{o}te d'Ivoire & OUEST & 10-14 & 113.42 & 83.08 & 151.76 & RW2 \\ 
  C\^{o}te d'Ivoire & OUEST & 15-19 & 88.99 & 32.88 & 219.30 & RW2 \\ 
  C\^{o}te d'Ivoire & SUD SANS ABIDJAN & 80-84 & 93.77 & 285.60 & 26.08 & HT-Direct \\ 
  C\^{o}te d'Ivoire & SUD SANS ABIDJAN & 80-84 & 192.62 & 112.24 & 309.48 & RW2 \\ 
  C\^{o}te d'Ivoire & SUD SANS ABIDJAN & 85-89 & 138.81 & 247.04 & 73.37 & HT-Direct \\ 
  C\^{o}te d'Ivoire & SUD SANS ABIDJAN & 85-89 & 152.46 & 105.12 & 215.84 & RW2 \\ 
  C\^{o}te d'Ivoire & SUD SANS ABIDJAN & 90-94 & 126.90 & 181.23 & 87.13 & HT-Direct \\ 
  C\^{o}te d'Ivoire & SUD SANS ABIDJAN & 90-94 & 142.81 & 110.56 & 182.44 & RW2 \\ 
  C\^{o}te d'Ivoire & SUD SANS ABIDJAN & 95-99 & 137.66 & 181.86 & 102.86 & HT-Direct \\ 
  C\^{o}te d'Ivoire & SUD SANS ABIDJAN & 95-99 & 141.15 & 117.83 & 168.01 & RW2 \\ 
  C\^{o}te d'Ivoire & SUD SANS ABIDJAN & 00-04 & 139.43 & 180.57 & 106.45 & HT-Direct \\ 
  C\^{o}te d'Ivoire & SUD SANS ABIDJAN & 00-04 & 134.73 & 115.82 & 156.35 & RW2 \\ 
  C\^{o}te d'Ivoire & SUD SANS ABIDJAN & 05-09 & 113.15 & 138.47 & 91.96 & HT-Direct \\ 
  C\^{o}te d'Ivoire & SUD SANS ABIDJAN & 05-09 & 114.67 & 96.08 & 136.44 & RW2 \\ 
  C\^{o}te d'Ivoire & SUD SANS ABIDJAN & 10-14 & 70.40 & 130.08 & 36.94 & HT-Direct \\ 
  C\^{o}te d'Ivoire & SUD SANS ABIDJAN & 10-14 & 98.13 & 70.65 & 134.37 & RW2 \\ 
  C\^{o}te d'Ivoire & SUD SANS ABIDJAN & 15-19 & 84.60 & 30.86 & 212.76 & RW2 \\ 
  C\^{o}te d'Ivoire & SUD-OUEST & 80-84 & 102.08 & 326.44 & 25.97 & HT-Direct \\ 
  C\^{o}te d'Ivoire & SUD-OUEST & 80-84 & 196.26 & 107.84 & 329.72 & RW2 \\ 
  C\^{o}te d'Ivoire & SUD-OUEST & 85-89 & 152.51 & 255.09 & 86.40 & HT-Direct \\ 
  C\^{o}te d'Ivoire & SUD-OUEST & 85-89 & 145.75 & 95.60 & 216.84 & RW2 \\ 
  C\^{o}te d'Ivoire & SUD-OUEST & 90-94 & 122.21 & 189.03 & 76.77 & HT-Direct \\ 
  C\^{o}te d'Ivoire & SUD-OUEST & 90-94 & 127.13 & 92.40 & 172.24 & RW2 \\ 
  C\^{o}te d'Ivoire & SUD-OUEST & 95-99 & 123.36 & 234.63 & 60.68 & HT-Direct \\ 
  C\^{o}te d'Ivoire & SUD-OUEST & 95-99 & 116.32 & 89.99 & 148.95 & RW2 \\ 
  C\^{o}te d'Ivoire & SUD-OUEST & 00-04 & 65.35 & 107.39 & 39.05 & HT-Direct \\ 
  C\^{o}te d'Ivoire & SUD-OUEST & 00-04 & 101.58 & 81.65 & 125.22 & RW2 \\ 
  C\^{o}te d'Ivoire & SUD-OUEST & 05-09 & 78.68 & 101.25 & 60.80 & HT-Direct \\ 
  C\^{o}te d'Ivoire & SUD-OUEST & 05-09 & 79.32 & 63.23 & 99.19 & RW2 \\ 
  C\^{o}te d'Ivoire & SUD-OUEST & 10-14 & 78.02 & 156.78 & 37.09 & HT-Direct \\ 
  C\^{o}te d'Ivoire & SUD-OUEST & 10-14 & 63.23 & 42.98 & 92.49 & RW2 \\ 
  C\^{o}te d'Ivoire & SUD-OUEST & 15-19 & 51.19 & 17.60 & 141.52 & RW2 \\ 
  C\^{o}te d'Ivoire & VILLE D'ABIDJAN & 80-84 & 125.46 & 286.42 & 48.77 & HT-Direct \\ 
  C\^{o}te d'Ivoire & VILLE D'ABIDJAN & 80-84 & 151.11 & 82.69 & 268.15 & RW2 \\ 
  C\^{o}te d'Ivoire & VILLE D'ABIDJAN & 85-89 & 133.37 & 225.15 & 75.36 & HT-Direct \\ 
  C\^{o}te d'Ivoire & VILLE D'ABIDJAN & 85-89 & 116.44 & 78.37 & 170.86 & RW2 \\ 
  C\^{o}te d'Ivoire & VILLE D'ABIDJAN & 90-94 & 88.03 & 135.03 & 56.33 & HT-Direct \\ 
  C\^{o}te d'Ivoire & VILLE D'ABIDJAN & 90-94 & 107.24 & 81.50 & 140.09 & RW2 \\ 
  C\^{o}te d'Ivoire & VILLE D'ABIDJAN & 95-99 & 103.45 & 138.84 & 76.28 & HT-Direct \\ 
  C\^{o}te d'Ivoire & VILLE D'ABIDJAN & 95-99 & 106.28 & 85.81 & 130.26 & RW2 \\ 
  C\^{o}te d'Ivoire & VILLE D'ABIDJAN & 00-04 & 83.74 & 117.33 & 59.13 & HT-Direct \\ 
  C\^{o}te d'Ivoire & VILLE D'ABIDJAN & 00-04 & 104.21 & 85.62 & 125.35 & RW2 \\ 
  C\^{o}te d'Ivoire & VILLE D'ABIDJAN & 05-09 & 99.29 & 128.09 & 76.39 & HT-Direct \\ 
  C\^{o}te d'Ivoire & VILLE D'ABIDJAN & 05-09 & 93.50 & 75.83 & 115.18 & RW2 \\ 
  C\^{o}te d'Ivoire & VILLE D'ABIDJAN & 10-14 & 80.05 & 132.59 & 47.20 & HT-Direct \\ 
  C\^{o}te d'Ivoire & VILLE D'ABIDJAN & 10-14 & 85.32 & 59.22 & 122.63 & RW2 \\ 
  C\^{o}te d'Ivoire & VILLE D'ABIDJAN & 15-19 & 78.87 & 27.29 & 211.63 & RW2 \\ 
  DRC & ALL & 80-84 & 177.53 & 170.43 & 184.58 & IHME \\ 
  DRC & ALL & 80-84 & 206.56 & 169.75 & 249.04 & RW2 \\ 
  DRC & ALL & 80-84 & 205.54 & 194.81 & 217.49 & UN \\ 
  DRC & ALL & 85-89 & 161.02 & 156.18 & 166.64 & IHME \\ 
  DRC & ALL & 85-89 & 190.70 & 167.27 & 216.23 & RW2 \\ 
  DRC & ALL & 85-89 & 192.63 & 184.31 & 202.27 & UN \\ 
  DRC & ALL & 90-94 & 165.89 & 160.54 & 171.62 & IHME \\ 
  DRC & ALL & 90-94 & 183.69 & 167.32 & 201.32 & RW2 \\ 
  DRC & ALL & 90-94 & 182.13 & 174.46 & 190.42 & UN \\ 
  DRC & ALL & 95-99 & 162.41 & 157.27 & 167.95 & IHME \\ 
  DRC & ALL & 95-99 & 171.81 & 159.01 & 185.31 & RW2 \\ 
  DRC & ALL & 95-99 & 171.66 & 164.49 & 179.34 & UN \\ 
  DRC & ALL & 00-04 & 142.43 & 137.67 & 147.57 & IHME \\ 
  DRC & ALL & 00-04 & 151.93 & 142.57 & 161.84 & RW2 \\ 
  DRC & ALL & 00-04 & 152.22 & 146.01 & 158.37 & UN \\ 
  DRC & ALL & 05-09 & 120.74 & 116.11 & 125.82 & IHME \\ 
  DRC & ALL & 05-09 & 129.30 & 120.20 & 138.96 & RW2 \\ 
  DRC & ALL & 05-09 & 129.24 & 122.26 & 136.71 & UN \\ 
  DRC & ALL & 10-14 & 99.73 & 94.09 & 106.03 & IHME \\ 
  DRC & ALL & 10-14 & 107.92 & 98.21 & 118.39 & RW2 \\ 
  DRC & ALL & 10-14 & 107.92 & 97.62 & 119.22 & UN \\ 
  DRC & BANDUNDU & 80-84 & 151.31 & 208.25 & 107.82 & HT-Direct \\ 
  DRC & BANDUNDU & 80-84 & 238.22 & 178.63 & 309.96 & RW2 \\ 
  DRC & BANDUNDU & 85-89 & 153.26 & 238.72 & 94.59 & HT-Direct \\ 
  DRC & BANDUNDU & 85-89 & 207.61 & 166.61 & 256.08 & RW2 \\ 
  DRC & BANDUNDU & 90-94 & 204.52 & 253.73 & 162.77 & HT-Direct \\ 
  DRC & BANDUNDU & 90-94 & 177.66 & 151.75 & 207.07 & RW2 \\ 
  DRC & BANDUNDU & 95-99 & 162.80 & 210.22 & 124.39 & HT-Direct \\ 
  DRC & BANDUNDU & 95-99 & 161.33 & 141.94 & 182.78 & RW2 \\ 
  DRC & BANDUNDU & 00-04 & 117.68 & 141.98 & 97.07 & HT-Direct \\ 
  DRC & BANDUNDU & 00-04 & 133.39 & 118.42 & 149.94 & RW2 \\ 
  DRC & BANDUNDU & 05-09 & 100.27 & 122.57 & 81.65 & HT-Direct \\ 
  DRC & BANDUNDU & 05-09 & 101.42 & 87.91 & 116.68 & RW2 \\ 
  DRC & BANDUNDU & 10-14 & 72.77 & 95.81 & 54.93 & HT-Direct \\ 
  DRC & BANDUNDU & 10-14 & 79.50 & 64.53 & 97.65 & RW2 \\ 
  DRC & BANDUNDU & 15-19 & 63.28 & 26.72 & 141.66 & RW2 \\ 
  DRC & BAS-CONGO & 80-84 & 183.53 & 273.26 & 118.47 & HT-Direct \\ 
  DRC & BAS-CONGO & 80-84 & 258.13 & 184.46 & 345.98 & RW2 \\ 
  DRC & BAS-CONGO & 85-89 & 186.01 & 251.34 & 134.61 & HT-Direct \\ 
  DRC & BAS-CONGO & 85-89 & 235.01 & 187.22 & 290.60 & RW2 \\ 
  DRC & BAS-CONGO & 90-94 & 226.00 & 295.06 & 169.23 & HT-Direct \\ 
  DRC & BAS-CONGO & 90-94 & 210.71 & 179.20 & 246.14 & RW2 \\ 
  DRC & BAS-CONGO & 95-99 & 210.85 & 253.40 & 173.78 & HT-Direct \\ 
  DRC & BAS-CONGO & 95-99 & 200.69 & 178.06 & 225.47 & RW2 \\ 
  DRC & BAS-CONGO & 00-04 & 164.02 & 190.52 & 140.56 & HT-Direct \\ 
  DRC & BAS-CONGO & 00-04 & 175.05 & 156.16 & 195.41 & RW2 \\ 
  DRC & BAS-CONGO & 05-09 & 131.12 & 168.66 & 100.92 & HT-Direct \\ 
  DRC & BAS-CONGO & 05-09 & 141.51 & 120.41 & 165.66 & RW2 \\ 
  DRC & BAS-CONGO & 10-14 & 117.10 & 163.64 & 82.50 & HT-Direct \\ 
  DRC & BAS-CONGO & 10-14 & 118.36 & 92.14 & 151.30 & RW2 \\ 
  DRC & BAS-CONGO & 15-19 & 100.38 & 42.18 & 222.50 & RW2 \\ 
  DRC & EQUATEUR & 80-84 & 141.17 & 208.45 & 93.05 & HT-Direct \\ 
  DRC & EQUATEUR & 80-84 & 190.05 & 139.32 & 253.49 & RW2 \\ 
  DRC & EQUATEUR & 85-89 & 160.67 & 206.67 & 123.31 & HT-Direct \\ 
  DRC & EQUATEUR & 85-89 & 184.06 & 147.56 & 226.12 & RW2 \\ 
  DRC & EQUATEUR & 90-94 & 217.44 & 293.70 & 156.59 & HT-Direct \\ 
  DRC & EQUATEUR & 90-94 & 175.86 & 148.75 & 206.62 & RW2 \\ 
  DRC & EQUATEUR & 95-99 & 166.05 & 217.22 & 125.01 & HT-Direct \\ 
  DRC & EQUATEUR & 95-99 & 180.37 & 158.03 & 204.95 & RW2 \\ 
  DRC & EQUATEUR & 00-04 & 148.06 & 178.74 & 121.86 & HT-Direct \\ 
  DRC & EQUATEUR & 00-04 & 170.23 & 151.24 & 190.91 & RW2 \\ 
  DRC & EQUATEUR & 05-09 & 141.63 & 172.73 & 115.36 & HT-Direct \\ 
  DRC & EQUATEUR & 05-09 & 149.23 & 130.25 & 170.35 & RW2 \\ 
  DRC & EQUATEUR & 10-14 & 141.25 & 178.11 & 110.99 & HT-Direct \\ 
  DRC & EQUATEUR & 10-14 & 135.07 & 111.87 & 162.41 & RW2 \\ 
  DRC & EQUATEUR & 15-19 & 123.79 & 55.20 & 257.08 & RW2 \\ 
  DRC & KASAI-OCCIDENTAL & 80-84 & 251.38 & 359.52 & 167.27 & HT-Direct \\ 
  DRC & KASAI-OCCIDENTAL & 80-84 & 258.25 & 193.80 & 336.20 & RW2 \\ 
  DRC & KASAI-OCCIDENTAL & 85-89 & 197.53 & 254.22 & 150.93 & HT-Direct \\ 
  DRC & KASAI-OCCIDENTAL & 85-89 & 236.00 & 194.24 & 284.10 & RW2 \\ 
  DRC & KASAI-OCCIDENTAL & 90-94 & 189.70 & 237.17 & 149.85 & HT-Direct \\ 
  DRC & KASAI-OCCIDENTAL & 90-94 & 212.96 & 186.35 & 242.36 & RW2 \\ 
  DRC & KASAI-OCCIDENTAL & 95-99 & 210.95 & 239.83 & 184.69 & HT-Direct \\ 
  DRC & KASAI-OCCIDENTAL & 95-99 & 205.70 & 187.10 & 225.68 & RW2 \\ 
  DRC & KASAI-OCCIDENTAL & 00-04 & 177.00 & 210.87 & 147.55 & HT-Direct \\ 
  DRC & KASAI-OCCIDENTAL & 00-04 & 181.81 & 163.43 & 201.93 & RW2 \\ 
  DRC & KASAI-OCCIDENTAL & 05-09 & 145.26 & 176.94 & 118.44 & HT-Direct \\ 
  DRC & KASAI-OCCIDENTAL & 05-09 & 147.30 & 126.76 & 170.26 & RW2 \\ 
  DRC & KASAI-OCCIDENTAL & 10-14 & 105.02 & 152.58 & 71.04 & HT-Direct \\ 
  DRC & KASAI-OCCIDENTAL & 10-14 & 121.73 & 96.17 & 152.45 & RW2 \\ 
  DRC & KASAI-OCCIDENTAL & 15-19 & 101.96 & 43.76 & 218.88 & RW2 \\ 
  DRC & KASAI-ORIENTAL & 80-84 & 129.05 & 195.97 & 82.63 & HT-Direct \\ 
  DRC & KASAI-ORIENTAL & 80-84 & 204.83 & 155.91 & 264.46 & RW2 \\ 
  DRC & KASAI-ORIENTAL & 85-89 & 194.34 & 237.70 & 157.25 & HT-Direct \\ 
  DRC & KASAI-ORIENTAL & 85-89 & 192.61 & 161.34 & 228.02 & RW2 \\ 
  DRC & KASAI-ORIENTAL & 90-94 & 174.12 & 209.68 & 143.49 & HT-Direct \\ 
  DRC & KASAI-ORIENTAL & 90-94 & 178.70 & 158.42 & 200.88 & RW2 \\ 
  DRC & KASAI-ORIENTAL & 95-99 & 170.51 & 192.33 & 150.70 & HT-Direct \\ 
  DRC & KASAI-ORIENTAL & 95-99 & 178.31 & 163.47 & 194.01 & RW2 \\ 
  DRC & KASAI-ORIENTAL & 00-04 & 161.36 & 186.33 & 139.17 & HT-Direct \\ 
  DRC & KASAI-ORIENTAL & 00-04 & 163.00 & 148.17 & 179.05 & RW2 \\ 
  DRC & KASAI-ORIENTAL & 05-09 & 131.50 & 158.56 & 108.46 & HT-Direct \\ 
  DRC & KASAI-ORIENTAL & 05-09 & 136.37 & 119.80 & 155.07 & RW2 \\ 
  DRC & KASAI-ORIENTAL & 10-14 & 111.25 & 145.55 & 84.23 & HT-Direct \\ 
  DRC & KASAI-ORIENTAL & 10-14 & 116.76 & 96.26 & 141.12 & RW2 \\ 
  DRC & KASAI-ORIENTAL & 15-19 & 101.05 & 44.28 & 216.24 & RW2 \\ 
  DRC & KATANGA & 80-84 & 144.81 & 215.69 & 94.42 & HT-Direct \\ 
  DRC & KATANGA & 80-84 & 209.83 & 157.40 & 272.74 & RW2 \\ 
  DRC & KATANGA & 85-89 & 180.11 & 231.00 & 138.41 & HT-Direct \\ 
  DRC & KATANGA & 85-89 & 200.51 & 165.63 & 240.44 & RW2 \\ 
  DRC & KATANGA & 90-94 & 193.08 & 235.59 & 156.68 & HT-Direct \\ 
  DRC & KATANGA & 90-94 & 189.09 & 165.59 & 214.82 & RW2 \\ 
  DRC & KATANGA & 95-99 & 184.51 & 217.86 & 155.26 & HT-Direct \\ 
  DRC & KATANGA & 95-99 & 190.95 & 172.51 & 210.88 & RW2 \\ 
  DRC & KATANGA & 00-04 & 173.01 & 199.14 & 149.67 & HT-Direct \\ 
  DRC & KATANGA & 00-04 & 175.55 & 159.87 & 192.77 & RW2 \\ 
  DRC & KATANGA & 05-09 & 130.85 & 154.29 & 110.51 & HT-Direct \\ 
  DRC & KATANGA & 05-09 & 147.41 & 132.73 & 163.24 & RW2 \\ 
  DRC & KATANGA & 10-14 & 124.20 & 140.95 & 109.19 & HT-Direct \\ 
  DRC & KATANGA & 10-14 & 126.71 & 112.48 & 142.38 & RW2 \\ 
  DRC & KATANGA & 15-19 & 109.94 & 50.36 & 225.34 & RW2 \\ 
  DRC & KINSHASA & 80-84 & 147.77 & 205.11 & 104.36 & HT-Direct \\ 
  DRC & KINSHASA & 80-84 & 129.09 & 94.08 & 176.80 & RW2 \\ 
  DRC & KINSHASA & 85-89 & 87.95 & 118.44 & 64.72 & HT-Direct \\ 
  DRC & KINSHASA & 85-89 & 119.38 & 96.71 & 147.12 & RW2 \\ 
  DRC & KINSHASA & 90-94 & 105.20 & 129.03 & 85.34 & HT-Direct \\ 
  DRC & KINSHASA & 90-94 & 110.22 & 95.95 & 126.42 & RW2 \\ 
  DRC & KINSHASA & 95-99 & 106.92 & 125.85 & 90.55 & HT-Direct \\ 
  DRC & KINSHASA & 95-99 & 111.32 & 99.79 & 123.89 & RW2 \\ 
  DRC & KINSHASA & 00-04 & 103.03 & 118.45 & 89.40 & HT-Direct \\ 
  DRC & KINSHASA & 00-04 & 104.33 & 93.79 & 115.66 & RW2 \\ 
  DRC & KINSHASA & 05-09 & 78.25 & 98.19 & 62.07 & HT-Direct \\ 
  DRC & KINSHASA & 05-09 & 91.42 & 79.68 & 104.73 & RW2 \\ 
  DRC & KINSHASA & 10-14 & 86.87 & 109.00 & 68.88 & HT-Direct \\ 
  DRC & KINSHASA & 10-14 & 84.18 & 69.27 & 102.46 & RW2 \\ 
  DRC & KINSHASA & 15-19 & 79.04 & 34.00 & 175.08 & RW2 \\ 
  DRC & MANIEMA & 80-84 & 131.24 & 212.84 & 77.83 & HT-Direct \\ 
  DRC & MANIEMA & 80-84 & 187.13 & 134.22 & 251.84 & RW2 \\ 
  DRC & MANIEMA & 85-89 & 163.64 & 228.40 & 114.52 & HT-Direct \\ 
  DRC & MANIEMA & 85-89 & 184.65 & 147.46 & 227.57 & RW2 \\ 
  DRC & MANIEMA & 90-94 & 140.57 & 174.98 & 112.01 & HT-Direct \\ 
  DRC & MANIEMA & 90-94 & 180.41 & 155.86 & 207.78 & RW2 \\ 
  DRC & MANIEMA & 95-99 & 223.91 & 262.36 & 189.64 & HT-Direct \\ 
  DRC & MANIEMA & 95-99 & 188.86 & 169.30 & 210.80 & RW2 \\ 
  DRC & MANIEMA & 00-04 & 196.16 & 249.20 & 152.13 & HT-Direct \\ 
  DRC & MANIEMA & 00-04 & 177.16 & 157.92 & 199.00 & RW2 \\ 
  DRC & MANIEMA & 05-09 & 135.61 & 162.29 & 112.73 & HT-Direct \\ 
  DRC & MANIEMA & 05-09 & 150.01 & 131.02 & 171.13 & RW2 \\ 
  DRC & MANIEMA & 10-14 & 110.82 & 144.08 & 84.49 & HT-Direct \\ 
  DRC & MANIEMA & 10-14 & 128.89 & 105.46 & 156.15 & RW2 \\ 
  DRC & MANIEMA & 15-19 & 111.89 & 48.97 & 235.38 & RW2 \\ 
  DRC & NORD-KIVU & 80-84 & 239.97 & 330.11 & 168.25 & HT-Direct \\ 
  DRC & NORD-KIVU & 80-84 & 266.50 & 194.62 & 353.65 & RW2 \\ 
  DRC & NORD-KIVU & 85-89 & 127.40 & 189.88 & 83.36 & HT-Direct \\ 
  DRC & NORD-KIVU & 85-89 & 217.27 & 170.83 & 272.13 & RW2 \\ 
  DRC & NORD-KIVU & 90-94 & 186.16 & 246.78 & 137.70 & HT-Direct \\ 
  DRC & NORD-KIVU & 90-94 & 173.61 & 145.03 & 206.27 & RW2 \\ 
  DRC & NORD-KIVU & 95-99 & 144.53 & 183.51 & 112.68 & HT-Direct \\ 
  DRC & NORD-KIVU & 95-99 & 147.51 & 127.73 & 169.66 & RW2 \\ 
  DRC & NORD-KIVU & 00-04 & 90.39 & 124.57 & 64.90 & HT-Direct \\ 
  DRC & NORD-KIVU & 00-04 & 112.25 & 96.80 & 130.03 & RW2 \\ 
  DRC & NORD-KIVU & 05-09 & 80.82 & 99.85 & 65.15 & HT-Direct \\ 
  DRC & NORD-KIVU & 05-09 & 76.40 & 63.83 & 91.27 & RW2 \\ 
  DRC & NORD-KIVU & 10-14 & 36.08 & 59.54 & 21.65 & HT-Direct \\ 
  DRC & NORD-KIVU & 10-14 & 52.47 & 39.46 & 69.28 & RW2 \\ 
  DRC & NORD-KIVU & 15-19 & 36.14 & 14.40 & 87.93 & RW2 \\ 
  DRC & ORIENTALE & 80-84 & 193.65 & 312.05 & 112.80 & HT-Direct \\ 
  DRC & ORIENTALE & 80-84 & 227.06 & 163.58 & 306.55 & RW2 \\ 
  DRC & ORIENTALE & 85-89 & 189.78 & 250.12 & 141.26 & HT-Direct \\ 
  DRC & ORIENTALE & 85-89 & 208.56 & 166.13 & 259.10 & RW2 \\ 
  DRC & ORIENTALE & 90-94 & 162.38 & 219.21 & 118.06 & HT-Direct \\ 
  DRC & ORIENTALE & 90-94 & 189.17 & 160.27 & 221.73 & RW2 \\ 
  DRC & ORIENTALE & 95-99 & 192.78 & 234.29 & 157.11 & HT-Direct \\ 
  DRC & ORIENTALE & 95-99 & 184.36 & 163.06 & 207.83 & RW2 \\ 
  DRC & ORIENTALE & 00-04 & 154.71 & 187.77 & 126.56 & HT-Direct \\ 
  DRC & ORIENTALE & 00-04 & 163.19 & 145.35 & 182.85 & RW2 \\ 
  DRC & ORIENTALE & 05-09 & 124.07 & 150.55 & 101.69 & HT-Direct \\ 
  DRC & ORIENTALE & 05-09 & 131.66 & 114.85 & 150.63 & RW2 \\ 
  DRC & ORIENTALE & 10-14 & 105.48 & 135.52 & 81.47 & HT-Direct \\ 
  DRC & ORIENTALE & 10-14 & 108.41 & 88.78 & 131.56 & RW2 \\ 
  DRC & ORIENTALE & 15-19 & 90.03 & 39.07 & 193.50 & RW2 \\ 
  DRC & SUD-KIVU & 80-84 & 252.32 & 331.34 & 186.88 & HT-Direct \\ 
  DRC & SUD-KIVU & 80-84 & 267.83 & 200.97 & 347.48 & RW2 \\ 
  DRC & SUD-KIVU & 85-89 & 137.93 & 208.20 & 88.72 & HT-Direct \\ 
  DRC & SUD-KIVU & 85-89 & 247.32 & 198.86 & 302.94 & RW2 \\ 
  DRC & SUD-KIVU & 90-94 & 226.61 & 290.05 & 173.66 & HT-Direct \\ 
  DRC & SUD-KIVU & 90-94 & 226.50 & 193.71 & 263.07 & RW2 \\ 
  DRC & SUD-KIVU & 95-99 & 242.19 & 295.45 & 195.86 & HT-Direct \\ 
  DRC & SUD-KIVU & 95-99 & 221.20 & 195.76 & 249.20 & RW2 \\ 
  DRC & SUD-KIVU & 00-04 & 178.09 & 217.00 & 144.87 & HT-Direct \\ 
  DRC & SUD-KIVU & 00-04 & 195.81 & 173.74 & 220.14 & RW2 \\ 
  DRC & SUD-KIVU & 05-09 & 145.94 & 181.41 & 116.42 & HT-Direct \\ 
  DRC & SUD-KIVU & 05-09 & 157.36 & 136.20 & 181.48 & RW2 \\ 
  DRC & SUD-KIVU & 10-14 & 129.41 & 166.65 & 99.49 & HT-Direct \\ 
  DRC & SUD-KIVU & 10-14 & 128.61 & 104.39 & 157.25 & RW2 \\ 
  DRC & SUD-KIVU & 15-19 & 105.81 & 46.06 & 225.51 & RW2 \\ 
  Egypt & ALL & 80-84 & 141.99 & 138.72 & 145.18 & IHME \\ 
  Egypt & ALL & 80-84 & 147.64 & 143.17 & 152.23 & RW2 \\ 
  Egypt & ALL & 80-84 & 147.62 & 144.54 & 150.81 & UN \\ 
  Egypt & ALL & 85-89 & 97.53 & 95.24 & 99.97 & IHME \\ 
  Egypt & ALL & 85-89 & 103.18 & 99.96 & 106.47 & RW2 \\ 
  Egypt & ALL & 85-89 & 103.23 & 101.14 & 105.40 & UN \\ 
  Egypt & ALL & 90-94 & 74.53 & 72.62 & 76.55 & IHME \\ 
  Egypt & ALL & 90-94 & 77.72 & 75.04 & 80.49 & RW2 \\ 
  Egypt & ALL & 90-94 & 77.64 & 76.01 & 79.24 & UN \\ 
  Egypt & ALL & 95-99 & 52.51 & 50.86 & 54.20 & IHME \\ 
  Egypt & ALL & 95-99 & 56.79 & 54.17 & 59.51 & RW2 \\ 
  Egypt & ALL & 95-99 & 56.85 & 55.60 & 58.34 & UN \\ 
  Egypt & ALL & 00-04 & 38.05 & 36.67 & 39.38 & IHME \\ 
  Egypt & ALL & 00-04 & 41.53 & 38.97 & 44.26 & RW2 \\ 
  Egypt & ALL & 00-04 & 41.53 & 40.30 & 42.88 & UN \\ 
  Egypt & ALL & 05-09 & 31.32 & 29.78 & 32.92 & IHME \\ 
  Egypt & ALL & 05-09 & 32.82 & 29.91 & 36.01 & RW2 \\ 
  Egypt & ALL & 05-09 & 32.82 & 31.32 & 34.10 & UN \\ 
  Egypt & ALL & 10-14 & 25.79 & 24.14 & 27.47 & IHME \\ 
  Egypt & ALL & 10-14 & 26.43 & 9.81 & 68.75 & RW2 \\ 
  Egypt & ALL & 10-14 & 26.84 & 24.98 & 28.86 & UN \\ 
  Egypt & FRONTIER GOVERNORATES & 80-84 & 98.02 & 117.07 & 81.78 & HT-Direct \\ 
  Egypt & FRONTIER GOVERNORATES & 80-84 & 99.33 & 85.54 & 115.16 & RW2 \\ 
  Egypt & FRONTIER GOVERNORATES & 85-89 & 70.59 & 82.61 & 60.20 & HT-Direct \\ 
  Egypt & FRONTIER GOVERNORATES & 85-89 & 72.36 & 65.10 & 80.44 & RW2 \\ 
  Egypt & FRONTIER GOVERNORATES & 90-94 & 53.94 & 64.01 & 45.38 & HT-Direct \\ 
  Egypt & FRONTIER GOVERNORATES & 90-94 & 56.98 & 51.09 & 63.27 & RW2 \\ 
  Egypt & FRONTIER GOVERNORATES & 95-99 & 44.93 & 55.25 & 36.47 & HT-Direct \\ 
  Egypt & FRONTIER GOVERNORATES & 95-99 & 45.38 & 39.93 & 51.45 & RW2 \\ 
  Egypt & FRONTIER GOVERNORATES & 00-04 & 32.98 & 42.89 & 25.30 & HT-Direct \\ 
  Egypt & FRONTIER GOVERNORATES & 00-04 & 36.73 & 31.05 & 43.51 & RW2 \\ 
  Egypt & FRONTIER GOVERNORATES & 05-09 & 35.12 & 49.99 & 24.56 & HT-Direct \\ 
  Egypt & FRONTIER GOVERNORATES & 05-09 & 33.46 & 25.97 & 43.34 & RW2 \\ 
  Egypt & FRONTIER GOVERNORATES & 10-14 & 31.67 & 12.53 & 79.63 & RW2 \\ 
  Egypt & FRONTIER GOVERNORATES & 15-19 & 29.83 & 3.12 & 236.19 & RW2 \\ 
  Egypt & LOWER EGYPT & 80-84 & 122.40 & 128.61 & 116.45 & HT-Direct \\ 
  Egypt & LOWER EGYPT & 80-84 & 125.83 & 119.93 & 132.05 & RW2 \\ 
  Egypt & LOWER EGYPT & 85-89 & 83.34 & 87.71 & 79.18 & HT-Direct \\ 
  Egypt & LOWER EGYPT & 85-89 & 86.65 & 82.87 & 90.64 & RW2 \\ 
  Egypt & LOWER EGYPT & 90-94 & 63.23 & 67.30 & 59.40 & HT-Direct \\ 
  Egypt & LOWER EGYPT & 90-94 & 63.14 & 60.01 & 66.40 & RW2 \\ 
  Egypt & LOWER EGYPT & 95-99 & 42.66 & 46.44 & 39.18 & HT-Direct \\ 
  Egypt & LOWER EGYPT & 95-99 & 45.84 & 42.86 & 48.97 & RW2 \\ 
  Egypt & LOWER EGYPT & 00-04 & 32.65 & 36.53 & 29.17 & HT-Direct \\ 
  Egypt & LOWER EGYPT & 00-04 & 33.76 & 30.90 & 36.90 & RW2 \\ 
  Egypt & LOWER EGYPT & 05-09 & 26.74 & 31.53 & 22.66 & HT-Direct \\ 
  Egypt & LOWER EGYPT & 05-09 & 27.74 & 24.13 & 31.92 & RW2 \\ 
  Egypt & LOWER EGYPT & 10-14 & 23.57 & 9.64 & 56.81 & RW2 \\ 
  Egypt & LOWER EGYPT & 15-19 & 20.07 & 2.16 & 167.70 & RW2 \\ 
  Egypt & UPPER EGYPT & 80-84 & 195.32 & 202.99 & 187.86 & HT-Direct \\ 
  Egypt & UPPER EGYPT & 80-84 & 198.20 & 190.74 & 205.86 & RW2 \\ 
  Egypt & UPPER EGYPT & 85-89 & 127.49 & 132.51 & 122.64 & HT-Direct \\ 
  Egypt & UPPER EGYPT & 85-89 & 136.80 & 131.86 & 141.74 & RW2 \\ 
  Egypt & UPPER EGYPT & 90-94 & 103.65 & 108.25 & 99.23 & HT-Direct \\ 
  Egypt & UPPER EGYPT & 90-94 & 102.00 & 98.14 & 106.06 & RW2 \\ 
  Egypt & UPPER EGYPT & 95-99 & 71.69 & 75.77 & 67.82 & HT-Direct \\ 
  Egypt & UPPER EGYPT & 95-99 & 74.22 & 70.57 & 78.09 & RW2 \\ 
  Egypt & UPPER EGYPT & 00-04 & 48.99 & 52.85 & 45.40 & HT-Direct \\ 
  Egypt & UPPER EGYPT & 00-04 & 52.71 & 49.20 & 56.46 & RW2 \\ 
  Egypt & UPPER EGYPT & 05-09 & 39.49 & 44.37 & 35.14 & HT-Direct \\ 
  Egypt & UPPER EGYPT & 05-09 & 41.13 & 36.90 & 45.82 & RW2 \\ 
  Egypt & UPPER EGYPT & 10-14 & 33.01 & 13.79 & 77.36 & RW2 \\ 
  Egypt & UPPER EGYPT & 15-19 & 26.46 & 2.81 & 208.44 & RW2 \\ 
  Egypt & URBAN GOVERNORATES & 80-84 & 86.39 & 94.19 & 79.19 & HT-Direct \\ 
  Egypt & URBAN GOVERNORATES & 80-84 & 90.03 & 83.04 & 97.51 & RW2 \\ 
  Egypt & URBAN GOVERNORATES & 85-89 & 66.50 & 72.91 & 60.62 & HT-Direct \\ 
  Egypt & URBAN GOVERNORATES & 85-89 & 65.27 & 61.02 & 69.98 & RW2 \\ 
  Egypt & URBAN GOVERNORATES & 90-94 & 46.74 & 52.60 & 41.50 & HT-Direct \\ 
  Egypt & URBAN GOVERNORATES & 90-94 & 50.40 & 46.41 & 54.53 & RW2 \\ 
  Egypt & URBAN GOVERNORATES & 95-99 & 37.91 & 44.53 & 32.24 & HT-Direct \\ 
  Egypt & URBAN GOVERNORATES & 95-99 & 39.56 & 35.63 & 43.72 & RW2 \\ 
  Egypt & URBAN GOVERNORATES & 00-04 & 29.24 & 35.49 & 24.07 & HT-Direct \\ 
  Egypt & URBAN GOVERNORATES & 00-04 & 31.84 & 27.90 & 36.38 & RW2 \\ 
  Egypt & URBAN GOVERNORATES & 05-09 & 30.64 & 40.19 & 23.30 & HT-Direct \\ 
  Egypt & URBAN GOVERNORATES & 05-09 & 28.99 & 23.64 & 35.86 & RW2 \\ 
  Egypt & URBAN GOVERNORATES & 10-14 & 27.49 & 10.98 & 67.89 & RW2 \\ 
  Egypt & URBAN GOVERNORATES & 15-19 & 26.12 & 2.75 & 213.27 & RW2 \\ 
  Ethiopia & ADDIS ABABA & 80-84 & 90.19 & 117.00 & 69.04 & HT-Direct \\ 
  Ethiopia & ADDIS ABABA & 80-84 & 89.89 & 72.59 & 110.43 & RW2 \\ 
  Ethiopia & ADDIS ABABA & 85-89 & 85.55 & 107.51 & 67.73 & HT-Direct \\ 
  Ethiopia & ADDIS ABABA & 85-89 & 89.29 & 77.43 & 102.73 & RW2 \\ 
  Ethiopia & ADDIS ABABA & 90-94 & 85.66 & 105.45 & 69.30 & HT-Direct \\ 
  Ethiopia & ADDIS ABABA & 90-94 & 90.05 & 79.75 & 101.70 & RW2 \\ 
  Ethiopia & ADDIS ABABA & 95-99 & 93.37 & 111.53 & 77.90 & HT-Direct \\ 
  Ethiopia & ADDIS ABABA & 95-99 & 79.49 & 70.09 & 90.62 & RW2 \\ 
  Ethiopia & ADDIS ABABA & 00-04 & 52.84 & 72.01 & 38.56 & HT-Direct \\ 
  Ethiopia & ADDIS ABABA & 00-04 & 64.17 & 54.90 & 74.96 & RW2 \\ 
  Ethiopia & ADDIS ABABA & 05-09 & 46.19 & 65.19 & 32.53 & HT-Direct \\ 
  Ethiopia & ADDIS ABABA & 05-09 & 47.08 & 37.85 & 58.23 & RW2 \\ 
  Ethiopia & ADDIS ABABA & 10-14 & 35.14 & 63.26 & 19.27 & HT-Direct \\ 
  Ethiopia & ADDIS ABABA & 10-14 & 34.81 & 24.26 & 48.59 & RW2 \\ 
  Ethiopia & ADDIS ABABA & 15-19 & 25.95 & 9.91 & 64.50 & RW2 \\ 
  Ethiopia & AFFAR & 80-84 & 328.98 & 380.06 & 281.64 & HT-Direct \\ 
  Ethiopia & AFFAR & 80-84 & 318.85 & 284.16 & 357.80 & RW2 \\ 
  Ethiopia & AFFAR & 85-89 & 292.96 & 327.95 & 260.26 & HT-Direct \\ 
  Ethiopia & AFFAR & 85-89 & 275.80 & 254.91 & 297.97 & RW2 \\ 
  Ethiopia & AFFAR & 90-94 & 240.69 & 271.11 & 212.68 & HT-Direct \\ 
  Ethiopia & AFFAR & 90-94 & 240.11 & 223.02 & 257.30 & RW2 \\ 
  Ethiopia & AFFAR & 95-99 & 179.55 & 202.24 & 158.90 & HT-Direct \\ 
  Ethiopia & AFFAR & 95-99 & 193.45 & 178.22 & 208.24 & RW2 \\ 
  Ethiopia & AFFAR & 00-04 & 154.54 & 178.86 & 132.99 & HT-Direct \\ 
  Ethiopia & AFFAR & 00-04 & 155.07 & 142.26 & 168.21 & RW2 \\ 
  Ethiopia & AFFAR & 05-09 & 122.21 & 141.98 & 104.86 & HT-Direct \\ 
  Ethiopia & AFFAR & 05-09 & 117.85 & 106.35 & 130.61 & RW2 \\ 
  Ethiopia & AFFAR & 10-14 & 114.45 & 145.45 & 89.36 & HT-Direct \\ 
  Ethiopia & AFFAR & 10-14 & 89.92 & 76.67 & 106.47 & RW2 \\ 
  Ethiopia & AFFAR & 15-19 & 68.89 & 31.38 & 146.58 & RW2 \\ 
  Ethiopia & ALL & 80-84 & 254.72 & 235.08 & 275.78 & IHME \\ 
  Ethiopia & ALL & 80-84 & 235.97 & 222.85 & 249.61 & RW2 \\ 
  Ethiopia & ALL & 80-84 & 235.81 & 227.46 & 244.60 & UN \\ 
  Ethiopia & ALL & 85-89 & 211.56 & 208.67 & 214.26 & IHME \\ 
  Ethiopia & ALL & 85-89 & 215.41 & 206.20 & 224.88 & RW2 \\ 
  Ethiopia & ALL & 85-89 & 215.75 & 208.65 & 222.99 & UN \\ 
  Ethiopia & ALL & 90-94 & 191.46 & 188.83 & 193.83 & IHME \\ 
  Ethiopia & ALL & 90-94 & 194.59 & 187.18 & 202.25 & RW2 \\ 
  Ethiopia & ALL & 90-94 & 194.17 & 188.18 & 201.32 & UN \\ 
  Ethiopia & ALL & 95-99 & 162.49 & 160.11 & 164.87 & IHME \\ 
  Ethiopia & ALL & 95-99 & 161.77 & 155.47 & 168.22 & RW2 \\ 
  Ethiopia & ALL & 95-99 & 162.00 & 156.75 & 167.62 & UN \\ 
  Ethiopia & ALL & 00-04 & 129.27 & 127.17 & 131.38 & IHME \\ 
  Ethiopia & ALL & 00-04 & 131.82 & 125.31 & 138.67 & RW2 \\ 
  Ethiopia & ALL & 00-04 & 131.72 & 126.68 & 136.42 & UN \\ 
  Ethiopia & ALL & 05-09 & 97.86 & 95.47 & 100.57 & IHME \\ 
  Ethiopia & ALL & 05-09 & 94.28 & 88.06 & 100.87 & RW2 \\ 
  Ethiopia & ALL & 05-09 & 94.39 & 89.20 & 99.54 & UN \\ 
  Ethiopia & ALL & 10-14 & 72.87 & 69.72 & 75.85 & IHME \\ 
  Ethiopia & ALL & 10-14 & 67.46 & 60.49 & 75.11 & RW2 \\ 
  Ethiopia & ALL & 10-14 & 67.36 & 60.19 & 75.41 & UN \\ 
  Ethiopia & AMHARA & 80-84 & 236.38 & 261.23 & 213.21 & HT-Direct \\ 
  Ethiopia & AMHARA & 80-84 & 234.19 & 215.15 & 254.36 & RW2 \\ 
  Ethiopia & AMHARA & 85-89 & 216.11 & 235.03 & 198.31 & HT-Direct \\ 
  Ethiopia & AMHARA & 85-89 & 212.94 & 200.38 & 225.65 & RW2 \\ 
  Ethiopia & AMHARA & 90-94 & 196.04 & 210.96 & 181.93 & HT-Direct \\ 
  Ethiopia & AMHARA & 90-94 & 196.03 & 186.01 & 206.13 & RW2 \\ 
  Ethiopia & AMHARA & 95-99 & 163.13 & 175.63 & 151.36 & HT-Direct \\ 
  Ethiopia & AMHARA & 95-99 & 166.19 & 157.47 & 175.42 & RW2 \\ 
  Ethiopia & AMHARA & 00-04 & 151.21 & 167.39 & 136.35 & HT-Direct \\ 
  Ethiopia & AMHARA & 00-04 & 136.62 & 127.32 & 147.04 & RW2 \\ 
  Ethiopia & AMHARA & 05-09 & 103.66 & 122.62 & 87.34 & HT-Direct \\ 
  Ethiopia & AMHARA & 05-09 & 101.01 & 91.17 & 111.79 & RW2 \\ 
  Ethiopia & AMHARA & 10-14 & 66.28 & 87.36 & 50.01 & HT-Direct \\ 
  Ethiopia & AMHARA & 10-14 & 72.58 & 60.36 & 85.57 & RW2 \\ 
  Ethiopia & AMHARA & 15-19 & 51.66 & 22.79 & 112.68 & RW2 \\ 
  Ethiopia & BENISHANGUL-GUMUZ & 80-84 & 273.63 & 323.95 & 228.49 & HT-Direct \\ 
  Ethiopia & BENISHANGUL-GUMUZ & 80-84 & 270.40 & 236.35 & 307.28 & RW2 \\ 
  Ethiopia & BENISHANGUL-GUMUZ & 85-89 & 255.95 & 291.44 & 223.42 & HT-Direct \\ 
  Ethiopia & BENISHANGUL-GUMUZ & 85-89 & 251.52 & 230.69 & 273.31 & RW2 \\ 
  Ethiopia & BENISHANGUL-GUMUZ & 90-94 & 240.45 & 265.79 & 216.81 & HT-Direct \\ 
  Ethiopia & BENISHANGUL-GUMUZ & 90-94 & 237.28 & 220.99 & 253.73 & RW2 \\ 
  Ethiopia & BENISHANGUL-GUMUZ & 95-99 & 194.10 & 219.21 & 171.24 & HT-Direct \\ 
  Ethiopia & BENISHANGUL-GUMUZ & 95-99 & 206.17 & 191.52 & 221.93 & RW2 \\ 
  Ethiopia & BENISHANGUL-GUMUZ & 00-04 & 190.56 & 214.51 & 168.70 & HT-Direct \\ 
  Ethiopia & BENISHANGUL-GUMUZ & 00-04 & 172.43 & 158.29 & 188.83 & RW2 \\ 
  Ethiopia & BENISHANGUL-GUMUZ & 05-09 & 135.41 & 165.59 & 110.01 & HT-Direct \\ 
  Ethiopia & BENISHANGUL-GUMUZ & 05-09 & 128.00 & 114.08 & 143.37 & RW2 \\ 
  Ethiopia & BENISHANGUL-GUMUZ & 10-14 & 90.34 & 116.33 & 69.70 & HT-Direct \\ 
  Ethiopia & BENISHANGUL-GUMUZ & 10-14 & 90.93 & 74.90 & 108.51 & RW2 \\ 
  Ethiopia & BENISHANGUL-GUMUZ & 15-19 & 63.66 & 27.95 & 137.99 & RW2 \\ 
  Ethiopia & DIRE DAWA & 80-84 & 204.66 & 256.04 & 161.36 & HT-Direct \\ 
  Ethiopia & DIRE DAWA & 80-84 & 223.81 & 189.76 & 260.90 & RW2 \\ 
  Ethiopia & DIRE DAWA & 85-89 & 219.10 & 252.79 & 188.77 & HT-Direct \\ 
  Ethiopia & DIRE DAWA & 85-89 & 210.13 & 189.76 & 231.98 & RW2 \\ 
  Ethiopia & DIRE DAWA & 90-94 & 203.24 & 233.63 & 175.89 & HT-Direct \\ 
  Ethiopia & DIRE DAWA & 90-94 & 197.41 & 181.20 & 215.11 & RW2 \\ 
  Ethiopia & DIRE DAWA & 95-99 & 169.82 & 197.15 & 145.59 & HT-Direct \\ 
  Ethiopia & DIRE DAWA & 95-99 & 165.01 & 151.11 & 180.46 & RW2 \\ 
  Ethiopia & DIRE DAWA & 00-04 & 129.26 & 150.21 & 110.85 & HT-Direct \\ 
  Ethiopia & DIRE DAWA & 00-04 & 131.23 & 119.30 & 144.20 & RW2 \\ 
  Ethiopia & DIRE DAWA & 05-09 & 96.29 & 115.74 & 79.82 & HT-Direct \\ 
  Ethiopia & DIRE DAWA & 05-09 & 96.36 & 85.68 & 108.33 & RW2 \\ 
  Ethiopia & DIRE DAWA & 10-14 & 81.13 & 105.84 & 61.79 & HT-Direct \\ 
  Ethiopia & DIRE DAWA & 10-14 & 71.09 & 59.05 & 85.10 & RW2 \\ 
  Ethiopia & DIRE DAWA & 15-19 & 52.69 & 23.29 & 115.28 & RW2 \\ 
  Ethiopia & GAMBELA & 80-84 & 265.93 & 312.11 & 224.35 & HT-Direct \\ 
  Ethiopia & GAMBELA & 80-84 & 273.14 & 238.32 & 309.68 & RW2 \\ 
  Ethiopia & GAMBELA & 85-89 & 245.68 & 285.64 & 209.67 & HT-Direct \\ 
  Ethiopia & GAMBELA & 85-89 & 254.73 & 232.31 & 278.42 & RW2 \\ 
  Ethiopia & GAMBELA & 90-94 & 261.14 & 294.62 & 230.23 & HT-Direct \\ 
  Ethiopia & GAMBELA & 90-94 & 235.05 & 217.17 & 254.78 & RW2 \\ 
  Ethiopia & GAMBELA & 95-99 & 193.44 & 219.53 & 169.77 & HT-Direct \\ 
  Ethiopia & GAMBELA & 95-99 & 189.37 & 174.51 & 205.72 & RW2 \\ 
  Ethiopia & GAMBELA & 00-04 & 126.24 & 149.32 & 106.29 & HT-Direct \\ 
  Ethiopia & GAMBELA & 00-04 & 143.74 & 130.05 & 158.09 & RW2 \\ 
  Ethiopia & GAMBELA & 05-09 & 106.75 & 130.40 & 86.96 & HT-Direct \\ 
  Ethiopia & GAMBELA & 05-09 & 102.65 & 90.83 & 115.57 & RW2 \\ 
  Ethiopia & GAMBELA & 10-14 & 87.32 & 111.89 & 67.73 & HT-Direct \\ 
  Ethiopia & GAMBELA & 10-14 & 74.36 & 62.20 & 88.78 & RW2 \\ 
  Ethiopia & GAMBELA & 15-19 & 54.28 & 24.06 & 118.42 & RW2 \\ 
  Ethiopia & HARARI & 80-84 & 265.94 & 335.32 & 206.46 & HT-Direct \\ 
  Ethiopia & HARARI & 80-84 & 249.58 & 208.16 & 295.88 & RW2 \\ 
  Ethiopia & HARARI & 85-89 & 194.98 & 228.33 & 165.45 & HT-Direct \\ 
  Ethiopia & HARARI & 85-89 & 219.81 & 197.45 & 243.88 & RW2 \\ 
  Ethiopia & HARARI & 90-94 & 222.73 & 251.20 & 196.63 & HT-Direct \\ 
  Ethiopia & HARARI & 90-94 & 195.38 & 179.24 & 213.19 & RW2 \\ 
  Ethiopia & HARARI & 95-99 & 151.24 & 175.09 & 130.13 & HT-Direct \\ 
  Ethiopia & HARARI & 95-99 & 150.92 & 137.31 & 165.39 & RW2 \\ 
  Ethiopia & HARARI & 00-04 & 98.46 & 117.28 & 82.38 & HT-Direct \\ 
  Ethiopia & HARARI & 00-04 & 111.37 & 98.88 & 124.19 & RW2 \\ 
  Ethiopia & HARARI & 05-09 & 77.26 & 96.34 & 61.71 & HT-Direct \\ 
  Ethiopia & HARARI & 05-09 & 79.87 & 69.60 & 91.36 & RW2 \\ 
  Ethiopia & HARARI & 10-14 & 78.87 & 105.46 & 58.54 & HT-Direct \\ 
  Ethiopia & HARARI & 10-14 & 60.24 & 48.71 & 75.24 & RW2 \\ 
  Ethiopia & HARARI & 15-19 & 46.20 & 19.78 & 106.68 & RW2 \\ 
  Ethiopia & OROMIYA & 80-84 & 244.25 & 268.72 & 221.34 & HT-Direct \\ 
  Ethiopia & OROMIYA & 80-84 & 235.24 & 217.63 & 254.13 & RW2 \\ 
  Ethiopia & OROMIYA & 85-89 & 204.02 & 220.58 & 188.40 & HT-Direct \\ 
  Ethiopia & OROMIYA & 85-89 & 213.14 & 201.51 & 225.33 & RW2 \\ 
  Ethiopia & OROMIYA & 90-94 & 207.72 & 224.96 & 191.48 & HT-Direct \\ 
  Ethiopia & OROMIYA & 90-94 & 195.54 & 185.93 & 205.55 & RW2 \\ 
  Ethiopia & OROMIYA & 95-99 & 158.25 & 170.20 & 146.99 & HT-Direct \\ 
  Ethiopia & OROMIYA & 95-99 & 160.71 & 152.72 & 169.09 & RW2 \\ 
  Ethiopia & OROMIYA & 00-04 & 127.96 & 140.86 & 116.08 & HT-Direct \\ 
  Ethiopia & OROMIYA & 00-04 & 126.69 & 119.10 & 134.74 & RW2 \\ 
  Ethiopia & OROMIYA & 05-09 & 93.37 & 105.08 & 82.85 & HT-Direct \\ 
  Ethiopia & OROMIYA & 05-09 & 92.57 & 85.31 & 100.29 & RW2 \\ 
  Ethiopia & OROMIYA & 10-14 & 78.51 & 96.58 & 63.58 & HT-Direct \\ 
  Ethiopia & OROMIYA & 10-14 & 67.84 & 59.62 & 77.01 & RW2 \\ 
  Ethiopia & OROMIYA & 15-19 & 50.09 & 22.95 & 105.51 & RW2 \\ 
  Ethiopia & SNNP & 80-84 & 237.15 & 270.34 & 206.88 & HT-Direct \\ 
  Ethiopia & SNNP & 80-84 & 245.36 & 219.90 & 271.79 & RW2 \\ 
  Ethiopia & SNNP & 85-89 & 237.31 & 259.20 & 216.72 & HT-Direct \\ 
  Ethiopia & SNNP & 85-89 & 229.81 & 215.19 & 245.26 & RW2 \\ 
  Ethiopia & SNNP & 90-94 & 218.67 & 235.95 & 202.33 & HT-Direct \\ 
  Ethiopia & SNNP & 90-94 & 212.86 & 201.59 & 224.88 & RW2 \\ 
  Ethiopia & SNNP & 95-99 & 173.44 & 187.58 & 160.15 & HT-Direct \\ 
  Ethiopia & SNNP & 95-99 & 174.54 & 164.85 & 184.81 & RW2 \\ 
  Ethiopia & SNNP & 00-04 & 135.26 & 148.19 & 123.30 & HT-Direct \\ 
  Ethiopia & SNNP & 00-04 & 136.37 & 127.38 & 145.70 & RW2 \\ 
  Ethiopia & SNNP & 05-09 & 102.83 & 118.89 & 88.72 & HT-Direct \\ 
  Ethiopia & SNNP & 05-09 & 99.72 & 90.51 & 109.75 & RW2 \\ 
  Ethiopia & SNNP & 10-14 & 83.34 & 106.93 & 64.59 & HT-Direct \\ 
  Ethiopia & SNNP & 10-14 & 73.60 & 62.29 & 87.04 & RW2 \\ 
  Ethiopia & SNNP & 15-19 & 54.61 & 24.32 & 119.06 & RW2 \\ 
  Ethiopia & SOMALI & 80-84 & 202.00 & 260.21 & 154.09 & HT-Direct \\ 
  Ethiopia & SOMALI & 80-84 & 178.96 & 148.71 & 214.11 & RW2 \\ 
  Ethiopia & SOMALI & 85-89 & 147.16 & 180.15 & 119.32 & HT-Direct \\ 
  Ethiopia & SOMALI & 85-89 & 171.42 & 151.18 & 193.44 & RW2 \\ 
  Ethiopia & SOMALI & 90-94 & 188.28 & 227.20 & 154.69 & HT-Direct \\ 
  Ethiopia & SOMALI & 90-94 & 167.32 & 151.57 & 184.22 & RW2 \\ 
  Ethiopia & SOMALI & 95-99 & 142.12 & 162.79 & 123.68 & HT-Direct \\ 
  Ethiopia & SOMALI & 95-99 & 147.61 & 135.33 & 160.61 & RW2 \\ 
  Ethiopia & SOMALI & 00-04 & 129.97 & 148.62 & 113.35 & HT-Direct \\ 
  Ethiopia & SOMALI & 00-04 & 126.73 & 116.43 & 137.83 & RW2 \\ 
  Ethiopia & SOMALI & 05-09 & 103.87 & 120.31 & 89.46 & HT-Direct \\ 
  Ethiopia & SOMALI & 05-09 & 101.56 & 92.28 & 111.81 & RW2 \\ 
  Ethiopia & SOMALI & 10-14 & 91.15 & 110.38 & 74.98 & HT-Direct \\ 
  Ethiopia & SOMALI & 10-14 & 81.86 & 71.20 & 94.02 & RW2 \\ 
  Ethiopia & SOMALI & 15-19 & 66.41 & 30.28 & 140.75 & RW2 \\ 
  Ethiopia & TIGRAY & 80-84 & 252.69 & 285.29 & 222.67 & HT-Direct \\ 
  Ethiopia & TIGRAY & 80-84 & 259.46 & 234.14 & 286.53 & RW2 \\ 
  Ethiopia & TIGRAY & 85-89 & 220.93 & 243.98 & 199.48 & HT-Direct \\ 
  Ethiopia & TIGRAY & 85-89 & 219.60 & 204.72 & 235.61 & RW2 \\ 
  Ethiopia & TIGRAY & 90-94 & 197.68 & 217.23 & 179.50 & HT-Direct \\ 
  Ethiopia & TIGRAY & 90-94 & 184.43 & 172.97 & 196.68 & RW2 \\ 
  Ethiopia & TIGRAY & 95-99 & 138.63 & 152.48 & 125.84 & HT-Direct \\ 
  Ethiopia & TIGRAY & 95-99 & 138.70 & 129.32 & 148.35 & RW2 \\ 
  Ethiopia & TIGRAY & 00-04 & 89.39 & 102.43 & 77.87 & HT-Direct \\ 
  Ethiopia & TIGRAY & 00-04 & 101.89 & 93.53 & 110.65 & RW2 \\ 
  Ethiopia & TIGRAY & 05-09 & 77.96 & 91.18 & 66.52 & HT-Direct \\ 
  Ethiopia & TIGRAY & 05-09 & 70.68 & 63.73 & 78.34 & RW2 \\ 
  Ethiopia & TIGRAY & 10-14 & 55.22 & 69.42 & 43.78 & HT-Direct \\ 
  Ethiopia & TIGRAY & 10-14 & 48.88 & 41.56 & 57.62 & RW2 \\ 
  Ethiopia & TIGRAY & 15-19 & 33.96 & 14.90 & 75.20 & RW2 \\ 
  Gabon & ALL & 80-84 & 105.56 & 102.43 & 108.52 & IHME \\ 
  Gabon & ALL & 80-84 & 110.48 & 95.37 & 127.67 & RW2 \\ 
  Gabon & ALL & 80-84 & 109.94 & 102.25 & 119.10 & UN \\ 
  Gabon & ALL & 85-89 & 89.66 & 87.53 & 91.86 & IHME \\ 
  Gabon & ALL & 85-89 & 96.12 & 84.72 & 108.69 & RW2 \\ 
  Gabon & ALL & 85-89 & 96.92 & 90.85 & 103.92 & UN \\ 
  Gabon & ALL & 90-94 & 79.90 & 78.23 & 81.80 & IHME \\ 
  Gabon & ALL & 90-94 & 92.83 & 83.69 & 102.90 & RW2 \\ 
  Gabon & ALL & 90-94 & 91.93 & 86.14 & 97.99 & UN \\ 
  Gabon & ALL & 95-99 & 73.53 & 71.86 & 75.24 & IHME \\ 
  Gabon & ALL & 95-99 & 88.84 & 78.51 & 100.29 & RW2 \\ 
  Gabon & ALL & 95-99 & 88.83 & 83.31 & 94.85 & UN \\ 
  Gabon & ALL & 00-04 & 69.99 & 68.33 & 71.76 & IHME \\ 
  Gabon & ALL & 00-04 & 82.35 & 66.09 & 102.83 & RW2 \\ 
  Gabon & ALL & 00-04 & 83.16 & 77.53 & 89.24 & UN \\ 
  Gabon & ALL & 05-09 & 63.23 & 61.13 & 64.97 & IHME \\ 
  Gabon & ALL & 05-09 & 70.22 & 57.95 & 84.82 & RW2 \\ 
  Gabon & ALL & 05-09 & 71.64 & 66.79 & 76.89 & UN \\ 
  Gabon & ALL & 10-14 & 52.38 & 49.61 & 55.02 & IHME \\ 
  Gabon & ALL & 10-14 & 58.55 & 50.24 & 68.07 & RW2 \\ 
  Gabon & ALL & 10-14 & 57.25 & 51.45 & 64.37 & UN \\ 
  Gabon & EAST & 80-84 & 71.07 & 99.51 & 50.30 & HT-Direct \\ 
  Gabon & EAST & 80-84 & 75.06 & 56.14 & 99.98 & RW2 \\ 
  Gabon & EAST & 85-89 & 59.77 & 81.58 & 43.52 & HT-Direct \\ 
  Gabon & EAST & 85-89 & 63.53 & 51.34 & 78.16 & RW2 \\ 
  Gabon & EAST & 90-94 & 53.77 & 66.88 & 43.10 & HT-Direct \\ 
  Gabon & EAST & 90-94 & 59.36 & 50.27 & 69.84 & RW2 \\ 
  Gabon & EAST & 95-99 & 62.94 & 78.44 & 50.34 & HT-Direct \\ 
  Gabon & EAST & 95-99 & 70.10 & 59.01 & 83.26 & RW2 \\ 
  Gabon & EAST & 00-04 & 60.95 & 81.52 & 45.32 & HT-Direct \\ 
  Gabon & EAST & 00-04 & 78.85 & 62.45 & 100.02 & RW2 \\ 
  Gabon & EAST & 05-09 & 53.92 & 75.62 & 38.19 & HT-Direct \\ 
  Gabon & EAST & 05-09 & 69.63 & 54.29 & 89.38 & RW2 \\ 
  Gabon & EAST & 10-14 & 77.36 & 118.42 & 49.74 & HT-Direct \\ 
  Gabon & EAST & 10-14 & 56.87 & 42.26 & 76.01 & RW2 \\ 
  Gabon & EAST & 15-19 & 45.98 & 16.17 & 124.15 & RW2 \\ 
  Gabon & LIBREVILLE,PORT-GENTIL & 80-84 & 98.38 & 125.53 & 76.59 & HT-Direct \\ 
  Gabon & LIBREVILLE,PORT-GENTIL & 80-84 & 113.60 & 88.89 & 142.62 & RW2 \\ 
  Gabon & LIBREVILLE,PORT-GENTIL & 85-89 & 83.11 & 100.47 & 68.52 & HT-Direct \\ 
  Gabon & LIBREVILLE,PORT-GENTIL & 85-89 & 101.20 & 86.08 & 118.69 & RW2 \\ 
  Gabon & LIBREVILLE,PORT-GENTIL & 90-94 & 104.24 & 122.98 & 88.07 & HT-Direct \\ 
  Gabon & LIBREVILLE,PORT-GENTIL & 90-94 & 92.58 & 80.02 & 107.94 & RW2 \\ 
  Gabon & LIBREVILLE,PORT-GENTIL & 95-99 & 73.79 & 91.46 & 59.31 & HT-Direct \\ 
  Gabon & LIBREVILLE,PORT-GENTIL & 95-99 & 93.51 & 78.65 & 110.96 & RW2 \\ 
  Gabon & LIBREVILLE,PORT-GENTIL & 00-04 & 54.34 & 79.61 & 36.78 & HT-Direct \\ 
  Gabon & LIBREVILLE,PORT-GENTIL & 00-04 & 88.14 & 66.74 & 113.27 & RW2 \\ 
  Gabon & LIBREVILLE,PORT-GENTIL & 05-09 & 47.17 & 63.14 & 35.10 & HT-Direct \\ 
  Gabon & LIBREVILLE,PORT-GENTIL & 05-09 & 69.07 & 52.97 & 87.86 & RW2 \\ 
  Gabon & LIBREVILLE,PORT-GENTIL & 10-14 & 75.53 & 107.89 & 52.30 & HT-Direct \\ 
  Gabon & LIBREVILLE,PORT-GENTIL & 10-14 & 53.11 & 40.84 & 68.68 & RW2 \\ 
  Gabon & LIBREVILLE,PORT-GENTIL & 15-19 & 41.09 & 14.67 & 113.80 & RW2 \\ 
  Gabon & NORTH & 80-84 & 125.59 & 166.00 & 93.90 & HT-Direct \\ 
  Gabon & NORTH & 80-84 & 136.81 & 107.53 & 172.54 & RW2 \\ 
  Gabon & NORTH & 85-89 & 101.25 & 123.63 & 82.54 & HT-Direct \\ 
  Gabon & NORTH & 85-89 & 110.70 & 93.78 & 130.39 & RW2 \\ 
  Gabon & NORTH & 90-94 & 89.93 & 108.09 & 74.56 & HT-Direct \\ 
  Gabon & NORTH & 90-94 & 97.69 & 84.75 & 112.04 & RW2 \\ 
  Gabon & NORTH & 95-99 & 97.87 & 117.82 & 80.98 & HT-Direct \\ 
  Gabon & NORTH & 95-99 & 107.45 & 92.36 & 125.18 & RW2 \\ 
  Gabon & NORTH & 00-04 & 90.92 & 123.90 & 66.05 & HT-Direct \\ 
  Gabon & NORTH & 00-04 & 112.83 & 91.41 & 140.24 & RW2 \\ 
  Gabon & NORTH & 05-09 & 75.44 & 90.90 & 62.44 & HT-Direct \\ 
  Gabon & NORTH & 05-09 & 94.59 & 78.57 & 113.70 & RW2 \\ 
  Gabon & NORTH & 10-14 & 100.25 & 126.73 & 78.81 & HT-Direct \\ 
  Gabon & NORTH & 10-14 & 73.60 & 61.84 & 87.44 & RW2 \\ 
  Gabon & NORTH & 15-19 & 56.53 & 21.68 & 141.29 & RW2 \\ 
  Gabon & SOUTH & 80-84 & 120.68 & 156.22 & 92.35 & HT-Direct \\ 
  Gabon & SOUTH & 80-84 & 125.02 & 98.72 & 156.98 & RW2 \\ 
  Gabon & SOUTH & 85-89 & 75.71 & 97.39 & 58.54 & HT-Direct \\ 
  Gabon & SOUTH & 85-89 & 98.76 & 82.51 & 117.67 & RW2 \\ 
  Gabon & SOUTH & 90-94 & 86.45 & 103.33 & 72.11 & HT-Direct \\ 
  Gabon & SOUTH & 90-94 & 85.95 & 74.62 & 98.70 & RW2 \\ 
  Gabon & SOUTH & 95-99 & 78.64 & 95.31 & 64.68 & HT-Direct \\ 
  Gabon & SOUTH & 95-99 & 92.22 & 79.03 & 108.19 & RW2 \\ 
  Gabon & SOUTH & 00-04 & 82.52 & 107.82 & 62.74 & HT-Direct \\ 
  Gabon & SOUTH & 00-04 & 93.57 & 74.92 & 117.84 & RW2 \\ 
  Gabon & SOUTH & 05-09 & 66.64 & 89.87 & 49.10 & HT-Direct \\ 
  Gabon & SOUTH & 05-09 & 74.15 & 59.11 & 93.09 & RW2 \\ 
  Gabon & SOUTH & 10-14 & 68.52 & 96.32 & 48.31 & HT-Direct \\ 
  Gabon & SOUTH & 10-14 & 53.66 & 41.87 & 68.19 & RW2 \\ 
  Gabon & SOUTH & 15-19 & 38.09 & 13.77 & 101.42 & RW2 \\ 
  Gabon & WEST & 80-84 & 154.47 & 195.61 & 120.68 & HT-Direct \\ 
  Gabon & WEST & 80-84 & 142.34 & 114.10 & 179.28 & RW2 \\ 
  Gabon & WEST & 85-89 & 97.53 & 118.17 & 80.17 & HT-Direct \\ 
  Gabon & WEST & 85-89 & 106.86 & 90.89 & 125.18 & RW2 \\ 
  Gabon & WEST & 90-94 & 67.05 & 83.30 & 53.79 & HT-Direct \\ 
  Gabon & WEST & 90-94 & 88.55 & 75.56 & 102.65 & RW2 \\ 
  Gabon & WEST & 95-99 & 88.25 & 106.38 & 72.95 & HT-Direct \\ 
  Gabon & WEST & 95-99 & 93.22 & 78.36 & 109.05 & RW2 \\ 
  Gabon & WEST & 00-04 & 64.81 & 106.86 & 38.59 & HT-Direct \\ 
  Gabon & WEST & 00-04 & 96.37 & 75.87 & 121.26 & RW2 \\ 
  Gabon & WEST & 05-09 & 57.56 & 76.43 & 43.13 & HT-Direct \\ 
  Gabon & WEST & 05-09 & 82.71 & 66.57 & 101.99 & RW2 \\ 
  Gabon & WEST & 10-14 & 101.09 & 132.91 & 76.21 & HT-Direct \\ 
  Gabon & WEST & 10-14 & 68.30 & 55.55 & 84.13 & RW2 \\ 
  Gabon & WEST & 15-19 & 56.61 & 21.55 & 143.38 & RW2 \\ 
  Gambia & ALL & 80-84 & 135.82 & 131.90 & 139.80 & IHME \\ 
  Gambia & ALL & 80-84 & 231.78 & 145.93 & 347.78 & RW2 \\ 
  Gambia & ALL & 80-84 & 228.47 & 214.24 & 242.39 & UN \\ 
  Gambia & ALL & 85-89 & 120.48 & 117.40 & 123.37 & IHME \\ 
  Gambia & ALL & 85-89 & 190.37 & 137.48 & 257.00 & RW2 \\ 
  Gambia & ALL & 85-89 & 191.14 & 180.53 & 202.28 & UN \\ 
  Gambia & ALL & 90-94 & 108.04 & 105.88 & 110.37 & IHME \\ 
  Gambia & ALL & 90-94 & 158.19 & 120.92 & 204.28 & RW2 \\ 
  Gambia & ALL & 90-94 & 158.24 & 150.73 & 166.37 & UN \\ 
  Gambia & ALL & 95-99 & 94.18 & 92.22 & 96.12 & IHME \\ 
  Gambia & ALL & 95-99 & 132.33 & 106.91 & 162.34 & RW2 \\ 
  Gambia & ALL & 95-99 & 132.66 & 126.03 & 139.90 & UN \\ 
  Gambia & ALL & 00-04 & 79.35 & 77.59 & 81.14 & IHME \\ 
  Gambia & ALL & 00-04 & 111.24 & 93.35 & 132.28 & RW2 \\ 
  Gambia & ALL & 00-04 & 110.82 & 104.31 & 117.45 & UN \\ 
  Gambia & ALL & 05-09 & 65.01 & 63.14 & 66.77 & IHME \\ 
  Gambia & ALL & 05-09 & 91.10 & 75.52 & 109.44 & RW2 \\ 
  Gambia & ALL & 05-09 & 90.98 & 83.48 & 100.36 & UN \\ 
  Gambia & ALL & 10-14 & 51.98 & 49.99 & 54.01 & IHME \\ 
  Gambia & ALL & 10-14 & 75.81 & 54.80 & 103.78 & RW2 \\ 
  Gambia & ALL & 10-14 & 76.27 & 66.30 & 88.83 & UN \\ 
  Gambia & BANJUL & 80-84 & 112.43 & 300.72 & 35.97 & HT-Direct \\ 
  Gambia & BANJUL & 80-84 & 128.94 & 50.54 & 297.19 & RW2 \\ 
  Gambia & BANJUL & 85-89 & 86.72 & 154.53 & 47.01 & HT-Direct \\ 
  Gambia & BANJUL & 85-89 & 110.04 & 55.28 & 206.32 & RW2 \\ 
  Gambia & BANJUL & 90-94 & 81.83 & 130.12 & 50.42 & HT-Direct \\ 
  Gambia & BANJUL & 90-94 & 97.03 & 58.20 & 157.85 & RW2 \\ 
  Gambia & BANJUL & 95-99 & 49.26 & 71.85 & 33.51 & HT-Direct \\ 
  Gambia & BANJUL & 95-99 & 88.03 & 59.66 & 127.26 & RW2 \\ 
  Gambia & BANJUL & 00-04 & 53.64 & 80.84 & 35.24 & HT-Direct \\ 
  Gambia & BANJUL & 00-04 & 82.55 & 59.45 & 112.99 & RW2 \\ 
  Gambia & BANJUL & 05-09 & 56.07 & 75.97 & 41.15 & HT-Direct \\ 
  Gambia & BANJUL & 05-09 & 75.02 & 53.23 & 105.09 & RW2 \\ 
  Gambia & BANJUL & 10-14 & 50.91 & 70.81 & 36.39 & HT-Direct \\ 
  Gambia & BANJUL & 10-14 & 68.14 & 40.94 & 112.05 & RW2 \\ 
  Gambia & BANJUL & 15-19 & 62.04 & 16.83 & 206.48 & RW2 \\ 
  Gambia & CENTRAL RIVER & 80-84 & 145.98 & 299.41 & 63.99 & HT-Direct \\ 
  Gambia & CENTRAL RIVER & 80-84 & 223.76 & 109.53 & 403.47 & RW2 \\ 
  Gambia & CENTRAL RIVER & 85-89 & 111.39 & 194.41 & 61.13 & HT-Direct \\ 
  Gambia & CENTRAL RIVER & 85-89 & 182.43 & 105.68 & 296.38 & RW2 \\ 
  Gambia & CENTRAL RIVER & 90-94 & 84.32 & 120.50 & 58.28 & HT-Direct \\ 
  Gambia & CENTRAL RIVER & 90-94 & 152.46 & 101.06 & 222.95 & RW2 \\ 
  Gambia & CENTRAL RIVER & 95-99 & 107.74 & 145.43 & 78.92 & HT-Direct \\ 
  Gambia & CENTRAL RIVER & 95-99 & 127.64 & 94.47 & 170.17 & RW2 \\ 
  Gambia & CENTRAL RIVER & 00-04 & 78.10 & 99.95 & 60.70 & HT-Direct \\ 
  Gambia & CENTRAL RIVER & 00-04 & 106.88 & 83.35 & 136.17 & RW2 \\ 
  Gambia & CENTRAL RIVER & 05-09 & 57.94 & 77.17 & 43.27 & HT-Direct \\ 
  Gambia & CENTRAL RIVER & 05-09 & 83.95 & 61.95 & 112.82 & RW2 \\ 
  Gambia & CENTRAL RIVER & 10-14 & 41.43 & 57.43 & 29.75 & HT-Direct \\ 
  Gambia & CENTRAL RIVER & 10-14 & 64.01 & 39.87 & 100.87 & RW2 \\ 
  Gambia & CENTRAL RIVER & 15-19 & 48.15 & 13.62 & 156.26 & RW2 \\ 
  Gambia & LOWER RIVER & 80-84 & 380.22 & 540.49 & 242.40 & HT-Direct \\ 
  Gambia & LOWER RIVER & 80-84 & 390.18 & 225.43 & 584.05 & RW2 \\ 
  Gambia & LOWER RIVER & 85-89 & 167.40 & 247.12 & 109.66 & HT-Direct \\ 
  Gambia & LOWER RIVER & 85-89 & 294.69 & 187.72 & 431.66 & RW2 \\ 
  Gambia & LOWER RIVER & 90-94 & 141.20 & 190.88 & 102.81 & HT-Direct \\ 
  Gambia & LOWER RIVER & 90-94 & 221.75 & 151.52 & 312.25 & RW2 \\ 
  Gambia & LOWER RIVER & 95-99 & 120.08 & 176.79 & 79.80 & HT-Direct \\ 
  Gambia & LOWER RIVER & 95-99 & 164.80 & 119.71 & 223.19 & RW2 \\ 
  Gambia & LOWER RIVER & 00-04 & 83.50 & 129.15 & 53.01 & HT-Direct \\ 
  Gambia & LOWER RIVER & 00-04 & 121.84 & 90.10 & 163.94 & RW2 \\ 
  Gambia & LOWER RIVER & 05-09 & 72.16 & 96.98 & 53.32 & HT-Direct \\ 
  Gambia & LOWER RIVER & 05-09 & 84.36 & 58.61 & 120.19 & RW2 \\ 
  Gambia & LOWER RIVER & 10-14 & 34.17 & 55.32 & 20.93 & HT-Direct \\ 
  Gambia & LOWER RIVER & 10-14 & 56.39 & 32.22 & 97.25 & RW2 \\ 
  Gambia & LOWER RIVER & 15-19 & 37.11 & 9.73 & 131.09 & RW2 \\ 
  Gambia & NORTH BANK & 80-84 & 206.13 & 350.18 & 111.20 & HT-Direct \\ 
  Gambia & NORTH BANK & 80-84 & 279.84 & 156.63 & 449.03 & RW2 \\ 
  Gambia & NORTH BANK & 85-89 & 148.72 & 201.16 & 108.10 & HT-Direct \\ 
  Gambia & NORTH BANK & 85-89 & 219.04 & 142.40 & 320.55 & RW2 \\ 
  Gambia & NORTH BANK & 90-94 & 101.70 & 136.95 & 74.75 & HT-Direct \\ 
  Gambia & NORTH BANK & 90-94 & 174.28 & 123.85 & 240.21 & RW2 \\ 
  Gambia & NORTH BANK & 95-99 & 96.34 & 125.47 & 73.40 & HT-Direct \\ 
  Gambia & NORTH BANK & 95-99 & 138.42 & 104.91 & 180.22 & RW2 \\ 
  Gambia & NORTH BANK & 00-04 & 93.08 & 119.46 & 72.04 & HT-Direct \\ 
  Gambia & NORTH BANK & 00-04 & 109.37 & 84.74 & 140.51 & RW2 \\ 
  Gambia & NORTH BANK & 05-09 & 55.23 & 75.04 & 40.43 & HT-Direct \\ 
  Gambia & NORTH BANK & 05-09 & 80.69 & 56.96 & 112.79 & RW2 \\ 
  Gambia & NORTH BANK & 10-14 & 28.97 & 51.08 & 16.27 & HT-Direct \\ 
  Gambia & NORTH BANK & 10-14 & 57.46 & 32.18 & 100.13 & RW2 \\ 
  Gambia & NORTH BANK & 15-19 & 40.74 & 10.24 & 145.43 & RW2 \\ 
  Gambia & UPPER RIVER & 80-84 & 240.89 & 356.14 & 154.02 & HT-Direct \\ 
  Gambia & UPPER RIVER & 80-84 & 334.64 & 199.47 & 502.82 & RW2 \\ 
  Gambia & UPPER RIVER & 85-89 & 166.20 & 274.60 & 94.99 & HT-Direct \\ 
  Gambia & UPPER RIVER & 85-89 & 279.53 & 182.41 & 402.97 & RW2 \\ 
  Gambia & UPPER RIVER & 90-94 & 212.51 & 265.72 & 167.53 & HT-Direct \\ 
  Gambia & UPPER RIVER & 90-94 & 237.08 & 170.44 & 318.85 & RW2 \\ 
  Gambia & UPPER RIVER & 95-99 & 131.26 & 157.74 & 108.65 & HT-Direct \\ 
  Gambia & UPPER RIVER & 95-99 & 199.11 & 155.37 & 251.94 & RW2 \\ 
  Gambia & UPPER RIVER & 00-04 & 121.49 & 164.61 & 88.48 & HT-Direct \\ 
  Gambia & UPPER RIVER & 00-04 & 166.42 & 129.07 & 213.01 & RW2 \\ 
  Gambia & UPPER RIVER & 05-09 & 117.91 & 158.34 & 86.74 & HT-Direct \\ 
  Gambia & UPPER RIVER & 05-09 & 130.30 & 95.51 & 175.26 & RW2 \\ 
  Gambia & UPPER RIVER & 10-14 & 58.91 & 76.75 & 45.01 & HT-Direct \\ 
  Gambia & UPPER RIVER & 10-14 & 97.83 & 63.05 & 148.30 & RW2 \\ 
  Gambia & UPPER RIVER & 15-19 & 72.49 & 21.34 & 216.61 & RW2 \\ 
  Gambia & WESTERN & 80-84 & 149.61 & 286.40 & 71.60 & HT-Direct \\ 
  Gambia & WESTERN & 80-84 & 198.99 & 99.28 & 362.69 & RW2 \\ 
  Gambia & WESTERN & 85-89 & 123.19 & 192.34 & 76.54 & HT-Direct \\ 
  Gambia & WESTERN & 85-89 & 163.43 & 96.63 & 263.97 & RW2 \\ 
  Gambia & WESTERN & 90-94 & 83.51 & 121.71 & 56.53 & HT-Direct \\ 
  Gambia & WESTERN & 90-94 & 138.66 & 92.07 & 203.67 & RW2 \\ 
  Gambia & WESTERN & 95-99 & 86.41 & 118.10 & 62.61 & HT-Direct \\ 
  Gambia & WESTERN & 95-99 & 119.59 & 87.74 & 160.30 & RW2 \\ 
  Gambia & WESTERN & 00-04 & 85.77 & 109.98 & 66.49 & HT-Direct \\ 
  Gambia & WESTERN & 00-04 & 105.00 & 82.19 & 133.14 & RW2 \\ 
  Gambia & WESTERN & 05-09 & 57.43 & 71.08 & 46.27 & HT-Direct \\ 
  Gambia & WESTERN & 05-09 & 88.24 & 67.94 & 113.79 & RW2 \\ 
  Gambia & WESTERN & 10-14 & 61.64 & 84.07 & 44.90 & HT-Direct \\ 
  Gambia & WESTERN & 10-14 & 73.53 & 47.19 & 113.42 & RW2 \\ 
  Gambia & WESTERN & 15-19 & 61.51 & 17.46 & 193.58 & RW2 \\ 
  Ghana & ALL & 80-84 & 148.92 & 146.86 & 151.03 & IHME \\ 
  Ghana & ALL & 80-84 & 162.85 & 154.80 & 171.24 & RW2 \\ 
  Ghana & ALL & 80-84 & 162.85 & 158.95 & 166.77 & UN \\ 
  Ghana & ALL & 85-89 & 134.21 & 132.48 & 135.87 & IHME \\ 
  Ghana & ALL & 85-89 & 143.76 & 136.58 & 151.25 & RW2 \\ 
  Ghana & ALL & 85-89 & 143.80 & 140.49 & 147.51 & UN \\ 
  Ghana & ALL & 90-94 & 116.65 & 115.07 & 118.43 & IHME \\ 
  Ghana & ALL & 90-94 & 120.47 & 113.87 & 127.37 & RW2 \\ 
  Ghana & ALL & 90-94 & 120.40 & 117.32 & 123.76 & UN \\ 
  Ghana & ALL & 95-99 & 105.14 & 103.63 & 106.60 & IHME \\ 
  Ghana & ALL & 95-99 & 108.99 & 101.93 & 116.45 & RW2 \\ 
  Ghana & ALL & 95-99 & 109.02 & 106.25 & 112.08 & UN \\ 
  Ghana & ALL & 00-04 & 93.30 & 91.80 & 94.80 & IHME \\ 
  Ghana & ALL & 00-04 & 94.18 & 87.62 & 101.19 & RW2 \\ 
  Ghana & ALL & 00-04 & 94.36 & 91.46 & 97.37 & UN \\ 
  Ghana & ALL & 05-09 & 79.49 & 77.85 & 81.08 & IHME \\ 
  Ghana & ALL & 05-09 & 82.60 & 75.16 & 90.73 & RW2 \\ 
  Ghana & ALL & 05-09 & 82.20 & 79.04 & 85.53 & UN \\ 
  Ghana & ALL & 10-14 & 64.38 & 62.12 & 66.82 & IHME \\ 
  Ghana & ALL & 10-14 & 68.64 & 58.60 & 80.17 & RW2 \\ 
  Ghana & ALL & 10-14 & 68.91 & 64.12 & 74.35 & UN \\ 
  Ghana & ASHANTI & 80-84 & 132.46 & 151.34 & 115.61 & HT-Direct \\ 
  Ghana & ASHANTI & 80-84 & 133.61 & 118.91 & 149.69 & RW2 \\ 
  Ghana & ASHANTI & 85-89 & 122.88 & 140.18 & 107.44 & HT-Direct \\ 
  Ghana & ASHANTI & 85-89 & 122.43 & 111.55 & 133.86 & RW2 \\ 
  Ghana & ASHANTI & 90-94 & 84.64 & 98.88 & 72.28 & HT-Direct \\ 
  Ghana & ASHANTI & 90-94 & 109.15 & 99.36 & 119.57 & RW2 \\ 
  Ghana & ASHANTI & 95-99 & 115.12 & 133.15 & 99.26 & HT-Direct \\ 
  Ghana & ASHANTI & 95-99 & 104.72 & 94.82 & 115.65 & RW2 \\ 
  Ghana & ASHANTI & 00-04 & 94.63 & 112.05 & 79.68 & HT-Direct \\ 
  Ghana & ASHANTI & 00-04 & 98.33 & 87.72 & 110.19 & RW2 \\ 
  Ghana & ASHANTI & 05-09 & 81.03 & 104.38 & 62.53 & HT-Direct \\ 
  Ghana & ASHANTI & 05-09 & 91.21 & 77.97 & 106.40 & RW2 \\ 
  Ghana & ASHANTI & 10-14 & 74.94 & 99.86 & 55.86 & HT-Direct \\ 
  Ghana & ASHANTI & 10-14 & 80.01 & 63.42 & 100.48 & RW2 \\ 
  Ghana & ASHANTI & 15-19 & 68.57 & 28.58 & 154.82 & RW2 \\ 
  Ghana & BRONG AHAFO & 80-84 & 132.03 & 156.16 & 111.14 & HT-Direct \\ 
  Ghana & BRONG AHAFO & 80-84 & 139.44 & 121.29 & 159.89 & RW2 \\ 
  Ghana & BRONG AHAFO & 85-89 & 122.97 & 144.36 & 104.36 & HT-Direct \\ 
  Ghana & BRONG AHAFO & 85-89 & 122.38 & 109.82 & 136.30 & RW2 \\ 
  Ghana & BRONG AHAFO & 90-94 & 90.98 & 112.15 & 73.47 & HT-Direct \\ 
  Ghana & BRONG AHAFO & 90-94 & 103.67 & 93.05 & 115.57 & RW2 \\ 
  Ghana & BRONG AHAFO & 95-99 & 90.91 & 111.74 & 73.64 & HT-Direct \\ 
  Ghana & BRONG AHAFO & 95-99 & 93.10 & 82.81 & 104.77 & RW2 \\ 
  Ghana & BRONG AHAFO & 00-04 & 84.62 & 103.83 & 68.69 & HT-Direct \\ 
  Ghana & BRONG AHAFO & 00-04 & 81.72 & 71.46 & 93.41 & RW2 \\ 
  Ghana & BRONG AHAFO & 05-09 & 65.47 & 89.43 & 47.60 & HT-Direct \\ 
  Ghana & BRONG AHAFO & 05-09 & 70.45 & 58.75 & 84.07 & RW2 \\ 
  Ghana & BRONG AHAFO & 10-14 & 44.70 & 61.67 & 32.24 & HT-Direct \\ 
  Ghana & BRONG AHAFO & 10-14 & 56.90 & 43.99 & 72.81 & RW2 \\ 
  Ghana & BRONG AHAFO & 15-19 & 44.99 & 18.46 & 103.33 & RW2 \\ 
  Ghana & CENTRAL & 80-84 & 184.65 & 211.69 & 160.37 & HT-Direct \\ 
  Ghana & CENTRAL & 80-84 & 184.82 & 164.09 & 207.91 & RW2 \\ 
  Ghana & CENTRAL & 85-89 & 154.43 & 177.15 & 134.15 & HT-Direct \\ 
  Ghana & CENTRAL & 85-89 & 155.30 & 141.18 & 170.43 & RW2 \\ 
  Ghana & CENTRAL & 90-94 & 111.73 & 130.22 & 95.57 & HT-Direct \\ 
  Ghana & CENTRAL & 90-94 & 126.57 & 114.43 & 139.52 & RW2 \\ 
  Ghana & CENTRAL & 95-99 & 102.00 & 127.72 & 80.98 & HT-Direct \\ 
  Ghana & CENTRAL & 95-99 & 110.55 & 97.96 & 124.16 & RW2 \\ 
  Ghana & CENTRAL & 00-04 & 90.56 & 111.93 & 72.94 & HT-Direct \\ 
  Ghana & CENTRAL & 00-04 & 95.88 & 83.48 & 110.00 & RW2 \\ 
  Ghana & CENTRAL & 05-09 & 89.72 & 118.57 & 67.35 & HT-Direct \\ 
  Ghana & CENTRAL & 05-09 & 83.13 & 68.75 & 100.24 & RW2 \\ 
  Ghana & CENTRAL & 10-14 & 68.17 & 110.51 & 41.30 & HT-Direct \\ 
  Ghana & CENTRAL & 10-14 & 68.04 & 50.71 & 92.22 & RW2 \\ 
  Ghana & CENTRAL & 15-19 & 54.70 & 21.51 & 131.46 & RW2 \\ 
  Ghana & EASTERN & 80-84 & 119.77 & 136.97 & 104.47 & HT-Direct \\ 
  Ghana & EASTERN & 80-84 & 124.93 & 110.83 & 140.59 & RW2 \\ 
  Ghana & EASTERN & 85-89 & 101.64 & 119.90 & 85.90 & HT-Direct \\ 
  Ghana & EASTERN & 85-89 & 114.66 & 103.77 & 126.69 & RW2 \\ 
  Ghana & EASTERN & 90-94 & 117.69 & 137.92 & 100.08 & HT-Direct \\ 
  Ghana & EASTERN & 90-94 & 101.75 & 91.87 & 112.89 & RW2 \\ 
  Ghana & EASTERN & 95-99 & 88.75 & 110.14 & 71.18 & HT-Direct \\ 
  Ghana & EASTERN & 95-99 & 94.95 & 84.27 & 106.88 & RW2 \\ 
  Ghana & EASTERN & 00-04 & 69.93 & 91.94 & 52.88 & HT-Direct \\ 
  Ghana & EASTERN & 00-04 & 86.89 & 75.37 & 99.90 & RW2 \\ 
  Ghana & EASTERN & 05-09 & 69.93 & 89.68 & 54.27 & HT-Direct \\ 
  Ghana & EASTERN & 05-09 & 79.20 & 66.45 & 94.04 & RW2 \\ 
  Ghana & EASTERN & 10-14 & 75.23 & 107.86 & 51.89 & HT-Direct \\ 
  Ghana & EASTERN & 10-14 & 68.56 & 52.76 & 88.71 & RW2 \\ 
  Ghana & EASTERN & 15-19 & 58.26 & 23.54 & 134.20 & RW2 \\ 
  Ghana & GREATER ACCRA & 80-84 & 123.51 & 145.58 & 104.37 & HT-Direct \\ 
  Ghana & GREATER ACCRA & 80-84 & 128.66 & 111.40 & 148.28 & RW2 \\ 
  Ghana & GREATER ACCRA & 85-89 & 109.33 & 131.46 & 90.54 & HT-Direct \\ 
  Ghana & GREATER ACCRA & 85-89 & 108.66 & 96.61 & 122.11 & RW2 \\ 
  Ghana & GREATER ACCRA & 90-94 & 83.73 & 104.17 & 67.00 & HT-Direct \\ 
  Ghana & GREATER ACCRA & 90-94 & 88.61 & 78.16 & 100.10 & RW2 \\ 
  Ghana & GREATER ACCRA & 95-99 & 73.28 & 94.28 & 56.67 & HT-Direct \\ 
  Ghana & GREATER ACCRA & 95-99 & 76.87 & 66.42 & 88.68 & RW2 \\ 
  Ghana & GREATER ACCRA & 00-04 & 48.57 & 67.82 & 34.58 & HT-Direct \\ 
  Ghana & GREATER ACCRA & 00-04 & 66.14 & 55.53 & 78.68 & RW2 \\ 
  Ghana & GREATER ACCRA & 05-09 & 68.10 & 93.02 & 49.50 & HT-Direct \\ 
  Ghana & GREATER ACCRA & 05-09 & 57.18 & 45.41 & 71.77 & RW2 \\ 
  Ghana & GREATER ACCRA & 10-14 & 30.81 & 62.30 & 14.98 & HT-Direct \\ 
  Ghana & GREATER ACCRA & 10-14 & 46.62 & 32.70 & 66.76 & RW2 \\ 
  Ghana & GREATER ACCRA & 15-19 & 37.24 & 14.03 & 95.01 & RW2 \\ 
  Ghana & UPPER W,E \& NORTHERN & 80-84 & 248.49 & 270.79 & 227.45 & HT-Direct \\ 
  Ghana & UPPER W,E \& NORTHERN & 80-84 & 253.95 & 234.53 & 274.30 & RW2 \\ 
  Ghana & UPPER W,E \& NORTHERN & 85-89 & 206.91 & 223.23 & 191.49 & HT-Direct \\ 
  Ghana & UPPER W,E \& NORTHERN & 85-89 & 215.39 & 202.40 & 228.85 & RW2 \\ 
  Ghana & UPPER W,E \& NORTHERN & 90-94 & 171.04 & 185.98 & 157.07 & HT-Direct \\ 
  Ghana & UPPER W,E \& NORTHERN & 90-94 & 176.59 & 165.09 & 188.64 & RW2 \\ 
  Ghana & UPPER W,E \& NORTHERN & 95-99 & 140.55 & 154.72 & 127.48 & HT-Direct \\ 
  Ghana & UPPER W,E \& NORTHERN & 95-99 & 152.88 & 141.69 & 164.63 & RW2 \\ 
  Ghana & UPPER W,E \& NORTHERN & 00-04 & 126.60 & 140.37 & 114.00 & HT-Direct \\ 
  Ghana & UPPER W,E \& NORTHERN & 00-04 & 131.91 & 121.36 & 143.22 & RW2 \\ 
  Ghana & UPPER W,E \& NORTHERN & 05-09 & 117.70 & 136.86 & 100.90 & HT-Direct \\ 
  Ghana & UPPER W,E \& NORTHERN & 05-09 & 112.93 & 100.60 & 126.79 & RW2 \\ 
  Ghana & UPPER W,E \& NORTHERN & 10-14 & 75.79 & 93.35 & 61.31 & HT-Direct \\ 
  Ghana & UPPER W,E \& NORTHERN & 10-14 & 90.32 & 74.71 & 109.01 & RW2 \\ 
  Ghana & UPPER W,E \& NORTHERN & 15-19 & 70.15 & 29.55 & 156.69 & RW2 \\ 
  Ghana & VOLTA & 80-84 & 155.46 & 177.78 & 135.48 & HT-Direct \\ 
  Ghana & VOLTA & 80-84 & 153.91 & 136.94 & 172.42 & RW2 \\ 
  Ghana & VOLTA & 85-89 & 118.64 & 134.55 & 104.39 & HT-Direct \\ 
  Ghana & VOLTA & 85-89 & 130.51 & 119.13 & 142.80 & RW2 \\ 
  Ghana & VOLTA & 90-94 & 107.58 & 128.13 & 89.99 & HT-Direct \\ 
  Ghana & VOLTA & 90-94 & 107.46 & 96.72 & 119.17 & RW2 \\ 
  Ghana & VOLTA & 95-99 & 88.21 & 114.48 & 67.51 & HT-Direct \\ 
  Ghana & VOLTA & 95-99 & 93.99 & 82.52 & 106.76 & RW2 \\ 
  Ghana & VOLTA & 00-04 & 80.22 & 104.66 & 61.09 & HT-Direct \\ 
  Ghana & VOLTA & 00-04 & 81.15 & 69.24 & 94.69 & RW2 \\ 
  Ghana & VOLTA & 05-09 & 64.31 & 87.12 & 47.16 & HT-Direct \\ 
  Ghana & VOLTA & 05-09 & 69.67 & 56.53 & 85.68 & RW2 \\ 
  Ghana & VOLTA & 10-14 & 55.65 & 97.16 & 31.26 & HT-Direct \\ 
  Ghana & VOLTA & 10-14 & 56.31 & 41.16 & 77.19 & RW2 \\ 
  Ghana & VOLTA & 15-19 & 44.45 & 17.45 & 108.42 & RW2 \\ 
  Ghana & WESTERN & 80-84 & 147.64 & 175.97 & 123.19 & HT-Direct \\ 
  Ghana & WESTERN & 80-84 & 155.85 & 134.45 & 179.72 & RW2 \\ 
  Ghana & WESTERN & 85-89 & 126.72 & 148.82 & 107.49 & HT-Direct \\ 
  Ghana & WESTERN & 85-89 & 135.99 & 122.12 & 151.25 & RW2 \\ 
  Ghana & WESTERN & 90-94 & 117.50 & 138.05 & 99.66 & HT-Direct \\ 
  Ghana & WESTERN & 90-94 & 114.39 & 102.93 & 127.24 & RW2 \\ 
  Ghana & WESTERN & 95-99 & 92.48 & 114.40 & 74.40 & HT-Direct \\ 
  Ghana & WESTERN & 95-99 & 101.06 & 89.55 & 114.34 & RW2 \\ 
  Ghana & WESTERN & 00-04 & 93.60 & 117.08 & 74.43 & HT-Direct \\ 
  Ghana & WESTERN & 00-04 & 86.59 & 75.05 & 99.85 & RW2 \\ 
  Ghana & WESTERN & 05-09 & 61.65 & 83.35 & 45.32 & HT-Direct \\ 
  Ghana & WESTERN & 05-09 & 72.27 & 59.51 & 87.38 & RW2 \\ 
  Ghana & WESTERN & 10-14 & 44.56 & 67.36 & 29.23 & HT-Direct \\ 
  Ghana & WESTERN & 10-14 & 56.55 & 41.79 & 75.19 & RW2 \\ 
  Ghana & WESTERN & 15-19 & 42.90 & 16.77 & 103.57 & RW2 \\ 
  Guinea & ALL & 80-84 & 266.35 & 262.09 & 270.46 & IHME \\ 
  Guinea & ALL & 80-84 & 277.54 & 261.89 & 293.76 & RW2 \\ 
  Guinea & ALL & 80-84 & 277.32 & 268.92 & 287.27 & UN \\ 
  Guinea & ALL & 85-89 & 243.38 & 239.92 & 247.15 & IHME \\ 
  Guinea & ALL & 85-89 & 252.93 & 240.41 & 265.82 & RW2 \\ 
  Guinea & ALL & 85-89 & 253.38 & 246.20 & 260.64 & UN \\ 
  Guinea & ALL & 90-94 & 218.40 & 215.40 & 221.44 & IHME \\ 
  Guinea & ALL & 90-94 & 225.75 & 216.66 & 235.12 & RW2 \\ 
  Guinea & ALL & 90-94 & 225.40 & 219.01 & 232.15 & UN \\ 
  Guinea & ALL & 95-99 & 191.12 & 188.41 & 193.91 & IHME \\ 
  Guinea & ALL & 95-99 & 191.68 & 183.93 & 199.61 & RW2 \\ 
  Guinea & ALL & 95-99 & 191.83 & 185.91 & 197.59 & UN \\ 
  Guinea & ALL & 00-04 & 164.93 & 162.06 & 167.72 & IHME \\ 
  Guinea & ALL & 00-04 & 155.90 & 148.49 & 163.64 & RW2 \\ 
  Guinea & ALL & 00-04 & 156.13 & 150.88 & 161.29 & UN \\ 
  Guinea & ALL & 05-09 & 141.98 & 138.53 & 145.73 & IHME \\ 
  Guinea & ALL & 05-09 & 127.16 & 117.49 & 137.51 & RW2 \\ 
  Guinea & ALL & 05-09 & 126.22 & 120.11 & 132.37 & UN \\ 
  Guinea & ALL & 10-14 & 120.45 & 116.00 & 125.13 & IHME \\ 
  Guinea & ALL & 10-14 & 103.98 & 91.81 & 117.45 & RW2 \\ 
  Guinea & ALL & 10-14 & 104.64 & 96.21 & 113.99 & UN \\ 
  Guinea & CENTRAL GUINEA & 80-84 & 263.88 & 294.91 & 235.03 & HT-Direct \\ 
  Guinea & CENTRAL GUINEA & 80-84 & 266.06 & 239.46 & 294.61 & RW2 \\ 
  Guinea & CENTRAL GUINEA & 85-89 & 220.23 & 243.53 & 198.57 & HT-Direct \\ 
  Guinea & CENTRAL GUINEA & 85-89 & 237.83 & 219.77 & 256.43 & RW2 \\ 
  Guinea & CENTRAL GUINEA & 90-94 & 206.07 & 223.56 & 189.62 & HT-Direct \\ 
  Guinea & CENTRAL GUINEA & 90-94 & 213.22 & 199.49 & 227.47 & RW2 \\ 
  Guinea & CENTRAL GUINEA & 95-99 & 182.20 & 198.39 & 167.07 & HT-Direct \\ 
  Guinea & CENTRAL GUINEA & 95-99 & 186.64 & 174.83 & 199.18 & RW2 \\ 
  Guinea & CENTRAL GUINEA & 00-04 & 163.77 & 176.99 & 151.35 & HT-Direct \\ 
  Guinea & CENTRAL GUINEA & 00-04 & 159.14 & 149.08 & 169.95 & RW2 \\ 
  Guinea & CENTRAL GUINEA & 05-09 & 139.19 & 165.35 & 116.58 & HT-Direct \\ 
  Guinea & CENTRAL GUINEA & 05-09 & 129.50 & 115.52 & 145.02 & RW2 \\ 
  Guinea & CENTRAL GUINEA & 10-14 & 98.27 & 131.01 & 73.02 & HT-Direct \\ 
  Guinea & CENTRAL GUINEA & 10-14 & 101.78 & 81.02 & 126.00 & RW2 \\ 
  Guinea & CENTRAL GUINEA & 15-19 & 78.69 & 32.32 & 176.51 & RW2 \\ 
  Guinea & CONAKRY & 80-84 & 209.49 & 248.24 & 175.38 & HT-Direct \\ 
  Guinea & CONAKRY & 80-84 & 215.12 & 183.40 & 250.92 & RW2 \\ 
  Guinea & CONAKRY & 85-89 & 162.95 & 190.09 & 139.02 & HT-Direct \\ 
  Guinea & CONAKRY & 85-89 & 176.29 & 156.85 & 197.57 & RW2 \\ 
  Guinea & CONAKRY & 90-94 & 147.80 & 170.04 & 128.03 & HT-Direct \\ 
  Guinea & CONAKRY & 90-94 & 144.29 & 129.70 & 160.50 & RW2 \\ 
  Guinea & CONAKRY & 95-99 & 109.77 & 126.99 & 94.63 & HT-Direct \\ 
  Guinea & CONAKRY & 95-99 & 112.83 & 100.18 & 126.48 & RW2 \\ 
  Guinea & CONAKRY & 00-04 & 63.38 & 86.06 & 46.38 & HT-Direct \\ 
  Guinea & CONAKRY & 00-04 & 87.36 & 73.91 & 102.42 & RW2 \\ 
  Guinea & CONAKRY & 05-09 & 88.62 & 123.66 & 62.80 & HT-Direct \\ 
  Guinea & CONAKRY & 05-09 & 68.52 & 54.75 & 85.52 & RW2 \\ 
  Guinea & CONAKRY & 10-14 & 56.44 & 100.38 & 31.07 & HT-Direct \\ 
  Guinea & CONAKRY & 10-14 & 51.52 & 34.64 & 77.16 & RW2 \\ 
  Guinea & CONAKRY & 15-19 & 38.09 & 12.90 & 108.01 & RW2 \\ 
  Guinea & FOREST GUINEA & 80-84 & 275.95 & 305.42 & 248.31 & HT-Direct \\ 
  Guinea & FOREST GUINEA & 80-84 & 288.06 & 259.96 & 317.83 & RW2 \\ 
  Guinea & FOREST GUINEA & 85-89 & 271.89 & 298.40 & 246.92 & HT-Direct \\ 
  Guinea & FOREST GUINEA & 85-89 & 290.70 & 269.42 & 313.23 & RW2 \\ 
  Guinea & FOREST GUINEA & 90-94 & 270.30 & 292.17 & 249.49 & HT-Direct \\ 
  Guinea & FOREST GUINEA & 90-94 & 270.41 & 252.94 & 289.05 & RW2 \\ 
  Guinea & FOREST GUINEA & 95-99 & 211.77 & 229.46 & 195.10 & HT-Direct \\ 
  Guinea & FOREST GUINEA & 95-99 & 217.86 & 203.69 & 233.09 & RW2 \\ 
  Guinea & FOREST GUINEA & 00-04 & 167.40 & 185.54 & 150.70 & HT-Direct \\ 
  Guinea & FOREST GUINEA & 00-04 & 162.06 & 148.87 & 175.98 & RW2 \\ 
  Guinea & FOREST GUINEA & 05-09 & 112.01 & 139.83 & 89.15 & HT-Direct \\ 
  Guinea & FOREST GUINEA & 05-09 & 119.51 & 103.97 & 136.41 & RW2 \\ 
  Guinea & FOREST GUINEA & 10-14 & 101.43 & 128.31 & 79.66 & HT-Direct \\ 
  Guinea & FOREST GUINEA & 10-14 & 93.99 & 76.40 & 114.98 & RW2 \\ 
  Guinea & FOREST GUINEA & 15-19 & 75.30 & 31.31 & 171.68 & RW2 \\ 
  Guinea & LOWER GUINEA & 80-84 & 272.29 & 305.73 & 241.25 & HT-Direct \\ 
  Guinea & LOWER GUINEA & 80-84 & 276.23 & 248.00 & 307.04 & RW2 \\ 
  Guinea & LOWER GUINEA & 85-89 & 229.17 & 250.90 & 208.79 & HT-Direct \\ 
  Guinea & LOWER GUINEA & 85-89 & 243.17 & 225.06 & 261.86 & RW2 \\ 
  Guinea & LOWER GUINEA & 90-94 & 206.08 & 224.88 & 188.47 & HT-Direct \\ 
  Guinea & LOWER GUINEA & 90-94 & 216.37 & 201.61 & 231.60 & RW2 \\ 
  Guinea & LOWER GUINEA & 95-99 & 184.50 & 201.52 & 168.62 & HT-Direct \\ 
  Guinea & LOWER GUINEA & 95-99 & 188.51 & 175.72 & 202.25 & RW2 \\ 
  Guinea & LOWER GUINEA & 00-04 & 164.27 & 181.69 & 148.21 & HT-Direct \\ 
  Guinea & LOWER GUINEA & 00-04 & 157.21 & 145.01 & 170.62 & RW2 \\ 
  Guinea & LOWER GUINEA & 05-09 & 135.99 & 167.08 & 109.93 & HT-Direct \\ 
  Guinea & LOWER GUINEA & 05-09 & 121.07 & 105.83 & 138.14 & RW2 \\ 
  Guinea & LOWER GUINEA & 10-14 & 72.30 & 102.61 & 50.44 & HT-Direct \\ 
  Guinea & LOWER GUINEA & 10-14 & 85.64 & 64.66 & 110.86 & RW2 \\ 
  Guinea & LOWER GUINEA & 15-19 & 58.40 & 22.40 & 140.62 & RW2 \\ 
  Guinea & UPPER GUINEA & 80-84 & 308.94 & 344.20 & 275.77 & HT-Direct \\ 
  Guinea & UPPER GUINEA & 80-84 & 314.90 & 284.19 & 347.63 & RW2 \\ 
  Guinea & UPPER GUINEA & 85-89 & 277.30 & 303.72 & 252.34 & HT-Direct \\ 
  Guinea & UPPER GUINEA & 85-89 & 286.40 & 266.33 & 307.90 & RW2 \\ 
  Guinea & UPPER GUINEA & 90-94 & 245.07 & 264.09 & 226.99 & HT-Direct \\ 
  Guinea & UPPER GUINEA & 90-94 & 251.76 & 236.81 & 267.15 & RW2 \\ 
  Guinea & UPPER GUINEA & 95-99 & 204.03 & 218.50 & 190.29 & HT-Direct \\ 
  Guinea & UPPER GUINEA & 95-99 & 212.02 & 199.72 & 224.68 & RW2 \\ 
  Guinea & UPPER GUINEA & 00-04 & 180.16 & 197.54 & 163.99 & HT-Direct \\ 
  Guinea & UPPER GUINEA & 00-04 & 183.84 & 171.19 & 197.10 & RW2 \\ 
  Guinea & UPPER GUINEA & 05-09 & 194.77 & 222.05 & 170.11 & HT-Direct \\ 
  Guinea & UPPER GUINEA & 05-09 & 169.26 & 153.44 & 186.47 & RW2 \\ 
  Guinea & UPPER GUINEA & 10-14 & 174.31 & 212.79 & 141.55 & HT-Direct \\ 
  Guinea & UPPER GUINEA & 10-14 & 163.51 & 137.31 & 194.68 & RW2 \\ 
  Guinea & UPPER GUINEA & 15-19 & 159.61 & 72.09 & 318.82 & RW2 \\ 
  Kenya & ALL & 80-84 & 101.94 & 100.33 & 103.62 & IHME \\ 
  Kenya & ALL & 80-84 & 102.59 & 95.32 & 110.36 & RW2 \\ 
  Kenya & ALL & 80-84 & 102.57 & 99.81 & 105.46 & UN \\ 
  Kenya & ALL & 85-89 & 93.67 & 92.16 & 95.14 & IHME \\ 
  Kenya & ALL & 85-89 & 97.40 & 91.44 & 103.65 & RW2 \\ 
  Kenya & ALL & 85-89 & 97.47 & 94.70 & 100.12 & UN \\ 
  Kenya & ALL & 90-94 & 93.48 & 91.93 & 95.02 & IHME \\ 
  Kenya & ALL & 90-94 & 108.35 & 102.29 & 114.75 & RW2 \\ 
  Kenya & ALL & 90-94 & 108.12 & 105.07 & 111.37 & UN \\ 
  Kenya & ALL & 95-99 & 93.71 & 92.03 & 95.27 & IHME \\ 
  Kenya & ALL & 95-99 & 113.65 & 106.96 & 120.65 & RW2 \\ 
  Kenya & ALL & 95-99 & 113.99 & 110.54 & 117.42 & UN \\ 
  Kenya & ALL & 00-04 & 81.88 & 80.14 & 83.73 & IHME \\ 
  Kenya & ALL & 00-04 & 100.00 & 93.65 & 106.79 & RW2 \\ 
  Kenya & ALL & 00-04 & 99.71 & 96.14 & 103.53 & UN \\ 
  Kenya & ALL & 05-09 & 63.75 & 62.16 & 65.45 & IHME \\ 
  Kenya & ALL & 05-09 & 75.16 & 68.68 & 82.16 & RW2 \\ 
  Kenya & ALL & 05-09 & 75.42 & 71.54 & 79.43 & UN \\ 
  Kenya & ALL & 10-14 & 52.82 & 50.68 & 54.98 & IHME \\ 
  Kenya & ALL & 10-14 & 56.44 & 51.21 & 62.13 & RW2 \\ 
  Kenya & ALL & 10-14 & 56.31 & 51.59 & 61.06 & UN \\ 
  Kenya & CENTRAL & 80-84 & 48.15 & 61.80 & 37.40 & HT-Direct \\ 
  Kenya & CENTRAL & 80-84 & 49.10 & 39.56 & 60.98 & RW2 \\ 
  Kenya & CENTRAL & 85-89 & 42.31 & 53.90 & 33.13 & HT-Direct \\ 
  Kenya & CENTRAL & 85-89 & 44.01 & 37.59 & 51.41 & RW2 \\ 
  Kenya & CENTRAL & 90-94 & 49.57 & 60.73 & 40.38 & HT-Direct \\ 
  Kenya & CENTRAL & 90-94 & 53.85 & 46.56 & 62.17 & RW2 \\ 
  Kenya & CENTRAL & 95-99 & 56.26 & 71.49 & 44.12 & HT-Direct \\ 
  Kenya & CENTRAL & 95-99 & 68.02 & 57.99 & 79.43 & RW2 \\ 
  Kenya & CENTRAL & 00-04 & 60.07 & 76.93 & 46.71 & HT-Direct \\ 
  Kenya & CENTRAL & 00-04 & 67.01 & 56.52 & 79.36 & RW2 \\ 
  Kenya & CENTRAL & 05-09 & 52.10 & 68.95 & 39.20 & HT-Direct \\ 
  Kenya & CENTRAL & 05-09 & 56.51 & 45.88 & 69.35 & RW2 \\ 
  Kenya & CENTRAL & 10-14 & 40.79 & 57.56 & 28.75 & HT-Direct \\ 
  Kenya & CENTRAL & 10-14 & 47.71 & 35.53 & 64.07 & RW2 \\ 
  Kenya & CENTRAL & 15-19 & 40.82 & 14.93 & 106.86 & RW2 \\ 
  Kenya & COAST & 80-84 & 156.99 & 181.03 & 135.62 & HT-Direct \\ 
  Kenya & COAST & 80-84 & 154.42 & 135.21 & 176.20 & RW2 \\ 
  Kenya & COAST & 85-89 & 105.60 & 120.65 & 92.24 & HT-Direct \\ 
  Kenya & COAST & 85-89 & 116.38 & 104.89 & 128.81 & RW2 \\ 
  Kenya & COAST & 90-94 & 107.92 & 122.93 & 94.54 & HT-Direct \\ 
  Kenya & COAST & 90-94 & 120.41 & 108.61 & 133.04 & RW2 \\ 
  Kenya & COAST & 95-99 & 137.79 & 163.20 & 115.79 & HT-Direct \\ 
  Kenya & COAST & 95-99 & 128.56 & 113.73 & 145.22 & RW2 \\ 
  Kenya & COAST & 00-04 & 80.76 & 97.76 & 66.49 & HT-Direct \\ 
  Kenya & COAST & 00-04 & 106.07 & 92.52 & 121.30 & RW2 \\ 
  Kenya & COAST & 05-09 & 66.19 & 79.86 & 54.73 & HT-Direct \\ 
  Kenya & COAST & 05-09 & 76.81 & 65.06 & 90.40 & RW2 \\ 
  Kenya & COAST & 10-14 & 54.96 & 74.98 & 40.06 & HT-Direct \\ 
  Kenya & COAST & 10-14 & 56.99 & 43.99 & 74.00 & RW2 \\ 
  Kenya & COAST & 15-19 & 42.81 & 16.01 & 110.65 & RW2 \\ 
  Kenya & EASTERN & 80-84 & 75.07 & 88.74 & 63.37 & HT-Direct \\ 
  Kenya & EASTERN & 80-84 & 76.36 & 65.74 & 88.73 & RW2 \\ 
  Kenya & EASTERN & 85-89 & 62.33 & 73.67 & 52.64 & HT-Direct \\ 
  Kenya & EASTERN & 85-89 & 64.77 & 57.59 & 72.76 & RW2 \\ 
  Kenya & EASTERN & 90-94 & 64.04 & 74.95 & 54.62 & HT-Direct \\ 
  Kenya & EASTERN & 90-94 & 75.33 & 67.43 & 84.03 & RW2 \\ 
  Kenya & EASTERN & 95-99 & 85.41 & 99.59 & 73.09 & HT-Direct \\ 
  Kenya & EASTERN & 95-99 & 89.90 & 79.98 & 100.96 & RW2 \\ 
  Kenya & EASTERN & 00-04 & 76.93 & 91.16 & 64.77 & HT-Direct \\ 
  Kenya & EASTERN & 00-04 & 81.91 & 71.94 & 93.13 & RW2 \\ 
  Kenya & EASTERN & 05-09 & 45.79 & 58.05 & 36.01 & HT-Direct \\ 
  Kenya & EASTERN & 05-09 & 63.68 & 53.82 & 75.12 & RW2 \\ 
  Kenya & EASTERN & 10-14 & 46.60 & 60.62 & 35.69 & HT-Direct \\ 
  Kenya & EASTERN & 10-14 & 50.32 & 39.94 & 63.33 & RW2 \\ 
  Kenya & EASTERN & 15-19 & 40.38 & 15.48 & 100.64 & RW2 \\ 
  Kenya & NAIROBI & 80-84 & 66.08 & 97.47 & 44.31 & HT-Direct \\ 
  Kenya & NAIROBI & 80-84 & 68.51 & 50.57 & 92.56 & RW2 \\ 
  Kenya & NAIROBI & 85-89 & 60.97 & 83.52 & 44.22 & HT-Direct \\ 
  Kenya & NAIROBI & 85-89 & 63.12 & 51.27 & 77.58 & RW2 \\ 
  Kenya & NAIROBI & 90-94 & 72.91 & 97.05 & 54.41 & HT-Direct \\ 
  Kenya & NAIROBI & 90-94 & 78.40 & 65.23 & 93.90 & RW2 \\ 
  Kenya & NAIROBI & 95-99 & 102.28 & 140.88 & 73.35 & HT-Direct \\ 
  Kenya & NAIROBI & 95-99 & 99.07 & 81.91 & 119.29 & RW2 \\ 
  Kenya & NAIROBI & 00-04 & 72.78 & 97.93 & 53.70 & HT-Direct \\ 
  Kenya & NAIROBI & 00-04 & 97.52 & 79.84 & 118.40 & RW2 \\ 
  Kenya & NAIROBI & 05-09 & 71.39 & 102.08 & 49.42 & HT-Direct \\ 
  Kenya & NAIROBI & 05-09 & 83.28 & 66.06 & 104.46 & RW2 \\ 
  Kenya & NAIROBI & 10-14 & 70.98 & 99.76 & 50.04 & HT-Direct \\ 
  Kenya & NAIROBI & 10-14 & 72.15 & 52.98 & 97.66 & RW2 \\ 
  Kenya & NAIROBI & 15-19 & 62.98 & 23.32 & 160.58 & RW2 \\ 
  Kenya & NORTHEASTERN & 80-84 & 219.17 & 312.84 & 147.53 & HT-Direct \\ 
  Kenya & NORTHEASTERN & 80-84 & 209.80 & 157.73 & 271.51 & RW2 \\ 
  Kenya & NORTHEASTERN & 85-89 & 117.52 & 150.44 & 91.04 & HT-Direct \\ 
  Kenya & NORTHEASTERN & 85-89 & 155.88 & 130.77 & 184.21 & RW2 \\ 
  Kenya & NORTHEASTERN & 90-94 & 169.72 & 200.82 & 142.57 & HT-Direct \\ 
  Kenya & NORTHEASTERN & 90-94 & 153.08 & 133.77 & 175.08 & RW2 \\ 
  Kenya & NORTHEASTERN & 95-99 & 133.24 & 162.55 & 108.54 & HT-Direct \\ 
  Kenya & NORTHEASTERN & 95-99 & 146.68 & 127.01 & 169.07 & RW2 \\ 
  Kenya & NORTHEASTERN & 00-04 & 77.82 & 97.41 & 61.90 & HT-Direct \\ 
  Kenya & NORTHEASTERN & 00-04 & 106.89 & 90.18 & 125.51 & RW2 \\ 
  Kenya & NORTHEASTERN & 05-09 & 56.48 & 74.33 & 42.71 & HT-Direct \\ 
  Kenya & NORTHEASTERN & 05-09 & 69.20 & 56.38 & 84.57 & RW2 \\ 
  Kenya & NORTHEASTERN & 10-14 & 47.82 & 65.18 & 34.90 & HT-Direct \\ 
  Kenya & NORTHEASTERN & 10-14 & 46.84 & 35.28 & 62.26 & RW2 \\ 
  Kenya & NORTHEASTERN & 15-19 & 32.16 & 11.68 & 87.41 & RW2 \\ 
  Kenya & NYANZA & 80-84 & 151.29 & 173.65 & 131.35 & HT-Direct \\ 
  Kenya & NYANZA & 80-84 & 163.28 & 142.92 & 185.21 & RW2 \\ 
  Kenya & NYANZA & 85-89 & 169.65 & 187.37 & 153.30 & HT-Direct \\ 
  Kenya & NYANZA & 85-89 & 174.04 & 159.89 & 189.36 & RW2 \\ 
  Kenya & NYANZA & 90-94 & 203.88 & 226.12 & 183.32 & HT-Direct \\ 
  Kenya & NYANZA & 90-94 & 217.41 & 199.74 & 236.37 & RW2 \\ 
  Kenya & NYANZA & 95-99 & 216.33 & 244.19 & 190.84 & HT-Direct \\ 
  Kenya & NYANZA & 95-99 & 250.29 & 227.06 & 275.19 & RW2 \\ 
  Kenya & NYANZA & 00-04 & 205.36 & 236.89 & 177.06 & HT-Direct \\ 
  Kenya & NYANZA & 00-04 & 215.27 & 191.98 & 241.26 & RW2 \\ 
  Kenya & NYANZA & 05-09 & 116.34 & 135.57 & 99.53 & HT-Direct \\ 
  Kenya & NYANZA & 05-09 & 139.38 & 121.77 & 159.33 & RW2 \\ 
  Kenya & NYANZA & 10-14 & 69.00 & 83.34 & 56.97 & HT-Direct \\ 
  Kenya & NYANZA & 10-14 & 79.02 & 65.33 & 95.20 & RW2 \\ 
  Kenya & NYANZA & 15-19 & 42.77 & 16.47 & 106.57 & RW2 \\ 
  Kenya & RIFT VALLEY & 80-84 & 79.34 & 93.04 & 67.50 & HT-Direct \\ 
  Kenya & RIFT VALLEY & 80-84 & 79.03 & 68.62 & 90.88 & RW2 \\ 
  Kenya & RIFT VALLEY & 85-89 & 61.90 & 71.65 & 53.40 & HT-Direct \\ 
  Kenya & RIFT VALLEY & 85-89 & 67.94 & 61.09 & 75.30 & RW2 \\ 
  Kenya & RIFT VALLEY & 90-94 & 75.98 & 86.61 & 66.56 & HT-Direct \\ 
  Kenya & RIFT VALLEY & 90-94 & 77.85 & 70.61 & 85.62 & RW2 \\ 
  Kenya & RIFT VALLEY & 95-99 & 73.32 & 85.01 & 63.13 & HT-Direct \\ 
  Kenya & RIFT VALLEY & 95-99 & 90.23 & 80.93 & 100.31 & RW2 \\ 
  Kenya & RIFT VALLEY & 00-04 & 71.86 & 82.30 & 62.65 & HT-Direct \\ 
  Kenya & RIFT VALLEY & 00-04 & 81.50 & 73.09 & 90.70 & RW2 \\ 
  Kenya & RIFT VALLEY & 05-09 & 49.53 & 57.60 & 42.54 & HT-Direct \\ 
  Kenya & RIFT VALLEY & 05-09 & 62.18 & 54.74 & 70.53 & RW2 \\ 
  Kenya & RIFT VALLEY & 10-14 & 45.68 & 53.97 & 38.61 & HT-Direct \\ 
  Kenya & RIFT VALLEY & 10-14 & 47.21 & 40.30 & 55.36 & RW2 \\ 
  Kenya & RIFT VALLEY & 15-19 & 35.99 & 14.43 & 88.01 & RW2 \\ 
  Kenya & WESTERN & 80-84 & 119.11 & 139.54 & 101.32 & HT-Direct \\ 
  Kenya & WESTERN & 80-84 & 124.21 & 107.17 & 143.54 & RW2 \\ 
  Kenya & WESTERN & 85-89 & 118.37 & 135.87 & 102.85 & HT-Direct \\ 
  Kenya & WESTERN & 85-89 & 124.33 & 111.99 & 138.08 & RW2 \\ 
  Kenya & WESTERN & 90-94 & 145.57 & 163.39 & 129.40 & HT-Direct \\ 
  Kenya & WESTERN & 90-94 & 152.43 & 138.58 & 167.49 & RW2 \\ 
  Kenya & WESTERN & 95-99 & 143.68 & 166.33 & 123.66 & HT-Direct \\ 
  Kenya & WESTERN & 95-99 & 177.16 & 158.68 & 197.25 & RW2 \\ 
  Kenya & WESTERN & 00-04 & 148.67 & 171.88 & 128.11 & HT-Direct \\ 
  Kenya & WESTERN & 00-04 & 155.27 & 137.37 & 175.84 & RW2 \\ 
  Kenya & WESTERN & 05-09 & 90.07 & 112.17 & 71.97 & HT-Direct \\ 
  Kenya & WESTERN & 05-09 & 103.70 & 88.49 & 121.27 & RW2 \\ 
  Kenya & WESTERN & 10-14 & 53.89 & 66.20 & 43.76 & HT-Direct \\ 
  Kenya & WESTERN & 10-14 & 61.85 & 50.28 & 75.48 & RW2 \\ 
  Kenya & WESTERN & 15-19 & 35.84 & 13.69 & 89.26 & RW2 \\ 
  Lesotho & ALL & 80-84 & 98.56 & 96.09 & 101.23 & IHME \\ 
  Lesotho & ALL & 80-84 & 109.45 & 88.41 & 134.86 & RW2 \\ 
  Lesotho & ALL & 80-84 & 109.69 & 105.00 & 115.30 & UN \\ 
  Lesotho & ALL & 85-89 & 94.41 & 92.09 & 96.83 & IHME \\ 
  Lesotho & ALL & 85-89 & 91.81 & 80.26 & 104.69 & RW2 \\ 
  Lesotho & ALL & 85-89 & 92.68 & 88.60 & 96.91 & UN \\ 
  Lesotho & ALL & 90-94 & 90.80 & 88.68 & 92.82 & IHME \\ 
  Lesotho & ALL & 90-94 & 91.73 & 83.05 & 101.22 & RW2 \\ 
  Lesotho & ALL & 90-94 & 90.01 & 86.45 & 93.80 & UN \\ 
  Lesotho & ALL & 95-99 & 97.19 & 95.08 & 99.45 & IHME \\ 
  Lesotho & ALL & 95-99 & 104.95 & 92.86 & 118.22 & RW2 \\ 
  Lesotho & ALL & 95-99 & 107.25 & 103.63 & 111.67 & UN \\ 
  Lesotho & ALL & 00-04 & 112.34 & 109.67 & 115.04 & IHME \\ 
  Lesotho & ALL & 00-04 & 121.07 & 111.35 & 131.59 & RW2 \\ 
  Lesotho & ALL & 00-04 & 120.24 & 115.87 & 124.57 & UN \\ 
  Lesotho & ALL & 05-09 & 116.61 & 112.84 & 120.42 & IHME \\ 
  Lesotho & ALL & 05-09 & 117.67 & 106.51 & 129.85 & RW2 \\ 
  Lesotho & ALL & 05-09 & 117.97 & 113.03 & 123.21 & UN \\ 
  Lesotho & ALL & 10-14 & 99.72 & 94.23 & 106.12 & IHME \\ 
  Lesotho & ALL & 10-14 & 95.68 & 81.15 & 112.32 & RW2 \\ 
  Lesotho & ALL & 10-14 & 95.61 & 88.80 & 103.23 & UN \\ 
  Lesotho & BEREA & 80-84 & 156.27 & 216.69 & 110.33 & HT-Direct \\ 
  Lesotho & BEREA & 80-84 & 153.18 & 110.39 & 208.38 & RW2 \\ 
  Lesotho & BEREA & 85-89 & 78.06 & 112.50 & 53.53 & HT-Direct \\ 
  Lesotho & BEREA & 85-89 & 113.26 & 88.62 & 143.50 & RW2 \\ 
  Lesotho & BEREA & 90-94 & 102.41 & 135.41 & 76.74 & HT-Direct \\ 
  Lesotho & BEREA & 90-94 & 103.95 & 86.38 & 124.33 & RW2 \\ 
  Lesotho & BEREA & 95-99 & 84.42 & 114.54 & 61.67 & HT-Direct \\ 
  Lesotho & BEREA & 95-99 & 107.93 & 90.61 & 127.98 & RW2 \\ 
  Lesotho & BEREA & 00-04 & 106.50 & 133.82 & 84.21 & HT-Direct \\ 
  Lesotho & BEREA & 00-04 & 109.00 & 93.36 & 127.13 & RW2 \\ 
  Lesotho & BEREA & 05-09 & 92.54 & 122.03 & 69.61 & HT-Direct \\ 
  Lesotho & BEREA & 05-09 & 100.89 & 83.03 & 121.89 & RW2 \\ 
  Lesotho & BEREA & 10-14 & 75.48 & 111.98 & 50.20 & HT-Direct \\ 
  Lesotho & BEREA & 10-14 & 75.59 & 56.37 & 100.65 & RW2 \\ 
  Lesotho & BEREA & 15-19 & 52.49 & 19.26 & 132.79 & RW2 \\ 
  Lesotho & BUTHA-BUTHE & 80-84 & 57.90 & 104.19 & 31.46 & HT-Direct \\ 
  Lesotho & BUTHA-BUTHE & 80-84 & 91.15 & 59.08 & 138.08 & RW2 \\ 
  Lesotho & BUTHA-BUTHE & 85-89 & 70.92 & 105.47 & 47.09 & HT-Direct \\ 
  Lesotho & BUTHA-BUTHE & 85-89 & 73.80 & 55.05 & 98.40 & RW2 \\ 
  Lesotho & BUTHA-BUTHE & 90-94 & 71.81 & 96.96 & 52.81 & HT-Direct \\ 
  Lesotho & BUTHA-BUTHE & 90-94 & 74.65 & 60.29 & 91.88 & RW2 \\ 
  Lesotho & BUTHA-BUTHE & 95-99 & 75.18 & 110.59 & 50.46 & HT-Direct \\ 
  Lesotho & BUTHA-BUTHE & 95-99 & 84.79 & 69.42 & 103.23 & RW2 \\ 
  Lesotho & BUTHA-BUTHE & 00-04 & 84.13 & 110.09 & 63.86 & HT-Direct \\ 
  Lesotho & BUTHA-BUTHE & 00-04 & 93.41 & 78.32 & 111.15 & RW2 \\ 
  Lesotho & BUTHA-BUTHE & 05-09 & 90.93 & 121.32 & 67.57 & HT-Direct \\ 
  Lesotho & BUTHA-BUTHE & 05-09 & 94.77 & 76.29 & 117.09 & RW2 \\ 
  Lesotho & BUTHA-BUTHE & 10-14 & 70.91 & 116.07 & 42.48 & HT-Direct \\ 
  Lesotho & BUTHA-BUTHE & 10-14 & 78.31 & 55.74 & 109.18 & RW2 \\ 
  Lesotho & BUTHA-BUTHE & 15-19 & 60.50 & 21.31 & 156.43 & RW2 \\ 
  Lesotho & LERIBE & 80-84 & 99.07 & 168.98 & 56.13 & HT-Direct \\ 
  Lesotho & LERIBE & 80-84 & 102.20 & 70.07 & 146.76 & RW2 \\ 
  Lesotho & LERIBE & 85-89 & 86.06 & 117.01 & 62.72 & HT-Direct \\ 
  Lesotho & LERIBE & 85-89 & 85.57 & 66.59 & 109.38 & RW2 \\ 
  Lesotho & LERIBE & 90-94 & 73.28 & 97.22 & 54.87 & HT-Direct \\ 
  Lesotho & LERIBE & 90-94 & 89.64 & 74.32 & 107.65 & RW2 \\ 
  Lesotho & LERIBE & 95-99 & 86.06 & 119.76 & 61.19 & HT-Direct \\ 
  Lesotho & LERIBE & 95-99 & 106.45 & 89.35 & 126.16 & RW2 \\ 
  Lesotho & LERIBE & 00-04 & 121.32 & 152.78 & 95.62 & HT-Direct \\ 
  Lesotho & LERIBE & 00-04 & 122.65 & 105.59 & 141.72 & RW2 \\ 
  Lesotho & LERIBE & 05-09 & 130.16 & 164.10 & 102.38 & HT-Direct \\ 
  Lesotho & LERIBE & 05-09 & 129.72 & 108.88 & 154.00 & RW2 \\ 
  Lesotho & LERIBE & 10-14 & 97.98 & 138.70 & 68.27 & HT-Direct \\ 
  Lesotho & LERIBE & 10-14 & 112.14 & 86.24 & 145.19 & RW2 \\ 
  Lesotho & LERIBE & 15-19 & 90.67 & 34.51 & 213.69 & RW2 \\ 
  Lesotho & MAFETENG & 80-84 & 59.65 & 107.92 & 32.19 & HT-Direct \\ 
  Lesotho & MAFETENG & 80-84 & 97.74 & 64.03 & 147.47 & RW2 \\ 
  Lesotho & MAFETENG & 85-89 & 90.78 & 128.73 & 63.21 & HT-Direct \\ 
  Lesotho & MAFETENG & 85-89 & 81.04 & 60.70 & 107.68 & RW2 \\ 
  Lesotho & MAFETENG & 90-94 & 78.96 & 110.97 & 55.60 & HT-Direct \\ 
  Lesotho & MAFETENG & 90-94 & 83.98 & 67.45 & 104.10 & RW2 \\ 
  Lesotho & MAFETENG & 95-99 & 68.47 & 103.28 & 44.80 & HT-Direct \\ 
  Lesotho & MAFETENG & 95-99 & 98.47 & 80.59 & 119.35 & RW2 \\ 
  Lesotho & MAFETENG & 00-04 & 96.49 & 125.32 & 73.74 & HT-Direct \\ 
  Lesotho & MAFETENG & 00-04 & 112.44 & 95.13 & 132.30 & RW2 \\ 
  Lesotho & MAFETENG & 05-09 & 121.57 & 151.84 & 96.64 & HT-Direct \\ 
  Lesotho & MAFETENG & 05-09 & 117.97 & 97.54 & 141.97 & RW2 \\ 
  Lesotho & MAFETENG & 10-14 & 89.59 & 155.58 & 49.93 & HT-Direct \\ 
  Lesotho & MAFETENG & 10-14 & 100.54 & 72.73 & 137.91 & RW2 \\ 
  Lesotho & MAFETENG & 15-19 & 80.22 & 28.23 & 200.48 & RW2 \\ 
  Lesotho & MASERU & 80-84 & 50.39 & 82.60 & 30.32 & HT-Direct \\ 
  Lesotho & MASERU & 80-84 & 76.38 & 52.27 & 109.44 & RW2 \\ 
  Lesotho & MASERU & 85-89 & 52.49 & 77.12 & 35.42 & HT-Direct \\ 
  Lesotho & MASERU & 85-89 & 68.27 & 52.47 & 88.01 & RW2 \\ 
  Lesotho & MASERU & 90-94 & 81.30 & 105.52 & 62.25 & HT-Direct \\ 
  Lesotho & MASERU & 90-94 & 76.65 & 63.90 & 91.68 & RW2 \\ 
  Lesotho & MASERU & 95-99 & 74.79 & 96.62 & 57.57 & HT-Direct \\ 
  Lesotho & MASERU & 95-99 & 96.44 & 81.99 & 113.17 & RW2 \\ 
  Lesotho & MASERU & 00-04 & 119.31 & 144.32 & 98.14 & HT-Direct \\ 
  Lesotho & MASERU & 00-04 & 116.47 & 101.39 & 133.61 & RW2 \\ 
  Lesotho & MASERU & 05-09 & 111.96 & 147.47 & 84.16 & HT-Direct \\ 
  Lesotho & MASERU & 05-09 & 127.45 & 106.63 & 151.96 & RW2 \\ 
  Lesotho & MASERU & 10-14 & 96.50 & 135.63 & 67.78 & HT-Direct \\ 
  Lesotho & MASERU & 10-14 & 113.38 & 86.83 & 146.47 & RW2 \\ 
  Lesotho & MASERU & 15-19 & 94.18 & 35.71 & 221.44 & RW2 \\ 
  Lesotho & MOHALE'S HOEK & 80-84 & 126.37 & 190.23 & 81.78 & HT-Direct \\ 
  Lesotho & MOHALE'S HOEK & 80-84 & 132.29 & 94.49 & 181.80 & RW2 \\ 
  Lesotho & MOHALE'S HOEK & 85-89 & 96.17 & 128.87 & 71.09 & HT-Direct \\ 
  Lesotho & MOHALE'S HOEK & 85-89 & 107.06 & 84.85 & 134.33 & RW2 \\ 
  Lesotho & MOHALE'S HOEK & 90-94 & 87.98 & 114.58 & 67.09 & HT-Direct \\ 
  Lesotho & MOHALE'S HOEK & 90-94 & 108.00 & 91.02 & 127.84 & RW2 \\ 
  Lesotho & MOHALE'S HOEK & 95-99 & 120.89 & 157.57 & 91.82 & HT-Direct \\ 
  Lesotho & MOHALE'S HOEK & 95-99 & 122.92 & 104.65 & 144.21 & RW2 \\ 
  Lesotho & MOHALE'S HOEK & 00-04 & 137.37 & 173.17 & 108.00 & HT-Direct \\ 
  Lesotho & MOHALE'S HOEK & 00-04 & 135.22 & 116.74 & 156.12 & RW2 \\ 
  Lesotho & MOHALE'S HOEK & 05-09 & 133.38 & 169.68 & 103.88 & HT-Direct \\ 
  Lesotho & MOHALE'S HOEK & 05-09 & 135.86 & 113.37 & 162.11 & RW2 \\ 
  Lesotho & MOHALE'S HOEK & 10-14 & 89.62 & 134.68 & 58.61 & HT-Direct \\ 
  Lesotho & MOHALE'S HOEK & 10-14 & 110.85 & 83.63 & 145.34 & RW2 \\ 
  Lesotho & MOHALE'S HOEK & 15-19 & 84.01 & 31.38 & 201.11 & RW2 \\ 
  Lesotho & MOKHOTLONG & 80-84 & 104.16 & 180.34 & 57.89 & HT-Direct \\ 
  Lesotho & MOKHOTLONG & 80-84 & 148.78 & 101.63 & 210.85 & RW2 \\ 
  Lesotho & MOKHOTLONG & 85-89 & 87.20 & 124.00 & 60.56 & HT-Direct \\ 
  Lesotho & MOKHOTLONG & 85-89 & 115.33 & 89.37 & 146.62 & RW2 \\ 
  Lesotho & MOKHOTLONG & 90-94 & 123.46 & 154.42 & 97.98 & HT-Direct \\ 
  Lesotho & MOKHOTLONG & 90-94 & 110.26 & 92.72 & 130.62 & RW2 \\ 
  Lesotho & MOKHOTLONG & 95-99 & 95.33 & 128.94 & 69.78 & HT-Direct \\ 
  Lesotho & MOKHOTLONG & 95-99 & 117.63 & 99.48 & 138.82 & RW2 \\ 
  Lesotho & MOKHOTLONG & 00-04 & 108.55 & 134.87 & 86.86 & HT-Direct \\ 
  Lesotho & MOKHOTLONG & 00-04 & 121.69 & 104.87 & 140.83 & RW2 \\ 
  Lesotho & MOKHOTLONG & 05-09 & 103.77 & 137.66 & 77.47 & HT-Direct \\ 
  Lesotho & MOKHOTLONG & 05-09 & 116.21 & 96.26 & 139.59 & RW2 \\ 
  Lesotho & MOKHOTLONG & 10-14 & 96.92 & 136.33 & 68.00 & HT-Direct \\ 
  Lesotho & MOKHOTLONG & 10-14 & 90.82 & 68.76 & 119.13 & RW2 \\ 
  Lesotho & MOKHOTLONG & 15-19 & 65.98 & 24.51 & 162.84 & RW2 \\ 
  Lesotho & QASHA'S NEK & 80-84 & 154.47 & 243.34 & 94.02 & HT-Direct \\ 
  Lesotho & QASHA'S NEK & 80-84 & 156.29 & 109.21 & 220.38 & RW2 \\ 
  Lesotho & QASHA'S NEK & 85-89 & 111.82 & 155.81 & 79.09 & HT-Direct \\ 
  Lesotho & QASHA'S NEK & 85-89 & 118.77 & 92.62 & 150.84 & RW2 \\ 
  Lesotho & QASHA'S NEK & 90-94 & 106.19 & 134.42 & 83.32 & HT-Direct \\ 
  Lesotho & QASHA'S NEK & 90-94 & 111.92 & 93.96 & 132.73 & RW2 \\ 
  Lesotho & QASHA'S NEK & 95-99 & 93.77 & 122.96 & 70.95 & HT-Direct \\ 
  Lesotho & QASHA'S NEK & 95-99 & 119.49 & 100.94 & 140.96 & RW2 \\ 
  Lesotho & QASHA'S NEK & 00-04 & 103.49 & 136.54 & 77.73 & HT-Direct \\ 
  Lesotho & QASHA'S NEK & 00-04 & 124.69 & 105.91 & 146.14 & RW2 \\ 
  Lesotho & QASHA'S NEK & 05-09 & 125.92 & 162.66 & 96.53 & HT-Direct \\ 
  Lesotho & QASHA'S NEK & 05-09 & 120.43 & 98.49 & 146.36 & RW2 \\ 
  Lesotho & QASHA'S NEK & 10-14 & 97.20 & 154.55 & 59.64 & HT-Direct \\ 
  Lesotho & QASHA'S NEK & 10-14 & 94.62 & 69.07 & 128.79 & RW2 \\ 
  Lesotho & QASHA'S NEK & 15-19 & 69.23 & 24.92 & 173.64 & RW2 \\ 
  Lesotho & QUTHING & 80-84 & 122.26 & 180.04 & 81.18 & HT-Direct \\ 
  Lesotho & QUTHING & 80-84 & 150.26 & 105.49 & 209.39 & RW2 \\ 
  Lesotho & QUTHING & 85-89 & 76.67 & 113.37 & 51.17 & HT-Direct \\ 
  Lesotho & QUTHING & 85-89 & 117.51 & 91.51 & 149.57 & RW2 \\ 
  Lesotho & QUTHING & 90-94 & 123.26 & 155.31 & 97.07 & HT-Direct \\ 
  Lesotho & QUTHING & 90-94 & 113.68 & 95.17 & 135.25 & RW2 \\ 
  Lesotho & QUTHING & 95-99 & 110.08 & 148.19 & 80.84 & HT-Direct \\ 
  Lesotho & QUTHING & 95-99 & 122.71 & 103.01 & 145.60 & RW2 \\ 
  Lesotho & QUTHING & 00-04 & 108.16 & 141.09 & 82.19 & HT-Direct \\ 
  Lesotho & QUTHING & 00-04 & 128.08 & 108.45 & 150.82 & RW2 \\ 
  Lesotho & QUTHING & 05-09 & 125.70 & 163.64 & 95.56 & HT-Direct \\ 
  Lesotho & QUTHING & 05-09 & 122.71 & 99.82 & 150.12 & RW2 \\ 
  Lesotho & QUTHING & 10-14 & 77.80 & 129.69 & 45.58 & HT-Direct \\ 
  Lesotho & QUTHING & 10-14 & 95.25 & 68.05 & 131.58 & RW2 \\ 
  Lesotho & QUTHING & 15-19 & 68.33 & 24.02 & 174.77 & RW2 \\ 
  Lesotho & THABA-TSEKA & 80-84 & 134.26 & 218.11 & 79.38 & HT-Direct \\ 
  Lesotho & THABA-TSEKA & 80-84 & 166.46 & 115.93 & 233.67 & RW2 \\ 
  Lesotho & THABA-TSEKA & 85-89 & 95.81 & 135.18 & 67.02 & HT-Direct \\ 
  Lesotho & THABA-TSEKA & 85-89 & 123.94 & 95.59 & 159.51 & RW2 \\ 
  Lesotho & THABA-TSEKA & 90-94 & 105.50 & 142.43 & 77.29 & HT-Direct \\ 
  Lesotho & THABA-TSEKA & 90-94 & 113.84 & 94.30 & 136.96 & RW2 \\ 
  Lesotho & THABA-TSEKA & 95-99 & 108.90 & 140.64 & 83.62 & HT-Direct \\ 
  Lesotho & THABA-TSEKA & 95-99 & 117.34 & 99.73 & 137.79 & RW2 \\ 
  Lesotho & THABA-TSEKA & 00-04 & 121.14 & 148.94 & 97.92 & HT-Direct \\ 
  Lesotho & THABA-TSEKA & 00-04 & 116.76 & 101.10 & 134.66 & RW2 \\ 
  Lesotho & THABA-TSEKA & 05-09 & 94.25 & 121.04 & 72.90 & HT-Direct \\ 
  Lesotho & THABA-TSEKA & 05-09 & 105.85 & 87.15 & 127.73 & RW2 \\ 
  Lesotho & THABA-TSEKA & 10-14 & 55.74 & 105.03 & 28.84 & HT-Direct \\ 
  Lesotho & THABA-TSEKA & 10-14 & 77.37 & 56.11 & 105.10 & RW2 \\ 
  Lesotho & THABA-TSEKA & 15-19 & 52.63 & 18.87 & 132.93 & RW2 \\ 
  Liberia & ALL & 80-84 & 236.32 & 232.47 & 239.96 & IHME \\ 
  Liberia & ALL & 80-84 & 237.19 & 207.24 & 270.04 & RW2 \\ 
  Liberia & ALL & 80-84 & 236.87 & 227.22 & 245.59 & UN \\ 
  Liberia & ALL & 85-89 & 227.05 & 223.85 & 230.33 & IHME \\ 
  Liberia & ALL & 85-89 & 242.19 & 222.41 & 262.82 & RW2 \\ 
  Liberia & ALL & 85-89 & 242.85 & 234.47 & 252.21 & UN \\ 
  Liberia & ALL & 90-94 & 220.60 & 216.62 & 224.48 & IHME \\ 
  Liberia & ALL & 90-94 & 254.77 & 238.92 & 271.47 & RW2 \\ 
  Liberia & ALL & 90-94 & 253.53 & 244.13 & 264.03 & UN \\ 
  Liberia & ALL & 95-99 & 195.34 & 192.46 & 198.27 & IHME \\ 
  Liberia & ALL & 95-99 & 214.35 & 200.55 & 228.66 & RW2 \\ 
  Liberia & ALL & 95-99 & 215.18 & 207.46 & 223.06 & UN \\ 
  Liberia & ALL & 00-04 & 146.95 & 144.51 & 149.52 & IHME \\ 
  Liberia & ALL & 00-04 & 157.99 & 147.43 & 169.26 & RW2 \\ 
  Liberia & ALL & 00-04 & 157.48 & 151.80 & 163.44 & UN \\ 
  Liberia & ALL & 05-09 & 106.28 & 104.22 & 108.45 & IHME \\ 
  Liberia & ALL & 05-09 & 107.67 & 96.48 & 119.83 & RW2 \\ 
  Liberia & ALL & 05-09 & 108.36 & 103.63 & 113.29 & UN \\ 
  Liberia & ALL & 10-14 & 83.09 & 80.24 & 85.99 & IHME \\ 
  Liberia & ALL & 10-14 & 81.53 & 71.56 & 92.67 & RW2 \\ 
  Liberia & ALL & 10-14 & 80.99 & 75.01 & 87.22 & UN \\ 
  Liberia & NORTH CENTRAL & 80-84 & 232.00 & 291.44 & 181.58 & HT-Direct \\ 
  Liberia & NORTH CENTRAL & 80-84 & 230.35 & 191.17 & 275.51 & RW2 \\ 
  Liberia & NORTH CENTRAL & 85-89 & 227.02 & 266.10 & 192.18 & HT-Direct \\ 
  Liberia & NORTH CENTRAL & 85-89 & 238.25 & 211.52 & 266.82 & RW2 \\ 
  Liberia & NORTH CENTRAL & 90-94 & 246.92 & 280.07 & 216.51 & HT-Direct \\ 
  Liberia & NORTH CENTRAL & 90-94 & 252.52 & 231.31 & 274.87 & RW2 \\ 
  Liberia & NORTH CENTRAL & 95-99 & 207.51 & 233.97 & 183.32 & HT-Direct \\ 
  Liberia & NORTH CENTRAL & 95-99 & 205.36 & 188.49 & 223.55 & RW2 \\ 
  Liberia & NORTH CENTRAL & 00-04 & 139.79 & 157.90 & 123.44 & HT-Direct \\ 
  Liberia & NORTH CENTRAL & 00-04 & 144.84 & 132.22 & 158.69 & RW2 \\ 
  Liberia & NORTH CENTRAL & 05-09 & 96.03 & 112.78 & 81.53 & HT-Direct \\ 
  Liberia & NORTH CENTRAL & 05-09 & 101.24 & 89.51 & 114.33 & RW2 \\ 
  Liberia & NORTH CENTRAL & 10-14 & 75.31 & 102.44 & 54.92 & HT-Direct \\ 
  Liberia & NORTH CENTRAL & 10-14 & 70.03 & 57.15 & 85.12 & RW2 \\ 
  Liberia & NORTH CENTRAL & 15-19 & 47.77 & 19.01 & 114.85 & RW2 \\ 
  Liberia & NORTH WESTERN & 80-84 & 213.26 & 300.87 & 145.84 & HT-Direct \\ 
  Liberia & NORTH WESTERN & 80-84 & 262.90 & 206.31 & 324.32 & RW2 \\ 
  Liberia & NORTH WESTERN & 85-89 & 273.61 & 323.02 & 229.21 & HT-Direct \\ 
  Liberia & NORTH WESTERN & 85-89 & 284.82 & 249.31 & 322.32 & RW2 \\ 
  Liberia & NORTH WESTERN & 90-94 & 328.52 & 380.27 & 280.61 & HT-Direct \\ 
  Liberia & NORTH WESTERN & 90-94 & 311.89 & 283.41 & 342.48 & RW2 \\ 
  Liberia & NORTH WESTERN & 95-99 & 271.35 & 308.30 & 237.31 & HT-Direct \\ 
  Liberia & NORTH WESTERN & 95-99 & 262.51 & 239.34 & 287.91 & RW2 \\ 
  Liberia & NORTH WESTERN & 00-04 & 172.95 & 199.87 & 148.99 & HT-Direct \\ 
  Liberia & NORTH WESTERN & 00-04 & 190.17 & 172.02 & 210.08 & RW2 \\ 
  Liberia & NORTH WESTERN & 05-09 & 125.63 & 151.83 & 103.40 & HT-Direct \\ 
  Liberia & NORTH WESTERN & 05-09 & 135.92 & 119.87 & 153.85 & RW2 \\ 
  Liberia & NORTH WESTERN & 10-14 & 110.16 & 134.53 & 89.74 & HT-Direct \\ 
  Liberia & NORTH WESTERN & 10-14 & 96.36 & 82.05 & 112.70 & RW2 \\ 
  Liberia & NORTH WESTERN & 15-19 & 67.29 & 27.42 & 154.38 & RW2 \\ 
  Liberia & SOUTH CENTRAL & 80-84 & 242.85 & 286.82 & 203.70 & HT-Direct \\ 
  Liberia & SOUTH CENTRAL & 80-84 & 245.77 & 211.24 & 284.01 & RW2 \\ 
  Liberia & SOUTH CENTRAL & 85-89 & 240.36 & 272.63 & 210.81 & HT-Direct \\ 
  Liberia & SOUTH CENTRAL & 85-89 & 253.02 & 229.26 & 278.46 & RW2 \\ 
  Liberia & SOUTH CENTRAL & 90-94 & 275.74 & 305.83 & 247.56 & HT-Direct \\ 
  Liberia & SOUTH CENTRAL & 90-94 & 267.05 & 247.24 & 287.64 & RW2 \\ 
  Liberia & SOUTH CENTRAL & 95-99 & 202.88 & 225.24 & 182.22 & HT-Direct \\ 
  Liberia & SOUTH CENTRAL & 95-99 & 217.32 & 201.04 & 234.54 & RW2 \\ 
  Liberia & SOUTH CENTRAL & 00-04 & 156.36 & 173.81 & 140.37 & HT-Direct \\ 
  Liberia & SOUTH CENTRAL & 00-04 & 155.26 & 142.59 & 169.02 & RW2 \\ 
  Liberia & SOUTH CENTRAL & 05-09 & 93.08 & 113.93 & 75.72 & HT-Direct \\ 
  Liberia & SOUTH CENTRAL & 05-09 & 110.59 & 97.25 & 125.51 & RW2 \\ 
  Liberia & SOUTH CENTRAL & 10-14 & 93.64 & 121.50 & 71.64 & HT-Direct \\ 
  Liberia & SOUTH CENTRAL & 10-14 & 78.53 & 65.24 & 94.38 & RW2 \\ 
  Liberia & SOUTH CENTRAL & 15-19 & 55.07 & 22.33 & 128.93 & RW2 \\ 
  Liberia & SOUTH EASTERN A & 80-84 & 184.28 & 270.36 & 121.06 & HT-Direct \\ 
  Liberia & SOUTH EASTERN A & 80-84 & 206.47 & 160.95 & 259.81 & RW2 \\ 
  Liberia & SOUTH EASTERN A & 85-89 & 204.74 & 248.18 & 167.21 & HT-Direct \\ 
  Liberia & SOUTH EASTERN A & 85-89 & 220.12 & 190.23 & 253.28 & RW2 \\ 
  Liberia & SOUTH EASTERN A & 90-94 & 258.51 & 305.74 & 216.30 & HT-Direct \\ 
  Liberia & SOUTH EASTERN A & 90-94 & 239.23 & 214.50 & 265.94 & RW2 \\ 
  Liberia & SOUTH EASTERN A & 95-99 & 185.08 & 221.54 & 153.43 & HT-Direct \\ 
  Liberia & SOUTH EASTERN A & 95-99 & 197.69 & 177.75 & 219.61 & RW2 \\ 
  Liberia & SOUTH EASTERN A & 00-04 & 139.61 & 163.62 & 118.62 & HT-Direct \\ 
  Liberia & SOUTH EASTERN A & 00-04 & 144.60 & 129.40 & 161.49 & RW2 \\ 
  Liberia & SOUTH EASTERN A & 05-09 & 111.05 & 135.66 & 90.44 & HT-Direct \\ 
  Liberia & SOUTH EASTERN A & 05-09 & 107.00 & 93.03 & 122.84 & RW2 \\ 
  Liberia & SOUTH EASTERN A & 10-14 & 78.28 & 105.72 & 57.51 & HT-Direct \\ 
  Liberia & SOUTH EASTERN A & 10-14 & 78.64 & 63.86 & 96.05 & RW2 \\ 
  Liberia & SOUTH EASTERN A & 15-19 & 56.95 & 22.77 & 134.56 & RW2 \\ 
  Liberia & SOUTH EASTERN B & 80-84 & 164.40 & 225.27 & 117.49 & HT-Direct \\ 
  Liberia & SOUTH EASTERN B & 80-84 & 174.08 & 135.86 & 222.32 & RW2 \\ 
  Liberia & SOUTH EASTERN B & 85-89 & 191.63 & 227.89 & 159.94 & HT-Direct \\ 
  Liberia & SOUTH EASTERN B & 85-89 & 194.06 & 169.35 & 221.75 & RW2 \\ 
  Liberia & SOUTH EASTERN B & 90-94 & 233.96 & 263.50 & 206.80 & HT-Direct \\ 
  Liberia & SOUTH EASTERN B & 90-94 & 219.46 & 199.96 & 240.57 & RW2 \\ 
  Liberia & SOUTH EASTERN B & 95-99 & 171.96 & 199.60 & 147.45 & HT-Direct \\ 
  Liberia & SOUTH EASTERN B & 95-99 & 190.45 & 170.01 & 211.17 & RW2 \\ 
  Liberia & SOUTH EASTERN B & 00-04 & 136.74 & 165.64 & 112.21 & HT-Direct \\ 
  Liberia & SOUTH EASTERN B & 00-04 & 154.88 & 136.91 & 173.20 & RW2 \\ 
  Liberia & SOUTH EASTERN B & 05-09 & 129.05 & 142.09 & 117.05 & HT-Direct \\ 
  Liberia & SOUTH EASTERN B & 05-09 & 136.46 & 124.59 & 149.16 & RW2 \\ 
  Liberia & SOUTH EASTERN B & 10-14 & 149.05 & 185.19 & 118.93 & HT-Direct \\ 
  Liberia & SOUTH EASTERN B & 10-14 & 124.34 & 105.06 & 147.78 & RW2 \\ 
  Liberia & SOUTH EASTERN B & 15-19 & 113.91 & 46.90 & 251.44 & RW2 \\ 
  Madagascar & ALL & 80-84 & 165.73 & 161.52 & 170.18 & IHME \\ 
  Madagascar & ALL & 80-84 & 180.00 & 170.61 & 189.79 & RW2 \\ 
  Madagascar & ALL & 80-84 & 180.07 & 174.50 & 186.07 & UN \\ 
  Madagascar & ALL & 85-89 & 161.67 & 157.73 & 165.74 & IHME \\ 
  Madagascar & ALL & 85-89 & 175.66 & 167.31 & 184.31 & RW2 \\ 
  Madagascar & ALL & 85-89 & 175.51 & 170.54 & 181.16 & UN \\ 
  Madagascar & ALL & 90-94 & 139.44 & 135.98 & 142.88 & IHME \\ 
  Madagascar & ALL & 90-94 & 151.48 & 143.57 & 159.73 & RW2 \\ 
  Madagascar & ALL & 90-94 & 151.62 & 146.90 & 156.19 & UN \\ 
  Madagascar & ALL & 95-99 & 119.85 & 116.48 & 123.52 & IHME \\ 
  Madagascar & ALL & 95-99 & 127.32 & 119.66 & 135.35 & RW2 \\ 
  Madagascar & ALL & 95-99 & 127.22 & 123.04 & 131.47 & UN \\ 
  Madagascar & ALL & 00-04 & 98.55 & 95.14 & 102.11 & IHME \\ 
  Madagascar & ALL & 00-04 & 97.00 & 90.08 & 104.43 & RW2 \\ 
  Madagascar & ALL & 00-04 & 97.07 & 92.82 & 101.48 & UN \\ 
  Madagascar & ALL & 05-09 & 78.84 & 75.52 & 82.79 & IHME \\ 
  Madagascar & ALL & 05-09 & 71.38 & 64.15 & 79.35 & RW2 \\ 
  Madagascar & ALL & 05-09 & 71.36 & 66.41 & 76.61 & UN \\ 
  Madagascar & ALL & 10-14 & 71.77 & 66.33 & 78.09 & IHME \\ 
  Madagascar & ALL & 10-14 & 51.97 & 18.99 & 133.66 & RW2 \\ 
  Madagascar & ALL & 10-14 & 55.71 & 48.63 & 64.70 & UN \\ 
  Madagascar & ANTANANARIVO & 80-84 & 141.96 & 158.62 & 126.78 & HT-Direct \\ 
  Madagascar & ANTANANARIVO & 80-84 & 140.62 & 126.67 & 155.59 & RW2 \\ 
  Madagascar & ANTANANARIVO & 85-89 & 156.52 & 174.76 & 139.86 & HT-Direct \\ 
  Madagascar & ANTANANARIVO & 85-89 & 143.21 & 130.49 & 157.71 & RW2 \\ 
  Madagascar & ANTANANARIVO & 90-94 & 113.62 & 129.92 & 99.14 & HT-Direct \\ 
  Madagascar & ANTANANARIVO & 90-94 & 120.51 & 108.82 & 133.49 & RW2 \\ 
  Madagascar & ANTANANARIVO & 95-99 & 88.48 & 102.74 & 76.03 & HT-Direct \\ 
  Madagascar & ANTANANARIVO & 95-99 & 92.06 & 81.19 & 103.57 & RW2 \\ 
  Madagascar & ANTANANARIVO & 00-04 & 61.07 & 72.39 & 51.42 & HT-Direct \\ 
  Madagascar & ANTANANARIVO & 00-04 & 74.98 & 64.97 & 85.89 & RW2 \\ 
  Madagascar & ANTANANARIVO & 05-09 & 65.24 & 81.38 & 52.13 & HT-Direct \\ 
  Madagascar & ANTANANARIVO & 05-09 & 59.98 & 48.16 & 75.34 & RW2 \\ 
  Madagascar & ANTANANARIVO & 10-14 & 47.35 & 17.26 & 125.17 & RW2 \\ 
  Madagascar & ANTANANARIVO & 15-19 & 37.36 & 3.13 & 320.63 & RW2 \\ 
  Madagascar & ANTSIRANANA & 80-84 & 151.14 & 176.79 & 128.63 & HT-Direct \\ 
  Madagascar & ANTSIRANANA & 80-84 & 142.51 & 123.14 & 164.36 & RW2 \\ 
  Madagascar & ANTSIRANANA & 85-89 & 134.70 & 156.36 & 115.63 & HT-Direct \\ 
  Madagascar & ANTSIRANANA & 85-89 & 137.92 & 122.33 & 154.87 & RW2 \\ 
  Madagascar & ANTSIRANANA & 90-94 & 134.46 & 158.92 & 113.26 & HT-Direct \\ 
  Madagascar & ANTSIRANANA & 90-94 & 129.46 & 113.44 & 147.54 & RW2 \\ 
  Madagascar & ANTSIRANANA & 95-99 & 95.35 & 115.91 & 78.12 & HT-Direct \\ 
  Madagascar & ANTSIRANANA & 95-99 & 108.69 & 93.26 & 126.17 & RW2 \\ 
  Madagascar & ANTSIRANANA & 00-04 & 98.12 & 126.83 & 75.35 & HT-Direct \\ 
  Madagascar & ANTSIRANANA & 00-04 & 91.89 & 74.81 & 112.86 & RW2 \\ 
  Madagascar & ANTSIRANANA & 05-09 & 51.31 & 84.34 & 30.79 & HT-Direct \\ 
  Madagascar & ANTSIRANANA & 05-09 & 64.33 & 42.73 & 95.70 & RW2 \\ 
  Madagascar & ANTSIRANANA & 10-14 & 41.33 & 12.09 & 129.80 & RW2 \\ 
  Madagascar & ANTSIRANANA & 15-19 & 26.11 & 1.69 & 281.80 & RW2 \\ 
  Madagascar & FIANARANTSOA & 80-84 & 250.64 & 278.62 & 224.60 & HT-Direct \\ 
  Madagascar & FIANARANTSOA & 80-84 & 234.90 & 212.39 & 259.50 & RW2 \\ 
  Madagascar & FIANARANTSOA & 85-89 & 214.89 & 236.24 & 194.97 & HT-Direct \\ 
  Madagascar & FIANARANTSOA & 85-89 & 215.22 & 198.66 & 232.31 & RW2 \\ 
  Madagascar & FIANARANTSOA & 90-94 & 184.59 & 202.88 & 167.59 & HT-Direct \\ 
  Madagascar & FIANARANTSOA & 90-94 & 194.09 & 178.69 & 210.00 & RW2 \\ 
  Madagascar & FIANARANTSOA & 95-99 & 168.89 & 188.07 & 151.31 & HT-Direct \\ 
  Madagascar & FIANARANTSOA & 95-99 & 172.13 & 157.34 & 188.49 & RW2 \\ 
  Madagascar & FIANARANTSOA & 00-04 & 139.18 & 155.72 & 124.14 & HT-Direct \\ 
  Madagascar & FIANARANTSOA & 00-04 & 144.43 & 130.68 & 159.77 & RW2 \\ 
  Madagascar & FIANARANTSOA & 05-09 & 89.04 & 102.63 & 77.09 & HT-Direct \\ 
  Madagascar & FIANARANTSOA & 05-09 & 97.96 & 84.68 & 112.99 & RW2 \\ 
  Madagascar & FIANARANTSOA & 10-14 & 60.10 & 22.99 & 144.54 & RW2 \\ 
  Madagascar & FIANARANTSOA & 15-19 & 35.53 & 3.08 & 292.81 & RW2 \\ 
  Madagascar & MAHAJANGA & 80-84 & 250.23 & 278.59 & 223.87 & HT-Direct \\ 
  Madagascar & MAHAJANGA & 80-84 & 233.52 & 210.75 & 258.38 & RW2 \\ 
  Madagascar & MAHAJANGA & 85-89 & 188.21 & 209.77 & 168.40 & HT-Direct \\ 
  Madagascar & MAHAJANGA & 85-89 & 189.91 & 173.92 & 206.31 & RW2 \\ 
  Madagascar & MAHAJANGA & 90-94 & 146.44 & 162.32 & 131.86 & HT-Direct \\ 
  Madagascar & MAHAJANGA & 90-94 & 152.58 & 139.45 & 166.50 & RW2 \\ 
  Madagascar & MAHAJANGA & 95-99 & 112.95 & 131.23 & 96.93 & HT-Direct \\ 
  Madagascar & MAHAJANGA & 95-99 & 121.72 & 108.61 & 136.06 & RW2 \\ 
  Madagascar & MAHAJANGA & 00-04 & 106.10 & 126.68 & 88.53 & HT-Direct \\ 
  Madagascar & MAHAJANGA & 00-04 & 99.07 & 86.26 & 114.32 & RW2 \\ 
  Madagascar & MAHAJANGA & 05-09 & 57.51 & 71.56 & 46.08 & HT-Direct \\ 
  Madagascar & MAHAJANGA & 05-09 & 65.98 & 53.45 & 81.19 & RW2 \\ 
  Madagascar & MAHAJANGA & 10-14 & 39.96 & 14.63 & 103.01 & RW2 \\ 
  Madagascar & MAHAJANGA & 15-19 & 24.02 & 2.02 & 216.61 & RW2 \\ 
  Madagascar & TOAMASINA & 80-84 & 176.87 & 203.15 & 153.35 & HT-Direct \\ 
  Madagascar & TOAMASINA & 80-84 & 173.46 & 152.35 & 196.62 & RW2 \\ 
  Madagascar & TOAMASINA & 85-89 & 185.37 & 206.17 & 166.22 & HT-Direct \\ 
  Madagascar & TOAMASINA & 85-89 & 184.68 & 169.63 & 201.04 & RW2 \\ 
  Madagascar & TOAMASINA & 90-94 & 175.92 & 196.20 & 157.32 & HT-Direct \\ 
  Madagascar & TOAMASINA & 90-94 & 170.81 & 155.19 & 187.99 & RW2 \\ 
  Madagascar & TOAMASINA & 95-99 & 121.03 & 137.76 & 106.08 & HT-Direct \\ 
  Madagascar & TOAMASINA & 95-99 & 126.56 & 113.75 & 140.57 & RW2 \\ 
  Madagascar & TOAMASINA & 00-04 & 72.86 & 89.03 & 59.44 & HT-Direct \\ 
  Madagascar & TOAMASINA & 00-04 & 85.90 & 73.16 & 100.29 & RW2 \\ 
  Madagascar & TOAMASINA & 05-09 & 48.62 & 67.71 & 34.71 & HT-Direct \\ 
  Madagascar & TOAMASINA & 05-09 & 50.33 & 37.58 & 66.58 & RW2 \\ 
  Madagascar & TOAMASINA & 10-14 & 27.76 & 9.50 & 76.87 & RW2 \\ 
  Madagascar & TOAMASINA & 15-19 & 15.22 & 1.24 & 153.91 & RW2 \\ 
  Madagascar & TOLIARY & 80-84 & 181.89 & 212.46 & 154.86 & HT-Direct \\ 
  Madagascar & TOLIARY & 80-84 & 175.79 & 152.72 & 201.63 & RW2 \\ 
  Madagascar & TOLIARY & 85-89 & 180.74 & 201.71 & 161.52 & HT-Direct \\ 
  Madagascar & TOLIARY & 85-89 & 176.15 & 160.94 & 192.55 & RW2 \\ 
  Madagascar & TOLIARY & 90-94 & 153.11 & 172.19 & 135.81 & HT-Direct \\ 
  Madagascar & TOLIARY & 90-94 & 162.98 & 148.07 & 178.96 & RW2 \\ 
  Madagascar & TOLIARY & 95-99 & 142.51 & 162.21 & 124.85 & HT-Direct \\ 
  Madagascar & TOLIARY & 95-99 & 142.32 & 128.18 & 158.18 & RW2 \\ 
  Madagascar & TOLIARY & 00-04 & 107.44 & 121.35 & 94.96 & HT-Direct \\ 
  Madagascar & TOLIARY & 00-04 & 113.11 & 101.43 & 126.13 & RW2 \\ 
  Madagascar & TOLIARY & 05-09 & 61.91 & 75.59 & 50.57 & HT-Direct \\ 
  Madagascar & TOLIARY & 05-09 & 70.46 & 57.76 & 85.25 & RW2 \\ 
  Madagascar & TOLIARY & 10-14 & 39.80 & 14.42 & 102.53 & RW2 \\ 
  Madagascar & TOLIARY & 15-19 & 22.26 & 1.82 & 204.12 & RW2 \\ 
  Malawi & ALL & 80-84 & 249.54 & 244.17 & 254.87 & IHME \\ 
  Malawi & ALL & 80-84 & 247.37 & 236.82 & 258.23 & RW2 \\ 
  Malawi & ALL & 80-84 & 247.38 & 240.91 & 254.48 & UN \\ 
  Malawi & ALL & 85-89 & 235.41 & 230.64 & 240.45 & IHME \\ 
  Malawi & ALL & 85-89 & 251.17 & 242.98 & 259.51 & RW2 \\ 
  Malawi & ALL & 85-89 & 251.19 & 244.39 & 258.13 & UN \\ 
  Malawi & ALL & 90-94 & 212.05 & 207.95 & 216.48 & IHME \\ 
  Malawi & ALL & 90-94 & 227.16 & 219.79 & 234.70 & RW2 \\ 
  Malawi & ALL & 90-94 & 227.09 & 221.44 & 233.44 & UN \\ 
  Malawi & ALL & 95-99 & 184.45 & 180.19 & 188.56 & IHME \\ 
  Malawi & ALL & 95-99 & 197.02 & 190.60 & 203.56 & RW2 \\ 
  Malawi & ALL & 95-99 & 197.10 & 192.43 & 202.12 & UN \\ 
  Malawi & ALL & 00-04 & 147.19 & 143.21 & 151.06 & IHME \\ 
  Malawi & ALL & 00-04 & 148.13 & 142.64 & 153.83 & RW2 \\ 
  Malawi & ALL & 00-04 & 148.18 & 143.93 & 152.99 & UN \\ 
  Malawi & ALL & 05-09 & 112.49 & 109.06 & 115.99 & IHME \\ 
  Malawi & ALL & 05-09 & 105.34 & 100.66 & 110.22 & RW2 \\ 
  Malawi & ALL & 05-09 & 105.24 & 101.01 & 109.83 & UN \\ 
  Malawi & ALL & 10-14 & 91.52 & 87.03 & 96.06 & IHME \\ 
  Malawi & ALL & 10-14 & 79.49 & 72.83 & 86.66 & RW2 \\ 
  Malawi & ALL & 10-14 & 79.65 & 72.89 & 87.10 & UN \\ 
  Malawi & CENTRAL REGION & 80-84 & 271.78 & 290.18 & 254.14 & HT-Direct \\ 
  Malawi & CENTRAL REGION & 80-84 & 278.09 & 261.82 & 295.12 & RW2 \\ 
  Malawi & CENTRAL REGION & 85-89 & 268.83 & 281.51 & 256.51 & HT-Direct \\ 
  Malawi & CENTRAL REGION & 85-89 & 273.04 & 262.22 & 284.25 & RW2 \\ 
  Malawi & CENTRAL REGION & 90-94 & 223.18 & 234.62 & 212.15 & HT-Direct \\ 
  Malawi & CENTRAL REGION & 90-94 & 239.50 & 229.67 & 249.57 & RW2 \\ 
  Malawi & CENTRAL REGION & 95-99 & 191.53 & 201.24 & 182.17 & HT-Direct \\ 
  Malawi & CENTRAL REGION & 95-99 & 202.53 & 193.94 & 211.21 & RW2 \\ 
  Malawi & CENTRAL REGION & 00-04 & 149.62 & 158.84 & 140.84 & HT-Direct \\ 
  Malawi & CENTRAL REGION & 00-04 & 152.67 & 145.41 & 160.30 & RW2 \\ 
  Malawi & CENTRAL REGION & 05-09 & 104.93 & 112.68 & 97.66 & HT-Direct \\ 
  Malawi & CENTRAL REGION & 05-09 & 110.62 & 104.09 & 117.53 & RW2 \\ 
  Malawi & CENTRAL REGION & 10-14 & 76.30 & 85.29 & 68.19 & HT-Direct \\ 
  Malawi & CENTRAL REGION & 10-14 & 84.62 & 75.72 & 94.60 & RW2 \\ 
  Malawi & CENTRAL REGION & 15-19 & 65.68 & 30.19 & 137.92 & RW2 \\ 
  Malawi & NORTHERN REGION & 80-84 & 193.59 & 216.41 & 172.65 & HT-Direct \\ 
  Malawi & NORTHERN REGION & 80-84 & 199.91 & 180.89 & 220.10 & RW2 \\ 
  Malawi & NORTHERN REGION & 85-89 & 196.25 & 212.94 & 180.57 & HT-Direct \\ 
  Malawi & NORTHERN REGION & 85-89 & 200.85 & 188.41 & 214.20 & RW2 \\ 
  Malawi & NORTHERN REGION & 90-94 & 171.04 & 186.48 & 156.63 & HT-Direct \\ 
  Malawi & NORTHERN REGION & 90-94 & 177.60 & 166.31 & 189.34 & RW2 \\ 
  Malawi & NORTHERN REGION & 95-99 & 135.85 & 149.65 & 123.13 & HT-Direct \\ 
  Malawi & NORTHERN REGION & 95-99 & 150.60 & 140.24 & 161.30 & RW2 \\ 
  Malawi & NORTHERN REGION & 00-04 & 116.03 & 129.02 & 104.20 & HT-Direct \\ 
  Malawi & NORTHERN REGION & 00-04 & 114.50 & 105.97 & 123.97 & RW2 \\ 
  Malawi & NORTHERN REGION & 05-09 & 79.94 & 92.20 & 69.18 & HT-Direct \\ 
  Malawi & NORTHERN REGION & 05-09 & 82.78 & 74.87 & 91.48 & RW2 \\ 
  Malawi & NORTHERN REGION & 10-14 & 52.18 & 61.64 & 44.11 & HT-Direct \\ 
  Malawi & NORTHERN REGION & 10-14 & 62.47 & 53.01 & 73.25 & RW2 \\ 
  Malawi & NORTHERN REGION & 15-19 & 47.77 & 21.03 & 104.96 & RW2 \\ 
  Malawi & SOUTHERN REGION & 80-84 & 229.31 & 243.00 & 216.17 & HT-Direct \\ 
  Malawi & SOUTHERN REGION & 80-84 & 232.05 & 219.40 & 245.33 & RW2 \\ 
  Malawi & SOUTHERN REGION & 85-89 & 229.98 & 241.28 & 219.05 & HT-Direct \\ 
  Malawi & SOUTHERN REGION & 85-89 & 243.97 & 233.64 & 254.33 & RW2 \\ 
  Malawi & SOUTHERN REGION & 90-94 & 223.45 & 233.16 & 214.04 & HT-Direct \\ 
  Malawi & SOUTHERN REGION & 90-94 & 230.84 & 222.01 & 240.01 & RW2 \\ 
  Malawi & SOUTHERN REGION & 95-99 & 193.89 & 201.82 & 186.20 & HT-Direct \\ 
  Malawi & SOUTHERN REGION & 95-99 & 201.87 & 194.41 & 209.67 & RW2 \\ 
  Malawi & SOUTHERN REGION & 00-04 & 142.00 & 149.15 & 135.14 & HT-Direct \\ 
  Malawi & SOUTHERN REGION & 00-04 & 149.94 & 143.53 & 156.61 & RW2 \\ 
  Malawi & SOUTHERN REGION & 05-09 & 103.86 & 110.13 & 97.91 & HT-Direct \\ 
  Malawi & SOUTHERN REGION & 05-09 & 106.17 & 100.59 & 112.00 & RW2 \\ 
  Malawi & SOUTHERN REGION & 10-14 & 65.79 & 73.50 & 58.83 & HT-Direct \\ 
  Malawi & SOUTHERN REGION & 10-14 & 78.05 & 69.58 & 87.34 & RW2 \\ 
  Malawi & SOUTHERN REGION & 15-19 & 57.78 & 26.21 & 123.23 & RW2 \\ 
  Mali & ALL & 80-84 & 264.91 & 261.52 & 267.92 & IHME \\ 
  Mali & ALL & 80-84 & 308.58 & 297.83 & 319.53 & RW2 \\ 
  Mali & ALL & 80-84 & 308.66 & 299.76 & 318.13 & UN \\ 
  Mali & ALL & 85-89 & 240.25 & 237.66 & 243.28 & IHME \\ 
  Mali & ALL & 85-89 & 271.57 & 263.33 & 279.94 & RW2 \\ 
  Mali & ALL & 85-89 & 271.40 & 263.51 & 279.20 & UN \\ 
  Mali & ALL & 90-94 & 222.21 & 219.55 & 225.14 & IHME \\ 
  Mali & ALL & 90-94 & 247.13 & 240.27 & 254.12 & RW2 \\ 
  Mali & ALL & 90-94 & 247.33 & 239.50 & 254.41 & UN \\ 
  Mali & ALL & 95-99 & 205.62 & 202.49 & 208.44 & IHME \\ 
  Mali & ALL & 95-99 & 234.72 & 226.41 & 243.18 & RW2 \\ 
  Mali & ALL & 95-99 & 234.26 & 227.42 & 242.23 & UN \\ 
  Mali & ALL & 00-04 & 182.24 & 179.26 & 185.52 & IHME \\ 
  Mali & ALL & 00-04 & 200.55 & 192.30 & 209.11 & RW2 \\ 
  Mali & ALL & 00-04 & 200.96 & 193.90 & 208.55 & UN \\ 
  Mali & ALL & 05-09 & 158.74 & 155.00 & 162.52 & IHME \\ 
  Mali & ALL & 05-09 & 155.74 & 142.41 & 170.07 & RW2 \\ 
  Mali & ALL & 05-09 & 155.29 & 143.82 & 166.77 & UN \\ 
  Mali & ALL & 10-14 & 140.35 & 134.80 & 145.97 & IHME \\ 
  Mali & ALL & 10-14 & 116.87 & 45.58 & 265.70 & RW2 \\ 
  Mali & ALL & 10-14 & 128.09 & 110.80 & 148.02 & UN \\ 
  Mali & BAMAKO & 80-84 & 173.88 & 192.18 & 156.97 & HT-Direct \\ 
  Mali & BAMAKO & 80-84 & 181.48 & 165.17 & 198.80 & RW2 \\ 
  Mali & BAMAKO & 85-89 & 148.58 & 166.26 & 132.47 & HT-Direct \\ 
  Mali & BAMAKO & 85-89 & 154.74 & 143.71 & 166.51 & RW2 \\ 
  Mali & BAMAKO & 90-94 & 144.98 & 161.06 & 130.25 & HT-Direct \\ 
  Mali & BAMAKO & 90-94 & 138.34 & 128.64 & 148.89 & RW2 \\ 
  Mali & BAMAKO & 95-99 & 130.87 & 149.49 & 114.26 & HT-Direct \\ 
  Mali & BAMAKO & 95-99 & 130.56 & 120.49 & 141.40 & RW2 \\ 
  Mali & BAMAKO & 00-04 & 109.22 & 124.45 & 95.66 & HT-Direct \\ 
  Mali & BAMAKO & 00-04 & 106.92 & 96.40 & 118.53 & RW2 \\ 
  Mali & BAMAKO & 05-09 & 77.40 & 113.98 & 51.88 & HT-Direct \\ 
  Mali & BAMAKO & 05-09 & 80.49 & 64.93 & 98.49 & RW2 \\ 
  Mali & BAMAKO & 10-14 & 59.45 & 23.40 & 141.68 & RW2 \\ 
  Mali & BAMAKO & 15-19 & 43.63 & 4.43 & 320.46 & RW2 \\ 
  Mali & KAYES, KOULIKORO & 80-84 & 294.57 & 311.15 & 278.52 & HT-Direct \\ 
  Mali & KAYES, KOULIKORO & 80-84 & 302.71 & 287.34 & 318.98 & RW2 \\ 
  Mali & KAYES, KOULIKORO & 85-89 & 257.13 & 271.77 & 243.01 & HT-Direct \\ 
  Mali & KAYES, KOULIKORO & 85-89 & 261.92 & 250.73 & 273.23 & RW2 \\ 
  Mali & KAYES, KOULIKORO & 90-94 & 232.98 & 244.69 & 221.66 & HT-Direct \\ 
  Mali & KAYES, KOULIKORO & 90-94 & 236.60 & 227.30 & 246.00 & RW2 \\ 
  Mali & KAYES, KOULIKORO & 95-99 & 238.16 & 250.52 & 226.22 & HT-Direct \\ 
  Mali & KAYES, KOULIKORO & 95-99 & 227.74 & 218.69 & 237.07 & RW2 \\ 
  Mali & KAYES, KOULIKORO & 00-04 & 197.55 & 212.32 & 183.58 & HT-Direct \\ 
  Mali & KAYES, KOULIKORO & 00-04 & 194.15 & 183.07 & 205.95 & RW2 \\ 
  Mali & KAYES, KOULIKORO & 05-09 & 165.21 & 201.03 & 134.70 & HT-Direct \\ 
  Mali & KAYES, KOULIKORO & 05-09 & 153.29 & 134.58 & 174.20 & RW2 \\ 
  Mali & KAYES, KOULIKORO & 10-14 & 118.41 & 50.67 & 252.50 & RW2 \\ 
  Mali & KAYES, KOULIKORO & 15-19 & 90.55 & 10.11 & 497.98 & RW2 \\ 
  Mali & MOPTI, TOMBOUCTOU, GAO, KIDAL & 80-84 & 383.04 & 406.82 & 359.81 & HT-Direct \\ 
  Mali & MOPTI, TOMBOUCTOU, GAO, KIDAL & 80-84 & 406.69 & 383.81 & 429.34 & RW2 \\ 
  Mali & MOPTI, TOMBOUCTOU, GAO, KIDAL & 85-89 & 351.97 & 372.29 & 332.18 & HT-Direct \\ 
  Mali & MOPTI, TOMBOUCTOU, GAO, KIDAL & 85-89 & 348.51 & 333.65 & 363.77 & RW2 \\ 
  Mali & MOPTI, TOMBOUCTOU, GAO, KIDAL & 90-94 & 296.41 & 311.90 & 281.37 & HT-Direct \\ 
  Mali & MOPTI, TOMBOUCTOU, GAO, KIDAL & 90-94 & 298.62 & 286.69 & 311.20 & RW2 \\ 
  Mali & MOPTI, TOMBOUCTOU, GAO, KIDAL & 95-99 & 279.53 & 297.16 & 262.56 & HT-Direct \\ 
  Mali & MOPTI, TOMBOUCTOU, GAO, KIDAL & 95-99 & 262.82 & 250.97 & 274.99 & RW2 \\ 
  Mali & MOPTI, TOMBOUCTOU, GAO, KIDAL & 00-04 & 191.87 & 207.24 & 177.38 & HT-Direct \\ 
  Mali & MOPTI, TOMBOUCTOU, GAO, KIDAL & 00-04 & 201.70 & 189.43 & 214.48 & RW2 \\ 
  Mali & MOPTI, TOMBOUCTOU, GAO, KIDAL & 05-09 & 173.69 & 207.09 & 144.70 & HT-Direct \\ 
  Mali & MOPTI, TOMBOUCTOU, GAO, KIDAL & 05-09 & 145.11 & 127.56 & 165.12 & RW2 \\ 
  Mali & MOPTI, TOMBOUCTOU, GAO, KIDAL & 10-14 & 101.98 & 43.37 & 222.30 & RW2 \\ 
  Mali & MOPTI, TOMBOUCTOU, GAO, KIDAL & 15-19 & 70.54 & 7.48 & 431.61 & RW2 \\ 
  Mali & SIKASSO, SEGOU & 80-84 & 279.83 & 295.21 & 264.96 & HT-Direct \\ 
  Mali & SIKASSO, SEGOU & 80-84 & 290.18 & 275.81 & 304.97 & RW2 \\ 
  Mali & SIKASSO, SEGOU & 85-89 & 264.39 & 277.07 & 252.09 & HT-Direct \\ 
  Mali & SIKASSO, SEGOU & 85-89 & 266.79 & 256.76 & 277.15 & RW2 \\ 
  Mali & SIKASSO, SEGOU & 90-94 & 249.90 & 261.24 & 238.90 & HT-Direct \\ 
  Mali & SIKASSO, SEGOU & 90-94 & 253.64 & 244.43 & 262.94 & RW2 \\ 
  Mali & SIKASSO, SEGOU & 95-99 & 265.37 & 283.18 & 248.29 & HT-Direct \\ 
  Mali & SIKASSO, SEGOU & 95-99 & 254.85 & 243.68 & 266.42 & RW2 \\ 
  Mali & SIKASSO, SEGOU & 00-04 & 234.67 & 251.49 & 218.65 & HT-Direct \\ 
  Mali & SIKASSO, SEGOU & 00-04 & 227.15 & 214.84 & 240.15 & RW2 \\ 
  Mali & SIKASSO, SEGOU & 05-09 & 197.36 & 224.43 & 172.82 & HT-Direct \\ 
  Mali & SIKASSO, SEGOU & 05-09 & 189.10 & 171.01 & 208.57 & RW2 \\ 
  Mali & SIKASSO, SEGOU & 10-14 & 155.15 & 69.90 & 313.00 & RW2 \\ 
  Mali & SIKASSO, SEGOU & 15-19 & 125.41 & 14.71 & 580.00 & RW2 \\ 
  Morocco & ALL & 80-84 & 110.92 & 109.27 & 112.75 & IHME \\ 
  Morocco & ALL & 80-84 & 120.67 & 113.75 & 127.96 & RW2 \\ 
  Morocco & ALL & 80-84 & 120.68 & 117.85 & 123.93 & UN \\ 
  Morocco & ALL & 85-89 & 83.10 & 81.61 & 84.46 & IHME \\ 
  Morocco & ALL & 85-89 & 93.08 & 86.59 & 99.98 & RW2 \\ 
  Morocco & ALL & 85-89 & 93.06 & 90.49 & 95.68 & UN \\ 
  Morocco & ALL & 90-94 & 63.58 & 62.39 & 64.93 & IHME \\ 
  Morocco & ALL & 90-94 & 73.04 & 66.93 & 79.66 & RW2 \\ 
  Morocco & ALL & 90-94 & 73.15 & 70.89 & 75.36 & UN \\ 
  Morocco & ALL & 95-99 & 50.42 & 49.10 & 51.85 & IHME \\ 
  Morocco & ALL & 95-99 & 57.95 & 51.95 & 64.54 & RW2 \\ 
  Morocco & ALL & 95-99 & 57.73 & 55.73 & 59.79 & UN \\ 
  Morocco & ALL & 00-04 & 40.28 & 38.49 & 42.03 & IHME \\ 
  Morocco & ALL & 00-04 & 45.74 & 39.40 & 53.09 & RW2 \\ 
  Morocco & ALL & 00-04 & 45.88 & 44.01 & 47.82 & UN \\ 
  Morocco & ALL & 05-09 & 32.25 & 30.20 & 34.41 & IHME \\ 
  Morocco & ALL & 05-09 & 35.82 & 10.21 & 117.54 & RW2 \\ 
  Morocco & ALL & 05-09 & 37.16 & 35.23 & 39.25 & UN \\ 
  Morocco & ALL & 10-14 & 25.87 & 23.78 & 28.26 & IHME \\ 
  Morocco & ALL & 10-14 & 27.81 & 1.22 & 388.29 & RW2 \\ 
  Morocco & ALL & 10-14 & 30.51 & 27.87 & 33.48 & UN \\ 
  Morocco & CENTRE & 80-84 & 85.05 & 95.98 & 75.27 & HT-Direct \\ 
  Morocco & CENTRE & 80-84 & 84.67 & 75.28 & 95.27 & RW2 \\ 
  Morocco & CENTRE & 85-89 & 56.63 & 65.84 & 48.64 & HT-Direct \\ 
  Morocco & CENTRE & 85-89 & 65.41 & 58.52 & 72.74 & RW2 \\ 
  Morocco & CENTRE & 90-94 & 49.56 & 60.37 & 40.61 & HT-Direct \\ 
  Morocco & CENTRE & 90-94 & 52.48 & 45.91 & 59.71 & RW2 \\ 
  Morocco & CENTRE & 95-99 & 45.35 & 58.21 & 35.22 & HT-Direct \\ 
  Morocco & CENTRE & 95-99 & 44.62 & 37.74 & 52.62 & RW2 \\ 
  Morocco & CENTRE & 00-04 & 39.73 & 53.39 & 29.46 & HT-Direct \\ 
  Morocco & CENTRE & 00-04 & 37.71 & 29.85 & 47.84 & RW2 \\ 
  Morocco & CENTRE & 05-09 & 31.80 & 10.52 & 91.14 & RW2 \\ 
  Morocco & CENTRE & 10-14 & 26.66 & 1.78 & 297.88 & RW2 \\ 
  Morocco & CENTRE & 15-19 & 22.69 & 0.19 & 728.81 & RW2 \\ 
  Morocco & CENTRE-NORD & 80-84 & 121.85 & 140.94 & 105.02 & HT-Direct \\ 
  Morocco & CENTRE-NORD & 80-84 & 131.00 & 114.04 & 150.03 & RW2 \\ 
  Morocco & CENTRE-NORD & 85-89 & 100.57 & 117.58 & 85.78 & HT-Direct \\ 
  Morocco & CENTRE-NORD & 85-89 & 104.53 & 93.39 & 117.13 & RW2 \\ 
  Morocco & CENTRE-NORD & 90-94 & 86.48 & 106.10 & 70.20 & HT-Direct \\ 
  Morocco & CENTRE-NORD & 90-94 & 82.99 & 72.41 & 95.24 & RW2 \\ 
  Morocco & CENTRE-NORD & 95-99 & 67.16 & 89.49 & 50.10 & HT-Direct \\ 
  Morocco & CENTRE-NORD & 95-99 & 66.87 & 55.71 & 80.17 & RW2 \\ 
  Morocco & CENTRE-NORD & 00-04 & 47.07 & 66.85 & 32.94 & HT-Direct \\ 
  Morocco & CENTRE-NORD & 00-04 & 52.17 & 39.68 & 68.06 & RW2 \\ 
  Morocco & CENTRE-NORD & 05-09 & 39.99 & 13.07 & 114.55 & RW2 \\ 
  Morocco & CENTRE-NORD & 10-14 & 30.55 & 2.01 & 320.97 & RW2 \\ 
  Morocco & CENTRE-NORD & 15-19 & 22.85 & 0.19 & 734.55 & RW2 \\ 
  Morocco & CENTRE-SUD & 80-84 & 95.88 & 114.94 & 79.70 & HT-Direct \\ 
  Morocco & CENTRE-SUD & 80-84 & 108.39 & 91.22 & 128.48 & RW2 \\ 
  Morocco & CENTRE-SUD & 85-89 & 106.44 & 134.79 & 83.47 & HT-Direct \\ 
  Morocco & CENTRE-SUD & 85-89 & 91.78 & 79.64 & 105.88 & RW2 \\ 
  Morocco & CENTRE-SUD & 90-94 & 75.23 & 97.13 & 57.95 & HT-Direct \\ 
  Morocco & CENTRE-SUD & 90-94 & 77.93 & 66.34 & 91.61 & RW2 \\ 
  Morocco & CENTRE-SUD & 95-99 & 64.17 & 94.64 & 43.05 & HT-Direct \\ 
  Morocco & CENTRE-SUD & 95-99 & 68.04 & 54.50 & 84.70 & RW2 \\ 
  Morocco & CENTRE-SUD & 00-04 & 58.27 & 92.24 & 36.31 & HT-Direct \\ 
  Morocco & CENTRE-SUD & 00-04 & 57.98 & 42.22 & 79.31 & RW2 \\ 
  Morocco & CENTRE-SUD & 05-09 & 48.60 & 15.72 & 138.48 & RW2 \\ 
  Morocco & CENTRE-SUD & 10-14 & 40.60 & 2.67 & 397.51 & RW2 \\ 
  Morocco & CENTRE-SUD & 15-19 & 34.38 & 0.28 & 800.25 & RW2 \\ 
  Morocco & NORD-OUEST & 80-84 & 115.10 & 127.84 & 103.49 & HT-Direct \\ 
  Morocco & NORD-OUEST & 80-84 & 123.57 & 111.44 & 136.72 & RW2 \\ 
  Morocco & NORD-OUEST & 85-89 & 97.96 & 111.66 & 85.77 & HT-Direct \\ 
  Morocco & NORD-OUEST & 85-89 & 94.30 & 85.47 & 104.19 & RW2 \\ 
  Morocco & NORD-OUEST & 90-94 & 61.40 & 74.85 & 50.24 & HT-Direct \\ 
  Morocco & NORD-OUEST & 90-94 & 71.65 & 62.92 & 81.34 & RW2 \\ 
  Morocco & NORD-OUEST & 95-99 & 57.30 & 72.67 & 45.02 & HT-Direct \\ 
  Morocco & NORD-OUEST & 95-99 & 56.48 & 47.81 & 66.62 & RW2 \\ 
  Morocco & NORD-OUEST & 00-04 & 45.13 & 63.68 & 31.80 & HT-Direct \\ 
  Morocco & NORD-OUEST & 00-04 & 43.89 & 34.09 & 56.53 & RW2 \\ 
  Morocco & NORD-OUEST & 05-09 & 33.84 & 11.04 & 98.48 & RW2 \\ 
  Morocco & NORD-OUEST & 10-14 & 25.95 & 1.75 & 290.68 & RW2 \\ 
  Morocco & NORD-OUEST & 15-19 & 19.60 & 0.17 & 695.69 & RW2 \\ 
  Morocco & ORIENTAL & 80-84 & 102.99 & 123.46 & 85.59 & HT-Direct \\ 
  Morocco & ORIENTAL & 80-84 & 103.01 & 86.56 & 122.28 & RW2 \\ 
  Morocco & ORIENTAL & 85-89 & 70.21 & 87.18 & 56.34 & HT-Direct \\ 
  Morocco & ORIENTAL & 85-89 & 82.79 & 71.20 & 96.04 & RW2 \\ 
  Morocco & ORIENTAL & 90-94 & 69.58 & 104.13 & 45.90 & HT-Direct \\ 
  Morocco & ORIENTAL & 90-94 & 67.78 & 55.69 & 82.07 & RW2 \\ 
  Morocco & ORIENTAL & 95-99 & 66.29 & 95.98 & 45.32 & HT-Direct \\ 
  Morocco & ORIENTAL & 95-99 & 57.16 & 44.22 & 73.62 & RW2 \\ 
  Morocco & ORIENTAL & 00-04 & 38.95 & 69.89 & 21.40 & HT-Direct \\ 
  Morocco & ORIENTAL & 00-04 & 46.80 & 31.97 & 68.14 & RW2 \\ 
  Morocco & ORIENTAL & 05-09 & 37.85 & 11.45 & 116.21 & RW2 \\ 
  Morocco & ORIENTAL & 10-14 & 30.37 & 1.87 & 337.87 & RW2 \\ 
  Morocco & ORIENTAL & 15-19 & 24.49 & 0.20 & 751.42 & RW2 \\ 
  Morocco & SUD & 80-84 & 159.28 & 183.91 & 137.39 & HT-Direct \\ 
  Morocco & SUD & 80-84 & 170.06 & 148.23 & 194.02 & RW2 \\ 
  Morocco & SUD & 85-89 & 120.16 & 140.38 & 102.50 & HT-Direct \\ 
  Morocco & SUD & 85-89 & 121.56 & 108.25 & 135.86 & RW2 \\ 
  Morocco & SUD & 90-94 & 80.53 & 100.37 & 64.33 & HT-Direct \\ 
  Morocco & SUD & 90-94 & 87.27 & 75.63 & 100.50 & RW2 \\ 
  Morocco & SUD & 95-99 & 70.57 & 97.65 & 50.58 & HT-Direct \\ 
  Morocco & SUD & 95-99 & 64.50 & 52.58 & 79.00 & RW2 \\ 
  Morocco & SUD & 00-04 & 42.53 & 66.38 & 27.00 & HT-Direct \\ 
  Morocco & SUD & 00-04 & 46.40 & 34.17 & 62.81 & RW2 \\ 
  Morocco & SUD & 05-09 & 32.86 & 10.47 & 97.05 & RW2 \\ 
  Morocco & SUD & 10-14 & 23.07 & 1.51 & 263.13 & RW2 \\ 
  Morocco & SUD & 15-19 & 16.08 & 0.13 & 662.37 & RW2 \\ 
  Morocco & TENSIFT & 80-84 & 138.66 & 158.16 & 121.23 & HT-Direct \\ 
  Morocco & TENSIFT & 80-84 & 143.08 & 125.55 & 162.42 & RW2 \\ 
  Morocco & TENSIFT & 85-89 & 91.41 & 110.61 & 75.27 & HT-Direct \\ 
  Morocco & TENSIFT & 85-89 & 108.09 & 95.41 & 122.02 & RW2 \\ 
  Morocco & TENSIFT & 90-94 & 95.43 & 118.56 & 76.43 & HT-Direct \\ 
  Morocco & TENSIFT & 90-94 & 83.72 & 72.15 & 97.08 & RW2 \\ 
  Morocco & TENSIFT & 95-99 & 63.52 & 89.90 & 44.51 & HT-Direct \\ 
  Morocco & TENSIFT & 95-99 & 66.76 & 54.76 & 81.18 & RW2 \\ 
  Morocco & TENSIFT & 00-04 & 49.43 & 68.51 & 35.46 & HT-Direct \\ 
  Morocco & TENSIFT & 00-04 & 51.81 & 39.45 & 67.50 & RW2 \\ 
  Morocco & TENSIFT & 05-09 & 39.64 & 12.88 & 115.30 & RW2 \\ 
  Morocco & TENSIFT & 10-14 & 30.01 & 2.00 & 329.13 & RW2 \\ 
  Morocco & TENSIFT & 15-19 & 22.69 & 0.20 & 732.27 & RW2 \\ 
  Mozambique & ALL & 80-84 & 250.42 & 237.09 & 264.07 & IHME \\ 
  Mozambique & ALL & 80-84 & 260.00 & 242.14 & 278.69 & RW2 \\ 
  Mozambique & ALL & 80-84 & 259.85 & 250.08 & 271.14 & UN \\ 
  Mozambique & ALL & 85-89 & 230.62 & 223.88 & 238.24 & IHME \\ 
  Mozambique & ALL & 85-89 & 248.51 & 232.21 & 265.48 & RW2 \\ 
  Mozambique & ALL & 85-89 & 248.72 & 238.91 & 257.88 & UN \\ 
  Mozambique & ALL & 90-94 & 205.17 & 202.81 & 207.71 & IHME \\ 
  Mozambique & ALL & 90-94 & 231.83 & 218.06 & 246.23 & RW2 \\ 
  Mozambique & ALL & 90-94 & 231.85 & 224.04 & 239.63 & UN \\ 
  Mozambique & ALL & 95-99 & 176.97 & 174.80 & 179.20 & IHME \\ 
  Mozambique & ALL & 95-99 & 198.19 & 188.48 & 208.19 & RW2 \\ 
  Mozambique & ALL & 95-99 & 198.20 & 192.02 & 205.08 & UN \\ 
  Mozambique & ALL & 00-04 & 144.16 & 141.98 & 146.21 & IHME \\ 
  Mozambique & ALL & 00-04 & 155.15 & 145.29 & 165.59 & RW2 \\ 
  Mozambique & ALL & 00-04 & 154.80 & 149.67 & 160.17 & UN \\ 
  Mozambique & ALL & 05-09 & 114.91 & 112.42 & 117.29 & IHME \\ 
  Mozambique & ALL & 05-09 & 119.68 & 109.50 & 130.67 & RW2 \\ 
  Mozambique & ALL & 05-09 & 120.29 & 115.63 & 125.70 & UN \\ 
  Mozambique & ALL & 10-14 & 90.78 & 87.44 & 93.88 & IHME \\ 
  Mozambique & ALL & 10-14 & 92.91 & 78.59 & 109.41 & RW2 \\ 
  Mozambique & ALL & 10-14 & 92.10 & 85.76 & 99.25 & UN \\ 
  Mozambique & CABO DELGADO & 80-84 & 301.69 & 372.89 & 238.91 & HT-Direct \\ 
  Mozambique & CABO DELGADO & 80-84 & 317.66 & 266.98 & 371.12 & RW2 \\ 
  Mozambique & CABO DELGADO & 85-89 & 287.41 & 334.14 & 244.81 & HT-Direct \\ 
  Mozambique & CABO DELGADO & 85-89 & 308.80 & 275.68 & 343.69 & RW2 \\ 
  Mozambique & CABO DELGADO & 90-94 & 272.10 & 311.44 & 236.02 & HT-Direct \\ 
  Mozambique & CABO DELGADO & 90-94 & 292.29 & 266.24 & 320.04 & RW2 \\ 
  Mozambique & CABO DELGADO & 95-99 & 242.70 & 277.86 & 210.68 & HT-Direct \\ 
  Mozambique & CABO DELGADO & 95-99 & 251.89 & 229.06 & 277.31 & RW2 \\ 
  Mozambique & CABO DELGADO & 00-04 & 194.02 & 226.87 & 164.92 & HT-Direct \\ 
  Mozambique & CABO DELGADO & 00-04 & 189.58 & 167.91 & 213.29 & RW2 \\ 
  Mozambique & CABO DELGADO & 05-09 & 100.98 & 128.53 & 78.80 & HT-Direct \\ 
  Mozambique & CABO DELGADO & 05-09 & 131.71 & 108.26 & 158.67 & RW2 \\ 
  Mozambique & CABO DELGADO & 10-14 & 60.13 & 106.56 & 33.18 & HT-Direct \\ 
  Mozambique & CABO DELGADO & 10-14 & 89.90 & 62.39 & 124.35 & RW2 \\ 
  Mozambique & CABO DELGADO & 15-19 & 60.37 & 22.09 & 150.69 & RW2 \\ 
  Mozambique & GAZA & 80-84 & 218.86 & 265.84 & 178.16 & HT-Direct \\ 
  Mozambique & GAZA & 80-84 & 219.87 & 186.79 & 257.49 & RW2 \\ 
  Mozambique & GAZA & 85-89 & 202.12 & 240.52 & 168.49 & HT-Direct \\ 
  Mozambique & GAZA & 85-89 & 212.13 & 188.00 & 238.55 & RW2 \\ 
  Mozambique & GAZA & 90-94 & 185.32 & 219.61 & 155.32 & HT-Direct \\ 
  Mozambique & GAZA & 90-94 & 197.90 & 178.18 & 218.99 & RW2 \\ 
  Mozambique & GAZA & 95-99 & 161.56 & 188.50 & 137.82 & HT-Direct \\ 
  Mozambique & GAZA & 95-99 & 172.77 & 155.41 & 191.08 & RW2 \\ 
  Mozambique & GAZA & 00-04 & 126.17 & 153.72 & 102.96 & HT-Direct \\ 
  Mozambique & GAZA & 00-04 & 142.66 & 125.36 & 161.99 & RW2 \\ 
  Mozambique & GAZA & 05-09 & 113.40 & 146.20 & 87.21 & HT-Direct \\ 
  Mozambique & GAZA & 05-09 & 117.72 & 98.16 & 140.87 & RW2 \\ 
  Mozambique & GAZA & 10-14 & 93.54 & 135.74 & 63.50 & HT-Direct \\ 
  Mozambique & GAZA & 10-14 & 98.35 & 74.83 & 128.94 & RW2 \\ 
  Mozambique & GAZA & 15-19 & 81.98 & 33.43 & 188.36 & RW2 \\ 
  Mozambique & INHAMBANE & 80-84 & 224.34 & 273.20 & 182.04 & HT-Direct \\ 
  Mozambique & INHAMBANE & 80-84 & 237.78 & 200.79 & 278.55 & RW2 \\ 
  Mozambique & INHAMBANE & 85-89 & 178.42 & 220.09 & 143.19 & HT-Direct \\ 
  Mozambique & INHAMBANE & 85-89 & 219.21 & 193.75 & 246.76 & RW2 \\ 
  Mozambique & INHAMBANE & 90-94 & 194.09 & 221.90 & 169.02 & HT-Direct \\ 
  Mozambique & INHAMBANE & 90-94 & 193.83 & 176.46 & 212.83 & RW2 \\ 
  Mozambique & INHAMBANE & 95-99 & 154.14 & 172.51 & 137.40 & HT-Direct \\ 
  Mozambique & INHAMBANE & 95-99 & 155.92 & 142.42 & 170.73 & RW2 \\ 
  Mozambique & INHAMBANE & 00-04 & 95.51 & 119.31 & 76.05 & HT-Direct \\ 
  Mozambique & INHAMBANE & 00-04 & 114.22 & 98.92 & 131.45 & RW2 \\ 
  Mozambique & INHAMBANE & 05-09 & 58.48 & 85.67 & 39.54 & HT-Direct \\ 
  Mozambique & INHAMBANE & 05-09 & 82.36 & 64.85 & 103.23 & RW2 \\ 
  Mozambique & INHAMBANE & 10-14 & 34.10 & 80.20 & 14.09 & HT-Direct \\ 
  Mozambique & INHAMBANE & 10-14 & 60.11 & 41.02 & 85.63 & RW2 \\ 
  Mozambique & INHAMBANE & 15-19 & 43.79 & 16.23 & 111.15 & RW2 \\ 
  Mozambique & MANICA & 80-84 & 270.94 & 338.61 & 212.45 & HT-Direct \\ 
  Mozambique & MANICA & 80-84 & 279.87 & 235.33 & 328.85 & RW2 \\ 
  Mozambique & MANICA & 85-89 & 265.20 & 319.32 & 217.33 & HT-Direct \\ 
  Mozambique & MANICA & 85-89 & 261.16 & 230.89 & 293.85 & RW2 \\ 
  Mozambique & MANICA & 90-94 & 214.34 & 244.31 & 187.12 & HT-Direct \\ 
  Mozambique & MANICA & 90-94 & 234.74 & 213.89 & 256.79 & RW2 \\ 
  Mozambique & MANICA & 95-99 & 193.56 & 222.21 & 167.81 & HT-Direct \\ 
  Mozambique & MANICA & 95-99 & 195.46 & 177.96 & 213.95 & RW2 \\ 
  Mozambique & MANICA & 00-04 & 123.43 & 146.75 & 103.37 & HT-Direct \\ 
  Mozambique & MANICA & 00-04 & 151.16 & 134.12 & 170.13 & RW2 \\ 
  Mozambique & MANICA & 05-09 & 104.27 & 132.37 & 81.57 & HT-Direct \\ 
  Mozambique & MANICA & 05-09 & 116.81 & 97.63 & 139.10 & RW2 \\ 
  Mozambique & MANICA & 10-14 & 113.10 & 176.77 & 70.41 & HT-Direct \\ 
  Mozambique & MANICA & 10-14 & 91.76 & 69.23 & 121.55 & RW2 \\ 
  Mozambique & MANICA & 15-19 & 72.01 & 29.18 & 169.07 & RW2 \\ 
  Mozambique & MAPUTO CIDADE & 80-84 & 95.34 & 126.69 & 71.12 & HT-Direct \\ 
  Mozambique & MAPUTO CIDADE & 80-84 & 98.61 & 77.76 & 124.95 & RW2 \\ 
  Mozambique & MAPUTO CIDADE & 85-89 & 99.97 & 127.17 & 78.07 & HT-Direct \\ 
  Mozambique & MAPUTO CIDADE & 85-89 & 102.89 & 87.17 & 121.15 & RW2 \\ 
  Mozambique & MAPUTO CIDADE & 90-94 & 95.91 & 119.13 & 76.83 & HT-Direct \\ 
  Mozambique & MAPUTO CIDADE & 90-94 & 104.07 & 89.90 & 119.77 & RW2 \\ 
  Mozambique & MAPUTO CIDADE & 95-99 & 90.57 & 111.96 & 72.94 & HT-Direct \\ 
  Mozambique & MAPUTO CIDADE & 95-99 & 100.77 & 87.45 & 115.35 & RW2 \\ 
  Mozambique & MAPUTO CIDADE & 00-04 & 92.35 & 112.34 & 75.62 & HT-Direct \\ 
  Mozambique & MAPUTO CIDADE & 00-04 & 94.80 & 81.68 & 109.95 & RW2 \\ 
  Mozambique & MAPUTO CIDADE & 05-09 & 75.06 & 96.89 & 57.83 & HT-Direct \\ 
  Mozambique & MAPUTO CIDADE & 05-09 & 88.22 & 71.52 & 108.91 & RW2 \\ 
  Mozambique & MAPUTO CIDADE & 10-14 & 66.54 & 116.45 & 37.12 & HT-Direct \\ 
  Mozambique & MAPUTO CIDADE & 10-14 & 82.25 & 56.65 & 117.58 & RW2 \\ 
  Mozambique & MAPUTO CIDADE & 15-19 & 76.67 & 27.51 & 194.23 & RW2 \\ 
  Mozambique & MAPUTO PROVINCIA & 80-84 & 134.79 & 190.48 & 93.51 & HT-Direct \\ 
  Mozambique & MAPUTO PROVINCIA & 80-84 & 130.37 & 100.97 & 168.02 & RW2 \\ 
  Mozambique & MAPUTO PROVINCIA & 85-89 & 129.04 & 164.42 & 100.37 & HT-Direct \\ 
  Mozambique & MAPUTO PROVINCIA & 85-89 & 131.48 & 110.64 & 155.98 & RW2 \\ 
  Mozambique & MAPUTO PROVINCIA & 90-94 & 118.37 & 146.97 & 94.73 & HT-Direct \\ 
  Mozambique & MAPUTO PROVINCIA & 90-94 & 128.76 & 111.65 & 147.50 & RW2 \\ 
  Mozambique & MAPUTO PROVINCIA & 95-99 & 93.94 & 118.88 & 73.80 & HT-Direct \\ 
  Mozambique & MAPUTO PROVINCIA & 95-99 & 120.75 & 105.07 & 137.57 & RW2 \\ 
  Mozambique & MAPUTO PROVINCIA & 00-04 & 114.01 & 137.97 & 93.77 & HT-Direct \\ 
  Mozambique & MAPUTO PROVINCIA & 00-04 & 109.85 & 95.19 & 126.51 & RW2 \\ 
  Mozambique & MAPUTO PROVINCIA & 05-09 & 85.20 & 107.47 & 67.21 & HT-Direct \\ 
  Mozambique & MAPUTO PROVINCIA & 05-09 & 99.65 & 81.92 & 121.03 & RW2 \\ 
  Mozambique & MAPUTO PROVINCIA & 10-14 & 76.66 & 133.78 & 42.72 & HT-Direct \\ 
  Mozambique & MAPUTO PROVINCIA & 10-14 & 90.96 & 65.27 & 125.72 & RW2 \\ 
  Mozambique & MAPUTO PROVINCIA & 15-19 & 83.08 & 31.64 & 199.73 & RW2 \\ 
  Mozambique & NAMPULA & 80-84 & 303.38 & 350.91 & 259.70 & HT-Direct \\ 
  Mozambique & NAMPULA & 80-84 & 311.24 & 271.62 & 352.85 & RW2 \\ 
  Mozambique & NAMPULA & 85-89 & 239.61 & 287.83 & 197.23 & HT-Direct \\ 
  Mozambique & NAMPULA & 85-89 & 292.09 & 260.89 & 324.15 & RW2 \\ 
  Mozambique & NAMPULA & 90-94 & 259.15 & 310.41 & 213.73 & HT-Direct \\ 
  Mozambique & NAMPULA & 90-94 & 266.36 & 241.83 & 292.92 & RW2 \\ 
  Mozambique & NAMPULA & 95-99 & 231.42 & 258.29 & 206.56 & HT-Direct \\ 
  Mozambique & NAMPULA & 95-99 & 219.73 & 200.64 & 240.83 & RW2 \\ 
  Mozambique & NAMPULA & 00-04 & 129.86 & 156.74 & 107.00 & HT-Direct \\ 
  Mozambique & NAMPULA & 00-04 & 157.77 & 139.25 & 178.36 & RW2 \\ 
  Mozambique & NAMPULA & 05-09 & 62.71 & 86.48 & 45.14 & HT-Direct \\ 
  Mozambique & NAMPULA & 05-09 & 106.48 & 86.77 & 129.10 & RW2 \\ 
  Mozambique & NAMPULA & 10-14 & 78.29 & 124.54 & 48.27 & HT-Direct \\ 
  Mozambique & NAMPULA & 10-14 & 71.55 & 51.13 & 97.12 & RW2 \\ 
  Mozambique & NAMPULA & 15-19 & 47.50 & 18.09 & 117.77 & RW2 \\ 
  Mozambique & NIASSA & 80-84 & 331.59 & 415.89 & 256.86 & HT-Direct \\ 
  Mozambique & NIASSA & 80-84 & 295.70 & 242.64 & 356.17 & RW2 \\ 
  Mozambique & NIASSA & 85-89 & 230.32 & 292.84 & 177.80 & HT-Direct \\ 
  Mozambique & NIASSA & 85-89 & 271.51 & 234.68 & 311.92 & RW2 \\ 
  Mozambique & NIASSA & 90-94 & 211.33 & 257.42 & 171.59 & HT-Direct \\ 
  Mozambique & NIASSA & 90-94 & 243.93 & 217.22 & 272.44 & RW2 \\ 
  Mozambique & NIASSA & 95-99 & 189.65 & 217.87 & 164.31 & HT-Direct \\ 
  Mozambique & NIASSA & 95-99 & 201.25 & 181.60 & 222.34 & RW2 \\ 
  Mozambique & NIASSA & 00-04 & 152.09 & 188.48 & 121.68 & HT-Direct \\ 
  Mozambique & NIASSA & 00-04 & 147.78 & 129.48 & 168.25 & RW2 \\ 
  Mozambique & NIASSA & 05-09 & 79.42 & 98.47 & 63.79 & HT-Direct \\ 
  Mozambique & NIASSA & 05-09 & 102.99 & 85.43 & 123.57 & RW2 \\ 
  Mozambique & NIASSA & 10-14 & 83.02 & 148.36 & 44.94 & HT-Direct \\ 
  Mozambique & NIASSA & 10-14 & 71.77 & 52.01 & 98.18 & RW2 \\ 
  Mozambique & NIASSA & 15-19 & 49.89 & 18.87 & 123.87 & RW2 \\ 
  Mozambique & SOFALA & 80-84 & 316.07 & 367.96 & 268.39 & HT-Direct \\ 
  Mozambique & SOFALA & 80-84 & 329.17 & 289.94 & 370.76 & RW2 \\ 
  Mozambique & SOFALA & 85-89 & 295.54 & 335.22 & 258.73 & HT-Direct \\ 
  Mozambique & SOFALA & 85-89 & 295.24 & 268.13 & 324.06 & RW2 \\ 
  Mozambique & SOFALA & 90-94 & 230.78 & 266.29 & 198.72 & HT-Direct \\ 
  Mozambique & SOFALA & 90-94 & 254.72 & 233.37 & 277.37 & RW2 \\ 
  Mozambique & SOFALA & 95-99 & 178.11 & 207.25 & 152.28 & HT-Direct \\ 
  Mozambique & SOFALA & 95-99 & 202.55 & 184.46 & 221.79 & RW2 \\ 
  Mozambique & SOFALA & 00-04 & 143.18 & 167.54 & 121.84 & HT-Direct \\ 
  Mozambique & SOFALA & 00-04 & 148.27 & 131.94 & 166.09 & RW2 \\ 
  Mozambique & SOFALA & 05-09 & 97.25 & 128.19 & 73.15 & HT-Direct \\ 
  Mozambique & SOFALA & 05-09 & 106.68 & 89.72 & 126.52 & RW2 \\ 
  Mozambique & SOFALA & 10-14 & 70.63 & 100.77 & 49.02 & HT-Direct \\ 
  Mozambique & SOFALA & 10-14 & 77.38 & 59.86 & 100.01 & RW2 \\ 
  Mozambique & SOFALA & 15-19 & 55.88 & 22.95 & 130.68 & RW2 \\ 
  Mozambique & TETE & 80-84 & 293.29 & 357.31 & 236.51 & HT-Direct \\ 
  Mozambique & TETE & 80-84 & 302.06 & 258.83 & 348.35 & RW2 \\ 
  Mozambique & TETE & 85-89 & 268.28 & 312.54 & 228.20 & HT-Direct \\ 
  Mozambique & TETE & 85-89 & 292.94 & 263.51 & 323.85 & RW2 \\ 
  Mozambique & TETE & 90-94 & 261.51 & 291.39 & 233.70 & HT-Direct \\ 
  Mozambique & TETE & 90-94 & 275.00 & 253.75 & 297.43 & RW2 \\ 
  Mozambique & TETE & 95-99 & 237.47 & 267.11 & 210.18 & HT-Direct \\ 
  Mozambique & TETE & 95-99 & 238.56 & 219.88 & 258.15 & RW2 \\ 
  Mozambique & TETE & 00-04 & 170.39 & 196.52 & 147.11 & HT-Direct \\ 
  Mozambique & TETE & 00-04 & 190.70 & 172.17 & 210.67 & RW2 \\ 
  Mozambique & TETE & 05-09 & 116.42 & 139.45 & 96.77 & HT-Direct \\ 
  Mozambique & TETE & 05-09 & 150.66 & 129.83 & 174.48 & RW2 \\ 
  Mozambique & TETE & 10-14 & 146.31 & 214.37 & 97.18 & HT-Direct \\ 
  Mozambique & TETE & 10-14 & 120.86 & 94.00 & 154.84 & RW2 \\ 
  Mozambique & TETE & 15-19 & 96.78 & 39.92 & 218.44 & RW2 \\ 
  Mozambique & ZAMBEZIA & 80-84 & 271.78 & 326.99 & 222.81 & HT-Direct \\ 
  Mozambique & ZAMBEZIA & 80-84 & 275.33 & 237.04 & 318.43 & RW2 \\ 
  Mozambique & ZAMBEZIA & 85-89 & 263.78 & 311.27 & 221.20 & HT-Direct \\ 
  Mozambique & ZAMBEZIA & 85-89 & 267.44 & 239.63 & 297.54 & RW2 \\ 
  Mozambique & ZAMBEZIA & 90-94 & 246.76 & 279.44 & 216.75 & HT-Direct \\ 
  Mozambique & ZAMBEZIA & 90-94 & 252.59 & 231.52 & 274.83 & RW2 \\ 
  Mozambique & ZAMBEZIA & 95-99 & 192.54 & 218.65 & 168.87 & HT-Direct \\ 
  Mozambique & ZAMBEZIA & 95-99 & 220.71 & 202.74 & 239.46 & RW2 \\ 
  Mozambique & ZAMBEZIA & 00-04 & 150.44 & 175.90 & 128.09 & HT-Direct \\ 
  Mozambique & ZAMBEZIA & 00-04 & 177.47 & 160.21 & 196.21 & RW2 \\ 
  Mozambique & ZAMBEZIA & 05-09 & 143.37 & 166.09 & 123.30 & HT-Direct \\ 
  Mozambique & ZAMBEZIA & 05-09 & 140.24 & 122.06 & 160.49 & RW2 \\ 
  Mozambique & ZAMBEZIA & 10-14 & 88.24 & 129.42 & 59.27 & HT-Direct \\ 
  Mozambique & ZAMBEZIA & 10-14 & 111.34 & 87.59 & 140.93 & RW2 \\ 
  Mozambique & ZAMBEZIA & 15-19 & 88.62 & 36.80 & 197.67 & RW2 \\ 
  Namibia & ALL & 80-84 & 92.76 & 89.63 & 95.98 & IHME \\ 
  Namibia & ALL & 80-84 & 94.40 & 77.46 & 114.60 & RW2 \\ 
  Namibia & ALL & 80-84 & 94.25 & 90.45 & 98.04 & UN \\ 
  Namibia & ALL & 85-89 & 77.18 & 74.67 & 79.77 & IHME \\ 
  Namibia & ALL & 85-89 & 80.38 & 69.75 & 92.45 & RW2 \\ 
  Namibia & ALL & 85-89 & 81.00 & 77.97 & 84.32 & UN \\ 
  Namibia & ALL & 90-94 & 66.03 & 64.06 & 68.03 & IHME \\ 
  Namibia & ALL & 90-94 & 71.83 & 64.15 & 80.34 & RW2 \\ 
  Namibia & ALL & 90-94 & 70.97 & 68.09 & 74.11 & UN \\ 
  Namibia & ALL & 95-99 & 61.14 & 59.16 & 63.31 & IHME \\ 
  Namibia & ALL & 95-99 & 71.72 & 62.53 & 81.83 & RW2 \\ 
  Namibia & ALL & 95-99 & 72.65 & 69.21 & 75.92 & UN \\ 
  Namibia & ALL & 00-04 & 64.68 & 62.33 & 67.00 & IHME \\ 
  Namibia & ALL & 00-04 & 76.05 & 68.56 & 84.46 & RW2 \\ 
  Namibia & ALL & 00-04 & 75.58 & 72.49 & 78.81 & UN \\ 
  Namibia & ALL & 05-09 & 61.58 & 58.75 & 64.51 & IHME \\ 
  Namibia & ALL & 05-09 & 64.42 & 58.34 & 71.06 & RW2 \\ 
  Namibia & ALL & 05-09 & 64.51 & 61.04 & 68.51 & UN \\ 
  Namibia & ALL & 10-14 & 48.89 & 45.65 & 52.49 & IHME \\ 
  Namibia & ALL & 10-14 & 50.33 & 42.12 & 59.97 & RW2 \\ 
  Namibia & ALL & 10-14 & 50.39 & 45.14 & 56.48 & UN \\ 
  Namibia & CAPRIVI & 80-84 & 101.08 & 190.45 & 51.00 & HT-Direct \\ 
  Namibia & CAPRIVI & 80-84 & 108.11 & 66.66 & 170.86 & RW2 \\ 
  Namibia & CAPRIVI & 85-89 & 80.31 & 125.07 & 50.64 & HT-Direct \\ 
  Namibia & CAPRIVI & 85-89 & 91.98 & 66.08 & 125.71 & RW2 \\ 
  Namibia & CAPRIVI & 90-94 & 74.33 & 106.00 & 51.58 & HT-Direct \\ 
  Namibia & CAPRIVI & 90-94 & 83.48 & 64.89 & 106.54 & RW2 \\ 
  Namibia & CAPRIVI & 95-99 & 64.11 & 97.96 & 41.42 & HT-Direct \\ 
  Namibia & CAPRIVI & 95-99 & 90.52 & 71.82 & 113.18 & RW2 \\ 
  Namibia & CAPRIVI & 00-04 & 113.30 & 161.55 & 78.11 & HT-Direct \\ 
  Namibia & CAPRIVI & 00-04 & 94.29 & 76.36 & 116.09 & RW2 \\ 
  Namibia & CAPRIVI & 05-09 & 79.73 & 109.06 & 57.78 & HT-Direct \\ 
  Namibia & CAPRIVI & 05-09 & 83.44 & 66.28 & 104.49 & RW2 \\ 
  Namibia & CAPRIVI & 10-14 & 76.48 & 133.14 & 42.74 & HT-Direct \\ 
  Namibia & CAPRIVI & 10-14 & 71.54 & 49.75 & 101.78 & RW2 \\ 
  Namibia & CAPRIVI & 15-19 & 61.32 & 21.45 & 164.58 & RW2 \\ 
  Namibia & ERONGO & 80-84 & 42.21 & 81.94 & 21.29 & HT-Direct \\ 
  Namibia & ERONGO & 80-84 & 57.82 & 36.62 & 90.84 & RW2 \\ 
  Namibia & ERONGO & 85-89 & 64.51 & 98.04 & 41.91 & HT-Direct \\ 
  Namibia & ERONGO & 85-89 & 51.50 & 36.83 & 71.78 & RW2 \\ 
  Namibia & ERONGO & 90-94 & 33.31 & 56.63 & 19.39 & HT-Direct \\ 
  Namibia & ERONGO & 90-94 & 48.88 & 37.35 & 63.73 & RW2 \\ 
  Namibia & ERONGO & 95-99 & 34.55 & 57.34 & 20.62 & HT-Direct \\ 
  Namibia & ERONGO & 95-99 & 55.24 & 43.54 & 69.44 & RW2 \\ 
  Namibia & ERONGO & 00-04 & 56.20 & 85.17 & 36.69 & HT-Direct \\ 
  Namibia & ERONGO & 00-04 & 59.64 & 47.70 & 74.11 & RW2 \\ 
  Namibia & ERONGO & 05-09 & 58.43 & 88.99 & 37.93 & HT-Direct \\ 
  Namibia & ERONGO & 05-09 & 55.05 & 42.61 & 70.64 & RW2 \\ 
  Namibia & ERONGO & 10-14 & 54.86 & 90.32 & 32.82 & HT-Direct \\ 
  Namibia & ERONGO & 10-14 & 49.62 & 35.01 & 69.77 & RW2 \\ 
  Namibia & ERONGO & 15-19 & 45.20 & 16.77 & 116.35 & RW2 \\ 
  Namibia & HARDAP & 80-84 & 131.68 & 213.15 & 78.26 & HT-Direct \\ 
  Namibia & HARDAP & 80-84 & 126.32 & 84.53 & 184.16 & RW2 \\ 
  Namibia & HARDAP & 85-89 & 76.08 & 116.26 & 49.02 & HT-Direct \\ 
  Namibia & HARDAP & 85-89 & 92.56 & 68.38 & 123.74 & RW2 \\ 
  Namibia & HARDAP & 90-94 & 57.58 & 83.96 & 39.13 & HT-Direct \\ 
  Namibia & HARDAP & 90-94 & 71.81 & 56.36 & 91.00 & RW2 \\ 
  Namibia & HARDAP & 95-99 & 66.81 & 108.39 & 40.46 & HT-Direct \\ 
  Namibia & HARDAP & 95-99 & 65.56 & 51.92 & 82.14 & RW2 \\ 
  Namibia & HARDAP & 00-04 & 50.20 & 85.41 & 29.04 & HT-Direct \\ 
  Namibia & HARDAP & 00-04 & 56.86 & 44.44 & 72.39 & RW2 \\ 
  Namibia & HARDAP & 05-09 & 45.87 & 73.59 & 28.27 & HT-Direct \\ 
  Namibia & HARDAP & 05-09 & 41.87 & 31.09 & 56.34 & RW2 \\ 
  Namibia & HARDAP & 10-14 & 30.41 & 60.43 & 15.07 & HT-Direct \\ 
  Namibia & HARDAP & 10-14 & 29.91 & 19.75 & 45.17 & RW2 \\ 
  Namibia & HARDAP & 15-19 & 21.39 & 7.40 & 61.15 & RW2 \\ 
  Namibia & KARAS & 80-84 & 107.64 & 154.83 & 73.57 & HT-Direct \\ 
  Namibia & KARAS & 80-84 & 117.21 & 82.36 & 164.30 & RW2 \\ 
  Namibia & KARAS & 85-89 & 74.77 & 109.87 & 50.25 & HT-Direct \\ 
  Namibia & KARAS & 85-89 & 89.16 & 68.45 & 115.35 & RW2 \\ 
  Namibia & KARAS & 90-94 & 74.91 & 103.52 & 53.74 & HT-Direct \\ 
  Namibia & KARAS & 90-94 & 72.02 & 57.02 & 90.14 & RW2 \\ 
  Namibia & KARAS & 95-99 & 41.06 & 64.96 & 25.72 & HT-Direct \\ 
  Namibia & KARAS & 95-99 & 69.05 & 53.77 & 87.08 & RW2 \\ 
  Namibia & KARAS & 00-04 & 57.75 & 87.00 & 37.93 & HT-Direct \\ 
  Namibia & KARAS & 00-04 & 64.22 & 50.10 & 81.67 & RW2 \\ 
  Namibia & KARAS & 05-09 & 59.32 & 90.66 & 38.36 & HT-Direct \\ 
  Namibia & KARAS & 05-09 & 51.91 & 38.87 & 68.92 & RW2 \\ 
  Namibia & KARAS & 10-14 & 46.36 & 93.53 & 22.39 & HT-Direct \\ 
  Namibia & KARAS & 10-14 & 40.94 & 26.46 & 63.81 & RW2 \\ 
  Namibia & KARAS & 15-19 & 32.32 & 10.70 & 97.43 & RW2 \\ 
  Namibia & KAVANGO & 80-84 & 111.48 & 175.10 & 69.04 & HT-Direct \\ 
  Namibia & KAVANGO & 80-84 & 119.90 & 85.77 & 166.03 & RW2 \\ 
  Namibia & KAVANGO & 85-89 & 103.51 & 138.03 & 76.85 & HT-Direct \\ 
  Namibia & KAVANGO & 85-89 & 100.73 & 79.95 & 126.45 & RW2 \\ 
  Namibia & KAVANGO & 90-94 & 77.45 & 103.94 & 57.27 & HT-Direct \\ 
  Namibia & KAVANGO & 90-94 & 90.03 & 75.37 & 107.24 & RW2 \\ 
  Namibia & KAVANGO & 95-99 & 83.17 & 99.83 & 69.08 & HT-Direct \\ 
  Namibia & KAVANGO & 95-99 & 95.72 & 82.46 & 110.67 & RW2 \\ 
  Namibia & KAVANGO & 00-04 & 68.11 & 90.04 & 51.21 & HT-Direct \\ 
  Namibia & KAVANGO & 00-04 & 97.85 & 84.06 & 113.39 & RW2 \\ 
  Namibia & KAVANGO & 05-09 & 106.12 & 133.89 & 83.56 & HT-Direct \\ 
  Namibia & KAVANGO & 05-09 & 85.95 & 72.12 & 102.20 & RW2 \\ 
  Namibia & KAVANGO & 10-14 & 77.96 & 122.95 & 48.53 & HT-Direct \\ 
  Namibia & KAVANGO & 10-14 & 73.23 & 55.45 & 96.60 & RW2 \\ 
  Namibia & KAVANGO & 15-19 & 62.41 & 23.98 & 155.00 & RW2 \\ 
  Namibia & KHOMAS & 80-84 & 79.70 & 145.33 & 42.25 & HT-Direct \\ 
  Namibia & KHOMAS & 80-84 & 78.53 & 49.17 & 122.95 & RW2 \\ 
  Namibia & KHOMAS & 85-89 & 43.35 & 76.73 & 24.12 & HT-Direct \\ 
  Namibia & KHOMAS & 85-89 & 64.27 & 45.21 & 90.31 & RW2 \\ 
  Namibia & KHOMAS & 90-94 & 52.37 & 77.59 & 35.04 & HT-Direct \\ 
  Namibia & KHOMAS & 90-94 & 56.01 & 42.89 & 72.80 & RW2 \\ 
  Namibia & KHOMAS & 95-99 & 47.14 & 82.78 & 26.40 & HT-Direct \\ 
  Namibia & KHOMAS & 95-99 & 57.54 & 45.80 & 72.27 & RW2 \\ 
  Namibia & KHOMAS & 00-04 & 51.76 & 79.01 & 33.56 & HT-Direct \\ 
  Namibia & KHOMAS & 00-04 & 55.74 & 44.89 & 69.05 & RW2 \\ 
  Namibia & KHOMAS & 05-09 & 47.98 & 66.85 & 34.24 & HT-Direct \\ 
  Namibia & KHOMAS & 05-09 & 45.67 & 35.54 & 58.42 & RW2 \\ 
  Namibia & KHOMAS & 10-14 & 33.69 & 71.75 & 15.49 & HT-Direct \\ 
  Namibia & KHOMAS & 10-14 & 36.34 & 24.80 & 52.90 & RW2 \\ 
  Namibia & KHOMAS & 15-19 & 29.06 & 10.19 & 80.25 & RW2 \\ 
  Namibia & KUNENE & 80-84 & 83.05 & 145.22 & 46.06 & HT-Direct \\ 
  Namibia & KUNENE & 80-84 & 114.32 & 77.19 & 167.01 & RW2 \\ 
  Namibia & KUNENE & 85-89 & 96.75 & 144.44 & 63.63 & HT-Direct \\ 
  Namibia & KUNENE & 85-89 & 89.82 & 67.09 & 119.37 & RW2 \\ 
  Namibia & KUNENE & 90-94 & 85.35 & 119.76 & 60.15 & HT-Direct \\ 
  Namibia & KUNENE & 90-94 & 74.40 & 59.27 & 92.91 & RW2 \\ 
  Namibia & KUNENE & 95-99 & 53.44 & 78.09 & 36.26 & HT-Direct \\ 
  Namibia & KUNENE & 95-99 & 72.64 & 59.08 & 88.36 & RW2 \\ 
  Namibia & KUNENE & 00-04 & 36.18 & 58.36 & 22.24 & HT-Direct \\ 
  Namibia & KUNENE & 00-04 & 67.97 & 55.12 & 83.12 & RW2 \\ 
  Namibia & KUNENE & 05-09 & 50.84 & 74.05 & 34.64 & HT-Direct \\ 
  Namibia & KUNENE & 05-09 & 54.53 & 42.98 & 69.02 & RW2 \\ 
  Namibia & KUNENE & 10-14 & 68.24 & 109.19 & 41.92 & HT-Direct \\ 
  Namibia & KUNENE & 10-14 & 42.94 & 30.92 & 59.60 & RW2 \\ 
  Namibia & KUNENE & 15-19 & 33.83 & 12.59 & 89.20 & RW2 \\ 
  Namibia & OHANGWENA & 80-84 & 149.73 & 211.99 & 103.37 & HT-Direct \\ 
  Namibia & OHANGWENA & 80-84 & 135.95 & 98.94 & 184.19 & RW2 \\ 
  Namibia & OHANGWENA & 85-89 & 74.38 & 107.77 & 50.74 & HT-Direct \\ 
  Namibia & OHANGWENA & 85-89 & 111.31 & 87.34 & 141.25 & RW2 \\ 
  Namibia & OHANGWENA & 90-94 & 87.90 & 117.98 & 64.93 & HT-Direct \\ 
  Namibia & OHANGWENA & 90-94 & 97.01 & 80.03 & 116.95 & RW2 \\ 
  Namibia & OHANGWENA & 95-99 & 71.00 & 96.98 & 51.58 & HT-Direct \\ 
  Namibia & OHANGWENA & 95-99 & 100.04 & 84.37 & 118.38 & RW2 \\ 
  Namibia & OHANGWENA & 00-04 & 123.21 & 155.85 & 96.63 & HT-Direct \\ 
  Namibia & OHANGWENA & 00-04 & 97.87 & 83.23 & 114.97 & RW2 \\ 
  Namibia & OHANGWENA & 05-09 & 70.09 & 94.42 & 51.67 & HT-Direct \\ 
  Namibia & OHANGWENA & 05-09 & 80.39 & 65.76 & 98.05 & RW2 \\ 
  Namibia & OHANGWENA & 10-14 & 71.68 & 124.08 & 40.39 & HT-Direct \\ 
  Namibia & OHANGWENA & 10-14 & 63.86 & 46.35 & 86.84 & RW2 \\ 
  Namibia & OHANGWENA & 15-19 & 50.62 & 18.71 & 128.87 & RW2 \\ 
  Namibia & OMAHEKE & 80-84 & 61.73 & 118.97 & 31.06 & HT-Direct \\ 
  Namibia & OMAHEKE & 80-84 & 86.79 & 54.47 & 131.47 & RW2 \\ 
  Namibia & OMAHEKE & 85-89 & 44.68 & 69.65 & 28.39 & HT-Direct \\ 
  Namibia & OMAHEKE & 85-89 & 76.17 & 56.52 & 101.42 & RW2 \\ 
  Namibia & OMAHEKE & 90-94 & 70.93 & 93.48 & 53.50 & HT-Direct \\ 
  Namibia & OMAHEKE & 90-94 & 70.96 & 58.03 & 86.63 & RW2 \\ 
  Namibia & OMAHEKE & 95-99 & 80.01 & 98.54 & 64.72 & HT-Direct \\ 
  Namibia & OMAHEKE & 95-99 & 76.15 & 63.86 & 91.26 & RW2 \\ 
  Namibia & OMAHEKE & 00-04 & 57.08 & 99.30 & 32.17 & HT-Direct \\ 
  Namibia & OMAHEKE & 00-04 & 75.42 & 60.88 & 93.34 & RW2 \\ 
  Namibia & OMAHEKE & 05-09 & 57.20 & 88.46 & 36.54 & HT-Direct \\ 
  Namibia & OMAHEKE & 05-09 & 62.26 & 47.08 & 82.18 & RW2 \\ 
  Namibia & OMAHEKE & 10-14 & 46.25 & 90.12 & 23.20 & HT-Direct \\ 
  Namibia & OMAHEKE & 10-14 & 49.78 & 32.70 & 74.17 & RW2 \\ 
  Namibia & OMAHEKE & 15-19 & 39.77 & 13.67 & 108.86 & RW2 \\ 
  Namibia & OMUSATI & 80-84 & 102.85 & 187.78 & 53.79 & HT-Direct \\ 
  Namibia & OMUSATI & 80-84 & 127.27 & 83.51 & 189.21 & RW2 \\ 
  Namibia & OMUSATI & 85-89 & 101.30 & 150.06 & 67.13 & HT-Direct \\ 
  Namibia & OMUSATI & 85-89 & 97.76 & 72.19 & 131.35 & RW2 \\ 
  Namibia & OMUSATI & 90-94 & 57.06 & 83.48 & 38.65 & HT-Direct \\ 
  Namibia & OMUSATI & 90-94 & 79.38 & 63.11 & 99.16 & RW2 \\ 
  Namibia & OMUSATI & 95-99 & 68.22 & 91.13 & 50.75 & HT-Direct \\ 
  Namibia & OMUSATI & 95-99 & 76.39 & 62.78 & 92.30 & RW2 \\ 
  Namibia & OMUSATI & 00-04 & 69.06 & 95.77 & 49.39 & HT-Direct \\ 
  Namibia & OMUSATI & 00-04 & 69.86 & 57.67 & 84.39 & RW2 \\ 
  Namibia & OMUSATI & 05-09 & 57.76 & 80.55 & 41.12 & HT-Direct \\ 
  Namibia & OMUSATI & 05-09 & 53.77 & 42.27 & 68.11 & RW2 \\ 
  Namibia & OMUSATI & 10-14 & 32.32 & 67.73 & 15.12 & HT-Direct \\ 
  Namibia & OMUSATI & 10-14 & 39.86 & 27.37 & 57.80 & RW2 \\ 
  Namibia & OMUSATI & 15-19 & 29.57 & 10.35 & 82.04 & RW2 \\ 
  Namibia & OSHANA & 80-84 & 105.26 & 154.18 & 70.57 & HT-Direct \\ 
  Namibia & OSHANA & 80-84 & 118.10 & 83.91 & 163.85 & RW2 \\ 
  Namibia & OSHANA & 85-89 & 90.19 & 135.89 & 58.82 & HT-Direct \\ 
  Namibia & OSHANA & 85-89 & 95.87 & 74.02 & 123.76 & RW2 \\ 
  Namibia & OSHANA & 90-94 & 64.90 & 88.85 & 47.06 & HT-Direct \\ 
  Namibia & OSHANA & 90-94 & 82.76 & 67.61 & 100.76 & RW2 \\ 
  Namibia & OSHANA & 95-99 & 81.35 & 107.09 & 61.37 & HT-Direct \\ 
  Namibia & OSHANA & 95-99 & 84.22 & 70.33 & 100.64 & RW2 \\ 
  Namibia & OSHANA & 00-04 & 76.67 & 106.52 & 54.67 & HT-Direct \\ 
  Namibia & OSHANA & 00-04 & 81.17 & 66.94 & 98.28 & RW2 \\ 
  Namibia & OSHANA & 05-09 & 74.36 & 108.05 & 50.58 & HT-Direct \\ 
  Namibia & OSHANA & 05-09 & 65.81 & 51.29 & 84.57 & RW2 \\ 
  Namibia & OSHANA & 10-14 & 39.32 & 88.60 & 16.94 & HT-Direct \\ 
  Namibia & OSHANA & 10-14 & 51.68 & 35.03 & 75.04 & RW2 \\ 
  Namibia & OSHANA & 15-19 & 40.50 & 14.27 & 110.29 & RW2 \\ 
  Namibia & OSHIKOTO & 80-84 & 38.42 & 79.53 & 18.14 & HT-Direct \\ 
  Namibia & OSHIKOTO & 80-84 & 87.42 & 57.73 & 128.14 & RW2 \\ 
  Namibia & OSHIKOTO & 85-89 & 64.98 & 92.35 & 45.32 & HT-Direct \\ 
  Namibia & OSHIKOTO & 85-89 & 78.89 & 60.01 & 103.00 & RW2 \\ 
  Namibia & OSHIKOTO & 90-94 & 79.69 & 108.33 & 58.12 & HT-Direct \\ 
  Namibia & OSHIKOTO & 90-94 & 75.20 & 61.60 & 91.54 & RW2 \\ 
  Namibia & OSHIKOTO & 95-99 & 76.50 & 97.17 & 59.94 & HT-Direct \\ 
  Namibia & OSHIKOTO & 95-99 & 83.98 & 71.21 & 98.93 & RW2 \\ 
  Namibia & OSHIKOTO & 00-04 & 92.63 & 118.77 & 71.78 & HT-Direct \\ 
  Namibia & OSHIKOTO & 00-04 & 87.92 & 74.81 & 103.29 & RW2 \\ 
  Namibia & OSHIKOTO & 05-09 & 66.24 & 94.08 & 46.22 & HT-Direct \\ 
  Namibia & OSHIKOTO & 05-09 & 77.40 & 62.78 & 95.17 & RW2 \\ 
  Namibia & OSHIKOTO & 10-14 & 67.35 & 108.58 & 41.06 & HT-Direct \\ 
  Namibia & OSHIKOTO & 10-14 & 66.15 & 48.45 & 89.35 & RW2 \\ 
  Namibia & OSHIKOTO & 15-19 & 56.47 & 21.38 & 142.25 & RW2 \\ 
  Namibia & OTJOZONDJUPA & 80-84 & 65.78 & 121.48 & 34.62 & HT-Direct \\ 
  Namibia & OTJOZONDJUPA & 80-84 & 82.84 & 55.47 & 120.75 & RW2 \\ 
  Namibia & OTJOZONDJUPA & 85-89 & 62.43 & 89.12 & 43.35 & HT-Direct \\ 
  Namibia & OTJOZONDJUPA & 85-89 & 70.44 & 53.33 & 92.29 & RW2 \\ 
  Namibia & OTJOZONDJUPA & 90-94 & 55.50 & 76.66 & 39.93 & HT-Direct \\ 
  Namibia & OTJOZONDJUPA & 90-94 & 63.69 & 51.56 & 78.36 & RW2 \\ 
  Namibia & OTJOZONDJUPA & 95-99 & 45.17 & 68.64 & 29.48 & HT-Direct \\ 
  Namibia & OTJOZONDJUPA & 95-99 & 67.66 & 56.23 & 81.30 & RW2 \\ 
  Namibia & OTJOZONDJUPA & 00-04 & 81.00 & 106.34 & 61.28 & HT-Direct \\ 
  Namibia & OTJOZONDJUPA & 00-04 & 67.91 & 56.75 & 81.12 & RW2 \\ 
  Namibia & OTJOZONDJUPA & 05-09 & 52.05 & 76.85 & 34.96 & HT-Direct \\ 
  Namibia & OTJOZONDJUPA & 05-09 & 57.32 & 45.38 & 72.19 & RW2 \\ 
  Namibia & OTJOZONDJUPA & 10-14 & 41.29 & 73.38 & 22.89 & HT-Direct \\ 
  Namibia & OTJOZONDJUPA & 10-14 & 46.92 & 33.40 & 65.46 & RW2 \\ 
  Namibia & OTJOZONDJUPA & 15-19 & 38.65 & 14.27 & 100.95 & RW2 \\ 
  Niger & ALL & 80-84 & 313.59 & 309.88 & 317.33 & IHME \\ 
  Niger & ALL & 80-84 & 323.38 & 310.60 & 336.43 & RW2 \\ 
  Niger & ALL & 80-84 & 323.45 & 313.75 & 332.99 & UN \\ 
  Niger & ALL & 85-89 & 306.35 & 303.06 & 309.70 & IHME \\ 
  Niger & ALL & 85-89 & 335.58 & 324.87 & 346.43 & RW2 \\ 
  Niger & ALL & 85-89 & 335.45 & 325.61 & 344.98 & UN \\ 
  Niger & ALL & 90-94 & 285.11 & 281.84 & 288.73 & IHME \\ 
  Niger & ALL & 90-94 & 311.54 & 301.20 & 322.09 & RW2 \\ 
  Niger & ALL & 90-94 & 311.64 & 303.32 & 320.12 & UN \\ 
  Niger & ALL & 95-99 & 248.85 & 245.15 & 252.33 & IHME \\ 
  Niger & ALL & 95-99 & 255.79 & 246.55 & 265.17 & RW2 \\ 
  Niger & ALL & 95-99 & 255.87 & 248.96 & 263.81 & UN \\ 
  Niger & ALL & 00-04 & 203.87 & 200.76 & 207.10 & IHME \\ 
  Niger & ALL & 00-04 & 206.12 & 197.38 & 215.21 & RW2 \\ 
  Niger & ALL & 00-04 & 205.98 & 199.46 & 212.56 & UN \\ 
  Niger & ALL & 05-09 & 158.01 & 154.33 & 161.76 & IHME \\ 
  Niger & ALL & 05-09 & 151.01 & 142.43 & 160.00 & RW2 \\ 
  Niger & ALL & 05-09 & 151.08 & 144.77 & 157.85 & UN \\ 
  Niger & ALL & 10-14 & 127.02 & 121.93 & 131.92 & IHME \\ 
  Niger & ALL & 10-14 & 104.71 & 40.31 & 241.38 & RW2 \\ 
  Niger & ALL & 10-14 & 110.69 & 101.57 & 121.65 & UN \\ 
  Niger & DOSSO & 80-84 & 270.51 & 295.75 & 246.66 & HT-Direct \\ 
  Niger & DOSSO & 80-84 & 272.11 & 249.69 & 296.03 & RW2 \\ 
  Niger & DOSSO & 85-89 & 276.17 & 298.99 & 254.46 & HT-Direct \\ 
  Niger & DOSSO & 85-89 & 281.29 & 264.79 & 298.71 & RW2 \\ 
  Niger & DOSSO & 90-94 & 237.39 & 256.62 & 219.18 & HT-Direct \\ 
  Niger & DOSSO & 90-94 & 259.36 & 243.02 & 275.58 & RW2 \\ 
  Niger & DOSSO & 95-99 & 225.22 & 246.18 & 205.56 & HT-Direct \\ 
  Niger & DOSSO & 95-99 & 229.97 & 215.00 & 245.26 & RW2 \\ 
  Niger & DOSSO & 00-04 & 214.44 & 231.03 & 198.74 & HT-Direct \\ 
  Niger & DOSSO & 00-04 & 214.48 & 201.91 & 227.82 & RW2 \\ 
  Niger & DOSSO & 05-09 & 185.22 & 203.99 & 167.81 & HT-Direct \\ 
  Niger & DOSSO & 05-09 & 188.53 & 171.96 & 206.53 & RW2 \\ 
  Niger & DOSSO & 10-14 & 160.11 & 71.27 & 322.36 & RW2 \\ 
  Niger & DOSSO & 15-19 & 134.17 & 15.05 & 603.94 & RW2 \\ 
  Niger & MARADI & 80-84 & 369.76 & 393.72 & 346.42 & HT-Direct \\ 
  Niger & MARADI & 80-84 & 375.33 & 352.86 & 398.35 & RW2 \\ 
  Niger & MARADI & 85-89 & 375.22 & 395.04 & 355.80 & HT-Direct \\ 
  Niger & MARADI & 85-89 & 398.54 & 381.57 & 415.32 & RW2 \\ 
  Niger & MARADI & 90-94 & 376.13 & 395.11 & 357.51 & HT-Direct \\ 
  Niger & MARADI & 90-94 & 382.33 & 366.17 & 399.22 & RW2 \\ 
  Niger & MARADI & 95-99 & 305.75 & 324.28 & 287.82 & HT-Direct \\ 
  Niger & MARADI & 95-99 & 314.46 & 299.29 & 330.54 & RW2 \\ 
  Niger & MARADI & 00-04 & 236.53 & 254.91 & 219.08 & HT-Direct \\ 
  Niger & MARADI & 00-04 & 237.16 & 223.37 & 251.57 & RW2 \\ 
  Niger & MARADI & 05-09 & 144.75 & 161.41 & 129.55 & HT-Direct \\ 
  Niger & MARADI & 05-09 & 153.00 & 137.78 & 169.23 & RW2 \\ 
  Niger & MARADI & 10-14 & 90.32 & 37.86 & 197.77 & RW2 \\ 
  Niger & MARADI & 15-19 & 51.11 & 5.17 & 351.10 & RW2 \\ 
  Niger & NIAMEY & 80-84 & 152.87 & 174.93 & 133.14 & HT-Direct \\ 
  Niger & NIAMEY & 80-84 & 150.60 & 132.52 & 170.68 & RW2 \\ 
  Niger & NIAMEY & 85-89 & 150.39 & 168.75 & 133.71 & HT-Direct \\ 
  Niger & NIAMEY & 85-89 & 163.96 & 149.81 & 178.84 & RW2 \\ 
  Niger & NIAMEY & 90-94 & 151.82 & 171.61 & 133.94 & HT-Direct \\ 
  Niger & NIAMEY & 90-94 & 156.24 & 142.28 & 171.19 & RW2 \\ 
  Niger & NIAMEY & 95-99 & 136.59 & 156.34 & 118.98 & HT-Direct \\ 
  Niger & NIAMEY & 95-99 & 137.68 & 124.62 & 152.15 & RW2 \\ 
  Niger & NIAMEY & 00-04 & 123.33 & 144.00 & 105.27 & HT-Direct \\ 
  Niger & NIAMEY & 00-04 & 122.21 & 108.74 & 137.27 & RW2 \\ 
  Niger & NIAMEY & 05-09 & 88.35 & 116.13 & 66.71 & HT-Direct \\ 
  Niger & NIAMEY & 05-09 & 98.41 & 78.89 & 121.94 & RW2 \\ 
  Niger & NIAMEY & 10-14 & 76.00 & 29.03 & 181.98 & RW2 \\ 
  Niger & NIAMEY & 15-19 & 58.40 & 5.33 & 405.34 & RW2 \\ 
  Niger & TASHOUA/AGADEZ & 80-84 & 320.93 & 347.19 & 295.75 & HT-Direct \\ 
  Niger & TASHOUA/AGADEZ & 80-84 & 328.74 & 305.67 & 352.76 & RW2 \\ 
  Niger & TASHOUA/AGADEZ & 85-89 & 322.83 & 346.89 & 299.67 & HT-Direct \\ 
  Niger & TASHOUA/AGADEZ & 85-89 & 334.35 & 316.88 & 352.25 & RW2 \\ 
  Niger & TASHOUA/AGADEZ & 90-94 & 294.34 & 318.36 & 271.42 & HT-Direct \\ 
  Niger & TASHOUA/AGADEZ & 90-94 & 301.01 & 284.76 & 318.01 & RW2 \\ 
  Niger & TASHOUA/AGADEZ & 95-99 & 230.70 & 248.34 & 213.96 & HT-Direct \\ 
  Niger & TASHOUA/AGADEZ & 95-99 & 242.58 & 228.70 & 256.81 & RW2 \\ 
  Niger & TASHOUA/AGADEZ & 00-04 & 193.26 & 217.28 & 171.32 & HT-Direct \\ 
  Niger & TASHOUA/AGADEZ & 00-04 & 193.51 & 179.50 & 208.45 & RW2 \\ 
  Niger & TASHOUA/AGADEZ & 05-09 & 136.33 & 155.42 & 119.25 & HT-Direct \\ 
  Niger & TASHOUA/AGADEZ & 05-09 & 139.86 & 124.65 & 156.53 & RW2 \\ 
  Niger & TASHOUA/AGADEZ & 10-14 & 95.44 & 40.48 & 208.86 & RW2 \\ 
  Niger & TASHOUA/AGADEZ & 15-19 & 64.34 & 6.69 & 403.03 & RW2 \\ 
  Niger & TILLABERI & 80-84 & 247.86 & 275.21 & 222.39 & HT-Direct \\ 
  Niger & TILLABERI & 80-84 & 276.19 & 251.62 & 301.68 & RW2 \\ 
  Niger & TILLABERI & 85-89 & 313.85 & 336.75 & 291.82 & HT-Direct \\ 
  Niger & TILLABERI & 85-89 & 292.60 & 275.76 & 310.81 & RW2 \\ 
  Niger & TILLABERI & 90-94 & 235.83 & 256.34 & 216.48 & HT-Direct \\ 
  Niger & TILLABERI & 90-94 & 263.56 & 247.50 & 279.90 & RW2 \\ 
  Niger & TILLABERI & 95-99 & 210.20 & 234.65 & 187.67 & HT-Direct \\ 
  Niger & TILLABERI & 95-99 & 219.95 & 204.28 & 235.88 & RW2 \\ 
  Niger & TILLABERI & 00-04 & 187.99 & 204.92 & 172.16 & HT-Direct \\ 
  Niger & TILLABERI & 00-04 & 189.53 & 176.87 & 202.94 & RW2 \\ 
  Niger & TILLABERI & 05-09 & 152.63 & 174.92 & 132.73 & HT-Direct \\ 
  Niger & TILLABERI & 05-09 & 152.42 & 135.48 & 171.32 & RW2 \\ 
  Niger & TILLABERI & 10-14 & 117.59 & 50.01 & 251.08 & RW2 \\ 
  Niger & TILLABERI & 15-19 & 89.90 & 9.66 & 492.96 & RW2 \\ 
  Niger & ZINDA/DIFFA & 80-84 & 360.69 & 392.59 & 329.98 & HT-Direct \\ 
  Niger & ZINDA/DIFFA & 80-84 & 368.98 & 340.90 & 398.49 & RW2 \\ 
  Niger & ZINDA/DIFFA & 85-89 & 367.41 & 392.18 & 343.33 & HT-Direct \\ 
  Niger & ZINDA/DIFFA & 85-89 & 376.26 & 357.19 & 395.77 & RW2 \\ 
  Niger & ZINDA/DIFFA & 90-94 & 329.65 & 352.07 & 307.99 & HT-Direct \\ 
  Niger & ZINDA/DIFFA & 90-94 & 344.43 & 326.77 & 362.58 & RW2 \\ 
  Niger & ZINDA/DIFFA & 95-99 & 263.60 & 284.43 & 243.78 & HT-Direct \\ 
  Niger & ZINDA/DIFFA & 95-99 & 279.07 & 262.87 & 295.55 & RW2 \\ 
  Niger & ZINDA/DIFFA & 00-04 & 223.15 & 245.37 & 202.41 & HT-Direct \\ 
  Niger & ZINDA/DIFFA & 00-04 & 219.19 & 204.06 & 235.18 & RW2 \\ 
  Niger & ZINDA/DIFFA & 05-09 & 148.38 & 168.86 & 130.00 & HT-Direct \\ 
  Niger & ZINDA/DIFFA & 05-09 & 152.73 & 135.91 & 171.28 & RW2 \\ 
  Niger & ZINDA/DIFFA & 10-14 & 99.14 & 41.55 & 217.74 & RW2 \\ 
  Niger & ZINDA/DIFFA & 15-19 & 63.22 & 6.40 & 400.32 & RW2 \\ 
  Nigeria & ALL & 80-84 & 218.11 & 214.92 & 221.28 & IHME \\ 
  Nigeria & ALL & 80-84 & 210.64 & 200.57 & 221.08 & RW2 \\ 
  Nigeria & ALL & 80-84 & 210.74 & 203.42 & 218.05 & UN \\ 
  Nigeria & ALL & 85-89 & 210.74 & 207.86 & 213.88 & IHME \\ 
  Nigeria & ALL & 85-89 & 211.64 & 203.72 & 219.76 & RW2 \\ 
  Nigeria & ALL & 85-89 & 211.46 & 205.53 & 217.77 & UN \\ 
  Nigeria & ALL & 90-94 & 204.18 & 201.54 & 206.97 & IHME \\ 
  Nigeria & ALL & 90-94 & 211.08 & 203.29 & 219.09 & RW2 \\ 
  Nigeria & ALL & 90-94 & 211.30 & 205.57 & 217.09 & UN \\ 
  Nigeria & ALL & 95-99 & 192.15 & 189.68 & 194.71 & IHME \\ 
  Nigeria & ALL & 95-99 & 200.57 & 194.44 & 206.80 & RW2 \\ 
  Nigeria & ALL & 95-99 & 200.45 & 194.94 & 205.96 & UN \\ 
  Nigeria & ALL & 00-04 & 175.49 & 172.92 & 178.00 & IHME \\ 
  Nigeria & ALL & 00-04 & 175.20 & 169.96 & 180.59 & RW2 \\ 
  Nigeria & ALL & 00-04 & 175.32 & 170.33 & 180.06 & UN \\ 
  Nigeria & ALL & 05-09 & 149.69 & 146.71 & 152.37 & IHME \\ 
  Nigeria & ALL & 05-09 & 146.59 & 141.58 & 151.74 & RW2 \\ 
  Nigeria & ALL & 05-09 & 146.49 & 141.68 & 151.76 & UN \\ 
  Nigeria & ALL & 10-14 & 120.69 & 116.52 & 124.95 & IHME \\ 
  Nigeria & ALL & 10-14 & 120.69 & 113.21 & 128.53 & RW2 \\ 
  Nigeria & ALL & 10-14 & 120.81 & 112.42 & 130.51 & UN \\ 
  Nigeria & NORTH CENTRAL & 80-84 & 169.54 & 194.77 & 146.98 & HT-Direct \\ 
  Nigeria & NORTH CENTRAL & 80-84 & 166.20 & 148.97 & 185.36 & RW2 \\ 
  Nigeria & NORTH CENTRAL & 85-89 & 159.27 & 175.08 & 144.64 & HT-Direct \\ 
  Nigeria & NORTH CENTRAL & 85-89 & 159.12 & 148.43 & 170.30 & RW2 \\ 
  Nigeria & NORTH CENTRAL & 90-94 & 149.02 & 164.20 & 135.02 & HT-Direct \\ 
  Nigeria & NORTH CENTRAL & 90-94 & 149.42 & 140.36 & 158.93 & RW2 \\ 
  Nigeria & NORTH CENTRAL & 95-99 & 145.76 & 157.89 & 134.41 & HT-Direct \\ 
  Nigeria & NORTH CENTRAL & 95-99 & 142.03 & 134.28 & 150.02 & RW2 \\ 
  Nigeria & NORTH CENTRAL & 00-04 & 123.67 & 134.21 & 113.85 & HT-Direct \\ 
  Nigeria & NORTH CENTRAL & 00-04 & 127.37 & 120.50 & 134.57 & RW2 \\ 
  Nigeria & NORTH CENTRAL & 05-09 & 121.60 & 133.53 & 110.60 & HT-Direct \\ 
  Nigeria & NORTH CENTRAL & 05-09 & 109.80 & 102.75 & 117.41 & RW2 \\ 
  Nigeria & NORTH CENTRAL & 10-14 & 82.33 & 94.25 & 71.80 & HT-Direct \\ 
  Nigeria & NORTH CENTRAL & 10-14 & 89.29 & 79.92 & 99.71 & RW2 \\ 
  Nigeria & NORTH CENTRAL & 15-19 & 71.04 & 33.09 & 144.88 & RW2 \\ 
  Nigeria & NORTH EAST & 80-84 & 253.01 & 274.84 & 232.36 & HT-Direct \\ 
  Nigeria & NORTH EAST & 80-84 & 254.96 & 236.08 & 274.61 & RW2 \\ 
  Nigeria & NORTH EAST & 85-89 & 270.05 & 289.94 & 251.04 & HT-Direct \\ 
  Nigeria & NORTH EAST & 85-89 & 265.23 & 251.31 & 279.63 & RW2 \\ 
  Nigeria & NORTH EAST & 90-94 & 255.21 & 272.22 & 238.92 & HT-Direct \\ 
  Nigeria & NORTH EAST & 90-94 & 263.60 & 251.02 & 276.45 & RW2 \\ 
  Nigeria & NORTH EAST & 95-99 & 261.63 & 274.25 & 249.39 & HT-Direct \\ 
  Nigeria & NORTH EAST & 95-99 & 253.81 & 244.14 & 263.88 & RW2 \\ 
  Nigeria & NORTH EAST & 00-04 & 227.90 & 239.77 & 216.44 & HT-Direct \\ 
  Nigeria & NORTH EAST & 00-04 & 218.86 & 210.13 & 228.06 & RW2 \\ 
  Nigeria & NORTH EAST & 05-09 & 172.43 & 184.84 & 160.69 & HT-Direct \\ 
  Nigeria & NORTH EAST & 05-09 & 173.99 & 164.55 & 183.73 & RW2 \\ 
  Nigeria & NORTH EAST & 10-14 & 134.37 & 154.03 & 116.87 & HT-Direct \\ 
  Nigeria & NORTH EAST & 10-14 & 133.35 & 118.82 & 149.33 & RW2 \\ 
  Nigeria & NORTH EAST & 15-19 & 100.16 & 47.14 & 199.66 & RW2 \\ 
  Nigeria & NORTH WEST & 80-84 & 279.66 & 300.69 & 259.56 & HT-Direct \\ 
  Nigeria & NORTH WEST & 80-84 & 285.94 & 266.87 & 305.61 & RW2 \\ 
  Nigeria & NORTH WEST & 85-89 & 302.35 & 319.22 & 285.98 & HT-Direct \\ 
  Nigeria & NORTH WEST & 85-89 & 299.24 & 286.24 & 312.77 & RW2 \\ 
  Nigeria & NORTH WEST & 90-94 & 301.66 & 318.89 & 284.97 & HT-Direct \\ 
  Nigeria & NORTH WEST & 90-94 & 293.41 & 280.54 & 307.03 & RW2 \\ 
  Nigeria & NORTH WEST & 95-99 & 273.90 & 287.43 & 260.78 & HT-Direct \\ 
  Nigeria & NORTH WEST & 95-99 & 270.01 & 259.80 & 280.56 & RW2 \\ 
  Nigeria & NORTH WEST & 00-04 & 229.44 & 241.20 & 218.08 & HT-Direct \\ 
  Nigeria & NORTH WEST & 00-04 & 227.94 & 218.73 & 237.25 & RW2 \\ 
  Nigeria & NORTH WEST & 05-09 & 193.05 & 204.21 & 182.36 & HT-Direct \\ 
  Nigeria & NORTH WEST & 05-09 & 186.68 & 178.24 & 195.41 & RW2 \\ 
  Nigeria & NORTH WEST & 10-14 & 145.66 & 159.50 & 132.83 & HT-Direct \\ 
  Nigeria & NORTH WEST & 10-14 & 148.72 & 136.64 & 161.77 & RW2 \\ 
  Nigeria & NORTH WEST & 15-19 & 116.12 & 55.93 & 225.60 & RW2 \\ 
  Nigeria & SOUTH EAST & 80-84 & 132.17 & 153.81 & 113.16 & HT-Direct \\ 
  Nigeria & SOUTH EAST & 80-84 & 133.83 & 117.67 & 152.11 & RW2 \\ 
  Nigeria & SOUTH EAST & 85-89 & 141.31 & 158.64 & 125.59 & HT-Direct \\ 
  Nigeria & SOUTH EAST & 85-89 & 137.33 & 126.43 & 148.92 & RW2 \\ 
  Nigeria & SOUTH EAST & 90-94 & 138.12 & 152.75 & 124.69 & HT-Direct \\ 
  Nigeria & SOUTH EAST & 90-94 & 139.52 & 129.77 & 149.82 & RW2 \\ 
  Nigeria & SOUTH EAST & 95-99 & 144.42 & 159.61 & 130.45 & HT-Direct \\ 
  Nigeria & SOUTH EAST & 95-99 & 145.87 & 136.26 & 155.87 & RW2 \\ 
  Nigeria & SOUTH EAST & 00-04 & 152.62 & 166.41 & 139.78 & HT-Direct \\ 
  Nigeria & SOUTH EAST & 00-04 & 144.92 & 136.09 & 154.42 & RW2 \\ 
  Nigeria & SOUTH EAST & 05-09 & 136.46 & 150.87 & 123.22 & HT-Direct \\ 
  Nigeria & SOUTH EAST & 05-09 & 134.75 & 124.73 & 145.38 & RW2 \\ 
  Nigeria & SOUTH EAST & 10-14 & 115.56 & 142.17 & 93.39 & HT-Direct \\ 
  Nigeria & SOUTH EAST & 10-14 & 117.76 & 100.44 & 137.33 & RW2 \\ 
  Nigeria & SOUTH EAST & 15-19 & 100.99 & 46.03 & 205.26 & RW2 \\ 
  Nigeria & SOUTH SOUTH & 80-84 & 147.74 & 169.07 & 128.69 & HT-Direct \\ 
  Nigeria & SOUTH SOUTH & 80-84 & 141.78 & 126.17 & 159.21 & RW2 \\ 
  Nigeria & SOUTH SOUTH & 85-89 & 126.75 & 143.78 & 111.48 & HT-Direct \\ 
  Nigeria & SOUTH SOUTH & 85-89 & 136.24 & 125.33 & 147.79 & RW2 \\ 
  Nigeria & SOUTH SOUTH & 90-94 & 133.12 & 147.38 & 120.05 & HT-Direct \\ 
  Nigeria & SOUTH SOUTH & 90-94 & 132.17 & 122.92 & 141.68 & RW2 \\ 
  Nigeria & SOUTH SOUTH & 95-99 & 134.25 & 147.92 & 121.67 & HT-Direct \\ 
  Nigeria & SOUTH SOUTH & 95-99 & 132.40 & 124.07 & 141.31 & RW2 \\ 
  Nigeria & SOUTH SOUTH & 00-04 & 131.85 & 143.64 & 120.89 & HT-Direct \\ 
  Nigeria & SOUTH SOUTH & 00-04 & 124.34 & 116.86 & 132.52 & RW2 \\ 
  Nigeria & SOUTH SOUTH & 05-09 & 109.43 & 121.73 & 98.23 & HT-Direct \\ 
  Nigeria & SOUTH SOUTH & 05-09 & 107.62 & 99.78 & 116.00 & RW2 \\ 
  Nigeria & SOUTH SOUTH & 10-14 & 80.06 & 93.99 & 68.04 & HT-Direct \\ 
  Nigeria & SOUTH SOUTH & 10-14 & 86.29 & 75.12 & 98.50 & RW2 \\ 
  Nigeria & SOUTH SOUTH & 15-19 & 67.36 & 30.93 & 139.27 & RW2 \\ 
  Nigeria & SOUTH WEST & 80-84 & 152.94 & 180.09 & 129.24 & HT-Direct \\ 
  Nigeria & SOUTH WEST & 80-84 & 151.87 & 132.13 & 174.91 & RW2 \\ 
  Nigeria & SOUTH WEST & 85-89 & 139.67 & 158.09 & 123.09 & HT-Direct \\ 
  Nigeria & SOUTH WEST & 85-89 & 131.22 & 120.17 & 143.29 & RW2 \\ 
  Nigeria & SOUTH WEST & 90-94 & 105.02 & 118.55 & 92.87 & HT-Direct \\ 
  Nigeria & SOUTH WEST & 90-94 & 113.96 & 104.97 & 123.33 & RW2 \\ 
  Nigeria & SOUTH WEST & 95-99 & 109.77 & 121.01 & 99.46 & HT-Direct \\ 
  Nigeria & SOUTH WEST & 95-99 & 106.23 & 98.81 & 113.89 & RW2 \\ 
  Nigeria & SOUTH WEST & 00-04 & 97.15 & 107.19 & 87.97 & HT-Direct \\ 
  Nigeria & SOUTH WEST & 00-04 & 97.26 & 90.73 & 104.17 & RW2 \\ 
  Nigeria & SOUTH WEST & 05-09 & 91.64 & 102.42 & 81.90 & HT-Direct \\ 
  Nigeria & SOUTH WEST & 05-09 & 86.97 & 79.80 & 94.78 & RW2 \\ 
  Nigeria & SOUTH WEST & 10-14 & 73.40 & 93.53 & 57.33 & HT-Direct \\ 
  Nigeria & SOUTH WEST & 10-14 & 73.92 & 62.15 & 88.19 & RW2 \\ 
  Nigeria & SOUTH WEST & 15-19 & 61.79 & 27.42 & 132.06 & RW2 \\ 
  Rwanda & ALL & 80-84 & 183.05 & 178.72 & 187.12 & IHME \\ 
  Rwanda & ALL & 80-84 & 188.89 & 175.97 & 202.53 & RW2 \\ 
  Rwanda & ALL & 80-84 & 188.63 & 182.96 & 193.81 & UN \\ 
  Rwanda & ALL & 85-89 & 156.43 & 152.45 & 159.97 & IHME \\ 
  Rwanda & ALL & 85-89 & 152.46 & 144.80 & 160.41 & RW2 \\ 
  Rwanda & ALL & 85-89 & 153.07 & 148.61 & 157.43 & UN \\ 
  Rwanda & ALL & 90-94 & 187.82 & 163.45 & 216.87 & IHME \\ 
  Rwanda & ALL & 90-94 & 183.25 & 177.74 & 188.89 & RW2 \\ 
  Rwanda & ALL & 90-94 & 182.69 & 176.51 & 189.45 & UN \\ 
  Rwanda & ALL & 95-99 & 176.38 & 171.82 & 180.55 & IHME \\ 
  Rwanda & ALL & 95-99 & 224.40 & 216.90 & 232.04 & RW2 \\ 
  Rwanda & ALL & 95-99 & 225.09 & 212.46 & 236.58 & UN \\ 
  Rwanda & ALL & 00-04 & 137.76 & 133.65 & 141.35 & IHME \\ 
  Rwanda & ALL & 00-04 & 154.22 & 148.40 & 160.24 & RW2 \\ 
  Rwanda & ALL & 00-04 & 153.90 & 149.44 & 158.95 & UN \\ 
  Rwanda & ALL & 05-09 & 86.74 & 83.70 & 89.98 & IHME \\ 
  Rwanda & ALL & 05-09 & 88.66 & 83.90 & 93.66 & RW2 \\ 
  Rwanda & ALL & 05-09 & 88.79 & 85.42 & 92.16 & UN \\ 
  Rwanda & ALL & 10-14 & 71.42 & 66.66 & 76.65 & IHME \\ 
  Rwanda & ALL & 10-14 & 53.39 & 47.04 & 60.49 & RW2 \\ 
  Rwanda & ALL & 10-14 & 53.34 & 49.12 & 57.82 & UN \\ 
  Rwanda & EAST & 80-84 & 181.85 & 210.81 & 156.09 & HT-Direct \\ 
  Rwanda & EAST & 80-84 & 183.93 & 160.50 & 210.53 & RW2 \\ 
  Rwanda & EAST & 85-89 & 163.34 & 179.61 & 148.28 & HT-Direct \\ 
  Rwanda & EAST & 85-89 & 160.32 & 148.41 & 173.05 & RW2 \\ 
  Rwanda & EAST & 90-94 & 236.73 & 252.10 & 222.02 & HT-Direct \\ 
  Rwanda & EAST & 90-94 & 205.65 & 195.50 & 216.05 & RW2 \\ 
  Rwanda & EAST & 95-99 & 249.38 & 264.70 & 234.66 & HT-Direct \\ 
  Rwanda & EAST & 95-99 & 261.20 & 247.79 & 274.60 & RW2 \\ 
  Rwanda & EAST & 00-04 & 202.32 & 215.71 & 189.57 & HT-Direct \\ 
  Rwanda & EAST & 00-04 & 198.54 & 187.33 & 210.89 & RW2 \\ 
  Rwanda & EAST & 05-09 & 108.27 & 118.56 & 98.77 & HT-Direct \\ 
  Rwanda & EAST & 05-09 & 117.21 & 107.88 & 127.10 & RW2 \\ 
  Rwanda & EAST & 10-14 & 60.79 & 74.03 & 49.79 & HT-Direct \\ 
  Rwanda & EAST & 10-14 & 67.50 & 55.53 & 81.22 & RW2 \\ 
  Rwanda & EAST & 15-19 & 38.44 & 12.23 & 112.22 & RW2 \\ 
  Rwanda & KIGALI & 80-84 & 116.56 & 151.21 & 89.01 & HT-Direct \\ 
  Rwanda & KIGALI & 80-84 & 127.18 & 100.77 & 157.85 & RW2 \\ 
  Rwanda & KIGALI & 85-89 & 109.49 & 133.06 & 89.67 & HT-Direct \\ 
  Rwanda & KIGALI & 85-89 & 111.23 & 97.52 & 126.95 & RW2 \\ 
  Rwanda & KIGALI & 90-94 & 174.20 & 198.31 & 152.46 & HT-Direct \\ 
  Rwanda & KIGALI & 90-94 & 137.94 & 125.59 & 152.13 & RW2 \\ 
  Rwanda & KIGALI & 95-99 & 135.05 & 152.81 & 119.07 & HT-Direct \\ 
  Rwanda & KIGALI & 95-99 & 156.54 & 141.39 & 171.67 & RW2 \\ 
  Rwanda & KIGALI & 00-04 & 108.13 & 127.26 & 91.57 & HT-Direct \\ 
  Rwanda & KIGALI & 00-04 & 104.51 & 93.09 & 116.87 & RW2 \\ 
  Rwanda & KIGALI & 05-09 & 56.50 & 68.32 & 46.62 & HT-Direct \\ 
  Rwanda & KIGALI & 05-09 & 59.23 & 50.92 & 68.98 & RW2 \\ 
  Rwanda & KIGALI & 10-14 & 31.42 & 44.99 & 21.85 & HT-Direct \\ 
  Rwanda & KIGALI & 10-14 & 34.75 & 25.95 & 46.74 & RW2 \\ 
  Rwanda & KIGALI & 15-19 & 20.52 & 6.15 & 66.80 & RW2 \\ 
  Rwanda & NORTH & 80-84 & 211.41 & 247.33 & 179.46 & HT-Direct \\ 
  Rwanda & NORTH & 80-84 & 202.24 & 175.19 & 234.28 & RW2 \\ 
  Rwanda & NORTH & 85-89 & 148.97 & 166.40 & 133.07 & HT-Direct \\ 
  Rwanda & NORTH & 85-89 & 152.04 & 139.51 & 165.18 & RW2 \\ 
  Rwanda & NORTH & 90-94 & 203.16 & 219.86 & 187.42 & HT-Direct \\ 
  Rwanda & NORTH & 90-94 & 179.05 & 168.64 & 189.69 & RW2 \\ 
  Rwanda & NORTH & 95-99 & 219.69 & 236.44 & 203.81 & HT-Direct \\ 
  Rwanda & NORTH & 95-99 & 213.14 & 200.60 & 226.91 & RW2 \\ 
  Rwanda & NORTH & 00-04 & 127.54 & 140.62 & 115.52 & HT-Direct \\ 
  Rwanda & NORTH & 00-04 & 140.73 & 130.36 & 151.61 & RW2 \\ 
  Rwanda & NORTH & 05-09 & 81.73 & 92.47 & 72.13 & HT-Direct \\ 
  Rwanda & NORTH & 05-09 & 79.17 & 71.18 & 88.09 & RW2 \\ 
  Rwanda & NORTH & 10-14 & 35.60 & 47.97 & 26.33 & HT-Direct \\ 
  Rwanda & NORTH & 10-14 & 45.08 & 35.44 & 56.94 & RW2 \\ 
  Rwanda & NORTH & 15-19 & 25.17 & 7.80 & 77.46 & RW2 \\ 
  Rwanda & SOUTH & 80-84 & 162.21 & 181.70 & 144.45 & HT-Direct \\ 
  Rwanda & SOUTH & 80-84 & 173.41 & 154.23 & 194.05 & RW2 \\ 
  Rwanda & SOUTH & 85-89 & 162.75 & 179.84 & 146.98 & HT-Direct \\ 
  Rwanda & SOUTH & 85-89 & 158.42 & 146.74 & 171.15 & RW2 \\ 
  Rwanda & SOUTH & 90-94 & 236.13 & 251.10 & 221.78 & HT-Direct \\ 
  Rwanda & SOUTH & 90-94 & 198.28 & 188.36 & 208.91 & RW2 \\ 
  Rwanda & SOUTH & 95-99 & 209.25 & 222.49 & 196.60 & HT-Direct \\ 
  Rwanda & SOUTH & 95-99 & 225.35 & 213.52 & 237.10 & RW2 \\ 
  Rwanda & SOUTH & 00-04 & 152.05 & 163.39 & 141.36 & HT-Direct \\ 
  Rwanda & SOUTH & 00-04 & 149.14 & 140.13 & 158.50 & RW2 \\ 
  Rwanda & SOUTH & 05-09 & 72.34 & 80.77 & 64.72 & HT-Direct \\ 
  Rwanda & SOUTH & 05-09 & 82.45 & 74.99 & 90.60 & RW2 \\ 
  Rwanda & SOUTH & 10-14 & 52.69 & 66.76 & 41.46 & HT-Direct \\ 
  Rwanda & SOUTH & 10-14 & 49.37 & 40.20 & 61.16 & RW2 \\ 
  Rwanda & SOUTH & 15-19 & 30.22 & 9.55 & 91.48 & RW2 \\ 
  Rwanda & WEST & 80-84 & 201.90 & 227.39 & 178.60 & HT-Direct \\ 
  Rwanda & WEST & 80-84 & 201.09 & 178.07 & 226.61 & RW2 \\ 
  Rwanda & WEST & 85-89 & 144.45 & 160.95 & 129.38 & HT-Direct \\ 
  Rwanda & WEST & 85-89 & 143.21 & 131.15 & 155.76 & RW2 \\ 
  Rwanda & WEST & 90-94 & 186.78 & 201.47 & 172.94 & HT-Direct \\ 
  Rwanda & WEST & 90-94 & 166.39 & 156.23 & 176.59 & RW2 \\ 
  Rwanda & WEST & 95-99 & 211.09 & 225.58 & 197.29 & HT-Direct \\ 
  Rwanda & WEST & 95-99 & 207.40 & 195.23 & 221.24 & RW2 \\ 
  Rwanda & WEST & 00-04 & 132.96 & 144.04 & 122.62 & HT-Direct \\ 
  Rwanda & WEST & 00-04 & 139.67 & 130.25 & 149.46 & RW2 \\ 
  Rwanda & WEST & 05-09 & 79.71 & 88.47 & 71.74 & HT-Direct \\ 
  Rwanda & WEST & 05-09 & 83.18 & 75.59 & 91.46 & RW2 \\ 
  Rwanda & WEST & 10-14 & 49.89 & 64.21 & 38.63 & HT-Direct \\ 
  Rwanda & WEST & 10-14 & 53.68 & 42.91 & 67.61 & RW2 \\ 
  Rwanda & WEST & 15-19 & 34.80 & 10.79 & 106.65 & RW2 \\ 
  Senegal & ALL & 80-84 & 182.52 & 180.58 & 184.68 & IHME \\ 
  Senegal & ALL & 80-84 & 192.12 & 183.26 & 201.29 & RW2 \\ 
  Senegal & ALL & 80-84 & 192.10 & 187.98 & 196.67 & UN \\ 
  Senegal & ALL & 85-89 & 155.77 & 154.15 & 157.31 & IHME \\ 
  Senegal & ALL & 85-89 & 157.11 & 150.44 & 163.99 & RW2 \\ 
  Senegal & ALL & 85-89 & 157.15 & 153.53 & 160.55 & UN \\ 
  Senegal & ALL & 90-94 & 139.81 & 138.06 & 141.32 & IHME \\ 
  Senegal & ALL & 90-94 & 139.54 & 134.03 & 145.23 & RW2 \\ 
  Senegal & ALL & 90-94 & 139.52 & 136.45 & 142.90 & UN \\ 
  Senegal & ALL & 95-99 & 128.38 & 126.78 & 129.94 & IHME \\ 
  Senegal & ALL & 95-99 & 142.37 & 136.89 & 148.01 & RW2 \\ 
  Senegal & ALL & 95-99 & 142.32 & 138.72 & 145.75 & UN \\ 
  Senegal & ALL & 00-04 & 107.37 & 106.02 & 108.76 & IHME \\ 
  Senegal & ALL & 00-04 & 119.97 & 115.18 & 124.97 & RW2 \\ 
  Senegal & ALL & 00-04 & 120.05 & 116.36 & 123.45 & UN \\ 
  Senegal & ALL & 05-09 & 81.94 & 80.70 & 83.26 & IHME \\ 
  Senegal & ALL & 05-09 & 82.31 & 78.33 & 86.47 & RW2 \\ 
  Senegal & ALL & 05-09 & 82.31 & 79.05 & 85.76 & UN \\ 
  Senegal & ALL & 10-14 & 61.65 & 59.91 & 63.20 & IHME \\ 
  Senegal & ALL & 10-14 & 56.69 & 52.88 & 60.73 & RW2 \\ 
  Senegal & ALL & 10-14 & 56.68 & 51.79 & 62.09 & UN \\ 
  Senegal & DAKAR & 80-84 & 130.04 & 150.57 & 111.95 & HT-Direct \\ 
  Senegal & DAKAR & 80-84 & 132.79 & 116.22 & 151.53 & RW2 \\ 
  Senegal & DAKAR & 85-89 & 106.71 & 122.05 & 93.09 & HT-Direct \\ 
  Senegal & DAKAR & 85-89 & 109.60 & 99.28 & 120.97 & RW2 \\ 
  Senegal & DAKAR & 90-94 & 94.71 & 108.77 & 82.31 & HT-Direct \\ 
  Senegal & DAKAR & 90-94 & 94.47 & 85.42 & 104.55 & RW2 \\ 
  Senegal & DAKAR & 95-99 & 92.34 & 109.25 & 77.82 & HT-Direct \\ 
  Senegal & DAKAR & 95-99 & 96.15 & 86.12 & 106.86 & RW2 \\ 
  Senegal & DAKAR & 00-04 & 68.79 & 80.44 & 58.72 & HT-Direct \\ 
  Senegal & DAKAR & 00-04 & 82.87 & 73.82 & 92.55 & RW2 \\ 
  Senegal & DAKAR & 05-09 & 60.75 & 71.50 & 51.52 & HT-Direct \\ 
  Senegal & DAKAR & 05-09 & 57.68 & 50.70 & 65.67 & RW2 \\ 
  Senegal & DAKAR & 10-14 & 38.15 & 50.98 & 28.46 & HT-Direct \\ 
  Senegal & DAKAR & 10-14 & 41.98 & 33.50 & 52.56 & RW2 \\ 
  Senegal & DAKAR & 15-19 & 31.52 & 12.15 & 78.43 & RW2 \\ 
  Senegal & DIOURBEL & 80-84 & 236.15 & 262.57 & 211.62 & HT-Direct \\ 
  Senegal & DIOURBEL & 80-84 & 227.48 & 206.68 & 250.16 & RW2 \\ 
  Senegal & DIOURBEL & 85-89 & 171.17 & 188.97 & 154.73 & HT-Direct \\ 
  Senegal & DIOURBEL & 85-89 & 185.31 & 172.20 & 198.81 & RW2 \\ 
  Senegal & DIOURBEL & 90-94 & 152.41 & 169.02 & 137.16 & HT-Direct \\ 
  Senegal & DIOURBEL & 90-94 & 161.28 & 150.10 & 172.68 & RW2 \\ 
  Senegal & DIOURBEL & 95-99 & 168.65 & 184.77 & 153.66 & HT-Direct \\ 
  Senegal & DIOURBEL & 95-99 & 168.94 & 158.04 & 180.43 & RW2 \\ 
  Senegal & DIOURBEL & 00-04 & 142.76 & 158.26 & 128.54 & HT-Direct \\ 
  Senegal & DIOURBEL & 00-04 & 148.89 & 138.38 & 160.03 & RW2 \\ 
  Senegal & DIOURBEL & 05-09 & 92.69 & 103.97 & 82.52 & HT-Direct \\ 
  Senegal & DIOURBEL & 05-09 & 103.31 & 94.70 & 112.68 & RW2 \\ 
  Senegal & DIOURBEL & 10-14 & 78.40 & 92.43 & 66.34 & HT-Direct \\ 
  Senegal & DIOURBEL & 10-14 & 76.73 & 67.05 & 87.85 & RW2 \\ 
  Senegal & DIOURBEL & 15-19 & 59.26 & 25.40 & 133.82 & RW2 \\ 
  Senegal & FATICK & 80-84 & 209.94 & 236.79 & 185.40 & HT-Direct \\ 
  Senegal & FATICK & 80-84 & 210.72 & 189.52 & 233.11 & RW2 \\ 
  Senegal & FATICK & 85-89 & 160.18 & 181.64 & 140.82 & HT-Direct \\ 
  Senegal & FATICK & 85-89 & 173.90 & 160.05 & 187.99 & RW2 \\ 
  Senegal & FATICK & 90-94 & 148.13 & 164.58 & 133.07 & HT-Direct \\ 
  Senegal & FATICK & 90-94 & 150.71 & 140.27 & 161.86 & RW2 \\ 
  Senegal & FATICK & 95-99 & 153.21 & 170.33 & 137.53 & HT-Direct \\ 
  Senegal & FATICK & 95-99 & 153.33 & 142.79 & 165.04 & RW2 \\ 
  Senegal & FATICK & 00-04 & 123.84 & 138.37 & 110.64 & HT-Direct \\ 
  Senegal & FATICK & 00-04 & 128.18 & 118.54 & 138.72 & RW2 \\ 
  Senegal & FATICK & 05-09 & 69.35 & 81.04 & 59.24 & HT-Direct \\ 
  Senegal & FATICK & 05-09 & 82.25 & 74.01 & 91.21 & RW2 \\ 
  Senegal & FATICK & 10-14 & 56.96 & 70.94 & 45.59 & HT-Direct \\ 
  Senegal & FATICK & 10-14 & 55.97 & 47.19 & 65.81 & RW2 \\ 
  Senegal & FATICK & 15-19 & 39.43 & 16.46 & 92.27 & RW2 \\ 
  Senegal & KAOLACK & 80-84 & 217.72 & 244.39 & 193.22 & HT-Direct \\ 
  Senegal & KAOLACK & 80-84 & 209.29 & 189.28 & 231.19 & RW2 \\ 
  Senegal & KAOLACK & 85-89 & 159.66 & 175.79 & 144.75 & HT-Direct \\ 
  Senegal & KAOLACK & 85-89 & 175.68 & 163.48 & 188.36 & RW2 \\ 
  Senegal & KAOLACK & 90-94 & 149.94 & 163.79 & 137.07 & HT-Direct \\ 
  Senegal & KAOLACK & 90-94 & 155.12 & 145.50 & 165.05 & RW2 \\ 
  Senegal & KAOLACK & 95-99 & 157.77 & 170.41 & 145.90 & HT-Direct \\ 
  Senegal & KAOLACK & 95-99 & 161.49 & 152.48 & 171.16 & RW2 \\ 
  Senegal & KAOLACK & 00-04 & 131.01 & 141.06 & 121.58 & HT-Direct \\ 
  Senegal & KAOLACK & 00-04 & 138.30 & 130.37 & 147.04 & RW2 \\ 
  Senegal & KAOLACK & 05-09 & 85.24 & 91.98 & 78.95 & HT-Direct \\ 
  Senegal & KAOLACK & 05-09 & 89.09 & 83.39 & 95.10 & RW2 \\ 
  Senegal & KAOLACK & 10-14 & 53.93 & 61.78 & 47.03 & HT-Direct \\ 
  Senegal & KAOLACK & 10-14 & 58.83 & 52.06 & 65.95 & RW2 \\ 
  Senegal & KAOLACK & 15-19 & 40.08 & 17.01 & 91.12 & RW2 \\ 
  Senegal & KOLDA & 80-84 & 247.03 & 275.21 & 220.85 & HT-Direct \\ 
  Senegal & KOLDA & 80-84 & 252.15 & 228.88 & 276.71 & RW2 \\ 
  Senegal & KOLDA & 85-89 & 224.04 & 242.06 & 206.99 & HT-Direct \\ 
  Senegal & KOLDA & 85-89 & 225.14 & 211.10 & 239.75 & RW2 \\ 
  Senegal & KOLDA & 90-94 & 188.44 & 205.14 & 172.80 & HT-Direct \\ 
  Senegal & KOLDA & 90-94 & 203.05 & 190.66 & 215.75 & RW2 \\ 
  Senegal & KOLDA & 95-99 & 209.79 & 224.41 & 195.87 & HT-Direct \\ 
  Senegal & KOLDA & 95-99 & 214.57 & 203.17 & 226.35 & RW2 \\ 
  Senegal & KOLDA & 00-04 & 180.67 & 193.60 & 168.42 & HT-Direct \\ 
  Senegal & KOLDA & 00-04 & 189.80 & 179.05 & 201.22 & RW2 \\ 
  Senegal & KOLDA & 05-09 & 115.91 & 127.45 & 105.28 & HT-Direct \\ 
  Senegal & KOLDA & 05-09 & 128.97 & 119.48 & 139.09 & RW2 \\ 
  Senegal & KOLDA & 10-14 & 95.21 & 108.59 & 83.32 & HT-Direct \\ 
  Senegal & KOLDA & 10-14 & 93.85 & 83.52 & 105.36 & RW2 \\ 
  Senegal & KOLDA & 15-19 & 71.32 & 30.48 & 160.04 & RW2 \\ 
  Senegal & LOUGA & 80-84 & 199.32 & 227.80 & 173.59 & HT-Direct \\ 
  Senegal & LOUGA & 80-84 & 194.62 & 173.50 & 218.22 & RW2 \\ 
  Senegal & LOUGA & 85-89 & 152.39 & 174.65 & 132.51 & HT-Direct \\ 
  Senegal & LOUGA & 85-89 & 153.63 & 140.91 & 167.35 & RW2 \\ 
  Senegal & LOUGA & 90-94 & 119.16 & 135.43 & 104.60 & HT-Direct \\ 
  Senegal & LOUGA & 90-94 & 124.77 & 114.96 & 134.91 & RW2 \\ 
  Senegal & LOUGA & 95-99 & 118.54 & 135.93 & 103.10 & HT-Direct \\ 
  Senegal & LOUGA & 95-99 & 121.45 & 110.89 & 132.04 & RW2 \\ 
  Senegal & LOUGA & 00-04 & 82.62 & 95.00 & 71.72 & HT-Direct \\ 
  Senegal & LOUGA & 00-04 & 101.64 & 92.24 & 111.38 & RW2 \\ 
  Senegal & LOUGA & 05-09 & 59.03 & 71.27 & 48.78 & HT-Direct \\ 
  Senegal & LOUGA & 05-09 & 69.30 & 62.21 & 77.05 & RW2 \\ 
  Senegal & LOUGA & 10-14 & 61.26 & 73.01 & 51.30 & HT-Direct \\ 
  Senegal & LOUGA & 10-14 & 52.40 & 45.00 & 61.35 & RW2 \\ 
  Senegal & LOUGA & 15-19 & 41.67 & 17.70 & 96.51 & RW2 \\ 
  Senegal & MATAM & 80-84 & 201.37 & 233.88 & 172.36 & HT-Direct \\ 
  Senegal & MATAM & 80-84 & 228.04 & 200.38 & 255.59 & RW2 \\ 
  Senegal & MATAM & 85-89 & 194.80 & 212.37 & 178.35 & HT-Direct \\ 
  Senegal & MATAM & 85-89 & 192.64 & 179.12 & 207.27 & RW2 \\ 
  Senegal & MATAM & 90-94 & 159.03 & 178.99 & 140.91 & HT-Direct \\ 
  Senegal & MATAM & 90-94 & 157.47 & 145.81 & 170.81 & RW2 \\ 
  Senegal & MATAM & 95-99 & 136.28 & 155.16 & 119.38 & HT-Direct \\ 
  Senegal & MATAM & 95-99 & 145.67 & 134.22 & 157.82 & RW2 \\ 
  Senegal & MATAM & 00-04 & 107.19 & 121.34 & 94.51 & HT-Direct \\ 
  Senegal & MATAM & 00-04 & 114.60 & 104.39 & 125.10 & RW2 \\ 
  Senegal & MATAM & 05-09 & 60.24 & 71.25 & 50.84 & HT-Direct \\ 
  Senegal & MATAM & 05-09 & 72.73 & 64.90 & 81.31 & RW2 \\ 
  Senegal & MATAM & 10-14 & 56.86 & 71.00 & 45.39 & HT-Direct \\ 
  Senegal & MATAM & 10-14 & 51.48 & 43.30 & 61.51 & RW2 \\ 
  Senegal & MATAM & 15-19 & 38.45 & 16.01 & 90.88 & RW2 \\ 
  Senegal & SAINT-LOUIS & 80-84 & 192.57 & 221.99 & 166.22 & HT-Direct \\ 
  Senegal & SAINT-LOUIS & 80-84 & 205.73 & 181.84 & 231.75 & RW2 \\ 
  Senegal & SAINT-LOUIS & 85-89 & 176.11 & 199.47 & 154.95 & HT-Direct \\ 
  Senegal & SAINT-LOUIS & 85-89 & 162.46 & 148.07 & 178.50 & RW2 \\ 
  Senegal & SAINT-LOUIS & 90-94 & 118.09 & 135.36 & 102.75 & HT-Direct \\ 
  Senegal & SAINT-LOUIS & 90-94 & 124.61 & 113.82 & 136.38 & RW2 \\ 
  Senegal & SAINT-LOUIS & 95-99 & 98.69 & 114.09 & 85.18 & HT-Direct \\ 
  Senegal & SAINT-LOUIS & 95-99 & 112.73 & 101.87 & 124.00 & RW2 \\ 
  Senegal & SAINT-LOUIS & 00-04 & 86.69 & 100.09 & 74.93 & HT-Direct \\ 
  Senegal & SAINT-LOUIS & 00-04 & 90.53 & 81.23 & 100.36 & RW2 \\ 
  Senegal & SAINT-LOUIS & 05-09 & 57.03 & 69.04 & 47.00 & HT-Direct \\ 
  Senegal & SAINT-LOUIS & 05-09 & 60.16 & 52.78 & 68.56 & RW2 \\ 
  Senegal & SAINT-LOUIS & 10-14 & 46.09 & 60.77 & 34.83 & HT-Direct \\ 
  Senegal & SAINT-LOUIS & 10-14 & 44.86 & 36.16 & 56.25 & RW2 \\ 
  Senegal & SAINT-LOUIS & 15-19 & 35.51 & 14.29 & 86.98 & RW2 \\ 
  Senegal & TAMBACOUNDA & 80-84 & 212.46 & 249.52 & 179.59 & HT-Direct \\ 
  Senegal & TAMBACOUNDA & 80-84 & 223.42 & 195.51 & 252.55 & RW2 \\ 
  Senegal & TAMBACOUNDA & 85-89 & 198.62 & 225.39 & 174.32 & HT-Direct \\ 
  Senegal & TAMBACOUNDA & 85-89 & 206.24 & 189.77 & 223.70 & RW2 \\ 
  Senegal & TAMBACOUNDA & 90-94 & 189.97 & 208.21 & 172.98 & HT-Direct \\ 
  Senegal & TAMBACOUNDA & 90-94 & 191.84 & 179.44 & 205.17 & RW2 \\ 
  Senegal & TAMBACOUNDA & 95-99 & 201.03 & 221.68 & 181.86 & HT-Direct \\ 
  Senegal & TAMBACOUNDA & 95-99 & 203.63 & 191.27 & 216.85 & RW2 \\ 
  Senegal & TAMBACOUNDA & 00-04 & 167.25 & 181.35 & 154.05 & HT-Direct \\ 
  Senegal & TAMBACOUNDA & 00-04 & 180.39 & 169.50 & 191.96 & RW2 \\ 
  Senegal & TAMBACOUNDA & 05-09 & 116.59 & 129.97 & 104.43 & HT-Direct \\ 
  Senegal & TAMBACOUNDA & 05-09 & 125.08 & 115.59 & 135.22 & RW2 \\ 
  Senegal & TAMBACOUNDA & 10-14 & 90.71 & 104.72 & 78.42 & HT-Direct \\ 
  Senegal & TAMBACOUNDA & 10-14 & 92.31 & 81.82 & 103.80 & RW2 \\ 
  Senegal & TAMBACOUNDA & 15-19 & 70.67 & 30.65 & 155.70 & RW2 \\ 
  Senegal & THIES & 80-84 & 165.76 & 192.82 & 141.84 & HT-Direct \\ 
  Senegal & THIES & 80-84 & 159.01 & 139.86 & 180.45 & RW2 \\ 
  Senegal & THIES & 85-89 & 107.64 & 126.01 & 91.66 & HT-Direct \\ 
  Senegal & THIES & 85-89 & 125.57 & 113.88 & 138.21 & RW2 \\ 
  Senegal & THIES & 90-94 & 102.53 & 116.56 & 90.02 & HT-Direct \\ 
  Senegal & THIES & 90-94 & 105.30 & 96.66 & 114.54 & RW2 \\ 
  Senegal & THIES & 95-99 & 108.17 & 123.43 & 94.59 & HT-Direct \\ 
  Senegal & THIES & 95-99 & 105.58 & 97.00 & 115.03 & RW2 \\ 
  Senegal & THIES & 00-04 & 76.93 & 88.51 & 66.75 & HT-Direct \\ 
  Senegal & THIES & 00-04 & 87.46 & 79.57 & 96.14 & RW2 \\ 
  Senegal & THIES & 05-09 & 60.40 & 72.47 & 50.24 & HT-Direct \\ 
  Senegal & THIES & 05-09 & 56.43 & 49.84 & 63.91 & RW2 \\ 
  Senegal & THIES & 10-14 & 31.22 & 41.39 & 23.49 & HT-Direct \\ 
  Senegal & THIES & 10-14 & 38.47 & 31.43 & 46.69 & RW2 \\ 
  Senegal & THIES & 15-19 & 27.17 & 11.06 & 64.87 & RW2 \\ 
  Senegal & ZUGUINCHOR & 80-84 & 215.24 & 258.63 & 177.38 & HT-Direct \\ 
  Senegal & ZUGUINCHOR & 80-84 & 210.13 & 178.39 & 246.09 & RW2 \\ 
  Senegal & ZUGUINCHOR & 85-89 & 155.69 & 183.78 & 131.21 & HT-Direct \\ 
  Senegal & ZUGUINCHOR & 85-89 & 173.96 & 155.40 & 194.10 & RW2 \\ 
  Senegal & ZUGUINCHOR & 90-94 & 146.86 & 167.08 & 128.72 & HT-Direct \\ 
  Senegal & ZUGUINCHOR & 90-94 & 147.17 & 133.75 & 161.66 & RW2 \\ 
  Senegal & ZUGUINCHOR & 95-99 & 142.88 & 160.92 & 126.55 & HT-Direct \\ 
  Senegal & ZUGUINCHOR & 95-99 & 145.68 & 133.27 & 159.23 & RW2 \\ 
  Senegal & ZUGUINCHOR & 00-04 & 108.17 & 124.43 & 93.80 & HT-Direct \\ 
  Senegal & ZUGUINCHOR & 00-04 & 117.46 & 105.73 & 130.36 & RW2 \\ 
  Senegal & ZUGUINCHOR & 05-09 & 64.92 & 80.18 & 52.40 & HT-Direct \\ 
  Senegal & ZUGUINCHOR & 05-09 & 71.20 & 61.30 & 82.60 & RW2 \\ 
  Senegal & ZUGUINCHOR & 10-14 & 46.67 & 65.19 & 33.22 & HT-Direct \\ 
  Senegal & ZUGUINCHOR & 10-14 & 46.79 & 36.15 & 60.32 & RW2 \\ 
  Senegal & ZUGUINCHOR & 15-19 & 32.37 & 12.32 & 83.21 & RW2 \\ 
  Sierra Leone & ALL & 80-84 & 270.31 & 265.06 & 276.10 & IHME \\ 
  Sierra Leone & ALL & 80-84 & 277.79 & 231.92 & 328.82 & RW2 \\ 
  Sierra Leone & ALL & 80-84 & 281.41 & 268.54 & 294.89 & UN \\ 
  Sierra Leone & ALL & 85-89 & 256.50 & 252.65 & 260.16 & IHME \\ 
  Sierra Leone & ALL & 85-89 & 271.75 & 244.82 & 300.40 & RW2 \\ 
  Sierra Leone & ALL & 85-89 & 268.02 & 258.06 & 278.78 & UN \\ 
  Sierra Leone & ALL & 90-94 & 238.75 & 235.46 & 241.96 & IHME \\ 
  Sierra Leone & ALL & 90-94 & 261.95 & 242.13 & 282.77 & RW2 \\ 
  Sierra Leone & ALL & 90-94 & 262.55 & 253.37 & 271.50 & UN \\ 
  Sierra Leone & ALL & 95-99 & 221.12 & 218.35 & 224.35 & IHME \\ 
  Sierra Leone & ALL & 95-99 & 249.36 & 231.48 & 268.06 & RW2 \\ 
  Sierra Leone & ALL & 95-99 & 250.10 & 242.01 & 258.61 & UN \\ 
  Sierra Leone & ALL & 00-04 & 198.59 & 196.17 & 201.45 & IHME \\ 
  Sierra Leone & ALL & 00-04 & 223.42 & 210.05 & 237.42 & RW2 \\ 
  Sierra Leone & ALL & 00-04 & 223.39 & 216.57 & 230.81 & UN \\ 
  Sierra Leone & ALL & 05-09 & 168.80 & 165.82 & 171.71 & IHME \\ 
  Sierra Leone & ALL & 05-09 & 187.88 & 178.30 & 197.86 & RW2 \\ 
  Sierra Leone & ALL & 05-09 & 187.57 & 180.45 & 194.59 & UN \\ 
  Sierra Leone & ALL & 10-14 & 138.76 & 135.08 & 142.62 & IHME \\ 
  Sierra Leone & ALL & 10-14 & 142.03 & 130.56 & 154.24 & RW2 \\ 
  Sierra Leone & ALL & 10-14 & 142.26 & 134.04 & 150.47 & UN \\ 
  Sierra Leone & EASTERN & 80-84 & 328.65 & 414.01 & 253.28 & HT-Direct \\ 
  Sierra Leone & EASTERN & 80-84 & 322.49 & 259.38 & 391.71 & RW2 \\ 
  Sierra Leone & EASTERN & 85-89 & 337.21 & 415.51 & 266.94 & HT-Direct \\ 
  Sierra Leone & EASTERN & 85-89 & 324.56 & 282.23 & 370.03 & RW2 \\ 
  Sierra Leone & EASTERN & 90-94 & 303.14 & 338.63 & 269.86 & HT-Direct \\ 
  Sierra Leone & EASTERN & 90-94 & 319.95 & 290.98 & 350.38 & RW2 \\ 
  Sierra Leone & EASTERN & 95-99 & 285.39 & 318.36 & 254.56 & HT-Direct \\ 
  Sierra Leone & EASTERN & 95-99 & 303.36 & 277.27 & 331.01 & RW2 \\ 
  Sierra Leone & EASTERN & 00-04 & 253.79 & 281.40 & 228.03 & HT-Direct \\ 
  Sierra Leone & EASTERN & 00-04 & 265.75 & 244.82 & 288.18 & RW2 \\ 
  Sierra Leone & EASTERN & 05-09 & 222.90 & 247.65 & 199.97 & HT-Direct \\ 
  Sierra Leone & EASTERN & 05-09 & 214.07 & 196.72 & 232.58 & RW2 \\ 
  Sierra Leone & EASTERN & 10-14 & 139.96 & 161.80 & 120.64 & HT-Direct \\ 
  Sierra Leone & EASTERN & 10-14 & 154.52 & 133.82 & 177.30 & RW2 \\ 
  Sierra Leone & EASTERN & 15-19 & 105.86 & 46.08 & 223.52 & RW2 \\ 
  Sierra Leone & NORTHERN & 80-84 & 172.52 & 239.84 & 121.08 & HT-Direct \\ 
  Sierra Leone & NORTHERN & 80-84 & 205.16 & 156.77 & 261.83 & RW2 \\ 
  Sierra Leone & NORTHERN & 85-89 & 251.41 & 307.73 & 202.38 & HT-Direct \\ 
  Sierra Leone & NORTHERN & 85-89 & 223.12 & 191.99 & 257.35 & RW2 \\ 
  Sierra Leone & NORTHERN & 90-94 & 221.01 & 251.41 & 193.34 & HT-Direct \\ 
  Sierra Leone & NORTHERN & 90-94 & 230.58 & 207.44 & 255.92 & RW2 \\ 
  Sierra Leone & NORTHERN & 95-99 & 214.10 & 240.19 & 190.13 & HT-Direct \\ 
  Sierra Leone & NORTHERN & 95-99 & 225.61 & 205.22 & 247.67 & RW2 \\ 
  Sierra Leone & NORTHERN & 00-04 & 190.87 & 213.22 & 170.35 & HT-Direct \\ 
  Sierra Leone & NORTHERN & 00-04 & 205.66 & 189.03 & 223.37 & RW2 \\ 
  Sierra Leone & NORTHERN & 05-09 & 183.47 & 199.37 & 168.57 & HT-Direct \\ 
  Sierra Leone & NORTHERN & 05-09 & 176.48 & 164.40 & 189.29 & RW2 \\ 
  Sierra Leone & NORTHERN & 10-14 & 124.95 & 141.90 & 109.77 & HT-Direct \\ 
  Sierra Leone & NORTHERN & 10-14 & 138.08 & 121.87 & 155.92 & RW2 \\ 
  Sierra Leone & NORTHERN & 15-19 & 103.85 & 46.29 & 216.34 & RW2 \\ 
  Sierra Leone & SOUTHERN & 80-84 & 363.86 & 490.77 & 253.43 & HT-Direct \\ 
  Sierra Leone & SOUTHERN & 80-84 & 328.15 & 253.86 & 411.73 & RW2 \\ 
  Sierra Leone & SOUTHERN & 85-89 & 311.62 & 379.99 & 250.58 & HT-Direct \\ 
  Sierra Leone & SOUTHERN & 85-89 & 317.25 & 274.63 & 362.03 & RW2 \\ 
  Sierra Leone & SOUTHERN & 90-94 & 291.24 & 328.34 & 256.73 & HT-Direct \\ 
  Sierra Leone & SOUTHERN & 90-94 & 302.78 & 274.29 & 332.66 & RW2 \\ 
  Sierra Leone & SOUTHERN & 95-99 & 251.08 & 278.09 & 225.87 & HT-Direct \\ 
  Sierra Leone & SOUTHERN & 95-99 & 281.74 & 258.59 & 306.34 & RW2 \\ 
  Sierra Leone & SOUTHERN & 00-04 & 258.84 & 286.52 & 232.96 & HT-Direct \\ 
  Sierra Leone & SOUTHERN & 00-04 & 243.86 & 224.97 & 264.87 & RW2 \\ 
  Sierra Leone & SOUTHERN & 05-09 & 186.96 & 203.68 & 171.32 & HT-Direct \\ 
  Sierra Leone & SOUTHERN & 05-09 & 189.42 & 176.28 & 203.42 & RW2 \\ 
  Sierra Leone & SOUTHERN & 10-14 & 125.42 & 144.05 & 108.90 & HT-Direct \\ 
  Sierra Leone & SOUTHERN & 10-14 & 133.93 & 116.97 & 152.52 & RW2 \\ 
  Sierra Leone & SOUTHERN & 15-19 & 90.86 & 39.57 & 193.15 & RW2 \\ 
  Sierra Leone & WESTERN & 80-84 & 318.37 & 506.39 & 175.35 & HT-Direct \\ 
  Sierra Leone & WESTERN & 80-84 & 244.26 & 158.87 & 370.67 & RW2 \\ 
  Sierra Leone & WESTERN & 85-89 & 250.80 & 346.08 & 174.75 & HT-Direct \\ 
  Sierra Leone & WESTERN & 85-89 & 217.07 & 168.09 & 277.45 & RW2 \\ 
  Sierra Leone & WESTERN & 90-94 & 172.34 & 251.84 & 114.11 & HT-Direct \\ 
  Sierra Leone & WESTERN & 90-94 & 191.67 & 152.50 & 234.99 & RW2 \\ 
  Sierra Leone & WESTERN & 95-99 & 123.36 & 173.41 & 86.26 & HT-Direct \\ 
  Sierra Leone & WESTERN & 95-99 & 175.31 & 138.35 & 213.90 & RW2 \\ 
  Sierra Leone & WESTERN & 00-04 & 152.96 & 193.32 & 119.78 & HT-Direct \\ 
  Sierra Leone & WESTERN & 00-04 & 165.93 & 138.02 & 195.69 & RW2 \\ 
  Sierra Leone & WESTERN & 05-09 & 165.60 & 201.26 & 135.18 & HT-Direct \\ 
  Sierra Leone & WESTERN & 05-09 & 158.17 & 136.82 & 182.13 & RW2 \\ 
  Sierra Leone & WESTERN & 10-14 & 142.23 & 180.87 & 110.72 & HT-Direct \\ 
  Sierra Leone & WESTERN & 10-14 & 144.12 & 113.92 & 183.57 & RW2 \\ 
  Sierra Leone & WESTERN & 15-19 & 128.75 & 53.45 & 284.57 & RW2 \\ 
  Tanzania & ALL & 80-84 & 171.36 & 169.32 & 173.54 & IHME \\ 
  Tanzania & ALL & 80-84 & 179.37 & 167.40 & 192.00 & RW2 \\ 
  Tanzania & ALL & 80-84 & 179.27 & 174.32 & 184.40 & UN \\ 
  Tanzania & ALL & 85-89 & 161.41 & 159.64 & 163.27 & IHME \\ 
  Tanzania & ALL & 85-89 & 171.57 & 162.25 & 181.27 & RW2 \\ 
  Tanzania & ALL & 85-89 & 171.80 & 167.30 & 176.39 & UN \\ 
  Tanzania & ALL & 90-94 & 150.46 & 148.53 & 152.19 & IHME \\ 
  Tanzania & ALL & 90-94 & 162.73 & 155.68 & 170.03 & RW2 \\ 
  Tanzania & ALL & 90-94 & 162.57 & 158.29 & 166.92 & UN \\ 
  Tanzania & ALL & 95-99 & 134.95 & 133.30 & 136.87 & IHME \\ 
  Tanzania & ALL & 95-99 & 149.01 & 142.48 & 155.75 & RW2 \\ 
  Tanzania & ALL & 95-99 & 149.00 & 144.89 & 152.97 & UN \\ 
  Tanzania & ALL & 00-04 & 110.67 & 108.67 & 112.55 & IHME \\ 
  Tanzania & ALL & 00-04 & 114.16 & 108.43 & 120.18 & RW2 \\ 
  Tanzania & ALL & 00-04 & 114.33 & 110.51 & 118.20 & UN \\ 
  Tanzania & ALL & 05-09 & 87.55 & 85.11 & 89.87 & IHME \\ 
  Tanzania & ALL & 05-09 & 79.67 & 74.92 & 84.69 & RW2 \\ 
  Tanzania & ALL & 05-09 & 79.59 & 75.50 & 84.15 & UN \\ 
  Tanzania & ALL & 10-14 & 69.44 & 66.37 & 72.79 & IHME \\ 
  Tanzania & ALL & 10-14 & 56.63 & 52.16 & 61.43 & RW2 \\ 
  Tanzania & ALL & 10-14 & 56.64 & 50.70 & 63.71 & UN \\ 
  Tanzania & ARUSHA & 80-84 & 93.90 & 124.83 & 70.02 & HT-Direct \\ 
  Tanzania & ARUSHA & 80-84 & 109.75 & 87.20 & 137.40 & RW2 \\ 
  Tanzania & ARUSHA & 85-89 & 103.58 & 136.04 & 78.17 & HT-Direct \\ 
  Tanzania & ARUSHA & 85-89 & 104.30 & 88.79 & 122.15 & RW2 \\ 
  Tanzania & ARUSHA & 90-94 & 97.58 & 121.12 & 78.21 & HT-Direct \\ 
  Tanzania & ARUSHA & 90-94 & 97.11 & 85.27 & 110.45 & RW2 \\ 
  Tanzania & ARUSHA & 95-99 & 86.27 & 106.67 & 69.47 & HT-Direct \\ 
  Tanzania & ARUSHA & 95-99 & 87.40 & 77.15 & 98.84 & RW2 \\ 
  Tanzania & ARUSHA & 00-04 & 56.77 & 71.58 & 44.87 & HT-Direct \\ 
  Tanzania & ARUSHA & 00-04 & 65.75 & 57.52 & 75.08 & RW2 \\ 
  Tanzania & ARUSHA & 05-09 & 57.41 & 75.99 & 43.16 & HT-Direct \\ 
  Tanzania & ARUSHA & 05-09 & 43.77 & 37.06 & 51.75 & RW2 \\ 
  Tanzania & ARUSHA & 10-14 & 27.76 & 43.03 & 17.81 & HT-Direct \\ 
  Tanzania & ARUSHA & 10-14 & 29.57 & 22.82 & 38.12 & RW2 \\ 
  Tanzania & ARUSHA & 15-19 & 20.21 & 8.29 & 48.48 & RW2 \\ 
  Tanzania & DAR ES SALAAM & 80-84 & 183.66 & 223.51 & 149.54 & HT-Direct \\ 
  Tanzania & DAR ES SALAAM & 80-84 & 199.32 & 164.11 & 240.00 & RW2 \\ 
  Tanzania & DAR ES SALAAM & 85-89 & 194.14 & 235.18 & 158.77 & HT-Direct \\ 
  Tanzania & DAR ES SALAAM & 85-89 & 175.21 & 151.66 & 202.78 & RW2 \\ 
  Tanzania & DAR ES SALAAM & 90-94 & 127.47 & 155.26 & 104.04 & HT-Direct \\ 
  Tanzania & DAR ES SALAAM & 90-94 & 141.76 & 122.63 & 162.60 & RW2 \\ 
  Tanzania & DAR ES SALAAM & 95-99 & 117.17 & 144.47 & 94.46 & HT-Direct \\ 
  Tanzania & DAR ES SALAAM & 95-99 & 123.94 & 106.22 & 143.51 & RW2 \\ 
  Tanzania & DAR ES SALAAM & 00-04 & 107.46 & 141.23 & 81.01 & HT-Direct \\ 
  Tanzania & DAR ES SALAAM & 00-04 & 105.29 & 88.73 & 124.39 & RW2 \\ 
  Tanzania & DAR ES SALAAM & 05-09 & 102.43 & 135.42 & 76.77 & HT-Direct \\ 
  Tanzania & DAR ES SALAAM & 05-09 & 86.90 & 73.42 & 102.98 & RW2 \\ 
  Tanzania & DAR ES SALAAM & 10-14 & 91.88 & 114.80 & 73.17 & HT-Direct \\ 
  Tanzania & DAR ES SALAAM & 10-14 & 74.88 & 62.54 & 89.42 & RW2 \\ 
  Tanzania & DAR ES SALAAM & 15-19 & 65.16 & 26.87 & 152.11 & RW2 \\ 
  Tanzania & DODOMA & 80-84 & 183.08 & 233.02 & 141.86 & HT-Direct \\ 
  Tanzania & DODOMA & 80-84 & 230.37 & 189.48 & 276.22 & RW2 \\ 
  Tanzania & DODOMA & 85-89 & 240.05 & 280.97 & 203.41 & HT-Direct \\ 
  Tanzania & DODOMA & 85-89 & 225.18 & 199.16 & 254.00 & RW2 \\ 
  Tanzania & DODOMA & 90-94 & 225.95 & 273.26 & 184.74 & HT-Direct \\ 
  Tanzania & DODOMA & 90-94 & 211.45 & 189.07 & 236.33 & RW2 \\ 
  Tanzania & DODOMA & 95-99 & 187.08 & 221.43 & 156.99 & HT-Direct \\ 
  Tanzania & DODOMA & 95-99 & 185.60 & 165.77 & 207.35 & RW2 \\ 
  Tanzania & DODOMA & 00-04 & 119.29 & 147.69 & 95.74 & HT-Direct \\ 
  Tanzania & DODOMA & 00-04 & 134.97 & 118.29 & 153.29 & RW2 \\ 
  Tanzania & DODOMA & 05-09 & 91.26 & 123.75 & 66.64 & HT-Direct \\ 
  Tanzania & DODOMA & 05-09 & 89.20 & 75.50 & 105.14 & RW2 \\ 
  Tanzania & DODOMA & 10-14 & 92.59 & 136.52 & 61.78 & HT-Direct \\ 
  Tanzania & DODOMA & 10-14 & 61.17 & 47.16 & 79.02 & RW2 \\ 
  Tanzania & DODOMA & 15-19 & 42.32 & 17.31 & 99.64 & RW2 \\ 
  Tanzania & IRINGA & 80-84 & 226.04 & 267.96 & 188.99 & HT-Direct \\ 
  Tanzania & IRINGA & 80-84 & 218.04 & 184.75 & 256.35 & RW2 \\ 
  Tanzania & IRINGA & 85-89 & 156.36 & 201.94 & 119.52 & HT-Direct \\ 
  Tanzania & IRINGA & 85-89 & 191.20 & 168.28 & 216.02 & RW2 \\ 
  Tanzania & IRINGA & 90-94 & 174.93 & 204.04 & 149.20 & HT-Direct \\ 
  Tanzania & IRINGA & 90-94 & 170.88 & 152.72 & 190.14 & RW2 \\ 
  Tanzania & IRINGA & 95-99 & 126.82 & 154.57 & 103.44 & HT-Direct \\ 
  Tanzania & IRINGA & 95-99 & 149.78 & 133.59 & 166.93 & RW2 \\ 
  Tanzania & IRINGA & 00-04 & 108.16 & 132.12 & 88.10 & HT-Direct \\ 
  Tanzania & IRINGA & 00-04 & 111.44 & 99.07 & 125.22 & RW2 \\ 
  Tanzania & IRINGA & 05-09 & 98.20 & 119.20 & 80.56 & HT-Direct \\ 
  Tanzania & IRINGA & 05-09 & 74.43 & 64.96 & 85.15 & RW2 \\ 
  Tanzania & IRINGA & 10-14 & 48.49 & 71.36 & 32.69 & HT-Direct \\ 
  Tanzania & IRINGA & 10-14 & 49.85 & 39.76 & 62.51 & RW2 \\ 
  Tanzania & IRINGA & 15-19 & 33.37 & 13.80 & 78.08 & RW2 \\ 
  Tanzania & KAGERA & 80-84 & 203.38 & 253.49 & 161.04 & HT-Direct \\ 
  Tanzania & KAGERA & 80-84 & 207.34 & 170.86 & 249.16 & RW2 \\ 
  Tanzania & KAGERA & 85-89 & 181.10 & 217.29 & 149.78 & HT-Direct \\ 
  Tanzania & KAGERA & 85-89 & 194.37 & 171.28 & 219.78 & RW2 \\ 
  Tanzania & KAGERA & 90-94 & 173.29 & 204.11 & 146.28 & HT-Direct \\ 
  Tanzania & KAGERA & 90-94 & 182.71 & 164.59 & 202.50 & RW2 \\ 
  Tanzania & KAGERA & 95-99 & 178.56 & 205.23 & 154.68 & HT-Direct \\ 
  Tanzania & KAGERA & 95-99 & 168.98 & 152.69 & 186.73 & RW2 \\ 
  Tanzania & KAGERA & 00-04 & 126.88 & 156.97 & 101.87 & HT-Direct \\ 
  Tanzania & KAGERA & 00-04 & 131.47 & 116.34 & 148.75 & RW2 \\ 
  Tanzania & KAGERA & 05-09 & 95.49 & 125.77 & 71.90 & HT-Direct \\ 
  Tanzania & KAGERA & 05-09 & 88.45 & 74.99 & 104.14 & RW2 \\ 
  Tanzania & KAGERA & 10-14 & 73.16 & 115.66 & 45.47 & HT-Direct \\ 
  Tanzania & KAGERA & 10-14 & 59.42 & 44.95 & 77.82 & RW2 \\ 
  Tanzania & KAGERA & 15-19 & 40.08 & 15.99 & 97.72 & RW2 \\ 
  Tanzania & KIGOMA & 80-84 & 175.48 & 238.27 & 126.48 & HT-Direct \\ 
  Tanzania & KIGOMA & 80-84 & 207.22 & 164.67 & 256.53 & RW2 \\ 
  Tanzania & KIGOMA & 85-89 & 187.69 & 224.26 & 155.88 & HT-Direct \\ 
  Tanzania & KIGOMA & 85-89 & 192.80 & 168.20 & 220.18 & RW2 \\ 
  Tanzania & KIGOMA & 90-94 & 187.56 & 221.91 & 157.45 & HT-Direct \\ 
  Tanzania & KIGOMA & 90-94 & 178.54 & 159.24 & 199.79 & RW2 \\ 
  Tanzania & KIGOMA & 95-99 & 137.33 & 171.51 & 109.07 & HT-Direct \\ 
  Tanzania & KIGOMA & 95-99 & 158.12 & 140.28 & 178.21 & RW2 \\ 
  Tanzania & KIGOMA & 00-04 & 123.11 & 149.15 & 101.08 & HT-Direct \\ 
  Tanzania & KIGOMA & 00-04 & 117.09 & 102.64 & 133.35 & RW2 \\ 
  Tanzania & KIGOMA & 05-09 & 91.19 & 122.40 & 67.33 & HT-Direct \\ 
  Tanzania & KIGOMA & 05-09 & 76.34 & 63.84 & 91.04 & RW2 \\ 
  Tanzania & KIGOMA & 10-14 & 47.76 & 82.77 & 27.12 & HT-Direct \\ 
  Tanzania & KIGOMA & 10-14 & 50.64 & 37.29 & 68.00 & RW2 \\ 
  Tanzania & KIGOMA & 15-19 & 33.95 & 13.24 & 83.70 & RW2 \\ 
  Tanzania & KILIMANJARO & 80-84 & 119.76 & 155.32 & 91.45 & HT-Direct \\ 
  Tanzania & KILIMANJARO & 80-84 & 107.24 & 83.79 & 137.03 & RW2 \\ 
  Tanzania & KILIMANJARO & 85-89 & 75.89 & 97.84 & 58.54 & HT-Direct \\ 
  Tanzania & KILIMANJARO & 85-89 & 87.98 & 74.19 & 104.11 & RW2 \\ 
  Tanzania & KILIMANJARO & 90-94 & 71.65 & 91.95 & 55.56 & HT-Direct \\ 
  Tanzania & KILIMANJARO & 90-94 & 74.21 & 62.13 & 87.58 & RW2 \\ 
  Tanzania & KILIMANJARO & 95-99 & 48.75 & 68.41 & 34.52 & HT-Direct \\ 
  Tanzania & KILIMANJARO & 95-99 & 66.94 & 54.85 & 80.49 & RW2 \\ 
  Tanzania & KILIMANJARO & 00-04 & 71.32 & 98.50 & 51.22 & HT-Direct \\ 
  Tanzania & KILIMANJARO & 00-04 & 56.03 & 45.21 & 69.27 & RW2 \\ 
  Tanzania & KILIMANJARO & 05-09 & 41.55 & 77.52 & 21.88 & HT-Direct \\ 
  Tanzania & KILIMANJARO & 05-09 & 43.41 & 32.06 & 58.84 & RW2 \\ 
  Tanzania & KILIMANJARO & 10-14 & 49.22 & 108.66 & 21.51 & HT-Direct \\ 
  Tanzania & KILIMANJARO & 10-14 & 34.77 & 21.72 & 56.64 & RW2 \\ 
  Tanzania & KILIMANJARO & 15-19 & 28.48 & 9.56 & 84.70 & RW2 \\ 
  Tanzania & LINDI & 80-84 & 228.32 & 270.63 & 190.89 & HT-Direct \\ 
  Tanzania & LINDI & 80-84 & 247.58 & 210.16 & 288.06 & RW2 \\ 
  Tanzania & LINDI & 85-89 & 241.50 & 289.26 & 199.42 & HT-Direct \\ 
  Tanzania & LINDI & 85-89 & 255.11 & 227.49 & 284.76 & RW2 \\ 
  Tanzania & LINDI & 90-94 & 257.03 & 299.28 & 218.88 & HT-Direct \\ 
  Tanzania & LINDI & 90-94 & 251.15 & 227.45 & 276.77 & RW2 \\ 
  Tanzania & LINDI & 95-99 & 228.34 & 260.03 & 199.47 & HT-Direct \\ 
  Tanzania & LINDI & 95-99 & 223.88 & 202.77 & 247.08 & RW2 \\ 
  Tanzania & LINDI & 00-04 & 151.11 & 191.98 & 117.67 & HT-Direct \\ 
  Tanzania & LINDI & 00-04 & 154.45 & 135.49 & 175.54 & RW2 \\ 
  Tanzania & LINDI & 05-09 & 100.97 & 135.81 & 74.30 & HT-Direct \\ 
  Tanzania & LINDI & 05-09 & 89.85 & 74.16 & 108.20 & RW2 \\ 
  Tanzania & LINDI & 10-14 & 55.71 & 101.48 & 29.89 & HT-Direct \\ 
  Tanzania & LINDI & 10-14 & 52.18 & 37.29 & 71.82 & RW2 \\ 
  Tanzania & LINDI & 15-19 & 30.21 & 11.38 & 77.02 & RW2 \\ 
  Tanzania & MARA & 80-84 & 194.42 & 244.65 & 152.42 & HT-Direct \\ 
  Tanzania & MARA & 80-84 & 207.16 & 170.67 & 249.27 & RW2 \\ 
  Tanzania & MARA & 85-89 & 211.38 & 250.79 & 176.71 & HT-Direct \\ 
  Tanzania & MARA & 85-89 & 201.00 & 178.51 & 225.60 & RW2 \\ 
  Tanzania & MARA & 90-94 & 180.99 & 204.33 & 159.78 & HT-Direct \\ 
  Tanzania & MARA & 90-94 & 194.69 & 177.89 & 212.40 & RW2 \\ 
  Tanzania & MARA & 95-99 & 187.63 & 216.05 & 162.17 & HT-Direct \\ 
  Tanzania & MARA & 95-99 & 188.54 & 172.15 & 206.19 & RW2 \\ 
  Tanzania & MARA & 00-04 & 168.89 & 195.15 & 145.52 & HT-Direct \\ 
  Tanzania & MARA & 00-04 & 155.57 & 140.53 & 172.42 & RW2 \\ 
  Tanzania & MARA & 05-09 & 121.70 & 149.45 & 98.51 & HT-Direct \\ 
  Tanzania & MARA & 05-09 & 108.38 & 95.27 & 123.19 & RW2 \\ 
  Tanzania & MARA & 10-14 & 80.97 & 112.97 & 57.45 & HT-Direct \\ 
  Tanzania & MARA & 10-14 & 73.00 & 58.34 & 90.47 & RW2 \\ 
  Tanzania & MARA & 15-19 & 48.64 & 20.09 & 112.28 & RW2 \\ 
  Tanzania & MBEYA & 80-84 & 145.34 & 184.21 & 113.54 & HT-Direct \\ 
  Tanzania & MBEYA & 80-84 & 151.04 & 122.02 & 185.29 & RW2 \\ 
  Tanzania & MBEYA & 85-89 & 127.93 & 163.27 & 99.32 & HT-Direct \\ 
  Tanzania & MBEYA & 85-89 & 149.79 & 129.11 & 173.03 & RW2 \\ 
  Tanzania & MBEYA & 90-94 & 168.16 & 204.57 & 137.12 & HT-Direct \\ 
  Tanzania & MBEYA & 90-94 & 150.80 & 132.79 & 170.97 & RW2 \\ 
  Tanzania & MBEYA & 95-99 & 140.71 & 173.24 & 113.46 & HT-Direct \\ 
  Tanzania & MBEYA & 95-99 & 144.93 & 127.84 & 163.91 & RW2 \\ 
  Tanzania & MBEYA & 00-04 & 114.27 & 141.01 & 92.06 & HT-Direct \\ 
  Tanzania & MBEYA & 00-04 & 116.52 & 101.71 & 133.06 & RW2 \\ 
  Tanzania & MBEYA & 05-09 & 87.43 & 121.73 & 62.11 & HT-Direct \\ 
  Tanzania & MBEYA & 05-09 & 86.27 & 72.78 & 102.09 & RW2 \\ 
  Tanzania & MBEYA & 10-14 & 90.23 & 132.68 & 60.41 & HT-Direct \\ 
  Tanzania & MBEYA & 10-14 & 67.71 & 52.26 & 87.27 & RW2 \\ 
  Tanzania & MBEYA & 15-19 & 54.13 & 22.22 & 126.50 & RW2 \\ 
  Tanzania & MOROGORO & 80-84 & 231.80 & 283.50 & 187.07 & HT-Direct \\ 
  Tanzania & MOROGORO & 80-84 & 219.87 & 184.45 & 259.66 & RW2 \\ 
  Tanzania & MOROGORO & 85-89 & 191.27 & 226.30 & 160.53 & HT-Direct \\ 
  Tanzania & MOROGORO & 85-89 & 210.86 & 187.75 & 235.96 & RW2 \\ 
  Tanzania & MOROGORO & 90-94 & 167.27 & 197.22 & 141.08 & HT-Direct \\ 
  Tanzania & MOROGORO & 90-94 & 198.78 & 180.36 & 218.52 & RW2 \\ 
  Tanzania & MOROGORO & 95-99 & 212.36 & 242.61 & 184.96 & HT-Direct \\ 
  Tanzania & MOROGORO & 95-99 & 179.86 & 162.56 & 198.65 & RW2 \\ 
  Tanzania & MOROGORO & 00-04 & 110.90 & 151.60 & 80.09 & HT-Direct \\ 
  Tanzania & MOROGORO & 00-04 & 130.67 & 114.05 & 149.55 & RW2 \\ 
  Tanzania & MOROGORO & 05-09 & 87.42 & 122.40 & 61.74 & HT-Direct \\ 
  Tanzania & MOROGORO & 05-09 & 82.21 & 67.96 & 99.19 & RW2 \\ 
  Tanzania & MOROGORO & 10-14 & 62.52 & 107.36 & 35.67 & HT-Direct \\ 
  Tanzania & MOROGORO & 10-14 & 52.10 & 38.57 & 69.65 & RW2 \\ 
  Tanzania & MOROGORO & 15-19 & 33.01 & 13.21 & 80.37 & RW2 \\ 
  Tanzania & MTWARA & 80-84 & 193.03 & 238.22 & 154.67 & HT-Direct \\ 
  Tanzania & MTWARA & 80-84 & 200.55 & 164.21 & 242.11 & RW2 \\ 
  Tanzania & MTWARA & 85-89 & 182.64 & 220.22 & 150.23 & HT-Direct \\ 
  Tanzania & MTWARA & 85-89 & 213.97 & 188.17 & 241.87 & RW2 \\ 
  Tanzania & MTWARA & 90-94 & 237.53 & 272.91 & 205.44 & HT-Direct \\ 
  Tanzania & MTWARA & 90-94 & 220.25 & 197.63 & 245.20 & RW2 \\ 
  Tanzania & MTWARA & 95-99 & 203.11 & 242.02 & 169.05 & HT-Direct \\ 
  Tanzania & MTWARA & 95-99 & 196.27 & 174.30 & 220.53 & RW2 \\ 
  Tanzania & MTWARA & 00-04 & 114.76 & 143.66 & 91.05 & HT-Direct \\ 
  Tanzania & MTWARA & 00-04 & 130.87 & 112.00 & 151.93 & RW2 \\ 
  Tanzania & MTWARA & 05-09 & 78.09 & 122.69 & 48.80 & HT-Direct \\ 
  Tanzania & MTWARA & 05-09 & 75.74 & 60.47 & 94.12 & RW2 \\ 
  Tanzania & MTWARA & 10-14 & 64.52 & 109.40 & 37.27 & HT-Direct \\ 
  Tanzania & MTWARA & 10-14 & 45.62 & 31.88 & 65.01 & RW2 \\ 
  Tanzania & MTWARA & 15-19 & 27.83 & 10.09 & 74.49 & RW2 \\ 
  Tanzania & MWANZA & 80-84 & 153.58 & 203.55 & 114.11 & HT-Direct \\ 
  Tanzania & MWANZA & 80-84 & 179.07 & 143.07 & 221.77 & RW2 \\ 
  Tanzania & MWANZA & 85-89 & 170.33 & 202.90 & 142.06 & HT-Direct \\ 
  Tanzania & MWANZA & 85-89 & 173.00 & 151.58 & 196.60 & RW2 \\ 
  Tanzania & MWANZA & 90-94 & 176.48 & 205.67 & 150.65 & HT-Direct \\ 
  Tanzania & MWANZA & 90-94 & 163.78 & 147.53 & 181.55 & RW2 \\ 
  Tanzania & MWANZA & 95-99 & 135.66 & 156.48 & 117.21 & HT-Direct \\ 
  Tanzania & MWANZA & 95-99 & 151.20 & 136.46 & 166.86 & RW2 \\ 
  Tanzania & MWANZA & 00-04 & 130.47 & 156.30 & 108.36 & HT-Direct \\ 
  Tanzania & MWANZA & 00-04 & 121.07 & 107.96 & 135.72 & RW2 \\ 
  Tanzania & MWANZA & 05-09 & 91.34 & 113.06 & 73.45 & HT-Direct \\ 
  Tanzania & MWANZA & 05-09 & 84.88 & 73.80 & 97.55 & RW2 \\ 
  Tanzania & MWANZA & 10-14 & 73.88 & 106.42 & 50.73 & HT-Direct \\ 
  Tanzania & MWANZA & 10-14 & 59.60 & 46.94 & 75.37 & RW2 \\ 
  Tanzania & MWANZA & 15-19 & 42.03 & 16.97 & 100.13 & RW2 \\ 
  Tanzania & PWANI & 80-84 & 181.74 & 238.36 & 136.17 & HT-Direct \\ 
  Tanzania & PWANI & 80-84 & 236.01 & 191.82 & 285.58 & RW2 \\ 
  Tanzania & PWANI & 85-89 & 261.14 & 303.19 & 223.06 & HT-Direct \\ 
  Tanzania & PWANI & 85-89 & 220.93 & 194.78 & 249.53 & RW2 \\ 
  Tanzania & PWANI & 90-94 & 157.87 & 189.81 & 130.44 & HT-Direct \\ 
  Tanzania & PWANI & 90-94 & 192.11 & 171.70 & 214.02 & RW2 \\ 
  Tanzania & PWANI & 95-99 & 174.53 & 209.38 & 144.41 & HT-Direct \\ 
  Tanzania & PWANI & 95-99 & 163.94 & 145.59 & 183.63 & RW2 \\ 
  Tanzania & PWANI & 00-04 & 113.70 & 139.14 & 92.42 & HT-Direct \\ 
  Tanzania & PWANI & 00-04 & 120.28 & 104.90 & 137.49 & RW2 \\ 
  Tanzania & PWANI & 05-09 & 81.29 & 121.76 & 53.46 & HT-Direct \\ 
  Tanzania & PWANI & 05-09 & 80.82 & 66.52 & 98.05 & RW2 \\ 
  Tanzania & PWANI & 10-14 & 85.21 & 137.96 & 51.43 & HT-Direct \\ 
  Tanzania & PWANI & 10-14 & 56.50 & 41.61 & 76.85 & RW2 \\ 
  Tanzania & PWANI & 15-19 & 40.17 & 15.78 & 99.52 & RW2 \\ 
  Tanzania & RUKWA & 80-84 & 232.14 & 290.25 & 182.67 & HT-Direct \\ 
  Tanzania & RUKWA & 80-84 & 235.44 & 193.79 & 284.17 & RW2 \\ 
  Tanzania & RUKWA & 85-89 & 195.34 & 229.90 & 164.87 & HT-Direct \\ 
  Tanzania & RUKWA & 85-89 & 204.70 & 182.09 & 229.68 & RW2 \\ 
  Tanzania & RUKWA & 90-94 & 179.72 & 203.26 & 158.36 & HT-Direct \\ 
  Tanzania & RUKWA & 90-94 & 180.83 & 164.38 & 198.56 & RW2 \\ 
  Tanzania & RUKWA & 95-99 & 169.25 & 205.57 & 138.24 & HT-Direct \\ 
  Tanzania & RUKWA & 95-99 & 155.79 & 139.59 & 173.36 & RW2 \\ 
  Tanzania & RUKWA & 00-04 & 106.35 & 124.23 & 90.77 & HT-Direct \\ 
  Tanzania & RUKWA & 00-04 & 115.63 & 102.85 & 129.19 & RW2 \\ 
  Tanzania & RUKWA & 05-09 & 86.17 & 106.89 & 69.15 & HT-Direct \\ 
  Tanzania & RUKWA & 05-09 & 83.24 & 73.05 & 94.61 & RW2 \\ 
  Tanzania & RUKWA & 10-14 & 93.70 & 122.01 & 71.42 & HT-Direct \\ 
  Tanzania & RUKWA & 10-14 & 66.63 & 54.73 & 81.31 & RW2 \\ 
  Tanzania & RUKWA & 15-19 & 55.23 & 23.51 & 127.35 & RW2 \\ 
  Tanzania & RUVUMA & 80-84 & 159.66 & 209.12 & 120.13 & HT-Direct \\ 
  Tanzania & RUVUMA & 80-84 & 147.41 & 116.94 & 184.18 & RW2 \\ 
  Tanzania & RUVUMA & 85-89 & 118.21 & 148.04 & 93.73 & HT-Direct \\ 
  Tanzania & RUVUMA & 85-89 & 157.00 & 134.96 & 181.88 & RW2 \\ 
  Tanzania & RUVUMA & 90-94 & 183.01 & 224.41 & 147.80 & HT-Direct \\ 
  Tanzania & RUVUMA & 90-94 & 166.26 & 146.64 & 188.20 & RW2 \\ 
  Tanzania & RUVUMA & 95-99 & 162.84 & 196.78 & 133.77 & HT-Direct \\ 
  Tanzania & RUVUMA & 95-99 & 160.62 & 142.16 & 181.54 & RW2 \\ 
  Tanzania & RUVUMA & 00-04 & 129.48 & 159.50 & 104.41 & HT-Direct \\ 
  Tanzania & RUVUMA & 00-04 & 119.77 & 104.20 & 137.16 & RW2 \\ 
  Tanzania & RUVUMA & 05-09 & 66.21 & 92.09 & 47.22 & HT-Direct \\ 
  Tanzania & RUVUMA & 05-09 & 76.35 & 62.98 & 92.14 & RW2 \\ 
  Tanzania & RUVUMA & 10-14 & 68.63 & 121.95 & 37.62 & HT-Direct \\ 
  Tanzania & RUVUMA & 10-14 & 49.89 & 35.80 & 68.20 & RW2 \\ 
  Tanzania & RUVUMA & 15-19 & 32.71 & 12.50 & 83.02 & RW2 \\ 
  Tanzania & SHINYANGA & 80-84 & 190.72 & 231.11 & 155.96 & HT-Direct \\ 
  Tanzania & SHINYANGA & 80-84 & 188.39 & 159.50 & 221.84 & RW2 \\ 
  Tanzania & SHINYANGA & 85-89 & 162.95 & 193.74 & 136.22 & HT-Direct \\ 
  Tanzania & SHINYANGA & 85-89 & 175.55 & 156.94 & 196.02 & RW2 \\ 
  Tanzania & SHINYANGA & 90-94 & 168.27 & 193.54 & 145.71 & HT-Direct \\ 
  Tanzania & SHINYANGA & 90-94 & 165.21 & 150.62 & 180.61 & RW2 \\ 
  Tanzania & SHINYANGA & 95-99 & 147.13 & 167.96 & 128.48 & HT-Direct \\ 
  Tanzania & SHINYANGA & 95-99 & 154.02 & 140.87 & 167.85 & RW2 \\ 
  Tanzania & SHINYANGA & 00-04 & 126.54 & 146.34 & 109.07 & HT-Direct \\ 
  Tanzania & SHINYANGA & 00-04 & 124.18 & 113.03 & 136.30 & RW2 \\ 
  Tanzania & SHINYANGA & 05-09 & 93.72 & 113.56 & 77.04 & HT-Direct \\ 
  Tanzania & SHINYANGA & 05-09 & 90.11 & 80.75 & 100.54 & RW2 \\ 
  Tanzania & SHINYANGA & 10-14 & 88.66 & 108.92 & 71.86 & HT-Direct \\ 
  Tanzania & SHINYANGA & 10-14 & 67.39 & 57.98 & 78.31 & RW2 \\ 
  Tanzania & SHINYANGA & 15-19 & 51.09 & 22.59 & 111.27 & RW2 \\ 
  Tanzania & SINGIDA & 80-84 & 173.27 & 228.70 & 129.03 & HT-Direct \\ 
  Tanzania & SINGIDA & 80-84 & 154.18 & 121.77 & 193.58 & RW2 \\ 
  Tanzania & SINGIDA & 85-89 & 92.39 & 123.10 & 68.74 & HT-Direct \\ 
  Tanzania & SINGIDA & 85-89 & 140.68 & 119.82 & 164.12 & RW2 \\ 
  Tanzania & SINGIDA & 90-94 & 133.16 & 159.69 & 110.46 & HT-Direct \\ 
  Tanzania & SINGIDA & 90-94 & 130.39 & 115.25 & 147.08 & RW2 \\ 
  Tanzania & SINGIDA & 95-99 & 121.56 & 148.91 & 98.64 & HT-Direct \\ 
  Tanzania & SINGIDA & 95-99 & 116.24 & 102.55 & 131.66 & RW2 \\ 
  Tanzania & SINGIDA & 00-04 & 90.11 & 122.18 & 65.83 & HT-Direct \\ 
  Tanzania & SINGIDA & 00-04 & 84.75 & 72.67 & 98.58 & RW2 \\ 
  Tanzania & SINGIDA & 05-09 & 58.54 & 82.86 & 41.04 & HT-Direct \\ 
  Tanzania & SINGIDA & 05-09 & 54.32 & 44.28 & 66.53 & RW2 \\ 
  Tanzania & SINGIDA & 10-14 & 33.32 & 60.69 & 18.06 & HT-Direct \\ 
  Tanzania & SINGIDA & 10-14 & 35.52 & 25.82 & 48.42 & RW2 \\ 
  Tanzania & SINGIDA & 15-19 & 23.51 & 9.22 & 59.03 & RW2 \\ 
  Tanzania & TABORA & 80-84 & 177.31 & 229.12 & 135.16 & HT-Direct \\ 
  Tanzania & TABORA & 80-84 & 174.31 & 139.19 & 215.18 & RW2 \\ 
  Tanzania & TABORA & 85-89 & 129.93 & 168.42 & 99.19 & HT-Direct \\ 
  Tanzania & TABORA & 85-89 & 165.19 & 142.16 & 190.87 & RW2 \\ 
  Tanzania & TABORA & 90-94 & 156.58 & 188.39 & 129.29 & HT-Direct \\ 
  Tanzania & TABORA & 90-94 & 159.56 & 142.77 & 177.80 & RW2 \\ 
  Tanzania & TABORA & 95-99 & 157.20 & 178.14 & 138.32 & HT-Direct \\ 
  Tanzania & TABORA & 95-99 & 147.37 & 133.92 & 162.04 & RW2 \\ 
  Tanzania & TABORA & 00-04 & 106.21 & 126.98 & 88.50 & HT-Direct \\ 
  Tanzania & TABORA & 00-04 & 111.05 & 99.69 & 123.48 & RW2 \\ 
  Tanzania & TABORA & 05-09 & 78.88 & 98.17 & 63.11 & HT-Direct \\ 
  Tanzania & TABORA & 05-09 & 74.86 & 65.31 & 85.71 & RW2 \\ 
  Tanzania & TABORA & 10-14 & 65.78 & 93.23 & 46.00 & HT-Direct \\ 
  Tanzania & TABORA & 10-14 & 52.80 & 42.28 & 65.61 & RW2 \\ 
  Tanzania & TABORA & 15-19 & 37.82 & 15.76 & 87.85 & RW2 \\ 
  Tanzania & TANGA & 80-84 & 140.51 & 178.03 & 109.84 & HT-Direct \\ 
  Tanzania & TANGA & 80-84 & 180.18 & 146.70 & 218.11 & RW2 \\ 
  Tanzania & TANGA & 85-89 & 197.98 & 237.71 & 163.47 & HT-Direct \\ 
  Tanzania & TANGA & 85-89 & 181.83 & 160.82 & 205.39 & RW2 \\ 
  Tanzania & TANGA & 90-94 & 179.36 & 204.06 & 157.06 & HT-Direct \\ 
  Tanzania & TANGA & 90-94 & 171.19 & 154.93 & 188.92 & RW2 \\ 
  Tanzania & TANGA & 95-99 & 131.61 & 159.45 & 108.01 & HT-Direct \\ 
  Tanzania & TANGA & 95-99 & 151.25 & 135.28 & 168.88 & RW2 \\ 
  Tanzania & TANGA & 00-04 & 124.57 & 153.27 & 100.60 & HT-Direct \\ 
  Tanzania & TANGA & 00-04 & 111.51 & 97.45 & 127.49 & RW2 \\ 
  Tanzania & TANGA & 05-09 & 71.88 & 99.55 & 51.46 & HT-Direct \\ 
  Tanzania & TANGA & 05-09 & 71.48 & 59.27 & 86.28 & RW2 \\ 
  Tanzania & TANGA & 10-14 & 55.12 & 96.47 & 30.88 & HT-Direct \\ 
  Tanzania & TANGA & 10-14 & 46.23 & 33.39 & 62.86 & RW2 \\ 
  Tanzania & TANGA & 15-19 & 29.98 & 11.47 & 75.66 & RW2 \\ 
  Togo & ALL & 80-84 & 162.29 & 159.60 & 164.74 & IHME \\ 
  Togo & ALL & 80-84 & 169.92 & 156.97 & 183.70 & RW2 \\ 
  Togo & ALL & 80-84 & 169.65 & 163.43 & 175.54 & UN \\ 
  Togo & ALL & 85-89 & 150.47 & 148.21 & 152.78 & IHME \\ 
  Togo & ALL & 85-89 & 152.75 & 142.29 & 163.77 & RW2 \\ 
  Togo & ALL & 85-89 & 153.33 & 148.78 & 158.66 & UN \\ 
  Togo & ALL & 90-94 & 140.73 & 138.55 & 143.15 & IHME \\ 
  Togo & ALL & 90-94 & 142.54 & 134.78 & 150.69 & RW2 \\ 
  Togo & ALL & 90-94 & 142.10 & 137.90 & 146.75 & UN \\ 
  Togo & ALL & 95-99 & 130.01 & 127.73 & 132.37 & IHME \\ 
  Togo & ALL & 95-99 & 129.65 & 121.30 & 138.41 & RW2 \\ 
  Togo & ALL & 95-99 & 129.92 & 125.92 & 134.00 & UN \\ 
  Togo & ALL & 00-04 & 117.78 & 115.57 & 120.26 & IHME \\ 
  Togo & ALL & 00-04 & 114.36 & 104.74 & 124.80 & RW2 \\ 
  Togo & ALL & 00-04 & 114.23 & 110.54 & 118.01 & UN \\ 
  Togo & ALL & 05-09 & 104.17 & 101.75 & 106.69 & IHME \\ 
  Togo & ALL & 05-09 & 99.05 & 90.33 & 108.52 & RW2 \\ 
  Togo & ALL & 05-09 & 99.20 & 95.31 & 103.13 & UN \\ 
  Togo & ALL & 10-14 & 88.39 & 85.23 & 91.33 & IHME \\ 
  Togo & ALL & 10-14 & 85.70 & 76.61 & 95.67 & RW2 \\ 
  Togo & ALL & 10-14 & 85.64 & 80.54 & 90.83 & UN \\ 
  Togo & CENTRALE & 80-84 & 193.64 & 235.13 & 157.97 & HT-Direct \\ 
  Togo & CENTRALE & 80-84 & 180.86 & 156.40 & 209.14 & RW2 \\ 
  Togo & CENTRALE & 85-89 & 165.58 & 193.29 & 141.14 & HT-Direct \\ 
  Togo & CENTRALE & 85-89 & 156.78 & 140.99 & 173.81 & RW2 \\ 
  Togo & CENTRALE & 90-94 & 150.06 & 172.90 & 129.76 & HT-Direct \\ 
  Togo & CENTRALE & 90-94 & 147.04 & 134.06 & 160.82 & RW2 \\ 
  Togo & CENTRALE & 95-99 & 127.34 & 152.62 & 105.74 & HT-Direct \\ 
  Togo & CENTRALE & 95-99 & 134.78 & 121.12 & 148.94 & RW2 \\ 
  Togo & CENTRALE & 00-04 & 147.89 & 186.12 & 116.39 & HT-Direct \\ 
  Togo & CENTRALE & 00-04 & 127.54 & 112.86 & 143.72 & RW2 \\ 
  Togo & CENTRALE & 05-09 & 104.04 & 126.91 & 84.89 & HT-Direct \\ 
  Togo & CENTRALE & 05-09 & 115.96 & 101.64 & 132.25 & RW2 \\ 
  Togo & CENTRALE & 10-14 & 114.98 & 142.30 & 92.34 & HT-Direct \\ 
  Togo & CENTRALE & 10-14 & 104.12 & 87.65 & 123.71 & RW2 \\ 
  Togo & CENTRALE & 15-19 & 93.28 & 40.99 & 198.68 & RW2 \\ 
  Togo & GRANDE AGGLOMÉRATION DE LOMÉ & 80-84 & 131.45 & 181.08 & 93.87 & HT-Direct \\ 
  Togo & GRANDE AGGLOMÉRATION DE LOMÉ & 80-84 & 122.97 & 97.23 & 153.14 & RW2 \\ 
  Togo & GRANDE AGGLOMÉRATION DE LOMÉ & 85-89 & 100.95 & 129.16 & 78.35 & HT-Direct \\ 
  Togo & GRANDE AGGLOMÉRATION DE LOMÉ & 85-89 & 107.76 & 92.25 & 125.57 & RW2 \\ 
  Togo & GRANDE AGGLOMÉRATION DE LOMÉ & 90-94 & 105.17 & 129.66 & 84.85 & HT-Direct \\ 
  Togo & GRANDE AGGLOMÉRATION DE LOMÉ & 90-94 & 102.03 & 89.75 & 115.91 & RW2 \\ 
  Togo & GRANDE AGGLOMÉRATION DE LOMÉ & 95-99 & 105.65 & 131.83 & 84.16 & HT-Direct \\ 
  Togo & GRANDE AGGLOMÉRATION DE LOMÉ & 95-99 & 91.29 & 80.18 & 104.41 & RW2 \\ 
  Togo & GRANDE AGGLOMÉRATION DE LOMÉ & 00-04 & 80.15 & 103.13 & 61.93 & HT-Direct \\ 
  Togo & GRANDE AGGLOMÉRATION DE LOMÉ & 00-04 & 80.29 & 69.44 & 92.84 & RW2 \\ 
  Togo & GRANDE AGGLOMÉRATION DE LOMÉ & 05-09 & 61.64 & 79.11 & 47.82 & HT-Direct \\ 
  Togo & GRANDE AGGLOMÉRATION DE LOMÉ & 05-09 & 67.19 & 56.91 & 79.00 & RW2 \\ 
  Togo & GRANDE AGGLOMÉRATION DE LOMÉ & 10-14 & 56.67 & 75.03 & 42.59 & HT-Direct \\ 
  Togo & GRANDE AGGLOMÉRATION DE LOMÉ & 10-14 & 56.31 & 45.04 & 70.22 & RW2 \\ 
  Togo & GRANDE AGGLOMÉRATION DE LOMÉ & 15-19 & 47.35 & 19.71 & 109.85 & RW2 \\ 
  Togo & KARA & 80-84 & 195.98 & 238.72 & 159.29 & HT-Direct \\ 
  Togo & KARA & 80-84 & 197.38 & 171.50 & 226.44 & RW2 \\ 
  Togo & KARA & 85-89 & 182.88 & 210.30 & 158.32 & HT-Direct \\ 
  Togo & KARA & 85-89 & 174.61 & 158.34 & 192.28 & RW2 \\ 
  Togo & KARA & 90-94 & 182.71 & 206.33 & 161.24 & HT-Direct \\ 
  Togo & KARA & 90-94 & 166.46 & 153.41 & 180.33 & RW2 \\ 
  Togo & KARA & 95-99 & 137.33 & 159.88 & 117.51 & HT-Direct \\ 
  Togo & KARA & 95-99 & 152.98 & 139.37 & 167.54 & RW2 \\ 
  Togo & KARA & 00-04 & 158.44 & 196.44 & 126.63 & HT-Direct \\ 
  Togo & KARA & 00-04 & 143.79 & 128.20 & 161.20 & RW2 \\ 
  Togo & KARA & 05-09 & 141.04 & 168.28 & 117.59 & HT-Direct \\ 
  Togo & KARA & 05-09 & 127.65 & 112.77 & 144.34 & RW2 \\ 
  Togo & KARA & 10-14 & 102.27 & 124.39 & 83.70 & HT-Direct \\ 
  Togo & KARA & 10-14 & 109.12 & 92.99 & 127.45 & RW2 \\ 
  Togo & KARA & 15-19 & 92.15 & 40.78 & 195.64 & RW2 \\ 
  Togo & MARITIME (SANS AGGLOMÉRATION DE LOMÉ) & 80-84 & 159.26 & 195.63 & 128.58 & HT-Direct \\ 
  Togo & MARITIME (SANS AGGLOMÉRATION DE LOMÉ) & 80-84 & 161.95 & 137.42 & 189.61 & RW2 \\ 
  Togo & MARITIME (SANS AGGLOMÉRATION DE LOMÉ) & 85-89 & 154.28 & 192.27 & 122.66 & HT-Direct \\ 
  Togo & MARITIME (SANS AGGLOMÉRATION DE LOMÉ) & 85-89 & 142.21 & 125.59 & 160.50 & RW2 \\ 
  Togo & MARITIME (SANS AGGLOMÉRATION DE LOMÉ) & 90-94 & 139.09 & 160.78 & 119.90 & HT-Direct \\ 
  Togo & MARITIME (SANS AGGLOMÉRATION DE LOMÉ) & 90-94 & 134.08 & 121.53 & 148.01 & RW2 \\ 
  Togo & MARITIME (SANS AGGLOMÉRATION DE LOMÉ) & 95-99 & 142.11 & 169.17 & 118.75 & HT-Direct \\ 
  Togo & MARITIME (SANS AGGLOMÉRATION DE LOMÉ) & 95-99 & 120.29 & 107.53 & 134.43 & RW2 \\ 
  Togo & MARITIME (SANS AGGLOMÉRATION DE LOMÉ) & 00-04 & 76.84 & 106.96 & 54.69 & HT-Direct \\ 
  Togo & MARITIME (SANS AGGLOMÉRATION DE LOMÉ) & 00-04 & 107.04 & 92.06 & 124.03 & RW2 \\ 
  Togo & MARITIME (SANS AGGLOMÉRATION DE LOMÉ) & 05-09 & 79.65 & 112.58 & 55.74 & HT-Direct \\ 
  Togo & MARITIME (SANS AGGLOMÉRATION DE LOMÉ) & 05-09 & 91.54 & 75.55 & 110.08 & RW2 \\ 
  Togo & MARITIME (SANS AGGLOMÉRATION DE LOMÉ) & 10-14 & 88.47 & 123.68 & 62.56 & HT-Direct \\ 
  Togo & MARITIME (SANS AGGLOMÉRATION DE LOMÉ) & 10-14 & 78.37 & 60.89 & 100.40 & RW2 \\ 
  Togo & MARITIME (SANS AGGLOMÉRATION DE LOMÉ) & 15-19 & 67.60 & 28.23 & 153.30 & RW2 \\ 
  Togo & PLATEAUX & 80-84 & 180.33 & 210.53 & 153.63 & HT-Direct \\ 
  Togo & PLATEAUX & 80-84 & 169.28 & 149.72 & 191.19 & RW2 \\ 
  Togo & PLATEAUX & 85-89 & 146.69 & 166.04 & 129.24 & HT-Direct \\ 
  Togo & PLATEAUX & 85-89 & 146.56 & 133.74 & 160.48 & RW2 \\ 
  Togo & PLATEAUX & 90-94 & 146.52 & 169.98 & 125.81 & HT-Direct \\ 
  Togo & PLATEAUX & 90-94 & 137.39 & 125.64 & 149.74 & RW2 \\ 
  Togo & PLATEAUX & 95-99 & 122.96 & 143.85 & 104.74 & HT-Direct \\ 
  Togo & PLATEAUX & 95-99 & 124.53 & 112.85 & 137.00 & RW2 \\ 
  Togo & PLATEAUX & 00-04 & 116.97 & 140.22 & 97.13 & HT-Direct \\ 
  Togo & PLATEAUX & 00-04 & 115.17 & 102.74 & 128.89 & RW2 \\ 
  Togo & PLATEAUX & 05-09 & 101.42 & 123.60 & 82.84 & HT-Direct \\ 
  Togo & PLATEAUX & 05-09 & 102.82 & 89.98 & 117.20 & RW2 \\ 
  Togo & PLATEAUX & 10-14 & 95.34 & 121.10 & 74.59 & HT-Direct \\ 
  Togo & PLATEAUX & 10-14 & 91.39 & 76.11 & 109.92 & RW2 \\ 
  Togo & PLATEAUX & 15-19 & 81.55 & 35.70 & 176.49 & RW2 \\ 
  Togo & SAVANES & 80-84 & 223.03 & 250.47 & 197.79 & HT-Direct \\ 
  Togo & SAVANES & 80-84 & 210.24 & 189.68 & 232.44 & RW2 \\ 
  Togo & SAVANES & 85-89 & 171.90 & 194.50 & 151.44 & HT-Direct \\ 
  Togo & SAVANES & 85-89 & 182.51 & 167.64 & 198.35 & RW2 \\ 
  Togo & SAVANES & 90-94 & 183.47 & 199.91 & 168.10 & HT-Direct \\ 
  Togo & SAVANES & 90-94 & 171.59 & 160.42 & 183.46 & RW2 \\ 
  Togo & SAVANES & 95-99 & 165.85 & 187.96 & 145.88 & HT-Direct \\ 
  Togo & SAVANES & 95-99 & 154.15 & 141.46 & 167.88 & RW2 \\ 
  Togo & SAVANES & 00-04 & 135.77 & 159.50 & 115.08 & HT-Direct \\ 
  Togo & SAVANES & 00-04 & 138.35 & 124.24 & 154.18 & RW2 \\ 
  Togo & SAVANES & 05-09 & 117.79 & 140.33 & 98.46 & HT-Direct \\ 
  Togo & SAVANES & 05-09 & 115.39 & 101.63 & 130.70 & RW2 \\ 
  Togo & SAVANES & 10-14 & 84.06 & 108.25 & 64.88 & HT-Direct \\ 
  Togo & SAVANES & 10-14 & 91.60 & 74.68 & 110.83 & RW2 \\ 
  Togo & SAVANES & 15-19 & 71.78 & 30.07 & 158.96 & RW2 \\ 
  Uganda & ALL & 80-84 & 192.81 & 189.88 & 195.87 & IHME \\ 
  Uganda & ALL & 80-84 & 209.25 & 199.14 & 219.74 & RW2 \\ 
  Uganda & ALL & 80-84 & 209.20 & 203.14 & 215.58 & UN \\ 
  Uganda & ALL & 85-89 & 177.00 & 174.76 & 179.15 & IHME \\ 
  Uganda & ALL & 85-89 & 192.10 & 182.98 & 201.52 & RW2 \\ 
  Uganda & ALL & 85-89 & 192.14 & 187.02 & 197.24 & UN \\ 
  Uganda & ALL & 90-94 & 162.28 & 160.44 & 164.15 & IHME \\ 
  Uganda & ALL & 90-94 & 179.53 & 171.17 & 188.21 & RW2 \\ 
  Uganda & ALL & 90-94 & 179.55 & 174.34 & 184.65 & UN \\ 
  Uganda & ALL & 95-99 & 143.81 & 141.82 & 145.60 & IHME \\ 
  Uganda & ALL & 95-99 & 163.10 & 155.34 & 171.11 & RW2 \\ 
  Uganda & ALL & 95-99 & 163.03 & 158.73 & 167.42 & UN \\ 
  Uganda & ALL & 00-04 & 121.66 & 119.87 & 123.38 & IHME \\ 
  Uganda & ALL & 00-04 & 132.13 & 125.76 & 138.81 & RW2 \\ 
  Uganda & ALL & 00-04 & 132.14 & 128.32 & 136.05 & UN \\ 
  Uganda & ALL & 05-09 & 99.98 & 97.90 & 102.17 & IHME \\ 
  Uganda & ALL & 05-09 & 92.64 & 86.77 & 98.85 & RW2 \\ 
  Uganda & ALL & 05-09 & 92.82 & 89.49 & 96.21 & UN \\ 
  Uganda & ALL & 10-14 & 80.52 & 77.72 & 83.31 & IHME \\ 
  Uganda & ALL & 10-14 & 66.06 & 56.63 & 76.87 & RW2 \\ 
  Uganda & ALL & 10-14 & 65.56 & 59.96 & 71.10 & UN \\ 
  Uganda & CENTRAL & 80-84 & 199.34 & 216.78 & 182.98 & HT-Direct \\ 
  Uganda & CENTRAL & 80-84 & 205.05 & 188.71 & 222.59 & RW2 \\ 
  Uganda & CENTRAL & 85-89 & 160.57 & 174.67 & 147.41 & HT-Direct \\ 
  Uganda & CENTRAL & 85-89 & 179.52 & 167.05 & 192.61 & RW2 \\ 
  Uganda & CENTRAL & 90-94 & 141.10 & 153.63 & 129.44 & HT-Direct \\ 
  Uganda & CENTRAL & 90-94 & 159.40 & 147.61 & 171.34 & RW2 \\ 
  Uganda & CENTRAL & 95-99 & 130.62 & 146.02 & 116.62 & HT-Direct \\ 
  Uganda & CENTRAL & 95-99 & 143.28 & 132.09 & 154.87 & RW2 \\ 
  Uganda & CENTRAL & 00-04 & 121.10 & 134.73 & 108.68 & HT-Direct \\ 
  Uganda & CENTRAL & 00-04 & 116.86 & 107.73 & 127.25 & RW2 \\ 
  Uganda & CENTRAL & 05-09 & 99.55 & 116.96 & 84.49 & HT-Direct \\ 
  Uganda & CENTRAL & 05-09 & 80.79 & 72.01 & 90.69 & RW2 \\ 
  Uganda & CENTRAL & 10-14 & 49.81 & 76.54 & 32.10 & HT-Direct \\ 
  Uganda & CENTRAL & 10-14 & 54.13 & 41.62 & 68.35 & RW2 \\ 
  Uganda & CENTRAL & 15-19 & 36.07 & 13.96 & 87.93 & RW2 \\ 
  Uganda & EASTERN & 80-84 & 206.16 & 224.77 & 188.72 & HT-Direct \\ 
  Uganda & EASTERN & 80-84 & 219.33 & 201.13 & 238.52 & RW2 \\ 
  Uganda & EASTERN & 85-89 & 183.74 & 198.11 & 170.20 & HT-Direct \\ 
  Uganda & EASTERN & 85-89 & 201.02 & 187.34 & 216.11 & RW2 \\ 
  Uganda & EASTERN & 90-94 & 164.58 & 178.53 & 151.51 & HT-Direct \\ 
  Uganda & EASTERN & 90-94 & 175.65 & 163.41 & 189.44 & RW2 \\ 
  Uganda & EASTERN & 95-99 & 136.88 & 148.83 & 125.75 & HT-Direct \\ 
  Uganda & EASTERN & 95-99 & 145.83 & 135.58 & 156.33 & RW2 \\ 
  Uganda & EASTERN & 00-04 & 108.13 & 119.40 & 97.81 & HT-Direct \\ 
  Uganda & EASTERN & 00-04 & 112.33 & 103.32 & 121.50 & RW2 \\ 
  Uganda & EASTERN & 05-09 & 92.10 & 104.86 & 80.75 & HT-Direct \\ 
  Uganda & EASTERN & 05-09 & 81.19 & 73.51 & 89.75 & RW2 \\ 
  Uganda & EASTERN & 10-14 & 92.51 & 133.51 & 63.18 & HT-Direct \\ 
  Uganda & EASTERN & 10-14 & 61.75 & 48.99 & 79.03 & RW2 \\ 
  Uganda & EASTERN & 15-19 & 47.94 & 19.37 & 117.89 & RW2 \\ 
  Uganda & NORTHERN & 80-84 & 240.53 & 267.97 & 215.08 & HT-Direct \\ 
  Uganda & NORTHERN & 80-84 & 249.69 & 225.48 & 275.59 & RW2 \\ 
  Uganda & NORTHERN & 85-89 & 207.01 & 226.98 & 188.36 & HT-Direct \\ 
  Uganda & NORTHERN & 85-89 & 229.40 & 212.64 & 246.74 & RW2 \\ 
  Uganda & NORTHERN & 90-94 & 187.23 & 202.19 & 173.14 & HT-Direct \\ 
  Uganda & NORTHERN & 90-94 & 210.86 & 197.25 & 225.08 & RW2 \\ 
  Uganda & NORTHERN & 95-99 & 183.51 & 198.25 & 169.62 & HT-Direct \\ 
  Uganda & NORTHERN & 95-99 & 191.99 & 180.23 & 204.40 & RW2 \\ 
  Uganda & NORTHERN & 00-04 & 160.86 & 174.74 & 147.88 & HT-Direct \\ 
  Uganda & NORTHERN & 00-04 & 155.41 & 145.46 & 166.10 & RW2 \\ 
  Uganda & NORTHERN & 05-09 & 113.60 & 130.90 & 98.32 & HT-Direct \\ 
  Uganda & NORTHERN & 05-09 & 108.29 & 97.62 & 120.03 & RW2 \\ 
  Uganda & NORTHERN & 10-14 & 119.45 & 185.76 & 74.64 & HT-Direct \\ 
  Uganda & NORTHERN & 10-14 & 77.15 & 60.67 & 97.50 & RW2 \\ 
  Uganda & NORTHERN & 15-19 & 55.68 & 22.33 & 132.93 & RW2 \\ 
  Uganda & WESTERN & 80-84 & 180.34 & 196.02 & 165.64 & HT-Direct \\ 
  Uganda & WESTERN & 80-84 & 186.09 & 171.22 & 201.88 & RW2 \\ 
  Uganda & WESTERN & 85-89 & 157.84 & 171.68 & 144.92 & HT-Direct \\ 
  Uganda & WESTERN & 85-89 & 180.95 & 167.60 & 194.57 & RW2 \\ 
  Uganda & WESTERN & 90-94 & 160.98 & 175.76 & 147.22 & HT-Direct \\ 
  Uganda & WESTERN & 90-94 & 180.79 & 167.74 & 194.52 & RW2 \\ 
  Uganda & WESTERN & 95-99 & 178.83 & 195.93 & 162.91 & HT-Direct \\ 
  Uganda & WESTERN & 95-99 & 177.12 & 164.32 & 191.62 & RW2 \\ 
  Uganda & WESTERN & 00-04 & 147.38 & 162.37 & 133.56 & HT-Direct \\ 
  Uganda & WESTERN & 00-04 & 146.32 & 135.77 & 157.89 & RW2 \\ 
  Uganda & WESTERN & 05-09 & 106.91 & 122.47 & 93.13 & HT-Direct \\ 
  Uganda & WESTERN & 05-09 & 101.81 & 91.88 & 112.39 & RW2 \\ 
  Uganda & WESTERN & 10-14 & 103.80 & 135.27 & 78.98 & HT-Direct \\ 
  Uganda & WESTERN & 10-14 & 72.82 & 60.48 & 87.38 & RW2 \\ 
  Uganda & WESTERN & 15-19 & 52.94 & 22.33 & 121.53 & RW2 \\ 
  Zambia & ALL & 80-84 & 160.63 & 156.43 & 164.93 & IHME \\ 
  Zambia & ALL & 80-84 & 162.11 & 154.29 & 170.24 & RW2 \\ 
  Zambia & ALL & 80-84 & 162.09 & 157.42 & 167.22 & UN \\ 
  Zambia & ALL & 85-89 & 170.45 & 166.39 & 174.58 & IHME \\ 
  Zambia & ALL & 85-89 & 183.49 & 176.41 & 190.76 & RW2 \\ 
  Zambia & ALL & 85-89 & 183.57 & 178.67 & 188.63 & UN \\ 
  Zambia & ALL & 90-94 & 168.58 & 164.39 & 172.06 & IHME \\ 
  Zambia & ALL & 90-94 & 188.48 & 181.48 & 195.69 & RW2 \\ 
  Zambia & ALL & 90-94 & 188.27 & 183.24 & 193.10 & UN \\ 
  Zambia & ALL & 95-99 & 154.77 & 150.93 & 158.70 & IHME \\ 
  Zambia & ALL & 95-99 & 174.79 & 166.50 & 183.32 & RW2 \\ 
  Zambia & ALL & 95-99 & 175.17 & 170.41 & 180.08 & UN \\ 
  Zambia & ALL & 00-04 & 130.72 & 126.83 & 134.32 & IHME \\ 
  Zambia & ALL & 00-04 & 142.23 & 134.50 & 150.41 & RW2 \\ 
  Zambia & ALL & 00-04 & 141.89 & 137.34 & 147.01 & UN \\ 
  Zambia & ALL & 05-09 & 98.98 & 95.39 & 102.72 & IHME \\ 
  Zambia & ALL & 05-09 & 98.44 & 90.12 & 107.34 & RW2 \\ 
  Zambia & ALL & 05-09 & 98.69 & 95.12 & 102.47 & UN \\ 
  Zambia & ALL & 10-14 & 74.07 & 70.17 & 78.06 & IHME \\ 
  Zambia & ALL & 10-14 & 74.47 & 67.75 & 81.74 & RW2 \\ 
  Zambia & ALL & 10-14 & 74.30 & 70.13 & 78.89 & UN \\ 
  Zambia & CENTRAL & 80-84 & 139.03 & 166.84 & 115.21 & HT-Direct \\ 
  Zambia & CENTRAL & 80-84 & 141.44 & 123.97 & 160.98 & RW2 \\ 
  Zambia & CENTRAL & 85-89 & 159.22 & 180.17 & 140.30 & HT-Direct \\ 
  Zambia & CENTRAL & 85-89 & 162.15 & 148.34 & 176.89 & RW2 \\ 
  Zambia & CENTRAL & 90-94 & 155.41 & 177.30 & 135.77 & HT-Direct \\ 
  Zambia & CENTRAL & 90-94 & 170.36 & 157.73 & 183.94 & RW2 \\ 
  Zambia & CENTRAL & 95-99 & 154.71 & 175.45 & 136.01 & HT-Direct \\ 
  Zambia & CENTRAL & 95-99 & 160.88 & 147.78 & 174.88 & RW2 \\ 
  Zambia & CENTRAL & 00-04 & 115.39 & 134.57 & 98.63 & HT-Direct \\ 
  Zambia & CENTRAL & 00-04 & 131.47 & 118.55 & 145.74 & RW2 \\ 
  Zambia & CENTRAL & 05-09 & 89.09 & 113.77 & 69.35 & HT-Direct \\ 
  Zambia & CENTRAL & 05-09 & 94.23 & 80.87 & 109.38 & RW2 \\ 
  Zambia & CENTRAL & 10-14 & 62.44 & 82.39 & 47.07 & HT-Direct \\ 
  Zambia & CENTRAL & 10-14 & 71.10 & 58.09 & 86.68 & RW2 \\ 
  Zambia & CENTRAL & 15-19 & 55.20 & 23.65 & 124.47 & RW2 \\ 
  Zambia & COPPERBELT & 80-84 & 115.91 & 135.07 & 99.16 & HT-Direct \\ 
  Zambia & COPPERBELT & 80-84 & 126.88 & 111.37 & 143.62 & RW2 \\ 
  Zambia & COPPERBELT & 85-89 & 153.23 & 170.44 & 137.47 & HT-Direct \\ 
  Zambia & COPPERBELT & 85-89 & 152.56 & 140.97 & 165.04 & RW2 \\ 
  Zambia & COPPERBELT & 90-94 & 163.17 & 180.78 & 146.97 & HT-Direct \\ 
  Zambia & COPPERBELT & 90-94 & 163.91 & 151.80 & 177.13 & RW2 \\ 
  Zambia & COPPERBELT & 95-99 & 135.13 & 155.96 & 116.69 & HT-Direct \\ 
  Zambia & COPPERBELT & 95-99 & 153.78 & 139.90 & 169.01 & RW2 \\ 
  Zambia & COPPERBELT & 00-04 & 113.06 & 134.66 & 94.54 & HT-Direct \\ 
  Zambia & COPPERBELT & 00-04 & 123.56 & 109.89 & 138.65 & RW2 \\ 
  Zambia & COPPERBELT & 05-09 & 61.52 & 78.03 & 48.33 & HT-Direct \\ 
  Zambia & COPPERBELT & 05-09 & 86.87 & 73.52 & 102.11 & RW2 \\ 
  Zambia & COPPERBELT & 10-14 & 68.20 & 92.24 & 50.09 & HT-Direct \\ 
  Zambia & COPPERBELT & 10-14 & 65.05 & 51.13 & 82.02 & RW2 \\ 
  Zambia & COPPERBELT & 15-19 & 50.17 & 20.84 & 117.01 & RW2 \\ 
  Zambia & EASTERN & 80-84 & 225.13 & 250.32 & 201.80 & HT-Direct \\ 
  Zambia & EASTERN & 80-84 & 224.73 & 204.93 & 246.95 & RW2 \\ 
  Zambia & EASTERN & 85-89 & 237.44 & 256.59 & 219.29 & HT-Direct \\ 
  Zambia & EASTERN & 85-89 & 237.56 & 224.00 & 251.88 & RW2 \\ 
  Zambia & EASTERN & 90-94 & 223.14 & 239.77 & 207.36 & HT-Direct \\ 
  Zambia & EASTERN & 90-94 & 233.48 & 220.22 & 246.87 & RW2 \\ 
  Zambia & EASTERN & 95-99 & 174.24 & 192.45 & 157.42 & HT-Direct \\ 
  Zambia & EASTERN & 95-99 & 212.28 & 196.76 & 227.70 & RW2 \\ 
  Zambia & EASTERN & 00-04 & 155.98 & 175.99 & 137.86 & HT-Direct \\ 
  Zambia & EASTERN & 00-04 & 174.63 & 160.20 & 189.88 & RW2 \\ 
  Zambia & EASTERN & 05-09 & 122.48 & 140.91 & 106.16 & HT-Direct \\ 
  Zambia & EASTERN & 05-09 & 129.50 & 115.13 & 145.49 & RW2 \\ 
  Zambia & EASTERN & 10-14 & 102.80 & 125.31 & 83.94 & HT-Direct \\ 
  Zambia & EASTERN & 10-14 & 101.32 & 85.70 & 120.46 & RW2 \\ 
  Zambia & EASTERN & 15-19 & 81.46 & 35.55 & 177.18 & RW2 \\ 
  Zambia & LUAPULA & 80-84 & 209.28 & 232.85 & 187.51 & HT-Direct \\ 
  Zambia & LUAPULA & 80-84 & 217.36 & 196.60 & 239.18 & RW2 \\ 
  Zambia & LUAPULA & 85-89 & 245.51 & 270.35 & 222.27 & HT-Direct \\ 
  Zambia & LUAPULA & 85-89 & 244.74 & 227.99 & 262.60 & RW2 \\ 
  Zambia & LUAPULA & 90-94 & 238.09 & 261.43 & 216.22 & HT-Direct \\ 
  Zambia & LUAPULA & 90-94 & 248.00 & 230.90 & 266.23 & RW2 \\ 
  Zambia & LUAPULA & 95-99 & 208.33 & 232.79 & 185.81 & HT-Direct \\ 
  Zambia & LUAPULA & 95-99 & 222.94 & 204.56 & 242.41 & RW2 \\ 
  Zambia & LUAPULA & 00-04 & 136.74 & 160.11 & 116.31 & HT-Direct \\ 
  Zambia & LUAPULA & 00-04 & 170.91 & 152.08 & 190.78 & RW2 \\ 
  Zambia & LUAPULA & 05-09 & 103.77 & 132.77 & 80.51 & HT-Direct \\ 
  Zambia & LUAPULA & 05-09 & 115.40 & 97.04 & 136.61 & RW2 \\ 
  Zambia & LUAPULA & 10-14 & 85.84 & 119.46 & 61.03 & HT-Direct \\ 
  Zambia & LUAPULA & 10-14 & 81.73 & 62.58 & 106.53 & RW2 \\ 
  Zambia & LUAPULA & 15-19 & 59.06 & 23.86 & 141.61 & RW2 \\ 
  Zambia & LUSAKA & 80-84 & 120.34 & 139.56 & 103.45 & HT-Direct \\ 
  Zambia & LUSAKA & 80-84 & 122.99 & 108.67 & 138.45 & RW2 \\ 
  Zambia & LUSAKA & 85-89 & 133.64 & 150.01 & 118.82 & HT-Direct \\ 
  Zambia & LUSAKA & 85-89 & 145.29 & 133.75 & 157.32 & RW2 \\ 
  Zambia & LUSAKA & 90-94 & 170.62 & 191.12 & 151.91 & HT-Direct \\ 
  Zambia & LUSAKA & 90-94 & 156.96 & 145.08 & 169.90 & RW2 \\ 
  Zambia & LUSAKA & 95-99 & 111.96 & 131.64 & 94.91 & HT-Direct \\ 
  Zambia & LUSAKA & 95-99 & 150.31 & 136.84 & 165.11 & RW2 \\ 
  Zambia & LUSAKA & 00-04 & 128.59 & 149.42 & 110.29 & HT-Direct \\ 
  Zambia & LUSAKA & 00-04 & 125.63 & 112.60 & 140.08 & RW2 \\ 
  Zambia & LUSAKA & 05-09 & 60.99 & 80.81 & 45.79 & HT-Direct \\ 
  Zambia & LUSAKA & 05-09 & 92.13 & 78.62 & 107.55 & RW2 \\ 
  Zambia & LUSAKA & 10-14 & 67.92 & 87.77 & 52.31 & HT-Direct \\ 
  Zambia & LUSAKA & 10-14 & 71.54 & 57.96 & 87.67 & RW2 \\ 
  Zambia & LUSAKA & 15-19 & 56.98 & 24.38 & 129.54 & RW2 \\ 
  Zambia & NORTH-WESTERN & 80-84 & 151.29 & 171.09 & 133.40 & HT-Direct \\ 
  Zambia & NORTH-WESTERN & 80-84 & 150.24 & 134.84 & 167.14 & RW2 \\ 
  Zambia & NORTH-WESTERN & 85-89 & 155.72 & 176.06 & 137.34 & HT-Direct \\ 
  Zambia & NORTH-WESTERN & 85-89 & 162.18 & 149.90 & 175.08 & RW2 \\ 
  Zambia & NORTH-WESTERN & 90-94 & 156.39 & 171.39 & 142.47 & HT-Direct \\ 
  Zambia & NORTH-WESTERN & 90-94 & 161.43 & 150.37 & 172.97 & RW2 \\ 
  Zambia & NORTH-WESTERN & 95-99 & 120.63 & 137.15 & 105.86 & HT-Direct \\ 
  Zambia & NORTH-WESTERN & 95-99 & 144.85 & 132.62 & 157.72 & RW2 \\ 
  Zambia & NORTH-WESTERN & 00-04 & 109.08 & 128.30 & 92.43 & HT-Direct \\ 
  Zambia & NORTH-WESTERN & 00-04 & 113.42 & 101.48 & 126.59 & RW2 \\ 
  Zambia & NORTH-WESTERN & 05-09 & 63.32 & 86.99 & 45.76 & HT-Direct \\ 
  Zambia & NORTH-WESTERN & 05-09 & 77.99 & 66.34 & 91.71 & RW2 \\ 
  Zambia & NORTH-WESTERN & 10-14 & 56.94 & 72.97 & 44.26 & HT-Direct \\ 
  Zambia & NORTH-WESTERN & 10-14 & 56.98 & 46.30 & 70.27 & RW2 \\ 
  Zambia & NORTH-WESTERN & 15-19 & 42.85 & 18.19 & 99.74 & RW2 \\ 
  Zambia & NORTHERN & 80-84 & 187.68 & 210.16 & 167.10 & HT-Direct \\ 
  Zambia & NORTHERN & 80-84 & 190.68 & 172.76 & 209.89 & RW2 \\ 
  Zambia & NORTHERN & 85-89 & 210.22 & 231.13 & 190.73 & HT-Direct \\ 
  Zambia & NORTHERN & 85-89 & 215.98 & 201.95 & 230.74 & RW2 \\ 
  Zambia & NORTHERN & 90-94 & 204.84 & 223.59 & 187.29 & HT-Direct \\ 
  Zambia & NORTHERN & 90-94 & 224.00 & 210.66 & 238.15 & RW2 \\ 
  Zambia & NORTHERN & 95-99 & 202.68 & 221.12 & 185.41 & HT-Direct \\ 
  Zambia & NORTHERN & 95-99 & 209.18 & 195.22 & 224.38 & RW2 \\ 
  Zambia & NORTHERN & 00-04 & 154.61 & 173.16 & 137.71 & HT-Direct \\ 
  Zambia & NORTHERN & 00-04 & 166.47 & 153.05 & 181.21 & RW2 \\ 
  Zambia & NORTHERN & 05-09 & 97.49 & 113.13 & 83.80 & HT-Direct \\ 
  Zambia & NORTHERN & 05-09 & 113.94 & 100.90 & 128.16 & RW2 \\ 
  Zambia & NORTHERN & 10-14 & 68.49 & 85.18 & 54.88 & HT-Direct \\ 
  Zambia & NORTHERN & 10-14 & 80.42 & 66.77 & 95.55 & RW2 \\ 
  Zambia & NORTHERN & 15-19 & 57.84 & 24.84 & 129.04 & RW2 \\ 
  Zambia & SOUTHERN & 80-84 & 133.64 & 152.85 & 116.51 & HT-Direct \\ 
  Zambia & SOUTHERN & 80-84 & 132.61 & 118.27 & 148.39 & RW2 \\ 
  Zambia & SOUTHERN & 85-89 & 140.17 & 156.59 & 125.21 & HT-Direct \\ 
  Zambia & SOUTHERN & 85-89 & 147.88 & 136.65 & 159.79 & RW2 \\ 
  Zambia & SOUTHERN & 90-94 & 151.73 & 170.32 & 134.84 & HT-Direct \\ 
  Zambia & SOUTHERN & 90-94 & 152.38 & 140.89 & 164.43 & RW2 \\ 
  Zambia & SOUTHERN & 95-99 & 122.63 & 141.13 & 106.25 & HT-Direct \\ 
  Zambia & SOUTHERN & 95-99 & 141.17 & 128.88 & 154.25 & RW2 \\ 
  Zambia & SOUTHERN & 00-04 & 101.32 & 116.45 & 87.96 & HT-Direct \\ 
  Zambia & SOUTHERN & 00-04 & 114.03 & 102.84 & 126.39 & RW2 \\ 
  Zambia & SOUTHERN & 05-09 & 65.83 & 83.22 & 51.88 & HT-Direct \\ 
  Zambia & SOUTHERN & 05-09 & 81.51 & 70.28 & 94.23 & RW2 \\ 
  Zambia & SOUTHERN & 10-14 & 64.05 & 80.96 & 50.48 & HT-Direct \\ 
  Zambia & SOUTHERN & 10-14 & 62.11 & 51.18 & 75.36 & RW2 \\ 
  Zambia & SOUTHERN & 15-19 & 48.71 & 20.96 & 111.45 & RW2 \\ 
  Zambia & WESTERN & 80-84 & 209.67 & 234.78 & 186.59 & HT-Direct \\ 
  Zambia & WESTERN & 80-84 & 208.63 & 188.23 & 230.49 & RW2 \\ 
  Zambia & WESTERN & 85-89 & 207.78 & 235.99 & 182.14 & HT-Direct \\ 
  Zambia & WESTERN & 85-89 & 217.38 & 200.70 & 234.89 & RW2 \\ 
  Zambia & WESTERN & 90-94 & 187.88 & 210.81 & 166.92 & HT-Direct \\ 
  Zambia & WESTERN & 90-94 & 211.43 & 196.01 & 227.72 & RW2 \\ 
  Zambia & WESTERN & 95-99 & 190.19 & 213.64 & 168.77 & HT-Direct \\ 
  Zambia & WESTERN & 95-99 & 186.99 & 171.42 & 204.08 & RW2 \\ 
  Zambia & WESTERN & 00-04 & 128.18 & 152.08 & 107.57 & HT-Direct \\ 
  Zambia & WESTERN & 00-04 & 141.85 & 126.27 & 159.09 & RW2 \\ 
  Zambia & WESTERN & 05-09 & 76.02 & 104.75 & 54.69 & HT-Direct \\ 
  Zambia & WESTERN & 05-09 & 92.84 & 77.56 & 110.72 & RW2 \\ 
  Zambia & WESTERN & 10-14 & 55.93 & 79.49 & 39.05 & HT-Direct \\ 
  Zambia & WESTERN & 10-14 & 63.51 & 48.98 & 81.41 & RW2 \\ 
  Zambia & WESTERN & 15-19 & 44.33 & 18.33 & 104.42 & RW2 \\ 
  Zimbabwe & ALL & 80-84 & 83.19 & 81.66 & 84.71 & IHME \\ 
  Zimbabwe & ALL & 80-84 & 94.31 & 86.38 & 102.91 & RW2 \\ 
  Zimbabwe & ALL & 80-84 & 94.36 & 91.13 & 97.64 & UN \\ 
  Zimbabwe & ALL & 85-89 & 69.14 & 68.10 & 70.27 & IHME \\ 
  Zimbabwe & ALL & 85-89 & 75.82 & 69.42 & 82.69 & RW2 \\ 
  Zimbabwe & ALL & 85-89 & 75.76 & 73.21 & 78.66 & UN \\ 
  Zimbabwe & ALL & 90-94 & 66.99 & 65.87 & 68.08 & IHME \\ 
  Zimbabwe & ALL & 90-94 & 83.27 & 75.84 & 91.36 & RW2 \\ 
  Zimbabwe & ALL & 90-94 & 83.14 & 80.33 & 86.01 & UN \\ 
  Zimbabwe & ALL & 95-99 & 74.12 & 72.87 & 75.48 & IHME \\ 
  Zimbabwe & ALL & 95-99 & 100.70 & 89.87 & 112.71 & RW2 \\ 
  Zimbabwe & ALL & 95-99 & 101.16 & 97.31 & 105.10 & UN \\ 
  Zimbabwe & ALL & 00-04 & 81.03 & 79.51 & 82.40 & IHME \\ 
  Zimbabwe & ALL & 00-04 & 104.27 & 92.34 & 117.61 & RW2 \\ 
  Zimbabwe & ALL & 00-04 & 104.55 & 100.33 & 109.08 & UN \\ 
  Zimbabwe & ALL & 05-09 & 81.90 & 79.93 & 83.63 & IHME \\ 
  Zimbabwe & ALL & 05-09 & 97.79 & 89.06 & 107.28 & RW2 \\ 
  Zimbabwe & ALL & 05-09 & 97.60 & 93.32 & 102.22 & UN \\ 
  Zimbabwe & ALL & 10-14 & 68.31 & 65.57 & 71.12 & IHME \\ 
  Zimbabwe & ALL & 10-14 & 89.65 & 30.17 & 235.98 & RW2 \\ 
  Zimbabwe & ALL & 10-14 & 81.03 & 73.89 & 88.62 & UN \\ 
  Zimbabwe & BULAWAYO & 80-84 & 56.78 & 80.95 & 39.52 & HT-Direct \\ 
  Zimbabwe & BULAWAYO & 80-84 & 57.19 & 42.20 & 77.19 & RW2 \\ 
  Zimbabwe & BULAWAYO & 85-89 & 37.93 & 56.14 & 25.46 & HT-Direct \\ 
  Zimbabwe & BULAWAYO & 85-89 & 46.86 & 36.77 & 59.54 & RW2 \\ 
  Zimbabwe & BULAWAYO & 90-94 & 41.02 & 60.73 & 27.52 & HT-Direct \\ 
  Zimbabwe & BULAWAYO & 90-94 & 55.60 & 44.28 & 69.69 & RW2 \\ 
  Zimbabwe & BULAWAYO & 95-99 & 77.88 & 105.46 & 57.06 & HT-Direct \\ 
  Zimbabwe & BULAWAYO & 95-99 & 77.64 & 61.27 & 98.21 & RW2 \\ 
  Zimbabwe & BULAWAYO & 00-04 & 73.18 & 105.05 & 50.44 & HT-Direct \\ 
  Zimbabwe & BULAWAYO & 00-04 & 76.29 & 58.43 & 99.21 & RW2 \\ 
  Zimbabwe & BULAWAYO & 05-09 & 48.23 & 64.27 & 36.04 & HT-Direct \\ 
  Zimbabwe & BULAWAYO & 05-09 & 64.13 & 47.44 & 86.20 & RW2 \\ 
  Zimbabwe & BULAWAYO & 10-14 & 53.07 & 16.78 & 160.27 & RW2 \\ 
  Zimbabwe & BULAWAYO & 15-19 & 43.35 & 2.57 & 463.19 & RW2 \\ 
  Zimbabwe & HARARE & 80-84 & 61.37 & 87.32 & 42.76 & HT-Direct \\ 
  Zimbabwe & HARARE & 80-84 & 68.29 & 51.50 & 89.95 & RW2 \\ 
  Zimbabwe & HARARE & 85-89 & 55.35 & 75.34 & 40.43 & HT-Direct \\ 
  Zimbabwe & HARARE & 85-89 & 57.53 & 46.29 & 71.19 & RW2 \\ 
  Zimbabwe & HARARE & 90-94 & 55.23 & 76.41 & 39.66 & HT-Direct \\ 
  Zimbabwe & HARARE & 90-94 & 68.99 & 56.38 & 84.34 & RW2 \\ 
  Zimbabwe & HARARE & 95-99 & 80.53 & 113.45 & 56.56 & HT-Direct \\ 
  Zimbabwe & HARARE & 95-99 & 96.65 & 77.48 & 120.57 & RW2 \\ 
  Zimbabwe & HARARE & 00-04 & 77.85 & 108.96 & 55.08 & HT-Direct \\ 
  Zimbabwe & HARARE & 00-04 & 96.16 & 74.29 & 123.64 & RW2 \\ 
  Zimbabwe & HARARE & 05-09 & 63.85 & 85.08 & 47.64 & HT-Direct \\ 
  Zimbabwe & HARARE & 05-09 & 82.78 & 61.47 & 110.32 & RW2 \\ 
  Zimbabwe & HARARE & 10-14 & 70.20 & 22.76 & 202.06 & RW2 \\ 
  Zimbabwe & HARARE & 15-19 & 59.55 & 3.63 & 541.77 & RW2 \\ 
  Zimbabwe & MANICALAND & 80-84 & 115.70 & 141.13 & 94.34 & HT-Direct \\ 
  Zimbabwe & MANICALAND & 80-84 & 111.94 & 93.48 & 133.71 & RW2 \\ 
  Zimbabwe & MANICALAND & 85-89 & 75.61 & 95.13 & 59.83 & HT-Direct \\ 
  Zimbabwe & MANICALAND & 85-89 & 93.95 & 80.70 & 109.12 & RW2 \\ 
  Zimbabwe & MANICALAND & 90-94 & 96.11 & 115.87 & 79.41 & HT-Direct \\ 
  Zimbabwe & MANICALAND & 90-94 & 112.84 & 96.86 & 130.65 & RW2 \\ 
  Zimbabwe & MANICALAND & 95-99 & 122.78 & 156.90 & 95.24 & HT-Direct \\ 
  Zimbabwe & MANICALAND & 95-99 & 157.80 & 131.55 & 188.56 & RW2 \\ 
  Zimbabwe & MANICALAND & 00-04 & 125.63 & 157.98 & 99.12 & HT-Direct \\ 
  Zimbabwe & MANICALAND & 00-04 & 160.39 & 130.72 & 195.71 & RW2 \\ 
  Zimbabwe & MANICALAND & 05-09 & 122.91 & 153.24 & 97.89 & HT-Direct \\ 
  Zimbabwe & MANICALAND & 05-09 & 142.10 & 112.54 & 177.86 & RW2 \\ 
  Zimbabwe & MANICALAND & 10-14 & 123.37 & 41.90 & 319.44 & RW2 \\ 
  Zimbabwe & MANICALAND & 15-19 & 106.67 & 6.88 & 688.63 & RW2 \\ 
  Zimbabwe & MASHONALAND CENTRAL & 80-84 & 143.98 & 167.49 & 123.28 & HT-Direct \\ 
  Zimbabwe & MASHONALAND CENTRAL & 80-84 & 144.76 & 125.40 & 166.79 & RW2 \\ 
  Zimbabwe & MASHONALAND CENTRAL & 85-89 & 96.14 & 118.66 & 77.52 & HT-Direct \\ 
  Zimbabwe & MASHONALAND CENTRAL & 85-89 & 110.65 & 96.27 & 126.98 & RW2 \\ 
  Zimbabwe & MASHONALAND CENTRAL & 90-94 & 96.26 & 120.80 & 76.28 & HT-Direct \\ 
  Zimbabwe & MASHONALAND CENTRAL & 90-94 & 118.62 & 101.74 & 137.69 & RW2 \\ 
  Zimbabwe & MASHONALAND CENTRAL & 95-99 & 111.69 & 140.45 & 88.21 & HT-Direct \\ 
  Zimbabwe & MASHONALAND CENTRAL & 95-99 & 148.64 & 124.45 & 176.39 & RW2 \\ 
  Zimbabwe & MASHONALAND CENTRAL & 00-04 & 104.70 & 133.12 & 81.77 & HT-Direct \\ 
  Zimbabwe & MASHONALAND CENTRAL & 00-04 & 135.94 & 111.26 & 165.03 & RW2 \\ 
  Zimbabwe & MASHONALAND CENTRAL & 05-09 & 94.18 & 114.24 & 77.34 & HT-Direct \\ 
  Zimbabwe & MASHONALAND CENTRAL & 05-09 & 108.00 & 87.77 & 132.19 & RW2 \\ 
  Zimbabwe & MASHONALAND CENTRAL & 10-14 & 83.94 & 28.27 & 231.48 & RW2 \\ 
  Zimbabwe & MASHONALAND CENTRAL & 15-19 & 65.22 & 4.00 & 558.63 & RW2 \\ 
  Zimbabwe & MASHONALAND EAST & 80-84 & 72.89 & 99.69 & 52.87 & HT-Direct \\ 
  Zimbabwe & MASHONALAND EAST & 80-84 & 86.79 & 68.40 & 109.15 & RW2 \\ 
  Zimbabwe & MASHONALAND EAST & 85-89 & 75.65 & 95.97 & 59.35 & HT-Direct \\ 
  Zimbabwe & MASHONALAND EAST & 85-89 & 74.52 & 62.27 & 88.99 & RW2 \\ 
  Zimbabwe & MASHONALAND EAST & 90-94 & 72.74 & 97.42 & 53.94 & HT-Direct \\ 
  Zimbabwe & MASHONALAND EAST & 90-94 & 90.61 & 76.77 & 106.79 & RW2 \\ 
  Zimbabwe & MASHONALAND EAST & 95-99 & 114.25 & 140.56 & 92.34 & HT-Direct \\ 
  Zimbabwe & MASHONALAND EAST & 95-99 & 127.98 & 107.46 & 152.02 & RW2 \\ 
  Zimbabwe & MASHONALAND EAST & 00-04 & 68.65 & 95.90 & 48.73 & HT-Direct \\ 
  Zimbabwe & MASHONALAND EAST & 00-04 & 129.48 & 105.37 & 157.67 & RW2 \\ 
  Zimbabwe & MASHONALAND EAST & 05-09 & 96.47 & 116.86 & 79.32 & HT-Direct \\ 
  Zimbabwe & MASHONALAND EAST & 05-09 & 113.82 & 92.45 & 139.48 & RW2 \\ 
  Zimbabwe & MASHONALAND EAST & 10-14 & 98.29 & 33.97 & 263.36 & RW2 \\ 
  Zimbabwe & MASHONALAND EAST & 15-19 & 84.35 & 5.55 & 627.45 & RW2 \\ 
  Zimbabwe & MASHONALAND WEST & 80-84 & 90.50 & 116.39 & 69.92 & HT-Direct \\ 
  Zimbabwe & MASHONALAND WEST & 80-84 & 109.32 & 90.15 & 131.52 & RW2 \\ 
  Zimbabwe & MASHONALAND WEST & 85-89 & 98.02 & 115.83 & 82.70 & HT-Direct \\ 
  Zimbabwe & MASHONALAND WEST & 85-89 & 92.03 & 79.95 & 105.35 & RW2 \\ 
  Zimbabwe & MASHONALAND WEST & 90-94 & 90.27 & 109.62 & 74.05 & HT-Direct \\ 
  Zimbabwe & MASHONALAND WEST & 90-94 & 108.46 & 94.43 & 124.26 & RW2 \\ 
  Zimbabwe & MASHONALAND WEST & 95-99 & 90.24 & 114.00 & 71.03 & HT-Direct \\ 
  Zimbabwe & MASHONALAND WEST & 95-99 & 148.28 & 125.64 & 174.55 & RW2 \\ 
  Zimbabwe & MASHONALAND WEST & 00-04 & 93.71 & 113.10 & 77.35 & HT-Direct \\ 
  Zimbabwe & MASHONALAND WEST & 00-04 & 147.78 & 122.51 & 177.38 & RW2 \\ 
  Zimbabwe & MASHONALAND WEST & 05-09 & 124.64 & 154.78 & 99.68 & HT-Direct \\ 
  Zimbabwe & MASHONALAND WEST & 05-09 & 129.14 & 103.69 & 159.74 & RW2 \\ 
  Zimbabwe & MASHONALAND WEST & 10-14 & 110.83 & 37.94 & 289.14 & RW2 \\ 
  Zimbabwe & MASHONALAND WEST & 15-19 & 94.23 & 6.08 & 661.02 & RW2 \\ 
  Zimbabwe & MASVINGO & 80-84 & 95.18 & 119.74 & 75.23 & HT-Direct \\ 
  Zimbabwe & MASVINGO & 80-84 & 97.23 & 80.40 & 116.95 & RW2 \\ 
  Zimbabwe & MASVINGO & 85-89 & 67.72 & 82.28 & 55.57 & HT-Direct \\ 
  Zimbabwe & MASVINGO & 85-89 & 75.27 & 64.95 & 87.13 & RW2 \\ 
  Zimbabwe & MASVINGO & 90-94 & 69.50 & 85.02 & 56.64 & HT-Direct \\ 
  Zimbabwe & MASVINGO & 90-94 & 83.29 & 71.93 & 96.37 & RW2 \\ 
  Zimbabwe & MASVINGO & 95-99 & 80.97 & 107.94 & 60.29 & HT-Direct \\ 
  Zimbabwe & MASVINGO & 95-99 & 108.23 & 90.50 & 129.47 & RW2 \\ 
  Zimbabwe & MASVINGO & 00-04 & 87.49 & 116.94 & 64.91 & HT-Direct \\ 
  Zimbabwe & MASVINGO & 00-04 & 100.73 & 81.96 & 123.09 & RW2 \\ 
  Zimbabwe & MASVINGO & 05-09 & 67.26 & 80.95 & 55.74 & HT-Direct \\ 
  Zimbabwe & MASVINGO & 05-09 & 80.86 & 66.06 & 98.65 & RW2 \\ 
  Zimbabwe & MASVINGO & 10-14 & 63.58 & 21.33 & 180.15 & RW2 \\ 
  Zimbabwe & MASVINGO & 15-19 & 49.57 & 3.12 & 486.81 & RW2 \\ 
  Zimbabwe & MATABELELAND NORTH & 80-84 & 100.95 & 129.63 & 78.04 & HT-Direct \\ 
  Zimbabwe & MATABELELAND NORTH & 80-84 & 95.93 & 77.69 & 118.28 & RW2 \\ 
  Zimbabwe & MATABELELAND NORTH & 85-89 & 65.79 & 82.38 & 52.35 & HT-Direct \\ 
  Zimbabwe & MATABELELAND NORTH & 85-89 & 74.03 & 62.50 & 87.59 & RW2 \\ 
  Zimbabwe & MATABELELAND NORTH & 90-94 & 57.62 & 80.42 & 40.99 & HT-Direct \\ 
  Zimbabwe & MATABELELAND NORTH & 90-94 & 81.95 & 68.44 & 97.49 & RW2 \\ 
  Zimbabwe & MATABELELAND NORTH & 95-99 & 69.66 & 95.20 & 50.59 & HT-Direct \\ 
  Zimbabwe & MATABELELAND NORTH & 95-99 & 107.35 & 87.37 & 130.98 & RW2 \\ 
  Zimbabwe & MATABELELAND NORTH & 00-04 & 95.51 & 134.91 & 66.74 & HT-Direct \\ 
  Zimbabwe & MATABELELAND NORTH & 00-04 & 101.50 & 80.18 & 127.79 & RW2 \\ 
  Zimbabwe & MATABELELAND NORTH & 05-09 & 73.65 & 93.36 & 57.84 & HT-Direct \\ 
  Zimbabwe & MATABELELAND NORTH & 05-09 & 83.37 & 64.54 & 106.81 & RW2 \\ 
  Zimbabwe & MATABELELAND NORTH & 10-14 & 67.05 & 21.78 & 190.59 & RW2 \\ 
  Zimbabwe & MATABELELAND NORTH & 15-19 & 54.33 & 3.39 & 504.17 & RW2 \\ 
  Zimbabwe & MATABELELAND SOUTH & 80-84 & 65.10 & 81.82 & 51.60 & HT-Direct \\ 
  Zimbabwe & MATABELELAND SOUTH & 80-84 & 68.33 & 55.45 & 83.79 & RW2 \\ 
  Zimbabwe & MATABELELAND SOUTH & 85-89 & 49.28 & 67.95 & 35.55 & HT-Direct \\ 
  Zimbabwe & MATABELELAND SOUTH & 85-89 & 55.58 & 46.28 & 66.47 & RW2 \\ 
  Zimbabwe & MATABELELAND SOUTH & 90-94 & 60.62 & 79.08 & 46.26 & HT-Direct \\ 
  Zimbabwe & MATABELELAND SOUTH & 90-94 & 65.09 & 54.37 & 77.75 & RW2 \\ 
  Zimbabwe & MATABELELAND SOUTH & 95-99 & 70.33 & 96.97 & 50.61 & HT-Direct \\ 
  Zimbabwe & MATABELELAND SOUTH & 95-99 & 89.49 & 72.42 & 110.10 & RW2 \\ 
  Zimbabwe & MATABELELAND SOUTH & 00-04 & 39.95 & 57.61 & 27.54 & HT-Direct \\ 
  Zimbabwe & MATABELELAND SOUTH & 00-04 & 87.56 & 68.27 & 111.61 & RW2 \\ 
  Zimbabwe & MATABELELAND SOUTH & 05-09 & 66.41 & 87.27 & 50.27 & HT-Direct \\ 
  Zimbabwe & MATABELELAND SOUTH & 05-09 & 74.18 & 56.12 & 97.70 & RW2 \\ 
  Zimbabwe & MATABELELAND SOUTH & 10-14 & 61.83 & 20.22 & 179.94 & RW2 \\ 
  Zimbabwe & MATABELELAND SOUTH & 15-19 & 51.27 & 3.12 & 508.54 & RW2 \\ 
  Zimbabwe & MIDLANDS & 80-84 & 98.47 & 120.91 & 79.82 & HT-Direct \\ 
  Zimbabwe & MIDLANDS & 80-84 & 99.51 & 82.92 & 118.72 & RW2 \\ 
  Zimbabwe & MIDLANDS & 85-89 & 66.85 & 84.91 & 52.41 & HT-Direct \\ 
  Zimbabwe & MIDLANDS & 85-89 & 79.19 & 67.94 & 92.05 & RW2 \\ 
  Zimbabwe & MIDLANDS & 90-94 & 82.64 & 100.54 & 67.69 & HT-Direct \\ 
  Zimbabwe & MIDLANDS & 90-94 & 89.63 & 77.53 & 103.26 & RW2 \\ 
  Zimbabwe & MIDLANDS & 95-99 & 80.85 & 100.77 & 64.59 & HT-Direct \\ 
  Zimbabwe & MIDLANDS & 95-99 & 118.64 & 100.19 & 140.11 & RW2 \\ 
  Zimbabwe & MIDLANDS & 00-04 & 75.77 & 95.00 & 60.18 & HT-Direct \\ 
  Zimbabwe & MIDLANDS & 00-04 & 113.26 & 93.30 & 137.10 & RW2 \\ 
  Zimbabwe & MIDLANDS & 05-09 & 83.25 & 100.92 & 68.43 & HT-Direct \\ 
  Zimbabwe & MIDLANDS & 05-09 & 94.13 & 76.82 & 114.67 & RW2 \\ 
  Zimbabwe & MIDLANDS & 10-14 & 76.81 & 25.89 & 211.80 & RW2 \\ 
  Zimbabwe & MIDLANDS & 15-19 & 62.08 & 3.97 & 549.10 & RW2 \\ 
  \hline
\caption{Complete results.} 
\label{fulltable}
\end{longtable}

}

\clearpage
\subsection{Table of All Results: 1-year Periods}
{\scriptsize
% latex table generated in R 3.4.3 by xtable 1.8-2 package
% Mon Apr  2 21:00:33 2018
\begin{longtable}{lllrrrl}
  \hline
Country & Region & Year & Median & Lower & Upper & Method \\ 
  \hline 
\endhead 
\hline 
{\footnotesize Continued on next page} 
\endfoot 
\endlastfoot 
Angola & ALL & 80-84 & 189.53 & 179.33 & 200.89 & IHME \\ 
  Angola & ALL & 80-84 & 251.28 & 192.05 & 322.16 & RW2 \\ 
  Angola & ALL & 80-84 & 232.95 & 214.36 & 255.00 & UN \\ 
  Angola & ALL & 85-89 & 187.44 & 179.24 & 196.32 & IHME \\ 
  Angola & ALL & 85-89 & 222.56 & 189.53 & 258.44 & RW2 \\ 
  Angola & ALL & 85-89 & 227.72 & 214.73 & 241.69 & UN \\ 
  Angola & ALL & 90-94 & 182.79 & 175.12 & 191.15 & IHME \\ 
  Angola & ALL & 90-94 & 226.02 & 200.25 & 254.31 & RW2 \\ 
  Angola & ALL & 90-94 & 227.00 & 215.43 & 238.75 & UN \\ 
  Angola & ALL & 95-99 & 161.43 & 154.52 & 168.39 & IHME \\ 
  Angola & ALL & 95-99 & 224.40 & 202.98 & 247.22 & RW2 \\ 
  Angola & ALL & 95-99 & 223.29 & 211.97 & 235.65 & UN \\ 
  Angola & ALL & 00-04 & 144.35 & 138.13 & 151.24 & IHME \\ 
  Angola & ALL & 00-04 & 212.56 & 185.32 & 242.76 & RW2 \\ 
  Angola & ALL & 00-04 & 212.39 & 198.39 & 226.68 & UN \\ 
  Angola & ALL & 05-09 & 113.54 & 107.89 & 119.83 & IHME \\ 
  Angola & ALL & 05-09 & 194.23 & 133.86 & 273.39 & RW2 \\ 
  Angola & ALL & 05-09 & 195.23 & 177.10 & 218.26 & UN \\ 
  Angola & ALL & 10-14 & 91.11 & 83.25 & 99.61 & IHME \\ 
  Angola & ALL & 10-14 & 173.94 & 150.46 & 200.03 & RW2 \\ 
  Angola & ALL & 10-14 & 173.83 & 144.34 & 207.56 & UN \\ 
  Angola & BENGO & 80-84 & 109.42 & 28.82 & 341.04 & RW2 \\ 
  Angola & BENGO & 85-89 & 98.46 & 430.20 & 15.55 & HT-Direct \\ 
  Angola & BENGO & 85-89 & 98.18 & 37.18 & 232.16 & RW2 \\ 
  Angola & BENGO & 90-94 & 37.81 & 145.36 & 9.00 & HT-Direct \\ 
  Angola & BENGO & 90-94 & 107.18 & 54.88 & 196.73 & RW2 \\ 
  Angola & BENGO & 95-99 & 99.96 & 213.76 & 43.40 & HT-Direct \\ 
  Angola & BENGO & 95-99 & 110.33 & 68.70 & 170.86 & RW2 \\ 
  Angola & BENGO & 00-04 & 131.53 & 220.77 & 74.90 & HT-Direct \\ 
  Angola & BENGO & 00-04 & 121.38 & 77.40 & 184.04 & RW2 \\ 
  Angola & BENGO & 05-09 & 66.52 & 113.31 & 38.22 & HT-Direct \\ 
  Angola & BENGO & 05-09 & 120.34 & 61.94 & 221.01 & RW2 \\ 
  Angola & BENGO & 10-14 & 28.58 & 52.50 & 15.39 & HT-Direct \\ 
  Angola & BENGO & 10-14 & 109.96 & 48.26 & 231.45 & RW2 \\ 
  Angola & BENGO & 15-19 & 99.26 & 19.43 & 382.74 & RW2 \\ 
  Angola & BENGUELA & 80-84 & 528.44 & 701.44 & 348.32 & HT-Direct \\ 
  Angola & BENGUELA & 80-84 & 382.03 & 295.80 & 472.34 & RW2 \\ 
  Angola & BENGUELA & 85-89 & 472.84 & 609.98 & 339.67 & HT-Direct \\ 
  Angola & BENGUELA & 85-89 & 371.05 & 308.05 & 437.98 & RW2 \\ 
  Angola & BENGUELA & 90-94 & 382.04 & 456.77 & 312.51 & HT-Direct \\ 
  Angola & BENGUELA & 90-94 & 399.95 & 350.79 & 452.10 & RW2 \\ 
  Angola & BENGUELA & 95-99 & 396.29 & 459.02 & 336.80 & HT-Direct \\ 
  Angola & BENGUELA & 95-99 & 391.60 & 345.92 & 441.97 & RW2 \\ 
  Angola & BENGUELA & 00-04 & 284.90 & 347.64 & 229.50 & HT-Direct \\ 
  Angola & BENGUELA & 00-04 & 388.81 & 331.42 & 450.89 & RW2 \\ 
  Angola & BENGUELA & 05-09 & 173.61 & 214.51 & 139.13 & HT-Direct \\ 
  Angola & BENGUELA & 05-09 & 356.35 & 259.79 & 466.03 & RW2 \\ 
  Angola & BENGUELA & 10-14 & 117.95 & 149.51 & 92.34 & HT-Direct \\ 
  Angola & BENGUELA & 10-14 & 306.23 & 239.98 & 380.60 & RW2 \\ 
  Angola & BENGUELA & 15-19 & 256.79 & 87.07 & 554.63 & RW2 \\ 
  Angola & BIÉ & 80-84 & 530.35 & 747.96 & 300.55 & HT-Direct \\ 
  Angola & BIÉ & 80-84 & 207.78 & 140.72 & 302.01 & RW2 \\ 
  Angola & BIÉ & 85-89 & 104.44 & 232.42 & 42.98 & HT-Direct \\ 
  Angola & BIÉ & 85-89 & 179.37 & 131.64 & 239.28 & RW2 \\ 
  Angola & BIÉ & 90-94 & 145.09 & 247.55 & 80.50 & HT-Direct \\ 
  Angola & BIÉ & 90-94 & 185.19 & 147.45 & 229.35 & RW2 \\ 
  Angola & BIÉ & 95-99 & 150.46 & 183.96 & 122.14 & HT-Direct \\ 
  Angola & BIÉ & 95-99 & 179.43 & 150.51 & 211.60 & RW2 \\ 
  Angola & BIÉ & 00-04 & 171.01 & 217.76 & 132.60 & HT-Direct \\ 
  Angola & BIÉ & 00-04 & 188.75 & 153.86 & 228.77 & RW2 \\ 
  Angola & BIÉ & 05-09 & 88.68 & 116.56 & 66.97 & HT-Direct \\ 
  Angola & BIÉ & 05-09 & 187.08 & 127.22 & 266.30 & RW2 \\ 
  Angola & BIÉ & 10-14 & 76.05 & 98.32 & 58.50 & HT-Direct \\ 
  Angola & BIÉ & 10-14 & 176.26 & 129.69 & 236.48 & RW2 \\ 
  Angola & BIÉ & 15-19 & 164.50 & 51.31 & 419.47 & RW2 \\ 
  Angola & CABINDA & 80-84 & 99.11 & 32.41 & 280.32 & RW2 \\ 
  Angola & CABINDA & 85-89 & 38.54 & 228.39 & 5.40 & HT-Direct \\ 
  Angola & CABINDA & 85-89 & 87.35 & 39.54 & 185.43 & RW2 \\ 
  Angola & CABINDA & 90-94 & 150.85 & 295.21 & 70.06 & HT-Direct \\ 
  Angola & CABINDA & 90-94 & 92.79 & 53.96 & 154.23 & RW2 \\ 
  Angola & CABINDA & 95-99 & 49.46 & 96.83 & 24.64 & HT-Direct \\ 
  Angola & CABINDA & 95-99 & 89.74 & 59.41 & 132.96 & RW2 \\ 
  Angola & CABINDA & 00-04 & 54.12 & 94.76 & 30.32 & HT-Direct \\ 
  Angola & CABINDA & 00-04 & 94.76 & 62.66 & 138.82 & RW2 \\ 
  Angola & CABINDA & 05-09 & 65.27 & 112.63 & 37.00 & HT-Direct \\ 
  Angola & CABINDA & 05-09 & 92.97 & 51.48 & 161.67 & RW2 \\ 
  Angola & CABINDA & 10-14 & 34.72 & 54.72 & 21.85 & HT-Direct \\ 
  Angola & CABINDA & 10-14 & 86.45 & 44.88 & 159.87 & RW2 \\ 
  Angola & CABINDA & 15-19 & 79.89 & 18.84 & 287.17 & RW2 \\ 
  Angola & CUANDO CUBANGO & 80-84 & 117.10 & 34.29 & 333.57 & RW2 \\ 
  Angola & CUANDO CUBANGO & 85-89 & 114.77 & 46.62 & 256.40 & RW2 \\ 
  Angola & CUANDO CUBANGO & 90-94 & 79.09 & 200.42 & 28.58 & HT-Direct \\ 
  Angola & CUANDO CUBANGO & 90-94 & 134.36 & 73.13 & 232.00 & RW2 \\ 
  Angola & CUANDO CUBANGO & 95-99 & 160.05 & 274.70 & 87.48 & HT-Direct \\ 
  Angola & CUANDO CUBANGO & 95-99 & 142.79 & 92.50 & 211.75 & RW2 \\ 
  Angola & CUANDO CUBANGO & 00-04 & 127.93 & 227.75 & 68.00 & HT-Direct \\ 
  Angola & CUANDO CUBANGO & 00-04 & 160.56 & 111.27 & 225.42 & RW2 \\ 
  Angola & CUANDO CUBANGO & 05-09 & 93.82 & 141.57 & 61.03 & HT-Direct \\ 
  Angola & CUANDO CUBANGO & 05-09 & 166.30 & 100.17 & 262.85 & RW2 \\ 
  Angola & CUANDO CUBANGO & 10-14 & 58.76 & 85.86 & 39.84 & HT-Direct \\ 
  Angola & CUANDO CUBANGO & 10-14 & 161.09 & 96.22 & 258.20 & RW2 \\ 
  Angola & CUANDO CUBANGO & 15-19 & 154.35 & 41.23 & 437.49 & RW2 \\ 
  Angola & CUANZA NORTE & 80-84 & 192.10 & 79.26 & 390.69 & RW2 \\ 
  Angola & CUANZA NORTE & 85-89 & 34.74 & 198.72 & 5.20 & HT-Direct \\ 
  Angola & CUANZA NORTE & 85-89 & 178.63 & 97.94 & 303.21 & RW2 \\ 
  Angola & CUANZA NORTE & 90-94 & 208.84 & 329.17 & 124.34 & HT-Direct \\ 
  Angola & CUANZA NORTE & 90-94 & 199.01 & 134.75 & 283.23 & RW2 \\ 
  Angola & CUANZA NORTE & 95-99 & 243.31 & 376.80 & 146.04 & HT-Direct \\ 
  Angola & CUANZA NORTE & 95-99 & 203.92 & 151.92 & 269.34 & RW2 \\ 
  Angola & CUANZA NORTE & 00-04 & 210.17 & 304.41 & 139.26 & HT-Direct \\ 
  Angola & CUANZA NORTE & 00-04 & 218.26 & 161.94 & 290.73 & RW2 \\ 
  Angola & CUANZA NORTE & 05-09 & 115.65 & 147.23 & 90.12 & HT-Direct \\ 
  Angola & CUANZA NORTE & 05-09 & 212.26 & 132.84 & 324.83 & RW2 \\ 
  Angola & CUANZA NORTE & 10-14 & 64.68 & 96.87 & 42.69 & HT-Direct \\ 
  Angola & CUANZA NORTE & 10-14 & 192.49 & 116.23 & 300.44 & RW2 \\ 
  Angola & CUANZA NORTE & 15-19 & 170.96 & 46.61 & 466.61 & RW2 \\ 
  Angola & CUANZA SUL & 80-84 & 238.79 & 689.39 & 42.46 & HT-Direct \\ 
  Angola & CUANZA SUL & 80-84 & 223.42 & 139.32 & 338.52 & RW2 \\ 
  Angola & CUANZA SUL & 85-89 & 263.21 & 392.14 & 165.15 & HT-Direct \\ 
  Angola & CUANZA SUL & 85-89 & 217.14 & 160.52 & 286.99 & RW2 \\ 
  Angola & CUANZA SUL & 90-94 & 268.58 & 376.47 & 182.56 & HT-Direct \\ 
  Angola & CUANZA SUL & 90-94 & 244.80 & 197.98 & 298.81 & RW2 \\ 
  Angola & CUANZA SUL & 95-99 & 225.53 & 285.73 & 174.91 & HT-Direct \\ 
  Angola & CUANZA SUL & 95-99 & 253.65 & 211.66 & 300.30 & RW2 \\ 
  Angola & CUANZA SUL & 00-04 & 201.75 & 284.21 & 138.59 & HT-Direct \\ 
  Angola & CUANZA SUL & 00-04 & 275.07 & 220.81 & 337.17 & RW2 \\ 
  Angola & CUANZA SUL & 05-09 & 173.43 & 207.23 & 144.15 & HT-Direct \\ 
  Angola & CUANZA SUL & 05-09 & 277.19 & 188.37 & 388.35 & RW2 \\ 
  Angola & CUANZA SUL & 10-14 & 111.11 & 154.93 & 78.53 & HT-Direct \\ 
  Angola & CUANZA SUL & 10-14 & 264.54 & 187.38 & 359.78 & RW2 \\ 
  Angola & CUANZA SUL & 15-19 & 248.55 & 80.73 & 557.06 & RW2 \\ 
  Angola & CUNENE & 80-84 & 114.44 & 46.21 & 244.93 & RW2 \\ 
  Angola & CUNENE & 85-89 & 26.46 & 158.80 & 3.90 & HT-Direct \\ 
  Angola & CUNENE & 85-89 & 119.82 & 68.96 & 196.42 & RW2 \\ 
  Angola & CUNENE & 90-94 & 162.50 & 228.99 & 112.50 & HT-Direct \\ 
  Angola & CUNENE & 90-94 & 147.94 & 110.37 & 195.19 & RW2 \\ 
  Angola & CUNENE & 95-99 & 139.97 & 182.02 & 106.36 & HT-Direct \\ 
  Angola & CUNENE & 95-99 & 161.94 & 131.41 & 197.84 & RW2 \\ 
  Angola & CUNENE & 00-04 & 178.73 & 242.58 & 128.83 & HT-Direct \\ 
  Angola & CUNENE & 00-04 & 183.52 & 140.49 & 237.25 & RW2 \\ 
  Angola & CUNENE & 05-09 & 100.54 & 153.70 & 64.36 & HT-Direct \\ 
  Angola & CUNENE & 05-09 & 188.07 & 117.10 & 288.73 & RW2 \\ 
  Angola & CUNENE & 10-14 & 59.55 & 89.76 & 39.07 & HT-Direct \\ 
  Angola & CUNENE & 10-14 & 179.08 & 108.03 & 282.03 & RW2 \\ 
  Angola & CUNENE & 15-19 & 167.24 & 44.99 & 460.88 & RW2 \\ 
  Angola & HUAMBO & 80-84 & 420.65 & 626.69 & 238.98 & HT-Direct \\ 
  Angola & HUAMBO & 80-84 & 301.64 & 221.49 & 394.39 & RW2 \\ 
  Angola & HUAMBO & 85-89 & 297.44 & 430.74 & 191.52 & HT-Direct \\ 
  Angola & HUAMBO & 85-89 & 276.24 & 218.44 & 342.88 & RW2 \\ 
  Angola & HUAMBO & 90-94 & 338.47 & 441.04 & 249.11 & HT-Direct \\ 
  Angola & HUAMBO & 90-94 & 287.52 & 238.45 & 342.94 & RW2 \\ 
  Angola & HUAMBO & 95-99 & 229.27 & 301.63 & 170.05 & HT-Direct \\ 
  Angola & HUAMBO & 95-99 & 270.72 & 225.77 & 322.69 & RW2 \\ 
  Angola & HUAMBO & 00-04 & 225.22 & 296.42 & 167.07 & HT-Direct \\ 
  Angola & HUAMBO & 00-04 & 263.53 & 211.78 & 323.65 & RW2 \\ 
  Angola & HUAMBO & 05-09 & 132.93 & 179.51 & 97.01 & HT-Direct \\ 
  Angola & HUAMBO & 05-09 & 236.06 & 160.68 & 333.28 & RW2 \\ 
  Angola & HUAMBO & 10-14 & 73.48 & 96.21 & 55.78 & HT-Direct \\ 
  Angola & HUAMBO & 10-14 & 197.53 & 143.12 & 265.31 & RW2 \\ 
  Angola & HUAMBO & 15-19 & 161.33 & 49.39 & 412.12 & RW2 \\ 
  Angola & HUÍLA & 80-84 & 489.05 & 924.26 & 69.83 & HT-Direct \\ 
  Angola & HUÍLA & 80-84 & 246.63 & 156.09 & 369.85 & RW2 \\ 
  Angola & HUÍLA & 85-89 & 268.10 & 401.89 & 166.45 & HT-Direct \\ 
  Angola & HUÍLA & 85-89 & 233.35 & 174.04 & 305.37 & RW2 \\ 
  Angola & HUÍLA & 90-94 & 248.32 & 328.09 & 182.68 & HT-Direct \\ 
  Angola & HUÍLA & 90-94 & 255.43 & 209.47 & 307.91 & RW2 \\ 
  Angola & HUÍLA & 95-99 & 247.66 & 322.29 & 185.58 & HT-Direct \\ 
  Angola & HUÍLA & 95-99 & 254.78 & 214.39 & 299.72 & RW2 \\ 
  Angola & HUÍLA & 00-04 & 185.56 & 228.04 & 149.46 & HT-Direct \\ 
  Angola & HUÍLA & 00-04 & 265.88 & 219.65 & 317.35 & RW2 \\ 
  Angola & HUÍLA & 05-09 & 140.70 & 182.53 & 107.20 & HT-Direct \\ 
  Angola & HUÍLA & 05-09 & 257.80 & 179.81 & 356.02 & RW2 \\ 
  Angola & HUÍLA & 10-14 & 99.42 & 129.59 & 75.66 & HT-Direct \\ 
  Angola & HUÍLA & 10-14 & 236.71 & 177.01 & 309.04 & RW2 \\ 
  Angola & HUÍLA & 15-19 & 214.04 & 69.37 & 498.84 & RW2 \\ 
  Angola & LUANDA & 80-84 & 336.21 & 625.39 & 133.20 & HT-Direct \\ 
  Angola & LUANDA & 80-84 & 191.71 & 117.26 & 294.76 & RW2 \\ 
  Angola & LUANDA & 85-89 & 129.35 & 262.95 & 58.27 & HT-Direct \\ 
  Angola & LUANDA & 85-89 & 164.54 & 115.98 & 226.65 & RW2 \\ 
  Angola & LUANDA & 90-94 & 145.78 & 219.95 & 93.62 & HT-Direct \\ 
  Angola & LUANDA & 90-94 & 169.20 & 132.42 & 213.09 & RW2 \\ 
  Angola & LUANDA & 95-99 & 170.36 & 217.74 & 131.55 & HT-Direct \\ 
  Angola & LUANDA & 95-99 & 160.13 & 131.39 & 194.23 & RW2 \\ 
  Angola & LUANDA & 00-04 & 111.93 & 144.81 & 85.76 & HT-Direct \\ 
  Angola & LUANDA & 00-04 & 157.62 & 123.93 & 199.09 & RW2 \\ 
  Angola & LUANDA & 05-09 & 73.93 & 103.88 & 52.11 & HT-Direct \\ 
  Angola & LUANDA & 05-09 & 139.66 & 88.33 & 213.93 & RW2 \\ 
  Angola & LUANDA & 10-14 & 40.39 & 59.59 & 27.20 & HT-Direct \\ 
  Angola & LUANDA & 10-14 & 114.17 & 70.24 & 179.05 & RW2 \\ 
  Angola & LUANDA & 15-19 & 91.90 & 23.73 & 293.55 & RW2 \\ 
  Angola & LUNDA NORTE & 80-84 & 193.46 & 83.79 & 387.91 & RW2 \\ 
  Angola & LUNDA NORTE & 85-89 & 171.61 & 351.28 & 73.44 & HT-Direct \\ 
  Angola & LUNDA NORTE & 85-89 & 173.83 & 103.34 & 278.07 & RW2 \\ 
  Angola & LUNDA NORTE & 90-94 & 211.78 & 348.48 & 118.92 & HT-Direct \\ 
  Angola & LUNDA NORTE & 90-94 & 183.91 & 129.87 & 253.73 & RW2 \\ 
  Angola & LUNDA NORTE & 95-99 & 167.23 & 244.71 & 110.68 & HT-Direct \\ 
  Angola & LUNDA NORTE & 95-99 & 176.81 & 132.62 & 231.84 & RW2 \\ 
  Angola & LUNDA NORTE & 00-04 & 115.43 & 185.27 & 69.66 & HT-Direct \\ 
  Angola & LUNDA NORTE & 00-04 & 180.85 & 129.36 & 245.83 & RW2 \\ 
  Angola & LUNDA NORTE & 05-09 & 62.90 & 107.24 & 36.15 & HT-Direct \\ 
  Angola & LUNDA NORTE & 05-09 & 173.75 & 104.84 & 273.15 & RW2 \\ 
  Angola & LUNDA NORTE & 10-14 & 66.14 & 99.90 & 43.25 & HT-Direct \\ 
  Angola & LUNDA NORTE & 10-14 & 158.73 & 93.56 & 256.28 & RW2 \\ 
  Angola & LUNDA NORTE & 15-19 & 143.19 & 37.82 & 423.32 & RW2 \\ 
  Angola & LUNDA SUL & 80-84 & 109.20 & 353.84 & 26.71 & HT-Direct \\ 
  Angola & LUNDA SUL & 80-84 & 108.90 & 53.92 & 204.54 & RW2 \\ 
  Angola & LUNDA SUL & 85-89 & 15.23 & 105.87 & 2.02 & HT-Direct \\ 
  Angola & LUNDA SUL & 85-89 & 105.99 & 63.36 & 171.73 & RW2 \\ 
  Angola & LUNDA SUL & 90-94 & 193.83 & 321.41 & 108.77 & HT-Direct \\ 
  Angola & LUNDA SUL & 90-94 & 123.01 & 84.78 & 174.77 & RW2 \\ 
  Angola & LUNDA SUL & 95-99 & 143.73 & 241.71 & 81.21 & HT-Direct \\ 
  Angola & LUNDA SUL & 95-99 & 127.56 & 93.35 & 172.49 & RW2 \\ 
  Angola & LUNDA SUL & 00-04 & 82.35 & 123.86 & 53.90 & HT-Direct \\ 
  Angola & LUNDA SUL & 00-04 & 139.75 & 101.44 & 189.40 & RW2 \\ 
  Angola & LUNDA SUL & 05-09 & 46.97 & 70.00 & 31.26 & HT-Direct \\ 
  Angola & LUNDA SUL & 05-09 & 141.73 & 86.66 & 222.70 & RW2 \\ 
  Angola & LUNDA SUL & 10-14 & 48.19 & 70.85 & 32.53 & HT-Direct \\ 
  Angola & LUNDA SUL & 10-14 & 136.08 & 82.95 & 215.43 & RW2 \\ 
  Angola & LUNDA SUL & 15-19 & 129.21 & 34.99 & 379.73 & RW2 \\ 
  Angola & MALANJE & 80-84 & 579.93 & 986.73 & 25.00 & HT-Direct \\ 
  Angola & MALANJE & 80-84 & 264.83 & 131.24 & 468.08 & RW2 \\ 
  Angola & MALANJE & 85-89 & 274.02 & 660.60 & 68.20 & HT-Direct \\ 
  Angola & MALANJE & 85-89 & 216.82 & 129.29 & 342.15 & RW2 \\ 
  Angola & MALANJE & 90-94 & 198.51 & 336.01 & 108.12 & HT-Direct \\ 
  Angola & MALANJE & 90-94 & 208.45 & 144.93 & 289.43 & RW2 \\ 
  Angola & MALANJE & 95-99 & 168.71 & 266.33 & 101.90 & HT-Direct \\ 
  Angola & MALANJE & 95-99 & 185.66 & 139.92 & 241.23 & RW2 \\ 
  Angola & MALANJE & 00-04 & 125.26 & 171.96 & 89.86 & HT-Direct \\ 
  Angola & MALANJE & 00-04 & 177.25 & 135.58 & 227.22 & RW2 \\ 
  Angola & MALANJE & 05-09 & 64.34 & 102.49 & 39.76 & HT-Direct \\ 
  Angola & MALANJE & 05-09 & 157.73 & 101.88 & 236.56 & RW2 \\ 
  Angola & MALANJE & 10-14 & 54.22 & 73.86 & 39.57 & HT-Direct \\ 
  Angola & MALANJE & 10-14 & 132.66 & 88.06 & 195.04 & RW2 \\ 
  Angola & MALANJE & 15-19 & 109.37 & 30.90 & 320.51 & RW2 \\ 
  Angola & MOXICO & 80-84 & 420.82 & 22.26 & 951.40 & RW2 \\ 
  Angola & MOXICO & 85-89 & 284.96 & 25.53 & 835.53 & RW2 \\ 
  Angola & MOXICO & 90-94 & 214.87 & 35.74 & 631.55 & RW2 \\ 
  Angola & MOXICO & 95-99 & 144.37 & 41.52 & 361.23 & RW2 \\ 
  Angola & MOXICO & 00-04 & 8.73 & 56.86 & 1.29 & HT-Direct \\ 
  Angola & MOXICO & 00-04 & 100.98 & 41.93 & 210.40 & RW2 \\ 
  Angola & MOXICO & 05-09 & 1.33 & 9.89 & 0.18 & HT-Direct \\ 
  Angola & MOXICO & 05-09 & 64.80 & 22.69 & 170.38 & RW2 \\ 
  Angola & MOXICO & 10-14 & 18.10 & 44.66 & 7.22 & HT-Direct \\ 
  Angola & MOXICO & 10-14 & 39.12 & 8.51 & 163.55 & RW2 \\ 
  Angola & MOXICO & 15-19 & 23.17 & 1.88 & 228.51 & RW2 \\ 
  Angola & NAMIBE & 80-84 & 291.47 & 586.81 & 106.47 & HT-Direct \\ 
  Angola & NAMIBE & 80-84 & 187.91 & 112.15 & 293.95 & RW2 \\ 
  Angola & NAMIBE & 85-89 & 150.73 & 320.17 & 62.69 & HT-Direct \\ 
  Angola & NAMIBE & 85-89 & 187.73 & 131.24 & 261.06 & RW2 \\ 
  Angola & NAMIBE & 90-94 & 223.44 & 324.81 & 146.82 & HT-Direct \\ 
  Angola & NAMIBE & 90-94 & 217.18 & 168.81 & 275.41 & RW2 \\ 
  Angola & NAMIBE & 95-99 & 218.53 & 294.91 & 157.51 & HT-Direct \\ 
  Angola & NAMIBE & 95-99 & 223.17 & 178.55 & 276.85 & RW2 \\ 
  Angola & NAMIBE & 00-04 & 183.89 & 259.15 & 126.74 & HT-Direct \\ 
  Angola & NAMIBE & 00-04 & 234.44 & 180.41 & 299.39 & RW2 \\ 
  Angola & NAMIBE & 05-09 & 132.11 & 187.89 & 91.03 & HT-Direct \\ 
  Angola & NAMIBE & 05-09 & 222.75 & 145.23 & 325.79 & RW2 \\ 
  Angola & NAMIBE & 10-14 & 71.44 & 100.25 & 50.44 & HT-Direct \\ 
  Angola & NAMIBE & 10-14 & 197.56 & 131.06 & 284.00 & RW2 \\ 
  Angola & NAMIBE & 15-19 & 170.59 & 49.46 & 448.96 & RW2 \\ 
  Angola & UÍGE & 80-84 & 694.11 & 878.49 & 415.95 & HT-Direct \\ 
  Angola & UÍGE & 80-84 & 390.28 & 277.44 & 524.09 & RW2 \\ 
  Angola & UÍGE & 85-89 & 431.39 & 609.64 & 269.31 & HT-Direct \\ 
  Angola & UÍGE & 85-89 & 293.74 & 222.37 & 376.10 & RW2 \\ 
  Angola & UÍGE & 90-94 & 182.55 & 285.37 & 111.02 & HT-Direct \\ 
  Angola & UÍGE & 90-94 & 254.00 & 195.77 & 318.83 & RW2 \\ 
  Angola & UÍGE & 95-99 & 157.17 & 237.29 & 100.54 & HT-Direct \\ 
  Angola & UÍGE & 95-99 & 210.63 & 162.59 & 263.11 & RW2 \\ 
  Angola & UÍGE & 00-04 & 155.05 & 205.94 & 114.91 & HT-Direct \\ 
  Angola & UÍGE & 00-04 & 190.29 & 146.06 & 242.81 & RW2 \\ 
  Angola & UÍGE & 05-09 & 100.91 & 150.54 & 66.36 & HT-Direct \\ 
  Angola & UÍGE & 05-09 & 160.43 & 103.89 & 238.66 & RW2 \\ 
  Angola & UÍGE & 10-14 & 58.50 & 78.62 & 43.29 & HT-Direct \\ 
  Angola & UÍGE & 10-14 & 126.76 & 86.46 & 183.65 & RW2 \\ 
  Angola & UÍGE & 15-19 & 97.68 & 27.61 & 292.41 & RW2 \\ 
  Angola & ZAIRE & 80-84 & 73.36 & 27.47 & 191.95 & RW2 \\ 
  Angola & ZAIRE & 85-89 & 76.72 & 204.35 & 26.18 & HT-Direct \\ 
  Angola & ZAIRE & 85-89 & 70.64 & 37.55 & 130.03 & RW2 \\ 
  Angola & ZAIRE & 90-94 & 67.85 & 134.50 & 32.97 & HT-Direct \\ 
  Angola & ZAIRE & 90-94 & 83.32 & 54.91 & 124.25 & RW2 \\ 
  Angola & ZAIRE & 95-99 & 85.15 & 130.15 & 54.73 & HT-Direct \\ 
  Angola & ZAIRE & 95-99 & 93.36 & 67.63 & 127.21 & RW2 \\ 
  Angola & ZAIRE & 00-04 & 111.82 & 175.54 & 69.29 & HT-Direct \\ 
  Angola & ZAIRE & 00-04 & 113.69 & 80.46 & 159.50 & RW2 \\ 
  Angola & ZAIRE & 05-09 & 78.01 & 128.45 & 46.32 & HT-Direct \\ 
  Angola & ZAIRE & 05-09 & 124.74 & 74.57 & 202.46 & RW2 \\ 
  Angola & ZAIRE & 10-14 & 44.31 & 64.82 & 30.08 & HT-Direct \\ 
  Angola & ZAIRE & 10-14 & 126.81 & 75.20 & 205.41 & RW2 \\ 
  Angola & ZAIRE & 15-19 & 126.76 & 32.44 & 383.78 & RW2 \\ 
  Benin & ALL & 80-84 & 213.87 & 210.96 & 216.86 & IHME \\ 
  Benin & ALL & 80-84 & 210.20 & 200.64 & 220.10 & RW2 \\ 
  Benin & ALL & 80-84 & 210.24 & 204.33 & 216.73 & UN \\ 
  Benin & ALL & 85-89 & 191.79 & 189.49 & 194.29 & IHME \\ 
  Benin & ALL & 85-89 & 192.37 & 184.32 & 200.65 & RW2 \\ 
  Benin & ALL & 85-89 & 192.27 & 186.66 & 198.04 & UN \\ 
  Benin & ALL & 90-94 & 172.46 & 170.20 & 174.63 & IHME \\ 
  Benin & ALL & 90-94 & 170.36 & 163.93 & 177.00 & RW2 \\ 
  Benin & ALL & 90-94 & 170.47 & 165.82 & 175.17 & UN \\ 
  Benin & ALL & 95-99 & 152.56 & 150.34 & 154.78 & IHME \\ 
  Benin & ALL & 95-99 & 153.61 & 147.78 & 159.58 & RW2 \\ 
  Benin & ALL & 95-99 & 153.58 & 149.23 & 158.09 & UN \\ 
  Benin & ALL & 00-04 & 130.49 & 128.21 & 132.69 & IHME \\ 
  Benin & ALL & 00-04 & 136.99 & 130.31 & 143.99 & RW2 \\ 
  Benin & ALL & 00-04 & 136.92 & 132.18 & 141.28 & UN \\ 
  Benin & ALL & 05-09 & 107.75 & 105.25 & 110.37 & IHME \\ 
  Benin & ALL & 05-09 & 118.12 & 103.83 & 134.08 & RW2 \\ 
  Benin & ALL & 05-09 & 118.35 & 113.03 & 124.29 & UN \\ 
  Benin & ALL & 10-14 & 88.79 & 85.44 & 92.52 & IHME \\ 
  Benin & ALL & 10-14 & 100.31 & 38.17 & 235.56 & RW2 \\ 
  Benin & ALL & 10-14 & 106.29 & 98.52 & 115.37 & UN \\ 
  Benin & ATACORA & 80-84 & 290.00 & 315.66 & 265.61 & HT-Direct \\ 
  Benin & ATACORA & 80-84 & 269.43 & 250.07 & 289.97 & RW2 \\ 
  Benin & ATACORA & 85-89 & 237.50 & 256.40 & 219.59 & HT-Direct \\ 
  Benin & ATACORA & 85-89 & 239.27 & 225.75 & 252.98 & RW2 \\ 
  Benin & ATACORA & 90-94 & 205.36 & 222.70 & 189.04 & HT-Direct \\ 
  Benin & ATACORA & 90-94 & 211.46 & 199.56 & 223.69 & RW2 \\ 
  Benin & ATACORA & 95-99 & 202.99 & 219.10 & 187.77 & HT-Direct \\ 
  Benin & ATACORA & 95-99 & 192.08 & 181.34 & 203.35 & RW2 \\ 
  Benin & ATACORA & 00-04 & 170.83 & 190.14 & 153.11 & HT-Direct \\ 
  Benin & ATACORA & 00-04 & 173.59 & 159.91 & 188.37 & RW2 \\ 
  Benin & ATACORA & 05-09 & 139.52 & 173.80 & 111.09 & HT-Direct \\ 
  Benin & ATACORA & 05-09 & 150.79 & 126.74 & 178.95 & RW2 \\ 
  Benin & ATACORA & 10-14 & 128.39 & 54.04 & 274.00 & RW2 \\ 
  Benin & ATACORA & 15-19 & 108.71 & 11.83 & 546.08 & RW2 \\ 
  Benin & ATLANTIQUE & 80-84 & 179.55 & 201.53 & 159.48 & HT-Direct \\ 
  Benin & ATLANTIQUE & 80-84 & 171.63 & 155.85 & 188.66 & RW2 \\ 
  Benin & ATLANTIQUE & 85-89 & 163.21 & 183.00 & 145.19 & HT-Direct \\ 
  Benin & ATLANTIQUE & 85-89 & 153.80 & 142.83 & 165.58 & RW2 \\ 
  Benin & ATLANTIQUE & 90-94 & 126.24 & 139.82 & 113.80 & HT-Direct \\ 
  Benin & ATLANTIQUE & 90-94 & 134.31 & 125.47 & 143.42 & RW2 \\ 
  Benin & ATLANTIQUE & 95-99 & 124.64 & 136.40 & 113.77 & HT-Direct \\ 
  Benin & ATLANTIQUE & 95-99 & 121.24 & 113.57 & 129.31 & RW2 \\ 
  Benin & ATLANTIQUE & 00-04 & 115.02 & 128.93 & 102.43 & HT-Direct \\ 
  Benin & ATLANTIQUE & 00-04 & 111.03 & 101.64 & 121.36 & RW2 \\ 
  Benin & ATLANTIQUE & 05-09 & 80.50 & 104.68 & 61.52 & HT-Direct \\ 
  Benin & ATLANTIQUE & 05-09 & 98.12 & 81.56 & 118.03 & RW2 \\ 
  Benin & ATLANTIQUE & 10-14 & 85.53 & 35.34 & 193.84 & RW2 \\ 
  Benin & ATLANTIQUE & 15-19 & 73.90 & 8.03 & 435.52 & RW2 \\ 
  Benin & BORGOU & 80-84 & 213.43 & 239.38 & 189.59 & HT-Direct \\ 
  Benin & BORGOU & 80-84 & 207.22 & 187.95 & 227.75 & RW2 \\ 
  Benin & BORGOU & 85-89 & 194.27 & 216.02 & 174.23 & HT-Direct \\ 
  Benin & BORGOU & 85-89 & 195.00 & 181.62 & 209.18 & RW2 \\ 
  Benin & BORGOU & 90-94 & 184.89 & 203.83 & 167.33 & HT-Direct \\ 
  Benin & BORGOU & 90-94 & 179.74 & 168.23 & 192.27 & RW2 \\ 
  Benin & BORGOU & 95-99 & 174.96 & 191.97 & 159.17 & HT-Direct \\ 
  Benin & BORGOU & 95-99 & 166.02 & 155.09 & 177.78 & RW2 \\ 
  Benin & BORGOU & 00-04 & 141.96 & 163.68 & 122.70 & HT-Direct \\ 
  Benin & BORGOU & 00-04 & 150.40 & 135.61 & 166.29 & RW2 \\ 
  Benin & BORGOU & 05-09 & 110.61 & 145.48 & 83.29 & HT-Direct \\ 
  Benin & BORGOU & 05-09 & 130.48 & 105.74 & 158.21 & RW2 \\ 
  Benin & BORGOU & 10-14 & 111.02 & 45.50 & 243.60 & RW2 \\ 
  Benin & BORGOU & 15-19 & 93.29 & 10.08 & 505.17 & RW2 \\ 
  Benin & MONO & 80-84 & 204.65 & 231.64 & 180.07 & HT-Direct \\ 
  Benin & MONO & 80-84 & 200.29 & 180.72 & 221.16 & RW2 \\ 
  Benin & MONO & 85-89 & 182.97 & 203.22 & 164.33 & HT-Direct \\ 
  Benin & MONO & 85-89 & 178.87 & 166.30 & 192.17 & RW2 \\ 
  Benin & MONO & 90-94 & 155.61 & 170.84 & 141.52 & HT-Direct \\ 
  Benin & MONO & 90-94 & 154.41 & 144.56 & 165.12 & RW2 \\ 
  Benin & MONO & 95-99 & 141.13 & 157.00 & 126.62 & HT-Direct \\ 
  Benin & MONO & 95-99 & 134.49 & 125.12 & 144.42 & RW2 \\ 
  Benin & MONO & 00-04 & 112.33 & 127.30 & 98.92 & HT-Direct \\ 
  Benin & MONO & 00-04 & 116.72 & 105.78 & 128.66 & RW2 \\ 
  Benin & MONO & 05-09 & 89.04 & 121.94 & 64.37 & HT-Direct \\ 
  Benin & MONO & 05-09 & 97.56 & 78.95 & 119.27 & RW2 \\ 
  Benin & MONO & 10-14 & 79.99 & 32.10 & 185.68 & RW2 \\ 
  Benin & MONO & 15-19 & 65.76 & 6.86 & 410.35 & RW2 \\ 
  Benin & OUEME & 80-84 & 212.06 & 240.44 & 186.20 & HT-Direct \\ 
  Benin & OUEME & 80-84 & 204.85 & 185.25 & 226.06 & RW2 \\ 
  Benin & OUEME & 85-89 & 194.26 & 213.37 & 176.48 & HT-Direct \\ 
  Benin & OUEME & 85-89 & 187.77 & 175.45 & 200.89 & RW2 \\ 
  Benin & OUEME & 90-94 & 165.29 & 179.90 & 151.64 & HT-Direct \\ 
  Benin & OUEME & 90-94 & 167.77 & 157.66 & 178.20 & RW2 \\ 
  Benin & OUEME & 95-99 & 160.48 & 176.84 & 145.37 & HT-Direct \\ 
  Benin & OUEME & 95-99 & 154.00 & 143.99 & 164.48 & RW2 \\ 
  Benin & OUEME & 00-04 & 138.31 & 154.71 & 123.40 & HT-Direct \\ 
  Benin & OUEME & 00-04 & 143.47 & 131.56 & 156.42 & RW2 \\ 
  Benin & OUEME & 05-09 & 130.33 & 162.18 & 103.96 & HT-Direct \\ 
  Benin & OUEME & 05-09 & 130.33 & 109.21 & 155.32 & RW2 \\ 
  Benin & OUEME & 10-14 & 116.77 & 48.78 & 253.84 & RW2 \\ 
  Benin & OUEME & 15-19 & 104.58 & 11.75 & 530.27 & RW2 \\ 
  Benin & ZOU & 80-84 & 233.74 & 260.97 & 208.55 & HT-Direct \\ 
  Benin & ZOU & 80-84 & 224.20 & 205.09 & 244.77 & RW2 \\ 
  Benin & ZOU & 85-89 & 201.71 & 224.15 & 181.00 & HT-Direct \\ 
  Benin & ZOU & 85-89 & 198.81 & 185.78 & 212.65 & RW2 \\ 
  Benin & ZOU & 90-94 & 179.20 & 194.89 & 164.52 & HT-Direct \\ 
  Benin & ZOU & 90-94 & 172.31 & 162.59 & 182.48 & RW2 \\ 
  Benin & ZOU & 95-99 & 144.26 & 156.87 & 132.50 & HT-Direct \\ 
  Benin & ZOU & 95-99 & 151.86 & 143.11 & 160.93 & RW2 \\ 
  Benin & ZOU & 00-04 & 147.49 & 163.21 & 133.04 & HT-Direct \\ 
  Benin & ZOU & 00-04 & 135.44 & 124.76 & 146.93 & RW2 \\ 
  Benin & ZOU & 05-09 & 82.36 & 113.91 & 58.97 & HT-Direct \\ 
  Benin & ZOU & 05-09 & 115.79 & 96.90 & 137.97 & RW2 \\ 
  Benin & ZOU & 10-14 & 97.41 & 40.36 & 216.81 & RW2 \\ 
  Benin & ZOU & 15-19 & 82.14 & 8.90 & 460.26 & RW2 \\ 
  Burkina Faso & ALL & 80-84 & 242.41 & 239.49 & 245.15 & IHME \\ 
  Burkina Faso & ALL & 80-84 & 232.08 & 222.75 & 241.67 & RW2 \\ 
  Burkina Faso & ALL & 80-84 & 232.03 & 224.90 & 238.70 & UN \\ 
  Burkina Faso & ALL & 85-89 & 218.13 & 215.86 & 220.79 & IHME \\ 
  Burkina Faso & ALL & 85-89 & 209.98 & 202.20 & 217.93 & RW2 \\ 
  Burkina Faso & ALL & 85-89 & 210.07 & 204.38 & 216.27 & UN \\ 
  Burkina Faso & ALL & 90-94 & 202.65 & 200.30 & 204.84 & IHME \\ 
  Burkina Faso & ALL & 90-94 & 201.84 & 195.42 & 208.43 & RW2 \\ 
  Burkina Faso & ALL & 90-94 & 201.77 & 196.36 & 207.54 & UN \\ 
  Burkina Faso & ALL & 95-99 & 188.88 & 186.70 & 191.42 & IHME \\ 
  Burkina Faso & ALL & 95-99 & 193.74 & 187.24 & 200.36 & RW2 \\ 
  Burkina Faso & ALL & 95-99 & 193.79 & 188.27 & 200.09 & UN \\ 
  Burkina Faso & ALL & 00-04 & 168.26 & 165.68 & 170.79 & IHME \\ 
  Burkina Faso & ALL & 00-04 & 176.96 & 169.51 & 184.71 & RW2 \\ 
  Burkina Faso & ALL & 00-04 & 176.96 & 170.60 & 183.88 & UN \\ 
  Burkina Faso & ALL & 05-09 & 144.04 & 140.79 & 147.39 & IHME \\ 
  Burkina Faso & ALL & 05-09 & 139.53 & 131.41 & 148.06 & RW2 \\ 
  Burkina Faso & ALL & 05-09 & 139.54 & 133.63 & 145.72 & UN \\ 
  Burkina Faso & ALL & 10-14 & 122.58 & 118.41 & 127.04 & IHME \\ 
  Burkina Faso & ALL & 10-14 & 103.58 & 40.39 & 236.89 & RW2 \\ 
  Burkina Faso & ALL & 10-14 & 102.06 & 93.11 & 111.53 & UN \\ 
  Burkina Faso & CENTRAL/SOUTH & 80-84 & 235.28 & 252.06 & 219.28 & HT-Direct \\ 
  Burkina Faso & CENTRAL/SOUTH & 80-84 & 216.67 & 202.70 & 231.13 & RW2 \\ 
  Burkina Faso & CENTRAL/SOUTH & 85-89 & 187.11 & 199.35 & 175.46 & HT-Direct \\ 
  Burkina Faso & CENTRAL/SOUTH & 85-89 & 203.32 & 193.13 & 214.06 & RW2 \\ 
  Burkina Faso & CENTRAL/SOUTH & 90-94 & 180.67 & 191.09 & 170.69 & HT-Direct \\ 
  Burkina Faso & CENTRAL/SOUTH & 90-94 & 201.46 & 191.83 & 211.50 & RW2 \\ 
  Burkina Faso & CENTRAL/SOUTH & 95-99 & 177.16 & 187.85 & 166.95 & HT-Direct \\ 
  Burkina Faso & CENTRAL/SOUTH & 95-99 & 195.31 & 185.52 & 205.52 & RW2 \\ 
  Burkina Faso & CENTRAL/SOUTH & 00-04 & 145.59 & 156.55 & 135.28 & HT-Direct \\ 
  Burkina Faso & CENTRAL/SOUTH & 00-04 & 179.66 & 168.84 & 191.14 & RW2 \\ 
  Burkina Faso & CENTRAL/SOUTH & 05-09 & 109.25 & 121.11 & 98.43 & HT-Direct \\ 
  Burkina Faso & CENTRAL/SOUTH & 05-09 & 142.91 & 129.38 & 157.46 & RW2 \\ 
  Burkina Faso & CENTRAL/SOUTH & 10-14 & 107.21 & 47.01 & 226.55 & RW2 \\ 
  Burkina Faso & CENTRAL/SOUTH & 15-19 & 78.90 & 8.82 & 443.68 & RW2 \\ 
  Burkina Faso & EAST & 80-84 & 263.51 & 287.21 & 241.11 & HT-Direct \\ 
  Burkina Faso & EAST & 80-84 & 257.92 & 239.19 & 277.88 & RW2 \\ 
  Burkina Faso & EAST & 85-89 & 222.35 & 242.60 & 203.34 & HT-Direct \\ 
  Burkina Faso & EAST & 85-89 & 226.38 & 213.11 & 240.16 & RW2 \\ 
  Burkina Faso & EAST & 90-94 & 213.19 & 228.46 & 198.68 & HT-Direct \\ 
  Burkina Faso & EAST & 90-94 & 210.81 & 200.04 & 221.98 & RW2 \\ 
  Burkina Faso & EAST & 95-99 & 193.41 & 207.86 & 179.75 & HT-Direct \\ 
  Burkina Faso & EAST & 95-99 & 194.16 & 183.89 & 204.83 & RW2 \\ 
  Burkina Faso & EAST & 00-04 & 168.41 & 187.29 & 151.09 & HT-Direct \\ 
  Burkina Faso & EAST & 00-04 & 170.34 & 158.43 & 183.07 & RW2 \\ 
  Burkina Faso & EAST & 05-09 & 125.02 & 142.50 & 109.40 & HT-Direct \\ 
  Burkina Faso & EAST & 05-09 & 129.97 & 115.93 & 145.58 & RW2 \\ 
  Burkina Faso & EAST & 10-14 & 93.32 & 39.90 & 201.19 & RW2 \\ 
  Burkina Faso & EAST & 15-19 & 66.11 & 7.28 & 403.03 & RW2 \\ 
  Burkina Faso & NORTH & 80-84 & 233.39 & 257.81 & 210.63 & HT-Direct \\ 
  Burkina Faso & NORTH & 80-84 & 224.74 & 211.26 & 238.88 & RW2 \\ 
  Burkina Faso & NORTH & 85-89 & 229.12 & 246.29 & 212.81 & HT-Direct \\ 
  Burkina Faso & NORTH & 85-89 & 195.25 & 185.86 & 204.99 & RW2 \\ 
  Burkina Faso & NORTH & 90-94 & 231.48 & 248.44 & 215.35 & HT-Direct \\ 
  Burkina Faso & NORTH & 90-94 & 182.83 & 174.85 & 191.00 & RW2 \\ 
  Burkina Faso & NORTH & 95-99 & 234.46 & 251.92 & 217.86 & HT-Direct \\ 
  Burkina Faso & NORTH & 95-99 & 170.71 & 163.02 & 178.72 & RW2 \\ 
  Burkina Faso & NORTH & 00-04 & 228.79 & 248.25 & 210.42 & HT-Direct \\ 
  Burkina Faso & NORTH & 00-04 & 150.03 & 141.69 & 158.90 & RW2 \\ 
  Burkina Faso & NORTH & 05-09 & 183.59 & 203.35 & 165.35 & HT-Direct \\ 
  Burkina Faso & NORTH & 05-09 & 114.15 & 104.13 & 125.00 & RW2 \\ 
  Burkina Faso & NORTH & 10-14 & 81.85 & 35.12 & 177.19 & RW2 \\ 
  Burkina Faso & NORTH & 15-19 & 57.96 & 6.36 & 367.23 & RW2 \\ 
  Burkina Faso & WEST & 80-84 & 219.42 & 236.24 & 203.49 & HT-Direct \\ 
  Burkina Faso & WEST & 80-84 & 233.50 & 215.16 & 252.69 & RW2 \\ 
  Burkina Faso & WEST & 85-89 & 200.83 & 214.64 & 187.69 & HT-Direct \\ 
  Burkina Faso & WEST & 85-89 & 226.80 & 214.52 & 239.47 & RW2 \\ 
  Burkina Faso & WEST & 90-94 & 205.41 & 218.87 & 192.57 & HT-Direct \\ 
  Burkina Faso & WEST & 90-94 & 232.52 & 221.20 & 244.37 & RW2 \\ 
  Burkina Faso & WEST & 95-99 & 191.78 & 205.57 & 178.70 & HT-Direct \\ 
  Burkina Faso & WEST & 95-99 & 235.74 & 224.02 & 247.81 & RW2 \\ 
  Burkina Faso & WEST & 00-04 & 181.98 & 197.44 & 167.48 & HT-Direct \\ 
  Burkina Faso & WEST & 00-04 & 228.14 & 214.80 & 242.14 & RW2 \\ 
  Burkina Faso & WEST & 05-09 & 133.38 & 148.88 & 119.26 & HT-Direct \\ 
  Burkina Faso & WEST & 05-09 & 193.66 & 176.90 & 211.83 & RW2 \\ 
  Burkina Faso & WEST & 10-14 & 155.58 & 70.09 & 308.32 & RW2 \\ 
  Burkina Faso & WEST & 15-19 & 123.44 & 14.09 & 570.02 & RW2 \\ 
  Burundi & ALL & 80-84 & 202.90 & 196.35 & 208.65 & IHME \\ 
  Burundi & ALL & 80-84 & 202.14 & 163.91 & 246.94 & RW2 \\ 
  Burundi & ALL & 80-84 & 201.57 & 193.32 & 211.06 & UN \\ 
  Burundi & ALL & 85-89 & 174.26 & 169.47 & 179.73 & IHME \\ 
  Burundi & ALL & 85-89 & 172.52 & 152.02 & 194.58 & RW2 \\ 
  Burundi & ALL & 85-89 & 172.35 & 164.06 & 180.82 & UN \\ 
  Burundi & ALL & 90-94 & 174.19 & 169.16 & 179.53 & IHME \\ 
  Burundi & ALL & 90-94 & 173.31 & 158.82 & 188.95 & RW2 \\ 
  Burundi & ALL & 90-94 & 173.74 & 165.42 & 182.23 & UN \\ 
  Burundi & ALL & 95-99 & 172.33 & 167.23 & 177.79 & IHME \\ 
  Burundi & ALL & 95-99 & 164.52 & 153.42 & 176.14 & RW2 \\ 
  Burundi & ALL & 95-99 & 164.77 & 156.21 & 173.93 & UN \\ 
  Burundi & ALL & 00-04 & 149.22 & 144.17 & 154.66 & IHME \\ 
  Burundi & ALL & 00-04 & 143.59 & 135.13 & 152.53 & RW2 \\ 
  Burundi & ALL & 00-04 & 142.94 & 134.94 & 151.10 & UN \\ 
  Burundi & ALL & 05-09 & 99.50 & 95.67 & 103.48 & IHME \\ 
  Burundi & ALL & 05-09 & 114.07 & 103.87 & 125.14 & RW2 \\ 
  Burundi & ALL & 05-09 & 114.34 & 104.29 & 124.91 & UN \\ 
  Burundi & ALL & 10-14 & 82.36 & 74.46 & 90.07 & IHME \\ 
  Burundi & ALL & 10-14 & 87.32 & 31.01 & 218.25 & RW2 \\ 
  Burundi & ALL & 10-14 & 92.23 & 78.38 & 109.08 & UN \\ 
  Burundi & BUJUMBURA & 80-84 & 237.56 & 389.25 & 132.18 & HT-Direct \\ 
  Burundi & BUJUMBURA & 80-84 & 186.14 & 118.57 & 288.34 & RW2 \\ 
  Burundi & BUJUMBURA & 85-89 & 138.40 & 240.16 & 75.47 & HT-Direct \\ 
  Burundi & BUJUMBURA & 85-89 & 133.40 & 96.33 & 181.29 & RW2 \\ 
  Burundi & BUJUMBURA & 90-94 & 101.43 & 154.75 & 65.07 & HT-Direct \\ 
  Burundi & BUJUMBURA & 90-94 & 114.59 & 87.59 & 146.46 & RW2 \\ 
  Burundi & BUJUMBURA & 95-99 & 148.69 & 218.95 & 98.14 & HT-Direct \\ 
  Burundi & BUJUMBURA & 95-99 & 98.61 & 78.39 & 122.21 & RW2 \\ 
  Burundi & BUJUMBURA & 00-04 & 86.48 & 122.71 & 60.22 & HT-Direct \\ 
  Burundi & BUJUMBURA & 00-04 & 77.61 & 62.65 & 95.46 & RW2 \\ 
  Burundi & BUJUMBURA & 05-09 & 56.89 & 77.28 & 41.63 & HT-Direct \\ 
  Burundi & BUJUMBURA & 05-09 & 58.74 & 43.17 & 80.34 & RW2 \\ 
  Burundi & BUJUMBURA & 10-14 & 44.27 & 14.32 & 134.58 & RW2 \\ 
  Burundi & BUJUMBURA & 15-19 & 33.49 & 2.36 & 358.63 & RW2 \\ 
  Burundi & CENTRE-EAST & 80-84 & 313.83 & 457.10 & 199.00 & HT-Direct \\ 
  Burundi & CENTRE-EAST & 80-84 & 203.27 & 146.40 & 285.79 & RW2 \\ 
  Burundi & CENTRE-EAST & 85-89 & 152.09 & 198.98 & 114.67 & HT-Direct \\ 
  Burundi & CENTRE-EAST & 85-89 & 161.60 & 134.30 & 193.81 & RW2 \\ 
  Burundi & CENTRE-EAST & 90-94 & 168.81 & 200.59 & 141.17 & HT-Direct \\ 
  Burundi & CENTRE-EAST & 90-94 & 154.14 & 135.59 & 173.90 & RW2 \\ 
  Burundi & CENTRE-EAST & 95-99 & 180.82 & 209.82 & 155.04 & HT-Direct \\ 
  Burundi & CENTRE-EAST & 95-99 & 148.11 & 133.88 & 163.11 & RW2 \\ 
  Burundi & CENTRE-EAST & 00-04 & 179.53 & 207.30 & 154.75 & HT-Direct \\ 
  Burundi & CENTRE-EAST & 00-04 & 134.06 & 121.55 & 148.22 & RW2 \\ 
  Burundi & CENTRE-EAST & 05-09 & 92.76 & 108.54 & 79.08 & HT-Direct \\ 
  Burundi & CENTRE-EAST & 05-09 & 109.01 & 93.01 & 127.46 & RW2 \\ 
  Burundi & CENTRE-EAST & 10-14 & 84.85 & 32.72 & 202.48 & RW2 \\ 
  Burundi & CENTRE-EAST & 15-19 & 65.02 & 5.63 & 451.28 & RW2 \\ 
  Burundi & NORTH & 80-84 & 207.97 & 314.69 & 130.54 & HT-Direct \\ 
  Burundi & NORTH & 80-84 & 184.52 & 130.46 & 250.53 & RW2 \\ 
  Burundi & NORTH & 85-89 & 160.34 & 213.88 & 118.18 & HT-Direct \\ 
  Burundi & NORTH & 85-89 & 174.22 & 142.68 & 209.40 & RW2 \\ 
  Burundi & NORTH & 90-94 & 223.90 & 265.46 & 187.18 & HT-Direct \\ 
  Burundi & NORTH & 90-94 & 191.07 & 168.67 & 216.47 & RW2 \\ 
  Burundi & NORTH & 95-99 & 250.90 & 295.09 & 211.34 & HT-Direct \\ 
  Burundi & NORTH & 95-99 & 192.42 & 173.37 & 213.87 & RW2 \\ 
  Burundi & NORTH & 00-04 & 210.64 & 239.95 & 184.04 & HT-Direct \\ 
  Burundi & NORTH & 00-04 & 172.62 & 157.31 & 188.95 & RW2 \\ 
  Burundi & NORTH & 05-09 & 130.72 & 148.74 & 114.59 & HT-Direct \\ 
  Burundi & NORTH & 05-09 & 144.22 & 125.62 & 165.16 & RW2 \\ 
  Burundi & NORTH & 10-14 & 116.80 & 45.84 & 265.72 & RW2 \\ 
  Burundi & NORTH & 15-19 & 93.55 & 8.04 & 559.64 & RW2 \\ 
  Burundi & SOUTH & 80-84 & 203.48 & 291.21 & 137.07 & HT-Direct \\ 
  Burundi & SOUTH & 80-84 & 218.88 & 164.37 & 287.38 & RW2 \\ 
  Burundi & SOUTH & 85-89 & 228.74 & 292.34 & 175.55 & HT-Direct \\ 
  Burundi & SOUTH & 85-89 & 173.87 & 143.34 & 210.32 & RW2 \\ 
  Burundi & SOUTH & 90-94 & 151.94 & 198.30 & 114.86 & HT-Direct \\ 
  Burundi & SOUTH & 90-94 & 152.59 & 130.89 & 177.25 & RW2 \\ 
  Burundi & SOUTH & 95-99 & 164.09 & 193.32 & 138.52 & HT-Direct \\ 
  Burundi & SOUTH & 95-99 & 132.11 & 117.45 & 147.28 & RW2 \\ 
  Burundi & SOUTH & 00-04 & 138.80 & 158.69 & 121.04 & HT-Direct \\ 
  Burundi & SOUTH & 00-04 & 110.82 & 100.65 & 121.93 & RW2 \\ 
  Burundi & SOUTH & 05-09 & 81.50 & 97.90 & 67.64 & HT-Direct \\ 
  Burundi & SOUTH & 05-09 & 87.93 & 73.18 & 106.22 & RW2 \\ 
  Burundi & SOUTH & 10-14 & 68.02 & 25.20 & 173.35 & RW2 \\ 
  Burundi & SOUTH & 15-19 & 52.71 & 4.31 & 413.21 & RW2 \\ 
  Burundi & WEST & 80-84 & 166.54 & 258.40 & 102.80 & HT-Direct \\ 
  Burundi & WEST & 80-84 & 195.53 & 135.00 & 266.75 & RW2 \\ 
  Burundi & WEST & 85-89 & 180.68 & 239.49 & 133.78 & HT-Direct \\ 
  Burundi & WEST & 85-89 & 187.12 & 153.64 & 225.21 & RW2 \\ 
  Burundi & WEST & 90-94 & 253.72 & 309.47 & 205.03 & HT-Direct \\ 
  Burundi & WEST & 90-94 & 200.39 & 174.87 & 229.88 & RW2 \\ 
  Burundi & WEST & 95-99 & 245.65 & 292.03 & 204.50 & HT-Direct \\ 
  Burundi & WEST & 95-99 & 193.27 & 172.41 & 217.85 & RW2 \\ 
  Burundi & WEST & 00-04 & 208.71 & 246.05 & 175.72 & HT-Direct \\ 
  Burundi & WEST & 00-04 & 161.80 & 144.84 & 180.47 & RW2 \\ 
  Burundi & WEST & 05-09 & 96.47 & 118.86 & 77.93 & HT-Direct \\ 
  Burundi & WEST & 05-09 & 120.06 & 96.81 & 146.00 & RW2 \\ 
  Burundi & WEST & 10-14 & 84.90 & 31.35 & 205.07 & RW2 \\ 
  Burundi & WEST & 15-19 & 58.55 & 4.90 & 430.22 & RW2 \\ 
  Cameroon & ADAM/NORD/EXT-NORD & 80-84 & 219.24 & 245.15 & 195.36 & HT-Direct \\ 
  Cameroon & ADAM/NORD/EXT-NORD & 80-84 & 242.28 & 218.86 & 267.71 & RW2 \\ 
  Cameroon & ADAM/NORD/EXT-NORD & 85-89 & 197.13 & 214.59 & 180.76 & HT-Direct \\ 
  Cameroon & ADAM/NORD/EXT-NORD & 85-89 & 198.95 & 185.03 & 213.61 & RW2 \\ 
  Cameroon & ADAM/NORD/EXT-NORD & 90-94 & 196.96 & 211.34 & 183.33 & HT-Direct \\ 
  Cameroon & ADAM/NORD/EXT-NORD & 90-94 & 195.57 & 184.65 & 207.00 & RW2 \\ 
  Cameroon & ADAM/NORD/EXT-NORD & 95-99 & 198.76 & 214.58 & 183.83 & HT-Direct \\ 
  Cameroon & ADAM/NORD/EXT-NORD & 95-99 & 202.45 & 190.76 & 214.52 & RW2 \\ 
  Cameroon & ADAM/NORD/EXT-NORD & 00-04 & 182.22 & 195.51 & 169.64 & HT-Direct \\ 
  Cameroon & ADAM/NORD/EXT-NORD & 00-04 & 185.29 & 175.08 & 195.88 & RW2 \\ 
  Cameroon & ADAM/NORD/EXT-NORD & 05-09 & 169.03 & 184.41 & 154.68 & HT-Direct \\ 
  Cameroon & ADAM/NORD/EXT-NORD & 05-09 & 152.49 & 142.35 & 163.26 & RW2 \\ 
  Cameroon & ADAM/NORD/EXT-NORD & 10-14 & 149.39 & 175.90 & 126.26 & HT-Direct \\ 
  Cameroon & ADAM/NORD/EXT-NORD & 10-14 & 134.00 & 117.11 & 153.30 & RW2 \\ 
  Cameroon & ADAM/NORD/EXT-NORD & 15-19 & 121.25 & 53.91 & 254.20 & RW2 \\ 
  Cameroon & ALL & 80-84 & 158.55 & 156.21 & 160.95 & IHME \\ 
  Cameroon & ALL & 80-84 & 169.36 & 155.99 & 183.63 & RW2 \\ 
  Cameroon & ALL & 80-84 & 169.28 & 163.27 & 175.03 & UN \\ 
  Cameroon & ALL & 85-89 & 140.84 & 138.75 & 142.90 & IHME \\ 
  Cameroon & ALL & 85-89 & 141.89 & 133.22 & 150.96 & RW2 \\ 
  Cameroon & ALL & 85-89 & 142.18 & 137.56 & 146.54 & UN \\ 
  Cameroon & ALL & 90-94 & 138.19 & 136.10 & 140.09 & IHME \\ 
  Cameroon & ALL & 90-94 & 143.56 & 136.57 & 150.85 & RW2 \\ 
  Cameroon & ALL & 90-94 & 143.13 & 138.46 & 148.01 & UN \\ 
  Cameroon & ALL & 95-99 & 140.21 & 137.84 & 142.57 & IHME \\ 
  Cameroon & ALL & 95-99 & 153.74 & 145.96 & 161.82 & RW2 \\ 
  Cameroon & ALL & 95-99 & 154.14 & 148.71 & 159.73 & UN \\ 
  Cameroon & ALL & 00-04 & 132.62 & 130.14 & 135.27 & IHME \\ 
  Cameroon & ALL & 00-04 & 139.93 & 133.35 & 146.81 & RW2 \\ 
  Cameroon & ALL & 00-04 & 139.95 & 133.56 & 146.00 & UN \\ 
  Cameroon & ALL & 05-09 & 117.53 & 114.49 & 120.78 & IHME \\ 
  Cameroon & ALL & 05-09 & 116.82 & 110.32 & 123.65 & RW2 \\ 
  Cameroon & ALL & 05-09 & 116.63 & 105.17 & 128.48 & UN \\ 
  Cameroon & ALL & 10-14 & 99.64 & 95.07 & 104.52 & IHME \\ 
  Cameroon & ALL & 10-14 & 98.48 & 87.41 & 110.70 & RW2 \\ 
  Cameroon & ALL & 10-14 & 98.85 & 80.91 & 120.50 & UN \\ 
  Cameroon & CENTRE/SUD/EST & 80-84 & 148.34 & 173.72 & 126.10 & HT-Direct \\ 
  Cameroon & CENTRE/SUD/EST & 80-84 & 159.13 & 138.62 & 181.61 & RW2 \\ 
  Cameroon & CENTRE/SUD/EST & 85-89 & 123.21 & 138.17 & 109.67 & HT-Direct \\ 
  Cameroon & CENTRE/SUD/EST & 85-89 & 131.98 & 120.48 & 144.19 & RW2 \\ 
  Cameroon & CENTRE/SUD/EST & 90-94 & 138.00 & 153.23 & 124.06 & HT-Direct \\ 
  Cameroon & CENTRE/SUD/EST & 90-94 & 134.73 & 125.08 & 145.12 & RW2 \\ 
  Cameroon & CENTRE/SUD/EST & 95-99 & 137.79 & 151.57 & 125.09 & HT-Direct \\ 
  Cameroon & CENTRE/SUD/EST & 95-99 & 144.35 & 134.58 & 154.87 & RW2 \\ 
  Cameroon & CENTRE/SUD/EST & 00-04 & 143.06 & 158.49 & 128.90 & HT-Direct \\ 
  Cameroon & CENTRE/SUD/EST & 00-04 & 132.44 & 123.17 & 142.55 & RW2 \\ 
  Cameroon & CENTRE/SUD/EST & 05-09 & 108.10 & 122.82 & 94.95 & HT-Direct \\ 
  Cameroon & CENTRE/SUD/EST & 05-09 & 103.32 & 94.02 & 113.50 & RW2 \\ 
  Cameroon & CENTRE/SUD/EST & 10-14 & 88.00 & 120.02 & 63.90 & HT-Direct \\ 
  Cameroon & CENTRE/SUD/EST & 10-14 & 84.67 & 69.13 & 102.25 & RW2 \\ 
  Cameroon & CENTRE/SUD/EST & 15-19 & 71.07 & 29.18 & 162.97 & RW2 \\ 
  Cameroon & NORD-OUEST/SUD-OUEST & 80-84 & 121.50 & 145.99 & 100.64 & HT-Direct \\ 
  Cameroon & NORD-OUEST/SUD-OUEST & 80-84 & 122.46 & 104.66 & 142.71 & RW2 \\ 
  Cameroon & NORD-OUEST/SUD-OUEST & 85-89 & 86.39 & 102.10 & 72.89 & HT-Direct \\ 
  Cameroon & NORD-OUEST/SUD-OUEST & 85-89 & 99.98 & 89.33 & 111.77 & RW2 \\ 
  Cameroon & NORD-OUEST/SUD-OUEST & 90-94 & 104.27 & 119.32 & 90.92 & HT-Direct \\ 
  Cameroon & NORD-OUEST/SUD-OUEST & 90-94 & 102.30 & 93.43 & 111.77 & RW2 \\ 
  Cameroon & NORD-OUEST/SUD-OUEST & 95-99 & 107.73 & 122.55 & 94.51 & HT-Direct \\ 
  Cameroon & NORD-OUEST/SUD-OUEST & 95-99 & 110.95 & 101.92 & 120.83 & RW2 \\ 
  Cameroon & NORD-OUEST/SUD-OUEST & 00-04 & 111.34 & 128.37 & 96.32 & HT-Direct \\ 
  Cameroon & NORD-OUEST/SUD-OUEST & 00-04 & 102.91 & 93.94 & 112.88 & RW2 \\ 
  Cameroon & NORD-OUEST/SUD-OUEST & 05-09 & 85.67 & 102.81 & 71.17 & HT-Direct \\ 
  Cameroon & NORD-OUEST/SUD-OUEST & 05-09 & 81.60 & 72.15 & 92.13 & RW2 \\ 
  Cameroon & NORD-OUEST/SUD-OUEST & 10-14 & 67.56 & 98.00 & 46.10 & HT-Direct \\ 
  Cameroon & NORD-OUEST/SUD-OUEST & 10-14 & 68.23 & 54.58 & 84.37 & RW2 \\ 
  Cameroon & NORD-OUEST/SUD-OUEST & 15-19 & 58.52 & 23.98 & 136.32 & RW2 \\ 
  Cameroon & OUEST/LITTORAL & 80-84 & 120.23 & 141.80 & 101.54 & HT-Direct \\ 
  Cameroon & OUEST/LITTORAL & 80-84 & 125.26 & 107.90 & 145.22 & RW2 \\ 
  Cameroon & OUEST/LITTORAL & 85-89 & 97.43 & 115.77 & 81.73 & HT-Direct \\ 
  Cameroon & OUEST/LITTORAL & 85-89 & 104.67 & 93.42 & 117.10 & RW2 \\ 
  Cameroon & OUEST/LITTORAL & 90-94 & 104.84 & 122.48 & 89.49 & HT-Direct \\ 
  Cameroon & OUEST/LITTORAL & 90-94 & 109.46 & 99.64 & 119.99 & RW2 \\ 
  Cameroon & OUEST/LITTORAL & 95-99 & 123.76 & 137.69 & 111.06 & HT-Direct \\ 
  Cameroon & OUEST/LITTORAL & 95-99 & 122.01 & 112.60 & 132.17 & RW2 \\ 
  Cameroon & OUEST/LITTORAL & 00-04 & 116.38 & 133.09 & 101.53 & HT-Direct \\ 
  Cameroon & OUEST/LITTORAL & 00-04 & 116.87 & 106.81 & 127.84 & RW2 \\ 
  Cameroon & OUEST/LITTORAL & 05-09 & 107.68 & 132.79 & 86.85 & HT-Direct \\ 
  Cameroon & OUEST/LITTORAL & 05-09 & 97.29 & 85.49 & 110.54 & RW2 \\ 
  Cameroon & OUEST/LITTORAL & 10-14 & 88.84 & 126.66 & 61.51 & HT-Direct \\ 
  Cameroon & OUEST/LITTORAL & 10-14 & 85.77 & 68.41 & 106.58 & RW2 \\ 
  Cameroon & OUEST/LITTORAL & 15-19 & 77.59 & 32.01 & 177.60 & RW2 \\ 
  Cameroon & YAOUNDE/DOUALA & 80-84 & 119.33 & 156.26 & 90.19 & HT-Direct \\ 
  Cameroon & YAOUNDE/DOUALA & 80-84 & 119.33 & 96.39 & 147.62 & RW2 \\ 
  Cameroon & YAOUNDE/DOUALA & 85-89 & 86.58 & 109.53 & 68.07 & HT-Direct \\ 
  Cameroon & YAOUNDE/DOUALA & 85-89 & 93.92 & 80.71 & 108.93 & RW2 \\ 
  Cameroon & YAOUNDE/DOUALA & 90-94 & 93.75 & 113.08 & 77.44 & HT-Direct \\ 
  Cameroon & YAOUNDE/DOUALA & 90-94 & 92.88 & 82.55 & 104.28 & RW2 \\ 
  Cameroon & YAOUNDE/DOUALA & 95-99 & 89.72 & 106.92 & 75.06 & HT-Direct \\ 
  Cameroon & YAOUNDE/DOUALA & 95-99 & 98.27 & 88.27 & 109.05 & RW2 \\ 
  Cameroon & YAOUNDE/DOUALA & 00-04 & 94.35 & 111.87 & 79.33 & HT-Direct \\ 
  Cameroon & YAOUNDE/DOUALA & 00-04 & 89.53 & 80.58 & 99.55 & RW2 \\ 
  Cameroon & YAOUNDE/DOUALA & 05-09 & 78.76 & 94.65 & 65.34 & HT-Direct \\ 
  Cameroon & YAOUNDE/DOUALA & 05-09 & 69.76 & 60.89 & 79.76 & RW2 \\ 
  Cameroon & YAOUNDE/DOUALA & 10-14 & 48.81 & 84.64 & 27.69 & HT-Direct \\ 
  Cameroon & YAOUNDE/DOUALA & 10-14 & 57.03 & 43.91 & 73.39 & RW2 \\ 
  Cameroon & YAOUNDE/DOUALA & 15-19 & 48.07 & 18.83 & 117.04 & RW2 \\ 
  Chad & ALL & 80-84 & 229.89 & 226.29 & 233.50 & IHME \\ 
  Chad & ALL & 80-84 & 237.13 & 209.02 & 267.75 & RW2 \\ 
  Chad & ALL & 80-84 & 236.47 & 227.15 & 245.73 & UN \\ 
  Chad & ALL & 85-89 & 212.02 & 208.56 & 215.04 & IHME \\ 
  Chad & ALL & 85-89 & 222.45 & 200.68 & 245.80 & RW2 \\ 
  Chad & ALL & 85-89 & 223.02 & 215.48 & 230.64 & UN \\ 
  Chad & ALL & 90-94 & 195.25 & 192.48 & 198.26 & IHME \\ 
  Chad & ALL & 90-94 & 209.21 & 193.84 & 225.43 & RW2 \\ 
  Chad & ALL & 90-94 & 209.68 & 202.88 & 216.80 & UN \\ 
  Chad & ALL & 95-99 & 186.83 & 183.80 & 190.07 & IHME \\ 
  Chad & ALL & 95-99 & 199.03 & 187.71 & 210.78 & RW2 \\ 
  Chad & ALL & 95-99 & 198.22 & 191.97 & 205.09 & UN \\ 
  Chad & ALL & 00-04 & 178.32 & 174.29 & 182.45 & IHME \\ 
  Chad & ALL & 00-04 & 183.50 & 172.37 & 195.20 & RW2 \\ 
  Chad & ALL & 00-04 & 184.23 & 176.95 & 191.97 & UN \\ 
  Chad & ALL & 05-09 & 159.79 & 154.39 & 165.02 & IHME \\ 
  Chad & ALL & 05-09 & 172.08 & 157.73 & 187.54 & RW2 \\ 
  Chad & ALL & 05-09 & 171.26 & 160.47 & 182.07 & UN \\ 
  Chad & ALL & 10-14 & 141.83 & 135.05 & 148.74 & IHME \\ 
  Chad & ALL & 10-14 & 152.93 & 142.70 & 163.68 & RW2 \\ 
  Chad & ALL & 10-14 & 153.04 & 134.67 & 175.78 & UN \\ 
  Chad & ZONE 1 & 80-84 & 167.45 & 224.83 & 122.40 & HT-Direct \\ 
  Chad & ZONE 1 & 80-84 & 190.73 & 148.14 & 242.34 & RW2 \\ 
  Chad & ZONE 1 & 85-89 & 148.35 & 193.06 & 112.54 & HT-Direct \\ 
  Chad & ZONE 1 & 85-89 & 189.30 & 159.13 & 224.28 & RW2 \\ 
  Chad & ZONE 1 & 90-94 & 193.20 & 226.56 & 163.72 & HT-Direct \\ 
  Chad & ZONE 1 & 90-94 & 183.50 & 162.50 & 207.00 & RW2 \\ 
  Chad & ZONE 1 & 95-99 & 164.39 & 187.35 & 143.75 & HT-Direct \\ 
  Chad & ZONE 1 & 95-99 & 171.58 & 156.07 & 188.15 & RW2 \\ 
  Chad & ZONE 1 & 00-04 & 156.96 & 177.17 & 138.67 & HT-Direct \\ 
  Chad & ZONE 1 & 00-04 & 165.34 & 150.90 & 180.71 & RW2 \\ 
  Chad & ZONE 1 & 05-09 & 157.94 & 185.07 & 134.13 & HT-Direct \\ 
  Chad & ZONE 1 & 05-09 & 164.45 & 146.88 & 183.76 & RW2 \\ 
  Chad & ZONE 1 & 10-14 & 141.00 & 171.24 & 115.35 & HT-Direct \\ 
  Chad & ZONE 1 & 10-14 & 158.05 & 131.74 & 189.07 & RW2 \\ 
  Chad & ZONE 1 & 15-19 & 149.13 & 67.40 & 298.47 & RW2 \\ 
  Chad & ZONE 2 & 80-84 & 242.74 & 306.55 & 188.60 & HT-Direct \\ 
  Chad & ZONE 2 & 80-84 & 254.60 & 206.28 & 310.88 & RW2 \\ 
  Chad & ZONE 2 & 85-89 & 184.89 & 234.51 & 143.80 & HT-Direct \\ 
  Chad & ZONE 2 & 85-89 & 222.74 & 188.95 & 260.52 & RW2 \\ 
  Chad & ZONE 2 & 90-94 & 171.45 & 208.28 & 139.98 & HT-Direct \\ 
  Chad & ZONE 2 & 90-94 & 192.84 & 170.04 & 218.15 & RW2 \\ 
  Chad & ZONE 2 & 95-99 & 170.44 & 196.51 & 147.19 & HT-Direct \\ 
  Chad & ZONE 2 & 95-99 & 163.08 & 146.99 & 180.35 & RW2 \\ 
  Chad & ZONE 2 & 00-04 & 125.39 & 148.89 & 105.13 & HT-Direct \\ 
  Chad & ZONE 2 & 00-04 & 138.55 & 124.19 & 154.08 & RW2 \\ 
  Chad & ZONE 2 & 05-09 & 110.69 & 135.88 & 89.69 & HT-Direct \\ 
  Chad & ZONE 2 & 05-09 & 121.67 & 106.58 & 138.26 & RW2 \\ 
  Chad & ZONE 2 & 10-14 & 98.12 & 119.27 & 80.38 & HT-Direct \\ 
  Chad & ZONE 2 & 10-14 & 105.82 & 87.59 & 128.10 & RW2 \\ 
  Chad & ZONE 2 & 15-19 & 91.60 & 40.21 & 195.05 & RW2 \\ 
  Chad & ZONE 3 & 80-84 & 299.22 & 372.66 & 234.84 & HT-Direct \\ 
  Chad & ZONE 3 & 80-84 & 296.11 & 243.13 & 357.48 & RW2 \\ 
  Chad & ZONE 3 & 85-89 & 236.85 & 299.84 & 183.63 & HT-Direct \\ 
  Chad & ZONE 3 & 85-89 & 262.23 & 224.99 & 303.66 & RW2 \\ 
  Chad & ZONE 3 & 90-94 & 191.19 & 222.18 & 163.62 & HT-Direct \\ 
  Chad & ZONE 3 & 90-94 & 230.30 & 206.03 & 256.07 & RW2 \\ 
  Chad & ZONE 3 & 95-99 & 206.20 & 235.94 & 179.34 & HT-Direct \\ 
  Chad & ZONE 3 & 95-99 & 199.03 & 181.51 & 217.60 & RW2 \\ 
  Chad & ZONE 3 & 00-04 & 164.77 & 187.62 & 144.21 & HT-Direct \\ 
  Chad & ZONE 3 & 00-04 & 173.39 & 158.75 & 189.05 & RW2 \\ 
  Chad & ZONE 3 & 05-09 & 146.73 & 167.58 & 128.07 & HT-Direct \\ 
  Chad & ZONE 3 & 05-09 & 154.59 & 139.69 & 170.70 & RW2 \\ 
  Chad & ZONE 3 & 10-14 & 122.50 & 145.93 & 102.38 & HT-Direct \\ 
  Chad & ZONE 3 & 10-14 & 134.96 & 115.01 & 158.55 & RW2 \\ 
  Chad & ZONE 3 & 15-19 & 116.53 & 52.95 & 237.10 & RW2 \\ 
  Chad & ZONE 4 & 80-84 & 203.05 & 272.46 & 147.73 & HT-Direct \\ 
  Chad & ZONE 4 & 80-84 & 211.81 & 162.77 & 271.71 & RW2 \\ 
  Chad & ZONE 4 & 85-89 & 140.08 & 192.51 & 100.16 & HT-Direct \\ 
  Chad & ZONE 4 & 85-89 & 186.80 & 153.48 & 225.67 & RW2 \\ 
  Chad & ZONE 4 & 90-94 & 147.21 & 183.84 & 116.83 & HT-Direct \\ 
  Chad & ZONE 4 & 90-94 & 164.02 & 141.50 & 189.42 & RW2 \\ 
  Chad & ZONE 4 & 95-99 & 145.52 & 173.16 & 121.64 & HT-Direct \\ 
  Chad & ZONE 4 & 95-99 & 140.78 & 125.15 & 158.13 & RW2 \\ 
  Chad & ZONE 4 & 00-04 & 120.58 & 141.55 & 102.35 & HT-Direct \\ 
  Chad & ZONE 4 & 00-04 & 120.64 & 108.09 & 134.36 & RW2 \\ 
  Chad & ZONE 4 & 05-09 & 93.41 & 109.41 & 79.54 & HT-Direct \\ 
  Chad & ZONE 4 & 05-09 & 106.11 & 94.24 & 119.04 & RW2 \\ 
  Chad & ZONE 4 & 10-14 & 85.28 & 102.31 & 70.86 & HT-Direct \\ 
  Chad & ZONE 4 & 10-14 & 93.23 & 78.00 & 111.58 & RW2 \\ 
  Chad & ZONE 4 & 15-19 & 81.80 & 35.57 & 176.12 & RW2 \\ 
  Chad & ZONE 5 & 80-84 & 144.19 & 207.32 & 97.90 & HT-Direct \\ 
  Chad & ZONE 5 & 80-84 & 189.37 & 145.88 & 241.28 & RW2 \\ 
  Chad & ZONE 5 & 85-89 & 189.24 & 248.24 & 141.62 & HT-Direct \\ 
  Chad & ZONE 5 & 85-89 & 184.43 & 152.60 & 221.26 & RW2 \\ 
  Chad & ZONE 5 & 90-94 & 162.12 & 197.39 & 132.12 & HT-Direct \\ 
  Chad & ZONE 5 & 90-94 & 176.27 & 153.12 & 202.03 & RW2 \\ 
  Chad & ZONE 5 & 95-99 & 152.60 & 192.95 & 119.43 & HT-Direct \\ 
  Chad & ZONE 5 & 95-99 & 163.89 & 145.64 & 183.76 & RW2 \\ 
  Chad & ZONE 5 & 00-04 & 156.70 & 184.85 & 132.14 & HT-Direct \\ 
  Chad & ZONE 5 & 00-04 & 153.98 & 138.06 & 171.11 & RW2 \\ 
  Chad & ZONE 5 & 05-09 & 123.34 & 153.27 & 98.58 & HT-Direct \\ 
  Chad & ZONE 5 & 05-09 & 146.67 & 129.40 & 165.87 & RW2 \\ 
  Chad & ZONE 5 & 10-14 & 119.85 & 141.31 & 101.26 & HT-Direct \\ 
  Chad & ZONE 5 & 10-14 & 134.83 & 114.50 & 158.49 & RW2 \\ 
  Chad & ZONE 5 & 15-19 & 121.95 & 56.40 & 244.40 & RW2 \\ 
  Chad & ZONE 6 & 80-84 & 190.20 & 266.92 & 131.57 & HT-Direct \\ 
  Chad & ZONE 6 & 80-84 & 175.28 & 133.38 & 230.10 & RW2 \\ 
  Chad & ZONE 6 & 85-89 & 159.45 & 202.65 & 124.03 & HT-Direct \\ 
  Chad & ZONE 6 & 85-89 & 171.25 & 141.66 & 205.18 & RW2 \\ 
  Chad & ZONE 6 & 90-94 & 153.84 & 190.64 & 123.07 & HT-Direct \\ 
  Chad & ZONE 6 & 90-94 & 165.42 & 143.77 & 189.27 & RW2 \\ 
  Chad & ZONE 6 & 95-99 & 125.35 & 151.68 & 103.04 & HT-Direct \\ 
  Chad & ZONE 6 & 95-99 & 158.68 & 140.89 & 176.98 & RW2 \\ 
  Chad & ZONE 6 & 00-04 & 158.13 & 178.92 & 139.35 & HT-Direct \\ 
  Chad & ZONE 6 & 00-04 & 159.16 & 145.27 & 173.91 & RW2 \\ 
  Chad & ZONE 6 & 05-09 & 165.67 & 186.22 & 146.98 & HT-Direct \\ 
  Chad & ZONE 6 & 05-09 & 163.04 & 148.40 & 178.91 & RW2 \\ 
  Chad & ZONE 6 & 10-14 & 129.62 & 151.71 & 110.33 & HT-Direct \\ 
  Chad & ZONE 6 & 10-14 & 158.47 & 136.73 & 183.30 & RW2 \\ 
  Chad & ZONE 6 & 15-19 & 150.81 & 70.01 & 295.52 & RW2 \\ 
  Chad & ZONE 7 & 80-84 & 210.70 & 276.54 & 157.13 & HT-Direct \\ 
  Chad & ZONE 7 & 80-84 & 238.72 & 187.39 & 295.14 & RW2 \\ 
  Chad & ZONE 7 & 85-89 & 202.04 & 249.50 & 161.66 & HT-Direct \\ 
  Chad & ZONE 7 & 85-89 & 250.18 & 214.16 & 289.37 & RW2 \\ 
  Chad & ZONE 7 & 90-94 & 221.09 & 255.93 & 189.78 & HT-Direct \\ 
  Chad & ZONE 7 & 90-94 & 257.58 & 232.41 & 284.47 & RW2 \\ 
  Chad & ZONE 7 & 95-99 & 268.87 & 294.52 & 244.67 & HT-Direct \\ 
  Chad & ZONE 7 & 95-99 & 255.67 & 237.31 & 275.40 & RW2 \\ 
  Chad & ZONE 7 & 00-04 & 236.21 & 265.94 & 208.86 & HT-Direct \\ 
  Chad & ZONE 7 & 00-04 & 248.96 & 229.45 & 270.85 & RW2 \\ 
  Chad & ZONE 7 & 05-09 & 247.06 & 291.14 & 207.71 & HT-Direct \\ 
  Chad & ZONE 7 & 05-09 & 238.38 & 216.39 & 261.89 & RW2 \\ 
  Chad & ZONE 7 & 10-14 & 175.32 & 195.23 & 157.05 & HT-Direct \\ 
  Chad & ZONE 7 & 10-14 & 214.12 & 189.32 & 240.23 & RW2 \\ 
  Chad & ZONE 7 & 15-19 & 187.90 & 90.63 & 345.34 & RW2 \\ 
  Chad & ZONE 8 & 80-84 & 224.78 & 300.63 & 163.59 & HT-Direct \\ 
  Chad & ZONE 8 & 80-84 & 261.49 & 207.35 & 321.46 & RW2 \\ 
  Chad & ZONE 8 & 85-89 & 226.53 & 271.48 & 187.10 & HT-Direct \\ 
  Chad & ZONE 8 & 85-89 & 253.54 & 217.09 & 293.07 & RW2 \\ 
  Chad & ZONE 8 & 90-94 & 216.01 & 265.49 & 173.56 & HT-Direct \\ 
  Chad & ZONE 8 & 90-94 & 241.89 & 214.23 & 271.82 & RW2 \\ 
  Chad & ZONE 8 & 95-99 & 220.46 & 252.68 & 191.30 & HT-Direct \\ 
  Chad & ZONE 8 & 95-99 & 223.75 & 203.33 & 245.79 & RW2 \\ 
  Chad & ZONE 8 & 00-04 & 202.47 & 235.33 & 173.15 & HT-Direct \\ 
  Chad & ZONE 8 & 00-04 & 205.60 & 187.58 & 225.59 & RW2 \\ 
  Chad & ZONE 8 & 05-09 & 186.19 & 209.14 & 165.24 & HT-Direct \\ 
  Chad & ZONE 8 & 05-09 & 187.45 & 170.95 & 205.41 & RW2 \\ 
  Chad & ZONE 8 & 10-14 & 130.85 & 150.99 & 113.04 & HT-Direct \\ 
  Chad & ZONE 8 & 10-14 & 161.61 & 140.20 & 185.09 & RW2 \\ 
  Chad & ZONE 8 & 15-19 & 135.92 & 62.65 & 269.76 & RW2 \\ 
  Comoros & ALL & 80-84 & 154.72 & 150.37 & 159.08 & IHME \\ 
  Comoros & ALL & 80-84 & 164.34 & 141.77 & 189.72 & RW2 \\ 
  Comoros & ALL & 80-84 & 164.08 & 156.29 & 172.55 & UN \\ 
  Comoros & ALL & 85-89 & 131.45 & 127.70 & 135.50 & IHME \\ 
  Comoros & ALL & 85-89 & 136.56 & 116.60 & 159.15 & RW2 \\ 
  Comoros & ALL & 85-89 & 137.66 & 130.88 & 145.53 & UN \\ 
  Comoros & ALL & 90-94 & 113.33 & 109.47 & 116.99 & IHME \\ 
  Comoros & ALL & 90-94 & 117.71 & 101.57 & 136.09 & RW2 \\ 
  Comoros & ALL & 90-94 & 116.85 & 110.74 & 123.03 & UN \\ 
  Comoros & ALL & 95-99 & 95.92 & 90.91 & 101.18 & IHME \\ 
  Comoros & ALL & 95-99 & 102.93 & 73.27 & 142.05 & RW2 \\ 
  Comoros & ALL & 95-99 & 101.90 & 94.26 & 109.80 & UN \\ 
  Comoros & ALL & 00-04 & 77.83 & 71.47 & 84.91 & IHME \\ 
  Comoros & ALL & 00-04 & 94.29 & 42.54 & 195.48 & RW2 \\ 
  Comoros & ALL & 00-04 & 97.73 & 87.11 & 108.71 & UN \\ 
  Comoros & ALL & 05-09 & 62.98 & 55.76 & 71.17 & IHME \\ 
  Comoros & ALL & 05-09 & 89.76 & 35.41 & 208.83 & RW2 \\ 
  Comoros & ALL & 05-09 & 95.07 & 79.54 & 113.51 & UN \\ 
  Comoros & ALL & 10-14 & 50.62 & 43.93 & 57.81 & IHME \\ 
  Comoros & ALL & 10-14 & 87.45 & 44.68 & 163.48 & RW2 \\ 
  Comoros & ALL & 10-14 & 85.93 & 66.84 & 110.09 & UN \\ 
  Comoros & MOHELI & 80-84 & 164.52 & 248.69 & 104.87 & HT-Direct \\ 
  Comoros & MOHELI & 80-84 & 168.02 & 110.21 & 247.31 & RW2 \\ 
  Comoros & MOHELI & 85-89 & 122.96 & 207.33 & 69.89 & HT-Direct \\ 
  Comoros & MOHELI & 85-89 & 129.48 & 92.23 & 179.12 & RW2 \\ 
  Comoros & MOHELI & 90-94 & 72.72 & 113.18 & 45.97 & HT-Direct \\ 
  Comoros & MOHELI & 90-94 & 107.54 & 75.37 & 150.51 & RW2 \\ 
  Comoros & MOHELI & 95-99 & 66.63 & 98.47 & 44.58 & HT-Direct \\ 
  Comoros & MOHELI & 95-99 & 92.97 & 56.36 & 149.07 & RW2 \\ 
  Comoros & MOHELI & 00-04 & 41.40 & 62.21 & 27.35 & HT-Direct \\ 
  Comoros & MOHELI & 00-04 & 84.28 & 35.11 & 189.88 & RW2 \\ 
  Comoros & MOHELI & 05-09 & 27.35 & 50.73 & 14.58 & HT-Direct \\ 
  Comoros & MOHELI & 05-09 & 79.90 & 26.20 & 219.10 & RW2 \\ 
  Comoros & MOHELI & 10-14 & 70.35 & 146.95 & 32.18 & HT-Direct \\ 
  Comoros & MOHELI & 10-14 & 77.37 & 22.68 & 234.68 & RW2 \\ 
  Comoros & MOHELI & 15-19 & 75.07 & 8.29 & 451.39 & RW2 \\ 
  Comoros & NDZOUANI & 80-84 & 174.50 & 221.34 & 135.85 & HT-Direct \\ 
  Comoros & NDZOUANI & 80-84 & 185.78 & 147.35 & 232.28 & RW2 \\ 
  Comoros & NDZOUANI & 85-89 & 122.95 & 149.11 & 100.83 & HT-Direct \\ 
  Comoros & NDZOUANI & 85-89 & 144.22 & 120.36 & 171.61 & RW2 \\ 
  Comoros & NDZOUANI & 90-94 & 101.84 & 122.39 & 84.41 & HT-Direct \\ 
  Comoros & NDZOUANI & 90-94 & 119.33 & 98.13 & 144.33 & RW2 \\ 
  Comoros & NDZOUANI & 95-99 & 58.10 & 81.23 & 41.27 & HT-Direct \\ 
  Comoros & NDZOUANI & 95-99 & 101.40 & 69.26 & 146.00 & RW2 \\ 
  Comoros & NDZOUANI & 00-04 & 52.66 & 76.39 & 36.01 & HT-Direct \\ 
  Comoros & NDZOUANI & 00-04 & 89.86 & 41.92 & 182.45 & RW2 \\ 
  Comoros & NDZOUANI & 05-09 & 42.27 & 63.31 & 28.01 & HT-Direct \\ 
  Comoros & NDZOUANI & 05-09 & 82.89 & 33.10 & 192.37 & RW2 \\ 
  Comoros & NDZOUANI & 10-14 & 55.60 & 88.31 & 34.55 & HT-Direct \\ 
  Comoros & NDZOUANI & 10-14 & 78.26 & 32.99 & 174.79 & RW2 \\ 
  Comoros & NDZOUANI & 15-19 & 73.86 & 11.22 & 365.63 & RW2 \\ 
  Comoros & NGAZIDJA & 80-84 & 137.42 & 159.67 & 117.84 & HT-Direct \\ 
  Comoros & NGAZIDJA & 80-84 & 146.63 & 125.37 & 170.73 & RW2 \\ 
  Comoros & NGAZIDJA & 85-89 & 105.57 & 129.30 & 85.77 & HT-Direct \\ 
  Comoros & NGAZIDJA & 85-89 & 124.42 & 103.95 & 148.14 & RW2 \\ 
  Comoros & NGAZIDJA & 90-94 & 94.66 & 112.25 & 79.58 & HT-Direct \\ 
  Comoros & NGAZIDJA & 90-94 & 113.53 & 94.68 & 135.93 & RW2 \\ 
  Comoros & NGAZIDJA & 95-99 & 80.42 & 105.43 & 60.93 & HT-Direct \\ 
  Comoros & NGAZIDJA & 95-99 & 106.48 & 74.67 & 149.49 & RW2 \\ 
  Comoros & NGAZIDJA & 00-04 & 51.51 & 69.91 & 37.75 & HT-Direct \\ 
  Comoros & NGAZIDJA & 00-04 & 102.91 & 49.45 & 201.94 & RW2 \\ 
  Comoros & NGAZIDJA & 05-09 & 57.07 & 76.20 & 42.52 & HT-Direct \\ 
  Comoros & NGAZIDJA & 05-09 & 102.84 & 42.91 & 225.84 & RW2 \\ 
  Comoros & NGAZIDJA & 10-14 & 59.22 & 90.94 & 38.10 & HT-Direct \\ 
  Comoros & NGAZIDJA & 10-14 & 104.23 & 46.46 & 215.53 & RW2 \\ 
  Comoros & NGAZIDJA & 15-19 & 105.93 & 16.93 & 452.55 & RW2 \\ 
  Congo & ALL & 80-84 & 112.06 & 107.16 & 117.11 & IHME \\ 
  Congo & ALL & 80-84 & 106.72 & 87.03 & 130.28 & RW2 \\ 
  Congo & ALL & 80-84 & 107.21 & 97.73 & 116.73 & UN \\ 
  Congo & ALL & 85-89 & 96.31 & 93.24 & 99.30 & IHME \\ 
  Congo & ALL & 85-89 & 95.22 & 83.68 & 108.10 & RW2 \\ 
  Congo & ALL & 85-89 & 94.68 & 88.56 & 101.05 & UN \\ 
  Congo & ALL & 90-94 & 95.27 & 92.60 & 97.85 & IHME \\ 
  Congo & ALL & 90-94 & 97.68 & 87.66 & 108.59 & RW2 \\ 
  Congo & ALL & 90-94 & 98.32 & 93.04 & 103.57 & UN \\ 
  Congo & ALL & 95-99 & 108.34 & 104.24 & 112.75 & IHME \\ 
  Congo & ALL & 95-99 & 116.21 & 107.24 & 125.76 & RW2 \\ 
  Congo & ALL & 95-99 & 115.58 & 110.27 & 120.96 & UN \\ 
  Congo & ALL & 00-04 & 103.83 & 100.74 & 106.98 & IHME \\ 
  Congo & ALL & 00-04 & 113.89 & 105.82 & 122.58 & RW2 \\ 
  Congo & ALL & 00-04 & 113.93 & 108.87 & 118.74 & UN \\ 
  Congo & ALL & 05-09 & 79.42 & 76.79 & 81.99 & IHME \\ 
  Congo & ALL & 05-09 & 79.17 & 70.08 & 89.16 & RW2 \\ 
  Congo & ALL & 05-09 & 79.66 & 75.41 & 84.77 & UN \\ 
  Congo & ALL & 10-14 & 62.66 & 58.56 & 67.24 & IHME \\ 
  Congo & ALL & 10-14 & 54.21 & 45.46 & 64.47 & RW2 \\ 
  Congo & ALL & 10-14 & 53.17 & 47.00 & 60.03 & UN \\ 
  Congo & BRAZZAVILLE & 80-84 & 80.37 & 121.51 & 52.32 & HT-Direct \\ 
  Congo & BRAZZAVILLE & 80-84 & 75.19 & 54.51 & 102.75 & RW2 \\ 
  Congo & BRAZZAVILLE & 85-89 & 64.78 & 86.74 & 48.08 & HT-Direct \\ 
  Congo & BRAZZAVILLE & 85-89 & 68.99 & 56.20 & 84.46 & RW2 \\ 
  Congo & BRAZZAVILLE & 90-94 & 61.65 & 79.69 & 47.48 & HT-Direct \\ 
  Congo & BRAZZAVILLE & 90-94 & 74.61 & 63.83 & 86.99 & RW2 \\ 
  Congo & BRAZZAVILLE & 95-99 & 128.71 & 151.23 & 109.11 & HT-Direct \\ 
  Congo & BRAZZAVILLE & 95-99 & 103.59 & 92.41 & 116.20 & RW2 \\ 
  Congo & BRAZZAVILLE & 00-04 & 93.21 & 110.92 & 78.08 & HT-Direct \\ 
  Congo & BRAZZAVILLE & 00-04 & 105.74 & 93.25 & 119.55 & RW2 \\ 
  Congo & BRAZZAVILLE & 05-09 & 68.38 & 90.44 & 51.40 & HT-Direct \\ 
  Congo & BRAZZAVILLE & 05-09 & 83.21 & 68.79 & 99.98 & RW2 \\ 
  Congo & BRAZZAVILLE & 10-14 & 80.14 & 127.51 & 49.37 & HT-Direct \\ 
  Congo & BRAZZAVILLE & 10-14 & 55.01 & 41.03 & 74.01 & RW2 \\ 
  Congo & BRAZZAVILLE & 15-19 & 34.40 & 11.71 & 97.51 & RW2 \\ 
  Congo & NORD & 80-84 & 119.86 & 164.48 & 86.10 & HT-Direct \\ 
  Congo & NORD & 80-84 & 123.68 & 95.56 & 159.28 & RW2 \\ 
  Congo & NORD & 85-89 & 108.94 & 133.69 & 88.30 & HT-Direct \\ 
  Congo & NORD & 85-89 & 104.64 & 89.35 & 122.42 & RW2 \\ 
  Congo & NORD & 90-94 & 94.11 & 111.34 & 79.30 & HT-Direct \\ 
  Congo & NORD & 90-94 & 102.75 & 90.80 & 115.70 & RW2 \\ 
  Congo & NORD & 95-99 & 142.46 & 160.16 & 126.42 & HT-Direct \\ 
  Congo & NORD & 95-99 & 130.48 & 119.58 & 142.10 & RW2 \\ 
  Congo & NORD & 00-04 & 122.69 & 133.67 & 112.49 & HT-Direct \\ 
  Congo & NORD & 00-04 & 126.53 & 117.51 & 136.33 & RW2 \\ 
  Congo & NORD & 05-09 & 92.19 & 104.62 & 81.09 & HT-Direct \\ 
  Congo & NORD & 05-09 & 93.22 & 83.15 & 104.51 & RW2 \\ 
  Congo & NORD & 10-14 & 60.70 & 77.28 & 47.50 & HT-Direct \\ 
  Congo & NORD & 10-14 & 54.59 & 45.70 & 64.95 & RW2 \\ 
  Congo & NORD & 15-19 & 29.42 & 10.58 & 77.40 & RW2 \\ 
  Congo & POINTE NOIRE & 80-84 & 93.52 & 142.88 & 60.02 & HT-Direct \\ 
  Congo & POINTE NOIRE & 80-84 & 110.51 & 80.47 & 151.23 & RW2 \\ 
  Congo & POINTE NOIRE & 85-89 & 105.98 & 142.81 & 77.79 & HT-Direct \\ 
  Congo & POINTE NOIRE & 85-89 & 90.97 & 73.50 & 112.29 & RW2 \\ 
  Congo & POINTE NOIRE & 90-94 & 79.97 & 103.85 & 61.21 & HT-Direct \\ 
  Congo & POINTE NOIRE & 90-94 & 85.81 & 72.95 & 100.87 & RW2 \\ 
  Congo & POINTE NOIRE & 95-99 & 115.66 & 140.18 & 94.96 & HT-Direct \\ 
  Congo & POINTE NOIRE & 95-99 & 103.98 & 91.35 & 117.86 & RW2 \\ 
  Congo & POINTE NOIRE & 00-04 & 88.65 & 108.93 & 71.84 & HT-Direct \\ 
  Congo & POINTE NOIRE & 00-04 & 96.18 & 83.30 & 110.19 & RW2 \\ 
  Congo & POINTE NOIRE & 05-09 & 52.13 & 69.27 & 39.05 & HT-Direct \\ 
  Congo & POINTE NOIRE & 05-09 & 69.52 & 57.36 & 83.91 & RW2 \\ 
  Congo & POINTE NOIRE & 10-14 & 68.53 & 105.75 & 43.78 & HT-Direct \\ 
  Congo & POINTE NOIRE & 10-14 & 42.53 & 31.87 & 58.01 & RW2 \\ 
  Congo & POINTE NOIRE & 15-19 & 24.68 & 8.39 & 72.50 & RW2 \\ 
  Congo & SUD & 80-84 & 131.98 & 175.90 & 97.72 & HT-Direct \\ 
  Congo & SUD & 80-84 & 139.28 & 109.73 & 174.23 & RW2 \\ 
  Congo & SUD & 85-89 & 100.64 & 126.63 & 79.50 & HT-Direct \\ 
  Congo & SUD & 85-89 & 115.62 & 98.43 & 135.04 & RW2 \\ 
  Congo & SUD & 90-94 & 120.06 & 139.19 & 103.25 & HT-Direct \\ 
  Congo & SUD & 90-94 & 110.69 & 98.74 & 123.93 & RW2 \\ 
  Congo & SUD & 95-99 & 133.53 & 152.73 & 116.40 & HT-Direct \\ 
  Congo & SUD & 95-99 & 134.12 & 122.67 & 146.66 & RW2 \\ 
  Congo & SUD & 00-04 & 133.71 & 150.67 & 118.39 & HT-Direct \\ 
  Congo & SUD & 00-04 & 122.96 & 112.37 & 134.60 & RW2 \\ 
  Congo & SUD & 05-09 & 72.61 & 85.61 & 61.46 & HT-Direct \\ 
  Congo & SUD & 05-09 & 85.04 & 74.82 & 96.63 & RW2 \\ 
  Congo & SUD & 10-14 & 55.61 & 71.04 & 43.37 & HT-Direct \\ 
  Congo & SUD & 10-14 & 47.91 & 40.19 & 56.91 & RW2 \\ 
  Congo & SUD & 15-19 & 25.11 & 9.06 & 66.07 & RW2 \\ 
  C\^{o}te d'Ivoire & ALL & 80-84 & 155.44 & 153.04 & 157.92 & IHME \\ 
  C\^{o}te d'Ivoire & ALL & 80-84 & 158.39 & 114.30 & 215.37 & RW2 \\ 
  C\^{o}te d'Ivoire & ALL & 80-84 & 160.36 & 154.83 & 165.77 & UN \\ 
  C\^{o}te d'Ivoire & ALL & 85-89 & 149.63 & 147.56 & 151.83 & IHME \\ 
  C\^{o}te d'Ivoire & ALL & 85-89 & 155.10 & 130.26 & 183.60 & RW2 \\ 
  C\^{o}te d'Ivoire & ALL & 85-89 & 152.92 & 148.31 & 157.65 & UN \\ 
  C\^{o}te d'Ivoire & ALL & 90-94 & 148.94 & 146.74 & 151.09 & IHME \\ 
  C\^{o}te d'Ivoire & ALL & 90-94 & 152.43 & 135.08 & 171.50 & RW2 \\ 
  C\^{o}te d'Ivoire & ALL & 90-94 & 153.01 & 148.26 & 157.66 & UN \\ 
  C\^{o}te d'Ivoire & ALL & 95-99 & 145.95 & 143.33 & 148.31 & IHME \\ 
  C\^{o}te d'Ivoire & ALL & 95-99 & 150.77 & 136.70 & 165.94 & RW2 \\ 
  C\^{o}te d'Ivoire & ALL & 95-99 & 150.89 & 145.62 & 156.27 & UN \\ 
  C\^{o}te d'Ivoire & ALL & 00-04 & 135.89 & 133.26 & 138.59 & IHME \\ 
  C\^{o}te d'Ivoire & ALL & 00-04 & 139.70 & 127.98 & 152.38 & RW2 \\ 
  C\^{o}te d'Ivoire & ALL & 00-04 & 139.57 & 133.91 & 145.15 & UN \\ 
  C\^{o}te d'Ivoire & ALL & 05-09 & 121.06 & 117.85 & 124.21 & IHME \\ 
  C\^{o}te d'Ivoire & ALL & 05-09 & 121.06 & 111.56 & 131.24 & RW2 \\ 
  C\^{o}te d'Ivoire & ALL & 05-09 & 121.06 & 115.65 & 126.46 & UN \\ 
  C\^{o}te d'Ivoire & ALL & 10-14 & 104.54 & 100.31 & 108.75 & IHME \\ 
  C\^{o}te d'Ivoire & ALL & 10-14 & 102.70 & 85.19 & 123.18 & RW2 \\ 
  C\^{o}te d'Ivoire & ALL & 10-14 & 102.89 & 94.95 & 110.57 & UN \\ 
  C\^{o}te d'Ivoire & CENTRE & 80-84 & 160.59 & 265.44 & 91.97 & HT-Direct \\ 
  C\^{o}te d'Ivoire & CENTRE & 80-84 & 168.41 & 106.67 & 257.22 & RW2 \\ 
  C\^{o}te d'Ivoire & CENTRE & 85-89 & 119.37 & 202.67 & 67.42 & HT-Direct \\ 
  C\^{o}te d'Ivoire & CENTRE & 85-89 & 140.18 & 100.26 & 193.04 & RW2 \\ 
  C\^{o}te d'Ivoire & CENTRE & 90-94 & 137.82 & 210.27 & 87.56 & HT-Direct \\ 
  C\^{o}te d'Ivoire & CENTRE & 90-94 & 139.94 & 109.68 & 176.96 & RW2 \\ 
  C\^{o}te d'Ivoire & CENTRE & 95-99 & 125.98 & 163.04 & 96.38 & HT-Direct \\ 
  C\^{o}te d'Ivoire & CENTRE & 95-99 & 148.69 & 125.26 & 175.23 & RW2 \\ 
  C\^{o}te d'Ivoire & CENTRE & 00-04 & 158.78 & 199.53 & 125.06 & HT-Direct \\ 
  C\^{o}te d'Ivoire & CENTRE & 00-04 & 153.45 & 131.88 & 177.89 & RW2 \\ 
  C\^{o}te d'Ivoire & CENTRE & 05-09 & 152.11 & 193.31 & 118.41 & HT-Direct \\ 
  C\^{o}te d'Ivoire & CENTRE & 05-09 & 141.61 & 115.94 & 171.82 & RW2 \\ 
  C\^{o}te d'Ivoire & CENTRE & 10-14 & 61.13 & 139.39 & 25.50 & HT-Direct \\ 
  C\^{o}te d'Ivoire & CENTRE & 10-14 & 131.33 & 92.27 & 183.38 & RW2 \\ 
  C\^{o}te d'Ivoire & CENTRE & 15-19 & 123.63 & 45.03 & 295.00 & RW2 \\ 
  C\^{o}te d'Ivoire & CENTRE-EST & 80-84 & 162.46 & 280.28 & 88.10 & HT-Direct \\ 
  C\^{o}te d'Ivoire & CENTRE-EST & 80-84 & 152.28 & 92.10 & 243.73 & RW2 \\ 
  C\^{o}te d'Ivoire & CENTRE-EST & 85-89 & 106.01 & 220.16 & 47.45 & HT-Direct \\ 
  C\^{o}te d'Ivoire & CENTRE-EST & 85-89 & 125.17 & 87.82 & 175.79 & RW2 \\ 
  C\^{o}te d'Ivoire & CENTRE-EST & 90-94 & 120.90 & 162.24 & 88.97 & HT-Direct \\ 
  C\^{o}te d'Ivoire & CENTRE-EST & 90-94 & 123.85 & 97.65 & 155.72 & RW2 \\ 
  C\^{o}te d'Ivoire & CENTRE-EST & 95-99 & 101.71 & 138.75 & 73.71 & HT-Direct \\ 
  C\^{o}te d'Ivoire & CENTRE-EST & 95-99 & 132.01 & 109.50 & 157.48 & RW2 \\ 
  C\^{o}te d'Ivoire & CENTRE-EST & 00-04 & 130.38 & 169.98 & 98.91 & HT-Direct \\ 
  C\^{o}te d'Ivoire & CENTRE-EST & 00-04 & 138.68 & 116.19 & 164.19 & RW2 \\ 
  C\^{o}te d'Ivoire & CENTRE-EST & 05-09 & 160.01 & 214.63 & 117.21 & HT-Direct \\ 
  C\^{o}te d'Ivoire & CENTRE-EST & 05-09 & 131.66 & 103.54 & 166.37 & RW2 \\ 
  C\^{o}te d'Ivoire & CENTRE-EST & 10-14 & 67.70 & 155.70 & 27.80 & HT-Direct \\ 
  C\^{o}te d'Ivoire & CENTRE-EST & 10-14 & 125.42 & 84.30 & 184.69 & RW2 \\ 
  C\^{o}te d'Ivoire & CENTRE-EST & 15-19 & 120.97 & 42.98 & 301.44 & RW2 \\ 
  C\^{o}te d'Ivoire & CENTRE-NORD & 80-84 & 127.06 & 189.30 & 83.19 & HT-Direct \\ 
  C\^{o}te d'Ivoire & CENTRE-NORD & 80-84 & 162.55 & 109.68 & 233.27 & RW2 \\ 
  C\^{o}te d'Ivoire & CENTRE-NORD & 85-89 & 116.88 & 176.47 & 75.57 & HT-Direct \\ 
  C\^{o}te d'Ivoire & CENTRE-NORD & 85-89 & 122.43 & 89.54 & 163.77 & RW2 \\ 
  C\^{o}te d'Ivoire & CENTRE-NORD & 90-94 & 82.78 & 137.49 & 48.62 & HT-Direct \\ 
  C\^{o}te d'Ivoire & CENTRE-NORD & 90-94 & 110.27 & 85.08 & 141.54 & RW2 \\ 
  C\^{o}te d'Ivoire & CENTRE-NORD & 95-99 & 132.57 & 218.07 & 77.27 & HT-Direct \\ 
  C\^{o}te d'Ivoire & CENTRE-NORD & 95-99 & 105.61 & 84.23 & 131.41 & RW2 \\ 
  C\^{o}te d'Ivoire & CENTRE-NORD & 00-04 & 71.74 & 109.00 & 46.55 & HT-Direct \\ 
  C\^{o}te d'Ivoire & CENTRE-NORD & 00-04 & 97.29 & 77.87 & 120.85 & RW2 \\ 
  C\^{o}te d'Ivoire & CENTRE-NORD & 05-09 & 81.70 & 117.29 & 56.22 & HT-Direct \\ 
  C\^{o}te d'Ivoire & CENTRE-NORD & 05-09 & 79.81 & 60.68 & 104.11 & RW2 \\ 
  C\^{o}te d'Ivoire & CENTRE-NORD & 10-14 & 57.26 & 112.87 & 28.17 & HT-Direct \\ 
  C\^{o}te d'Ivoire & CENTRE-NORD & 10-14 & 66.34 & 44.38 & 98.12 & RW2 \\ 
  C\^{o}te d'Ivoire & CENTRE-NORD & 15-19 & 55.93 & 19.71 & 151.53 & RW2 \\ 
  C\^{o}te d'Ivoire & CENTRE-OUEST & 80-84 & 53.46 & 189.78 & 13.44 & HT-Direct \\ 
  C\^{o}te d'Ivoire & CENTRE-OUEST & 80-84 & 174.79 & 99.11 & 288.65 & RW2 \\ 
  C\^{o}te d'Ivoire & CENTRE-OUEST & 85-89 & 95.55 & 203.46 & 41.86 & HT-Direct \\ 
  C\^{o}te d'Ivoire & CENTRE-OUEST & 85-89 & 135.56 & 91.25 & 197.01 & RW2 \\ 
  C\^{o}te d'Ivoire & CENTRE-OUEST & 90-94 & 143.56 & 197.48 & 102.48 & HT-Direct \\ 
  C\^{o}te d'Ivoire & CENTRE-OUEST & 90-94 & 124.87 & 95.59 & 161.67 & RW2 \\ 
  C\^{o}te d'Ivoire & CENTRE-OUEST & 95-99 & 92.43 & 129.04 & 65.42 & HT-Direct \\ 
  C\^{o}te d'Ivoire & CENTRE-OUEST & 95-99 & 120.62 & 99.04 & 146.20 & RW2 \\ 
  C\^{o}te d'Ivoire & CENTRE-OUEST & 00-04 & 119.58 & 168.48 & 83.46 & HT-Direct \\ 
  C\^{o}te d'Ivoire & CENTRE-OUEST & 00-04 & 111.23 & 93.04 & 132.57 & RW2 \\ 
  C\^{o}te d'Ivoire & CENTRE-OUEST & 05-09 & 84.60 & 110.52 & 64.33 & HT-Direct \\ 
  C\^{o}te d'Ivoire & CENTRE-OUEST & 05-09 & 90.84 & 72.46 & 113.01 & RW2 \\ 
  C\^{o}te d'Ivoire & CENTRE-OUEST & 10-14 & 70.92 & 139.11 & 34.81 & HT-Direct \\ 
  C\^{o}te d'Ivoire & CENTRE-OUEST & 10-14 & 75.03 & 51.53 & 107.52 & RW2 \\ 
  C\^{o}te d'Ivoire & CENTRE-OUEST & 15-19 & 63.23 & 22.53 & 165.02 & RW2 \\ 
  C\^{o}te d'Ivoire & NORD & 80-84 & 234.48 & 394.16 & 126.03 & HT-Direct \\ 
  C\^{o}te d'Ivoire & NORD & 80-84 & 247.92 & 158.70 & 364.52 & RW2 \\ 
  C\^{o}te d'Ivoire & NORD & 85-89 & 185.93 & 296.60 & 110.09 & HT-Direct \\ 
  C\^{o}te d'Ivoire & NORD & 85-89 & 213.44 & 156.04 & 283.80 & RW2 \\ 
  C\^{o}te d'Ivoire & NORD & 90-94 & 192.67 & 277.88 & 128.93 & HT-Direct \\ 
  C\^{o}te d'Ivoire & NORD & 90-94 & 217.05 & 174.60 & 266.40 & RW2 \\ 
  C\^{o}te d'Ivoire & NORD & 95-99 & 223.74 & 273.84 & 180.52 & HT-Direct \\ 
  C\^{o}te d'Ivoire & NORD & 95-99 & 232.72 & 200.91 & 267.63 & RW2 \\ 
  C\^{o}te d'Ivoire & NORD & 00-04 & 244.98 & 289.95 & 204.97 & HT-Direct \\ 
  C\^{o}te d'Ivoire & NORD & 00-04 & 240.39 & 214.03 & 268.99 & RW2 \\ 
  C\^{o}te d'Ivoire & NORD & 05-09 & 211.03 & 246.95 & 179.10 & HT-Direct \\ 
  C\^{o}te d'Ivoire & NORD & 05-09 & 223.92 & 197.19 & 253.48 & RW2 \\ 
  C\^{o}te d'Ivoire & NORD & 10-14 & 211.81 & 274.95 & 159.96 & HT-Direct \\ 
  C\^{o}te d'Ivoire & NORD & 10-14 & 211.57 & 170.01 & 260.34 & RW2 \\ 
  C\^{o}te d'Ivoire & NORD & 15-19 & 202.53 & 85.46 & 413.40 & RW2 \\ 
  C\^{o}te d'Ivoire & NORD-EST & 80-84 & 274.70 & 461.90 & 143.19 & HT-Direct \\ 
  C\^{o}te d'Ivoire & NORD-EST & 80-84 & 216.17 & 129.74 & 340.78 & RW2 \\ 
  C\^{o}te d'Ivoire & NORD-EST & 85-89 & 117.48 & 224.25 & 57.76 & HT-Direct \\ 
  C\^{o}te d'Ivoire & NORD-EST & 85-89 & 167.35 & 115.35 & 236.99 & RW2 \\ 
  C\^{o}te d'Ivoire & NORD-EST & 90-94 & 182.93 & 269.00 & 119.89 & HT-Direct \\ 
  C\^{o}te d'Ivoire & NORD-EST & 90-94 & 153.76 & 118.15 & 197.33 & RW2 \\ 
  C\^{o}te d'Ivoire & NORD-EST & 95-99 & 118.32 & 157.40 & 87.93 & HT-Direct \\ 
  C\^{o}te d'Ivoire & NORD-EST & 95-99 & 151.15 & 124.25 & 182.14 & RW2 \\ 
  C\^{o}te d'Ivoire & NORD-EST & 00-04 & 145.72 & 199.23 & 104.71 & HT-Direct \\ 
  C\^{o}te d'Ivoire & NORD-EST & 00-04 & 145.94 & 120.96 & 175.15 & RW2 \\ 
  C\^{o}te d'Ivoire & NORD-EST & 05-09 & 129.59 & 191.22 & 85.72 & HT-Direct \\ 
  C\^{o}te d'Ivoire & NORD-EST & 05-09 & 127.42 & 99.84 & 160.99 & RW2 \\ 
  C\^{o}te d'Ivoire & NORD-EST & 10-14 & 116.29 & 179.33 & 73.42 & HT-Direct \\ 
  C\^{o}te d'Ivoire & NORD-EST & 10-14 & 112.60 & 78.96 & 158.41 & RW2 \\ 
  C\^{o}te d'Ivoire & NORD-EST & 15-19 & 100.67 & 37.67 & 246.80 & RW2 \\ 
  C\^{o}te d'Ivoire & NORD-OUEST & 80-84 & 224.51 & 321.45 & 150.33 & HT-Direct \\ 
  C\^{o}te d'Ivoire & NORD-OUEST & 80-84 & 226.71 & 158.30 & 311.51 & RW2 \\ 
  C\^{o}te d'Ivoire & NORD-OUEST & 85-89 & 168.60 & 235.20 & 117.95 & HT-Direct \\ 
  C\^{o}te d'Ivoire & NORD-OUEST & 85-89 & 195.99 & 153.51 & 246.58 & RW2 \\ 
  C\^{o}te d'Ivoire & NORD-OUEST & 90-94 & 144.36 & 194.88 & 105.22 & HT-Direct \\ 
  C\^{o}te d'Ivoire & NORD-OUEST & 90-94 & 200.70 & 167.86 & 238.36 & RW2 \\ 
  C\^{o}te d'Ivoire & NORD-OUEST & 95-99 & 226.21 & 272.04 & 186.12 & HT-Direct \\ 
  C\^{o}te d'Ivoire & NORD-OUEST & 95-99 & 215.87 & 190.17 & 244.69 & RW2 \\ 
  C\^{o}te d'Ivoire & NORD-OUEST & 00-04 & 223.28 & 254.77 & 194.67 & HT-Direct \\ 
  C\^{o}te d'Ivoire & NORD-OUEST & 00-04 & 220.26 & 198.75 & 243.46 & RW2 \\ 
  C\^{o}te d'Ivoire & NORD-OUEST & 05-09 & 184.22 & 219.80 & 153.28 & HT-Direct \\ 
  C\^{o}te d'Ivoire & NORD-OUEST & 05-09 & 200.06 & 174.52 & 228.26 & RW2 \\ 
  C\^{o}te d'Ivoire & NORD-OUEST & 10-14 & 183.99 & 260.85 & 125.91 & HT-Direct \\ 
  C\^{o}te d'Ivoire & NORD-OUEST & 10-14 & 184.08 & 142.92 & 232.85 & RW2 \\ 
  C\^{o}te d'Ivoire & NORD-OUEST & 15-19 & 171.47 & 69.78 & 366.07 & RW2 \\ 
  C\^{o}te d'Ivoire & OUEST & 80-84 & 135.65 & 295.00 & 55.59 & HT-Direct \\ 
  C\^{o}te d'Ivoire & OUEST & 80-84 & 319.33 & 201.30 & 448.68 & RW2 \\ 
  C\^{o}te d'Ivoire & OUEST & 85-89 & 226.42 & 334.19 & 145.79 & HT-Direct \\ 
  C\^{o}te d'Ivoire & OUEST & 85-89 & 253.74 & 191.66 & 323.72 & RW2 \\ 
  C\^{o}te d'Ivoire & OUEST & 90-94 & 207.04 & 257.76 & 164.09 & HT-Direct \\ 
  C\^{o}te d'Ivoire & OUEST & 90-94 & 231.59 & 195.31 & 272.27 & RW2 \\ 
  C\^{o}te d'Ivoire & OUEST & 95-99 & 220.44 & 256.47 & 188.19 & HT-Direct \\ 
  C\^{o}te d'Ivoire & OUEST & 95-99 & 217.31 & 192.95 & 244.81 & RW2 \\ 
  C\^{o}te d'Ivoire & OUEST & 00-04 & 193.03 & 224.33 & 165.17 & HT-Direct \\ 
  C\^{o}te d'Ivoire & OUEST & 00-04 & 190.30 & 170.03 & 212.67 & RW2 \\ 
  C\^{o}te d'Ivoire & OUEST & 05-09 & 123.79 & 153.40 & 99.22 & HT-Direct \\ 
  C\^{o}te d'Ivoire & OUEST & 05-09 & 146.19 & 123.99 & 171.27 & RW2 \\ 
  C\^{o}te d'Ivoire & OUEST & 10-14 & 132.87 & 208.99 & 81.61 & HT-Direct \\ 
  C\^{o}te d'Ivoire & OUEST & 10-14 & 113.42 & 83.08 & 151.76 & RW2 \\ 
  C\^{o}te d'Ivoire & OUEST & 15-19 & 88.99 & 32.88 & 219.30 & RW2 \\ 
  C\^{o}te d'Ivoire & SUD SANS ABIDJAN & 80-84 & 93.77 & 285.60 & 26.08 & HT-Direct \\ 
  C\^{o}te d'Ivoire & SUD SANS ABIDJAN & 80-84 & 192.62 & 112.24 & 309.48 & RW2 \\ 
  C\^{o}te d'Ivoire & SUD SANS ABIDJAN & 85-89 & 138.81 & 247.04 & 73.37 & HT-Direct \\ 
  C\^{o}te d'Ivoire & SUD SANS ABIDJAN & 85-89 & 152.46 & 105.12 & 215.84 & RW2 \\ 
  C\^{o}te d'Ivoire & SUD SANS ABIDJAN & 90-94 & 126.90 & 181.23 & 87.13 & HT-Direct \\ 
  C\^{o}te d'Ivoire & SUD SANS ABIDJAN & 90-94 & 142.81 & 110.56 & 182.44 & RW2 \\ 
  C\^{o}te d'Ivoire & SUD SANS ABIDJAN & 95-99 & 137.66 & 181.86 & 102.86 & HT-Direct \\ 
  C\^{o}te d'Ivoire & SUD SANS ABIDJAN & 95-99 & 141.15 & 117.83 & 168.01 & RW2 \\ 
  C\^{o}te d'Ivoire & SUD SANS ABIDJAN & 00-04 & 139.43 & 180.57 & 106.45 & HT-Direct \\ 
  C\^{o}te d'Ivoire & SUD SANS ABIDJAN & 00-04 & 134.73 & 115.82 & 156.35 & RW2 \\ 
  C\^{o}te d'Ivoire & SUD SANS ABIDJAN & 05-09 & 113.15 & 138.47 & 91.96 & HT-Direct \\ 
  C\^{o}te d'Ivoire & SUD SANS ABIDJAN & 05-09 & 114.67 & 96.08 & 136.44 & RW2 \\ 
  C\^{o}te d'Ivoire & SUD SANS ABIDJAN & 10-14 & 70.40 & 130.08 & 36.94 & HT-Direct \\ 
  C\^{o}te d'Ivoire & SUD SANS ABIDJAN & 10-14 & 98.13 & 70.65 & 134.37 & RW2 \\ 
  C\^{o}te d'Ivoire & SUD SANS ABIDJAN & 15-19 & 84.60 & 30.86 & 212.76 & RW2 \\ 
  C\^{o}te d'Ivoire & SUD-OUEST & 80-84 & 102.08 & 326.44 & 25.97 & HT-Direct \\ 
  C\^{o}te d'Ivoire & SUD-OUEST & 80-84 & 196.26 & 107.84 & 329.72 & RW2 \\ 
  C\^{o}te d'Ivoire & SUD-OUEST & 85-89 & 152.51 & 255.09 & 86.40 & HT-Direct \\ 
  C\^{o}te d'Ivoire & SUD-OUEST & 85-89 & 145.75 & 95.60 & 216.84 & RW2 \\ 
  C\^{o}te d'Ivoire & SUD-OUEST & 90-94 & 122.21 & 189.03 & 76.77 & HT-Direct \\ 
  C\^{o}te d'Ivoire & SUD-OUEST & 90-94 & 127.13 & 92.40 & 172.24 & RW2 \\ 
  C\^{o}te d'Ivoire & SUD-OUEST & 95-99 & 123.36 & 234.63 & 60.68 & HT-Direct \\ 
  C\^{o}te d'Ivoire & SUD-OUEST & 95-99 & 116.32 & 89.99 & 148.95 & RW2 \\ 
  C\^{o}te d'Ivoire & SUD-OUEST & 00-04 & 65.35 & 107.39 & 39.05 & HT-Direct \\ 
  C\^{o}te d'Ivoire & SUD-OUEST & 00-04 & 101.58 & 81.65 & 125.22 & RW2 \\ 
  C\^{o}te d'Ivoire & SUD-OUEST & 05-09 & 78.68 & 101.25 & 60.80 & HT-Direct \\ 
  C\^{o}te d'Ivoire & SUD-OUEST & 05-09 & 79.32 & 63.23 & 99.19 & RW2 \\ 
  C\^{o}te d'Ivoire & SUD-OUEST & 10-14 & 78.02 & 156.78 & 37.09 & HT-Direct \\ 
  C\^{o}te d'Ivoire & SUD-OUEST & 10-14 & 63.23 & 42.98 & 92.49 & RW2 \\ 
  C\^{o}te d'Ivoire & SUD-OUEST & 15-19 & 51.19 & 17.60 & 141.52 & RW2 \\ 
  C\^{o}te d'Ivoire & VILLE D'ABIDJAN & 80-84 & 125.46 & 286.42 & 48.77 & HT-Direct \\ 
  C\^{o}te d'Ivoire & VILLE D'ABIDJAN & 80-84 & 151.11 & 82.69 & 268.15 & RW2 \\ 
  C\^{o}te d'Ivoire & VILLE D'ABIDJAN & 85-89 & 133.37 & 225.15 & 75.36 & HT-Direct \\ 
  C\^{o}te d'Ivoire & VILLE D'ABIDJAN & 85-89 & 116.44 & 78.37 & 170.86 & RW2 \\ 
  C\^{o}te d'Ivoire & VILLE D'ABIDJAN & 90-94 & 88.03 & 135.03 & 56.33 & HT-Direct \\ 
  C\^{o}te d'Ivoire & VILLE D'ABIDJAN & 90-94 & 107.24 & 81.50 & 140.09 & RW2 \\ 
  C\^{o}te d'Ivoire & VILLE D'ABIDJAN & 95-99 & 103.45 & 138.84 & 76.28 & HT-Direct \\ 
  C\^{o}te d'Ivoire & VILLE D'ABIDJAN & 95-99 & 106.28 & 85.81 & 130.26 & RW2 \\ 
  C\^{o}te d'Ivoire & VILLE D'ABIDJAN & 00-04 & 83.74 & 117.33 & 59.13 & HT-Direct \\ 
  C\^{o}te d'Ivoire & VILLE D'ABIDJAN & 00-04 & 104.21 & 85.62 & 125.35 & RW2 \\ 
  C\^{o}te d'Ivoire & VILLE D'ABIDJAN & 05-09 & 99.29 & 128.09 & 76.39 & HT-Direct \\ 
  C\^{o}te d'Ivoire & VILLE D'ABIDJAN & 05-09 & 93.50 & 75.83 & 115.18 & RW2 \\ 
  C\^{o}te d'Ivoire & VILLE D'ABIDJAN & 10-14 & 80.05 & 132.59 & 47.20 & HT-Direct \\ 
  C\^{o}te d'Ivoire & VILLE D'ABIDJAN & 10-14 & 85.32 & 59.22 & 122.63 & RW2 \\ 
  C\^{o}te d'Ivoire & VILLE D'ABIDJAN & 15-19 & 78.87 & 27.29 & 211.63 & RW2 \\ 
  DRC & ALL & 80-84 & 177.53 & 170.43 & 184.58 & IHME \\ 
  DRC & ALL & 80-84 & 206.56 & 169.75 & 249.04 & RW2 \\ 
  DRC & ALL & 80-84 & 205.54 & 194.81 & 217.49 & UN \\ 
  DRC & ALL & 85-89 & 161.02 & 156.18 & 166.64 & IHME \\ 
  DRC & ALL & 85-89 & 190.70 & 167.27 & 216.23 & RW2 \\ 
  DRC & ALL & 85-89 & 192.63 & 184.31 & 202.27 & UN \\ 
  DRC & ALL & 90-94 & 165.89 & 160.54 & 171.62 & IHME \\ 
  DRC & ALL & 90-94 & 183.69 & 167.32 & 201.32 & RW2 \\ 
  DRC & ALL & 90-94 & 182.13 & 174.46 & 190.42 & UN \\ 
  DRC & ALL & 95-99 & 162.41 & 157.27 & 167.95 & IHME \\ 
  DRC & ALL & 95-99 & 171.81 & 159.01 & 185.31 & RW2 \\ 
  DRC & ALL & 95-99 & 171.66 & 164.49 & 179.34 & UN \\ 
  DRC & ALL & 00-04 & 142.43 & 137.67 & 147.57 & IHME \\ 
  DRC & ALL & 00-04 & 151.93 & 142.57 & 161.84 & RW2 \\ 
  DRC & ALL & 00-04 & 152.22 & 146.01 & 158.37 & UN \\ 
  DRC & ALL & 05-09 & 120.74 & 116.11 & 125.82 & IHME \\ 
  DRC & ALL & 05-09 & 129.30 & 120.20 & 138.96 & RW2 \\ 
  DRC & ALL & 05-09 & 129.24 & 122.26 & 136.71 & UN \\ 
  DRC & ALL & 10-14 & 99.73 & 94.09 & 106.03 & IHME \\ 
  DRC & ALL & 10-14 & 107.92 & 98.21 & 118.39 & RW2 \\ 
  DRC & ALL & 10-14 & 107.92 & 97.62 & 119.22 & UN \\ 
  DRC & BANDUNDU & 80-84 & 151.31 & 208.25 & 107.82 & HT-Direct \\ 
  DRC & BANDUNDU & 80-84 & 238.22 & 178.63 & 309.96 & RW2 \\ 
  DRC & BANDUNDU & 85-89 & 153.26 & 238.72 & 94.59 & HT-Direct \\ 
  DRC & BANDUNDU & 85-89 & 207.61 & 166.61 & 256.08 & RW2 \\ 
  DRC & BANDUNDU & 90-94 & 204.52 & 253.73 & 162.77 & HT-Direct \\ 
  DRC & BANDUNDU & 90-94 & 177.66 & 151.75 & 207.07 & RW2 \\ 
  DRC & BANDUNDU & 95-99 & 162.80 & 210.22 & 124.39 & HT-Direct \\ 
  DRC & BANDUNDU & 95-99 & 161.33 & 141.94 & 182.78 & RW2 \\ 
  DRC & BANDUNDU & 00-04 & 117.68 & 141.98 & 97.07 & HT-Direct \\ 
  DRC & BANDUNDU & 00-04 & 133.39 & 118.42 & 149.94 & RW2 \\ 
  DRC & BANDUNDU & 05-09 & 100.27 & 122.57 & 81.65 & HT-Direct \\ 
  DRC & BANDUNDU & 05-09 & 101.42 & 87.91 & 116.68 & RW2 \\ 
  DRC & BANDUNDU & 10-14 & 72.77 & 95.81 & 54.93 & HT-Direct \\ 
  DRC & BANDUNDU & 10-14 & 79.50 & 64.53 & 97.65 & RW2 \\ 
  DRC & BANDUNDU & 15-19 & 63.28 & 26.72 & 141.66 & RW2 \\ 
  DRC & BAS-CONGO & 80-84 & 183.53 & 273.26 & 118.47 & HT-Direct \\ 
  DRC & BAS-CONGO & 80-84 & 258.13 & 184.46 & 345.98 & RW2 \\ 
  DRC & BAS-CONGO & 85-89 & 186.01 & 251.34 & 134.61 & HT-Direct \\ 
  DRC & BAS-CONGO & 85-89 & 235.01 & 187.22 & 290.60 & RW2 \\ 
  DRC & BAS-CONGO & 90-94 & 226.00 & 295.06 & 169.23 & HT-Direct \\ 
  DRC & BAS-CONGO & 90-94 & 210.71 & 179.20 & 246.14 & RW2 \\ 
  DRC & BAS-CONGO & 95-99 & 210.85 & 253.40 & 173.78 & HT-Direct \\ 
  DRC & BAS-CONGO & 95-99 & 200.69 & 178.06 & 225.47 & RW2 \\ 
  DRC & BAS-CONGO & 00-04 & 164.02 & 190.52 & 140.56 & HT-Direct \\ 
  DRC & BAS-CONGO & 00-04 & 175.05 & 156.16 & 195.41 & RW2 \\ 
  DRC & BAS-CONGO & 05-09 & 131.12 & 168.66 & 100.92 & HT-Direct \\ 
  DRC & BAS-CONGO & 05-09 & 141.51 & 120.41 & 165.66 & RW2 \\ 
  DRC & BAS-CONGO & 10-14 & 117.10 & 163.64 & 82.50 & HT-Direct \\ 
  DRC & BAS-CONGO & 10-14 & 118.36 & 92.14 & 151.30 & RW2 \\ 
  DRC & BAS-CONGO & 15-19 & 100.38 & 42.18 & 222.50 & RW2 \\ 
  DRC & EQUATEUR & 80-84 & 141.17 & 208.45 & 93.05 & HT-Direct \\ 
  DRC & EQUATEUR & 80-84 & 190.05 & 139.32 & 253.49 & RW2 \\ 
  DRC & EQUATEUR & 85-89 & 160.67 & 206.67 & 123.31 & HT-Direct \\ 
  DRC & EQUATEUR & 85-89 & 184.06 & 147.56 & 226.12 & RW2 \\ 
  DRC & EQUATEUR & 90-94 & 217.44 & 293.70 & 156.59 & HT-Direct \\ 
  DRC & EQUATEUR & 90-94 & 175.86 & 148.75 & 206.62 & RW2 \\ 
  DRC & EQUATEUR & 95-99 & 166.05 & 217.22 & 125.01 & HT-Direct \\ 
  DRC & EQUATEUR & 95-99 & 180.37 & 158.03 & 204.95 & RW2 \\ 
  DRC & EQUATEUR & 00-04 & 148.06 & 178.74 & 121.86 & HT-Direct \\ 
  DRC & EQUATEUR & 00-04 & 170.23 & 151.24 & 190.91 & RW2 \\ 
  DRC & EQUATEUR & 05-09 & 141.63 & 172.73 & 115.36 & HT-Direct \\ 
  DRC & EQUATEUR & 05-09 & 149.23 & 130.25 & 170.35 & RW2 \\ 
  DRC & EQUATEUR & 10-14 & 141.25 & 178.11 & 110.99 & HT-Direct \\ 
  DRC & EQUATEUR & 10-14 & 135.07 & 111.87 & 162.41 & RW2 \\ 
  DRC & EQUATEUR & 15-19 & 123.79 & 55.20 & 257.08 & RW2 \\ 
  DRC & KASAI-OCCIDENTAL & 80-84 & 251.38 & 359.52 & 167.27 & HT-Direct \\ 
  DRC & KASAI-OCCIDENTAL & 80-84 & 258.25 & 193.80 & 336.20 & RW2 \\ 
  DRC & KASAI-OCCIDENTAL & 85-89 & 197.53 & 254.22 & 150.93 & HT-Direct \\ 
  DRC & KASAI-OCCIDENTAL & 85-89 & 236.00 & 194.24 & 284.10 & RW2 \\ 
  DRC & KASAI-OCCIDENTAL & 90-94 & 189.70 & 237.17 & 149.85 & HT-Direct \\ 
  DRC & KASAI-OCCIDENTAL & 90-94 & 212.96 & 186.35 & 242.36 & RW2 \\ 
  DRC & KASAI-OCCIDENTAL & 95-99 & 210.95 & 239.83 & 184.69 & HT-Direct \\ 
  DRC & KASAI-OCCIDENTAL & 95-99 & 205.70 & 187.10 & 225.68 & RW2 \\ 
  DRC & KASAI-OCCIDENTAL & 00-04 & 177.00 & 210.87 & 147.55 & HT-Direct \\ 
  DRC & KASAI-OCCIDENTAL & 00-04 & 181.81 & 163.43 & 201.93 & RW2 \\ 
  DRC & KASAI-OCCIDENTAL & 05-09 & 145.26 & 176.94 & 118.44 & HT-Direct \\ 
  DRC & KASAI-OCCIDENTAL & 05-09 & 147.30 & 126.76 & 170.26 & RW2 \\ 
  DRC & KASAI-OCCIDENTAL & 10-14 & 105.02 & 152.58 & 71.04 & HT-Direct \\ 
  DRC & KASAI-OCCIDENTAL & 10-14 & 121.73 & 96.17 & 152.45 & RW2 \\ 
  DRC & KASAI-OCCIDENTAL & 15-19 & 101.96 & 43.76 & 218.88 & RW2 \\ 
  DRC & KASAI-ORIENTAL & 80-84 & 129.05 & 195.97 & 82.63 & HT-Direct \\ 
  DRC & KASAI-ORIENTAL & 80-84 & 204.83 & 155.91 & 264.46 & RW2 \\ 
  DRC & KASAI-ORIENTAL & 85-89 & 194.34 & 237.70 & 157.25 & HT-Direct \\ 
  DRC & KASAI-ORIENTAL & 85-89 & 192.61 & 161.34 & 228.02 & RW2 \\ 
  DRC & KASAI-ORIENTAL & 90-94 & 174.12 & 209.68 & 143.49 & HT-Direct \\ 
  DRC & KASAI-ORIENTAL & 90-94 & 178.70 & 158.42 & 200.88 & RW2 \\ 
  DRC & KASAI-ORIENTAL & 95-99 & 170.51 & 192.33 & 150.70 & HT-Direct \\ 
  DRC & KASAI-ORIENTAL & 95-99 & 178.31 & 163.47 & 194.01 & RW2 \\ 
  DRC & KASAI-ORIENTAL & 00-04 & 161.36 & 186.33 & 139.17 & HT-Direct \\ 
  DRC & KASAI-ORIENTAL & 00-04 & 163.00 & 148.17 & 179.05 & RW2 \\ 
  DRC & KASAI-ORIENTAL & 05-09 & 131.50 & 158.56 & 108.46 & HT-Direct \\ 
  DRC & KASAI-ORIENTAL & 05-09 & 136.37 & 119.80 & 155.07 & RW2 \\ 
  DRC & KASAI-ORIENTAL & 10-14 & 111.25 & 145.55 & 84.23 & HT-Direct \\ 
  DRC & KASAI-ORIENTAL & 10-14 & 116.76 & 96.26 & 141.12 & RW2 \\ 
  DRC & KASAI-ORIENTAL & 15-19 & 101.05 & 44.28 & 216.24 & RW2 \\ 
  DRC & KATANGA & 80-84 & 144.81 & 215.69 & 94.42 & HT-Direct \\ 
  DRC & KATANGA & 80-84 & 209.83 & 157.40 & 272.74 & RW2 \\ 
  DRC & KATANGA & 85-89 & 180.11 & 231.00 & 138.41 & HT-Direct \\ 
  DRC & KATANGA & 85-89 & 200.51 & 165.63 & 240.44 & RW2 \\ 
  DRC & KATANGA & 90-94 & 193.08 & 235.59 & 156.68 & HT-Direct \\ 
  DRC & KATANGA & 90-94 & 189.09 & 165.59 & 214.82 & RW2 \\ 
  DRC & KATANGA & 95-99 & 184.51 & 217.86 & 155.26 & HT-Direct \\ 
  DRC & KATANGA & 95-99 & 190.95 & 172.51 & 210.88 & RW2 \\ 
  DRC & KATANGA & 00-04 & 173.01 & 199.14 & 149.67 & HT-Direct \\ 
  DRC & KATANGA & 00-04 & 175.55 & 159.87 & 192.77 & RW2 \\ 
  DRC & KATANGA & 05-09 & 130.85 & 154.29 & 110.51 & HT-Direct \\ 
  DRC & KATANGA & 05-09 & 147.41 & 132.73 & 163.24 & RW2 \\ 
  DRC & KATANGA & 10-14 & 124.20 & 140.95 & 109.19 & HT-Direct \\ 
  DRC & KATANGA & 10-14 & 126.71 & 112.48 & 142.38 & RW2 \\ 
  DRC & KATANGA & 15-19 & 109.94 & 50.36 & 225.34 & RW2 \\ 
  DRC & KINSHASA & 80-84 & 147.77 & 205.11 & 104.36 & HT-Direct \\ 
  DRC & KINSHASA & 80-84 & 129.09 & 94.08 & 176.80 & RW2 \\ 
  DRC & KINSHASA & 85-89 & 87.95 & 118.44 & 64.72 & HT-Direct \\ 
  DRC & KINSHASA & 85-89 & 119.38 & 96.71 & 147.12 & RW2 \\ 
  DRC & KINSHASA & 90-94 & 105.20 & 129.03 & 85.34 & HT-Direct \\ 
  DRC & KINSHASA & 90-94 & 110.22 & 95.95 & 126.42 & RW2 \\ 
  DRC & KINSHASA & 95-99 & 106.92 & 125.85 & 90.55 & HT-Direct \\ 
  DRC & KINSHASA & 95-99 & 111.32 & 99.79 & 123.89 & RW2 \\ 
  DRC & KINSHASA & 00-04 & 103.03 & 118.45 & 89.40 & HT-Direct \\ 
  DRC & KINSHASA & 00-04 & 104.33 & 93.79 & 115.66 & RW2 \\ 
  DRC & KINSHASA & 05-09 & 78.25 & 98.19 & 62.07 & HT-Direct \\ 
  DRC & KINSHASA & 05-09 & 91.42 & 79.68 & 104.73 & RW2 \\ 
  DRC & KINSHASA & 10-14 & 86.87 & 109.00 & 68.88 & HT-Direct \\ 
  DRC & KINSHASA & 10-14 & 84.18 & 69.27 & 102.46 & RW2 \\ 
  DRC & KINSHASA & 15-19 & 79.04 & 34.00 & 175.08 & RW2 \\ 
  DRC & MANIEMA & 80-84 & 131.24 & 212.84 & 77.83 & HT-Direct \\ 
  DRC & MANIEMA & 80-84 & 187.13 & 134.22 & 251.84 & RW2 \\ 
  DRC & MANIEMA & 85-89 & 163.64 & 228.40 & 114.52 & HT-Direct \\ 
  DRC & MANIEMA & 85-89 & 184.65 & 147.46 & 227.57 & RW2 \\ 
  DRC & MANIEMA & 90-94 & 140.57 & 174.98 & 112.01 & HT-Direct \\ 
  DRC & MANIEMA & 90-94 & 180.41 & 155.86 & 207.78 & RW2 \\ 
  DRC & MANIEMA & 95-99 & 223.91 & 262.36 & 189.64 & HT-Direct \\ 
  DRC & MANIEMA & 95-99 & 188.86 & 169.30 & 210.80 & RW2 \\ 
  DRC & MANIEMA & 00-04 & 196.16 & 249.20 & 152.13 & HT-Direct \\ 
  DRC & MANIEMA & 00-04 & 177.16 & 157.92 & 199.00 & RW2 \\ 
  DRC & MANIEMA & 05-09 & 135.61 & 162.29 & 112.73 & HT-Direct \\ 
  DRC & MANIEMA & 05-09 & 150.01 & 131.02 & 171.13 & RW2 \\ 
  DRC & MANIEMA & 10-14 & 110.82 & 144.08 & 84.49 & HT-Direct \\ 
  DRC & MANIEMA & 10-14 & 128.89 & 105.46 & 156.15 & RW2 \\ 
  DRC & MANIEMA & 15-19 & 111.89 & 48.97 & 235.38 & RW2 \\ 
  DRC & NORD-KIVU & 80-84 & 239.97 & 330.11 & 168.25 & HT-Direct \\ 
  DRC & NORD-KIVU & 80-84 & 266.50 & 194.62 & 353.65 & RW2 \\ 
  DRC & NORD-KIVU & 85-89 & 127.40 & 189.88 & 83.36 & HT-Direct \\ 
  DRC & NORD-KIVU & 85-89 & 217.27 & 170.83 & 272.13 & RW2 \\ 
  DRC & NORD-KIVU & 90-94 & 186.16 & 246.78 & 137.70 & HT-Direct \\ 
  DRC & NORD-KIVU & 90-94 & 173.61 & 145.03 & 206.27 & RW2 \\ 
  DRC & NORD-KIVU & 95-99 & 144.53 & 183.51 & 112.68 & HT-Direct \\ 
  DRC & NORD-KIVU & 95-99 & 147.51 & 127.73 & 169.66 & RW2 \\ 
  DRC & NORD-KIVU & 00-04 & 90.39 & 124.57 & 64.90 & HT-Direct \\ 
  DRC & NORD-KIVU & 00-04 & 112.25 & 96.80 & 130.03 & RW2 \\ 
  DRC & NORD-KIVU & 05-09 & 80.82 & 99.85 & 65.15 & HT-Direct \\ 
  DRC & NORD-KIVU & 05-09 & 76.40 & 63.83 & 91.27 & RW2 \\ 
  DRC & NORD-KIVU & 10-14 & 36.08 & 59.54 & 21.65 & HT-Direct \\ 
  DRC & NORD-KIVU & 10-14 & 52.47 & 39.46 & 69.28 & RW2 \\ 
  DRC & NORD-KIVU & 15-19 & 36.14 & 14.40 & 87.93 & RW2 \\ 
  DRC & ORIENTALE & 80-84 & 193.65 & 312.05 & 112.80 & HT-Direct \\ 
  DRC & ORIENTALE & 80-84 & 227.06 & 163.58 & 306.55 & RW2 \\ 
  DRC & ORIENTALE & 85-89 & 189.78 & 250.12 & 141.26 & HT-Direct \\ 
  DRC & ORIENTALE & 85-89 & 208.56 & 166.13 & 259.10 & RW2 \\ 
  DRC & ORIENTALE & 90-94 & 162.38 & 219.21 & 118.06 & HT-Direct \\ 
  DRC & ORIENTALE & 90-94 & 189.17 & 160.27 & 221.73 & RW2 \\ 
  DRC & ORIENTALE & 95-99 & 192.78 & 234.29 & 157.11 & HT-Direct \\ 
  DRC & ORIENTALE & 95-99 & 184.36 & 163.06 & 207.83 & RW2 \\ 
  DRC & ORIENTALE & 00-04 & 154.71 & 187.77 & 126.56 & HT-Direct \\ 
  DRC & ORIENTALE & 00-04 & 163.19 & 145.35 & 182.85 & RW2 \\ 
  DRC & ORIENTALE & 05-09 & 124.07 & 150.55 & 101.69 & HT-Direct \\ 
  DRC & ORIENTALE & 05-09 & 131.66 & 114.85 & 150.63 & RW2 \\ 
  DRC & ORIENTALE & 10-14 & 105.48 & 135.52 & 81.47 & HT-Direct \\ 
  DRC & ORIENTALE & 10-14 & 108.41 & 88.78 & 131.56 & RW2 \\ 
  DRC & ORIENTALE & 15-19 & 90.03 & 39.07 & 193.50 & RW2 \\ 
  DRC & SUD-KIVU & 80-84 & 252.32 & 331.34 & 186.88 & HT-Direct \\ 
  DRC & SUD-KIVU & 80-84 & 267.83 & 200.97 & 347.48 & RW2 \\ 
  DRC & SUD-KIVU & 85-89 & 137.93 & 208.20 & 88.72 & HT-Direct \\ 
  DRC & SUD-KIVU & 85-89 & 247.32 & 198.86 & 302.94 & RW2 \\ 
  DRC & SUD-KIVU & 90-94 & 226.61 & 290.05 & 173.66 & HT-Direct \\ 
  DRC & SUD-KIVU & 90-94 & 226.50 & 193.71 & 263.07 & RW2 \\ 
  DRC & SUD-KIVU & 95-99 & 242.19 & 295.45 & 195.86 & HT-Direct \\ 
  DRC & SUD-KIVU & 95-99 & 221.20 & 195.76 & 249.20 & RW2 \\ 
  DRC & SUD-KIVU & 00-04 & 178.09 & 217.00 & 144.87 & HT-Direct \\ 
  DRC & SUD-KIVU & 00-04 & 195.81 & 173.74 & 220.14 & RW2 \\ 
  DRC & SUD-KIVU & 05-09 & 145.94 & 181.41 & 116.42 & HT-Direct \\ 
  DRC & SUD-KIVU & 05-09 & 157.36 & 136.20 & 181.48 & RW2 \\ 
  DRC & SUD-KIVU & 10-14 & 129.41 & 166.65 & 99.49 & HT-Direct \\ 
  DRC & SUD-KIVU & 10-14 & 128.61 & 104.39 & 157.25 & RW2 \\ 
  DRC & SUD-KIVU & 15-19 & 105.81 & 46.06 & 225.51 & RW2 \\ 
  Egypt & ALL & 80-84 & 141.99 & 138.72 & 145.18 & IHME \\ 
  Egypt & ALL & 80-84 & 147.64 & 143.17 & 152.23 & RW2 \\ 
  Egypt & ALL & 80-84 & 147.62 & 144.54 & 150.81 & UN \\ 
  Egypt & ALL & 85-89 & 97.53 & 95.24 & 99.97 & IHME \\ 
  Egypt & ALL & 85-89 & 103.18 & 99.96 & 106.47 & RW2 \\ 
  Egypt & ALL & 85-89 & 103.23 & 101.14 & 105.40 & UN \\ 
  Egypt & ALL & 90-94 & 74.53 & 72.62 & 76.55 & IHME \\ 
  Egypt & ALL & 90-94 & 77.72 & 75.04 & 80.49 & RW2 \\ 
  Egypt & ALL & 90-94 & 77.64 & 76.01 & 79.24 & UN \\ 
  Egypt & ALL & 95-99 & 52.51 & 50.86 & 54.20 & IHME \\ 
  Egypt & ALL & 95-99 & 56.79 & 54.17 & 59.51 & RW2 \\ 
  Egypt & ALL & 95-99 & 56.85 & 55.60 & 58.34 & UN \\ 
  Egypt & ALL & 00-04 & 38.05 & 36.67 & 39.38 & IHME \\ 
  Egypt & ALL & 00-04 & 41.53 & 38.97 & 44.26 & RW2 \\ 
  Egypt & ALL & 00-04 & 41.53 & 40.30 & 42.88 & UN \\ 
  Egypt & ALL & 05-09 & 31.32 & 29.78 & 32.92 & IHME \\ 
  Egypt & ALL & 05-09 & 32.82 & 29.91 & 36.01 & RW2 \\ 
  Egypt & ALL & 05-09 & 32.82 & 31.32 & 34.10 & UN \\ 
  Egypt & ALL & 10-14 & 25.79 & 24.14 & 27.47 & IHME \\ 
  Egypt & ALL & 10-14 & 26.43 & 9.81 & 68.75 & RW2 \\ 
  Egypt & ALL & 10-14 & 26.84 & 24.98 & 28.86 & UN \\ 
  Egypt & FRONTIER GOVERNORATES & 80-84 & 98.02 & 117.07 & 81.78 & HT-Direct \\ 
  Egypt & FRONTIER GOVERNORATES & 80-84 & 99.33 & 85.54 & 115.16 & RW2 \\ 
  Egypt & FRONTIER GOVERNORATES & 85-89 & 70.59 & 82.61 & 60.20 & HT-Direct \\ 
  Egypt & FRONTIER GOVERNORATES & 85-89 & 72.36 & 65.10 & 80.44 & RW2 \\ 
  Egypt & FRONTIER GOVERNORATES & 90-94 & 53.94 & 64.01 & 45.38 & HT-Direct \\ 
  Egypt & FRONTIER GOVERNORATES & 90-94 & 56.98 & 51.09 & 63.27 & RW2 \\ 
  Egypt & FRONTIER GOVERNORATES & 95-99 & 44.93 & 55.25 & 36.47 & HT-Direct \\ 
  Egypt & FRONTIER GOVERNORATES & 95-99 & 45.38 & 39.93 & 51.45 & RW2 \\ 
  Egypt & FRONTIER GOVERNORATES & 00-04 & 32.98 & 42.89 & 25.30 & HT-Direct \\ 
  Egypt & FRONTIER GOVERNORATES & 00-04 & 36.73 & 31.05 & 43.51 & RW2 \\ 
  Egypt & FRONTIER GOVERNORATES & 05-09 & 35.12 & 49.99 & 24.56 & HT-Direct \\ 
  Egypt & FRONTIER GOVERNORATES & 05-09 & 33.46 & 25.97 & 43.34 & RW2 \\ 
  Egypt & FRONTIER GOVERNORATES & 10-14 & 31.67 & 12.53 & 79.63 & RW2 \\ 
  Egypt & FRONTIER GOVERNORATES & 15-19 & 29.83 & 3.12 & 236.19 & RW2 \\ 
  Egypt & LOWER EGYPT & 80-84 & 122.40 & 128.61 & 116.45 & HT-Direct \\ 
  Egypt & LOWER EGYPT & 80-84 & 125.83 & 119.93 & 132.05 & RW2 \\ 
  Egypt & LOWER EGYPT & 85-89 & 83.34 & 87.71 & 79.18 & HT-Direct \\ 
  Egypt & LOWER EGYPT & 85-89 & 86.65 & 82.87 & 90.64 & RW2 \\ 
  Egypt & LOWER EGYPT & 90-94 & 63.23 & 67.30 & 59.40 & HT-Direct \\ 
  Egypt & LOWER EGYPT & 90-94 & 63.14 & 60.01 & 66.40 & RW2 \\ 
  Egypt & LOWER EGYPT & 95-99 & 42.66 & 46.44 & 39.18 & HT-Direct \\ 
  Egypt & LOWER EGYPT & 95-99 & 45.84 & 42.86 & 48.97 & RW2 \\ 
  Egypt & LOWER EGYPT & 00-04 & 32.65 & 36.53 & 29.17 & HT-Direct \\ 
  Egypt & LOWER EGYPT & 00-04 & 33.76 & 30.90 & 36.90 & RW2 \\ 
  Egypt & LOWER EGYPT & 05-09 & 26.74 & 31.53 & 22.66 & HT-Direct \\ 
  Egypt & LOWER EGYPT & 05-09 & 27.74 & 24.13 & 31.92 & RW2 \\ 
  Egypt & LOWER EGYPT & 10-14 & 23.57 & 9.64 & 56.81 & RW2 \\ 
  Egypt & LOWER EGYPT & 15-19 & 20.07 & 2.16 & 167.70 & RW2 \\ 
  Egypt & UPPER EGYPT & 80-84 & 195.32 & 202.99 & 187.86 & HT-Direct \\ 
  Egypt & UPPER EGYPT & 80-84 & 198.20 & 190.74 & 205.86 & RW2 \\ 
  Egypt & UPPER EGYPT & 85-89 & 127.49 & 132.51 & 122.64 & HT-Direct \\ 
  Egypt & UPPER EGYPT & 85-89 & 136.80 & 131.86 & 141.74 & RW2 \\ 
  Egypt & UPPER EGYPT & 90-94 & 103.65 & 108.25 & 99.23 & HT-Direct \\ 
  Egypt & UPPER EGYPT & 90-94 & 102.00 & 98.14 & 106.06 & RW2 \\ 
  Egypt & UPPER EGYPT & 95-99 & 71.69 & 75.77 & 67.82 & HT-Direct \\ 
  Egypt & UPPER EGYPT & 95-99 & 74.22 & 70.57 & 78.09 & RW2 \\ 
  Egypt & UPPER EGYPT & 00-04 & 48.99 & 52.85 & 45.40 & HT-Direct \\ 
  Egypt & UPPER EGYPT & 00-04 & 52.71 & 49.20 & 56.46 & RW2 \\ 
  Egypt & UPPER EGYPT & 05-09 & 39.49 & 44.37 & 35.14 & HT-Direct \\ 
  Egypt & UPPER EGYPT & 05-09 & 41.13 & 36.90 & 45.82 & RW2 \\ 
  Egypt & UPPER EGYPT & 10-14 & 33.01 & 13.79 & 77.36 & RW2 \\ 
  Egypt & UPPER EGYPT & 15-19 & 26.46 & 2.81 & 208.44 & RW2 \\ 
  Egypt & URBAN GOVERNORATES & 80-84 & 86.39 & 94.19 & 79.19 & HT-Direct \\ 
  Egypt & URBAN GOVERNORATES & 80-84 & 90.03 & 83.04 & 97.51 & RW2 \\ 
  Egypt & URBAN GOVERNORATES & 85-89 & 66.50 & 72.91 & 60.62 & HT-Direct \\ 
  Egypt & URBAN GOVERNORATES & 85-89 & 65.27 & 61.02 & 69.98 & RW2 \\ 
  Egypt & URBAN GOVERNORATES & 90-94 & 46.74 & 52.60 & 41.50 & HT-Direct \\ 
  Egypt & URBAN GOVERNORATES & 90-94 & 50.40 & 46.41 & 54.53 & RW2 \\ 
  Egypt & URBAN GOVERNORATES & 95-99 & 37.91 & 44.53 & 32.24 & HT-Direct \\ 
  Egypt & URBAN GOVERNORATES & 95-99 & 39.56 & 35.63 & 43.72 & RW2 \\ 
  Egypt & URBAN GOVERNORATES & 00-04 & 29.24 & 35.49 & 24.07 & HT-Direct \\ 
  Egypt & URBAN GOVERNORATES & 00-04 & 31.84 & 27.90 & 36.38 & RW2 \\ 
  Egypt & URBAN GOVERNORATES & 05-09 & 30.64 & 40.19 & 23.30 & HT-Direct \\ 
  Egypt & URBAN GOVERNORATES & 05-09 & 28.99 & 23.64 & 35.86 & RW2 \\ 
  Egypt & URBAN GOVERNORATES & 10-14 & 27.49 & 10.98 & 67.89 & RW2 \\ 
  Egypt & URBAN GOVERNORATES & 15-19 & 26.12 & 2.75 & 213.27 & RW2 \\ 
  Ethiopia & ADDIS ABABA & 80-84 & 90.19 & 117.00 & 69.04 & HT-Direct \\ 
  Ethiopia & ADDIS ABABA & 80-84 & 89.89 & 72.59 & 110.43 & RW2 \\ 
  Ethiopia & ADDIS ABABA & 85-89 & 85.55 & 107.51 & 67.73 & HT-Direct \\ 
  Ethiopia & ADDIS ABABA & 85-89 & 89.29 & 77.43 & 102.73 & RW2 \\ 
  Ethiopia & ADDIS ABABA & 90-94 & 85.66 & 105.45 & 69.30 & HT-Direct \\ 
  Ethiopia & ADDIS ABABA & 90-94 & 90.05 & 79.75 & 101.70 & RW2 \\ 
  Ethiopia & ADDIS ABABA & 95-99 & 93.37 & 111.53 & 77.90 & HT-Direct \\ 
  Ethiopia & ADDIS ABABA & 95-99 & 79.49 & 70.09 & 90.62 & RW2 \\ 
  Ethiopia & ADDIS ABABA & 00-04 & 52.84 & 72.01 & 38.56 & HT-Direct \\ 
  Ethiopia & ADDIS ABABA & 00-04 & 64.17 & 54.90 & 74.96 & RW2 \\ 
  Ethiopia & ADDIS ABABA & 05-09 & 46.19 & 65.19 & 32.53 & HT-Direct \\ 
  Ethiopia & ADDIS ABABA & 05-09 & 47.08 & 37.85 & 58.23 & RW2 \\ 
  Ethiopia & ADDIS ABABA & 10-14 & 35.14 & 63.26 & 19.27 & HT-Direct \\ 
  Ethiopia & ADDIS ABABA & 10-14 & 34.81 & 24.26 & 48.59 & RW2 \\ 
  Ethiopia & ADDIS ABABA & 15-19 & 25.95 & 9.91 & 64.50 & RW2 \\ 
  Ethiopia & AFFAR & 80-84 & 328.98 & 380.06 & 281.64 & HT-Direct \\ 
  Ethiopia & AFFAR & 80-84 & 318.85 & 284.16 & 357.80 & RW2 \\ 
  Ethiopia & AFFAR & 85-89 & 292.96 & 327.95 & 260.26 & HT-Direct \\ 
  Ethiopia & AFFAR & 85-89 & 275.80 & 254.91 & 297.97 & RW2 \\ 
  Ethiopia & AFFAR & 90-94 & 240.69 & 271.11 & 212.68 & HT-Direct \\ 
  Ethiopia & AFFAR & 90-94 & 240.11 & 223.02 & 257.30 & RW2 \\ 
  Ethiopia & AFFAR & 95-99 & 179.55 & 202.24 & 158.90 & HT-Direct \\ 
  Ethiopia & AFFAR & 95-99 & 193.45 & 178.22 & 208.24 & RW2 \\ 
  Ethiopia & AFFAR & 00-04 & 154.54 & 178.86 & 132.99 & HT-Direct \\ 
  Ethiopia & AFFAR & 00-04 & 155.07 & 142.26 & 168.21 & RW2 \\ 
  Ethiopia & AFFAR & 05-09 & 122.21 & 141.98 & 104.86 & HT-Direct \\ 
  Ethiopia & AFFAR & 05-09 & 117.85 & 106.35 & 130.61 & RW2 \\ 
  Ethiopia & AFFAR & 10-14 & 114.45 & 145.45 & 89.36 & HT-Direct \\ 
  Ethiopia & AFFAR & 10-14 & 89.92 & 76.67 & 106.47 & RW2 \\ 
  Ethiopia & AFFAR & 15-19 & 68.89 & 31.38 & 146.58 & RW2 \\ 
  Ethiopia & ALL & 80-84 & 254.72 & 235.08 & 275.78 & IHME \\ 
  Ethiopia & ALL & 80-84 & 235.97 & 222.85 & 249.61 & RW2 \\ 
  Ethiopia & ALL & 80-84 & 235.81 & 227.46 & 244.60 & UN \\ 
  Ethiopia & ALL & 85-89 & 211.56 & 208.67 & 214.26 & IHME \\ 
  Ethiopia & ALL & 85-89 & 215.41 & 206.20 & 224.88 & RW2 \\ 
  Ethiopia & ALL & 85-89 & 215.75 & 208.65 & 222.99 & UN \\ 
  Ethiopia & ALL & 90-94 & 191.46 & 188.83 & 193.83 & IHME \\ 
  Ethiopia & ALL & 90-94 & 194.59 & 187.18 & 202.25 & RW2 \\ 
  Ethiopia & ALL & 90-94 & 194.17 & 188.18 & 201.32 & UN \\ 
  Ethiopia & ALL & 95-99 & 162.49 & 160.11 & 164.87 & IHME \\ 
  Ethiopia & ALL & 95-99 & 161.77 & 155.47 & 168.22 & RW2 \\ 
  Ethiopia & ALL & 95-99 & 162.00 & 156.75 & 167.62 & UN \\ 
  Ethiopia & ALL & 00-04 & 129.27 & 127.17 & 131.38 & IHME \\ 
  Ethiopia & ALL & 00-04 & 131.82 & 125.31 & 138.67 & RW2 \\ 
  Ethiopia & ALL & 00-04 & 131.72 & 126.68 & 136.42 & UN \\ 
  Ethiopia & ALL & 05-09 & 97.86 & 95.47 & 100.57 & IHME \\ 
  Ethiopia & ALL & 05-09 & 94.28 & 88.06 & 100.87 & RW2 \\ 
  Ethiopia & ALL & 05-09 & 94.39 & 89.20 & 99.54 & UN \\ 
  Ethiopia & ALL & 10-14 & 72.87 & 69.72 & 75.85 & IHME \\ 
  Ethiopia & ALL & 10-14 & 67.46 & 60.49 & 75.11 & RW2 \\ 
  Ethiopia & ALL & 10-14 & 67.36 & 60.19 & 75.41 & UN \\ 
  Ethiopia & AMHARA & 80-84 & 236.38 & 261.23 & 213.21 & HT-Direct \\ 
  Ethiopia & AMHARA & 80-84 & 234.19 & 215.15 & 254.36 & RW2 \\ 
  Ethiopia & AMHARA & 85-89 & 216.11 & 235.03 & 198.31 & HT-Direct \\ 
  Ethiopia & AMHARA & 85-89 & 212.94 & 200.38 & 225.65 & RW2 \\ 
  Ethiopia & AMHARA & 90-94 & 196.04 & 210.96 & 181.93 & HT-Direct \\ 
  Ethiopia & AMHARA & 90-94 & 196.03 & 186.01 & 206.13 & RW2 \\ 
  Ethiopia & AMHARA & 95-99 & 163.13 & 175.63 & 151.36 & HT-Direct \\ 
  Ethiopia & AMHARA & 95-99 & 166.19 & 157.47 & 175.42 & RW2 \\ 
  Ethiopia & AMHARA & 00-04 & 151.21 & 167.39 & 136.35 & HT-Direct \\ 
  Ethiopia & AMHARA & 00-04 & 136.62 & 127.32 & 147.04 & RW2 \\ 
  Ethiopia & AMHARA & 05-09 & 103.66 & 122.62 & 87.34 & HT-Direct \\ 
  Ethiopia & AMHARA & 05-09 & 101.01 & 91.17 & 111.79 & RW2 \\ 
  Ethiopia & AMHARA & 10-14 & 66.28 & 87.36 & 50.01 & HT-Direct \\ 
  Ethiopia & AMHARA & 10-14 & 72.58 & 60.36 & 85.57 & RW2 \\ 
  Ethiopia & AMHARA & 15-19 & 51.66 & 22.79 & 112.68 & RW2 \\ 
  Ethiopia & BENISHANGUL-GUMUZ & 80-84 & 273.63 & 323.95 & 228.49 & HT-Direct \\ 
  Ethiopia & BENISHANGUL-GUMUZ & 80-84 & 270.40 & 236.35 & 307.28 & RW2 \\ 
  Ethiopia & BENISHANGUL-GUMUZ & 85-89 & 255.95 & 291.44 & 223.42 & HT-Direct \\ 
  Ethiopia & BENISHANGUL-GUMUZ & 85-89 & 251.52 & 230.69 & 273.31 & RW2 \\ 
  Ethiopia & BENISHANGUL-GUMUZ & 90-94 & 240.45 & 265.79 & 216.81 & HT-Direct \\ 
  Ethiopia & BENISHANGUL-GUMUZ & 90-94 & 237.28 & 220.99 & 253.73 & RW2 \\ 
  Ethiopia & BENISHANGUL-GUMUZ & 95-99 & 194.10 & 219.21 & 171.24 & HT-Direct \\ 
  Ethiopia & BENISHANGUL-GUMUZ & 95-99 & 206.17 & 191.52 & 221.93 & RW2 \\ 
  Ethiopia & BENISHANGUL-GUMUZ & 00-04 & 190.56 & 214.51 & 168.70 & HT-Direct \\ 
  Ethiopia & BENISHANGUL-GUMUZ & 00-04 & 172.43 & 158.29 & 188.83 & RW2 \\ 
  Ethiopia & BENISHANGUL-GUMUZ & 05-09 & 135.41 & 165.59 & 110.01 & HT-Direct \\ 
  Ethiopia & BENISHANGUL-GUMUZ & 05-09 & 128.00 & 114.08 & 143.37 & RW2 \\ 
  Ethiopia & BENISHANGUL-GUMUZ & 10-14 & 90.34 & 116.33 & 69.70 & HT-Direct \\ 
  Ethiopia & BENISHANGUL-GUMUZ & 10-14 & 90.93 & 74.90 & 108.51 & RW2 \\ 
  Ethiopia & BENISHANGUL-GUMUZ & 15-19 & 63.66 & 27.95 & 137.99 & RW2 \\ 
  Ethiopia & DIRE DAWA & 80-84 & 204.66 & 256.04 & 161.36 & HT-Direct \\ 
  Ethiopia & DIRE DAWA & 80-84 & 223.81 & 189.76 & 260.90 & RW2 \\ 
  Ethiopia & DIRE DAWA & 85-89 & 219.10 & 252.79 & 188.77 & HT-Direct \\ 
  Ethiopia & DIRE DAWA & 85-89 & 210.13 & 189.76 & 231.98 & RW2 \\ 
  Ethiopia & DIRE DAWA & 90-94 & 203.24 & 233.63 & 175.89 & HT-Direct \\ 
  Ethiopia & DIRE DAWA & 90-94 & 197.41 & 181.20 & 215.11 & RW2 \\ 
  Ethiopia & DIRE DAWA & 95-99 & 169.82 & 197.15 & 145.59 & HT-Direct \\ 
  Ethiopia & DIRE DAWA & 95-99 & 165.01 & 151.11 & 180.46 & RW2 \\ 
  Ethiopia & DIRE DAWA & 00-04 & 129.26 & 150.21 & 110.85 & HT-Direct \\ 
  Ethiopia & DIRE DAWA & 00-04 & 131.23 & 119.30 & 144.20 & RW2 \\ 
  Ethiopia & DIRE DAWA & 05-09 & 96.29 & 115.74 & 79.82 & HT-Direct \\ 
  Ethiopia & DIRE DAWA & 05-09 & 96.36 & 85.68 & 108.33 & RW2 \\ 
  Ethiopia & DIRE DAWA & 10-14 & 81.13 & 105.84 & 61.79 & HT-Direct \\ 
  Ethiopia & DIRE DAWA & 10-14 & 71.09 & 59.05 & 85.10 & RW2 \\ 
  Ethiopia & DIRE DAWA & 15-19 & 52.69 & 23.29 & 115.28 & RW2 \\ 
  Ethiopia & GAMBELA & 80-84 & 265.93 & 312.11 & 224.35 & HT-Direct \\ 
  Ethiopia & GAMBELA & 80-84 & 273.14 & 238.32 & 309.68 & RW2 \\ 
  Ethiopia & GAMBELA & 85-89 & 245.68 & 285.64 & 209.67 & HT-Direct \\ 
  Ethiopia & GAMBELA & 85-89 & 254.73 & 232.31 & 278.42 & RW2 \\ 
  Ethiopia & GAMBELA & 90-94 & 261.14 & 294.62 & 230.23 & HT-Direct \\ 
  Ethiopia & GAMBELA & 90-94 & 235.05 & 217.17 & 254.78 & RW2 \\ 
  Ethiopia & GAMBELA & 95-99 & 193.44 & 219.53 & 169.77 & HT-Direct \\ 
  Ethiopia & GAMBELA & 95-99 & 189.37 & 174.51 & 205.72 & RW2 \\ 
  Ethiopia & GAMBELA & 00-04 & 126.24 & 149.32 & 106.29 & HT-Direct \\ 
  Ethiopia & GAMBELA & 00-04 & 143.74 & 130.05 & 158.09 & RW2 \\ 
  Ethiopia & GAMBELA & 05-09 & 106.75 & 130.40 & 86.96 & HT-Direct \\ 
  Ethiopia & GAMBELA & 05-09 & 102.65 & 90.83 & 115.57 & RW2 \\ 
  Ethiopia & GAMBELA & 10-14 & 87.32 & 111.89 & 67.73 & HT-Direct \\ 
  Ethiopia & GAMBELA & 10-14 & 74.36 & 62.20 & 88.78 & RW2 \\ 
  Ethiopia & GAMBELA & 15-19 & 54.28 & 24.06 & 118.42 & RW2 \\ 
  Ethiopia & HARARI & 80-84 & 265.94 & 335.32 & 206.46 & HT-Direct \\ 
  Ethiopia & HARARI & 80-84 & 249.58 & 208.16 & 295.88 & RW2 \\ 
  Ethiopia & HARARI & 85-89 & 194.98 & 228.33 & 165.45 & HT-Direct \\ 
  Ethiopia & HARARI & 85-89 & 219.81 & 197.45 & 243.88 & RW2 \\ 
  Ethiopia & HARARI & 90-94 & 222.73 & 251.20 & 196.63 & HT-Direct \\ 
  Ethiopia & HARARI & 90-94 & 195.38 & 179.24 & 213.19 & RW2 \\ 
  Ethiopia & HARARI & 95-99 & 151.24 & 175.09 & 130.13 & HT-Direct \\ 
  Ethiopia & HARARI & 95-99 & 150.92 & 137.31 & 165.39 & RW2 \\ 
  Ethiopia & HARARI & 00-04 & 98.46 & 117.28 & 82.38 & HT-Direct \\ 
  Ethiopia & HARARI & 00-04 & 111.37 & 98.88 & 124.19 & RW2 \\ 
  Ethiopia & HARARI & 05-09 & 77.26 & 96.34 & 61.71 & HT-Direct \\ 
  Ethiopia & HARARI & 05-09 & 79.87 & 69.60 & 91.36 & RW2 \\ 
  Ethiopia & HARARI & 10-14 & 78.87 & 105.46 & 58.54 & HT-Direct \\ 
  Ethiopia & HARARI & 10-14 & 60.24 & 48.71 & 75.24 & RW2 \\ 
  Ethiopia & HARARI & 15-19 & 46.20 & 19.78 & 106.68 & RW2 \\ 
  Ethiopia & OROMIYA & 80-84 & 244.25 & 268.72 & 221.34 & HT-Direct \\ 
  Ethiopia & OROMIYA & 80-84 & 235.24 & 217.63 & 254.13 & RW2 \\ 
  Ethiopia & OROMIYA & 85-89 & 204.02 & 220.58 & 188.40 & HT-Direct \\ 
  Ethiopia & OROMIYA & 85-89 & 213.14 & 201.51 & 225.33 & RW2 \\ 
  Ethiopia & OROMIYA & 90-94 & 207.72 & 224.96 & 191.48 & HT-Direct \\ 
  Ethiopia & OROMIYA & 90-94 & 195.54 & 185.93 & 205.55 & RW2 \\ 
  Ethiopia & OROMIYA & 95-99 & 158.25 & 170.20 & 146.99 & HT-Direct \\ 
  Ethiopia & OROMIYA & 95-99 & 160.71 & 152.72 & 169.09 & RW2 \\ 
  Ethiopia & OROMIYA & 00-04 & 127.96 & 140.86 & 116.08 & HT-Direct \\ 
  Ethiopia & OROMIYA & 00-04 & 126.69 & 119.10 & 134.74 & RW2 \\ 
  Ethiopia & OROMIYA & 05-09 & 93.37 & 105.08 & 82.85 & HT-Direct \\ 
  Ethiopia & OROMIYA & 05-09 & 92.57 & 85.31 & 100.29 & RW2 \\ 
  Ethiopia & OROMIYA & 10-14 & 78.51 & 96.58 & 63.58 & HT-Direct \\ 
  Ethiopia & OROMIYA & 10-14 & 67.84 & 59.62 & 77.01 & RW2 \\ 
  Ethiopia & OROMIYA & 15-19 & 50.09 & 22.95 & 105.51 & RW2 \\ 
  Ethiopia & SNNP & 80-84 & 237.15 & 270.34 & 206.88 & HT-Direct \\ 
  Ethiopia & SNNP & 80-84 & 245.36 & 219.90 & 271.79 & RW2 \\ 
  Ethiopia & SNNP & 85-89 & 237.31 & 259.20 & 216.72 & HT-Direct \\ 
  Ethiopia & SNNP & 85-89 & 229.81 & 215.19 & 245.26 & RW2 \\ 
  Ethiopia & SNNP & 90-94 & 218.67 & 235.95 & 202.33 & HT-Direct \\ 
  Ethiopia & SNNP & 90-94 & 212.86 & 201.59 & 224.88 & RW2 \\ 
  Ethiopia & SNNP & 95-99 & 173.44 & 187.58 & 160.15 & HT-Direct \\ 
  Ethiopia & SNNP & 95-99 & 174.54 & 164.85 & 184.81 & RW2 \\ 
  Ethiopia & SNNP & 00-04 & 135.26 & 148.19 & 123.30 & HT-Direct \\ 
  Ethiopia & SNNP & 00-04 & 136.37 & 127.38 & 145.70 & RW2 \\ 
  Ethiopia & SNNP & 05-09 & 102.83 & 118.89 & 88.72 & HT-Direct \\ 
  Ethiopia & SNNP & 05-09 & 99.72 & 90.51 & 109.75 & RW2 \\ 
  Ethiopia & SNNP & 10-14 & 83.34 & 106.93 & 64.59 & HT-Direct \\ 
  Ethiopia & SNNP & 10-14 & 73.60 & 62.29 & 87.04 & RW2 \\ 
  Ethiopia & SNNP & 15-19 & 54.61 & 24.32 & 119.06 & RW2 \\ 
  Ethiopia & SOMALI & 80-84 & 202.00 & 260.21 & 154.09 & HT-Direct \\ 
  Ethiopia & SOMALI & 80-84 & 178.96 & 148.71 & 214.11 & RW2 \\ 
  Ethiopia & SOMALI & 85-89 & 147.16 & 180.15 & 119.32 & HT-Direct \\ 
  Ethiopia & SOMALI & 85-89 & 171.42 & 151.18 & 193.44 & RW2 \\ 
  Ethiopia & SOMALI & 90-94 & 188.28 & 227.20 & 154.69 & HT-Direct \\ 
  Ethiopia & SOMALI & 90-94 & 167.32 & 151.57 & 184.22 & RW2 \\ 
  Ethiopia & SOMALI & 95-99 & 142.12 & 162.79 & 123.68 & HT-Direct \\ 
  Ethiopia & SOMALI & 95-99 & 147.61 & 135.33 & 160.61 & RW2 \\ 
  Ethiopia & SOMALI & 00-04 & 129.97 & 148.62 & 113.35 & HT-Direct \\ 
  Ethiopia & SOMALI & 00-04 & 126.73 & 116.43 & 137.83 & RW2 \\ 
  Ethiopia & SOMALI & 05-09 & 103.87 & 120.31 & 89.46 & HT-Direct \\ 
  Ethiopia & SOMALI & 05-09 & 101.56 & 92.28 & 111.81 & RW2 \\ 
  Ethiopia & SOMALI & 10-14 & 91.15 & 110.38 & 74.98 & HT-Direct \\ 
  Ethiopia & SOMALI & 10-14 & 81.86 & 71.20 & 94.02 & RW2 \\ 
  Ethiopia & SOMALI & 15-19 & 66.41 & 30.28 & 140.75 & RW2 \\ 
  Ethiopia & TIGRAY & 80-84 & 252.69 & 285.29 & 222.67 & HT-Direct \\ 
  Ethiopia & TIGRAY & 80-84 & 259.46 & 234.14 & 286.53 & RW2 \\ 
  Ethiopia & TIGRAY & 85-89 & 220.93 & 243.98 & 199.48 & HT-Direct \\ 
  Ethiopia & TIGRAY & 85-89 & 219.60 & 204.72 & 235.61 & RW2 \\ 
  Ethiopia & TIGRAY & 90-94 & 197.68 & 217.23 & 179.50 & HT-Direct \\ 
  Ethiopia & TIGRAY & 90-94 & 184.43 & 172.97 & 196.68 & RW2 \\ 
  Ethiopia & TIGRAY & 95-99 & 138.63 & 152.48 & 125.84 & HT-Direct \\ 
  Ethiopia & TIGRAY & 95-99 & 138.70 & 129.32 & 148.35 & RW2 \\ 
  Ethiopia & TIGRAY & 00-04 & 89.39 & 102.43 & 77.87 & HT-Direct \\ 
  Ethiopia & TIGRAY & 00-04 & 101.89 & 93.53 & 110.65 & RW2 \\ 
  Ethiopia & TIGRAY & 05-09 & 77.96 & 91.18 & 66.52 & HT-Direct \\ 
  Ethiopia & TIGRAY & 05-09 & 70.68 & 63.73 & 78.34 & RW2 \\ 
  Ethiopia & TIGRAY & 10-14 & 55.22 & 69.42 & 43.78 & HT-Direct \\ 
  Ethiopia & TIGRAY & 10-14 & 48.88 & 41.56 & 57.62 & RW2 \\ 
  Ethiopia & TIGRAY & 15-19 & 33.96 & 14.90 & 75.20 & RW2 \\ 
  Gabon & ALL & 80-84 & 105.56 & 102.43 & 108.52 & IHME \\ 
  Gabon & ALL & 80-84 & 110.48 & 95.37 & 127.67 & RW2 \\ 
  Gabon & ALL & 80-84 & 109.94 & 102.25 & 119.10 & UN \\ 
  Gabon & ALL & 85-89 & 89.66 & 87.53 & 91.86 & IHME \\ 
  Gabon & ALL & 85-89 & 96.12 & 84.72 & 108.69 & RW2 \\ 
  Gabon & ALL & 85-89 & 96.92 & 90.85 & 103.92 & UN \\ 
  Gabon & ALL & 90-94 & 79.90 & 78.23 & 81.80 & IHME \\ 
  Gabon & ALL & 90-94 & 92.83 & 83.69 & 102.90 & RW2 \\ 
  Gabon & ALL & 90-94 & 91.93 & 86.14 & 97.99 & UN \\ 
  Gabon & ALL & 95-99 & 73.53 & 71.86 & 75.24 & IHME \\ 
  Gabon & ALL & 95-99 & 88.84 & 78.51 & 100.29 & RW2 \\ 
  Gabon & ALL & 95-99 & 88.83 & 83.31 & 94.85 & UN \\ 
  Gabon & ALL & 00-04 & 69.99 & 68.33 & 71.76 & IHME \\ 
  Gabon & ALL & 00-04 & 82.35 & 66.09 & 102.83 & RW2 \\ 
  Gabon & ALL & 00-04 & 83.16 & 77.53 & 89.24 & UN \\ 
  Gabon & ALL & 05-09 & 63.23 & 61.13 & 64.97 & IHME \\ 
  Gabon & ALL & 05-09 & 70.22 & 57.95 & 84.82 & RW2 \\ 
  Gabon & ALL & 05-09 & 71.64 & 66.79 & 76.89 & UN \\ 
  Gabon & ALL & 10-14 & 52.38 & 49.61 & 55.02 & IHME \\ 
  Gabon & ALL & 10-14 & 58.55 & 50.24 & 68.07 & RW2 \\ 
  Gabon & ALL & 10-14 & 57.25 & 51.45 & 64.37 & UN \\ 
  Gabon & EAST & 80-84 & 71.07 & 99.51 & 50.30 & HT-Direct \\ 
  Gabon & EAST & 80-84 & 75.06 & 56.14 & 99.98 & RW2 \\ 
  Gabon & EAST & 85-89 & 59.77 & 81.58 & 43.52 & HT-Direct \\ 
  Gabon & EAST & 85-89 & 63.53 & 51.34 & 78.16 & RW2 \\ 
  Gabon & EAST & 90-94 & 53.77 & 66.88 & 43.10 & HT-Direct \\ 
  Gabon & EAST & 90-94 & 59.36 & 50.27 & 69.84 & RW2 \\ 
  Gabon & EAST & 95-99 & 62.94 & 78.44 & 50.34 & HT-Direct \\ 
  Gabon & EAST & 95-99 & 70.10 & 59.01 & 83.26 & RW2 \\ 
  Gabon & EAST & 00-04 & 60.95 & 81.52 & 45.32 & HT-Direct \\ 
  Gabon & EAST & 00-04 & 78.85 & 62.45 & 100.02 & RW2 \\ 
  Gabon & EAST & 05-09 & 53.92 & 75.62 & 38.19 & HT-Direct \\ 
  Gabon & EAST & 05-09 & 69.63 & 54.29 & 89.38 & RW2 \\ 
  Gabon & EAST & 10-14 & 77.36 & 118.42 & 49.74 & HT-Direct \\ 
  Gabon & EAST & 10-14 & 56.87 & 42.26 & 76.01 & RW2 \\ 
  Gabon & EAST & 15-19 & 45.98 & 16.17 & 124.15 & RW2 \\ 
  Gabon & LIBREVILLE,PORT-GENTIL & 80-84 & 98.38 & 125.53 & 76.59 & HT-Direct \\ 
  Gabon & LIBREVILLE,PORT-GENTIL & 80-84 & 113.60 & 88.89 & 142.62 & RW2 \\ 
  Gabon & LIBREVILLE,PORT-GENTIL & 85-89 & 83.11 & 100.47 & 68.52 & HT-Direct \\ 
  Gabon & LIBREVILLE,PORT-GENTIL & 85-89 & 101.20 & 86.08 & 118.69 & RW2 \\ 
  Gabon & LIBREVILLE,PORT-GENTIL & 90-94 & 104.24 & 122.98 & 88.07 & HT-Direct \\ 
  Gabon & LIBREVILLE,PORT-GENTIL & 90-94 & 92.58 & 80.02 & 107.94 & RW2 \\ 
  Gabon & LIBREVILLE,PORT-GENTIL & 95-99 & 73.79 & 91.46 & 59.31 & HT-Direct \\ 
  Gabon & LIBREVILLE,PORT-GENTIL & 95-99 & 93.51 & 78.65 & 110.96 & RW2 \\ 
  Gabon & LIBREVILLE,PORT-GENTIL & 00-04 & 54.34 & 79.61 & 36.78 & HT-Direct \\ 
  Gabon & LIBREVILLE,PORT-GENTIL & 00-04 & 88.14 & 66.74 & 113.27 & RW2 \\ 
  Gabon & LIBREVILLE,PORT-GENTIL & 05-09 & 47.17 & 63.14 & 35.10 & HT-Direct \\ 
  Gabon & LIBREVILLE,PORT-GENTIL & 05-09 & 69.07 & 52.97 & 87.86 & RW2 \\ 
  Gabon & LIBREVILLE,PORT-GENTIL & 10-14 & 75.53 & 107.89 & 52.30 & HT-Direct \\ 
  Gabon & LIBREVILLE,PORT-GENTIL & 10-14 & 53.11 & 40.84 & 68.68 & RW2 \\ 
  Gabon & LIBREVILLE,PORT-GENTIL & 15-19 & 41.09 & 14.67 & 113.80 & RW2 \\ 
  Gabon & NORTH & 80-84 & 125.59 & 166.00 & 93.90 & HT-Direct \\ 
  Gabon & NORTH & 80-84 & 136.81 & 107.53 & 172.54 & RW2 \\ 
  Gabon & NORTH & 85-89 & 101.25 & 123.63 & 82.54 & HT-Direct \\ 
  Gabon & NORTH & 85-89 & 110.70 & 93.78 & 130.39 & RW2 \\ 
  Gabon & NORTH & 90-94 & 89.93 & 108.09 & 74.56 & HT-Direct \\ 
  Gabon & NORTH & 90-94 & 97.69 & 84.75 & 112.04 & RW2 \\ 
  Gabon & NORTH & 95-99 & 97.87 & 117.82 & 80.98 & HT-Direct \\ 
  Gabon & NORTH & 95-99 & 107.45 & 92.36 & 125.18 & RW2 \\ 
  Gabon & NORTH & 00-04 & 90.92 & 123.90 & 66.05 & HT-Direct \\ 
  Gabon & NORTH & 00-04 & 112.83 & 91.41 & 140.24 & RW2 \\ 
  Gabon & NORTH & 05-09 & 75.44 & 90.90 & 62.44 & HT-Direct \\ 
  Gabon & NORTH & 05-09 & 94.59 & 78.57 & 113.70 & RW2 \\ 
  Gabon & NORTH & 10-14 & 100.25 & 126.73 & 78.81 & HT-Direct \\ 
  Gabon & NORTH & 10-14 & 73.60 & 61.84 & 87.44 & RW2 \\ 
  Gabon & NORTH & 15-19 & 56.53 & 21.68 & 141.29 & RW2 \\ 
  Gabon & SOUTH & 80-84 & 120.68 & 156.22 & 92.35 & HT-Direct \\ 
  Gabon & SOUTH & 80-84 & 125.02 & 98.72 & 156.98 & RW2 \\ 
  Gabon & SOUTH & 85-89 & 75.71 & 97.39 & 58.54 & HT-Direct \\ 
  Gabon & SOUTH & 85-89 & 98.76 & 82.51 & 117.67 & RW2 \\ 
  Gabon & SOUTH & 90-94 & 86.45 & 103.33 & 72.11 & HT-Direct \\ 
  Gabon & SOUTH & 90-94 & 85.95 & 74.62 & 98.70 & RW2 \\ 
  Gabon & SOUTH & 95-99 & 78.64 & 95.31 & 64.68 & HT-Direct \\ 
  Gabon & SOUTH & 95-99 & 92.22 & 79.03 & 108.19 & RW2 \\ 
  Gabon & SOUTH & 00-04 & 82.52 & 107.82 & 62.74 & HT-Direct \\ 
  Gabon & SOUTH & 00-04 & 93.57 & 74.92 & 117.84 & RW2 \\ 
  Gabon & SOUTH & 05-09 & 66.64 & 89.87 & 49.10 & HT-Direct \\ 
  Gabon & SOUTH & 05-09 & 74.15 & 59.11 & 93.09 & RW2 \\ 
  Gabon & SOUTH & 10-14 & 68.52 & 96.32 & 48.31 & HT-Direct \\ 
  Gabon & SOUTH & 10-14 & 53.66 & 41.87 & 68.19 & RW2 \\ 
  Gabon & SOUTH & 15-19 & 38.09 & 13.77 & 101.42 & RW2 \\ 
  Gabon & WEST & 80-84 & 154.47 & 195.61 & 120.68 & HT-Direct \\ 
  Gabon & WEST & 80-84 & 142.34 & 114.10 & 179.28 & RW2 \\ 
  Gabon & WEST & 85-89 & 97.53 & 118.17 & 80.17 & HT-Direct \\ 
  Gabon & WEST & 85-89 & 106.86 & 90.89 & 125.18 & RW2 \\ 
  Gabon & WEST & 90-94 & 67.05 & 83.30 & 53.79 & HT-Direct \\ 
  Gabon & WEST & 90-94 & 88.55 & 75.56 & 102.65 & RW2 \\ 
  Gabon & WEST & 95-99 & 88.25 & 106.38 & 72.95 & HT-Direct \\ 
  Gabon & WEST & 95-99 & 93.22 & 78.36 & 109.05 & RW2 \\ 
  Gabon & WEST & 00-04 & 64.81 & 106.86 & 38.59 & HT-Direct \\ 
  Gabon & WEST & 00-04 & 96.37 & 75.87 & 121.26 & RW2 \\ 
  Gabon & WEST & 05-09 & 57.56 & 76.43 & 43.13 & HT-Direct \\ 
  Gabon & WEST & 05-09 & 82.71 & 66.57 & 101.99 & RW2 \\ 
  Gabon & WEST & 10-14 & 101.09 & 132.91 & 76.21 & HT-Direct \\ 
  Gabon & WEST & 10-14 & 68.30 & 55.55 & 84.13 & RW2 \\ 
  Gabon & WEST & 15-19 & 56.61 & 21.55 & 143.38 & RW2 \\ 
  Gambia & ALL & 80-84 & 135.82 & 131.90 & 139.80 & IHME \\ 
  Gambia & ALL & 80-84 & 231.78 & 145.93 & 347.78 & RW2 \\ 
  Gambia & ALL & 80-84 & 228.47 & 214.24 & 242.39 & UN \\ 
  Gambia & ALL & 85-89 & 120.48 & 117.40 & 123.37 & IHME \\ 
  Gambia & ALL & 85-89 & 190.37 & 137.48 & 257.00 & RW2 \\ 
  Gambia & ALL & 85-89 & 191.14 & 180.53 & 202.28 & UN \\ 
  Gambia & ALL & 90-94 & 108.04 & 105.88 & 110.37 & IHME \\ 
  Gambia & ALL & 90-94 & 158.19 & 120.92 & 204.28 & RW2 \\ 
  Gambia & ALL & 90-94 & 158.24 & 150.73 & 166.37 & UN \\ 
  Gambia & ALL & 95-99 & 94.18 & 92.22 & 96.12 & IHME \\ 
  Gambia & ALL & 95-99 & 132.33 & 106.91 & 162.34 & RW2 \\ 
  Gambia & ALL & 95-99 & 132.66 & 126.03 & 139.90 & UN \\ 
  Gambia & ALL & 00-04 & 79.35 & 77.59 & 81.14 & IHME \\ 
  Gambia & ALL & 00-04 & 111.24 & 93.35 & 132.28 & RW2 \\ 
  Gambia & ALL & 00-04 & 110.82 & 104.31 & 117.45 & UN \\ 
  Gambia & ALL & 05-09 & 65.01 & 63.14 & 66.77 & IHME \\ 
  Gambia & ALL & 05-09 & 91.10 & 75.52 & 109.44 & RW2 \\ 
  Gambia & ALL & 05-09 & 90.98 & 83.48 & 100.36 & UN \\ 
  Gambia & ALL & 10-14 & 51.98 & 49.99 & 54.01 & IHME \\ 
  Gambia & ALL & 10-14 & 75.81 & 54.80 & 103.78 & RW2 \\ 
  Gambia & ALL & 10-14 & 76.27 & 66.30 & 88.83 & UN \\ 
  Gambia & BANJUL & 80-84 & 112.43 & 300.72 & 35.97 & HT-Direct \\ 
  Gambia & BANJUL & 80-84 & 128.94 & 50.54 & 297.19 & RW2 \\ 
  Gambia & BANJUL & 85-89 & 86.72 & 154.53 & 47.01 & HT-Direct \\ 
  Gambia & BANJUL & 85-89 & 110.04 & 55.28 & 206.32 & RW2 \\ 
  Gambia & BANJUL & 90-94 & 81.83 & 130.12 & 50.42 & HT-Direct \\ 
  Gambia & BANJUL & 90-94 & 97.03 & 58.20 & 157.85 & RW2 \\ 
  Gambia & BANJUL & 95-99 & 49.26 & 71.85 & 33.51 & HT-Direct \\ 
  Gambia & BANJUL & 95-99 & 88.03 & 59.66 & 127.26 & RW2 \\ 
  Gambia & BANJUL & 00-04 & 53.64 & 80.84 & 35.24 & HT-Direct \\ 
  Gambia & BANJUL & 00-04 & 82.55 & 59.45 & 112.99 & RW2 \\ 
  Gambia & BANJUL & 05-09 & 56.07 & 75.97 & 41.15 & HT-Direct \\ 
  Gambia & BANJUL & 05-09 & 75.02 & 53.23 & 105.09 & RW2 \\ 
  Gambia & BANJUL & 10-14 & 50.91 & 70.81 & 36.39 & HT-Direct \\ 
  Gambia & BANJUL & 10-14 & 68.14 & 40.94 & 112.05 & RW2 \\ 
  Gambia & BANJUL & 15-19 & 62.04 & 16.83 & 206.48 & RW2 \\ 
  Gambia & CENTRAL RIVER & 80-84 & 145.98 & 299.41 & 63.99 & HT-Direct \\ 
  Gambia & CENTRAL RIVER & 80-84 & 223.76 & 109.53 & 403.47 & RW2 \\ 
  Gambia & CENTRAL RIVER & 85-89 & 111.39 & 194.41 & 61.13 & HT-Direct \\ 
  Gambia & CENTRAL RIVER & 85-89 & 182.43 & 105.68 & 296.38 & RW2 \\ 
  Gambia & CENTRAL RIVER & 90-94 & 84.32 & 120.50 & 58.28 & HT-Direct \\ 
  Gambia & CENTRAL RIVER & 90-94 & 152.46 & 101.06 & 222.95 & RW2 \\ 
  Gambia & CENTRAL RIVER & 95-99 & 107.74 & 145.43 & 78.92 & HT-Direct \\ 
  Gambia & CENTRAL RIVER & 95-99 & 127.64 & 94.47 & 170.17 & RW2 \\ 
  Gambia & CENTRAL RIVER & 00-04 & 78.10 & 99.95 & 60.70 & HT-Direct \\ 
  Gambia & CENTRAL RIVER & 00-04 & 106.88 & 83.35 & 136.17 & RW2 \\ 
  Gambia & CENTRAL RIVER & 05-09 & 57.94 & 77.17 & 43.27 & HT-Direct \\ 
  Gambia & CENTRAL RIVER & 05-09 & 83.95 & 61.95 & 112.82 & RW2 \\ 
  Gambia & CENTRAL RIVER & 10-14 & 41.43 & 57.43 & 29.75 & HT-Direct \\ 
  Gambia & CENTRAL RIVER & 10-14 & 64.01 & 39.87 & 100.87 & RW2 \\ 
  Gambia & CENTRAL RIVER & 15-19 & 48.15 & 13.62 & 156.26 & RW2 \\ 
  Gambia & LOWER RIVER & 80-84 & 380.22 & 540.49 & 242.40 & HT-Direct \\ 
  Gambia & LOWER RIVER & 80-84 & 390.18 & 225.43 & 584.05 & RW2 \\ 
  Gambia & LOWER RIVER & 85-89 & 167.40 & 247.12 & 109.66 & HT-Direct \\ 
  Gambia & LOWER RIVER & 85-89 & 294.69 & 187.72 & 431.66 & RW2 \\ 
  Gambia & LOWER RIVER & 90-94 & 141.20 & 190.88 & 102.81 & HT-Direct \\ 
  Gambia & LOWER RIVER & 90-94 & 221.75 & 151.52 & 312.25 & RW2 \\ 
  Gambia & LOWER RIVER & 95-99 & 120.08 & 176.79 & 79.80 & HT-Direct \\ 
  Gambia & LOWER RIVER & 95-99 & 164.80 & 119.71 & 223.19 & RW2 \\ 
  Gambia & LOWER RIVER & 00-04 & 83.50 & 129.15 & 53.01 & HT-Direct \\ 
  Gambia & LOWER RIVER & 00-04 & 121.84 & 90.10 & 163.94 & RW2 \\ 
  Gambia & LOWER RIVER & 05-09 & 72.16 & 96.98 & 53.32 & HT-Direct \\ 
  Gambia & LOWER RIVER & 05-09 & 84.36 & 58.61 & 120.19 & RW2 \\ 
  Gambia & LOWER RIVER & 10-14 & 34.17 & 55.32 & 20.93 & HT-Direct \\ 
  Gambia & LOWER RIVER & 10-14 & 56.39 & 32.22 & 97.25 & RW2 \\ 
  Gambia & LOWER RIVER & 15-19 & 37.11 & 9.73 & 131.09 & RW2 \\ 
  Gambia & NORTH BANK & 80-84 & 206.13 & 350.18 & 111.20 & HT-Direct \\ 
  Gambia & NORTH BANK & 80-84 & 279.84 & 156.63 & 449.03 & RW2 \\ 
  Gambia & NORTH BANK & 85-89 & 148.72 & 201.16 & 108.10 & HT-Direct \\ 
  Gambia & NORTH BANK & 85-89 & 219.04 & 142.40 & 320.55 & RW2 \\ 
  Gambia & NORTH BANK & 90-94 & 101.70 & 136.95 & 74.75 & HT-Direct \\ 
  Gambia & NORTH BANK & 90-94 & 174.28 & 123.85 & 240.21 & RW2 \\ 
  Gambia & NORTH BANK & 95-99 & 96.34 & 125.47 & 73.40 & HT-Direct \\ 
  Gambia & NORTH BANK & 95-99 & 138.42 & 104.91 & 180.22 & RW2 \\ 
  Gambia & NORTH BANK & 00-04 & 93.08 & 119.46 & 72.04 & HT-Direct \\ 
  Gambia & NORTH BANK & 00-04 & 109.37 & 84.74 & 140.51 & RW2 \\ 
  Gambia & NORTH BANK & 05-09 & 55.23 & 75.04 & 40.43 & HT-Direct \\ 
  Gambia & NORTH BANK & 05-09 & 80.69 & 56.96 & 112.79 & RW2 \\ 
  Gambia & NORTH BANK & 10-14 & 28.97 & 51.08 & 16.27 & HT-Direct \\ 
  Gambia & NORTH BANK & 10-14 & 57.46 & 32.18 & 100.13 & RW2 \\ 
  Gambia & NORTH BANK & 15-19 & 40.74 & 10.24 & 145.43 & RW2 \\ 
  Gambia & UPPER RIVER & 80-84 & 240.89 & 356.14 & 154.02 & HT-Direct \\ 
  Gambia & UPPER RIVER & 80-84 & 334.64 & 199.47 & 502.82 & RW2 \\ 
  Gambia & UPPER RIVER & 85-89 & 166.20 & 274.60 & 94.99 & HT-Direct \\ 
  Gambia & UPPER RIVER & 85-89 & 279.53 & 182.41 & 402.97 & RW2 \\ 
  Gambia & UPPER RIVER & 90-94 & 212.51 & 265.72 & 167.53 & HT-Direct \\ 
  Gambia & UPPER RIVER & 90-94 & 237.08 & 170.44 & 318.85 & RW2 \\ 
  Gambia & UPPER RIVER & 95-99 & 131.26 & 157.74 & 108.65 & HT-Direct \\ 
  Gambia & UPPER RIVER & 95-99 & 199.11 & 155.37 & 251.94 & RW2 \\ 
  Gambia & UPPER RIVER & 00-04 & 121.49 & 164.61 & 88.48 & HT-Direct \\ 
  Gambia & UPPER RIVER & 00-04 & 166.42 & 129.07 & 213.01 & RW2 \\ 
  Gambia & UPPER RIVER & 05-09 & 117.91 & 158.34 & 86.74 & HT-Direct \\ 
  Gambia & UPPER RIVER & 05-09 & 130.30 & 95.51 & 175.26 & RW2 \\ 
  Gambia & UPPER RIVER & 10-14 & 58.91 & 76.75 & 45.01 & HT-Direct \\ 
  Gambia & UPPER RIVER & 10-14 & 97.83 & 63.05 & 148.30 & RW2 \\ 
  Gambia & UPPER RIVER & 15-19 & 72.49 & 21.34 & 216.61 & RW2 \\ 
  Gambia & WESTERN & 80-84 & 149.61 & 286.40 & 71.60 & HT-Direct \\ 
  Gambia & WESTERN & 80-84 & 198.99 & 99.28 & 362.69 & RW2 \\ 
  Gambia & WESTERN & 85-89 & 123.19 & 192.34 & 76.54 & HT-Direct \\ 
  Gambia & WESTERN & 85-89 & 163.43 & 96.63 & 263.97 & RW2 \\ 
  Gambia & WESTERN & 90-94 & 83.51 & 121.71 & 56.53 & HT-Direct \\ 
  Gambia & WESTERN & 90-94 & 138.66 & 92.07 & 203.67 & RW2 \\ 
  Gambia & WESTERN & 95-99 & 86.41 & 118.10 & 62.61 & HT-Direct \\ 
  Gambia & WESTERN & 95-99 & 119.59 & 87.74 & 160.30 & RW2 \\ 
  Gambia & WESTERN & 00-04 & 85.77 & 109.98 & 66.49 & HT-Direct \\ 
  Gambia & WESTERN & 00-04 & 105.00 & 82.19 & 133.14 & RW2 \\ 
  Gambia & WESTERN & 05-09 & 57.43 & 71.08 & 46.27 & HT-Direct \\ 
  Gambia & WESTERN & 05-09 & 88.24 & 67.94 & 113.79 & RW2 \\ 
  Gambia & WESTERN & 10-14 & 61.64 & 84.07 & 44.90 & HT-Direct \\ 
  Gambia & WESTERN & 10-14 & 73.53 & 47.19 & 113.42 & RW2 \\ 
  Gambia & WESTERN & 15-19 & 61.51 & 17.46 & 193.58 & RW2 \\ 
  Ghana & ALL & 80-84 & 148.92 & 146.86 & 151.03 & IHME \\ 
  Ghana & ALL & 80-84 & 162.85 & 154.80 & 171.24 & RW2 \\ 
  Ghana & ALL & 80-84 & 162.85 & 158.95 & 166.77 & UN \\ 
  Ghana & ALL & 85-89 & 134.21 & 132.48 & 135.87 & IHME \\ 
  Ghana & ALL & 85-89 & 143.76 & 136.58 & 151.25 & RW2 \\ 
  Ghana & ALL & 85-89 & 143.80 & 140.49 & 147.51 & UN \\ 
  Ghana & ALL & 90-94 & 116.65 & 115.07 & 118.43 & IHME \\ 
  Ghana & ALL & 90-94 & 120.47 & 113.87 & 127.37 & RW2 \\ 
  Ghana & ALL & 90-94 & 120.40 & 117.32 & 123.76 & UN \\ 
  Ghana & ALL & 95-99 & 105.14 & 103.63 & 106.60 & IHME \\ 
  Ghana & ALL & 95-99 & 108.99 & 101.93 & 116.45 & RW2 \\ 
  Ghana & ALL & 95-99 & 109.02 & 106.25 & 112.08 & UN \\ 
  Ghana & ALL & 00-04 & 93.30 & 91.80 & 94.80 & IHME \\ 
  Ghana & ALL & 00-04 & 94.18 & 87.62 & 101.19 & RW2 \\ 
  Ghana & ALL & 00-04 & 94.36 & 91.46 & 97.37 & UN \\ 
  Ghana & ALL & 05-09 & 79.49 & 77.85 & 81.08 & IHME \\ 
  Ghana & ALL & 05-09 & 82.60 & 75.16 & 90.73 & RW2 \\ 
  Ghana & ALL & 05-09 & 82.20 & 79.04 & 85.53 & UN \\ 
  Ghana & ALL & 10-14 & 64.38 & 62.12 & 66.82 & IHME \\ 
  Ghana & ALL & 10-14 & 68.64 & 58.60 & 80.17 & RW2 \\ 
  Ghana & ALL & 10-14 & 68.91 & 64.12 & 74.35 & UN \\ 
  Ghana & ASHANTI & 80-84 & 132.46 & 151.34 & 115.61 & HT-Direct \\ 
  Ghana & ASHANTI & 80-84 & 133.61 & 118.91 & 149.69 & RW2 \\ 
  Ghana & ASHANTI & 85-89 & 122.88 & 140.18 & 107.44 & HT-Direct \\ 
  Ghana & ASHANTI & 85-89 & 122.43 & 111.55 & 133.86 & RW2 \\ 
  Ghana & ASHANTI & 90-94 & 84.64 & 98.88 & 72.28 & HT-Direct \\ 
  Ghana & ASHANTI & 90-94 & 109.15 & 99.36 & 119.57 & RW2 \\ 
  Ghana & ASHANTI & 95-99 & 115.12 & 133.15 & 99.26 & HT-Direct \\ 
  Ghana & ASHANTI & 95-99 & 104.72 & 94.82 & 115.65 & RW2 \\ 
  Ghana & ASHANTI & 00-04 & 94.63 & 112.05 & 79.68 & HT-Direct \\ 
  Ghana & ASHANTI & 00-04 & 98.33 & 87.72 & 110.19 & RW2 \\ 
  Ghana & ASHANTI & 05-09 & 81.03 & 104.38 & 62.53 & HT-Direct \\ 
  Ghana & ASHANTI & 05-09 & 91.21 & 77.97 & 106.40 & RW2 \\ 
  Ghana & ASHANTI & 10-14 & 74.94 & 99.86 & 55.86 & HT-Direct \\ 
  Ghana & ASHANTI & 10-14 & 80.01 & 63.42 & 100.48 & RW2 \\ 
  Ghana & ASHANTI & 15-19 & 68.57 & 28.58 & 154.82 & RW2 \\ 
  Ghana & BRONG AHAFO & 80-84 & 132.03 & 156.16 & 111.14 & HT-Direct \\ 
  Ghana & BRONG AHAFO & 80-84 & 139.44 & 121.29 & 159.89 & RW2 \\ 
  Ghana & BRONG AHAFO & 85-89 & 122.97 & 144.36 & 104.36 & HT-Direct \\ 
  Ghana & BRONG AHAFO & 85-89 & 122.38 & 109.82 & 136.30 & RW2 \\ 
  Ghana & BRONG AHAFO & 90-94 & 90.98 & 112.15 & 73.47 & HT-Direct \\ 
  Ghana & BRONG AHAFO & 90-94 & 103.67 & 93.05 & 115.57 & RW2 \\ 
  Ghana & BRONG AHAFO & 95-99 & 90.91 & 111.74 & 73.64 & HT-Direct \\ 
  Ghana & BRONG AHAFO & 95-99 & 93.10 & 82.81 & 104.77 & RW2 \\ 
  Ghana & BRONG AHAFO & 00-04 & 84.62 & 103.83 & 68.69 & HT-Direct \\ 
  Ghana & BRONG AHAFO & 00-04 & 81.72 & 71.46 & 93.41 & RW2 \\ 
  Ghana & BRONG AHAFO & 05-09 & 65.47 & 89.43 & 47.60 & HT-Direct \\ 
  Ghana & BRONG AHAFO & 05-09 & 70.45 & 58.75 & 84.07 & RW2 \\ 
  Ghana & BRONG AHAFO & 10-14 & 44.70 & 61.67 & 32.24 & HT-Direct \\ 
  Ghana & BRONG AHAFO & 10-14 & 56.90 & 43.99 & 72.81 & RW2 \\ 
  Ghana & BRONG AHAFO & 15-19 & 44.99 & 18.46 & 103.33 & RW2 \\ 
  Ghana & CENTRAL & 80-84 & 184.65 & 211.69 & 160.37 & HT-Direct \\ 
  Ghana & CENTRAL & 80-84 & 184.82 & 164.09 & 207.91 & RW2 \\ 
  Ghana & CENTRAL & 85-89 & 154.43 & 177.15 & 134.15 & HT-Direct \\ 
  Ghana & CENTRAL & 85-89 & 155.30 & 141.18 & 170.43 & RW2 \\ 
  Ghana & CENTRAL & 90-94 & 111.73 & 130.22 & 95.57 & HT-Direct \\ 
  Ghana & CENTRAL & 90-94 & 126.57 & 114.43 & 139.52 & RW2 \\ 
  Ghana & CENTRAL & 95-99 & 102.00 & 127.72 & 80.98 & HT-Direct \\ 
  Ghana & CENTRAL & 95-99 & 110.55 & 97.96 & 124.16 & RW2 \\ 
  Ghana & CENTRAL & 00-04 & 90.56 & 111.93 & 72.94 & HT-Direct \\ 
  Ghana & CENTRAL & 00-04 & 95.88 & 83.48 & 110.00 & RW2 \\ 
  Ghana & CENTRAL & 05-09 & 89.72 & 118.57 & 67.35 & HT-Direct \\ 
  Ghana & CENTRAL & 05-09 & 83.13 & 68.75 & 100.24 & RW2 \\ 
  Ghana & CENTRAL & 10-14 & 68.17 & 110.51 & 41.30 & HT-Direct \\ 
  Ghana & CENTRAL & 10-14 & 68.04 & 50.71 & 92.22 & RW2 \\ 
  Ghana & CENTRAL & 15-19 & 54.70 & 21.51 & 131.46 & RW2 \\ 
  Ghana & EASTERN & 80-84 & 119.77 & 136.97 & 104.47 & HT-Direct \\ 
  Ghana & EASTERN & 80-84 & 124.93 & 110.83 & 140.59 & RW2 \\ 
  Ghana & EASTERN & 85-89 & 101.64 & 119.90 & 85.90 & HT-Direct \\ 
  Ghana & EASTERN & 85-89 & 114.66 & 103.77 & 126.69 & RW2 \\ 
  Ghana & EASTERN & 90-94 & 117.69 & 137.92 & 100.08 & HT-Direct \\ 
  Ghana & EASTERN & 90-94 & 101.75 & 91.87 & 112.89 & RW2 \\ 
  Ghana & EASTERN & 95-99 & 88.75 & 110.14 & 71.18 & HT-Direct \\ 
  Ghana & EASTERN & 95-99 & 94.95 & 84.27 & 106.88 & RW2 \\ 
  Ghana & EASTERN & 00-04 & 69.93 & 91.94 & 52.88 & HT-Direct \\ 
  Ghana & EASTERN & 00-04 & 86.89 & 75.37 & 99.90 & RW2 \\ 
  Ghana & EASTERN & 05-09 & 69.93 & 89.68 & 54.27 & HT-Direct \\ 
  Ghana & EASTERN & 05-09 & 79.20 & 66.45 & 94.04 & RW2 \\ 
  Ghana & EASTERN & 10-14 & 75.23 & 107.86 & 51.89 & HT-Direct \\ 
  Ghana & EASTERN & 10-14 & 68.56 & 52.76 & 88.71 & RW2 \\ 
  Ghana & EASTERN & 15-19 & 58.26 & 23.54 & 134.20 & RW2 \\ 
  Ghana & GREATER ACCRA & 80-84 & 123.51 & 145.58 & 104.37 & HT-Direct \\ 
  Ghana & GREATER ACCRA & 80-84 & 128.66 & 111.40 & 148.28 & RW2 \\ 
  Ghana & GREATER ACCRA & 85-89 & 109.33 & 131.46 & 90.54 & HT-Direct \\ 
  Ghana & GREATER ACCRA & 85-89 & 108.66 & 96.61 & 122.11 & RW2 \\ 
  Ghana & GREATER ACCRA & 90-94 & 83.73 & 104.17 & 67.00 & HT-Direct \\ 
  Ghana & GREATER ACCRA & 90-94 & 88.61 & 78.16 & 100.10 & RW2 \\ 
  Ghana & GREATER ACCRA & 95-99 & 73.28 & 94.28 & 56.67 & HT-Direct \\ 
  Ghana & GREATER ACCRA & 95-99 & 76.87 & 66.42 & 88.68 & RW2 \\ 
  Ghana & GREATER ACCRA & 00-04 & 48.57 & 67.82 & 34.58 & HT-Direct \\ 
  Ghana & GREATER ACCRA & 00-04 & 66.14 & 55.53 & 78.68 & RW2 \\ 
  Ghana & GREATER ACCRA & 05-09 & 68.10 & 93.02 & 49.50 & HT-Direct \\ 
  Ghana & GREATER ACCRA & 05-09 & 57.18 & 45.41 & 71.77 & RW2 \\ 
  Ghana & GREATER ACCRA & 10-14 & 30.81 & 62.30 & 14.98 & HT-Direct \\ 
  Ghana & GREATER ACCRA & 10-14 & 46.62 & 32.70 & 66.76 & RW2 \\ 
  Ghana & GREATER ACCRA & 15-19 & 37.24 & 14.03 & 95.01 & RW2 \\ 
  Ghana & UPPER W,E \& NORTHERN & 80-84 & 248.49 & 270.79 & 227.45 & HT-Direct \\ 
  Ghana & UPPER W,E \& NORTHERN & 80-84 & 253.95 & 234.53 & 274.30 & RW2 \\ 
  Ghana & UPPER W,E \& NORTHERN & 85-89 & 206.91 & 223.23 & 191.49 & HT-Direct \\ 
  Ghana & UPPER W,E \& NORTHERN & 85-89 & 215.39 & 202.40 & 228.85 & RW2 \\ 
  Ghana & UPPER W,E \& NORTHERN & 90-94 & 171.04 & 185.98 & 157.07 & HT-Direct \\ 
  Ghana & UPPER W,E \& NORTHERN & 90-94 & 176.59 & 165.09 & 188.64 & RW2 \\ 
  Ghana & UPPER W,E \& NORTHERN & 95-99 & 140.55 & 154.72 & 127.48 & HT-Direct \\ 
  Ghana & UPPER W,E \& NORTHERN & 95-99 & 152.88 & 141.69 & 164.63 & RW2 \\ 
  Ghana & UPPER W,E \& NORTHERN & 00-04 & 126.60 & 140.37 & 114.00 & HT-Direct \\ 
  Ghana & UPPER W,E \& NORTHERN & 00-04 & 131.91 & 121.36 & 143.22 & RW2 \\ 
  Ghana & UPPER W,E \& NORTHERN & 05-09 & 117.70 & 136.86 & 100.90 & HT-Direct \\ 
  Ghana & UPPER W,E \& NORTHERN & 05-09 & 112.93 & 100.60 & 126.79 & RW2 \\ 
  Ghana & UPPER W,E \& NORTHERN & 10-14 & 75.79 & 93.35 & 61.31 & HT-Direct \\ 
  Ghana & UPPER W,E \& NORTHERN & 10-14 & 90.32 & 74.71 & 109.01 & RW2 \\ 
  Ghana & UPPER W,E \& NORTHERN & 15-19 & 70.15 & 29.55 & 156.69 & RW2 \\ 
  Ghana & VOLTA & 80-84 & 155.46 & 177.78 & 135.48 & HT-Direct \\ 
  Ghana & VOLTA & 80-84 & 153.91 & 136.94 & 172.42 & RW2 \\ 
  Ghana & VOLTA & 85-89 & 118.64 & 134.55 & 104.39 & HT-Direct \\ 
  Ghana & VOLTA & 85-89 & 130.51 & 119.13 & 142.80 & RW2 \\ 
  Ghana & VOLTA & 90-94 & 107.58 & 128.13 & 89.99 & HT-Direct \\ 
  Ghana & VOLTA & 90-94 & 107.46 & 96.72 & 119.17 & RW2 \\ 
  Ghana & VOLTA & 95-99 & 88.21 & 114.48 & 67.51 & HT-Direct \\ 
  Ghana & VOLTA & 95-99 & 93.99 & 82.52 & 106.76 & RW2 \\ 
  Ghana & VOLTA & 00-04 & 80.22 & 104.66 & 61.09 & HT-Direct \\ 
  Ghana & VOLTA & 00-04 & 81.15 & 69.24 & 94.69 & RW2 \\ 
  Ghana & VOLTA & 05-09 & 64.31 & 87.12 & 47.16 & HT-Direct \\ 
  Ghana & VOLTA & 05-09 & 69.67 & 56.53 & 85.68 & RW2 \\ 
  Ghana & VOLTA & 10-14 & 55.65 & 97.16 & 31.26 & HT-Direct \\ 
  Ghana & VOLTA & 10-14 & 56.31 & 41.16 & 77.19 & RW2 \\ 
  Ghana & VOLTA & 15-19 & 44.45 & 17.45 & 108.42 & RW2 \\ 
  Ghana & WESTERN & 80-84 & 147.64 & 175.97 & 123.19 & HT-Direct \\ 
  Ghana & WESTERN & 80-84 & 155.85 & 134.45 & 179.72 & RW2 \\ 
  Ghana & WESTERN & 85-89 & 126.72 & 148.82 & 107.49 & HT-Direct \\ 
  Ghana & WESTERN & 85-89 & 135.99 & 122.12 & 151.25 & RW2 \\ 
  Ghana & WESTERN & 90-94 & 117.50 & 138.05 & 99.66 & HT-Direct \\ 
  Ghana & WESTERN & 90-94 & 114.39 & 102.93 & 127.24 & RW2 \\ 
  Ghana & WESTERN & 95-99 & 92.48 & 114.40 & 74.40 & HT-Direct \\ 
  Ghana & WESTERN & 95-99 & 101.06 & 89.55 & 114.34 & RW2 \\ 
  Ghana & WESTERN & 00-04 & 93.60 & 117.08 & 74.43 & HT-Direct \\ 
  Ghana & WESTERN & 00-04 & 86.59 & 75.05 & 99.85 & RW2 \\ 
  Ghana & WESTERN & 05-09 & 61.65 & 83.35 & 45.32 & HT-Direct \\ 
  Ghana & WESTERN & 05-09 & 72.27 & 59.51 & 87.38 & RW2 \\ 
  Ghana & WESTERN & 10-14 & 44.56 & 67.36 & 29.23 & HT-Direct \\ 
  Ghana & WESTERN & 10-14 & 56.55 & 41.79 & 75.19 & RW2 \\ 
  Ghana & WESTERN & 15-19 & 42.90 & 16.77 & 103.57 & RW2 \\ 
  Guinea & ALL & 80-84 & 266.35 & 262.09 & 270.46 & IHME \\ 
  Guinea & ALL & 80-84 & 277.54 & 261.89 & 293.76 & RW2 \\ 
  Guinea & ALL & 80-84 & 277.32 & 268.92 & 287.27 & UN \\ 
  Guinea & ALL & 85-89 & 243.38 & 239.92 & 247.15 & IHME \\ 
  Guinea & ALL & 85-89 & 252.93 & 240.41 & 265.82 & RW2 \\ 
  Guinea & ALL & 85-89 & 253.38 & 246.20 & 260.64 & UN \\ 
  Guinea & ALL & 90-94 & 218.40 & 215.40 & 221.44 & IHME \\ 
  Guinea & ALL & 90-94 & 225.75 & 216.66 & 235.12 & RW2 \\ 
  Guinea & ALL & 90-94 & 225.40 & 219.01 & 232.15 & UN \\ 
  Guinea & ALL & 95-99 & 191.12 & 188.41 & 193.91 & IHME \\ 
  Guinea & ALL & 95-99 & 191.68 & 183.93 & 199.61 & RW2 \\ 
  Guinea & ALL & 95-99 & 191.83 & 185.91 & 197.59 & UN \\ 
  Guinea & ALL & 00-04 & 164.93 & 162.06 & 167.72 & IHME \\ 
  Guinea & ALL & 00-04 & 155.90 & 148.49 & 163.64 & RW2 \\ 
  Guinea & ALL & 00-04 & 156.13 & 150.88 & 161.29 & UN \\ 
  Guinea & ALL & 05-09 & 141.98 & 138.53 & 145.73 & IHME \\ 
  Guinea & ALL & 05-09 & 127.16 & 117.49 & 137.51 & RW2 \\ 
  Guinea & ALL & 05-09 & 126.22 & 120.11 & 132.37 & UN \\ 
  Guinea & ALL & 10-14 & 120.45 & 116.00 & 125.13 & IHME \\ 
  Guinea & ALL & 10-14 & 103.98 & 91.81 & 117.45 & RW2 \\ 
  Guinea & ALL & 10-14 & 104.64 & 96.21 & 113.99 & UN \\ 
  Guinea & CENTRAL GUINEA & 80-84 & 263.88 & 294.91 & 235.03 & HT-Direct \\ 
  Guinea & CENTRAL GUINEA & 80-84 & 266.06 & 239.46 & 294.61 & RW2 \\ 
  Guinea & CENTRAL GUINEA & 85-89 & 220.23 & 243.53 & 198.57 & HT-Direct \\ 
  Guinea & CENTRAL GUINEA & 85-89 & 237.83 & 219.77 & 256.43 & RW2 \\ 
  Guinea & CENTRAL GUINEA & 90-94 & 206.07 & 223.56 & 189.62 & HT-Direct \\ 
  Guinea & CENTRAL GUINEA & 90-94 & 213.22 & 199.49 & 227.47 & RW2 \\ 
  Guinea & CENTRAL GUINEA & 95-99 & 182.20 & 198.39 & 167.07 & HT-Direct \\ 
  Guinea & CENTRAL GUINEA & 95-99 & 186.64 & 174.83 & 199.18 & RW2 \\ 
  Guinea & CENTRAL GUINEA & 00-04 & 163.77 & 176.99 & 151.35 & HT-Direct \\ 
  Guinea & CENTRAL GUINEA & 00-04 & 159.14 & 149.08 & 169.95 & RW2 \\ 
  Guinea & CENTRAL GUINEA & 05-09 & 139.19 & 165.35 & 116.58 & HT-Direct \\ 
  Guinea & CENTRAL GUINEA & 05-09 & 129.50 & 115.52 & 145.02 & RW2 \\ 
  Guinea & CENTRAL GUINEA & 10-14 & 98.27 & 131.01 & 73.02 & HT-Direct \\ 
  Guinea & CENTRAL GUINEA & 10-14 & 101.78 & 81.02 & 126.00 & RW2 \\ 
  Guinea & CENTRAL GUINEA & 15-19 & 78.69 & 32.32 & 176.51 & RW2 \\ 
  Guinea & CONAKRY & 80-84 & 209.49 & 248.24 & 175.38 & HT-Direct \\ 
  Guinea & CONAKRY & 80-84 & 215.12 & 183.40 & 250.92 & RW2 \\ 
  Guinea & CONAKRY & 85-89 & 162.95 & 190.09 & 139.02 & HT-Direct \\ 
  Guinea & CONAKRY & 85-89 & 176.29 & 156.85 & 197.57 & RW2 \\ 
  Guinea & CONAKRY & 90-94 & 147.80 & 170.04 & 128.03 & HT-Direct \\ 
  Guinea & CONAKRY & 90-94 & 144.29 & 129.70 & 160.50 & RW2 \\ 
  Guinea & CONAKRY & 95-99 & 109.77 & 126.99 & 94.63 & HT-Direct \\ 
  Guinea & CONAKRY & 95-99 & 112.83 & 100.18 & 126.48 & RW2 \\ 
  Guinea & CONAKRY & 00-04 & 63.38 & 86.06 & 46.38 & HT-Direct \\ 
  Guinea & CONAKRY & 00-04 & 87.36 & 73.91 & 102.42 & RW2 \\ 
  Guinea & CONAKRY & 05-09 & 88.62 & 123.66 & 62.80 & HT-Direct \\ 
  Guinea & CONAKRY & 05-09 & 68.52 & 54.75 & 85.52 & RW2 \\ 
  Guinea & CONAKRY & 10-14 & 56.44 & 100.38 & 31.07 & HT-Direct \\ 
  Guinea & CONAKRY & 10-14 & 51.52 & 34.64 & 77.16 & RW2 \\ 
  Guinea & CONAKRY & 15-19 & 38.09 & 12.90 & 108.01 & RW2 \\ 
  Guinea & FOREST GUINEA & 80-84 & 275.95 & 305.42 & 248.31 & HT-Direct \\ 
  Guinea & FOREST GUINEA & 80-84 & 288.06 & 259.96 & 317.83 & RW2 \\ 
  Guinea & FOREST GUINEA & 85-89 & 271.89 & 298.40 & 246.92 & HT-Direct \\ 
  Guinea & FOREST GUINEA & 85-89 & 290.70 & 269.42 & 313.23 & RW2 \\ 
  Guinea & FOREST GUINEA & 90-94 & 270.30 & 292.17 & 249.49 & HT-Direct \\ 
  Guinea & FOREST GUINEA & 90-94 & 270.41 & 252.94 & 289.05 & RW2 \\ 
  Guinea & FOREST GUINEA & 95-99 & 211.77 & 229.46 & 195.10 & HT-Direct \\ 
  Guinea & FOREST GUINEA & 95-99 & 217.86 & 203.69 & 233.09 & RW2 \\ 
  Guinea & FOREST GUINEA & 00-04 & 167.40 & 185.54 & 150.70 & HT-Direct \\ 
  Guinea & FOREST GUINEA & 00-04 & 162.06 & 148.87 & 175.98 & RW2 \\ 
  Guinea & FOREST GUINEA & 05-09 & 112.01 & 139.83 & 89.15 & HT-Direct \\ 
  Guinea & FOREST GUINEA & 05-09 & 119.51 & 103.97 & 136.41 & RW2 \\ 
  Guinea & FOREST GUINEA & 10-14 & 101.43 & 128.31 & 79.66 & HT-Direct \\ 
  Guinea & FOREST GUINEA & 10-14 & 93.99 & 76.40 & 114.98 & RW2 \\ 
  Guinea & FOREST GUINEA & 15-19 & 75.30 & 31.31 & 171.68 & RW2 \\ 
  Guinea & LOWER GUINEA & 80-84 & 272.29 & 305.73 & 241.25 & HT-Direct \\ 
  Guinea & LOWER GUINEA & 80-84 & 276.23 & 248.00 & 307.04 & RW2 \\ 
  Guinea & LOWER GUINEA & 85-89 & 229.17 & 250.90 & 208.79 & HT-Direct \\ 
  Guinea & LOWER GUINEA & 85-89 & 243.17 & 225.06 & 261.86 & RW2 \\ 
  Guinea & LOWER GUINEA & 90-94 & 206.08 & 224.88 & 188.47 & HT-Direct \\ 
  Guinea & LOWER GUINEA & 90-94 & 216.37 & 201.61 & 231.60 & RW2 \\ 
  Guinea & LOWER GUINEA & 95-99 & 184.50 & 201.52 & 168.62 & HT-Direct \\ 
  Guinea & LOWER GUINEA & 95-99 & 188.51 & 175.72 & 202.25 & RW2 \\ 
  Guinea & LOWER GUINEA & 00-04 & 164.27 & 181.69 & 148.21 & HT-Direct \\ 
  Guinea & LOWER GUINEA & 00-04 & 157.21 & 145.01 & 170.62 & RW2 \\ 
  Guinea & LOWER GUINEA & 05-09 & 135.99 & 167.08 & 109.93 & HT-Direct \\ 
  Guinea & LOWER GUINEA & 05-09 & 121.07 & 105.83 & 138.14 & RW2 \\ 
  Guinea & LOWER GUINEA & 10-14 & 72.30 & 102.61 & 50.44 & HT-Direct \\ 
  Guinea & LOWER GUINEA & 10-14 & 85.64 & 64.66 & 110.86 & RW2 \\ 
  Guinea & LOWER GUINEA & 15-19 & 58.40 & 22.40 & 140.62 & RW2 \\ 
  Guinea & UPPER GUINEA & 80-84 & 308.94 & 344.20 & 275.77 & HT-Direct \\ 
  Guinea & UPPER GUINEA & 80-84 & 314.90 & 284.19 & 347.63 & RW2 \\ 
  Guinea & UPPER GUINEA & 85-89 & 277.30 & 303.72 & 252.34 & HT-Direct \\ 
  Guinea & UPPER GUINEA & 85-89 & 286.40 & 266.33 & 307.90 & RW2 \\ 
  Guinea & UPPER GUINEA & 90-94 & 245.07 & 264.09 & 226.99 & HT-Direct \\ 
  Guinea & UPPER GUINEA & 90-94 & 251.76 & 236.81 & 267.15 & RW2 \\ 
  Guinea & UPPER GUINEA & 95-99 & 204.03 & 218.50 & 190.29 & HT-Direct \\ 
  Guinea & UPPER GUINEA & 95-99 & 212.02 & 199.72 & 224.68 & RW2 \\ 
  Guinea & UPPER GUINEA & 00-04 & 180.16 & 197.54 & 163.99 & HT-Direct \\ 
  Guinea & UPPER GUINEA & 00-04 & 183.84 & 171.19 & 197.10 & RW2 \\ 
  Guinea & UPPER GUINEA & 05-09 & 194.77 & 222.05 & 170.11 & HT-Direct \\ 
  Guinea & UPPER GUINEA & 05-09 & 169.26 & 153.44 & 186.47 & RW2 \\ 
  Guinea & UPPER GUINEA & 10-14 & 174.31 & 212.79 & 141.55 & HT-Direct \\ 
  Guinea & UPPER GUINEA & 10-14 & 163.51 & 137.31 & 194.68 & RW2 \\ 
  Guinea & UPPER GUINEA & 15-19 & 159.61 & 72.09 & 318.82 & RW2 \\ 
  Kenya & ALL & 80-84 & 101.94 & 100.33 & 103.62 & IHME \\ 
  Kenya & ALL & 80-84 & 102.59 & 95.32 & 110.36 & RW2 \\ 
  Kenya & ALL & 80-84 & 102.57 & 99.81 & 105.46 & UN \\ 
  Kenya & ALL & 85-89 & 93.67 & 92.16 & 95.14 & IHME \\ 
  Kenya & ALL & 85-89 & 97.40 & 91.44 & 103.65 & RW2 \\ 
  Kenya & ALL & 85-89 & 97.47 & 94.70 & 100.12 & UN \\ 
  Kenya & ALL & 90-94 & 93.48 & 91.93 & 95.02 & IHME \\ 
  Kenya & ALL & 90-94 & 108.35 & 102.29 & 114.75 & RW2 \\ 
  Kenya & ALL & 90-94 & 108.12 & 105.07 & 111.37 & UN \\ 
  Kenya & ALL & 95-99 & 93.71 & 92.03 & 95.27 & IHME \\ 
  Kenya & ALL & 95-99 & 113.65 & 106.96 & 120.65 & RW2 \\ 
  Kenya & ALL & 95-99 & 113.99 & 110.54 & 117.42 & UN \\ 
  Kenya & ALL & 00-04 & 81.88 & 80.14 & 83.73 & IHME \\ 
  Kenya & ALL & 00-04 & 100.00 & 93.65 & 106.79 & RW2 \\ 
  Kenya & ALL & 00-04 & 99.71 & 96.14 & 103.53 & UN \\ 
  Kenya & ALL & 05-09 & 63.75 & 62.16 & 65.45 & IHME \\ 
  Kenya & ALL & 05-09 & 75.16 & 68.68 & 82.16 & RW2 \\ 
  Kenya & ALL & 05-09 & 75.42 & 71.54 & 79.43 & UN \\ 
  Kenya & ALL & 10-14 & 52.82 & 50.68 & 54.98 & IHME \\ 
  Kenya & ALL & 10-14 & 56.44 & 51.21 & 62.13 & RW2 \\ 
  Kenya & ALL & 10-14 & 56.31 & 51.59 & 61.06 & UN \\ 
  Kenya & CENTRAL & 80-84 & 48.15 & 61.80 & 37.40 & HT-Direct \\ 
  Kenya & CENTRAL & 80-84 & 49.10 & 39.56 & 60.98 & RW2 \\ 
  Kenya & CENTRAL & 85-89 & 42.31 & 53.90 & 33.13 & HT-Direct \\ 
  Kenya & CENTRAL & 85-89 & 44.01 & 37.59 & 51.41 & RW2 \\ 
  Kenya & CENTRAL & 90-94 & 49.57 & 60.73 & 40.38 & HT-Direct \\ 
  Kenya & CENTRAL & 90-94 & 53.85 & 46.56 & 62.17 & RW2 \\ 
  Kenya & CENTRAL & 95-99 & 56.26 & 71.49 & 44.12 & HT-Direct \\ 
  Kenya & CENTRAL & 95-99 & 68.02 & 57.99 & 79.43 & RW2 \\ 
  Kenya & CENTRAL & 00-04 & 60.07 & 76.93 & 46.71 & HT-Direct \\ 
  Kenya & CENTRAL & 00-04 & 67.01 & 56.52 & 79.36 & RW2 \\ 
  Kenya & CENTRAL & 05-09 & 52.10 & 68.95 & 39.20 & HT-Direct \\ 
  Kenya & CENTRAL & 05-09 & 56.51 & 45.88 & 69.35 & RW2 \\ 
  Kenya & CENTRAL & 10-14 & 40.79 & 57.56 & 28.75 & HT-Direct \\ 
  Kenya & CENTRAL & 10-14 & 47.71 & 35.53 & 64.07 & RW2 \\ 
  Kenya & CENTRAL & 15-19 & 40.82 & 14.93 & 106.86 & RW2 \\ 
  Kenya & COAST & 80-84 & 156.99 & 181.03 & 135.62 & HT-Direct \\ 
  Kenya & COAST & 80-84 & 154.42 & 135.21 & 176.20 & RW2 \\ 
  Kenya & COAST & 85-89 & 105.60 & 120.65 & 92.24 & HT-Direct \\ 
  Kenya & COAST & 85-89 & 116.38 & 104.89 & 128.81 & RW2 \\ 
  Kenya & COAST & 90-94 & 107.92 & 122.93 & 94.54 & HT-Direct \\ 
  Kenya & COAST & 90-94 & 120.41 & 108.61 & 133.04 & RW2 \\ 
  Kenya & COAST & 95-99 & 137.79 & 163.20 & 115.79 & HT-Direct \\ 
  Kenya & COAST & 95-99 & 128.56 & 113.73 & 145.22 & RW2 \\ 
  Kenya & COAST & 00-04 & 80.76 & 97.76 & 66.49 & HT-Direct \\ 
  Kenya & COAST & 00-04 & 106.07 & 92.52 & 121.30 & RW2 \\ 
  Kenya & COAST & 05-09 & 66.19 & 79.86 & 54.73 & HT-Direct \\ 
  Kenya & COAST & 05-09 & 76.81 & 65.06 & 90.40 & RW2 \\ 
  Kenya & COAST & 10-14 & 54.96 & 74.98 & 40.06 & HT-Direct \\ 
  Kenya & COAST & 10-14 & 56.99 & 43.99 & 74.00 & RW2 \\ 
  Kenya & COAST & 15-19 & 42.81 & 16.01 & 110.65 & RW2 \\ 
  Kenya & EASTERN & 80-84 & 75.07 & 88.74 & 63.37 & HT-Direct \\ 
  Kenya & EASTERN & 80-84 & 76.36 & 65.74 & 88.73 & RW2 \\ 
  Kenya & EASTERN & 85-89 & 62.33 & 73.67 & 52.64 & HT-Direct \\ 
  Kenya & EASTERN & 85-89 & 64.77 & 57.59 & 72.76 & RW2 \\ 
  Kenya & EASTERN & 90-94 & 64.04 & 74.95 & 54.62 & HT-Direct \\ 
  Kenya & EASTERN & 90-94 & 75.33 & 67.43 & 84.03 & RW2 \\ 
  Kenya & EASTERN & 95-99 & 85.41 & 99.59 & 73.09 & HT-Direct \\ 
  Kenya & EASTERN & 95-99 & 89.90 & 79.98 & 100.96 & RW2 \\ 
  Kenya & EASTERN & 00-04 & 76.93 & 91.16 & 64.77 & HT-Direct \\ 
  Kenya & EASTERN & 00-04 & 81.91 & 71.94 & 93.13 & RW2 \\ 
  Kenya & EASTERN & 05-09 & 45.79 & 58.05 & 36.01 & HT-Direct \\ 
  Kenya & EASTERN & 05-09 & 63.68 & 53.82 & 75.12 & RW2 \\ 
  Kenya & EASTERN & 10-14 & 46.60 & 60.62 & 35.69 & HT-Direct \\ 
  Kenya & EASTERN & 10-14 & 50.32 & 39.94 & 63.33 & RW2 \\ 
  Kenya & EASTERN & 15-19 & 40.38 & 15.48 & 100.64 & RW2 \\ 
  Kenya & NAIROBI & 80-84 & 66.08 & 97.47 & 44.31 & HT-Direct \\ 
  Kenya & NAIROBI & 80-84 & 68.51 & 50.57 & 92.56 & RW2 \\ 
  Kenya & NAIROBI & 85-89 & 60.97 & 83.52 & 44.22 & HT-Direct \\ 
  Kenya & NAIROBI & 85-89 & 63.12 & 51.27 & 77.58 & RW2 \\ 
  Kenya & NAIROBI & 90-94 & 72.91 & 97.05 & 54.41 & HT-Direct \\ 
  Kenya & NAIROBI & 90-94 & 78.40 & 65.23 & 93.90 & RW2 \\ 
  Kenya & NAIROBI & 95-99 & 102.28 & 140.88 & 73.35 & HT-Direct \\ 
  Kenya & NAIROBI & 95-99 & 99.07 & 81.91 & 119.29 & RW2 \\ 
  Kenya & NAIROBI & 00-04 & 72.78 & 97.93 & 53.70 & HT-Direct \\ 
  Kenya & NAIROBI & 00-04 & 97.52 & 79.84 & 118.40 & RW2 \\ 
  Kenya & NAIROBI & 05-09 & 71.39 & 102.08 & 49.42 & HT-Direct \\ 
  Kenya & NAIROBI & 05-09 & 83.28 & 66.06 & 104.46 & RW2 \\ 
  Kenya & NAIROBI & 10-14 & 70.98 & 99.76 & 50.04 & HT-Direct \\ 
  Kenya & NAIROBI & 10-14 & 72.15 & 52.98 & 97.66 & RW2 \\ 
  Kenya & NAIROBI & 15-19 & 62.98 & 23.32 & 160.58 & RW2 \\ 
  Kenya & NORTHEASTERN & 80-84 & 219.17 & 312.84 & 147.53 & HT-Direct \\ 
  Kenya & NORTHEASTERN & 80-84 & 209.80 & 157.73 & 271.51 & RW2 \\ 
  Kenya & NORTHEASTERN & 85-89 & 117.52 & 150.44 & 91.04 & HT-Direct \\ 
  Kenya & NORTHEASTERN & 85-89 & 155.88 & 130.77 & 184.21 & RW2 \\ 
  Kenya & NORTHEASTERN & 90-94 & 169.72 & 200.82 & 142.57 & HT-Direct \\ 
  Kenya & NORTHEASTERN & 90-94 & 153.08 & 133.77 & 175.08 & RW2 \\ 
  Kenya & NORTHEASTERN & 95-99 & 133.24 & 162.55 & 108.54 & HT-Direct \\ 
  Kenya & NORTHEASTERN & 95-99 & 146.68 & 127.01 & 169.07 & RW2 \\ 
  Kenya & NORTHEASTERN & 00-04 & 77.82 & 97.41 & 61.90 & HT-Direct \\ 
  Kenya & NORTHEASTERN & 00-04 & 106.89 & 90.18 & 125.51 & RW2 \\ 
  Kenya & NORTHEASTERN & 05-09 & 56.48 & 74.33 & 42.71 & HT-Direct \\ 
  Kenya & NORTHEASTERN & 05-09 & 69.20 & 56.38 & 84.57 & RW2 \\ 
  Kenya & NORTHEASTERN & 10-14 & 47.82 & 65.18 & 34.90 & HT-Direct \\ 
  Kenya & NORTHEASTERN & 10-14 & 46.84 & 35.28 & 62.26 & RW2 \\ 
  Kenya & NORTHEASTERN & 15-19 & 32.16 & 11.68 & 87.41 & RW2 \\ 
  Kenya & NYANZA & 80-84 & 151.29 & 173.65 & 131.35 & HT-Direct \\ 
  Kenya & NYANZA & 80-84 & 163.28 & 142.92 & 185.21 & RW2 \\ 
  Kenya & NYANZA & 85-89 & 169.65 & 187.37 & 153.30 & HT-Direct \\ 
  Kenya & NYANZA & 85-89 & 174.04 & 159.89 & 189.36 & RW2 \\ 
  Kenya & NYANZA & 90-94 & 203.88 & 226.12 & 183.32 & HT-Direct \\ 
  Kenya & NYANZA & 90-94 & 217.41 & 199.74 & 236.37 & RW2 \\ 
  Kenya & NYANZA & 95-99 & 216.33 & 244.19 & 190.84 & HT-Direct \\ 
  Kenya & NYANZA & 95-99 & 250.29 & 227.06 & 275.19 & RW2 \\ 
  Kenya & NYANZA & 00-04 & 205.36 & 236.89 & 177.06 & HT-Direct \\ 
  Kenya & NYANZA & 00-04 & 215.27 & 191.98 & 241.26 & RW2 \\ 
  Kenya & NYANZA & 05-09 & 116.34 & 135.57 & 99.53 & HT-Direct \\ 
  Kenya & NYANZA & 05-09 & 139.38 & 121.77 & 159.33 & RW2 \\ 
  Kenya & NYANZA & 10-14 & 69.00 & 83.34 & 56.97 & HT-Direct \\ 
  Kenya & NYANZA & 10-14 & 79.02 & 65.33 & 95.20 & RW2 \\ 
  Kenya & NYANZA & 15-19 & 42.77 & 16.47 & 106.57 & RW2 \\ 
  Kenya & RIFT VALLEY & 80-84 & 79.34 & 93.04 & 67.50 & HT-Direct \\ 
  Kenya & RIFT VALLEY & 80-84 & 79.03 & 68.62 & 90.88 & RW2 \\ 
  Kenya & RIFT VALLEY & 85-89 & 61.90 & 71.65 & 53.40 & HT-Direct \\ 
  Kenya & RIFT VALLEY & 85-89 & 67.94 & 61.09 & 75.30 & RW2 \\ 
  Kenya & RIFT VALLEY & 90-94 & 75.98 & 86.61 & 66.56 & HT-Direct \\ 
  Kenya & RIFT VALLEY & 90-94 & 77.85 & 70.61 & 85.62 & RW2 \\ 
  Kenya & RIFT VALLEY & 95-99 & 73.32 & 85.01 & 63.13 & HT-Direct \\ 
  Kenya & RIFT VALLEY & 95-99 & 90.23 & 80.93 & 100.31 & RW2 \\ 
  Kenya & RIFT VALLEY & 00-04 & 71.86 & 82.30 & 62.65 & HT-Direct \\ 
  Kenya & RIFT VALLEY & 00-04 & 81.50 & 73.09 & 90.70 & RW2 \\ 
  Kenya & RIFT VALLEY & 05-09 & 49.53 & 57.60 & 42.54 & HT-Direct \\ 
  Kenya & RIFT VALLEY & 05-09 & 62.18 & 54.74 & 70.53 & RW2 \\ 
  Kenya & RIFT VALLEY & 10-14 & 45.68 & 53.97 & 38.61 & HT-Direct \\ 
  Kenya & RIFT VALLEY & 10-14 & 47.21 & 40.30 & 55.36 & RW2 \\ 
  Kenya & RIFT VALLEY & 15-19 & 35.99 & 14.43 & 88.01 & RW2 \\ 
  Kenya & WESTERN & 80-84 & 119.11 & 139.54 & 101.32 & HT-Direct \\ 
  Kenya & WESTERN & 80-84 & 124.21 & 107.17 & 143.54 & RW2 \\ 
  Kenya & WESTERN & 85-89 & 118.37 & 135.87 & 102.85 & HT-Direct \\ 
  Kenya & WESTERN & 85-89 & 124.33 & 111.99 & 138.08 & RW2 \\ 
  Kenya & WESTERN & 90-94 & 145.57 & 163.39 & 129.40 & HT-Direct \\ 
  Kenya & WESTERN & 90-94 & 152.43 & 138.58 & 167.49 & RW2 \\ 
  Kenya & WESTERN & 95-99 & 143.68 & 166.33 & 123.66 & HT-Direct \\ 
  Kenya & WESTERN & 95-99 & 177.16 & 158.68 & 197.25 & RW2 \\ 
  Kenya & WESTERN & 00-04 & 148.67 & 171.88 & 128.11 & HT-Direct \\ 
  Kenya & WESTERN & 00-04 & 155.27 & 137.37 & 175.84 & RW2 \\ 
  Kenya & WESTERN & 05-09 & 90.07 & 112.17 & 71.97 & HT-Direct \\ 
  Kenya & WESTERN & 05-09 & 103.70 & 88.49 & 121.27 & RW2 \\ 
  Kenya & WESTERN & 10-14 & 53.89 & 66.20 & 43.76 & HT-Direct \\ 
  Kenya & WESTERN & 10-14 & 61.85 & 50.28 & 75.48 & RW2 \\ 
  Kenya & WESTERN & 15-19 & 35.84 & 13.69 & 89.26 & RW2 \\ 
  Lesotho & ALL & 80-84 & 98.56 & 96.09 & 101.23 & IHME \\ 
  Lesotho & ALL & 80-84 & 109.45 & 88.41 & 134.86 & RW2 \\ 
  Lesotho & ALL & 80-84 & 109.69 & 105.00 & 115.30 & UN \\ 
  Lesotho & ALL & 85-89 & 94.41 & 92.09 & 96.83 & IHME \\ 
  Lesotho & ALL & 85-89 & 91.81 & 80.26 & 104.69 & RW2 \\ 
  Lesotho & ALL & 85-89 & 92.68 & 88.60 & 96.91 & UN \\ 
  Lesotho & ALL & 90-94 & 90.80 & 88.68 & 92.82 & IHME \\ 
  Lesotho & ALL & 90-94 & 91.73 & 83.05 & 101.22 & RW2 \\ 
  Lesotho & ALL & 90-94 & 90.01 & 86.45 & 93.80 & UN \\ 
  Lesotho & ALL & 95-99 & 97.19 & 95.08 & 99.45 & IHME \\ 
  Lesotho & ALL & 95-99 & 104.95 & 92.86 & 118.22 & RW2 \\ 
  Lesotho & ALL & 95-99 & 107.25 & 103.63 & 111.67 & UN \\ 
  Lesotho & ALL & 00-04 & 112.34 & 109.67 & 115.04 & IHME \\ 
  Lesotho & ALL & 00-04 & 121.07 & 111.35 & 131.59 & RW2 \\ 
  Lesotho & ALL & 00-04 & 120.24 & 115.87 & 124.57 & UN \\ 
  Lesotho & ALL & 05-09 & 116.61 & 112.84 & 120.42 & IHME \\ 
  Lesotho & ALL & 05-09 & 117.67 & 106.51 & 129.85 & RW2 \\ 
  Lesotho & ALL & 05-09 & 117.97 & 113.03 & 123.21 & UN \\ 
  Lesotho & ALL & 10-14 & 99.72 & 94.23 & 106.12 & IHME \\ 
  Lesotho & ALL & 10-14 & 95.68 & 81.15 & 112.32 & RW2 \\ 
  Lesotho & ALL & 10-14 & 95.61 & 88.80 & 103.23 & UN \\ 
  Lesotho & BEREA & 80-84 & 156.27 & 216.69 & 110.33 & HT-Direct \\ 
  Lesotho & BEREA & 80-84 & 153.18 & 110.39 & 208.38 & RW2 \\ 
  Lesotho & BEREA & 85-89 & 78.06 & 112.50 & 53.53 & HT-Direct \\ 
  Lesotho & BEREA & 85-89 & 113.26 & 88.62 & 143.50 & RW2 \\ 
  Lesotho & BEREA & 90-94 & 102.41 & 135.41 & 76.74 & HT-Direct \\ 
  Lesotho & BEREA & 90-94 & 103.95 & 86.38 & 124.33 & RW2 \\ 
  Lesotho & BEREA & 95-99 & 84.42 & 114.54 & 61.67 & HT-Direct \\ 
  Lesotho & BEREA & 95-99 & 107.93 & 90.61 & 127.98 & RW2 \\ 
  Lesotho & BEREA & 00-04 & 106.50 & 133.82 & 84.21 & HT-Direct \\ 
  Lesotho & BEREA & 00-04 & 109.00 & 93.36 & 127.13 & RW2 \\ 
  Lesotho & BEREA & 05-09 & 92.54 & 122.03 & 69.61 & HT-Direct \\ 
  Lesotho & BEREA & 05-09 & 100.89 & 83.03 & 121.89 & RW2 \\ 
  Lesotho & BEREA & 10-14 & 75.48 & 111.98 & 50.20 & HT-Direct \\ 
  Lesotho & BEREA & 10-14 & 75.59 & 56.37 & 100.65 & RW2 \\ 
  Lesotho & BEREA & 15-19 & 52.49 & 19.26 & 132.79 & RW2 \\ 
  Lesotho & BUTHA-BUTHE & 80-84 & 57.90 & 104.19 & 31.46 & HT-Direct \\ 
  Lesotho & BUTHA-BUTHE & 80-84 & 91.15 & 59.08 & 138.08 & RW2 \\ 
  Lesotho & BUTHA-BUTHE & 85-89 & 70.92 & 105.47 & 47.09 & HT-Direct \\ 
  Lesotho & BUTHA-BUTHE & 85-89 & 73.80 & 55.05 & 98.40 & RW2 \\ 
  Lesotho & BUTHA-BUTHE & 90-94 & 71.81 & 96.96 & 52.81 & HT-Direct \\ 
  Lesotho & BUTHA-BUTHE & 90-94 & 74.65 & 60.29 & 91.88 & RW2 \\ 
  Lesotho & BUTHA-BUTHE & 95-99 & 75.18 & 110.59 & 50.46 & HT-Direct \\ 
  Lesotho & BUTHA-BUTHE & 95-99 & 84.79 & 69.42 & 103.23 & RW2 \\ 
  Lesotho & BUTHA-BUTHE & 00-04 & 84.13 & 110.09 & 63.86 & HT-Direct \\ 
  Lesotho & BUTHA-BUTHE & 00-04 & 93.41 & 78.32 & 111.15 & RW2 \\ 
  Lesotho & BUTHA-BUTHE & 05-09 & 90.93 & 121.32 & 67.57 & HT-Direct \\ 
  Lesotho & BUTHA-BUTHE & 05-09 & 94.77 & 76.29 & 117.09 & RW2 \\ 
  Lesotho & BUTHA-BUTHE & 10-14 & 70.91 & 116.07 & 42.48 & HT-Direct \\ 
  Lesotho & BUTHA-BUTHE & 10-14 & 78.31 & 55.74 & 109.18 & RW2 \\ 
  Lesotho & BUTHA-BUTHE & 15-19 & 60.50 & 21.31 & 156.43 & RW2 \\ 
  Lesotho & LERIBE & 80-84 & 99.07 & 168.98 & 56.13 & HT-Direct \\ 
  Lesotho & LERIBE & 80-84 & 102.20 & 70.07 & 146.76 & RW2 \\ 
  Lesotho & LERIBE & 85-89 & 86.06 & 117.01 & 62.72 & HT-Direct \\ 
  Lesotho & LERIBE & 85-89 & 85.57 & 66.59 & 109.38 & RW2 \\ 
  Lesotho & LERIBE & 90-94 & 73.28 & 97.22 & 54.87 & HT-Direct \\ 
  Lesotho & LERIBE & 90-94 & 89.64 & 74.32 & 107.65 & RW2 \\ 
  Lesotho & LERIBE & 95-99 & 86.06 & 119.76 & 61.19 & HT-Direct \\ 
  Lesotho & LERIBE & 95-99 & 106.45 & 89.35 & 126.16 & RW2 \\ 
  Lesotho & LERIBE & 00-04 & 121.32 & 152.78 & 95.62 & HT-Direct \\ 
  Lesotho & LERIBE & 00-04 & 122.65 & 105.59 & 141.72 & RW2 \\ 
  Lesotho & LERIBE & 05-09 & 130.16 & 164.10 & 102.38 & HT-Direct \\ 
  Lesotho & LERIBE & 05-09 & 129.72 & 108.88 & 154.00 & RW2 \\ 
  Lesotho & LERIBE & 10-14 & 97.98 & 138.70 & 68.27 & HT-Direct \\ 
  Lesotho & LERIBE & 10-14 & 112.14 & 86.24 & 145.19 & RW2 \\ 
  Lesotho & LERIBE & 15-19 & 90.67 & 34.51 & 213.69 & RW2 \\ 
  Lesotho & MAFETENG & 80-84 & 59.65 & 107.92 & 32.19 & HT-Direct \\ 
  Lesotho & MAFETENG & 80-84 & 97.74 & 64.03 & 147.47 & RW2 \\ 
  Lesotho & MAFETENG & 85-89 & 90.78 & 128.73 & 63.21 & HT-Direct \\ 
  Lesotho & MAFETENG & 85-89 & 81.04 & 60.70 & 107.68 & RW2 \\ 
  Lesotho & MAFETENG & 90-94 & 78.96 & 110.97 & 55.60 & HT-Direct \\ 
  Lesotho & MAFETENG & 90-94 & 83.98 & 67.45 & 104.10 & RW2 \\ 
  Lesotho & MAFETENG & 95-99 & 68.47 & 103.28 & 44.80 & HT-Direct \\ 
  Lesotho & MAFETENG & 95-99 & 98.47 & 80.59 & 119.35 & RW2 \\ 
  Lesotho & MAFETENG & 00-04 & 96.49 & 125.32 & 73.74 & HT-Direct \\ 
  Lesotho & MAFETENG & 00-04 & 112.44 & 95.13 & 132.30 & RW2 \\ 
  Lesotho & MAFETENG & 05-09 & 121.57 & 151.84 & 96.64 & HT-Direct \\ 
  Lesotho & MAFETENG & 05-09 & 117.97 & 97.54 & 141.97 & RW2 \\ 
  Lesotho & MAFETENG & 10-14 & 89.59 & 155.58 & 49.93 & HT-Direct \\ 
  Lesotho & MAFETENG & 10-14 & 100.54 & 72.73 & 137.91 & RW2 \\ 
  Lesotho & MAFETENG & 15-19 & 80.22 & 28.23 & 200.48 & RW2 \\ 
  Lesotho & MASERU & 80-84 & 50.39 & 82.60 & 30.32 & HT-Direct \\ 
  Lesotho & MASERU & 80-84 & 76.38 & 52.27 & 109.44 & RW2 \\ 
  Lesotho & MASERU & 85-89 & 52.49 & 77.12 & 35.42 & HT-Direct \\ 
  Lesotho & MASERU & 85-89 & 68.27 & 52.47 & 88.01 & RW2 \\ 
  Lesotho & MASERU & 90-94 & 81.30 & 105.52 & 62.25 & HT-Direct \\ 
  Lesotho & MASERU & 90-94 & 76.65 & 63.90 & 91.68 & RW2 \\ 
  Lesotho & MASERU & 95-99 & 74.79 & 96.62 & 57.57 & HT-Direct \\ 
  Lesotho & MASERU & 95-99 & 96.44 & 81.99 & 113.17 & RW2 \\ 
  Lesotho & MASERU & 00-04 & 119.31 & 144.32 & 98.14 & HT-Direct \\ 
  Lesotho & MASERU & 00-04 & 116.47 & 101.39 & 133.61 & RW2 \\ 
  Lesotho & MASERU & 05-09 & 111.96 & 147.47 & 84.16 & HT-Direct \\ 
  Lesotho & MASERU & 05-09 & 127.45 & 106.63 & 151.96 & RW2 \\ 
  Lesotho & MASERU & 10-14 & 96.50 & 135.63 & 67.78 & HT-Direct \\ 
  Lesotho & MASERU & 10-14 & 113.38 & 86.83 & 146.47 & RW2 \\ 
  Lesotho & MASERU & 15-19 & 94.18 & 35.71 & 221.44 & RW2 \\ 
  Lesotho & MOHALE'S HOEK & 80-84 & 126.37 & 190.23 & 81.78 & HT-Direct \\ 
  Lesotho & MOHALE'S HOEK & 80-84 & 132.29 & 94.49 & 181.80 & RW2 \\ 
  Lesotho & MOHALE'S HOEK & 85-89 & 96.17 & 128.87 & 71.09 & HT-Direct \\ 
  Lesotho & MOHALE'S HOEK & 85-89 & 107.06 & 84.85 & 134.33 & RW2 \\ 
  Lesotho & MOHALE'S HOEK & 90-94 & 87.98 & 114.58 & 67.09 & HT-Direct \\ 
  Lesotho & MOHALE'S HOEK & 90-94 & 108.00 & 91.02 & 127.84 & RW2 \\ 
  Lesotho & MOHALE'S HOEK & 95-99 & 120.89 & 157.57 & 91.82 & HT-Direct \\ 
  Lesotho & MOHALE'S HOEK & 95-99 & 122.92 & 104.65 & 144.21 & RW2 \\ 
  Lesotho & MOHALE'S HOEK & 00-04 & 137.37 & 173.17 & 108.00 & HT-Direct \\ 
  Lesotho & MOHALE'S HOEK & 00-04 & 135.22 & 116.74 & 156.12 & RW2 \\ 
  Lesotho & MOHALE'S HOEK & 05-09 & 133.38 & 169.68 & 103.88 & HT-Direct \\ 
  Lesotho & MOHALE'S HOEK & 05-09 & 135.86 & 113.37 & 162.11 & RW2 \\ 
  Lesotho & MOHALE'S HOEK & 10-14 & 89.62 & 134.68 & 58.61 & HT-Direct \\ 
  Lesotho & MOHALE'S HOEK & 10-14 & 110.85 & 83.63 & 145.34 & RW2 \\ 
  Lesotho & MOHALE'S HOEK & 15-19 & 84.01 & 31.38 & 201.11 & RW2 \\ 
  Lesotho & MOKHOTLONG & 80-84 & 104.16 & 180.34 & 57.89 & HT-Direct \\ 
  Lesotho & MOKHOTLONG & 80-84 & 148.78 & 101.63 & 210.85 & RW2 \\ 
  Lesotho & MOKHOTLONG & 85-89 & 87.20 & 124.00 & 60.56 & HT-Direct \\ 
  Lesotho & MOKHOTLONG & 85-89 & 115.33 & 89.37 & 146.62 & RW2 \\ 
  Lesotho & MOKHOTLONG & 90-94 & 123.46 & 154.42 & 97.98 & HT-Direct \\ 
  Lesotho & MOKHOTLONG & 90-94 & 110.26 & 92.72 & 130.62 & RW2 \\ 
  Lesotho & MOKHOTLONG & 95-99 & 95.33 & 128.94 & 69.78 & HT-Direct \\ 
  Lesotho & MOKHOTLONG & 95-99 & 117.63 & 99.48 & 138.82 & RW2 \\ 
  Lesotho & MOKHOTLONG & 00-04 & 108.55 & 134.87 & 86.86 & HT-Direct \\ 
  Lesotho & MOKHOTLONG & 00-04 & 121.69 & 104.87 & 140.83 & RW2 \\ 
  Lesotho & MOKHOTLONG & 05-09 & 103.77 & 137.66 & 77.47 & HT-Direct \\ 
  Lesotho & MOKHOTLONG & 05-09 & 116.21 & 96.26 & 139.59 & RW2 \\ 
  Lesotho & MOKHOTLONG & 10-14 & 96.92 & 136.33 & 68.00 & HT-Direct \\ 
  Lesotho & MOKHOTLONG & 10-14 & 90.82 & 68.76 & 119.13 & RW2 \\ 
  Lesotho & MOKHOTLONG & 15-19 & 65.98 & 24.51 & 162.84 & RW2 \\ 
  Lesotho & QASHA'S NEK & 80-84 & 154.47 & 243.34 & 94.02 & HT-Direct \\ 
  Lesotho & QASHA'S NEK & 80-84 & 156.29 & 109.21 & 220.38 & RW2 \\ 
  Lesotho & QASHA'S NEK & 85-89 & 111.82 & 155.81 & 79.09 & HT-Direct \\ 
  Lesotho & QASHA'S NEK & 85-89 & 118.77 & 92.62 & 150.84 & RW2 \\ 
  Lesotho & QASHA'S NEK & 90-94 & 106.19 & 134.42 & 83.32 & HT-Direct \\ 
  Lesotho & QASHA'S NEK & 90-94 & 111.92 & 93.96 & 132.73 & RW2 \\ 
  Lesotho & QASHA'S NEK & 95-99 & 93.77 & 122.96 & 70.95 & HT-Direct \\ 
  Lesotho & QASHA'S NEK & 95-99 & 119.49 & 100.94 & 140.96 & RW2 \\ 
  Lesotho & QASHA'S NEK & 00-04 & 103.49 & 136.54 & 77.73 & HT-Direct \\ 
  Lesotho & QASHA'S NEK & 00-04 & 124.69 & 105.91 & 146.14 & RW2 \\ 
  Lesotho & QASHA'S NEK & 05-09 & 125.92 & 162.66 & 96.53 & HT-Direct \\ 
  Lesotho & QASHA'S NEK & 05-09 & 120.43 & 98.49 & 146.36 & RW2 \\ 
  Lesotho & QASHA'S NEK & 10-14 & 97.20 & 154.55 & 59.64 & HT-Direct \\ 
  Lesotho & QASHA'S NEK & 10-14 & 94.62 & 69.07 & 128.79 & RW2 \\ 
  Lesotho & QASHA'S NEK & 15-19 & 69.23 & 24.92 & 173.64 & RW2 \\ 
  Lesotho & QUTHING & 80-84 & 122.26 & 180.04 & 81.18 & HT-Direct \\ 
  Lesotho & QUTHING & 80-84 & 150.26 & 105.49 & 209.39 & RW2 \\ 
  Lesotho & QUTHING & 85-89 & 76.67 & 113.37 & 51.17 & HT-Direct \\ 
  Lesotho & QUTHING & 85-89 & 117.51 & 91.51 & 149.57 & RW2 \\ 
  Lesotho & QUTHING & 90-94 & 123.26 & 155.31 & 97.07 & HT-Direct \\ 
  Lesotho & QUTHING & 90-94 & 113.68 & 95.17 & 135.25 & RW2 \\ 
  Lesotho & QUTHING & 95-99 & 110.08 & 148.19 & 80.84 & HT-Direct \\ 
  Lesotho & QUTHING & 95-99 & 122.71 & 103.01 & 145.60 & RW2 \\ 
  Lesotho & QUTHING & 00-04 & 108.16 & 141.09 & 82.19 & HT-Direct \\ 
  Lesotho & QUTHING & 00-04 & 128.08 & 108.45 & 150.82 & RW2 \\ 
  Lesotho & QUTHING & 05-09 & 125.70 & 163.64 & 95.56 & HT-Direct \\ 
  Lesotho & QUTHING & 05-09 & 122.71 & 99.82 & 150.12 & RW2 \\ 
  Lesotho & QUTHING & 10-14 & 77.80 & 129.69 & 45.58 & HT-Direct \\ 
  Lesotho & QUTHING & 10-14 & 95.25 & 68.05 & 131.58 & RW2 \\ 
  Lesotho & QUTHING & 15-19 & 68.33 & 24.02 & 174.77 & RW2 \\ 
  Lesotho & THABA-TSEKA & 80-84 & 134.26 & 218.11 & 79.38 & HT-Direct \\ 
  Lesotho & THABA-TSEKA & 80-84 & 166.46 & 115.93 & 233.67 & RW2 \\ 
  Lesotho & THABA-TSEKA & 85-89 & 95.81 & 135.18 & 67.02 & HT-Direct \\ 
  Lesotho & THABA-TSEKA & 85-89 & 123.94 & 95.59 & 159.51 & RW2 \\ 
  Lesotho & THABA-TSEKA & 90-94 & 105.50 & 142.43 & 77.29 & HT-Direct \\ 
  Lesotho & THABA-TSEKA & 90-94 & 113.84 & 94.30 & 136.96 & RW2 \\ 
  Lesotho & THABA-TSEKA & 95-99 & 108.90 & 140.64 & 83.62 & HT-Direct \\ 
  Lesotho & THABA-TSEKA & 95-99 & 117.34 & 99.73 & 137.79 & RW2 \\ 
  Lesotho & THABA-TSEKA & 00-04 & 121.14 & 148.94 & 97.92 & HT-Direct \\ 
  Lesotho & THABA-TSEKA & 00-04 & 116.76 & 101.10 & 134.66 & RW2 \\ 
  Lesotho & THABA-TSEKA & 05-09 & 94.25 & 121.04 & 72.90 & HT-Direct \\ 
  Lesotho & THABA-TSEKA & 05-09 & 105.85 & 87.15 & 127.73 & RW2 \\ 
  Lesotho & THABA-TSEKA & 10-14 & 55.74 & 105.03 & 28.84 & HT-Direct \\ 
  Lesotho & THABA-TSEKA & 10-14 & 77.37 & 56.11 & 105.10 & RW2 \\ 
  Lesotho & THABA-TSEKA & 15-19 & 52.63 & 18.87 & 132.93 & RW2 \\ 
  Liberia & ALL & 80-84 & 236.32 & 232.47 & 239.96 & IHME \\ 
  Liberia & ALL & 80-84 & 237.19 & 207.24 & 270.04 & RW2 \\ 
  Liberia & ALL & 80-84 & 236.87 & 227.22 & 245.59 & UN \\ 
  Liberia & ALL & 85-89 & 227.05 & 223.85 & 230.33 & IHME \\ 
  Liberia & ALL & 85-89 & 242.19 & 222.41 & 262.82 & RW2 \\ 
  Liberia & ALL & 85-89 & 242.85 & 234.47 & 252.21 & UN \\ 
  Liberia & ALL & 90-94 & 220.60 & 216.62 & 224.48 & IHME \\ 
  Liberia & ALL & 90-94 & 254.77 & 238.92 & 271.47 & RW2 \\ 
  Liberia & ALL & 90-94 & 253.53 & 244.13 & 264.03 & UN \\ 
  Liberia & ALL & 95-99 & 195.34 & 192.46 & 198.27 & IHME \\ 
  Liberia & ALL & 95-99 & 214.35 & 200.55 & 228.66 & RW2 \\ 
  Liberia & ALL & 95-99 & 215.18 & 207.46 & 223.06 & UN \\ 
  Liberia & ALL & 00-04 & 146.95 & 144.51 & 149.52 & IHME \\ 
  Liberia & ALL & 00-04 & 157.99 & 147.43 & 169.26 & RW2 \\ 
  Liberia & ALL & 00-04 & 157.48 & 151.80 & 163.44 & UN \\ 
  Liberia & ALL & 05-09 & 106.28 & 104.22 & 108.45 & IHME \\ 
  Liberia & ALL & 05-09 & 107.67 & 96.48 & 119.83 & RW2 \\ 
  Liberia & ALL & 05-09 & 108.36 & 103.63 & 113.29 & UN \\ 
  Liberia & ALL & 10-14 & 83.09 & 80.24 & 85.99 & IHME \\ 
  Liberia & ALL & 10-14 & 81.53 & 71.56 & 92.67 & RW2 \\ 
  Liberia & ALL & 10-14 & 80.99 & 75.01 & 87.22 & UN \\ 
  Liberia & NORTH CENTRAL & 80-84 & 232.00 & 291.44 & 181.58 & HT-Direct \\ 
  Liberia & NORTH CENTRAL & 80-84 & 230.35 & 191.17 & 275.51 & RW2 \\ 
  Liberia & NORTH CENTRAL & 85-89 & 227.02 & 266.10 & 192.18 & HT-Direct \\ 
  Liberia & NORTH CENTRAL & 85-89 & 238.25 & 211.52 & 266.82 & RW2 \\ 
  Liberia & NORTH CENTRAL & 90-94 & 246.92 & 280.07 & 216.51 & HT-Direct \\ 
  Liberia & NORTH CENTRAL & 90-94 & 252.52 & 231.31 & 274.87 & RW2 \\ 
  Liberia & NORTH CENTRAL & 95-99 & 207.51 & 233.97 & 183.32 & HT-Direct \\ 
  Liberia & NORTH CENTRAL & 95-99 & 205.36 & 188.49 & 223.55 & RW2 \\ 
  Liberia & NORTH CENTRAL & 00-04 & 139.79 & 157.90 & 123.44 & HT-Direct \\ 
  Liberia & NORTH CENTRAL & 00-04 & 144.84 & 132.22 & 158.69 & RW2 \\ 
  Liberia & NORTH CENTRAL & 05-09 & 96.03 & 112.78 & 81.53 & HT-Direct \\ 
  Liberia & NORTH CENTRAL & 05-09 & 101.24 & 89.51 & 114.33 & RW2 \\ 
  Liberia & NORTH CENTRAL & 10-14 & 75.31 & 102.44 & 54.92 & HT-Direct \\ 
  Liberia & NORTH CENTRAL & 10-14 & 70.03 & 57.15 & 85.12 & RW2 \\ 
  Liberia & NORTH CENTRAL & 15-19 & 47.77 & 19.01 & 114.85 & RW2 \\ 
  Liberia & NORTH WESTERN & 80-84 & 213.26 & 300.87 & 145.84 & HT-Direct \\ 
  Liberia & NORTH WESTERN & 80-84 & 262.90 & 206.31 & 324.32 & RW2 \\ 
  Liberia & NORTH WESTERN & 85-89 & 273.61 & 323.02 & 229.21 & HT-Direct \\ 
  Liberia & NORTH WESTERN & 85-89 & 284.82 & 249.31 & 322.32 & RW2 \\ 
  Liberia & NORTH WESTERN & 90-94 & 328.52 & 380.27 & 280.61 & HT-Direct \\ 
  Liberia & NORTH WESTERN & 90-94 & 311.89 & 283.41 & 342.48 & RW2 \\ 
  Liberia & NORTH WESTERN & 95-99 & 271.35 & 308.30 & 237.31 & HT-Direct \\ 
  Liberia & NORTH WESTERN & 95-99 & 262.51 & 239.34 & 287.91 & RW2 \\ 
  Liberia & NORTH WESTERN & 00-04 & 172.95 & 199.87 & 148.99 & HT-Direct \\ 
  Liberia & NORTH WESTERN & 00-04 & 190.17 & 172.02 & 210.08 & RW2 \\ 
  Liberia & NORTH WESTERN & 05-09 & 125.63 & 151.83 & 103.40 & HT-Direct \\ 
  Liberia & NORTH WESTERN & 05-09 & 135.92 & 119.87 & 153.85 & RW2 \\ 
  Liberia & NORTH WESTERN & 10-14 & 110.16 & 134.53 & 89.74 & HT-Direct \\ 
  Liberia & NORTH WESTERN & 10-14 & 96.36 & 82.05 & 112.70 & RW2 \\ 
  Liberia & NORTH WESTERN & 15-19 & 67.29 & 27.42 & 154.38 & RW2 \\ 
  Liberia & SOUTH CENTRAL & 80-84 & 242.85 & 286.82 & 203.70 & HT-Direct \\ 
  Liberia & SOUTH CENTRAL & 80-84 & 245.77 & 211.24 & 284.01 & RW2 \\ 
  Liberia & SOUTH CENTRAL & 85-89 & 240.36 & 272.63 & 210.81 & HT-Direct \\ 
  Liberia & SOUTH CENTRAL & 85-89 & 253.02 & 229.26 & 278.46 & RW2 \\ 
  Liberia & SOUTH CENTRAL & 90-94 & 275.74 & 305.83 & 247.56 & HT-Direct \\ 
  Liberia & SOUTH CENTRAL & 90-94 & 267.05 & 247.24 & 287.64 & RW2 \\ 
  Liberia & SOUTH CENTRAL & 95-99 & 202.88 & 225.24 & 182.22 & HT-Direct \\ 
  Liberia & SOUTH CENTRAL & 95-99 & 217.32 & 201.04 & 234.54 & RW2 \\ 
  Liberia & SOUTH CENTRAL & 00-04 & 156.36 & 173.81 & 140.37 & HT-Direct \\ 
  Liberia & SOUTH CENTRAL & 00-04 & 155.26 & 142.59 & 169.02 & RW2 \\ 
  Liberia & SOUTH CENTRAL & 05-09 & 93.08 & 113.93 & 75.72 & HT-Direct \\ 
  Liberia & SOUTH CENTRAL & 05-09 & 110.59 & 97.25 & 125.51 & RW2 \\ 
  Liberia & SOUTH CENTRAL & 10-14 & 93.64 & 121.50 & 71.64 & HT-Direct \\ 
  Liberia & SOUTH CENTRAL & 10-14 & 78.53 & 65.24 & 94.38 & RW2 \\ 
  Liberia & SOUTH CENTRAL & 15-19 & 55.07 & 22.33 & 128.93 & RW2 \\ 
  Liberia & SOUTH EASTERN A & 80-84 & 184.28 & 270.36 & 121.06 & HT-Direct \\ 
  Liberia & SOUTH EASTERN A & 80-84 & 206.47 & 160.95 & 259.81 & RW2 \\ 
  Liberia & SOUTH EASTERN A & 85-89 & 204.74 & 248.18 & 167.21 & HT-Direct \\ 
  Liberia & SOUTH EASTERN A & 85-89 & 220.12 & 190.23 & 253.28 & RW2 \\ 
  Liberia & SOUTH EASTERN A & 90-94 & 258.51 & 305.74 & 216.30 & HT-Direct \\ 
  Liberia & SOUTH EASTERN A & 90-94 & 239.23 & 214.50 & 265.94 & RW2 \\ 
  Liberia & SOUTH EASTERN A & 95-99 & 185.08 & 221.54 & 153.43 & HT-Direct \\ 
  Liberia & SOUTH EASTERN A & 95-99 & 197.69 & 177.75 & 219.61 & RW2 \\ 
  Liberia & SOUTH EASTERN A & 00-04 & 139.61 & 163.62 & 118.62 & HT-Direct \\ 
  Liberia & SOUTH EASTERN A & 00-04 & 144.60 & 129.40 & 161.49 & RW2 \\ 
  Liberia & SOUTH EASTERN A & 05-09 & 111.05 & 135.66 & 90.44 & HT-Direct \\ 
  Liberia & SOUTH EASTERN A & 05-09 & 107.00 & 93.03 & 122.84 & RW2 \\ 
  Liberia & SOUTH EASTERN A & 10-14 & 78.28 & 105.72 & 57.51 & HT-Direct \\ 
  Liberia & SOUTH EASTERN A & 10-14 & 78.64 & 63.86 & 96.05 & RW2 \\ 
  Liberia & SOUTH EASTERN A & 15-19 & 56.95 & 22.77 & 134.56 & RW2 \\ 
  Liberia & SOUTH EASTERN B & 80-84 & 164.40 & 225.27 & 117.49 & HT-Direct \\ 
  Liberia & SOUTH EASTERN B & 80-84 & 174.08 & 135.86 & 222.32 & RW2 \\ 
  Liberia & SOUTH EASTERN B & 85-89 & 191.63 & 227.89 & 159.94 & HT-Direct \\ 
  Liberia & SOUTH EASTERN B & 85-89 & 194.06 & 169.35 & 221.75 & RW2 \\ 
  Liberia & SOUTH EASTERN B & 90-94 & 233.96 & 263.50 & 206.80 & HT-Direct \\ 
  Liberia & SOUTH EASTERN B & 90-94 & 219.46 & 199.96 & 240.57 & RW2 \\ 
  Liberia & SOUTH EASTERN B & 95-99 & 171.96 & 199.60 & 147.45 & HT-Direct \\ 
  Liberia & SOUTH EASTERN B & 95-99 & 190.45 & 170.01 & 211.17 & RW2 \\ 
  Liberia & SOUTH EASTERN B & 00-04 & 136.74 & 165.64 & 112.21 & HT-Direct \\ 
  Liberia & SOUTH EASTERN B & 00-04 & 154.88 & 136.91 & 173.20 & RW2 \\ 
  Liberia & SOUTH EASTERN B & 05-09 & 129.05 & 142.09 & 117.05 & HT-Direct \\ 
  Liberia & SOUTH EASTERN B & 05-09 & 136.46 & 124.59 & 149.16 & RW2 \\ 
  Liberia & SOUTH EASTERN B & 10-14 & 149.05 & 185.19 & 118.93 & HT-Direct \\ 
  Liberia & SOUTH EASTERN B & 10-14 & 124.34 & 105.06 & 147.78 & RW2 \\ 
  Liberia & SOUTH EASTERN B & 15-19 & 113.91 & 46.90 & 251.44 & RW2 \\ 
  Madagascar & ALL & 80-84 & 165.73 & 161.52 & 170.18 & IHME \\ 
  Madagascar & ALL & 80-84 & 180.00 & 170.61 & 189.79 & RW2 \\ 
  Madagascar & ALL & 80-84 & 180.07 & 174.50 & 186.07 & UN \\ 
  Madagascar & ALL & 85-89 & 161.67 & 157.73 & 165.74 & IHME \\ 
  Madagascar & ALL & 85-89 & 175.66 & 167.31 & 184.31 & RW2 \\ 
  Madagascar & ALL & 85-89 & 175.51 & 170.54 & 181.16 & UN \\ 
  Madagascar & ALL & 90-94 & 139.44 & 135.98 & 142.88 & IHME \\ 
  Madagascar & ALL & 90-94 & 151.48 & 143.57 & 159.73 & RW2 \\ 
  Madagascar & ALL & 90-94 & 151.62 & 146.90 & 156.19 & UN \\ 
  Madagascar & ALL & 95-99 & 119.85 & 116.48 & 123.52 & IHME \\ 
  Madagascar & ALL & 95-99 & 127.32 & 119.66 & 135.35 & RW2 \\ 
  Madagascar & ALL & 95-99 & 127.22 & 123.04 & 131.47 & UN \\ 
  Madagascar & ALL & 00-04 & 98.55 & 95.14 & 102.11 & IHME \\ 
  Madagascar & ALL & 00-04 & 97.00 & 90.08 & 104.43 & RW2 \\ 
  Madagascar & ALL & 00-04 & 97.07 & 92.82 & 101.48 & UN \\ 
  Madagascar & ALL & 05-09 & 78.84 & 75.52 & 82.79 & IHME \\ 
  Madagascar & ALL & 05-09 & 71.38 & 64.15 & 79.35 & RW2 \\ 
  Madagascar & ALL & 05-09 & 71.36 & 66.41 & 76.61 & UN \\ 
  Madagascar & ALL & 10-14 & 71.77 & 66.33 & 78.09 & IHME \\ 
  Madagascar & ALL & 10-14 & 51.97 & 18.99 & 133.66 & RW2 \\ 
  Madagascar & ALL & 10-14 & 55.71 & 48.63 & 64.70 & UN \\ 
  Madagascar & ANTANANARIVO & 80-84 & 141.96 & 158.62 & 126.78 & HT-Direct \\ 
  Madagascar & ANTANANARIVO & 80-84 & 140.62 & 126.67 & 155.59 & RW2 \\ 
  Madagascar & ANTANANARIVO & 85-89 & 156.52 & 174.76 & 139.86 & HT-Direct \\ 
  Madagascar & ANTANANARIVO & 85-89 & 143.21 & 130.49 & 157.71 & RW2 \\ 
  Madagascar & ANTANANARIVO & 90-94 & 113.62 & 129.92 & 99.14 & HT-Direct \\ 
  Madagascar & ANTANANARIVO & 90-94 & 120.51 & 108.82 & 133.49 & RW2 \\ 
  Madagascar & ANTANANARIVO & 95-99 & 88.48 & 102.74 & 76.03 & HT-Direct \\ 
  Madagascar & ANTANANARIVO & 95-99 & 92.06 & 81.19 & 103.57 & RW2 \\ 
  Madagascar & ANTANANARIVO & 00-04 & 61.07 & 72.39 & 51.42 & HT-Direct \\ 
  Madagascar & ANTANANARIVO & 00-04 & 74.98 & 64.97 & 85.89 & RW2 \\ 
  Madagascar & ANTANANARIVO & 05-09 & 65.24 & 81.38 & 52.13 & HT-Direct \\ 
  Madagascar & ANTANANARIVO & 05-09 & 59.98 & 48.16 & 75.34 & RW2 \\ 
  Madagascar & ANTANANARIVO & 10-14 & 47.35 & 17.26 & 125.17 & RW2 \\ 
  Madagascar & ANTANANARIVO & 15-19 & 37.36 & 3.13 & 320.63 & RW2 \\ 
  Madagascar & ANTSIRANANA & 80-84 & 151.14 & 176.79 & 128.63 & HT-Direct \\ 
  Madagascar & ANTSIRANANA & 80-84 & 142.51 & 123.14 & 164.36 & RW2 \\ 
  Madagascar & ANTSIRANANA & 85-89 & 134.70 & 156.36 & 115.63 & HT-Direct \\ 
  Madagascar & ANTSIRANANA & 85-89 & 137.92 & 122.33 & 154.87 & RW2 \\ 
  Madagascar & ANTSIRANANA & 90-94 & 134.46 & 158.92 & 113.26 & HT-Direct \\ 
  Madagascar & ANTSIRANANA & 90-94 & 129.46 & 113.44 & 147.54 & RW2 \\ 
  Madagascar & ANTSIRANANA & 95-99 & 95.35 & 115.91 & 78.12 & HT-Direct \\ 
  Madagascar & ANTSIRANANA & 95-99 & 108.69 & 93.26 & 126.17 & RW2 \\ 
  Madagascar & ANTSIRANANA & 00-04 & 98.12 & 126.83 & 75.35 & HT-Direct \\ 
  Madagascar & ANTSIRANANA & 00-04 & 91.89 & 74.81 & 112.86 & RW2 \\ 
  Madagascar & ANTSIRANANA & 05-09 & 51.31 & 84.34 & 30.79 & HT-Direct \\ 
  Madagascar & ANTSIRANANA & 05-09 & 64.33 & 42.73 & 95.70 & RW2 \\ 
  Madagascar & ANTSIRANANA & 10-14 & 41.33 & 12.09 & 129.80 & RW2 \\ 
  Madagascar & ANTSIRANANA & 15-19 & 26.11 & 1.69 & 281.80 & RW2 \\ 
  Madagascar & FIANARANTSOA & 80-84 & 250.64 & 278.62 & 224.60 & HT-Direct \\ 
  Madagascar & FIANARANTSOA & 80-84 & 234.90 & 212.39 & 259.50 & RW2 \\ 
  Madagascar & FIANARANTSOA & 85-89 & 214.89 & 236.24 & 194.97 & HT-Direct \\ 
  Madagascar & FIANARANTSOA & 85-89 & 215.22 & 198.66 & 232.31 & RW2 \\ 
  Madagascar & FIANARANTSOA & 90-94 & 184.59 & 202.88 & 167.59 & HT-Direct \\ 
  Madagascar & FIANARANTSOA & 90-94 & 194.09 & 178.69 & 210.00 & RW2 \\ 
  Madagascar & FIANARANTSOA & 95-99 & 168.89 & 188.07 & 151.31 & HT-Direct \\ 
  Madagascar & FIANARANTSOA & 95-99 & 172.13 & 157.34 & 188.49 & RW2 \\ 
  Madagascar & FIANARANTSOA & 00-04 & 139.18 & 155.72 & 124.14 & HT-Direct \\ 
  Madagascar & FIANARANTSOA & 00-04 & 144.43 & 130.68 & 159.77 & RW2 \\ 
  Madagascar & FIANARANTSOA & 05-09 & 89.04 & 102.63 & 77.09 & HT-Direct \\ 
  Madagascar & FIANARANTSOA & 05-09 & 97.96 & 84.68 & 112.99 & RW2 \\ 
  Madagascar & FIANARANTSOA & 10-14 & 60.10 & 22.99 & 144.54 & RW2 \\ 
  Madagascar & FIANARANTSOA & 15-19 & 35.53 & 3.08 & 292.81 & RW2 \\ 
  Madagascar & MAHAJANGA & 80-84 & 250.23 & 278.59 & 223.87 & HT-Direct \\ 
  Madagascar & MAHAJANGA & 80-84 & 233.52 & 210.75 & 258.38 & RW2 \\ 
  Madagascar & MAHAJANGA & 85-89 & 188.21 & 209.77 & 168.40 & HT-Direct \\ 
  Madagascar & MAHAJANGA & 85-89 & 189.91 & 173.92 & 206.31 & RW2 \\ 
  Madagascar & MAHAJANGA & 90-94 & 146.44 & 162.32 & 131.86 & HT-Direct \\ 
  Madagascar & MAHAJANGA & 90-94 & 152.58 & 139.45 & 166.50 & RW2 \\ 
  Madagascar & MAHAJANGA & 95-99 & 112.95 & 131.23 & 96.93 & HT-Direct \\ 
  Madagascar & MAHAJANGA & 95-99 & 121.72 & 108.61 & 136.06 & RW2 \\ 
  Madagascar & MAHAJANGA & 00-04 & 106.10 & 126.68 & 88.53 & HT-Direct \\ 
  Madagascar & MAHAJANGA & 00-04 & 99.07 & 86.26 & 114.32 & RW2 \\ 
  Madagascar & MAHAJANGA & 05-09 & 57.51 & 71.56 & 46.08 & HT-Direct \\ 
  Madagascar & MAHAJANGA & 05-09 & 65.98 & 53.45 & 81.19 & RW2 \\ 
  Madagascar & MAHAJANGA & 10-14 & 39.96 & 14.63 & 103.01 & RW2 \\ 
  Madagascar & MAHAJANGA & 15-19 & 24.02 & 2.02 & 216.61 & RW2 \\ 
  Madagascar & TOAMASINA & 80-84 & 176.87 & 203.15 & 153.35 & HT-Direct \\ 
  Madagascar & TOAMASINA & 80-84 & 173.46 & 152.35 & 196.62 & RW2 \\ 
  Madagascar & TOAMASINA & 85-89 & 185.37 & 206.17 & 166.22 & HT-Direct \\ 
  Madagascar & TOAMASINA & 85-89 & 184.68 & 169.63 & 201.04 & RW2 \\ 
  Madagascar & TOAMASINA & 90-94 & 175.92 & 196.20 & 157.32 & HT-Direct \\ 
  Madagascar & TOAMASINA & 90-94 & 170.81 & 155.19 & 187.99 & RW2 \\ 
  Madagascar & TOAMASINA & 95-99 & 121.03 & 137.76 & 106.08 & HT-Direct \\ 
  Madagascar & TOAMASINA & 95-99 & 126.56 & 113.75 & 140.57 & RW2 \\ 
  Madagascar & TOAMASINA & 00-04 & 72.86 & 89.03 & 59.44 & HT-Direct \\ 
  Madagascar & TOAMASINA & 00-04 & 85.90 & 73.16 & 100.29 & RW2 \\ 
  Madagascar & TOAMASINA & 05-09 & 48.62 & 67.71 & 34.71 & HT-Direct \\ 
  Madagascar & TOAMASINA & 05-09 & 50.33 & 37.58 & 66.58 & RW2 \\ 
  Madagascar & TOAMASINA & 10-14 & 27.76 & 9.50 & 76.87 & RW2 \\ 
  Madagascar & TOAMASINA & 15-19 & 15.22 & 1.24 & 153.91 & RW2 \\ 
  Madagascar & TOLIARY & 80-84 & 181.89 & 212.46 & 154.86 & HT-Direct \\ 
  Madagascar & TOLIARY & 80-84 & 175.79 & 152.72 & 201.63 & RW2 \\ 
  Madagascar & TOLIARY & 85-89 & 180.74 & 201.71 & 161.52 & HT-Direct \\ 
  Madagascar & TOLIARY & 85-89 & 176.15 & 160.94 & 192.55 & RW2 \\ 
  Madagascar & TOLIARY & 90-94 & 153.11 & 172.19 & 135.81 & HT-Direct \\ 
  Madagascar & TOLIARY & 90-94 & 162.98 & 148.07 & 178.96 & RW2 \\ 
  Madagascar & TOLIARY & 95-99 & 142.51 & 162.21 & 124.85 & HT-Direct \\ 
  Madagascar & TOLIARY & 95-99 & 142.32 & 128.18 & 158.18 & RW2 \\ 
  Madagascar & TOLIARY & 00-04 & 107.44 & 121.35 & 94.96 & HT-Direct \\ 
  Madagascar & TOLIARY & 00-04 & 113.11 & 101.43 & 126.13 & RW2 \\ 
  Madagascar & TOLIARY & 05-09 & 61.91 & 75.59 & 50.57 & HT-Direct \\ 
  Madagascar & TOLIARY & 05-09 & 70.46 & 57.76 & 85.25 & RW2 \\ 
  Madagascar & TOLIARY & 10-14 & 39.80 & 14.42 & 102.53 & RW2 \\ 
  Madagascar & TOLIARY & 15-19 & 22.26 & 1.82 & 204.12 & RW2 \\ 
  Malawi & ALL & 80-84 & 249.54 & 244.17 & 254.87 & IHME \\ 
  Malawi & ALL & 80-84 & 247.37 & 236.82 & 258.23 & RW2 \\ 
  Malawi & ALL & 80-84 & 247.38 & 240.91 & 254.48 & UN \\ 
  Malawi & ALL & 85-89 & 235.41 & 230.64 & 240.45 & IHME \\ 
  Malawi & ALL & 85-89 & 251.17 & 242.98 & 259.51 & RW2 \\ 
  Malawi & ALL & 85-89 & 251.19 & 244.39 & 258.13 & UN \\ 
  Malawi & ALL & 90-94 & 212.05 & 207.95 & 216.48 & IHME \\ 
  Malawi & ALL & 90-94 & 227.16 & 219.79 & 234.70 & RW2 \\ 
  Malawi & ALL & 90-94 & 227.09 & 221.44 & 233.44 & UN \\ 
  Malawi & ALL & 95-99 & 184.45 & 180.19 & 188.56 & IHME \\ 
  Malawi & ALL & 95-99 & 197.02 & 190.60 & 203.56 & RW2 \\ 
  Malawi & ALL & 95-99 & 197.10 & 192.43 & 202.12 & UN \\ 
  Malawi & ALL & 00-04 & 147.19 & 143.21 & 151.06 & IHME \\ 
  Malawi & ALL & 00-04 & 148.13 & 142.64 & 153.83 & RW2 \\ 
  Malawi & ALL & 00-04 & 148.18 & 143.93 & 152.99 & UN \\ 
  Malawi & ALL & 05-09 & 112.49 & 109.06 & 115.99 & IHME \\ 
  Malawi & ALL & 05-09 & 105.34 & 100.66 & 110.22 & RW2 \\ 
  Malawi & ALL & 05-09 & 105.24 & 101.01 & 109.83 & UN \\ 
  Malawi & ALL & 10-14 & 91.52 & 87.03 & 96.06 & IHME \\ 
  Malawi & ALL & 10-14 & 79.49 & 72.83 & 86.66 & RW2 \\ 
  Malawi & ALL & 10-14 & 79.65 & 72.89 & 87.10 & UN \\ 
  Malawi & CENTRAL REGION & 80-84 & 271.78 & 290.18 & 254.14 & HT-Direct \\ 
  Malawi & CENTRAL REGION & 80-84 & 278.09 & 261.82 & 295.12 & RW2 \\ 
  Malawi & CENTRAL REGION & 85-89 & 268.83 & 281.51 & 256.51 & HT-Direct \\ 
  Malawi & CENTRAL REGION & 85-89 & 273.04 & 262.22 & 284.25 & RW2 \\ 
  Malawi & CENTRAL REGION & 90-94 & 223.18 & 234.62 & 212.15 & HT-Direct \\ 
  Malawi & CENTRAL REGION & 90-94 & 239.50 & 229.67 & 249.57 & RW2 \\ 
  Malawi & CENTRAL REGION & 95-99 & 191.53 & 201.24 & 182.17 & HT-Direct \\ 
  Malawi & CENTRAL REGION & 95-99 & 202.53 & 193.94 & 211.21 & RW2 \\ 
  Malawi & CENTRAL REGION & 00-04 & 149.62 & 158.84 & 140.84 & HT-Direct \\ 
  Malawi & CENTRAL REGION & 00-04 & 152.67 & 145.41 & 160.30 & RW2 \\ 
  Malawi & CENTRAL REGION & 05-09 & 104.93 & 112.68 & 97.66 & HT-Direct \\ 
  Malawi & CENTRAL REGION & 05-09 & 110.62 & 104.09 & 117.53 & RW2 \\ 
  Malawi & CENTRAL REGION & 10-14 & 76.30 & 85.29 & 68.19 & HT-Direct \\ 
  Malawi & CENTRAL REGION & 10-14 & 84.62 & 75.72 & 94.60 & RW2 \\ 
  Malawi & CENTRAL REGION & 15-19 & 65.68 & 30.19 & 137.92 & RW2 \\ 
  Malawi & NORTHERN REGION & 80-84 & 193.59 & 216.41 & 172.65 & HT-Direct \\ 
  Malawi & NORTHERN REGION & 80-84 & 199.91 & 180.89 & 220.10 & RW2 \\ 
  Malawi & NORTHERN REGION & 85-89 & 196.25 & 212.94 & 180.57 & HT-Direct \\ 
  Malawi & NORTHERN REGION & 85-89 & 200.85 & 188.41 & 214.20 & RW2 \\ 
  Malawi & NORTHERN REGION & 90-94 & 171.04 & 186.48 & 156.63 & HT-Direct \\ 
  Malawi & NORTHERN REGION & 90-94 & 177.60 & 166.31 & 189.34 & RW2 \\ 
  Malawi & NORTHERN REGION & 95-99 & 135.85 & 149.65 & 123.13 & HT-Direct \\ 
  Malawi & NORTHERN REGION & 95-99 & 150.60 & 140.24 & 161.30 & RW2 \\ 
  Malawi & NORTHERN REGION & 00-04 & 116.03 & 129.02 & 104.20 & HT-Direct \\ 
  Malawi & NORTHERN REGION & 00-04 & 114.50 & 105.97 & 123.97 & RW2 \\ 
  Malawi & NORTHERN REGION & 05-09 & 79.94 & 92.20 & 69.18 & HT-Direct \\ 
  Malawi & NORTHERN REGION & 05-09 & 82.78 & 74.87 & 91.48 & RW2 \\ 
  Malawi & NORTHERN REGION & 10-14 & 52.18 & 61.64 & 44.11 & HT-Direct \\ 
  Malawi & NORTHERN REGION & 10-14 & 62.47 & 53.01 & 73.25 & RW2 \\ 
  Malawi & NORTHERN REGION & 15-19 & 47.77 & 21.03 & 104.96 & RW2 \\ 
  Malawi & SOUTHERN REGION & 80-84 & 229.31 & 243.00 & 216.17 & HT-Direct \\ 
  Malawi & SOUTHERN REGION & 80-84 & 232.05 & 219.40 & 245.33 & RW2 \\ 
  Malawi & SOUTHERN REGION & 85-89 & 229.98 & 241.28 & 219.05 & HT-Direct \\ 
  Malawi & SOUTHERN REGION & 85-89 & 243.97 & 233.64 & 254.33 & RW2 \\ 
  Malawi & SOUTHERN REGION & 90-94 & 223.45 & 233.16 & 214.04 & HT-Direct \\ 
  Malawi & SOUTHERN REGION & 90-94 & 230.84 & 222.01 & 240.01 & RW2 \\ 
  Malawi & SOUTHERN REGION & 95-99 & 193.89 & 201.82 & 186.20 & HT-Direct \\ 
  Malawi & SOUTHERN REGION & 95-99 & 201.87 & 194.41 & 209.67 & RW2 \\ 
  Malawi & SOUTHERN REGION & 00-04 & 142.00 & 149.15 & 135.14 & HT-Direct \\ 
  Malawi & SOUTHERN REGION & 00-04 & 149.94 & 143.53 & 156.61 & RW2 \\ 
  Malawi & SOUTHERN REGION & 05-09 & 103.86 & 110.13 & 97.91 & HT-Direct \\ 
  Malawi & SOUTHERN REGION & 05-09 & 106.17 & 100.59 & 112.00 & RW2 \\ 
  Malawi & SOUTHERN REGION & 10-14 & 65.79 & 73.50 & 58.83 & HT-Direct \\ 
  Malawi & SOUTHERN REGION & 10-14 & 78.05 & 69.58 & 87.34 & RW2 \\ 
  Malawi & SOUTHERN REGION & 15-19 & 57.78 & 26.21 & 123.23 & RW2 \\ 
  Mali & ALL & 80-84 & 264.91 & 261.52 & 267.92 & IHME \\ 
  Mali & ALL & 80-84 & 308.58 & 297.83 & 319.53 & RW2 \\ 
  Mali & ALL & 80-84 & 308.66 & 299.76 & 318.13 & UN \\ 
  Mali & ALL & 85-89 & 240.25 & 237.66 & 243.28 & IHME \\ 
  Mali & ALL & 85-89 & 271.57 & 263.33 & 279.94 & RW2 \\ 
  Mali & ALL & 85-89 & 271.40 & 263.51 & 279.20 & UN \\ 
  Mali & ALL & 90-94 & 222.21 & 219.55 & 225.14 & IHME \\ 
  Mali & ALL & 90-94 & 247.13 & 240.27 & 254.12 & RW2 \\ 
  Mali & ALL & 90-94 & 247.33 & 239.50 & 254.41 & UN \\ 
  Mali & ALL & 95-99 & 205.62 & 202.49 & 208.44 & IHME \\ 
  Mali & ALL & 95-99 & 234.72 & 226.41 & 243.18 & RW2 \\ 
  Mali & ALL & 95-99 & 234.26 & 227.42 & 242.23 & UN \\ 
  Mali & ALL & 00-04 & 182.24 & 179.26 & 185.52 & IHME \\ 
  Mali & ALL & 00-04 & 200.55 & 192.30 & 209.11 & RW2 \\ 
  Mali & ALL & 00-04 & 200.96 & 193.90 & 208.55 & UN \\ 
  Mali & ALL & 05-09 & 158.74 & 155.00 & 162.52 & IHME \\ 
  Mali & ALL & 05-09 & 155.74 & 142.41 & 170.07 & RW2 \\ 
  Mali & ALL & 05-09 & 155.29 & 143.82 & 166.77 & UN \\ 
  Mali & ALL & 10-14 & 140.35 & 134.80 & 145.97 & IHME \\ 
  Mali & ALL & 10-14 & 116.87 & 45.58 & 265.70 & RW2 \\ 
  Mali & ALL & 10-14 & 128.09 & 110.80 & 148.02 & UN \\ 
  Mali & BAMAKO & 80-84 & 173.88 & 192.18 & 156.97 & HT-Direct \\ 
  Mali & BAMAKO & 80-84 & 181.48 & 165.17 & 198.80 & RW2 \\ 
  Mali & BAMAKO & 85-89 & 148.58 & 166.26 & 132.47 & HT-Direct \\ 
  Mali & BAMAKO & 85-89 & 154.74 & 143.71 & 166.51 & RW2 \\ 
  Mali & BAMAKO & 90-94 & 144.98 & 161.06 & 130.25 & HT-Direct \\ 
  Mali & BAMAKO & 90-94 & 138.34 & 128.64 & 148.89 & RW2 \\ 
  Mali & BAMAKO & 95-99 & 130.87 & 149.49 & 114.26 & HT-Direct \\ 
  Mali & BAMAKO & 95-99 & 130.56 & 120.49 & 141.40 & RW2 \\ 
  Mali & BAMAKO & 00-04 & 109.22 & 124.45 & 95.66 & HT-Direct \\ 
  Mali & BAMAKO & 00-04 & 106.92 & 96.40 & 118.53 & RW2 \\ 
  Mali & BAMAKO & 05-09 & 77.40 & 113.98 & 51.88 & HT-Direct \\ 
  Mali & BAMAKO & 05-09 & 80.49 & 64.93 & 98.49 & RW2 \\ 
  Mali & BAMAKO & 10-14 & 59.45 & 23.40 & 141.68 & RW2 \\ 
  Mali & BAMAKO & 15-19 & 43.63 & 4.43 & 320.46 & RW2 \\ 
  Mali & KAYES, KOULIKORO & 80-84 & 294.57 & 311.15 & 278.52 & HT-Direct \\ 
  Mali & KAYES, KOULIKORO & 80-84 & 302.71 & 287.34 & 318.98 & RW2 \\ 
  Mali & KAYES, KOULIKORO & 85-89 & 257.13 & 271.77 & 243.01 & HT-Direct \\ 
  Mali & KAYES, KOULIKORO & 85-89 & 261.92 & 250.73 & 273.23 & RW2 \\ 
  Mali & KAYES, KOULIKORO & 90-94 & 232.98 & 244.69 & 221.66 & HT-Direct \\ 
  Mali & KAYES, KOULIKORO & 90-94 & 236.60 & 227.30 & 246.00 & RW2 \\ 
  Mali & KAYES, KOULIKORO & 95-99 & 238.16 & 250.52 & 226.22 & HT-Direct \\ 
  Mali & KAYES, KOULIKORO & 95-99 & 227.74 & 218.69 & 237.07 & RW2 \\ 
  Mali & KAYES, KOULIKORO & 00-04 & 197.55 & 212.32 & 183.58 & HT-Direct \\ 
  Mali & KAYES, KOULIKORO & 00-04 & 194.15 & 183.07 & 205.95 & RW2 \\ 
  Mali & KAYES, KOULIKORO & 05-09 & 165.21 & 201.03 & 134.70 & HT-Direct \\ 
  Mali & KAYES, KOULIKORO & 05-09 & 153.29 & 134.58 & 174.20 & RW2 \\ 
  Mali & KAYES, KOULIKORO & 10-14 & 118.41 & 50.67 & 252.50 & RW2 \\ 
  Mali & KAYES, KOULIKORO & 15-19 & 90.55 & 10.11 & 497.98 & RW2 \\ 
  Mali & MOPTI, TOMBOUCTOU, GAO, KIDAL & 80-84 & 383.04 & 406.82 & 359.81 & HT-Direct \\ 
  Mali & MOPTI, TOMBOUCTOU, GAO, KIDAL & 80-84 & 406.69 & 383.81 & 429.34 & RW2 \\ 
  Mali & MOPTI, TOMBOUCTOU, GAO, KIDAL & 85-89 & 351.97 & 372.29 & 332.18 & HT-Direct \\ 
  Mali & MOPTI, TOMBOUCTOU, GAO, KIDAL & 85-89 & 348.51 & 333.65 & 363.77 & RW2 \\ 
  Mali & MOPTI, TOMBOUCTOU, GAO, KIDAL & 90-94 & 296.41 & 311.90 & 281.37 & HT-Direct \\ 
  Mali & MOPTI, TOMBOUCTOU, GAO, KIDAL & 90-94 & 298.62 & 286.69 & 311.20 & RW2 \\ 
  Mali & MOPTI, TOMBOUCTOU, GAO, KIDAL & 95-99 & 279.53 & 297.16 & 262.56 & HT-Direct \\ 
  Mali & MOPTI, TOMBOUCTOU, GAO, KIDAL & 95-99 & 262.82 & 250.97 & 274.99 & RW2 \\ 
  Mali & MOPTI, TOMBOUCTOU, GAO, KIDAL & 00-04 & 191.87 & 207.24 & 177.38 & HT-Direct \\ 
  Mali & MOPTI, TOMBOUCTOU, GAO, KIDAL & 00-04 & 201.70 & 189.43 & 214.48 & RW2 \\ 
  Mali & MOPTI, TOMBOUCTOU, GAO, KIDAL & 05-09 & 173.69 & 207.09 & 144.70 & HT-Direct \\ 
  Mali & MOPTI, TOMBOUCTOU, GAO, KIDAL & 05-09 & 145.11 & 127.56 & 165.12 & RW2 \\ 
  Mali & MOPTI, TOMBOUCTOU, GAO, KIDAL & 10-14 & 101.98 & 43.37 & 222.30 & RW2 \\ 
  Mali & MOPTI, TOMBOUCTOU, GAO, KIDAL & 15-19 & 70.54 & 7.48 & 431.61 & RW2 \\ 
  Mali & SIKASSO, SEGOU & 80-84 & 279.83 & 295.21 & 264.96 & HT-Direct \\ 
  Mali & SIKASSO, SEGOU & 80-84 & 290.18 & 275.81 & 304.97 & RW2 \\ 
  Mali & SIKASSO, SEGOU & 85-89 & 264.39 & 277.07 & 252.09 & HT-Direct \\ 
  Mali & SIKASSO, SEGOU & 85-89 & 266.79 & 256.76 & 277.15 & RW2 \\ 
  Mali & SIKASSO, SEGOU & 90-94 & 249.90 & 261.24 & 238.90 & HT-Direct \\ 
  Mali & SIKASSO, SEGOU & 90-94 & 253.64 & 244.43 & 262.94 & RW2 \\ 
  Mali & SIKASSO, SEGOU & 95-99 & 265.37 & 283.18 & 248.29 & HT-Direct \\ 
  Mali & SIKASSO, SEGOU & 95-99 & 254.85 & 243.68 & 266.42 & RW2 \\ 
  Mali & SIKASSO, SEGOU & 00-04 & 234.67 & 251.49 & 218.65 & HT-Direct \\ 
  Mali & SIKASSO, SEGOU & 00-04 & 227.15 & 214.84 & 240.15 & RW2 \\ 
  Mali & SIKASSO, SEGOU & 05-09 & 197.36 & 224.43 & 172.82 & HT-Direct \\ 
  Mali & SIKASSO, SEGOU & 05-09 & 189.10 & 171.01 & 208.57 & RW2 \\ 
  Mali & SIKASSO, SEGOU & 10-14 & 155.15 & 69.90 & 313.00 & RW2 \\ 
  Mali & SIKASSO, SEGOU & 15-19 & 125.41 & 14.71 & 580.00 & RW2 \\ 
  Morocco & ALL & 80-84 & 110.92 & 109.27 & 112.75 & IHME \\ 
  Morocco & ALL & 80-84 & 120.67 & 113.75 & 127.96 & RW2 \\ 
  Morocco & ALL & 80-84 & 120.68 & 117.85 & 123.93 & UN \\ 
  Morocco & ALL & 85-89 & 83.10 & 81.61 & 84.46 & IHME \\ 
  Morocco & ALL & 85-89 & 93.08 & 86.59 & 99.98 & RW2 \\ 
  Morocco & ALL & 85-89 & 93.06 & 90.49 & 95.68 & UN \\ 
  Morocco & ALL & 90-94 & 63.58 & 62.39 & 64.93 & IHME \\ 
  Morocco & ALL & 90-94 & 73.04 & 66.93 & 79.66 & RW2 \\ 
  Morocco & ALL & 90-94 & 73.15 & 70.89 & 75.36 & UN \\ 
  Morocco & ALL & 95-99 & 50.42 & 49.10 & 51.85 & IHME \\ 
  Morocco & ALL & 95-99 & 57.95 & 51.95 & 64.54 & RW2 \\ 
  Morocco & ALL & 95-99 & 57.73 & 55.73 & 59.79 & UN \\ 
  Morocco & ALL & 00-04 & 40.28 & 38.49 & 42.03 & IHME \\ 
  Morocco & ALL & 00-04 & 45.74 & 39.40 & 53.09 & RW2 \\ 
  Morocco & ALL & 00-04 & 45.88 & 44.01 & 47.82 & UN \\ 
  Morocco & ALL & 05-09 & 32.25 & 30.20 & 34.41 & IHME \\ 
  Morocco & ALL & 05-09 & 35.82 & 10.21 & 117.54 & RW2 \\ 
  Morocco & ALL & 05-09 & 37.16 & 35.23 & 39.25 & UN \\ 
  Morocco & ALL & 10-14 & 25.87 & 23.78 & 28.26 & IHME \\ 
  Morocco & ALL & 10-14 & 27.81 & 1.22 & 388.29 & RW2 \\ 
  Morocco & ALL & 10-14 & 30.51 & 27.87 & 33.48 & UN \\ 
  Morocco & CENTRE & 80-84 & 85.05 & 95.98 & 75.27 & HT-Direct \\ 
  Morocco & CENTRE & 80-84 & 84.67 & 75.28 & 95.27 & RW2 \\ 
  Morocco & CENTRE & 85-89 & 56.63 & 65.84 & 48.64 & HT-Direct \\ 
  Morocco & CENTRE & 85-89 & 65.41 & 58.52 & 72.74 & RW2 \\ 
  Morocco & CENTRE & 90-94 & 49.56 & 60.37 & 40.61 & HT-Direct \\ 
  Morocco & CENTRE & 90-94 & 52.48 & 45.91 & 59.71 & RW2 \\ 
  Morocco & CENTRE & 95-99 & 45.35 & 58.21 & 35.22 & HT-Direct \\ 
  Morocco & CENTRE & 95-99 & 44.62 & 37.74 & 52.62 & RW2 \\ 
  Morocco & CENTRE & 00-04 & 39.73 & 53.39 & 29.46 & HT-Direct \\ 
  Morocco & CENTRE & 00-04 & 37.71 & 29.85 & 47.84 & RW2 \\ 
  Morocco & CENTRE & 05-09 & 31.80 & 10.52 & 91.14 & RW2 \\ 
  Morocco & CENTRE & 10-14 & 26.66 & 1.78 & 297.88 & RW2 \\ 
  Morocco & CENTRE & 15-19 & 22.69 & 0.19 & 728.81 & RW2 \\ 
  Morocco & CENTRE-NORD & 80-84 & 121.85 & 140.94 & 105.02 & HT-Direct \\ 
  Morocco & CENTRE-NORD & 80-84 & 131.00 & 114.04 & 150.03 & RW2 \\ 
  Morocco & CENTRE-NORD & 85-89 & 100.57 & 117.58 & 85.78 & HT-Direct \\ 
  Morocco & CENTRE-NORD & 85-89 & 104.53 & 93.39 & 117.13 & RW2 \\ 
  Morocco & CENTRE-NORD & 90-94 & 86.48 & 106.10 & 70.20 & HT-Direct \\ 
  Morocco & CENTRE-NORD & 90-94 & 82.99 & 72.41 & 95.24 & RW2 \\ 
  Morocco & CENTRE-NORD & 95-99 & 67.16 & 89.49 & 50.10 & HT-Direct \\ 
  Morocco & CENTRE-NORD & 95-99 & 66.87 & 55.71 & 80.17 & RW2 \\ 
  Morocco & CENTRE-NORD & 00-04 & 47.07 & 66.85 & 32.94 & HT-Direct \\ 
  Morocco & CENTRE-NORD & 00-04 & 52.17 & 39.68 & 68.06 & RW2 \\ 
  Morocco & CENTRE-NORD & 05-09 & 39.99 & 13.07 & 114.55 & RW2 \\ 
  Morocco & CENTRE-NORD & 10-14 & 30.55 & 2.01 & 320.97 & RW2 \\ 
  Morocco & CENTRE-NORD & 15-19 & 22.85 & 0.19 & 734.55 & RW2 \\ 
  Morocco & CENTRE-SUD & 80-84 & 95.88 & 114.94 & 79.70 & HT-Direct \\ 
  Morocco & CENTRE-SUD & 80-84 & 108.39 & 91.22 & 128.48 & RW2 \\ 
  Morocco & CENTRE-SUD & 85-89 & 106.44 & 134.79 & 83.47 & HT-Direct \\ 
  Morocco & CENTRE-SUD & 85-89 & 91.78 & 79.64 & 105.88 & RW2 \\ 
  Morocco & CENTRE-SUD & 90-94 & 75.23 & 97.13 & 57.95 & HT-Direct \\ 
  Morocco & CENTRE-SUD & 90-94 & 77.93 & 66.34 & 91.61 & RW2 \\ 
  Morocco & CENTRE-SUD & 95-99 & 64.17 & 94.64 & 43.05 & HT-Direct \\ 
  Morocco & CENTRE-SUD & 95-99 & 68.04 & 54.50 & 84.70 & RW2 \\ 
  Morocco & CENTRE-SUD & 00-04 & 58.27 & 92.24 & 36.31 & HT-Direct \\ 
  Morocco & CENTRE-SUD & 00-04 & 57.98 & 42.22 & 79.31 & RW2 \\ 
  Morocco & CENTRE-SUD & 05-09 & 48.60 & 15.72 & 138.48 & RW2 \\ 
  Morocco & CENTRE-SUD & 10-14 & 40.60 & 2.67 & 397.51 & RW2 \\ 
  Morocco & CENTRE-SUD & 15-19 & 34.38 & 0.28 & 800.25 & RW2 \\ 
  Morocco & NORD-OUEST & 80-84 & 115.10 & 127.84 & 103.49 & HT-Direct \\ 
  Morocco & NORD-OUEST & 80-84 & 123.57 & 111.44 & 136.72 & RW2 \\ 
  Morocco & NORD-OUEST & 85-89 & 97.96 & 111.66 & 85.77 & HT-Direct \\ 
  Morocco & NORD-OUEST & 85-89 & 94.30 & 85.47 & 104.19 & RW2 \\ 
  Morocco & NORD-OUEST & 90-94 & 61.40 & 74.85 & 50.24 & HT-Direct \\ 
  Morocco & NORD-OUEST & 90-94 & 71.65 & 62.92 & 81.34 & RW2 \\ 
  Morocco & NORD-OUEST & 95-99 & 57.30 & 72.67 & 45.02 & HT-Direct \\ 
  Morocco & NORD-OUEST & 95-99 & 56.48 & 47.81 & 66.62 & RW2 \\ 
  Morocco & NORD-OUEST & 00-04 & 45.13 & 63.68 & 31.80 & HT-Direct \\ 
  Morocco & NORD-OUEST & 00-04 & 43.89 & 34.09 & 56.53 & RW2 \\ 
  Morocco & NORD-OUEST & 05-09 & 33.84 & 11.04 & 98.48 & RW2 \\ 
  Morocco & NORD-OUEST & 10-14 & 25.95 & 1.75 & 290.68 & RW2 \\ 
  Morocco & NORD-OUEST & 15-19 & 19.60 & 0.17 & 695.69 & RW2 \\ 
  Morocco & ORIENTAL & 80-84 & 102.99 & 123.46 & 85.59 & HT-Direct \\ 
  Morocco & ORIENTAL & 80-84 & 103.01 & 86.56 & 122.28 & RW2 \\ 
  Morocco & ORIENTAL & 85-89 & 70.21 & 87.18 & 56.34 & HT-Direct \\ 
  Morocco & ORIENTAL & 85-89 & 82.79 & 71.20 & 96.04 & RW2 \\ 
  Morocco & ORIENTAL & 90-94 & 69.58 & 104.13 & 45.90 & HT-Direct \\ 
  Morocco & ORIENTAL & 90-94 & 67.78 & 55.69 & 82.07 & RW2 \\ 
  Morocco & ORIENTAL & 95-99 & 66.29 & 95.98 & 45.32 & HT-Direct \\ 
  Morocco & ORIENTAL & 95-99 & 57.16 & 44.22 & 73.62 & RW2 \\ 
  Morocco & ORIENTAL & 00-04 & 38.95 & 69.89 & 21.40 & HT-Direct \\ 
  Morocco & ORIENTAL & 00-04 & 46.80 & 31.97 & 68.14 & RW2 \\ 
  Morocco & ORIENTAL & 05-09 & 37.85 & 11.45 & 116.21 & RW2 \\ 
  Morocco & ORIENTAL & 10-14 & 30.37 & 1.87 & 337.87 & RW2 \\ 
  Morocco & ORIENTAL & 15-19 & 24.49 & 0.20 & 751.42 & RW2 \\ 
  Morocco & SUD & 80-84 & 159.28 & 183.91 & 137.39 & HT-Direct \\ 
  Morocco & SUD & 80-84 & 170.06 & 148.23 & 194.02 & RW2 \\ 
  Morocco & SUD & 85-89 & 120.16 & 140.38 & 102.50 & HT-Direct \\ 
  Morocco & SUD & 85-89 & 121.56 & 108.25 & 135.86 & RW2 \\ 
  Morocco & SUD & 90-94 & 80.53 & 100.37 & 64.33 & HT-Direct \\ 
  Morocco & SUD & 90-94 & 87.27 & 75.63 & 100.50 & RW2 \\ 
  Morocco & SUD & 95-99 & 70.57 & 97.65 & 50.58 & HT-Direct \\ 
  Morocco & SUD & 95-99 & 64.50 & 52.58 & 79.00 & RW2 \\ 
  Morocco & SUD & 00-04 & 42.53 & 66.38 & 27.00 & HT-Direct \\ 
  Morocco & SUD & 00-04 & 46.40 & 34.17 & 62.81 & RW2 \\ 
  Morocco & SUD & 05-09 & 32.86 & 10.47 & 97.05 & RW2 \\ 
  Morocco & SUD & 10-14 & 23.07 & 1.51 & 263.13 & RW2 \\ 
  Morocco & SUD & 15-19 & 16.08 & 0.13 & 662.37 & RW2 \\ 
  Morocco & TENSIFT & 80-84 & 138.66 & 158.16 & 121.23 & HT-Direct \\ 
  Morocco & TENSIFT & 80-84 & 143.08 & 125.55 & 162.42 & RW2 \\ 
  Morocco & TENSIFT & 85-89 & 91.41 & 110.61 & 75.27 & HT-Direct \\ 
  Morocco & TENSIFT & 85-89 & 108.09 & 95.41 & 122.02 & RW2 \\ 
  Morocco & TENSIFT & 90-94 & 95.43 & 118.56 & 76.43 & HT-Direct \\ 
  Morocco & TENSIFT & 90-94 & 83.72 & 72.15 & 97.08 & RW2 \\ 
  Morocco & TENSIFT & 95-99 & 63.52 & 89.90 & 44.51 & HT-Direct \\ 
  Morocco & TENSIFT & 95-99 & 66.76 & 54.76 & 81.18 & RW2 \\ 
  Morocco & TENSIFT & 00-04 & 49.43 & 68.51 & 35.46 & HT-Direct \\ 
  Morocco & TENSIFT & 00-04 & 51.81 & 39.45 & 67.50 & RW2 \\ 
  Morocco & TENSIFT & 05-09 & 39.64 & 12.88 & 115.30 & RW2 \\ 
  Morocco & TENSIFT & 10-14 & 30.01 & 2.00 & 329.13 & RW2 \\ 
  Morocco & TENSIFT & 15-19 & 22.69 & 0.20 & 732.27 & RW2 \\ 
  Mozambique & ALL & 80-84 & 250.42 & 237.09 & 264.07 & IHME \\ 
  Mozambique & ALL & 80-84 & 260.00 & 242.14 & 278.69 & RW2 \\ 
  Mozambique & ALL & 80-84 & 259.85 & 250.08 & 271.14 & UN \\ 
  Mozambique & ALL & 85-89 & 230.62 & 223.88 & 238.24 & IHME \\ 
  Mozambique & ALL & 85-89 & 248.51 & 232.21 & 265.48 & RW2 \\ 
  Mozambique & ALL & 85-89 & 248.72 & 238.91 & 257.88 & UN \\ 
  Mozambique & ALL & 90-94 & 205.17 & 202.81 & 207.71 & IHME \\ 
  Mozambique & ALL & 90-94 & 231.83 & 218.06 & 246.23 & RW2 \\ 
  Mozambique & ALL & 90-94 & 231.85 & 224.04 & 239.63 & UN \\ 
  Mozambique & ALL & 95-99 & 176.97 & 174.80 & 179.20 & IHME \\ 
  Mozambique & ALL & 95-99 & 198.19 & 188.48 & 208.19 & RW2 \\ 
  Mozambique & ALL & 95-99 & 198.20 & 192.02 & 205.08 & UN \\ 
  Mozambique & ALL & 00-04 & 144.16 & 141.98 & 146.21 & IHME \\ 
  Mozambique & ALL & 00-04 & 155.15 & 145.29 & 165.59 & RW2 \\ 
  Mozambique & ALL & 00-04 & 154.80 & 149.67 & 160.17 & UN \\ 
  Mozambique & ALL & 05-09 & 114.91 & 112.42 & 117.29 & IHME \\ 
  Mozambique & ALL & 05-09 & 119.68 & 109.50 & 130.67 & RW2 \\ 
  Mozambique & ALL & 05-09 & 120.29 & 115.63 & 125.70 & UN \\ 
  Mozambique & ALL & 10-14 & 90.78 & 87.44 & 93.88 & IHME \\ 
  Mozambique & ALL & 10-14 & 92.91 & 78.59 & 109.41 & RW2 \\ 
  Mozambique & ALL & 10-14 & 92.10 & 85.76 & 99.25 & UN \\ 
  Mozambique & CABO DELGADO & 80-84 & 301.69 & 372.89 & 238.91 & HT-Direct \\ 
  Mozambique & CABO DELGADO & 80-84 & 317.66 & 266.98 & 371.12 & RW2 \\ 
  Mozambique & CABO DELGADO & 85-89 & 287.41 & 334.14 & 244.81 & HT-Direct \\ 
  Mozambique & CABO DELGADO & 85-89 & 308.80 & 275.68 & 343.69 & RW2 \\ 
  Mozambique & CABO DELGADO & 90-94 & 272.10 & 311.44 & 236.02 & HT-Direct \\ 
  Mozambique & CABO DELGADO & 90-94 & 292.29 & 266.24 & 320.04 & RW2 \\ 
  Mozambique & CABO DELGADO & 95-99 & 242.70 & 277.86 & 210.68 & HT-Direct \\ 
  Mozambique & CABO DELGADO & 95-99 & 251.89 & 229.06 & 277.31 & RW2 \\ 
  Mozambique & CABO DELGADO & 00-04 & 194.02 & 226.87 & 164.92 & HT-Direct \\ 
  Mozambique & CABO DELGADO & 00-04 & 189.58 & 167.91 & 213.29 & RW2 \\ 
  Mozambique & CABO DELGADO & 05-09 & 100.98 & 128.53 & 78.80 & HT-Direct \\ 
  Mozambique & CABO DELGADO & 05-09 & 131.71 & 108.26 & 158.67 & RW2 \\ 
  Mozambique & CABO DELGADO & 10-14 & 60.13 & 106.56 & 33.18 & HT-Direct \\ 
  Mozambique & CABO DELGADO & 10-14 & 89.90 & 62.39 & 124.35 & RW2 \\ 
  Mozambique & CABO DELGADO & 15-19 & 60.37 & 22.09 & 150.69 & RW2 \\ 
  Mozambique & GAZA & 80-84 & 218.86 & 265.84 & 178.16 & HT-Direct \\ 
  Mozambique & GAZA & 80-84 & 219.87 & 186.79 & 257.49 & RW2 \\ 
  Mozambique & GAZA & 85-89 & 202.12 & 240.52 & 168.49 & HT-Direct \\ 
  Mozambique & GAZA & 85-89 & 212.13 & 188.00 & 238.55 & RW2 \\ 
  Mozambique & GAZA & 90-94 & 185.32 & 219.61 & 155.32 & HT-Direct \\ 
  Mozambique & GAZA & 90-94 & 197.90 & 178.18 & 218.99 & RW2 \\ 
  Mozambique & GAZA & 95-99 & 161.56 & 188.50 & 137.82 & HT-Direct \\ 
  Mozambique & GAZA & 95-99 & 172.77 & 155.41 & 191.08 & RW2 \\ 
  Mozambique & GAZA & 00-04 & 126.17 & 153.72 & 102.96 & HT-Direct \\ 
  Mozambique & GAZA & 00-04 & 142.66 & 125.36 & 161.99 & RW2 \\ 
  Mozambique & GAZA & 05-09 & 113.40 & 146.20 & 87.21 & HT-Direct \\ 
  Mozambique & GAZA & 05-09 & 117.72 & 98.16 & 140.87 & RW2 \\ 
  Mozambique & GAZA & 10-14 & 93.54 & 135.74 & 63.50 & HT-Direct \\ 
  Mozambique & GAZA & 10-14 & 98.35 & 74.83 & 128.94 & RW2 \\ 
  Mozambique & GAZA & 15-19 & 81.98 & 33.43 & 188.36 & RW2 \\ 
  Mozambique & INHAMBANE & 80-84 & 224.34 & 273.20 & 182.04 & HT-Direct \\ 
  Mozambique & INHAMBANE & 80-84 & 237.78 & 200.79 & 278.55 & RW2 \\ 
  Mozambique & INHAMBANE & 85-89 & 178.42 & 220.09 & 143.19 & HT-Direct \\ 
  Mozambique & INHAMBANE & 85-89 & 219.21 & 193.75 & 246.76 & RW2 \\ 
  Mozambique & INHAMBANE & 90-94 & 194.09 & 221.90 & 169.02 & HT-Direct \\ 
  Mozambique & INHAMBANE & 90-94 & 193.83 & 176.46 & 212.83 & RW2 \\ 
  Mozambique & INHAMBANE & 95-99 & 154.14 & 172.51 & 137.40 & HT-Direct \\ 
  Mozambique & INHAMBANE & 95-99 & 155.92 & 142.42 & 170.73 & RW2 \\ 
  Mozambique & INHAMBANE & 00-04 & 95.51 & 119.31 & 76.05 & HT-Direct \\ 
  Mozambique & INHAMBANE & 00-04 & 114.22 & 98.92 & 131.45 & RW2 \\ 
  Mozambique & INHAMBANE & 05-09 & 58.48 & 85.67 & 39.54 & HT-Direct \\ 
  Mozambique & INHAMBANE & 05-09 & 82.36 & 64.85 & 103.23 & RW2 \\ 
  Mozambique & INHAMBANE & 10-14 & 34.10 & 80.20 & 14.09 & HT-Direct \\ 
  Mozambique & INHAMBANE & 10-14 & 60.11 & 41.02 & 85.63 & RW2 \\ 
  Mozambique & INHAMBANE & 15-19 & 43.79 & 16.23 & 111.15 & RW2 \\ 
  Mozambique & MANICA & 80-84 & 270.94 & 338.61 & 212.45 & HT-Direct \\ 
  Mozambique & MANICA & 80-84 & 279.87 & 235.33 & 328.85 & RW2 \\ 
  Mozambique & MANICA & 85-89 & 265.20 & 319.32 & 217.33 & HT-Direct \\ 
  Mozambique & MANICA & 85-89 & 261.16 & 230.89 & 293.85 & RW2 \\ 
  Mozambique & MANICA & 90-94 & 214.34 & 244.31 & 187.12 & HT-Direct \\ 
  Mozambique & MANICA & 90-94 & 234.74 & 213.89 & 256.79 & RW2 \\ 
  Mozambique & MANICA & 95-99 & 193.56 & 222.21 & 167.81 & HT-Direct \\ 
  Mozambique & MANICA & 95-99 & 195.46 & 177.96 & 213.95 & RW2 \\ 
  Mozambique & MANICA & 00-04 & 123.43 & 146.75 & 103.37 & HT-Direct \\ 
  Mozambique & MANICA & 00-04 & 151.16 & 134.12 & 170.13 & RW2 \\ 
  Mozambique & MANICA & 05-09 & 104.27 & 132.37 & 81.57 & HT-Direct \\ 
  Mozambique & MANICA & 05-09 & 116.81 & 97.63 & 139.10 & RW2 \\ 
  Mozambique & MANICA & 10-14 & 113.10 & 176.77 & 70.41 & HT-Direct \\ 
  Mozambique & MANICA & 10-14 & 91.76 & 69.23 & 121.55 & RW2 \\ 
  Mozambique & MANICA & 15-19 & 72.01 & 29.18 & 169.07 & RW2 \\ 
  Mozambique & MAPUTO CIDADE & 80-84 & 95.34 & 126.69 & 71.12 & HT-Direct \\ 
  Mozambique & MAPUTO CIDADE & 80-84 & 98.61 & 77.76 & 124.95 & RW2 \\ 
  Mozambique & MAPUTO CIDADE & 85-89 & 99.97 & 127.17 & 78.07 & HT-Direct \\ 
  Mozambique & MAPUTO CIDADE & 85-89 & 102.89 & 87.17 & 121.15 & RW2 \\ 
  Mozambique & MAPUTO CIDADE & 90-94 & 95.91 & 119.13 & 76.83 & HT-Direct \\ 
  Mozambique & MAPUTO CIDADE & 90-94 & 104.07 & 89.90 & 119.77 & RW2 \\ 
  Mozambique & MAPUTO CIDADE & 95-99 & 90.57 & 111.96 & 72.94 & HT-Direct \\ 
  Mozambique & MAPUTO CIDADE & 95-99 & 100.77 & 87.45 & 115.35 & RW2 \\ 
  Mozambique & MAPUTO CIDADE & 00-04 & 92.35 & 112.34 & 75.62 & HT-Direct \\ 
  Mozambique & MAPUTO CIDADE & 00-04 & 94.80 & 81.68 & 109.95 & RW2 \\ 
  Mozambique & MAPUTO CIDADE & 05-09 & 75.06 & 96.89 & 57.83 & HT-Direct \\ 
  Mozambique & MAPUTO CIDADE & 05-09 & 88.22 & 71.52 & 108.91 & RW2 \\ 
  Mozambique & MAPUTO CIDADE & 10-14 & 66.54 & 116.45 & 37.12 & HT-Direct \\ 
  Mozambique & MAPUTO CIDADE & 10-14 & 82.25 & 56.65 & 117.58 & RW2 \\ 
  Mozambique & MAPUTO CIDADE & 15-19 & 76.67 & 27.51 & 194.23 & RW2 \\ 
  Mozambique & MAPUTO PROVINCIA & 80-84 & 134.79 & 190.48 & 93.51 & HT-Direct \\ 
  Mozambique & MAPUTO PROVINCIA & 80-84 & 130.37 & 100.97 & 168.02 & RW2 \\ 
  Mozambique & MAPUTO PROVINCIA & 85-89 & 129.04 & 164.42 & 100.37 & HT-Direct \\ 
  Mozambique & MAPUTO PROVINCIA & 85-89 & 131.48 & 110.64 & 155.98 & RW2 \\ 
  Mozambique & MAPUTO PROVINCIA & 90-94 & 118.37 & 146.97 & 94.73 & HT-Direct \\ 
  Mozambique & MAPUTO PROVINCIA & 90-94 & 128.76 & 111.65 & 147.50 & RW2 \\ 
  Mozambique & MAPUTO PROVINCIA & 95-99 & 93.94 & 118.88 & 73.80 & HT-Direct \\ 
  Mozambique & MAPUTO PROVINCIA & 95-99 & 120.75 & 105.07 & 137.57 & RW2 \\ 
  Mozambique & MAPUTO PROVINCIA & 00-04 & 114.01 & 137.97 & 93.77 & HT-Direct \\ 
  Mozambique & MAPUTO PROVINCIA & 00-04 & 109.85 & 95.19 & 126.51 & RW2 \\ 
  Mozambique & MAPUTO PROVINCIA & 05-09 & 85.20 & 107.47 & 67.21 & HT-Direct \\ 
  Mozambique & MAPUTO PROVINCIA & 05-09 & 99.65 & 81.92 & 121.03 & RW2 \\ 
  Mozambique & MAPUTO PROVINCIA & 10-14 & 76.66 & 133.78 & 42.72 & HT-Direct \\ 
  Mozambique & MAPUTO PROVINCIA & 10-14 & 90.96 & 65.27 & 125.72 & RW2 \\ 
  Mozambique & MAPUTO PROVINCIA & 15-19 & 83.08 & 31.64 & 199.73 & RW2 \\ 
  Mozambique & NAMPULA & 80-84 & 303.38 & 350.91 & 259.70 & HT-Direct \\ 
  Mozambique & NAMPULA & 80-84 & 311.24 & 271.62 & 352.85 & RW2 \\ 
  Mozambique & NAMPULA & 85-89 & 239.61 & 287.83 & 197.23 & HT-Direct \\ 
  Mozambique & NAMPULA & 85-89 & 292.09 & 260.89 & 324.15 & RW2 \\ 
  Mozambique & NAMPULA & 90-94 & 259.15 & 310.41 & 213.73 & HT-Direct \\ 
  Mozambique & NAMPULA & 90-94 & 266.36 & 241.83 & 292.92 & RW2 \\ 
  Mozambique & NAMPULA & 95-99 & 231.42 & 258.29 & 206.56 & HT-Direct \\ 
  Mozambique & NAMPULA & 95-99 & 219.73 & 200.64 & 240.83 & RW2 \\ 
  Mozambique & NAMPULA & 00-04 & 129.86 & 156.74 & 107.00 & HT-Direct \\ 
  Mozambique & NAMPULA & 00-04 & 157.77 & 139.25 & 178.36 & RW2 \\ 
  Mozambique & NAMPULA & 05-09 & 62.71 & 86.48 & 45.14 & HT-Direct \\ 
  Mozambique & NAMPULA & 05-09 & 106.48 & 86.77 & 129.10 & RW2 \\ 
  Mozambique & NAMPULA & 10-14 & 78.29 & 124.54 & 48.27 & HT-Direct \\ 
  Mozambique & NAMPULA & 10-14 & 71.55 & 51.13 & 97.12 & RW2 \\ 
  Mozambique & NAMPULA & 15-19 & 47.50 & 18.09 & 117.77 & RW2 \\ 
  Mozambique & NIASSA & 80-84 & 331.59 & 415.89 & 256.86 & HT-Direct \\ 
  Mozambique & NIASSA & 80-84 & 295.70 & 242.64 & 356.17 & RW2 \\ 
  Mozambique & NIASSA & 85-89 & 230.32 & 292.84 & 177.80 & HT-Direct \\ 
  Mozambique & NIASSA & 85-89 & 271.51 & 234.68 & 311.92 & RW2 \\ 
  Mozambique & NIASSA & 90-94 & 211.33 & 257.42 & 171.59 & HT-Direct \\ 
  Mozambique & NIASSA & 90-94 & 243.93 & 217.22 & 272.44 & RW2 \\ 
  Mozambique & NIASSA & 95-99 & 189.65 & 217.87 & 164.31 & HT-Direct \\ 
  Mozambique & NIASSA & 95-99 & 201.25 & 181.60 & 222.34 & RW2 \\ 
  Mozambique & NIASSA & 00-04 & 152.09 & 188.48 & 121.68 & HT-Direct \\ 
  Mozambique & NIASSA & 00-04 & 147.78 & 129.48 & 168.25 & RW2 \\ 
  Mozambique & NIASSA & 05-09 & 79.42 & 98.47 & 63.79 & HT-Direct \\ 
  Mozambique & NIASSA & 05-09 & 102.99 & 85.43 & 123.57 & RW2 \\ 
  Mozambique & NIASSA & 10-14 & 83.02 & 148.36 & 44.94 & HT-Direct \\ 
  Mozambique & NIASSA & 10-14 & 71.77 & 52.01 & 98.18 & RW2 \\ 
  Mozambique & NIASSA & 15-19 & 49.89 & 18.87 & 123.87 & RW2 \\ 
  Mozambique & SOFALA & 80-84 & 316.07 & 367.96 & 268.39 & HT-Direct \\ 
  Mozambique & SOFALA & 80-84 & 329.17 & 289.94 & 370.76 & RW2 \\ 
  Mozambique & SOFALA & 85-89 & 295.54 & 335.22 & 258.73 & HT-Direct \\ 
  Mozambique & SOFALA & 85-89 & 295.24 & 268.13 & 324.06 & RW2 \\ 
  Mozambique & SOFALA & 90-94 & 230.78 & 266.29 & 198.72 & HT-Direct \\ 
  Mozambique & SOFALA & 90-94 & 254.72 & 233.37 & 277.37 & RW2 \\ 
  Mozambique & SOFALA & 95-99 & 178.11 & 207.25 & 152.28 & HT-Direct \\ 
  Mozambique & SOFALA & 95-99 & 202.55 & 184.46 & 221.79 & RW2 \\ 
  Mozambique & SOFALA & 00-04 & 143.18 & 167.54 & 121.84 & HT-Direct \\ 
  Mozambique & SOFALA & 00-04 & 148.27 & 131.94 & 166.09 & RW2 \\ 
  Mozambique & SOFALA & 05-09 & 97.25 & 128.19 & 73.15 & HT-Direct \\ 
  Mozambique & SOFALA & 05-09 & 106.68 & 89.72 & 126.52 & RW2 \\ 
  Mozambique & SOFALA & 10-14 & 70.63 & 100.77 & 49.02 & HT-Direct \\ 
  Mozambique & SOFALA & 10-14 & 77.38 & 59.86 & 100.01 & RW2 \\ 
  Mozambique & SOFALA & 15-19 & 55.88 & 22.95 & 130.68 & RW2 \\ 
  Mozambique & TETE & 80-84 & 293.29 & 357.31 & 236.51 & HT-Direct \\ 
  Mozambique & TETE & 80-84 & 302.06 & 258.83 & 348.35 & RW2 \\ 
  Mozambique & TETE & 85-89 & 268.28 & 312.54 & 228.20 & HT-Direct \\ 
  Mozambique & TETE & 85-89 & 292.94 & 263.51 & 323.85 & RW2 \\ 
  Mozambique & TETE & 90-94 & 261.51 & 291.39 & 233.70 & HT-Direct \\ 
  Mozambique & TETE & 90-94 & 275.00 & 253.75 & 297.43 & RW2 \\ 
  Mozambique & TETE & 95-99 & 237.47 & 267.11 & 210.18 & HT-Direct \\ 
  Mozambique & TETE & 95-99 & 238.56 & 219.88 & 258.15 & RW2 \\ 
  Mozambique & TETE & 00-04 & 170.39 & 196.52 & 147.11 & HT-Direct \\ 
  Mozambique & TETE & 00-04 & 190.70 & 172.17 & 210.67 & RW2 \\ 
  Mozambique & TETE & 05-09 & 116.42 & 139.45 & 96.77 & HT-Direct \\ 
  Mozambique & TETE & 05-09 & 150.66 & 129.83 & 174.48 & RW2 \\ 
  Mozambique & TETE & 10-14 & 146.31 & 214.37 & 97.18 & HT-Direct \\ 
  Mozambique & TETE & 10-14 & 120.86 & 94.00 & 154.84 & RW2 \\ 
  Mozambique & TETE & 15-19 & 96.78 & 39.92 & 218.44 & RW2 \\ 
  Mozambique & ZAMBEZIA & 80-84 & 271.78 & 326.99 & 222.81 & HT-Direct \\ 
  Mozambique & ZAMBEZIA & 80-84 & 275.33 & 237.04 & 318.43 & RW2 \\ 
  Mozambique & ZAMBEZIA & 85-89 & 263.78 & 311.27 & 221.20 & HT-Direct \\ 
  Mozambique & ZAMBEZIA & 85-89 & 267.44 & 239.63 & 297.54 & RW2 \\ 
  Mozambique & ZAMBEZIA & 90-94 & 246.76 & 279.44 & 216.75 & HT-Direct \\ 
  Mozambique & ZAMBEZIA & 90-94 & 252.59 & 231.52 & 274.83 & RW2 \\ 
  Mozambique & ZAMBEZIA & 95-99 & 192.54 & 218.65 & 168.87 & HT-Direct \\ 
  Mozambique & ZAMBEZIA & 95-99 & 220.71 & 202.74 & 239.46 & RW2 \\ 
  Mozambique & ZAMBEZIA & 00-04 & 150.44 & 175.90 & 128.09 & HT-Direct \\ 
  Mozambique & ZAMBEZIA & 00-04 & 177.47 & 160.21 & 196.21 & RW2 \\ 
  Mozambique & ZAMBEZIA & 05-09 & 143.37 & 166.09 & 123.30 & HT-Direct \\ 
  Mozambique & ZAMBEZIA & 05-09 & 140.24 & 122.06 & 160.49 & RW2 \\ 
  Mozambique & ZAMBEZIA & 10-14 & 88.24 & 129.42 & 59.27 & HT-Direct \\ 
  Mozambique & ZAMBEZIA & 10-14 & 111.34 & 87.59 & 140.93 & RW2 \\ 
  Mozambique & ZAMBEZIA & 15-19 & 88.62 & 36.80 & 197.67 & RW2 \\ 
  Namibia & ALL & 80-84 & 92.76 & 89.63 & 95.98 & IHME \\ 
  Namibia & ALL & 80-84 & 94.40 & 77.46 & 114.60 & RW2 \\ 
  Namibia & ALL & 80-84 & 94.25 & 90.45 & 98.04 & UN \\ 
  Namibia & ALL & 85-89 & 77.18 & 74.67 & 79.77 & IHME \\ 
  Namibia & ALL & 85-89 & 80.38 & 69.75 & 92.45 & RW2 \\ 
  Namibia & ALL & 85-89 & 81.00 & 77.97 & 84.32 & UN \\ 
  Namibia & ALL & 90-94 & 66.03 & 64.06 & 68.03 & IHME \\ 
  Namibia & ALL & 90-94 & 71.83 & 64.15 & 80.34 & RW2 \\ 
  Namibia & ALL & 90-94 & 70.97 & 68.09 & 74.11 & UN \\ 
  Namibia & ALL & 95-99 & 61.14 & 59.16 & 63.31 & IHME \\ 
  Namibia & ALL & 95-99 & 71.72 & 62.53 & 81.83 & RW2 \\ 
  Namibia & ALL & 95-99 & 72.65 & 69.21 & 75.92 & UN \\ 
  Namibia & ALL & 00-04 & 64.68 & 62.33 & 67.00 & IHME \\ 
  Namibia & ALL & 00-04 & 76.05 & 68.56 & 84.46 & RW2 \\ 
  Namibia & ALL & 00-04 & 75.58 & 72.49 & 78.81 & UN \\ 
  Namibia & ALL & 05-09 & 61.58 & 58.75 & 64.51 & IHME \\ 
  Namibia & ALL & 05-09 & 64.42 & 58.34 & 71.06 & RW2 \\ 
  Namibia & ALL & 05-09 & 64.51 & 61.04 & 68.51 & UN \\ 
  Namibia & ALL & 10-14 & 48.89 & 45.65 & 52.49 & IHME \\ 
  Namibia & ALL & 10-14 & 50.33 & 42.12 & 59.97 & RW2 \\ 
  Namibia & ALL & 10-14 & 50.39 & 45.14 & 56.48 & UN \\ 
  Namibia & CAPRIVI & 80-84 & 101.08 & 190.45 & 51.00 & HT-Direct \\ 
  Namibia & CAPRIVI & 80-84 & 108.11 & 66.66 & 170.86 & RW2 \\ 
  Namibia & CAPRIVI & 85-89 & 80.31 & 125.07 & 50.64 & HT-Direct \\ 
  Namibia & CAPRIVI & 85-89 & 91.98 & 66.08 & 125.71 & RW2 \\ 
  Namibia & CAPRIVI & 90-94 & 74.33 & 106.00 & 51.58 & HT-Direct \\ 
  Namibia & CAPRIVI & 90-94 & 83.48 & 64.89 & 106.54 & RW2 \\ 
  Namibia & CAPRIVI & 95-99 & 64.11 & 97.96 & 41.42 & HT-Direct \\ 
  Namibia & CAPRIVI & 95-99 & 90.52 & 71.82 & 113.18 & RW2 \\ 
  Namibia & CAPRIVI & 00-04 & 113.30 & 161.55 & 78.11 & HT-Direct \\ 
  Namibia & CAPRIVI & 00-04 & 94.29 & 76.36 & 116.09 & RW2 \\ 
  Namibia & CAPRIVI & 05-09 & 79.73 & 109.06 & 57.78 & HT-Direct \\ 
  Namibia & CAPRIVI & 05-09 & 83.44 & 66.28 & 104.49 & RW2 \\ 
  Namibia & CAPRIVI & 10-14 & 76.48 & 133.14 & 42.74 & HT-Direct \\ 
  Namibia & CAPRIVI & 10-14 & 71.54 & 49.75 & 101.78 & RW2 \\ 
  Namibia & CAPRIVI & 15-19 & 61.32 & 21.45 & 164.58 & RW2 \\ 
  Namibia & ERONGO & 80-84 & 42.21 & 81.94 & 21.29 & HT-Direct \\ 
  Namibia & ERONGO & 80-84 & 57.82 & 36.62 & 90.84 & RW2 \\ 
  Namibia & ERONGO & 85-89 & 64.51 & 98.04 & 41.91 & HT-Direct \\ 
  Namibia & ERONGO & 85-89 & 51.50 & 36.83 & 71.78 & RW2 \\ 
  Namibia & ERONGO & 90-94 & 33.31 & 56.63 & 19.39 & HT-Direct \\ 
  Namibia & ERONGO & 90-94 & 48.88 & 37.35 & 63.73 & RW2 \\ 
  Namibia & ERONGO & 95-99 & 34.55 & 57.34 & 20.62 & HT-Direct \\ 
  Namibia & ERONGO & 95-99 & 55.24 & 43.54 & 69.44 & RW2 \\ 
  Namibia & ERONGO & 00-04 & 56.20 & 85.17 & 36.69 & HT-Direct \\ 
  Namibia & ERONGO & 00-04 & 59.64 & 47.70 & 74.11 & RW2 \\ 
  Namibia & ERONGO & 05-09 & 58.43 & 88.99 & 37.93 & HT-Direct \\ 
  Namibia & ERONGO & 05-09 & 55.05 & 42.61 & 70.64 & RW2 \\ 
  Namibia & ERONGO & 10-14 & 54.86 & 90.32 & 32.82 & HT-Direct \\ 
  Namibia & ERONGO & 10-14 & 49.62 & 35.01 & 69.77 & RW2 \\ 
  Namibia & ERONGO & 15-19 & 45.20 & 16.77 & 116.35 & RW2 \\ 
  Namibia & HARDAP & 80-84 & 131.68 & 213.15 & 78.26 & HT-Direct \\ 
  Namibia & HARDAP & 80-84 & 126.32 & 84.53 & 184.16 & RW2 \\ 
  Namibia & HARDAP & 85-89 & 76.08 & 116.26 & 49.02 & HT-Direct \\ 
  Namibia & HARDAP & 85-89 & 92.56 & 68.38 & 123.74 & RW2 \\ 
  Namibia & HARDAP & 90-94 & 57.58 & 83.96 & 39.13 & HT-Direct \\ 
  Namibia & HARDAP & 90-94 & 71.81 & 56.36 & 91.00 & RW2 \\ 
  Namibia & HARDAP & 95-99 & 66.81 & 108.39 & 40.46 & HT-Direct \\ 
  Namibia & HARDAP & 95-99 & 65.56 & 51.92 & 82.14 & RW2 \\ 
  Namibia & HARDAP & 00-04 & 50.20 & 85.41 & 29.04 & HT-Direct \\ 
  Namibia & HARDAP & 00-04 & 56.86 & 44.44 & 72.39 & RW2 \\ 
  Namibia & HARDAP & 05-09 & 45.87 & 73.59 & 28.27 & HT-Direct \\ 
  Namibia & HARDAP & 05-09 & 41.87 & 31.09 & 56.34 & RW2 \\ 
  Namibia & HARDAP & 10-14 & 30.41 & 60.43 & 15.07 & HT-Direct \\ 
  Namibia & HARDAP & 10-14 & 29.91 & 19.75 & 45.17 & RW2 \\ 
  Namibia & HARDAP & 15-19 & 21.39 & 7.40 & 61.15 & RW2 \\ 
  Namibia & KARAS & 80-84 & 107.64 & 154.83 & 73.57 & HT-Direct \\ 
  Namibia & KARAS & 80-84 & 117.21 & 82.36 & 164.30 & RW2 \\ 
  Namibia & KARAS & 85-89 & 74.77 & 109.87 & 50.25 & HT-Direct \\ 
  Namibia & KARAS & 85-89 & 89.16 & 68.45 & 115.35 & RW2 \\ 
  Namibia & KARAS & 90-94 & 74.91 & 103.52 & 53.74 & HT-Direct \\ 
  Namibia & KARAS & 90-94 & 72.02 & 57.02 & 90.14 & RW2 \\ 
  Namibia & KARAS & 95-99 & 41.06 & 64.96 & 25.72 & HT-Direct \\ 
  Namibia & KARAS & 95-99 & 69.05 & 53.77 & 87.08 & RW2 \\ 
  Namibia & KARAS & 00-04 & 57.75 & 87.00 & 37.93 & HT-Direct \\ 
  Namibia & KARAS & 00-04 & 64.22 & 50.10 & 81.67 & RW2 \\ 
  Namibia & KARAS & 05-09 & 59.32 & 90.66 & 38.36 & HT-Direct \\ 
  Namibia & KARAS & 05-09 & 51.91 & 38.87 & 68.92 & RW2 \\ 
  Namibia & KARAS & 10-14 & 46.36 & 93.53 & 22.39 & HT-Direct \\ 
  Namibia & KARAS & 10-14 & 40.94 & 26.46 & 63.81 & RW2 \\ 
  Namibia & KARAS & 15-19 & 32.32 & 10.70 & 97.43 & RW2 \\ 
  Namibia & KAVANGO & 80-84 & 111.48 & 175.10 & 69.04 & HT-Direct \\ 
  Namibia & KAVANGO & 80-84 & 119.90 & 85.77 & 166.03 & RW2 \\ 
  Namibia & KAVANGO & 85-89 & 103.51 & 138.03 & 76.85 & HT-Direct \\ 
  Namibia & KAVANGO & 85-89 & 100.73 & 79.95 & 126.45 & RW2 \\ 
  Namibia & KAVANGO & 90-94 & 77.45 & 103.94 & 57.27 & HT-Direct \\ 
  Namibia & KAVANGO & 90-94 & 90.03 & 75.37 & 107.24 & RW2 \\ 
  Namibia & KAVANGO & 95-99 & 83.17 & 99.83 & 69.08 & HT-Direct \\ 
  Namibia & KAVANGO & 95-99 & 95.72 & 82.46 & 110.67 & RW2 \\ 
  Namibia & KAVANGO & 00-04 & 68.11 & 90.04 & 51.21 & HT-Direct \\ 
  Namibia & KAVANGO & 00-04 & 97.85 & 84.06 & 113.39 & RW2 \\ 
  Namibia & KAVANGO & 05-09 & 106.12 & 133.89 & 83.56 & HT-Direct \\ 
  Namibia & KAVANGO & 05-09 & 85.95 & 72.12 & 102.20 & RW2 \\ 
  Namibia & KAVANGO & 10-14 & 77.96 & 122.95 & 48.53 & HT-Direct \\ 
  Namibia & KAVANGO & 10-14 & 73.23 & 55.45 & 96.60 & RW2 \\ 
  Namibia & KAVANGO & 15-19 & 62.41 & 23.98 & 155.00 & RW2 \\ 
  Namibia & KHOMAS & 80-84 & 79.70 & 145.33 & 42.25 & HT-Direct \\ 
  Namibia & KHOMAS & 80-84 & 78.53 & 49.17 & 122.95 & RW2 \\ 
  Namibia & KHOMAS & 85-89 & 43.35 & 76.73 & 24.12 & HT-Direct \\ 
  Namibia & KHOMAS & 85-89 & 64.27 & 45.21 & 90.31 & RW2 \\ 
  Namibia & KHOMAS & 90-94 & 52.37 & 77.59 & 35.04 & HT-Direct \\ 
  Namibia & KHOMAS & 90-94 & 56.01 & 42.89 & 72.80 & RW2 \\ 
  Namibia & KHOMAS & 95-99 & 47.14 & 82.78 & 26.40 & HT-Direct \\ 
  Namibia & KHOMAS & 95-99 & 57.54 & 45.80 & 72.27 & RW2 \\ 
  Namibia & KHOMAS & 00-04 & 51.76 & 79.01 & 33.56 & HT-Direct \\ 
  Namibia & KHOMAS & 00-04 & 55.74 & 44.89 & 69.05 & RW2 \\ 
  Namibia & KHOMAS & 05-09 & 47.98 & 66.85 & 34.24 & HT-Direct \\ 
  Namibia & KHOMAS & 05-09 & 45.67 & 35.54 & 58.42 & RW2 \\ 
  Namibia & KHOMAS & 10-14 & 33.69 & 71.75 & 15.49 & HT-Direct \\ 
  Namibia & KHOMAS & 10-14 & 36.34 & 24.80 & 52.90 & RW2 \\ 
  Namibia & KHOMAS & 15-19 & 29.06 & 10.19 & 80.25 & RW2 \\ 
  Namibia & KUNENE & 80-84 & 83.05 & 145.22 & 46.06 & HT-Direct \\ 
  Namibia & KUNENE & 80-84 & 114.32 & 77.19 & 167.01 & RW2 \\ 
  Namibia & KUNENE & 85-89 & 96.75 & 144.44 & 63.63 & HT-Direct \\ 
  Namibia & KUNENE & 85-89 & 89.82 & 67.09 & 119.37 & RW2 \\ 
  Namibia & KUNENE & 90-94 & 85.35 & 119.76 & 60.15 & HT-Direct \\ 
  Namibia & KUNENE & 90-94 & 74.40 & 59.27 & 92.91 & RW2 \\ 
  Namibia & KUNENE & 95-99 & 53.44 & 78.09 & 36.26 & HT-Direct \\ 
  Namibia & KUNENE & 95-99 & 72.64 & 59.08 & 88.36 & RW2 \\ 
  Namibia & KUNENE & 00-04 & 36.18 & 58.36 & 22.24 & HT-Direct \\ 
  Namibia & KUNENE & 00-04 & 67.97 & 55.12 & 83.12 & RW2 \\ 
  Namibia & KUNENE & 05-09 & 50.84 & 74.05 & 34.64 & HT-Direct \\ 
  Namibia & KUNENE & 05-09 & 54.53 & 42.98 & 69.02 & RW2 \\ 
  Namibia & KUNENE & 10-14 & 68.24 & 109.19 & 41.92 & HT-Direct \\ 
  Namibia & KUNENE & 10-14 & 42.94 & 30.92 & 59.60 & RW2 \\ 
  Namibia & KUNENE & 15-19 & 33.83 & 12.59 & 89.20 & RW2 \\ 
  Namibia & OHANGWENA & 80-84 & 149.73 & 211.99 & 103.37 & HT-Direct \\ 
  Namibia & OHANGWENA & 80-84 & 135.95 & 98.94 & 184.19 & RW2 \\ 
  Namibia & OHANGWENA & 85-89 & 74.38 & 107.77 & 50.74 & HT-Direct \\ 
  Namibia & OHANGWENA & 85-89 & 111.31 & 87.34 & 141.25 & RW2 \\ 
  Namibia & OHANGWENA & 90-94 & 87.90 & 117.98 & 64.93 & HT-Direct \\ 
  Namibia & OHANGWENA & 90-94 & 97.01 & 80.03 & 116.95 & RW2 \\ 
  Namibia & OHANGWENA & 95-99 & 71.00 & 96.98 & 51.58 & HT-Direct \\ 
  Namibia & OHANGWENA & 95-99 & 100.04 & 84.37 & 118.38 & RW2 \\ 
  Namibia & OHANGWENA & 00-04 & 123.21 & 155.85 & 96.63 & HT-Direct \\ 
  Namibia & OHANGWENA & 00-04 & 97.87 & 83.23 & 114.97 & RW2 \\ 
  Namibia & OHANGWENA & 05-09 & 70.09 & 94.42 & 51.67 & HT-Direct \\ 
  Namibia & OHANGWENA & 05-09 & 80.39 & 65.76 & 98.05 & RW2 \\ 
  Namibia & OHANGWENA & 10-14 & 71.68 & 124.08 & 40.39 & HT-Direct \\ 
  Namibia & OHANGWENA & 10-14 & 63.86 & 46.35 & 86.84 & RW2 \\ 
  Namibia & OHANGWENA & 15-19 & 50.62 & 18.71 & 128.87 & RW2 \\ 
  Namibia & OMAHEKE & 80-84 & 61.73 & 118.97 & 31.06 & HT-Direct \\ 
  Namibia & OMAHEKE & 80-84 & 86.79 & 54.47 & 131.47 & RW2 \\ 
  Namibia & OMAHEKE & 85-89 & 44.68 & 69.65 & 28.39 & HT-Direct \\ 
  Namibia & OMAHEKE & 85-89 & 76.17 & 56.52 & 101.42 & RW2 \\ 
  Namibia & OMAHEKE & 90-94 & 70.93 & 93.48 & 53.50 & HT-Direct \\ 
  Namibia & OMAHEKE & 90-94 & 70.96 & 58.03 & 86.63 & RW2 \\ 
  Namibia & OMAHEKE & 95-99 & 80.01 & 98.54 & 64.72 & HT-Direct \\ 
  Namibia & OMAHEKE & 95-99 & 76.15 & 63.86 & 91.26 & RW2 \\ 
  Namibia & OMAHEKE & 00-04 & 57.08 & 99.30 & 32.17 & HT-Direct \\ 
  Namibia & OMAHEKE & 00-04 & 75.42 & 60.88 & 93.34 & RW2 \\ 
  Namibia & OMAHEKE & 05-09 & 57.20 & 88.46 & 36.54 & HT-Direct \\ 
  Namibia & OMAHEKE & 05-09 & 62.26 & 47.08 & 82.18 & RW2 \\ 
  Namibia & OMAHEKE & 10-14 & 46.25 & 90.12 & 23.20 & HT-Direct \\ 
  Namibia & OMAHEKE & 10-14 & 49.78 & 32.70 & 74.17 & RW2 \\ 
  Namibia & OMAHEKE & 15-19 & 39.77 & 13.67 & 108.86 & RW2 \\ 
  Namibia & OMUSATI & 80-84 & 102.85 & 187.78 & 53.79 & HT-Direct \\ 
  Namibia & OMUSATI & 80-84 & 127.27 & 83.51 & 189.21 & RW2 \\ 
  Namibia & OMUSATI & 85-89 & 101.30 & 150.06 & 67.13 & HT-Direct \\ 
  Namibia & OMUSATI & 85-89 & 97.76 & 72.19 & 131.35 & RW2 \\ 
  Namibia & OMUSATI & 90-94 & 57.06 & 83.48 & 38.65 & HT-Direct \\ 
  Namibia & OMUSATI & 90-94 & 79.38 & 63.11 & 99.16 & RW2 \\ 
  Namibia & OMUSATI & 95-99 & 68.22 & 91.13 & 50.75 & HT-Direct \\ 
  Namibia & OMUSATI & 95-99 & 76.39 & 62.78 & 92.30 & RW2 \\ 
  Namibia & OMUSATI & 00-04 & 69.06 & 95.77 & 49.39 & HT-Direct \\ 
  Namibia & OMUSATI & 00-04 & 69.86 & 57.67 & 84.39 & RW2 \\ 
  Namibia & OMUSATI & 05-09 & 57.76 & 80.55 & 41.12 & HT-Direct \\ 
  Namibia & OMUSATI & 05-09 & 53.77 & 42.27 & 68.11 & RW2 \\ 
  Namibia & OMUSATI & 10-14 & 32.32 & 67.73 & 15.12 & HT-Direct \\ 
  Namibia & OMUSATI & 10-14 & 39.86 & 27.37 & 57.80 & RW2 \\ 
  Namibia & OMUSATI & 15-19 & 29.57 & 10.35 & 82.04 & RW2 \\ 
  Namibia & OSHANA & 80-84 & 105.26 & 154.18 & 70.57 & HT-Direct \\ 
  Namibia & OSHANA & 80-84 & 118.10 & 83.91 & 163.85 & RW2 \\ 
  Namibia & OSHANA & 85-89 & 90.19 & 135.89 & 58.82 & HT-Direct \\ 
  Namibia & OSHANA & 85-89 & 95.87 & 74.02 & 123.76 & RW2 \\ 
  Namibia & OSHANA & 90-94 & 64.90 & 88.85 & 47.06 & HT-Direct \\ 
  Namibia & OSHANA & 90-94 & 82.76 & 67.61 & 100.76 & RW2 \\ 
  Namibia & OSHANA & 95-99 & 81.35 & 107.09 & 61.37 & HT-Direct \\ 
  Namibia & OSHANA & 95-99 & 84.22 & 70.33 & 100.64 & RW2 \\ 
  Namibia & OSHANA & 00-04 & 76.67 & 106.52 & 54.67 & HT-Direct \\ 
  Namibia & OSHANA & 00-04 & 81.17 & 66.94 & 98.28 & RW2 \\ 
  Namibia & OSHANA & 05-09 & 74.36 & 108.05 & 50.58 & HT-Direct \\ 
  Namibia & OSHANA & 05-09 & 65.81 & 51.29 & 84.57 & RW2 \\ 
  Namibia & OSHANA & 10-14 & 39.32 & 88.60 & 16.94 & HT-Direct \\ 
  Namibia & OSHANA & 10-14 & 51.68 & 35.03 & 75.04 & RW2 \\ 
  Namibia & OSHANA & 15-19 & 40.50 & 14.27 & 110.29 & RW2 \\ 
  Namibia & OSHIKOTO & 80-84 & 38.42 & 79.53 & 18.14 & HT-Direct \\ 
  Namibia & OSHIKOTO & 80-84 & 87.42 & 57.73 & 128.14 & RW2 \\ 
  Namibia & OSHIKOTO & 85-89 & 64.98 & 92.35 & 45.32 & HT-Direct \\ 
  Namibia & OSHIKOTO & 85-89 & 78.89 & 60.01 & 103.00 & RW2 \\ 
  Namibia & OSHIKOTO & 90-94 & 79.69 & 108.33 & 58.12 & HT-Direct \\ 
  Namibia & OSHIKOTO & 90-94 & 75.20 & 61.60 & 91.54 & RW2 \\ 
  Namibia & OSHIKOTO & 95-99 & 76.50 & 97.17 & 59.94 & HT-Direct \\ 
  Namibia & OSHIKOTO & 95-99 & 83.98 & 71.21 & 98.93 & RW2 \\ 
  Namibia & OSHIKOTO & 00-04 & 92.63 & 118.77 & 71.78 & HT-Direct \\ 
  Namibia & OSHIKOTO & 00-04 & 87.92 & 74.81 & 103.29 & RW2 \\ 
  Namibia & OSHIKOTO & 05-09 & 66.24 & 94.08 & 46.22 & HT-Direct \\ 
  Namibia & OSHIKOTO & 05-09 & 77.40 & 62.78 & 95.17 & RW2 \\ 
  Namibia & OSHIKOTO & 10-14 & 67.35 & 108.58 & 41.06 & HT-Direct \\ 
  Namibia & OSHIKOTO & 10-14 & 66.15 & 48.45 & 89.35 & RW2 \\ 
  Namibia & OSHIKOTO & 15-19 & 56.47 & 21.38 & 142.25 & RW2 \\ 
  Namibia & OTJOZONDJUPA & 80-84 & 65.78 & 121.48 & 34.62 & HT-Direct \\ 
  Namibia & OTJOZONDJUPA & 80-84 & 82.84 & 55.47 & 120.75 & RW2 \\ 
  Namibia & OTJOZONDJUPA & 85-89 & 62.43 & 89.12 & 43.35 & HT-Direct \\ 
  Namibia & OTJOZONDJUPA & 85-89 & 70.44 & 53.33 & 92.29 & RW2 \\ 
  Namibia & OTJOZONDJUPA & 90-94 & 55.50 & 76.66 & 39.93 & HT-Direct \\ 
  Namibia & OTJOZONDJUPA & 90-94 & 63.69 & 51.56 & 78.36 & RW2 \\ 
  Namibia & OTJOZONDJUPA & 95-99 & 45.17 & 68.64 & 29.48 & HT-Direct \\ 
  Namibia & OTJOZONDJUPA & 95-99 & 67.66 & 56.23 & 81.30 & RW2 \\ 
  Namibia & OTJOZONDJUPA & 00-04 & 81.00 & 106.34 & 61.28 & HT-Direct \\ 
  Namibia & OTJOZONDJUPA & 00-04 & 67.91 & 56.75 & 81.12 & RW2 \\ 
  Namibia & OTJOZONDJUPA & 05-09 & 52.05 & 76.85 & 34.96 & HT-Direct \\ 
  Namibia & OTJOZONDJUPA & 05-09 & 57.32 & 45.38 & 72.19 & RW2 \\ 
  Namibia & OTJOZONDJUPA & 10-14 & 41.29 & 73.38 & 22.89 & HT-Direct \\ 
  Namibia & OTJOZONDJUPA & 10-14 & 46.92 & 33.40 & 65.46 & RW2 \\ 
  Namibia & OTJOZONDJUPA & 15-19 & 38.65 & 14.27 & 100.95 & RW2 \\ 
  Niger & ALL & 80-84 & 313.59 & 309.88 & 317.33 & IHME \\ 
  Niger & ALL & 80-84 & 323.38 & 310.60 & 336.43 & RW2 \\ 
  Niger & ALL & 80-84 & 323.45 & 313.75 & 332.99 & UN \\ 
  Niger & ALL & 85-89 & 306.35 & 303.06 & 309.70 & IHME \\ 
  Niger & ALL & 85-89 & 335.58 & 324.87 & 346.43 & RW2 \\ 
  Niger & ALL & 85-89 & 335.45 & 325.61 & 344.98 & UN \\ 
  Niger & ALL & 90-94 & 285.11 & 281.84 & 288.73 & IHME \\ 
  Niger & ALL & 90-94 & 311.54 & 301.20 & 322.09 & RW2 \\ 
  Niger & ALL & 90-94 & 311.64 & 303.32 & 320.12 & UN \\ 
  Niger & ALL & 95-99 & 248.85 & 245.15 & 252.33 & IHME \\ 
  Niger & ALL & 95-99 & 255.79 & 246.55 & 265.17 & RW2 \\ 
  Niger & ALL & 95-99 & 255.87 & 248.96 & 263.81 & UN \\ 
  Niger & ALL & 00-04 & 203.87 & 200.76 & 207.10 & IHME \\ 
  Niger & ALL & 00-04 & 206.12 & 197.38 & 215.21 & RW2 \\ 
  Niger & ALL & 00-04 & 205.98 & 199.46 & 212.56 & UN \\ 
  Niger & ALL & 05-09 & 158.01 & 154.33 & 161.76 & IHME \\ 
  Niger & ALL & 05-09 & 151.01 & 142.43 & 160.00 & RW2 \\ 
  Niger & ALL & 05-09 & 151.08 & 144.77 & 157.85 & UN \\ 
  Niger & ALL & 10-14 & 127.02 & 121.93 & 131.92 & IHME \\ 
  Niger & ALL & 10-14 & 104.71 & 40.31 & 241.38 & RW2 \\ 
  Niger & ALL & 10-14 & 110.69 & 101.57 & 121.65 & UN \\ 
  Niger & DOSSO & 80-84 & 270.51 & 295.75 & 246.66 & HT-Direct \\ 
  Niger & DOSSO & 80-84 & 272.11 & 249.69 & 296.03 & RW2 \\ 
  Niger & DOSSO & 85-89 & 276.17 & 298.99 & 254.46 & HT-Direct \\ 
  Niger & DOSSO & 85-89 & 281.29 & 264.79 & 298.71 & RW2 \\ 
  Niger & DOSSO & 90-94 & 237.39 & 256.62 & 219.18 & HT-Direct \\ 
  Niger & DOSSO & 90-94 & 259.36 & 243.02 & 275.58 & RW2 \\ 
  Niger & DOSSO & 95-99 & 225.22 & 246.18 & 205.56 & HT-Direct \\ 
  Niger & DOSSO & 95-99 & 229.97 & 215.00 & 245.26 & RW2 \\ 
  Niger & DOSSO & 00-04 & 214.44 & 231.03 & 198.74 & HT-Direct \\ 
  Niger & DOSSO & 00-04 & 214.48 & 201.91 & 227.82 & RW2 \\ 
  Niger & DOSSO & 05-09 & 185.22 & 203.99 & 167.81 & HT-Direct \\ 
  Niger & DOSSO & 05-09 & 188.53 & 171.96 & 206.53 & RW2 \\ 
  Niger & DOSSO & 10-14 & 160.11 & 71.27 & 322.36 & RW2 \\ 
  Niger & DOSSO & 15-19 & 134.17 & 15.05 & 603.94 & RW2 \\ 
  Niger & MARADI & 80-84 & 369.76 & 393.72 & 346.42 & HT-Direct \\ 
  Niger & MARADI & 80-84 & 375.33 & 352.86 & 398.35 & RW2 \\ 
  Niger & MARADI & 85-89 & 375.22 & 395.04 & 355.80 & HT-Direct \\ 
  Niger & MARADI & 85-89 & 398.54 & 381.57 & 415.32 & RW2 \\ 
  Niger & MARADI & 90-94 & 376.13 & 395.11 & 357.51 & HT-Direct \\ 
  Niger & MARADI & 90-94 & 382.33 & 366.17 & 399.22 & RW2 \\ 
  Niger & MARADI & 95-99 & 305.75 & 324.28 & 287.82 & HT-Direct \\ 
  Niger & MARADI & 95-99 & 314.46 & 299.29 & 330.54 & RW2 \\ 
  Niger & MARADI & 00-04 & 236.53 & 254.91 & 219.08 & HT-Direct \\ 
  Niger & MARADI & 00-04 & 237.16 & 223.37 & 251.57 & RW2 \\ 
  Niger & MARADI & 05-09 & 144.75 & 161.41 & 129.55 & HT-Direct \\ 
  Niger & MARADI & 05-09 & 153.00 & 137.78 & 169.23 & RW2 \\ 
  Niger & MARADI & 10-14 & 90.32 & 37.86 & 197.77 & RW2 \\ 
  Niger & MARADI & 15-19 & 51.11 & 5.17 & 351.10 & RW2 \\ 
  Niger & NIAMEY & 80-84 & 152.87 & 174.93 & 133.14 & HT-Direct \\ 
  Niger & NIAMEY & 80-84 & 150.60 & 132.52 & 170.68 & RW2 \\ 
  Niger & NIAMEY & 85-89 & 150.39 & 168.75 & 133.71 & HT-Direct \\ 
  Niger & NIAMEY & 85-89 & 163.96 & 149.81 & 178.84 & RW2 \\ 
  Niger & NIAMEY & 90-94 & 151.82 & 171.61 & 133.94 & HT-Direct \\ 
  Niger & NIAMEY & 90-94 & 156.24 & 142.28 & 171.19 & RW2 \\ 
  Niger & NIAMEY & 95-99 & 136.59 & 156.34 & 118.98 & HT-Direct \\ 
  Niger & NIAMEY & 95-99 & 137.68 & 124.62 & 152.15 & RW2 \\ 
  Niger & NIAMEY & 00-04 & 123.33 & 144.00 & 105.27 & HT-Direct \\ 
  Niger & NIAMEY & 00-04 & 122.21 & 108.74 & 137.27 & RW2 \\ 
  Niger & NIAMEY & 05-09 & 88.35 & 116.13 & 66.71 & HT-Direct \\ 
  Niger & NIAMEY & 05-09 & 98.41 & 78.89 & 121.94 & RW2 \\ 
  Niger & NIAMEY & 10-14 & 76.00 & 29.03 & 181.98 & RW2 \\ 
  Niger & NIAMEY & 15-19 & 58.40 & 5.33 & 405.34 & RW2 \\ 
  Niger & TASHOUA/AGADEZ & 80-84 & 320.93 & 347.19 & 295.75 & HT-Direct \\ 
  Niger & TASHOUA/AGADEZ & 80-84 & 328.74 & 305.67 & 352.76 & RW2 \\ 
  Niger & TASHOUA/AGADEZ & 85-89 & 322.83 & 346.89 & 299.67 & HT-Direct \\ 
  Niger & TASHOUA/AGADEZ & 85-89 & 334.35 & 316.88 & 352.25 & RW2 \\ 
  Niger & TASHOUA/AGADEZ & 90-94 & 294.34 & 318.36 & 271.42 & HT-Direct \\ 
  Niger & TASHOUA/AGADEZ & 90-94 & 301.01 & 284.76 & 318.01 & RW2 \\ 
  Niger & TASHOUA/AGADEZ & 95-99 & 230.70 & 248.34 & 213.96 & HT-Direct \\ 
  Niger & TASHOUA/AGADEZ & 95-99 & 242.58 & 228.70 & 256.81 & RW2 \\ 
  Niger & TASHOUA/AGADEZ & 00-04 & 193.26 & 217.28 & 171.32 & HT-Direct \\ 
  Niger & TASHOUA/AGADEZ & 00-04 & 193.51 & 179.50 & 208.45 & RW2 \\ 
  Niger & TASHOUA/AGADEZ & 05-09 & 136.33 & 155.42 & 119.25 & HT-Direct \\ 
  Niger & TASHOUA/AGADEZ & 05-09 & 139.86 & 124.65 & 156.53 & RW2 \\ 
  Niger & TASHOUA/AGADEZ & 10-14 & 95.44 & 40.48 & 208.86 & RW2 \\ 
  Niger & TASHOUA/AGADEZ & 15-19 & 64.34 & 6.69 & 403.03 & RW2 \\ 
  Niger & TILLABERI & 80-84 & 247.86 & 275.21 & 222.39 & HT-Direct \\ 
  Niger & TILLABERI & 80-84 & 276.19 & 251.62 & 301.68 & RW2 \\ 
  Niger & TILLABERI & 85-89 & 313.85 & 336.75 & 291.82 & HT-Direct \\ 
  Niger & TILLABERI & 85-89 & 292.60 & 275.76 & 310.81 & RW2 \\ 
  Niger & TILLABERI & 90-94 & 235.83 & 256.34 & 216.48 & HT-Direct \\ 
  Niger & TILLABERI & 90-94 & 263.56 & 247.50 & 279.90 & RW2 \\ 
  Niger & TILLABERI & 95-99 & 210.20 & 234.65 & 187.67 & HT-Direct \\ 
  Niger & TILLABERI & 95-99 & 219.95 & 204.28 & 235.88 & RW2 \\ 
  Niger & TILLABERI & 00-04 & 187.99 & 204.92 & 172.16 & HT-Direct \\ 
  Niger & TILLABERI & 00-04 & 189.53 & 176.87 & 202.94 & RW2 \\ 
  Niger & TILLABERI & 05-09 & 152.63 & 174.92 & 132.73 & HT-Direct \\ 
  Niger & TILLABERI & 05-09 & 152.42 & 135.48 & 171.32 & RW2 \\ 
  Niger & TILLABERI & 10-14 & 117.59 & 50.01 & 251.08 & RW2 \\ 
  Niger & TILLABERI & 15-19 & 89.90 & 9.66 & 492.96 & RW2 \\ 
  Niger & ZINDA/DIFFA & 80-84 & 360.69 & 392.59 & 329.98 & HT-Direct \\ 
  Niger & ZINDA/DIFFA & 80-84 & 368.98 & 340.90 & 398.49 & RW2 \\ 
  Niger & ZINDA/DIFFA & 85-89 & 367.41 & 392.18 & 343.33 & HT-Direct \\ 
  Niger & ZINDA/DIFFA & 85-89 & 376.26 & 357.19 & 395.77 & RW2 \\ 
  Niger & ZINDA/DIFFA & 90-94 & 329.65 & 352.07 & 307.99 & HT-Direct \\ 
  Niger & ZINDA/DIFFA & 90-94 & 344.43 & 326.77 & 362.58 & RW2 \\ 
  Niger & ZINDA/DIFFA & 95-99 & 263.60 & 284.43 & 243.78 & HT-Direct \\ 
  Niger & ZINDA/DIFFA & 95-99 & 279.07 & 262.87 & 295.55 & RW2 \\ 
  Niger & ZINDA/DIFFA & 00-04 & 223.15 & 245.37 & 202.41 & HT-Direct \\ 
  Niger & ZINDA/DIFFA & 00-04 & 219.19 & 204.06 & 235.18 & RW2 \\ 
  Niger & ZINDA/DIFFA & 05-09 & 148.38 & 168.86 & 130.00 & HT-Direct \\ 
  Niger & ZINDA/DIFFA & 05-09 & 152.73 & 135.91 & 171.28 & RW2 \\ 
  Niger & ZINDA/DIFFA & 10-14 & 99.14 & 41.55 & 217.74 & RW2 \\ 
  Niger & ZINDA/DIFFA & 15-19 & 63.22 & 6.40 & 400.32 & RW2 \\ 
  Nigeria & ALL & 80-84 & 218.11 & 214.92 & 221.28 & IHME \\ 
  Nigeria & ALL & 80-84 & 210.64 & 200.57 & 221.08 & RW2 \\ 
  Nigeria & ALL & 80-84 & 210.74 & 203.42 & 218.05 & UN \\ 
  Nigeria & ALL & 85-89 & 210.74 & 207.86 & 213.88 & IHME \\ 
  Nigeria & ALL & 85-89 & 211.64 & 203.72 & 219.76 & RW2 \\ 
  Nigeria & ALL & 85-89 & 211.46 & 205.53 & 217.77 & UN \\ 
  Nigeria & ALL & 90-94 & 204.18 & 201.54 & 206.97 & IHME \\ 
  Nigeria & ALL & 90-94 & 211.08 & 203.29 & 219.09 & RW2 \\ 
  Nigeria & ALL & 90-94 & 211.30 & 205.57 & 217.09 & UN \\ 
  Nigeria & ALL & 95-99 & 192.15 & 189.68 & 194.71 & IHME \\ 
  Nigeria & ALL & 95-99 & 200.57 & 194.44 & 206.80 & RW2 \\ 
  Nigeria & ALL & 95-99 & 200.45 & 194.94 & 205.96 & UN \\ 
  Nigeria & ALL & 00-04 & 175.49 & 172.92 & 178.00 & IHME \\ 
  Nigeria & ALL & 00-04 & 175.20 & 169.96 & 180.59 & RW2 \\ 
  Nigeria & ALL & 00-04 & 175.32 & 170.33 & 180.06 & UN \\ 
  Nigeria & ALL & 05-09 & 149.69 & 146.71 & 152.37 & IHME \\ 
  Nigeria & ALL & 05-09 & 146.59 & 141.58 & 151.74 & RW2 \\ 
  Nigeria & ALL & 05-09 & 146.49 & 141.68 & 151.76 & UN \\ 
  Nigeria & ALL & 10-14 & 120.69 & 116.52 & 124.95 & IHME \\ 
  Nigeria & ALL & 10-14 & 120.69 & 113.21 & 128.53 & RW2 \\ 
  Nigeria & ALL & 10-14 & 120.81 & 112.42 & 130.51 & UN \\ 
  Nigeria & NORTH CENTRAL & 80-84 & 169.54 & 194.77 & 146.98 & HT-Direct \\ 
  Nigeria & NORTH CENTRAL & 80-84 & 166.20 & 148.97 & 185.36 & RW2 \\ 
  Nigeria & NORTH CENTRAL & 85-89 & 159.27 & 175.08 & 144.64 & HT-Direct \\ 
  Nigeria & NORTH CENTRAL & 85-89 & 159.12 & 148.43 & 170.30 & RW2 \\ 
  Nigeria & NORTH CENTRAL & 90-94 & 149.02 & 164.20 & 135.02 & HT-Direct \\ 
  Nigeria & NORTH CENTRAL & 90-94 & 149.42 & 140.36 & 158.93 & RW2 \\ 
  Nigeria & NORTH CENTRAL & 95-99 & 145.76 & 157.89 & 134.41 & HT-Direct \\ 
  Nigeria & NORTH CENTRAL & 95-99 & 142.03 & 134.28 & 150.02 & RW2 \\ 
  Nigeria & NORTH CENTRAL & 00-04 & 123.67 & 134.21 & 113.85 & HT-Direct \\ 
  Nigeria & NORTH CENTRAL & 00-04 & 127.37 & 120.50 & 134.57 & RW2 \\ 
  Nigeria & NORTH CENTRAL & 05-09 & 121.60 & 133.53 & 110.60 & HT-Direct \\ 
  Nigeria & NORTH CENTRAL & 05-09 & 109.80 & 102.75 & 117.41 & RW2 \\ 
  Nigeria & NORTH CENTRAL & 10-14 & 82.33 & 94.25 & 71.80 & HT-Direct \\ 
  Nigeria & NORTH CENTRAL & 10-14 & 89.29 & 79.92 & 99.71 & RW2 \\ 
  Nigeria & NORTH CENTRAL & 15-19 & 71.04 & 33.09 & 144.88 & RW2 \\ 
  Nigeria & NORTH EAST & 80-84 & 253.01 & 274.84 & 232.36 & HT-Direct \\ 
  Nigeria & NORTH EAST & 80-84 & 254.96 & 236.08 & 274.61 & RW2 \\ 
  Nigeria & NORTH EAST & 85-89 & 270.05 & 289.94 & 251.04 & HT-Direct \\ 
  Nigeria & NORTH EAST & 85-89 & 265.23 & 251.31 & 279.63 & RW2 \\ 
  Nigeria & NORTH EAST & 90-94 & 255.21 & 272.22 & 238.92 & HT-Direct \\ 
  Nigeria & NORTH EAST & 90-94 & 263.60 & 251.02 & 276.45 & RW2 \\ 
  Nigeria & NORTH EAST & 95-99 & 261.63 & 274.25 & 249.39 & HT-Direct \\ 
  Nigeria & NORTH EAST & 95-99 & 253.81 & 244.14 & 263.88 & RW2 \\ 
  Nigeria & NORTH EAST & 00-04 & 227.90 & 239.77 & 216.44 & HT-Direct \\ 
  Nigeria & NORTH EAST & 00-04 & 218.86 & 210.13 & 228.06 & RW2 \\ 
  Nigeria & NORTH EAST & 05-09 & 172.43 & 184.84 & 160.69 & HT-Direct \\ 
  Nigeria & NORTH EAST & 05-09 & 173.99 & 164.55 & 183.73 & RW2 \\ 
  Nigeria & NORTH EAST & 10-14 & 134.37 & 154.03 & 116.87 & HT-Direct \\ 
  Nigeria & NORTH EAST & 10-14 & 133.35 & 118.82 & 149.33 & RW2 \\ 
  Nigeria & NORTH EAST & 15-19 & 100.16 & 47.14 & 199.66 & RW2 \\ 
  Nigeria & NORTH WEST & 80-84 & 279.66 & 300.69 & 259.56 & HT-Direct \\ 
  Nigeria & NORTH WEST & 80-84 & 285.94 & 266.87 & 305.61 & RW2 \\ 
  Nigeria & NORTH WEST & 85-89 & 302.35 & 319.22 & 285.98 & HT-Direct \\ 
  Nigeria & NORTH WEST & 85-89 & 299.24 & 286.24 & 312.77 & RW2 \\ 
  Nigeria & NORTH WEST & 90-94 & 301.66 & 318.89 & 284.97 & HT-Direct \\ 
  Nigeria & NORTH WEST & 90-94 & 293.41 & 280.54 & 307.03 & RW2 \\ 
  Nigeria & NORTH WEST & 95-99 & 273.90 & 287.43 & 260.78 & HT-Direct \\ 
  Nigeria & NORTH WEST & 95-99 & 270.01 & 259.80 & 280.56 & RW2 \\ 
  Nigeria & NORTH WEST & 00-04 & 229.44 & 241.20 & 218.08 & HT-Direct \\ 
  Nigeria & NORTH WEST & 00-04 & 227.94 & 218.73 & 237.25 & RW2 \\ 
  Nigeria & NORTH WEST & 05-09 & 193.05 & 204.21 & 182.36 & HT-Direct \\ 
  Nigeria & NORTH WEST & 05-09 & 186.68 & 178.24 & 195.41 & RW2 \\ 
  Nigeria & NORTH WEST & 10-14 & 145.66 & 159.50 & 132.83 & HT-Direct \\ 
  Nigeria & NORTH WEST & 10-14 & 148.72 & 136.64 & 161.77 & RW2 \\ 
  Nigeria & NORTH WEST & 15-19 & 116.12 & 55.93 & 225.60 & RW2 \\ 
  Nigeria & SOUTH EAST & 80-84 & 132.17 & 153.81 & 113.16 & HT-Direct \\ 
  Nigeria & SOUTH EAST & 80-84 & 133.83 & 117.67 & 152.11 & RW2 \\ 
  Nigeria & SOUTH EAST & 85-89 & 141.31 & 158.64 & 125.59 & HT-Direct \\ 
  Nigeria & SOUTH EAST & 85-89 & 137.33 & 126.43 & 148.92 & RW2 \\ 
  Nigeria & SOUTH EAST & 90-94 & 138.12 & 152.75 & 124.69 & HT-Direct \\ 
  Nigeria & SOUTH EAST & 90-94 & 139.52 & 129.77 & 149.82 & RW2 \\ 
  Nigeria & SOUTH EAST & 95-99 & 144.42 & 159.61 & 130.45 & HT-Direct \\ 
  Nigeria & SOUTH EAST & 95-99 & 145.87 & 136.26 & 155.87 & RW2 \\ 
  Nigeria & SOUTH EAST & 00-04 & 152.62 & 166.41 & 139.78 & HT-Direct \\ 
  Nigeria & SOUTH EAST & 00-04 & 144.92 & 136.09 & 154.42 & RW2 \\ 
  Nigeria & SOUTH EAST & 05-09 & 136.46 & 150.87 & 123.22 & HT-Direct \\ 
  Nigeria & SOUTH EAST & 05-09 & 134.75 & 124.73 & 145.38 & RW2 \\ 
  Nigeria & SOUTH EAST & 10-14 & 115.56 & 142.17 & 93.39 & HT-Direct \\ 
  Nigeria & SOUTH EAST & 10-14 & 117.76 & 100.44 & 137.33 & RW2 \\ 
  Nigeria & SOUTH EAST & 15-19 & 100.99 & 46.03 & 205.26 & RW2 \\ 
  Nigeria & SOUTH SOUTH & 80-84 & 147.74 & 169.07 & 128.69 & HT-Direct \\ 
  Nigeria & SOUTH SOUTH & 80-84 & 141.78 & 126.17 & 159.21 & RW2 \\ 
  Nigeria & SOUTH SOUTH & 85-89 & 126.75 & 143.78 & 111.48 & HT-Direct \\ 
  Nigeria & SOUTH SOUTH & 85-89 & 136.24 & 125.33 & 147.79 & RW2 \\ 
  Nigeria & SOUTH SOUTH & 90-94 & 133.12 & 147.38 & 120.05 & HT-Direct \\ 
  Nigeria & SOUTH SOUTH & 90-94 & 132.17 & 122.92 & 141.68 & RW2 \\ 
  Nigeria & SOUTH SOUTH & 95-99 & 134.25 & 147.92 & 121.67 & HT-Direct \\ 
  Nigeria & SOUTH SOUTH & 95-99 & 132.40 & 124.07 & 141.31 & RW2 \\ 
  Nigeria & SOUTH SOUTH & 00-04 & 131.85 & 143.64 & 120.89 & HT-Direct \\ 
  Nigeria & SOUTH SOUTH & 00-04 & 124.34 & 116.86 & 132.52 & RW2 \\ 
  Nigeria & SOUTH SOUTH & 05-09 & 109.43 & 121.73 & 98.23 & HT-Direct \\ 
  Nigeria & SOUTH SOUTH & 05-09 & 107.62 & 99.78 & 116.00 & RW2 \\ 
  Nigeria & SOUTH SOUTH & 10-14 & 80.06 & 93.99 & 68.04 & HT-Direct \\ 
  Nigeria & SOUTH SOUTH & 10-14 & 86.29 & 75.12 & 98.50 & RW2 \\ 
  Nigeria & SOUTH SOUTH & 15-19 & 67.36 & 30.93 & 139.27 & RW2 \\ 
  Nigeria & SOUTH WEST & 80-84 & 152.94 & 180.09 & 129.24 & HT-Direct \\ 
  Nigeria & SOUTH WEST & 80-84 & 151.87 & 132.13 & 174.91 & RW2 \\ 
  Nigeria & SOUTH WEST & 85-89 & 139.67 & 158.09 & 123.09 & HT-Direct \\ 
  Nigeria & SOUTH WEST & 85-89 & 131.22 & 120.17 & 143.29 & RW2 \\ 
  Nigeria & SOUTH WEST & 90-94 & 105.02 & 118.55 & 92.87 & HT-Direct \\ 
  Nigeria & SOUTH WEST & 90-94 & 113.96 & 104.97 & 123.33 & RW2 \\ 
  Nigeria & SOUTH WEST & 95-99 & 109.77 & 121.01 & 99.46 & HT-Direct \\ 
  Nigeria & SOUTH WEST & 95-99 & 106.23 & 98.81 & 113.89 & RW2 \\ 
  Nigeria & SOUTH WEST & 00-04 & 97.15 & 107.19 & 87.97 & HT-Direct \\ 
  Nigeria & SOUTH WEST & 00-04 & 97.26 & 90.73 & 104.17 & RW2 \\ 
  Nigeria & SOUTH WEST & 05-09 & 91.64 & 102.42 & 81.90 & HT-Direct \\ 
  Nigeria & SOUTH WEST & 05-09 & 86.97 & 79.80 & 94.78 & RW2 \\ 
  Nigeria & SOUTH WEST & 10-14 & 73.40 & 93.53 & 57.33 & HT-Direct \\ 
  Nigeria & SOUTH WEST & 10-14 & 73.92 & 62.15 & 88.19 & RW2 \\ 
  Nigeria & SOUTH WEST & 15-19 & 61.79 & 27.42 & 132.06 & RW2 \\ 
  Rwanda & ALL & 80-84 & 183.05 & 178.72 & 187.12 & IHME \\ 
  Rwanda & ALL & 80-84 & 188.89 & 175.97 & 202.53 & RW2 \\ 
  Rwanda & ALL & 80-84 & 188.63 & 182.96 & 193.81 & UN \\ 
  Rwanda & ALL & 85-89 & 156.43 & 152.45 & 159.97 & IHME \\ 
  Rwanda & ALL & 85-89 & 152.46 & 144.80 & 160.41 & RW2 \\ 
  Rwanda & ALL & 85-89 & 153.07 & 148.61 & 157.43 & UN \\ 
  Rwanda & ALL & 90-94 & 187.82 & 163.45 & 216.87 & IHME \\ 
  Rwanda & ALL & 90-94 & 183.25 & 177.74 & 188.89 & RW2 \\ 
  Rwanda & ALL & 90-94 & 182.69 & 176.51 & 189.45 & UN \\ 
  Rwanda & ALL & 95-99 & 176.38 & 171.82 & 180.55 & IHME \\ 
  Rwanda & ALL & 95-99 & 224.40 & 216.90 & 232.04 & RW2 \\ 
  Rwanda & ALL & 95-99 & 225.09 & 212.46 & 236.58 & UN \\ 
  Rwanda & ALL & 00-04 & 137.76 & 133.65 & 141.35 & IHME \\ 
  Rwanda & ALL & 00-04 & 154.22 & 148.40 & 160.24 & RW2 \\ 
  Rwanda & ALL & 00-04 & 153.90 & 149.44 & 158.95 & UN \\ 
  Rwanda & ALL & 05-09 & 86.74 & 83.70 & 89.98 & IHME \\ 
  Rwanda & ALL & 05-09 & 88.66 & 83.90 & 93.66 & RW2 \\ 
  Rwanda & ALL & 05-09 & 88.79 & 85.42 & 92.16 & UN \\ 
  Rwanda & ALL & 10-14 & 71.42 & 66.66 & 76.65 & IHME \\ 
  Rwanda & ALL & 10-14 & 53.39 & 47.04 & 60.49 & RW2 \\ 
  Rwanda & ALL & 10-14 & 53.34 & 49.12 & 57.82 & UN \\ 
  Rwanda & EAST & 80-84 & 181.85 & 210.81 & 156.09 & HT-Direct \\ 
  Rwanda & EAST & 80-84 & 183.93 & 160.50 & 210.53 & RW2 \\ 
  Rwanda & EAST & 85-89 & 163.34 & 179.61 & 148.28 & HT-Direct \\ 
  Rwanda & EAST & 85-89 & 160.32 & 148.41 & 173.05 & RW2 \\ 
  Rwanda & EAST & 90-94 & 236.73 & 252.10 & 222.02 & HT-Direct \\ 
  Rwanda & EAST & 90-94 & 205.65 & 195.50 & 216.05 & RW2 \\ 
  Rwanda & EAST & 95-99 & 249.38 & 264.70 & 234.66 & HT-Direct \\ 
  Rwanda & EAST & 95-99 & 261.20 & 247.79 & 274.60 & RW2 \\ 
  Rwanda & EAST & 00-04 & 202.32 & 215.71 & 189.57 & HT-Direct \\ 
  Rwanda & EAST & 00-04 & 198.54 & 187.33 & 210.89 & RW2 \\ 
  Rwanda & EAST & 05-09 & 108.27 & 118.56 & 98.77 & HT-Direct \\ 
  Rwanda & EAST & 05-09 & 117.21 & 107.88 & 127.10 & RW2 \\ 
  Rwanda & EAST & 10-14 & 60.79 & 74.03 & 49.79 & HT-Direct \\ 
  Rwanda & EAST & 10-14 & 67.50 & 55.53 & 81.22 & RW2 \\ 
  Rwanda & EAST & 15-19 & 38.44 & 12.23 & 112.22 & RW2 \\ 
  Rwanda & KIGALI & 80-84 & 116.56 & 151.21 & 89.01 & HT-Direct \\ 
  Rwanda & KIGALI & 80-84 & 127.18 & 100.77 & 157.85 & RW2 \\ 
  Rwanda & KIGALI & 85-89 & 109.49 & 133.06 & 89.67 & HT-Direct \\ 
  Rwanda & KIGALI & 85-89 & 111.23 & 97.52 & 126.95 & RW2 \\ 
  Rwanda & KIGALI & 90-94 & 174.20 & 198.31 & 152.46 & HT-Direct \\ 
  Rwanda & KIGALI & 90-94 & 137.94 & 125.59 & 152.13 & RW2 \\ 
  Rwanda & KIGALI & 95-99 & 135.05 & 152.81 & 119.07 & HT-Direct \\ 
  Rwanda & KIGALI & 95-99 & 156.54 & 141.39 & 171.67 & RW2 \\ 
  Rwanda & KIGALI & 00-04 & 108.13 & 127.26 & 91.57 & HT-Direct \\ 
  Rwanda & KIGALI & 00-04 & 104.51 & 93.09 & 116.87 & RW2 \\ 
  Rwanda & KIGALI & 05-09 & 56.50 & 68.32 & 46.62 & HT-Direct \\ 
  Rwanda & KIGALI & 05-09 & 59.23 & 50.92 & 68.98 & RW2 \\ 
  Rwanda & KIGALI & 10-14 & 31.42 & 44.99 & 21.85 & HT-Direct \\ 
  Rwanda & KIGALI & 10-14 & 34.75 & 25.95 & 46.74 & RW2 \\ 
  Rwanda & KIGALI & 15-19 & 20.52 & 6.15 & 66.80 & RW2 \\ 
  Rwanda & NORTH & 80-84 & 211.41 & 247.33 & 179.46 & HT-Direct \\ 
  Rwanda & NORTH & 80-84 & 202.24 & 175.19 & 234.28 & RW2 \\ 
  Rwanda & NORTH & 85-89 & 148.97 & 166.40 & 133.07 & HT-Direct \\ 
  Rwanda & NORTH & 85-89 & 152.04 & 139.51 & 165.18 & RW2 \\ 
  Rwanda & NORTH & 90-94 & 203.16 & 219.86 & 187.42 & HT-Direct \\ 
  Rwanda & NORTH & 90-94 & 179.05 & 168.64 & 189.69 & RW2 \\ 
  Rwanda & NORTH & 95-99 & 219.69 & 236.44 & 203.81 & HT-Direct \\ 
  Rwanda & NORTH & 95-99 & 213.14 & 200.60 & 226.91 & RW2 \\ 
  Rwanda & NORTH & 00-04 & 127.54 & 140.62 & 115.52 & HT-Direct \\ 
  Rwanda & NORTH & 00-04 & 140.73 & 130.36 & 151.61 & RW2 \\ 
  Rwanda & NORTH & 05-09 & 81.73 & 92.47 & 72.13 & HT-Direct \\ 
  Rwanda & NORTH & 05-09 & 79.17 & 71.18 & 88.09 & RW2 \\ 
  Rwanda & NORTH & 10-14 & 35.60 & 47.97 & 26.33 & HT-Direct \\ 
  Rwanda & NORTH & 10-14 & 45.08 & 35.44 & 56.94 & RW2 \\ 
  Rwanda & NORTH & 15-19 & 25.17 & 7.80 & 77.46 & RW2 \\ 
  Rwanda & SOUTH & 80-84 & 162.21 & 181.70 & 144.45 & HT-Direct \\ 
  Rwanda & SOUTH & 80-84 & 173.41 & 154.23 & 194.05 & RW2 \\ 
  Rwanda & SOUTH & 85-89 & 162.75 & 179.84 & 146.98 & HT-Direct \\ 
  Rwanda & SOUTH & 85-89 & 158.42 & 146.74 & 171.15 & RW2 \\ 
  Rwanda & SOUTH & 90-94 & 236.13 & 251.10 & 221.78 & HT-Direct \\ 
  Rwanda & SOUTH & 90-94 & 198.28 & 188.36 & 208.91 & RW2 \\ 
  Rwanda & SOUTH & 95-99 & 209.25 & 222.49 & 196.60 & HT-Direct \\ 
  Rwanda & SOUTH & 95-99 & 225.35 & 213.52 & 237.10 & RW2 \\ 
  Rwanda & SOUTH & 00-04 & 152.05 & 163.39 & 141.36 & HT-Direct \\ 
  Rwanda & SOUTH & 00-04 & 149.14 & 140.13 & 158.50 & RW2 \\ 
  Rwanda & SOUTH & 05-09 & 72.34 & 80.77 & 64.72 & HT-Direct \\ 
  Rwanda & SOUTH & 05-09 & 82.45 & 74.99 & 90.60 & RW2 \\ 
  Rwanda & SOUTH & 10-14 & 52.69 & 66.76 & 41.46 & HT-Direct \\ 
  Rwanda & SOUTH & 10-14 & 49.37 & 40.20 & 61.16 & RW2 \\ 
  Rwanda & SOUTH & 15-19 & 30.22 & 9.55 & 91.48 & RW2 \\ 
  Rwanda & WEST & 80-84 & 201.90 & 227.39 & 178.60 & HT-Direct \\ 
  Rwanda & WEST & 80-84 & 201.09 & 178.07 & 226.61 & RW2 \\ 
  Rwanda & WEST & 85-89 & 144.45 & 160.95 & 129.38 & HT-Direct \\ 
  Rwanda & WEST & 85-89 & 143.21 & 131.15 & 155.76 & RW2 \\ 
  Rwanda & WEST & 90-94 & 186.78 & 201.47 & 172.94 & HT-Direct \\ 
  Rwanda & WEST & 90-94 & 166.39 & 156.23 & 176.59 & RW2 \\ 
  Rwanda & WEST & 95-99 & 211.09 & 225.58 & 197.29 & HT-Direct \\ 
  Rwanda & WEST & 95-99 & 207.40 & 195.23 & 221.24 & RW2 \\ 
  Rwanda & WEST & 00-04 & 132.96 & 144.04 & 122.62 & HT-Direct \\ 
  Rwanda & WEST & 00-04 & 139.67 & 130.25 & 149.46 & RW2 \\ 
  Rwanda & WEST & 05-09 & 79.71 & 88.47 & 71.74 & HT-Direct \\ 
  Rwanda & WEST & 05-09 & 83.18 & 75.59 & 91.46 & RW2 \\ 
  Rwanda & WEST & 10-14 & 49.89 & 64.21 & 38.63 & HT-Direct \\ 
  Rwanda & WEST & 10-14 & 53.68 & 42.91 & 67.61 & RW2 \\ 
  Rwanda & WEST & 15-19 & 34.80 & 10.79 & 106.65 & RW2 \\ 
  Senegal & ALL & 80-84 & 182.52 & 180.58 & 184.68 & IHME \\ 
  Senegal & ALL & 80-84 & 192.12 & 183.26 & 201.29 & RW2 \\ 
  Senegal & ALL & 80-84 & 192.10 & 187.98 & 196.67 & UN \\ 
  Senegal & ALL & 85-89 & 155.77 & 154.15 & 157.31 & IHME \\ 
  Senegal & ALL & 85-89 & 157.11 & 150.44 & 163.99 & RW2 \\ 
  Senegal & ALL & 85-89 & 157.15 & 153.53 & 160.55 & UN \\ 
  Senegal & ALL & 90-94 & 139.81 & 138.06 & 141.32 & IHME \\ 
  Senegal & ALL & 90-94 & 139.54 & 134.03 & 145.23 & RW2 \\ 
  Senegal & ALL & 90-94 & 139.52 & 136.45 & 142.90 & UN \\ 
  Senegal & ALL & 95-99 & 128.38 & 126.78 & 129.94 & IHME \\ 
  Senegal & ALL & 95-99 & 142.37 & 136.89 & 148.01 & RW2 \\ 
  Senegal & ALL & 95-99 & 142.32 & 138.72 & 145.75 & UN \\ 
  Senegal & ALL & 00-04 & 107.37 & 106.02 & 108.76 & IHME \\ 
  Senegal & ALL & 00-04 & 119.97 & 115.18 & 124.97 & RW2 \\ 
  Senegal & ALL & 00-04 & 120.05 & 116.36 & 123.45 & UN \\ 
  Senegal & ALL & 05-09 & 81.94 & 80.70 & 83.26 & IHME \\ 
  Senegal & ALL & 05-09 & 82.31 & 78.33 & 86.47 & RW2 \\ 
  Senegal & ALL & 05-09 & 82.31 & 79.05 & 85.76 & UN \\ 
  Senegal & ALL & 10-14 & 61.65 & 59.91 & 63.20 & IHME \\ 
  Senegal & ALL & 10-14 & 56.69 & 52.88 & 60.73 & RW2 \\ 
  Senegal & ALL & 10-14 & 56.68 & 51.79 & 62.09 & UN \\ 
  Senegal & DAKAR & 80-84 & 130.04 & 150.57 & 111.95 & HT-Direct \\ 
  Senegal & DAKAR & 80-84 & 132.79 & 116.22 & 151.53 & RW2 \\ 
  Senegal & DAKAR & 85-89 & 106.71 & 122.05 & 93.09 & HT-Direct \\ 
  Senegal & DAKAR & 85-89 & 109.60 & 99.28 & 120.97 & RW2 \\ 
  Senegal & DAKAR & 90-94 & 94.71 & 108.77 & 82.31 & HT-Direct \\ 
  Senegal & DAKAR & 90-94 & 94.47 & 85.42 & 104.55 & RW2 \\ 
  Senegal & DAKAR & 95-99 & 92.34 & 109.25 & 77.82 & HT-Direct \\ 
  Senegal & DAKAR & 95-99 & 96.15 & 86.12 & 106.86 & RW2 \\ 
  Senegal & DAKAR & 00-04 & 68.79 & 80.44 & 58.72 & HT-Direct \\ 
  Senegal & DAKAR & 00-04 & 82.87 & 73.82 & 92.55 & RW2 \\ 
  Senegal & DAKAR & 05-09 & 60.75 & 71.50 & 51.52 & HT-Direct \\ 
  Senegal & DAKAR & 05-09 & 57.68 & 50.70 & 65.67 & RW2 \\ 
  Senegal & DAKAR & 10-14 & 38.15 & 50.98 & 28.46 & HT-Direct \\ 
  Senegal & DAKAR & 10-14 & 41.98 & 33.50 & 52.56 & RW2 \\ 
  Senegal & DAKAR & 15-19 & 31.52 & 12.15 & 78.43 & RW2 \\ 
  Senegal & DIOURBEL & 80-84 & 236.15 & 262.57 & 211.62 & HT-Direct \\ 
  Senegal & DIOURBEL & 80-84 & 227.48 & 206.68 & 250.16 & RW2 \\ 
  Senegal & DIOURBEL & 85-89 & 171.17 & 188.97 & 154.73 & HT-Direct \\ 
  Senegal & DIOURBEL & 85-89 & 185.31 & 172.20 & 198.81 & RW2 \\ 
  Senegal & DIOURBEL & 90-94 & 152.41 & 169.02 & 137.16 & HT-Direct \\ 
  Senegal & DIOURBEL & 90-94 & 161.28 & 150.10 & 172.68 & RW2 \\ 
  Senegal & DIOURBEL & 95-99 & 168.65 & 184.77 & 153.66 & HT-Direct \\ 
  Senegal & DIOURBEL & 95-99 & 168.94 & 158.04 & 180.43 & RW2 \\ 
  Senegal & DIOURBEL & 00-04 & 142.76 & 158.26 & 128.54 & HT-Direct \\ 
  Senegal & DIOURBEL & 00-04 & 148.89 & 138.38 & 160.03 & RW2 \\ 
  Senegal & DIOURBEL & 05-09 & 92.69 & 103.97 & 82.52 & HT-Direct \\ 
  Senegal & DIOURBEL & 05-09 & 103.31 & 94.70 & 112.68 & RW2 \\ 
  Senegal & DIOURBEL & 10-14 & 78.40 & 92.43 & 66.34 & HT-Direct \\ 
  Senegal & DIOURBEL & 10-14 & 76.73 & 67.05 & 87.85 & RW2 \\ 
  Senegal & DIOURBEL & 15-19 & 59.26 & 25.40 & 133.82 & RW2 \\ 
  Senegal & FATICK & 80-84 & 209.94 & 236.79 & 185.40 & HT-Direct \\ 
  Senegal & FATICK & 80-84 & 210.72 & 189.52 & 233.11 & RW2 \\ 
  Senegal & FATICK & 85-89 & 160.18 & 181.64 & 140.82 & HT-Direct \\ 
  Senegal & FATICK & 85-89 & 173.90 & 160.05 & 187.99 & RW2 \\ 
  Senegal & FATICK & 90-94 & 148.13 & 164.58 & 133.07 & HT-Direct \\ 
  Senegal & FATICK & 90-94 & 150.71 & 140.27 & 161.86 & RW2 \\ 
  Senegal & FATICK & 95-99 & 153.21 & 170.33 & 137.53 & HT-Direct \\ 
  Senegal & FATICK & 95-99 & 153.33 & 142.79 & 165.04 & RW2 \\ 
  Senegal & FATICK & 00-04 & 123.84 & 138.37 & 110.64 & HT-Direct \\ 
  Senegal & FATICK & 00-04 & 128.18 & 118.54 & 138.72 & RW2 \\ 
  Senegal & FATICK & 05-09 & 69.35 & 81.04 & 59.24 & HT-Direct \\ 
  Senegal & FATICK & 05-09 & 82.25 & 74.01 & 91.21 & RW2 \\ 
  Senegal & FATICK & 10-14 & 56.96 & 70.94 & 45.59 & HT-Direct \\ 
  Senegal & FATICK & 10-14 & 55.97 & 47.19 & 65.81 & RW2 \\ 
  Senegal & FATICK & 15-19 & 39.43 & 16.46 & 92.27 & RW2 \\ 
  Senegal & KAOLACK & 80-84 & 217.72 & 244.39 & 193.22 & HT-Direct \\ 
  Senegal & KAOLACK & 80-84 & 209.29 & 189.28 & 231.19 & RW2 \\ 
  Senegal & KAOLACK & 85-89 & 159.66 & 175.79 & 144.75 & HT-Direct \\ 
  Senegal & KAOLACK & 85-89 & 175.68 & 163.48 & 188.36 & RW2 \\ 
  Senegal & KAOLACK & 90-94 & 149.94 & 163.79 & 137.07 & HT-Direct \\ 
  Senegal & KAOLACK & 90-94 & 155.12 & 145.50 & 165.05 & RW2 \\ 
  Senegal & KAOLACK & 95-99 & 157.77 & 170.41 & 145.90 & HT-Direct \\ 
  Senegal & KAOLACK & 95-99 & 161.49 & 152.48 & 171.16 & RW2 \\ 
  Senegal & KAOLACK & 00-04 & 131.01 & 141.06 & 121.58 & HT-Direct \\ 
  Senegal & KAOLACK & 00-04 & 138.30 & 130.37 & 147.04 & RW2 \\ 
  Senegal & KAOLACK & 05-09 & 85.24 & 91.98 & 78.95 & HT-Direct \\ 
  Senegal & KAOLACK & 05-09 & 89.09 & 83.39 & 95.10 & RW2 \\ 
  Senegal & KAOLACK & 10-14 & 53.93 & 61.78 & 47.03 & HT-Direct \\ 
  Senegal & KAOLACK & 10-14 & 58.83 & 52.06 & 65.95 & RW2 \\ 
  Senegal & KAOLACK & 15-19 & 40.08 & 17.01 & 91.12 & RW2 \\ 
  Senegal & KOLDA & 80-84 & 247.03 & 275.21 & 220.85 & HT-Direct \\ 
  Senegal & KOLDA & 80-84 & 252.15 & 228.88 & 276.71 & RW2 \\ 
  Senegal & KOLDA & 85-89 & 224.04 & 242.06 & 206.99 & HT-Direct \\ 
  Senegal & KOLDA & 85-89 & 225.14 & 211.10 & 239.75 & RW2 \\ 
  Senegal & KOLDA & 90-94 & 188.44 & 205.14 & 172.80 & HT-Direct \\ 
  Senegal & KOLDA & 90-94 & 203.05 & 190.66 & 215.75 & RW2 \\ 
  Senegal & KOLDA & 95-99 & 209.79 & 224.41 & 195.87 & HT-Direct \\ 
  Senegal & KOLDA & 95-99 & 214.57 & 203.17 & 226.35 & RW2 \\ 
  Senegal & KOLDA & 00-04 & 180.67 & 193.60 & 168.42 & HT-Direct \\ 
  Senegal & KOLDA & 00-04 & 189.80 & 179.05 & 201.22 & RW2 \\ 
  Senegal & KOLDA & 05-09 & 115.91 & 127.45 & 105.28 & HT-Direct \\ 
  Senegal & KOLDA & 05-09 & 128.97 & 119.48 & 139.09 & RW2 \\ 
  Senegal & KOLDA & 10-14 & 95.21 & 108.59 & 83.32 & HT-Direct \\ 
  Senegal & KOLDA & 10-14 & 93.85 & 83.52 & 105.36 & RW2 \\ 
  Senegal & KOLDA & 15-19 & 71.32 & 30.48 & 160.04 & RW2 \\ 
  Senegal & LOUGA & 80-84 & 199.32 & 227.80 & 173.59 & HT-Direct \\ 
  Senegal & LOUGA & 80-84 & 194.62 & 173.50 & 218.22 & RW2 \\ 
  Senegal & LOUGA & 85-89 & 152.39 & 174.65 & 132.51 & HT-Direct \\ 
  Senegal & LOUGA & 85-89 & 153.63 & 140.91 & 167.35 & RW2 \\ 
  Senegal & LOUGA & 90-94 & 119.16 & 135.43 & 104.60 & HT-Direct \\ 
  Senegal & LOUGA & 90-94 & 124.77 & 114.96 & 134.91 & RW2 \\ 
  Senegal & LOUGA & 95-99 & 118.54 & 135.93 & 103.10 & HT-Direct \\ 
  Senegal & LOUGA & 95-99 & 121.45 & 110.89 & 132.04 & RW2 \\ 
  Senegal & LOUGA & 00-04 & 82.62 & 95.00 & 71.72 & HT-Direct \\ 
  Senegal & LOUGA & 00-04 & 101.64 & 92.24 & 111.38 & RW2 \\ 
  Senegal & LOUGA & 05-09 & 59.03 & 71.27 & 48.78 & HT-Direct \\ 
  Senegal & LOUGA & 05-09 & 69.30 & 62.21 & 77.05 & RW2 \\ 
  Senegal & LOUGA & 10-14 & 61.26 & 73.01 & 51.30 & HT-Direct \\ 
  Senegal & LOUGA & 10-14 & 52.40 & 45.00 & 61.35 & RW2 \\ 
  Senegal & LOUGA & 15-19 & 41.67 & 17.70 & 96.51 & RW2 \\ 
  Senegal & MATAM & 80-84 & 201.37 & 233.88 & 172.36 & HT-Direct \\ 
  Senegal & MATAM & 80-84 & 228.04 & 200.38 & 255.59 & RW2 \\ 
  Senegal & MATAM & 85-89 & 194.80 & 212.37 & 178.35 & HT-Direct \\ 
  Senegal & MATAM & 85-89 & 192.64 & 179.12 & 207.27 & RW2 \\ 
  Senegal & MATAM & 90-94 & 159.03 & 178.99 & 140.91 & HT-Direct \\ 
  Senegal & MATAM & 90-94 & 157.47 & 145.81 & 170.81 & RW2 \\ 
  Senegal & MATAM & 95-99 & 136.28 & 155.16 & 119.38 & HT-Direct \\ 
  Senegal & MATAM & 95-99 & 145.67 & 134.22 & 157.82 & RW2 \\ 
  Senegal & MATAM & 00-04 & 107.19 & 121.34 & 94.51 & HT-Direct \\ 
  Senegal & MATAM & 00-04 & 114.60 & 104.39 & 125.10 & RW2 \\ 
  Senegal & MATAM & 05-09 & 60.24 & 71.25 & 50.84 & HT-Direct \\ 
  Senegal & MATAM & 05-09 & 72.73 & 64.90 & 81.31 & RW2 \\ 
  Senegal & MATAM & 10-14 & 56.86 & 71.00 & 45.39 & HT-Direct \\ 
  Senegal & MATAM & 10-14 & 51.48 & 43.30 & 61.51 & RW2 \\ 
  Senegal & MATAM & 15-19 & 38.45 & 16.01 & 90.88 & RW2 \\ 
  Senegal & SAINT-LOUIS & 80-84 & 192.57 & 221.99 & 166.22 & HT-Direct \\ 
  Senegal & SAINT-LOUIS & 80-84 & 205.73 & 181.84 & 231.75 & RW2 \\ 
  Senegal & SAINT-LOUIS & 85-89 & 176.11 & 199.47 & 154.95 & HT-Direct \\ 
  Senegal & SAINT-LOUIS & 85-89 & 162.46 & 148.07 & 178.50 & RW2 \\ 
  Senegal & SAINT-LOUIS & 90-94 & 118.09 & 135.36 & 102.75 & HT-Direct \\ 
  Senegal & SAINT-LOUIS & 90-94 & 124.61 & 113.82 & 136.38 & RW2 \\ 
  Senegal & SAINT-LOUIS & 95-99 & 98.69 & 114.09 & 85.18 & HT-Direct \\ 
  Senegal & SAINT-LOUIS & 95-99 & 112.73 & 101.87 & 124.00 & RW2 \\ 
  Senegal & SAINT-LOUIS & 00-04 & 86.69 & 100.09 & 74.93 & HT-Direct \\ 
  Senegal & SAINT-LOUIS & 00-04 & 90.53 & 81.23 & 100.36 & RW2 \\ 
  Senegal & SAINT-LOUIS & 05-09 & 57.03 & 69.04 & 47.00 & HT-Direct \\ 
  Senegal & SAINT-LOUIS & 05-09 & 60.16 & 52.78 & 68.56 & RW2 \\ 
  Senegal & SAINT-LOUIS & 10-14 & 46.09 & 60.77 & 34.83 & HT-Direct \\ 
  Senegal & SAINT-LOUIS & 10-14 & 44.86 & 36.16 & 56.25 & RW2 \\ 
  Senegal & SAINT-LOUIS & 15-19 & 35.51 & 14.29 & 86.98 & RW2 \\ 
  Senegal & TAMBACOUNDA & 80-84 & 212.46 & 249.52 & 179.59 & HT-Direct \\ 
  Senegal & TAMBACOUNDA & 80-84 & 223.42 & 195.51 & 252.55 & RW2 \\ 
  Senegal & TAMBACOUNDA & 85-89 & 198.62 & 225.39 & 174.32 & HT-Direct \\ 
  Senegal & TAMBACOUNDA & 85-89 & 206.24 & 189.77 & 223.70 & RW2 \\ 
  Senegal & TAMBACOUNDA & 90-94 & 189.97 & 208.21 & 172.98 & HT-Direct \\ 
  Senegal & TAMBACOUNDA & 90-94 & 191.84 & 179.44 & 205.17 & RW2 \\ 
  Senegal & TAMBACOUNDA & 95-99 & 201.03 & 221.68 & 181.86 & HT-Direct \\ 
  Senegal & TAMBACOUNDA & 95-99 & 203.63 & 191.27 & 216.85 & RW2 \\ 
  Senegal & TAMBACOUNDA & 00-04 & 167.25 & 181.35 & 154.05 & HT-Direct \\ 
  Senegal & TAMBACOUNDA & 00-04 & 180.39 & 169.50 & 191.96 & RW2 \\ 
  Senegal & TAMBACOUNDA & 05-09 & 116.59 & 129.97 & 104.43 & HT-Direct \\ 
  Senegal & TAMBACOUNDA & 05-09 & 125.08 & 115.59 & 135.22 & RW2 \\ 
  Senegal & TAMBACOUNDA & 10-14 & 90.71 & 104.72 & 78.42 & HT-Direct \\ 
  Senegal & TAMBACOUNDA & 10-14 & 92.31 & 81.82 & 103.80 & RW2 \\ 
  Senegal & TAMBACOUNDA & 15-19 & 70.67 & 30.65 & 155.70 & RW2 \\ 
  Senegal & THIES & 80-84 & 165.76 & 192.82 & 141.84 & HT-Direct \\ 
  Senegal & THIES & 80-84 & 159.01 & 139.86 & 180.45 & RW2 \\ 
  Senegal & THIES & 85-89 & 107.64 & 126.01 & 91.66 & HT-Direct \\ 
  Senegal & THIES & 85-89 & 125.57 & 113.88 & 138.21 & RW2 \\ 
  Senegal & THIES & 90-94 & 102.53 & 116.56 & 90.02 & HT-Direct \\ 
  Senegal & THIES & 90-94 & 105.30 & 96.66 & 114.54 & RW2 \\ 
  Senegal & THIES & 95-99 & 108.17 & 123.43 & 94.59 & HT-Direct \\ 
  Senegal & THIES & 95-99 & 105.58 & 97.00 & 115.03 & RW2 \\ 
  Senegal & THIES & 00-04 & 76.93 & 88.51 & 66.75 & HT-Direct \\ 
  Senegal & THIES & 00-04 & 87.46 & 79.57 & 96.14 & RW2 \\ 
  Senegal & THIES & 05-09 & 60.40 & 72.47 & 50.24 & HT-Direct \\ 
  Senegal & THIES & 05-09 & 56.43 & 49.84 & 63.91 & RW2 \\ 
  Senegal & THIES & 10-14 & 31.22 & 41.39 & 23.49 & HT-Direct \\ 
  Senegal & THIES & 10-14 & 38.47 & 31.43 & 46.69 & RW2 \\ 
  Senegal & THIES & 15-19 & 27.17 & 11.06 & 64.87 & RW2 \\ 
  Senegal & ZUGUINCHOR & 80-84 & 215.24 & 258.63 & 177.38 & HT-Direct \\ 
  Senegal & ZUGUINCHOR & 80-84 & 210.13 & 178.39 & 246.09 & RW2 \\ 
  Senegal & ZUGUINCHOR & 85-89 & 155.69 & 183.78 & 131.21 & HT-Direct \\ 
  Senegal & ZUGUINCHOR & 85-89 & 173.96 & 155.40 & 194.10 & RW2 \\ 
  Senegal & ZUGUINCHOR & 90-94 & 146.86 & 167.08 & 128.72 & HT-Direct \\ 
  Senegal & ZUGUINCHOR & 90-94 & 147.17 & 133.75 & 161.66 & RW2 \\ 
  Senegal & ZUGUINCHOR & 95-99 & 142.88 & 160.92 & 126.55 & HT-Direct \\ 
  Senegal & ZUGUINCHOR & 95-99 & 145.68 & 133.27 & 159.23 & RW2 \\ 
  Senegal & ZUGUINCHOR & 00-04 & 108.17 & 124.43 & 93.80 & HT-Direct \\ 
  Senegal & ZUGUINCHOR & 00-04 & 117.46 & 105.73 & 130.36 & RW2 \\ 
  Senegal & ZUGUINCHOR & 05-09 & 64.92 & 80.18 & 52.40 & HT-Direct \\ 
  Senegal & ZUGUINCHOR & 05-09 & 71.20 & 61.30 & 82.60 & RW2 \\ 
  Senegal & ZUGUINCHOR & 10-14 & 46.67 & 65.19 & 33.22 & HT-Direct \\ 
  Senegal & ZUGUINCHOR & 10-14 & 46.79 & 36.15 & 60.32 & RW2 \\ 
  Senegal & ZUGUINCHOR & 15-19 & 32.37 & 12.32 & 83.21 & RW2 \\ 
  Sierra Leone & ALL & 80-84 & 270.31 & 265.06 & 276.10 & IHME \\ 
  Sierra Leone & ALL & 80-84 & 277.79 & 231.92 & 328.82 & RW2 \\ 
  Sierra Leone & ALL & 80-84 & 281.41 & 268.54 & 294.89 & UN \\ 
  Sierra Leone & ALL & 85-89 & 256.50 & 252.65 & 260.16 & IHME \\ 
  Sierra Leone & ALL & 85-89 & 271.75 & 244.82 & 300.40 & RW2 \\ 
  Sierra Leone & ALL & 85-89 & 268.02 & 258.06 & 278.78 & UN \\ 
  Sierra Leone & ALL & 90-94 & 238.75 & 235.46 & 241.96 & IHME \\ 
  Sierra Leone & ALL & 90-94 & 261.95 & 242.13 & 282.77 & RW2 \\ 
  Sierra Leone & ALL & 90-94 & 262.55 & 253.37 & 271.50 & UN \\ 
  Sierra Leone & ALL & 95-99 & 221.12 & 218.35 & 224.35 & IHME \\ 
  Sierra Leone & ALL & 95-99 & 249.36 & 231.48 & 268.06 & RW2 \\ 
  Sierra Leone & ALL & 95-99 & 250.10 & 242.01 & 258.61 & UN \\ 
  Sierra Leone & ALL & 00-04 & 198.59 & 196.17 & 201.45 & IHME \\ 
  Sierra Leone & ALL & 00-04 & 223.42 & 210.05 & 237.42 & RW2 \\ 
  Sierra Leone & ALL & 00-04 & 223.39 & 216.57 & 230.81 & UN \\ 
  Sierra Leone & ALL & 05-09 & 168.80 & 165.82 & 171.71 & IHME \\ 
  Sierra Leone & ALL & 05-09 & 187.88 & 178.30 & 197.86 & RW2 \\ 
  Sierra Leone & ALL & 05-09 & 187.57 & 180.45 & 194.59 & UN \\ 
  Sierra Leone & ALL & 10-14 & 138.76 & 135.08 & 142.62 & IHME \\ 
  Sierra Leone & ALL & 10-14 & 142.03 & 130.56 & 154.24 & RW2 \\ 
  Sierra Leone & ALL & 10-14 & 142.26 & 134.04 & 150.47 & UN \\ 
  Sierra Leone & EASTERN & 80-84 & 328.65 & 414.01 & 253.28 & HT-Direct \\ 
  Sierra Leone & EASTERN & 80-84 & 322.49 & 259.38 & 391.71 & RW2 \\ 
  Sierra Leone & EASTERN & 85-89 & 337.21 & 415.51 & 266.94 & HT-Direct \\ 
  Sierra Leone & EASTERN & 85-89 & 324.56 & 282.23 & 370.03 & RW2 \\ 
  Sierra Leone & EASTERN & 90-94 & 303.14 & 338.63 & 269.86 & HT-Direct \\ 
  Sierra Leone & EASTERN & 90-94 & 319.95 & 290.98 & 350.38 & RW2 \\ 
  Sierra Leone & EASTERN & 95-99 & 285.39 & 318.36 & 254.56 & HT-Direct \\ 
  Sierra Leone & EASTERN & 95-99 & 303.36 & 277.27 & 331.01 & RW2 \\ 
  Sierra Leone & EASTERN & 00-04 & 253.79 & 281.40 & 228.03 & HT-Direct \\ 
  Sierra Leone & EASTERN & 00-04 & 265.75 & 244.82 & 288.18 & RW2 \\ 
  Sierra Leone & EASTERN & 05-09 & 222.90 & 247.65 & 199.97 & HT-Direct \\ 
  Sierra Leone & EASTERN & 05-09 & 214.07 & 196.72 & 232.58 & RW2 \\ 
  Sierra Leone & EASTERN & 10-14 & 139.96 & 161.80 & 120.64 & HT-Direct \\ 
  Sierra Leone & EASTERN & 10-14 & 154.52 & 133.82 & 177.30 & RW2 \\ 
  Sierra Leone & EASTERN & 15-19 & 105.86 & 46.08 & 223.52 & RW2 \\ 
  Sierra Leone & NORTHERN & 80-84 & 172.52 & 239.84 & 121.08 & HT-Direct \\ 
  Sierra Leone & NORTHERN & 80-84 & 205.16 & 156.77 & 261.83 & RW2 \\ 
  Sierra Leone & NORTHERN & 85-89 & 251.41 & 307.73 & 202.38 & HT-Direct \\ 
  Sierra Leone & NORTHERN & 85-89 & 223.12 & 191.99 & 257.35 & RW2 \\ 
  Sierra Leone & NORTHERN & 90-94 & 221.01 & 251.41 & 193.34 & HT-Direct \\ 
  Sierra Leone & NORTHERN & 90-94 & 230.58 & 207.44 & 255.92 & RW2 \\ 
  Sierra Leone & NORTHERN & 95-99 & 214.10 & 240.19 & 190.13 & HT-Direct \\ 
  Sierra Leone & NORTHERN & 95-99 & 225.61 & 205.22 & 247.67 & RW2 \\ 
  Sierra Leone & NORTHERN & 00-04 & 190.87 & 213.22 & 170.35 & HT-Direct \\ 
  Sierra Leone & NORTHERN & 00-04 & 205.66 & 189.03 & 223.37 & RW2 \\ 
  Sierra Leone & NORTHERN & 05-09 & 183.47 & 199.37 & 168.57 & HT-Direct \\ 
  Sierra Leone & NORTHERN & 05-09 & 176.48 & 164.40 & 189.29 & RW2 \\ 
  Sierra Leone & NORTHERN & 10-14 & 124.95 & 141.90 & 109.77 & HT-Direct \\ 
  Sierra Leone & NORTHERN & 10-14 & 138.08 & 121.87 & 155.92 & RW2 \\ 
  Sierra Leone & NORTHERN & 15-19 & 103.85 & 46.29 & 216.34 & RW2 \\ 
  Sierra Leone & SOUTHERN & 80-84 & 363.86 & 490.77 & 253.43 & HT-Direct \\ 
  Sierra Leone & SOUTHERN & 80-84 & 328.15 & 253.86 & 411.73 & RW2 \\ 
  Sierra Leone & SOUTHERN & 85-89 & 311.62 & 379.99 & 250.58 & HT-Direct \\ 
  Sierra Leone & SOUTHERN & 85-89 & 317.25 & 274.63 & 362.03 & RW2 \\ 
  Sierra Leone & SOUTHERN & 90-94 & 291.24 & 328.34 & 256.73 & HT-Direct \\ 
  Sierra Leone & SOUTHERN & 90-94 & 302.78 & 274.29 & 332.66 & RW2 \\ 
  Sierra Leone & SOUTHERN & 95-99 & 251.08 & 278.09 & 225.87 & HT-Direct \\ 
  Sierra Leone & SOUTHERN & 95-99 & 281.74 & 258.59 & 306.34 & RW2 \\ 
  Sierra Leone & SOUTHERN & 00-04 & 258.84 & 286.52 & 232.96 & HT-Direct \\ 
  Sierra Leone & SOUTHERN & 00-04 & 243.86 & 224.97 & 264.87 & RW2 \\ 
  Sierra Leone & SOUTHERN & 05-09 & 186.96 & 203.68 & 171.32 & HT-Direct \\ 
  Sierra Leone & SOUTHERN & 05-09 & 189.42 & 176.28 & 203.42 & RW2 \\ 
  Sierra Leone & SOUTHERN & 10-14 & 125.42 & 144.05 & 108.90 & HT-Direct \\ 
  Sierra Leone & SOUTHERN & 10-14 & 133.93 & 116.97 & 152.52 & RW2 \\ 
  Sierra Leone & SOUTHERN & 15-19 & 90.86 & 39.57 & 193.15 & RW2 \\ 
  Sierra Leone & WESTERN & 80-84 & 318.37 & 506.39 & 175.35 & HT-Direct \\ 
  Sierra Leone & WESTERN & 80-84 & 244.26 & 158.87 & 370.67 & RW2 \\ 
  Sierra Leone & WESTERN & 85-89 & 250.80 & 346.08 & 174.75 & HT-Direct \\ 
  Sierra Leone & WESTERN & 85-89 & 217.07 & 168.09 & 277.45 & RW2 \\ 
  Sierra Leone & WESTERN & 90-94 & 172.34 & 251.84 & 114.11 & HT-Direct \\ 
  Sierra Leone & WESTERN & 90-94 & 191.67 & 152.50 & 234.99 & RW2 \\ 
  Sierra Leone & WESTERN & 95-99 & 123.36 & 173.41 & 86.26 & HT-Direct \\ 
  Sierra Leone & WESTERN & 95-99 & 175.31 & 138.35 & 213.90 & RW2 \\ 
  Sierra Leone & WESTERN & 00-04 & 152.96 & 193.32 & 119.78 & HT-Direct \\ 
  Sierra Leone & WESTERN & 00-04 & 165.93 & 138.02 & 195.69 & RW2 \\ 
  Sierra Leone & WESTERN & 05-09 & 165.60 & 201.26 & 135.18 & HT-Direct \\ 
  Sierra Leone & WESTERN & 05-09 & 158.17 & 136.82 & 182.13 & RW2 \\ 
  Sierra Leone & WESTERN & 10-14 & 142.23 & 180.87 & 110.72 & HT-Direct \\ 
  Sierra Leone & WESTERN & 10-14 & 144.12 & 113.92 & 183.57 & RW2 \\ 
  Sierra Leone & WESTERN & 15-19 & 128.75 & 53.45 & 284.57 & RW2 \\ 
  Tanzania & ALL & 80-84 & 171.36 & 169.32 & 173.54 & IHME \\ 
  Tanzania & ALL & 80-84 & 179.37 & 167.40 & 192.00 & RW2 \\ 
  Tanzania & ALL & 80-84 & 179.27 & 174.32 & 184.40 & UN \\ 
  Tanzania & ALL & 85-89 & 161.41 & 159.64 & 163.27 & IHME \\ 
  Tanzania & ALL & 85-89 & 171.57 & 162.25 & 181.27 & RW2 \\ 
  Tanzania & ALL & 85-89 & 171.80 & 167.30 & 176.39 & UN \\ 
  Tanzania & ALL & 90-94 & 150.46 & 148.53 & 152.19 & IHME \\ 
  Tanzania & ALL & 90-94 & 162.73 & 155.68 & 170.03 & RW2 \\ 
  Tanzania & ALL & 90-94 & 162.57 & 158.29 & 166.92 & UN \\ 
  Tanzania & ALL & 95-99 & 134.95 & 133.30 & 136.87 & IHME \\ 
  Tanzania & ALL & 95-99 & 149.01 & 142.48 & 155.75 & RW2 \\ 
  Tanzania & ALL & 95-99 & 149.00 & 144.89 & 152.97 & UN \\ 
  Tanzania & ALL & 00-04 & 110.67 & 108.67 & 112.55 & IHME \\ 
  Tanzania & ALL & 00-04 & 114.16 & 108.43 & 120.18 & RW2 \\ 
  Tanzania & ALL & 00-04 & 114.33 & 110.51 & 118.20 & UN \\ 
  Tanzania & ALL & 05-09 & 87.55 & 85.11 & 89.87 & IHME \\ 
  Tanzania & ALL & 05-09 & 79.67 & 74.92 & 84.69 & RW2 \\ 
  Tanzania & ALL & 05-09 & 79.59 & 75.50 & 84.15 & UN \\ 
  Tanzania & ALL & 10-14 & 69.44 & 66.37 & 72.79 & IHME \\ 
  Tanzania & ALL & 10-14 & 56.63 & 52.16 & 61.43 & RW2 \\ 
  Tanzania & ALL & 10-14 & 56.64 & 50.70 & 63.71 & UN \\ 
  Tanzania & ARUSHA & 80-84 & 93.90 & 124.83 & 70.02 & HT-Direct \\ 
  Tanzania & ARUSHA & 80-84 & 109.75 & 87.20 & 137.40 & RW2 \\ 
  Tanzania & ARUSHA & 85-89 & 103.58 & 136.04 & 78.17 & HT-Direct \\ 
  Tanzania & ARUSHA & 85-89 & 104.30 & 88.79 & 122.15 & RW2 \\ 
  Tanzania & ARUSHA & 90-94 & 97.58 & 121.12 & 78.21 & HT-Direct \\ 
  Tanzania & ARUSHA & 90-94 & 97.11 & 85.27 & 110.45 & RW2 \\ 
  Tanzania & ARUSHA & 95-99 & 86.27 & 106.67 & 69.47 & HT-Direct \\ 
  Tanzania & ARUSHA & 95-99 & 87.40 & 77.15 & 98.84 & RW2 \\ 
  Tanzania & ARUSHA & 00-04 & 56.77 & 71.58 & 44.87 & HT-Direct \\ 
  Tanzania & ARUSHA & 00-04 & 65.75 & 57.52 & 75.08 & RW2 \\ 
  Tanzania & ARUSHA & 05-09 & 57.41 & 75.99 & 43.16 & HT-Direct \\ 
  Tanzania & ARUSHA & 05-09 & 43.77 & 37.06 & 51.75 & RW2 \\ 
  Tanzania & ARUSHA & 10-14 & 27.76 & 43.03 & 17.81 & HT-Direct \\ 
  Tanzania & ARUSHA & 10-14 & 29.57 & 22.82 & 38.12 & RW2 \\ 
  Tanzania & ARUSHA & 15-19 & 20.21 & 8.29 & 48.48 & RW2 \\ 
  Tanzania & DAR ES SALAAM & 80-84 & 183.66 & 223.51 & 149.54 & HT-Direct \\ 
  Tanzania & DAR ES SALAAM & 80-84 & 199.32 & 164.11 & 240.00 & RW2 \\ 
  Tanzania & DAR ES SALAAM & 85-89 & 194.14 & 235.18 & 158.77 & HT-Direct \\ 
  Tanzania & DAR ES SALAAM & 85-89 & 175.21 & 151.66 & 202.78 & RW2 \\ 
  Tanzania & DAR ES SALAAM & 90-94 & 127.47 & 155.26 & 104.04 & HT-Direct \\ 
  Tanzania & DAR ES SALAAM & 90-94 & 141.76 & 122.63 & 162.60 & RW2 \\ 
  Tanzania & DAR ES SALAAM & 95-99 & 117.17 & 144.47 & 94.46 & HT-Direct \\ 
  Tanzania & DAR ES SALAAM & 95-99 & 123.94 & 106.22 & 143.51 & RW2 \\ 
  Tanzania & DAR ES SALAAM & 00-04 & 107.46 & 141.23 & 81.01 & HT-Direct \\ 
  Tanzania & DAR ES SALAAM & 00-04 & 105.29 & 88.73 & 124.39 & RW2 \\ 
  Tanzania & DAR ES SALAAM & 05-09 & 102.43 & 135.42 & 76.77 & HT-Direct \\ 
  Tanzania & DAR ES SALAAM & 05-09 & 86.90 & 73.42 & 102.98 & RW2 \\ 
  Tanzania & DAR ES SALAAM & 10-14 & 91.88 & 114.80 & 73.17 & HT-Direct \\ 
  Tanzania & DAR ES SALAAM & 10-14 & 74.88 & 62.54 & 89.42 & RW2 \\ 
  Tanzania & DAR ES SALAAM & 15-19 & 65.16 & 26.87 & 152.11 & RW2 \\ 
  Tanzania & DODOMA & 80-84 & 183.08 & 233.02 & 141.86 & HT-Direct \\ 
  Tanzania & DODOMA & 80-84 & 230.37 & 189.48 & 276.22 & RW2 \\ 
  Tanzania & DODOMA & 85-89 & 240.05 & 280.97 & 203.41 & HT-Direct \\ 
  Tanzania & DODOMA & 85-89 & 225.18 & 199.16 & 254.00 & RW2 \\ 
  Tanzania & DODOMA & 90-94 & 225.95 & 273.26 & 184.74 & HT-Direct \\ 
  Tanzania & DODOMA & 90-94 & 211.45 & 189.07 & 236.33 & RW2 \\ 
  Tanzania & DODOMA & 95-99 & 187.08 & 221.43 & 156.99 & HT-Direct \\ 
  Tanzania & DODOMA & 95-99 & 185.60 & 165.77 & 207.35 & RW2 \\ 
  Tanzania & DODOMA & 00-04 & 119.29 & 147.69 & 95.74 & HT-Direct \\ 
  Tanzania & DODOMA & 00-04 & 134.97 & 118.29 & 153.29 & RW2 \\ 
  Tanzania & DODOMA & 05-09 & 91.26 & 123.75 & 66.64 & HT-Direct \\ 
  Tanzania & DODOMA & 05-09 & 89.20 & 75.50 & 105.14 & RW2 \\ 
  Tanzania & DODOMA & 10-14 & 92.59 & 136.52 & 61.78 & HT-Direct \\ 
  Tanzania & DODOMA & 10-14 & 61.17 & 47.16 & 79.02 & RW2 \\ 
  Tanzania & DODOMA & 15-19 & 42.32 & 17.31 & 99.64 & RW2 \\ 
  Tanzania & IRINGA & 80-84 & 226.04 & 267.96 & 188.99 & HT-Direct \\ 
  Tanzania & IRINGA & 80-84 & 218.04 & 184.75 & 256.35 & RW2 \\ 
  Tanzania & IRINGA & 85-89 & 156.36 & 201.94 & 119.52 & HT-Direct \\ 
  Tanzania & IRINGA & 85-89 & 191.20 & 168.28 & 216.02 & RW2 \\ 
  Tanzania & IRINGA & 90-94 & 174.93 & 204.04 & 149.20 & HT-Direct \\ 
  Tanzania & IRINGA & 90-94 & 170.88 & 152.72 & 190.14 & RW2 \\ 
  Tanzania & IRINGA & 95-99 & 126.82 & 154.57 & 103.44 & HT-Direct \\ 
  Tanzania & IRINGA & 95-99 & 149.78 & 133.59 & 166.93 & RW2 \\ 
  Tanzania & IRINGA & 00-04 & 108.16 & 132.12 & 88.10 & HT-Direct \\ 
  Tanzania & IRINGA & 00-04 & 111.44 & 99.07 & 125.22 & RW2 \\ 
  Tanzania & IRINGA & 05-09 & 98.20 & 119.20 & 80.56 & HT-Direct \\ 
  Tanzania & IRINGA & 05-09 & 74.43 & 64.96 & 85.15 & RW2 \\ 
  Tanzania & IRINGA & 10-14 & 48.49 & 71.36 & 32.69 & HT-Direct \\ 
  Tanzania & IRINGA & 10-14 & 49.85 & 39.76 & 62.51 & RW2 \\ 
  Tanzania & IRINGA & 15-19 & 33.37 & 13.80 & 78.08 & RW2 \\ 
  Tanzania & KAGERA & 80-84 & 203.38 & 253.49 & 161.04 & HT-Direct \\ 
  Tanzania & KAGERA & 80-84 & 207.34 & 170.86 & 249.16 & RW2 \\ 
  Tanzania & KAGERA & 85-89 & 181.10 & 217.29 & 149.78 & HT-Direct \\ 
  Tanzania & KAGERA & 85-89 & 194.37 & 171.28 & 219.78 & RW2 \\ 
  Tanzania & KAGERA & 90-94 & 173.29 & 204.11 & 146.28 & HT-Direct \\ 
  Tanzania & KAGERA & 90-94 & 182.71 & 164.59 & 202.50 & RW2 \\ 
  Tanzania & KAGERA & 95-99 & 178.56 & 205.23 & 154.68 & HT-Direct \\ 
  Tanzania & KAGERA & 95-99 & 168.98 & 152.69 & 186.73 & RW2 \\ 
  Tanzania & KAGERA & 00-04 & 126.88 & 156.97 & 101.87 & HT-Direct \\ 
  Tanzania & KAGERA & 00-04 & 131.47 & 116.34 & 148.75 & RW2 \\ 
  Tanzania & KAGERA & 05-09 & 95.49 & 125.77 & 71.90 & HT-Direct \\ 
  Tanzania & KAGERA & 05-09 & 88.45 & 74.99 & 104.14 & RW2 \\ 
  Tanzania & KAGERA & 10-14 & 73.16 & 115.66 & 45.47 & HT-Direct \\ 
  Tanzania & KAGERA & 10-14 & 59.42 & 44.95 & 77.82 & RW2 \\ 
  Tanzania & KAGERA & 15-19 & 40.08 & 15.99 & 97.72 & RW2 \\ 
  Tanzania & KIGOMA & 80-84 & 175.48 & 238.27 & 126.48 & HT-Direct \\ 
  Tanzania & KIGOMA & 80-84 & 207.22 & 164.67 & 256.53 & RW2 \\ 
  Tanzania & KIGOMA & 85-89 & 187.69 & 224.26 & 155.88 & HT-Direct \\ 
  Tanzania & KIGOMA & 85-89 & 192.80 & 168.20 & 220.18 & RW2 \\ 
  Tanzania & KIGOMA & 90-94 & 187.56 & 221.91 & 157.45 & HT-Direct \\ 
  Tanzania & KIGOMA & 90-94 & 178.54 & 159.24 & 199.79 & RW2 \\ 
  Tanzania & KIGOMA & 95-99 & 137.33 & 171.51 & 109.07 & HT-Direct \\ 
  Tanzania & KIGOMA & 95-99 & 158.12 & 140.28 & 178.21 & RW2 \\ 
  Tanzania & KIGOMA & 00-04 & 123.11 & 149.15 & 101.08 & HT-Direct \\ 
  Tanzania & KIGOMA & 00-04 & 117.09 & 102.64 & 133.35 & RW2 \\ 
  Tanzania & KIGOMA & 05-09 & 91.19 & 122.40 & 67.33 & HT-Direct \\ 
  Tanzania & KIGOMA & 05-09 & 76.34 & 63.84 & 91.04 & RW2 \\ 
  Tanzania & KIGOMA & 10-14 & 47.76 & 82.77 & 27.12 & HT-Direct \\ 
  Tanzania & KIGOMA & 10-14 & 50.64 & 37.29 & 68.00 & RW2 \\ 
  Tanzania & KIGOMA & 15-19 & 33.95 & 13.24 & 83.70 & RW2 \\ 
  Tanzania & KILIMANJARO & 80-84 & 119.76 & 155.32 & 91.45 & HT-Direct \\ 
  Tanzania & KILIMANJARO & 80-84 & 107.24 & 83.79 & 137.03 & RW2 \\ 
  Tanzania & KILIMANJARO & 85-89 & 75.89 & 97.84 & 58.54 & HT-Direct \\ 
  Tanzania & KILIMANJARO & 85-89 & 87.98 & 74.19 & 104.11 & RW2 \\ 
  Tanzania & KILIMANJARO & 90-94 & 71.65 & 91.95 & 55.56 & HT-Direct \\ 
  Tanzania & KILIMANJARO & 90-94 & 74.21 & 62.13 & 87.58 & RW2 \\ 
  Tanzania & KILIMANJARO & 95-99 & 48.75 & 68.41 & 34.52 & HT-Direct \\ 
  Tanzania & KILIMANJARO & 95-99 & 66.94 & 54.85 & 80.49 & RW2 \\ 
  Tanzania & KILIMANJARO & 00-04 & 71.32 & 98.50 & 51.22 & HT-Direct \\ 
  Tanzania & KILIMANJARO & 00-04 & 56.03 & 45.21 & 69.27 & RW2 \\ 
  Tanzania & KILIMANJARO & 05-09 & 41.55 & 77.52 & 21.88 & HT-Direct \\ 
  Tanzania & KILIMANJARO & 05-09 & 43.41 & 32.06 & 58.84 & RW2 \\ 
  Tanzania & KILIMANJARO & 10-14 & 49.22 & 108.66 & 21.51 & HT-Direct \\ 
  Tanzania & KILIMANJARO & 10-14 & 34.77 & 21.72 & 56.64 & RW2 \\ 
  Tanzania & KILIMANJARO & 15-19 & 28.48 & 9.56 & 84.70 & RW2 \\ 
  Tanzania & LINDI & 80-84 & 228.32 & 270.63 & 190.89 & HT-Direct \\ 
  Tanzania & LINDI & 80-84 & 247.58 & 210.16 & 288.06 & RW2 \\ 
  Tanzania & LINDI & 85-89 & 241.50 & 289.26 & 199.42 & HT-Direct \\ 
  Tanzania & LINDI & 85-89 & 255.11 & 227.49 & 284.76 & RW2 \\ 
  Tanzania & LINDI & 90-94 & 257.03 & 299.28 & 218.88 & HT-Direct \\ 
  Tanzania & LINDI & 90-94 & 251.15 & 227.45 & 276.77 & RW2 \\ 
  Tanzania & LINDI & 95-99 & 228.34 & 260.03 & 199.47 & HT-Direct \\ 
  Tanzania & LINDI & 95-99 & 223.88 & 202.77 & 247.08 & RW2 \\ 
  Tanzania & LINDI & 00-04 & 151.11 & 191.98 & 117.67 & HT-Direct \\ 
  Tanzania & LINDI & 00-04 & 154.45 & 135.49 & 175.54 & RW2 \\ 
  Tanzania & LINDI & 05-09 & 100.97 & 135.81 & 74.30 & HT-Direct \\ 
  Tanzania & LINDI & 05-09 & 89.85 & 74.16 & 108.20 & RW2 \\ 
  Tanzania & LINDI & 10-14 & 55.71 & 101.48 & 29.89 & HT-Direct \\ 
  Tanzania & LINDI & 10-14 & 52.18 & 37.29 & 71.82 & RW2 \\ 
  Tanzania & LINDI & 15-19 & 30.21 & 11.38 & 77.02 & RW2 \\ 
  Tanzania & MARA & 80-84 & 194.42 & 244.65 & 152.42 & HT-Direct \\ 
  Tanzania & MARA & 80-84 & 207.16 & 170.67 & 249.27 & RW2 \\ 
  Tanzania & MARA & 85-89 & 211.38 & 250.79 & 176.71 & HT-Direct \\ 
  Tanzania & MARA & 85-89 & 201.00 & 178.51 & 225.60 & RW2 \\ 
  Tanzania & MARA & 90-94 & 180.99 & 204.33 & 159.78 & HT-Direct \\ 
  Tanzania & MARA & 90-94 & 194.69 & 177.89 & 212.40 & RW2 \\ 
  Tanzania & MARA & 95-99 & 187.63 & 216.05 & 162.17 & HT-Direct \\ 
  Tanzania & MARA & 95-99 & 188.54 & 172.15 & 206.19 & RW2 \\ 
  Tanzania & MARA & 00-04 & 168.89 & 195.15 & 145.52 & HT-Direct \\ 
  Tanzania & MARA & 00-04 & 155.57 & 140.53 & 172.42 & RW2 \\ 
  Tanzania & MARA & 05-09 & 121.70 & 149.45 & 98.51 & HT-Direct \\ 
  Tanzania & MARA & 05-09 & 108.38 & 95.27 & 123.19 & RW2 \\ 
  Tanzania & MARA & 10-14 & 80.97 & 112.97 & 57.45 & HT-Direct \\ 
  Tanzania & MARA & 10-14 & 73.00 & 58.34 & 90.47 & RW2 \\ 
  Tanzania & MARA & 15-19 & 48.64 & 20.09 & 112.28 & RW2 \\ 
  Tanzania & MBEYA & 80-84 & 145.34 & 184.21 & 113.54 & HT-Direct \\ 
  Tanzania & MBEYA & 80-84 & 151.04 & 122.02 & 185.29 & RW2 \\ 
  Tanzania & MBEYA & 85-89 & 127.93 & 163.27 & 99.32 & HT-Direct \\ 
  Tanzania & MBEYA & 85-89 & 149.79 & 129.11 & 173.03 & RW2 \\ 
  Tanzania & MBEYA & 90-94 & 168.16 & 204.57 & 137.12 & HT-Direct \\ 
  Tanzania & MBEYA & 90-94 & 150.80 & 132.79 & 170.97 & RW2 \\ 
  Tanzania & MBEYA & 95-99 & 140.71 & 173.24 & 113.46 & HT-Direct \\ 
  Tanzania & MBEYA & 95-99 & 144.93 & 127.84 & 163.91 & RW2 \\ 
  Tanzania & MBEYA & 00-04 & 114.27 & 141.01 & 92.06 & HT-Direct \\ 
  Tanzania & MBEYA & 00-04 & 116.52 & 101.71 & 133.06 & RW2 \\ 
  Tanzania & MBEYA & 05-09 & 87.43 & 121.73 & 62.11 & HT-Direct \\ 
  Tanzania & MBEYA & 05-09 & 86.27 & 72.78 & 102.09 & RW2 \\ 
  Tanzania & MBEYA & 10-14 & 90.23 & 132.68 & 60.41 & HT-Direct \\ 
  Tanzania & MBEYA & 10-14 & 67.71 & 52.26 & 87.27 & RW2 \\ 
  Tanzania & MBEYA & 15-19 & 54.13 & 22.22 & 126.50 & RW2 \\ 
  Tanzania & MOROGORO & 80-84 & 231.80 & 283.50 & 187.07 & HT-Direct \\ 
  Tanzania & MOROGORO & 80-84 & 219.87 & 184.45 & 259.66 & RW2 \\ 
  Tanzania & MOROGORO & 85-89 & 191.27 & 226.30 & 160.53 & HT-Direct \\ 
  Tanzania & MOROGORO & 85-89 & 210.86 & 187.75 & 235.96 & RW2 \\ 
  Tanzania & MOROGORO & 90-94 & 167.27 & 197.22 & 141.08 & HT-Direct \\ 
  Tanzania & MOROGORO & 90-94 & 198.78 & 180.36 & 218.52 & RW2 \\ 
  Tanzania & MOROGORO & 95-99 & 212.36 & 242.61 & 184.96 & HT-Direct \\ 
  Tanzania & MOROGORO & 95-99 & 179.86 & 162.56 & 198.65 & RW2 \\ 
  Tanzania & MOROGORO & 00-04 & 110.90 & 151.60 & 80.09 & HT-Direct \\ 
  Tanzania & MOROGORO & 00-04 & 130.67 & 114.05 & 149.55 & RW2 \\ 
  Tanzania & MOROGORO & 05-09 & 87.42 & 122.40 & 61.74 & HT-Direct \\ 
  Tanzania & MOROGORO & 05-09 & 82.21 & 67.96 & 99.19 & RW2 \\ 
  Tanzania & MOROGORO & 10-14 & 62.52 & 107.36 & 35.67 & HT-Direct \\ 
  Tanzania & MOROGORO & 10-14 & 52.10 & 38.57 & 69.65 & RW2 \\ 
  Tanzania & MOROGORO & 15-19 & 33.01 & 13.21 & 80.37 & RW2 \\ 
  Tanzania & MTWARA & 80-84 & 193.03 & 238.22 & 154.67 & HT-Direct \\ 
  Tanzania & MTWARA & 80-84 & 200.55 & 164.21 & 242.11 & RW2 \\ 
  Tanzania & MTWARA & 85-89 & 182.64 & 220.22 & 150.23 & HT-Direct \\ 
  Tanzania & MTWARA & 85-89 & 213.97 & 188.17 & 241.87 & RW2 \\ 
  Tanzania & MTWARA & 90-94 & 237.53 & 272.91 & 205.44 & HT-Direct \\ 
  Tanzania & MTWARA & 90-94 & 220.25 & 197.63 & 245.20 & RW2 \\ 
  Tanzania & MTWARA & 95-99 & 203.11 & 242.02 & 169.05 & HT-Direct \\ 
  Tanzania & MTWARA & 95-99 & 196.27 & 174.30 & 220.53 & RW2 \\ 
  Tanzania & MTWARA & 00-04 & 114.76 & 143.66 & 91.05 & HT-Direct \\ 
  Tanzania & MTWARA & 00-04 & 130.87 & 112.00 & 151.93 & RW2 \\ 
  Tanzania & MTWARA & 05-09 & 78.09 & 122.69 & 48.80 & HT-Direct \\ 
  Tanzania & MTWARA & 05-09 & 75.74 & 60.47 & 94.12 & RW2 \\ 
  Tanzania & MTWARA & 10-14 & 64.52 & 109.40 & 37.27 & HT-Direct \\ 
  Tanzania & MTWARA & 10-14 & 45.62 & 31.88 & 65.01 & RW2 \\ 
  Tanzania & MTWARA & 15-19 & 27.83 & 10.09 & 74.49 & RW2 \\ 
  Tanzania & MWANZA & 80-84 & 153.58 & 203.55 & 114.11 & HT-Direct \\ 
  Tanzania & MWANZA & 80-84 & 179.07 & 143.07 & 221.77 & RW2 \\ 
  Tanzania & MWANZA & 85-89 & 170.33 & 202.90 & 142.06 & HT-Direct \\ 
  Tanzania & MWANZA & 85-89 & 173.00 & 151.58 & 196.60 & RW2 \\ 
  Tanzania & MWANZA & 90-94 & 176.48 & 205.67 & 150.65 & HT-Direct \\ 
  Tanzania & MWANZA & 90-94 & 163.78 & 147.53 & 181.55 & RW2 \\ 
  Tanzania & MWANZA & 95-99 & 135.66 & 156.48 & 117.21 & HT-Direct \\ 
  Tanzania & MWANZA & 95-99 & 151.20 & 136.46 & 166.86 & RW2 \\ 
  Tanzania & MWANZA & 00-04 & 130.47 & 156.30 & 108.36 & HT-Direct \\ 
  Tanzania & MWANZA & 00-04 & 121.07 & 107.96 & 135.72 & RW2 \\ 
  Tanzania & MWANZA & 05-09 & 91.34 & 113.06 & 73.45 & HT-Direct \\ 
  Tanzania & MWANZA & 05-09 & 84.88 & 73.80 & 97.55 & RW2 \\ 
  Tanzania & MWANZA & 10-14 & 73.88 & 106.42 & 50.73 & HT-Direct \\ 
  Tanzania & MWANZA & 10-14 & 59.60 & 46.94 & 75.37 & RW2 \\ 
  Tanzania & MWANZA & 15-19 & 42.03 & 16.97 & 100.13 & RW2 \\ 
  Tanzania & PWANI & 80-84 & 181.74 & 238.36 & 136.17 & HT-Direct \\ 
  Tanzania & PWANI & 80-84 & 236.01 & 191.82 & 285.58 & RW2 \\ 
  Tanzania & PWANI & 85-89 & 261.14 & 303.19 & 223.06 & HT-Direct \\ 
  Tanzania & PWANI & 85-89 & 220.93 & 194.78 & 249.53 & RW2 \\ 
  Tanzania & PWANI & 90-94 & 157.87 & 189.81 & 130.44 & HT-Direct \\ 
  Tanzania & PWANI & 90-94 & 192.11 & 171.70 & 214.02 & RW2 \\ 
  Tanzania & PWANI & 95-99 & 174.53 & 209.38 & 144.41 & HT-Direct \\ 
  Tanzania & PWANI & 95-99 & 163.94 & 145.59 & 183.63 & RW2 \\ 
  Tanzania & PWANI & 00-04 & 113.70 & 139.14 & 92.42 & HT-Direct \\ 
  Tanzania & PWANI & 00-04 & 120.28 & 104.90 & 137.49 & RW2 \\ 
  Tanzania & PWANI & 05-09 & 81.29 & 121.76 & 53.46 & HT-Direct \\ 
  Tanzania & PWANI & 05-09 & 80.82 & 66.52 & 98.05 & RW2 \\ 
  Tanzania & PWANI & 10-14 & 85.21 & 137.96 & 51.43 & HT-Direct \\ 
  Tanzania & PWANI & 10-14 & 56.50 & 41.61 & 76.85 & RW2 \\ 
  Tanzania & PWANI & 15-19 & 40.17 & 15.78 & 99.52 & RW2 \\ 
  Tanzania & RUKWA & 80-84 & 232.14 & 290.25 & 182.67 & HT-Direct \\ 
  Tanzania & RUKWA & 80-84 & 235.44 & 193.79 & 284.17 & RW2 \\ 
  Tanzania & RUKWA & 85-89 & 195.34 & 229.90 & 164.87 & HT-Direct \\ 
  Tanzania & RUKWA & 85-89 & 204.70 & 182.09 & 229.68 & RW2 \\ 
  Tanzania & RUKWA & 90-94 & 179.72 & 203.26 & 158.36 & HT-Direct \\ 
  Tanzania & RUKWA & 90-94 & 180.83 & 164.38 & 198.56 & RW2 \\ 
  Tanzania & RUKWA & 95-99 & 169.25 & 205.57 & 138.24 & HT-Direct \\ 
  Tanzania & RUKWA & 95-99 & 155.79 & 139.59 & 173.36 & RW2 \\ 
  Tanzania & RUKWA & 00-04 & 106.35 & 124.23 & 90.77 & HT-Direct \\ 
  Tanzania & RUKWA & 00-04 & 115.63 & 102.85 & 129.19 & RW2 \\ 
  Tanzania & RUKWA & 05-09 & 86.17 & 106.89 & 69.15 & HT-Direct \\ 
  Tanzania & RUKWA & 05-09 & 83.24 & 73.05 & 94.61 & RW2 \\ 
  Tanzania & RUKWA & 10-14 & 93.70 & 122.01 & 71.42 & HT-Direct \\ 
  Tanzania & RUKWA & 10-14 & 66.63 & 54.73 & 81.31 & RW2 \\ 
  Tanzania & RUKWA & 15-19 & 55.23 & 23.51 & 127.35 & RW2 \\ 
  Tanzania & RUVUMA & 80-84 & 159.66 & 209.12 & 120.13 & HT-Direct \\ 
  Tanzania & RUVUMA & 80-84 & 147.41 & 116.94 & 184.18 & RW2 \\ 
  Tanzania & RUVUMA & 85-89 & 118.21 & 148.04 & 93.73 & HT-Direct \\ 
  Tanzania & RUVUMA & 85-89 & 157.00 & 134.96 & 181.88 & RW2 \\ 
  Tanzania & RUVUMA & 90-94 & 183.01 & 224.41 & 147.80 & HT-Direct \\ 
  Tanzania & RUVUMA & 90-94 & 166.26 & 146.64 & 188.20 & RW2 \\ 
  Tanzania & RUVUMA & 95-99 & 162.84 & 196.78 & 133.77 & HT-Direct \\ 
  Tanzania & RUVUMA & 95-99 & 160.62 & 142.16 & 181.54 & RW2 \\ 
  Tanzania & RUVUMA & 00-04 & 129.48 & 159.50 & 104.41 & HT-Direct \\ 
  Tanzania & RUVUMA & 00-04 & 119.77 & 104.20 & 137.16 & RW2 \\ 
  Tanzania & RUVUMA & 05-09 & 66.21 & 92.09 & 47.22 & HT-Direct \\ 
  Tanzania & RUVUMA & 05-09 & 76.35 & 62.98 & 92.14 & RW2 \\ 
  Tanzania & RUVUMA & 10-14 & 68.63 & 121.95 & 37.62 & HT-Direct \\ 
  Tanzania & RUVUMA & 10-14 & 49.89 & 35.80 & 68.20 & RW2 \\ 
  Tanzania & RUVUMA & 15-19 & 32.71 & 12.50 & 83.02 & RW2 \\ 
  Tanzania & SHINYANGA & 80-84 & 190.72 & 231.11 & 155.96 & HT-Direct \\ 
  Tanzania & SHINYANGA & 80-84 & 188.39 & 159.50 & 221.84 & RW2 \\ 
  Tanzania & SHINYANGA & 85-89 & 162.95 & 193.74 & 136.22 & HT-Direct \\ 
  Tanzania & SHINYANGA & 85-89 & 175.55 & 156.94 & 196.02 & RW2 \\ 
  Tanzania & SHINYANGA & 90-94 & 168.27 & 193.54 & 145.71 & HT-Direct \\ 
  Tanzania & SHINYANGA & 90-94 & 165.21 & 150.62 & 180.61 & RW2 \\ 
  Tanzania & SHINYANGA & 95-99 & 147.13 & 167.96 & 128.48 & HT-Direct \\ 
  Tanzania & SHINYANGA & 95-99 & 154.02 & 140.87 & 167.85 & RW2 \\ 
  Tanzania & SHINYANGA & 00-04 & 126.54 & 146.34 & 109.07 & HT-Direct \\ 
  Tanzania & SHINYANGA & 00-04 & 124.18 & 113.03 & 136.30 & RW2 \\ 
  Tanzania & SHINYANGA & 05-09 & 93.72 & 113.56 & 77.04 & HT-Direct \\ 
  Tanzania & SHINYANGA & 05-09 & 90.11 & 80.75 & 100.54 & RW2 \\ 
  Tanzania & SHINYANGA & 10-14 & 88.66 & 108.92 & 71.86 & HT-Direct \\ 
  Tanzania & SHINYANGA & 10-14 & 67.39 & 57.98 & 78.31 & RW2 \\ 
  Tanzania & SHINYANGA & 15-19 & 51.09 & 22.59 & 111.27 & RW2 \\ 
  Tanzania & SINGIDA & 80-84 & 173.27 & 228.70 & 129.03 & HT-Direct \\ 
  Tanzania & SINGIDA & 80-84 & 154.18 & 121.77 & 193.58 & RW2 \\ 
  Tanzania & SINGIDA & 85-89 & 92.39 & 123.10 & 68.74 & HT-Direct \\ 
  Tanzania & SINGIDA & 85-89 & 140.68 & 119.82 & 164.12 & RW2 \\ 
  Tanzania & SINGIDA & 90-94 & 133.16 & 159.69 & 110.46 & HT-Direct \\ 
  Tanzania & SINGIDA & 90-94 & 130.39 & 115.25 & 147.08 & RW2 \\ 
  Tanzania & SINGIDA & 95-99 & 121.56 & 148.91 & 98.64 & HT-Direct \\ 
  Tanzania & SINGIDA & 95-99 & 116.24 & 102.55 & 131.66 & RW2 \\ 
  Tanzania & SINGIDA & 00-04 & 90.11 & 122.18 & 65.83 & HT-Direct \\ 
  Tanzania & SINGIDA & 00-04 & 84.75 & 72.67 & 98.58 & RW2 \\ 
  Tanzania & SINGIDA & 05-09 & 58.54 & 82.86 & 41.04 & HT-Direct \\ 
  Tanzania & SINGIDA & 05-09 & 54.32 & 44.28 & 66.53 & RW2 \\ 
  Tanzania & SINGIDA & 10-14 & 33.32 & 60.69 & 18.06 & HT-Direct \\ 
  Tanzania & SINGIDA & 10-14 & 35.52 & 25.82 & 48.42 & RW2 \\ 
  Tanzania & SINGIDA & 15-19 & 23.51 & 9.22 & 59.03 & RW2 \\ 
  Tanzania & TABORA & 80-84 & 177.31 & 229.12 & 135.16 & HT-Direct \\ 
  Tanzania & TABORA & 80-84 & 174.31 & 139.19 & 215.18 & RW2 \\ 
  Tanzania & TABORA & 85-89 & 129.93 & 168.42 & 99.19 & HT-Direct \\ 
  Tanzania & TABORA & 85-89 & 165.19 & 142.16 & 190.87 & RW2 \\ 
  Tanzania & TABORA & 90-94 & 156.58 & 188.39 & 129.29 & HT-Direct \\ 
  Tanzania & TABORA & 90-94 & 159.56 & 142.77 & 177.80 & RW2 \\ 
  Tanzania & TABORA & 95-99 & 157.20 & 178.14 & 138.32 & HT-Direct \\ 
  Tanzania & TABORA & 95-99 & 147.37 & 133.92 & 162.04 & RW2 \\ 
  Tanzania & TABORA & 00-04 & 106.21 & 126.98 & 88.50 & HT-Direct \\ 
  Tanzania & TABORA & 00-04 & 111.05 & 99.69 & 123.48 & RW2 \\ 
  Tanzania & TABORA & 05-09 & 78.88 & 98.17 & 63.11 & HT-Direct \\ 
  Tanzania & TABORA & 05-09 & 74.86 & 65.31 & 85.71 & RW2 \\ 
  Tanzania & TABORA & 10-14 & 65.78 & 93.23 & 46.00 & HT-Direct \\ 
  Tanzania & TABORA & 10-14 & 52.80 & 42.28 & 65.61 & RW2 \\ 
  Tanzania & TABORA & 15-19 & 37.82 & 15.76 & 87.85 & RW2 \\ 
  Tanzania & TANGA & 80-84 & 140.51 & 178.03 & 109.84 & HT-Direct \\ 
  Tanzania & TANGA & 80-84 & 180.18 & 146.70 & 218.11 & RW2 \\ 
  Tanzania & TANGA & 85-89 & 197.98 & 237.71 & 163.47 & HT-Direct \\ 
  Tanzania & TANGA & 85-89 & 181.83 & 160.82 & 205.39 & RW2 \\ 
  Tanzania & TANGA & 90-94 & 179.36 & 204.06 & 157.06 & HT-Direct \\ 
  Tanzania & TANGA & 90-94 & 171.19 & 154.93 & 188.92 & RW2 \\ 
  Tanzania & TANGA & 95-99 & 131.61 & 159.45 & 108.01 & HT-Direct \\ 
  Tanzania & TANGA & 95-99 & 151.25 & 135.28 & 168.88 & RW2 \\ 
  Tanzania & TANGA & 00-04 & 124.57 & 153.27 & 100.60 & HT-Direct \\ 
  Tanzania & TANGA & 00-04 & 111.51 & 97.45 & 127.49 & RW2 \\ 
  Tanzania & TANGA & 05-09 & 71.88 & 99.55 & 51.46 & HT-Direct \\ 
  Tanzania & TANGA & 05-09 & 71.48 & 59.27 & 86.28 & RW2 \\ 
  Tanzania & TANGA & 10-14 & 55.12 & 96.47 & 30.88 & HT-Direct \\ 
  Tanzania & TANGA & 10-14 & 46.23 & 33.39 & 62.86 & RW2 \\ 
  Tanzania & TANGA & 15-19 & 29.98 & 11.47 & 75.66 & RW2 \\ 
  Togo & ALL & 80-84 & 162.29 & 159.60 & 164.74 & IHME \\ 
  Togo & ALL & 80-84 & 169.92 & 156.97 & 183.70 & RW2 \\ 
  Togo & ALL & 80-84 & 169.65 & 163.43 & 175.54 & UN \\ 
  Togo & ALL & 85-89 & 150.47 & 148.21 & 152.78 & IHME \\ 
  Togo & ALL & 85-89 & 152.75 & 142.29 & 163.77 & RW2 \\ 
  Togo & ALL & 85-89 & 153.33 & 148.78 & 158.66 & UN \\ 
  Togo & ALL & 90-94 & 140.73 & 138.55 & 143.15 & IHME \\ 
  Togo & ALL & 90-94 & 142.54 & 134.78 & 150.69 & RW2 \\ 
  Togo & ALL & 90-94 & 142.10 & 137.90 & 146.75 & UN \\ 
  Togo & ALL & 95-99 & 130.01 & 127.73 & 132.37 & IHME \\ 
  Togo & ALL & 95-99 & 129.65 & 121.30 & 138.41 & RW2 \\ 
  Togo & ALL & 95-99 & 129.92 & 125.92 & 134.00 & UN \\ 
  Togo & ALL & 00-04 & 117.78 & 115.57 & 120.26 & IHME \\ 
  Togo & ALL & 00-04 & 114.36 & 104.74 & 124.80 & RW2 \\ 
  Togo & ALL & 00-04 & 114.23 & 110.54 & 118.01 & UN \\ 
  Togo & ALL & 05-09 & 104.17 & 101.75 & 106.69 & IHME \\ 
  Togo & ALL & 05-09 & 99.05 & 90.33 & 108.52 & RW2 \\ 
  Togo & ALL & 05-09 & 99.20 & 95.31 & 103.13 & UN \\ 
  Togo & ALL & 10-14 & 88.39 & 85.23 & 91.33 & IHME \\ 
  Togo & ALL & 10-14 & 85.70 & 76.61 & 95.67 & RW2 \\ 
  Togo & ALL & 10-14 & 85.64 & 80.54 & 90.83 & UN \\ 
  Togo & CENTRALE & 80-84 & 193.64 & 235.13 & 157.97 & HT-Direct \\ 
  Togo & CENTRALE & 80-84 & 180.86 & 156.40 & 209.14 & RW2 \\ 
  Togo & CENTRALE & 85-89 & 165.58 & 193.29 & 141.14 & HT-Direct \\ 
  Togo & CENTRALE & 85-89 & 156.78 & 140.99 & 173.81 & RW2 \\ 
  Togo & CENTRALE & 90-94 & 150.06 & 172.90 & 129.76 & HT-Direct \\ 
  Togo & CENTRALE & 90-94 & 147.04 & 134.06 & 160.82 & RW2 \\ 
  Togo & CENTRALE & 95-99 & 127.34 & 152.62 & 105.74 & HT-Direct \\ 
  Togo & CENTRALE & 95-99 & 134.78 & 121.12 & 148.94 & RW2 \\ 
  Togo & CENTRALE & 00-04 & 147.89 & 186.12 & 116.39 & HT-Direct \\ 
  Togo & CENTRALE & 00-04 & 127.54 & 112.86 & 143.72 & RW2 \\ 
  Togo & CENTRALE & 05-09 & 104.04 & 126.91 & 84.89 & HT-Direct \\ 
  Togo & CENTRALE & 05-09 & 115.96 & 101.64 & 132.25 & RW2 \\ 
  Togo & CENTRALE & 10-14 & 114.98 & 142.30 & 92.34 & HT-Direct \\ 
  Togo & CENTRALE & 10-14 & 104.12 & 87.65 & 123.71 & RW2 \\ 
  Togo & CENTRALE & 15-19 & 93.28 & 40.99 & 198.68 & RW2 \\ 
  Togo & GRANDE AGGLOMÉRATION DE LOMÉ & 80-84 & 131.45 & 181.08 & 93.87 & HT-Direct \\ 
  Togo & GRANDE AGGLOMÉRATION DE LOMÉ & 80-84 & 122.97 & 97.23 & 153.14 & RW2 \\ 
  Togo & GRANDE AGGLOMÉRATION DE LOMÉ & 85-89 & 100.95 & 129.16 & 78.35 & HT-Direct \\ 
  Togo & GRANDE AGGLOMÉRATION DE LOMÉ & 85-89 & 107.76 & 92.25 & 125.57 & RW2 \\ 
  Togo & GRANDE AGGLOMÉRATION DE LOMÉ & 90-94 & 105.17 & 129.66 & 84.85 & HT-Direct \\ 
  Togo & GRANDE AGGLOMÉRATION DE LOMÉ & 90-94 & 102.03 & 89.75 & 115.91 & RW2 \\ 
  Togo & GRANDE AGGLOMÉRATION DE LOMÉ & 95-99 & 105.65 & 131.83 & 84.16 & HT-Direct \\ 
  Togo & GRANDE AGGLOMÉRATION DE LOMÉ & 95-99 & 91.29 & 80.18 & 104.41 & RW2 \\ 
  Togo & GRANDE AGGLOMÉRATION DE LOMÉ & 00-04 & 80.15 & 103.13 & 61.93 & HT-Direct \\ 
  Togo & GRANDE AGGLOMÉRATION DE LOMÉ & 00-04 & 80.29 & 69.44 & 92.84 & RW2 \\ 
  Togo & GRANDE AGGLOMÉRATION DE LOMÉ & 05-09 & 61.64 & 79.11 & 47.82 & HT-Direct \\ 
  Togo & GRANDE AGGLOMÉRATION DE LOMÉ & 05-09 & 67.19 & 56.91 & 79.00 & RW2 \\ 
  Togo & GRANDE AGGLOMÉRATION DE LOMÉ & 10-14 & 56.67 & 75.03 & 42.59 & HT-Direct \\ 
  Togo & GRANDE AGGLOMÉRATION DE LOMÉ & 10-14 & 56.31 & 45.04 & 70.22 & RW2 \\ 
  Togo & GRANDE AGGLOMÉRATION DE LOMÉ & 15-19 & 47.35 & 19.71 & 109.85 & RW2 \\ 
  Togo & KARA & 80-84 & 195.98 & 238.72 & 159.29 & HT-Direct \\ 
  Togo & KARA & 80-84 & 197.38 & 171.50 & 226.44 & RW2 \\ 
  Togo & KARA & 85-89 & 182.88 & 210.30 & 158.32 & HT-Direct \\ 
  Togo & KARA & 85-89 & 174.61 & 158.34 & 192.28 & RW2 \\ 
  Togo & KARA & 90-94 & 182.71 & 206.33 & 161.24 & HT-Direct \\ 
  Togo & KARA & 90-94 & 166.46 & 153.41 & 180.33 & RW2 \\ 
  Togo & KARA & 95-99 & 137.33 & 159.88 & 117.51 & HT-Direct \\ 
  Togo & KARA & 95-99 & 152.98 & 139.37 & 167.54 & RW2 \\ 
  Togo & KARA & 00-04 & 158.44 & 196.44 & 126.63 & HT-Direct \\ 
  Togo & KARA & 00-04 & 143.79 & 128.20 & 161.20 & RW2 \\ 
  Togo & KARA & 05-09 & 141.04 & 168.28 & 117.59 & HT-Direct \\ 
  Togo & KARA & 05-09 & 127.65 & 112.77 & 144.34 & RW2 \\ 
  Togo & KARA & 10-14 & 102.27 & 124.39 & 83.70 & HT-Direct \\ 
  Togo & KARA & 10-14 & 109.12 & 92.99 & 127.45 & RW2 \\ 
  Togo & KARA & 15-19 & 92.15 & 40.78 & 195.64 & RW2 \\ 
  Togo & MARITIME (SANS AGGLOMÉRATION DE LOMÉ) & 80-84 & 159.26 & 195.63 & 128.58 & HT-Direct \\ 
  Togo & MARITIME (SANS AGGLOMÉRATION DE LOMÉ) & 80-84 & 161.95 & 137.42 & 189.61 & RW2 \\ 
  Togo & MARITIME (SANS AGGLOMÉRATION DE LOMÉ) & 85-89 & 154.28 & 192.27 & 122.66 & HT-Direct \\ 
  Togo & MARITIME (SANS AGGLOMÉRATION DE LOMÉ) & 85-89 & 142.21 & 125.59 & 160.50 & RW2 \\ 
  Togo & MARITIME (SANS AGGLOMÉRATION DE LOMÉ) & 90-94 & 139.09 & 160.78 & 119.90 & HT-Direct \\ 
  Togo & MARITIME (SANS AGGLOMÉRATION DE LOMÉ) & 90-94 & 134.08 & 121.53 & 148.01 & RW2 \\ 
  Togo & MARITIME (SANS AGGLOMÉRATION DE LOMÉ) & 95-99 & 142.11 & 169.17 & 118.75 & HT-Direct \\ 
  Togo & MARITIME (SANS AGGLOMÉRATION DE LOMÉ) & 95-99 & 120.29 & 107.53 & 134.43 & RW2 \\ 
  Togo & MARITIME (SANS AGGLOMÉRATION DE LOMÉ) & 00-04 & 76.84 & 106.96 & 54.69 & HT-Direct \\ 
  Togo & MARITIME (SANS AGGLOMÉRATION DE LOMÉ) & 00-04 & 107.04 & 92.06 & 124.03 & RW2 \\ 
  Togo & MARITIME (SANS AGGLOMÉRATION DE LOMÉ) & 05-09 & 79.65 & 112.58 & 55.74 & HT-Direct \\ 
  Togo & MARITIME (SANS AGGLOMÉRATION DE LOMÉ) & 05-09 & 91.54 & 75.55 & 110.08 & RW2 \\ 
  Togo & MARITIME (SANS AGGLOMÉRATION DE LOMÉ) & 10-14 & 88.47 & 123.68 & 62.56 & HT-Direct \\ 
  Togo & MARITIME (SANS AGGLOMÉRATION DE LOMÉ) & 10-14 & 78.37 & 60.89 & 100.40 & RW2 \\ 
  Togo & MARITIME (SANS AGGLOMÉRATION DE LOMÉ) & 15-19 & 67.60 & 28.23 & 153.30 & RW2 \\ 
  Togo & PLATEAUX & 80-84 & 180.33 & 210.53 & 153.63 & HT-Direct \\ 
  Togo & PLATEAUX & 80-84 & 169.28 & 149.72 & 191.19 & RW2 \\ 
  Togo & PLATEAUX & 85-89 & 146.69 & 166.04 & 129.24 & HT-Direct \\ 
  Togo & PLATEAUX & 85-89 & 146.56 & 133.74 & 160.48 & RW2 \\ 
  Togo & PLATEAUX & 90-94 & 146.52 & 169.98 & 125.81 & HT-Direct \\ 
  Togo & PLATEAUX & 90-94 & 137.39 & 125.64 & 149.74 & RW2 \\ 
  Togo & PLATEAUX & 95-99 & 122.96 & 143.85 & 104.74 & HT-Direct \\ 
  Togo & PLATEAUX & 95-99 & 124.53 & 112.85 & 137.00 & RW2 \\ 
  Togo & PLATEAUX & 00-04 & 116.97 & 140.22 & 97.13 & HT-Direct \\ 
  Togo & PLATEAUX & 00-04 & 115.17 & 102.74 & 128.89 & RW2 \\ 
  Togo & PLATEAUX & 05-09 & 101.42 & 123.60 & 82.84 & HT-Direct \\ 
  Togo & PLATEAUX & 05-09 & 102.82 & 89.98 & 117.20 & RW2 \\ 
  Togo & PLATEAUX & 10-14 & 95.34 & 121.10 & 74.59 & HT-Direct \\ 
  Togo & PLATEAUX & 10-14 & 91.39 & 76.11 & 109.92 & RW2 \\ 
  Togo & PLATEAUX & 15-19 & 81.55 & 35.70 & 176.49 & RW2 \\ 
  Togo & SAVANES & 80-84 & 223.03 & 250.47 & 197.79 & HT-Direct \\ 
  Togo & SAVANES & 80-84 & 210.24 & 189.68 & 232.44 & RW2 \\ 
  Togo & SAVANES & 85-89 & 171.90 & 194.50 & 151.44 & HT-Direct \\ 
  Togo & SAVANES & 85-89 & 182.51 & 167.64 & 198.35 & RW2 \\ 
  Togo & SAVANES & 90-94 & 183.47 & 199.91 & 168.10 & HT-Direct \\ 
  Togo & SAVANES & 90-94 & 171.59 & 160.42 & 183.46 & RW2 \\ 
  Togo & SAVANES & 95-99 & 165.85 & 187.96 & 145.88 & HT-Direct \\ 
  Togo & SAVANES & 95-99 & 154.15 & 141.46 & 167.88 & RW2 \\ 
  Togo & SAVANES & 00-04 & 135.77 & 159.50 & 115.08 & HT-Direct \\ 
  Togo & SAVANES & 00-04 & 138.35 & 124.24 & 154.18 & RW2 \\ 
  Togo & SAVANES & 05-09 & 117.79 & 140.33 & 98.46 & HT-Direct \\ 
  Togo & SAVANES & 05-09 & 115.39 & 101.63 & 130.70 & RW2 \\ 
  Togo & SAVANES & 10-14 & 84.06 & 108.25 & 64.88 & HT-Direct \\ 
  Togo & SAVANES & 10-14 & 91.60 & 74.68 & 110.83 & RW2 \\ 
  Togo & SAVANES & 15-19 & 71.78 & 30.07 & 158.96 & RW2 \\ 
  Uganda & ALL & 80-84 & 192.81 & 189.88 & 195.87 & IHME \\ 
  Uganda & ALL & 80-84 & 209.25 & 199.14 & 219.74 & RW2 \\ 
  Uganda & ALL & 80-84 & 209.20 & 203.14 & 215.58 & UN \\ 
  Uganda & ALL & 85-89 & 177.00 & 174.76 & 179.15 & IHME \\ 
  Uganda & ALL & 85-89 & 192.10 & 182.98 & 201.52 & RW2 \\ 
  Uganda & ALL & 85-89 & 192.14 & 187.02 & 197.24 & UN \\ 
  Uganda & ALL & 90-94 & 162.28 & 160.44 & 164.15 & IHME \\ 
  Uganda & ALL & 90-94 & 179.53 & 171.17 & 188.21 & RW2 \\ 
  Uganda & ALL & 90-94 & 179.55 & 174.34 & 184.65 & UN \\ 
  Uganda & ALL & 95-99 & 143.81 & 141.82 & 145.60 & IHME \\ 
  Uganda & ALL & 95-99 & 163.10 & 155.34 & 171.11 & RW2 \\ 
  Uganda & ALL & 95-99 & 163.03 & 158.73 & 167.42 & UN \\ 
  Uganda & ALL & 00-04 & 121.66 & 119.87 & 123.38 & IHME \\ 
  Uganda & ALL & 00-04 & 132.13 & 125.76 & 138.81 & RW2 \\ 
  Uganda & ALL & 00-04 & 132.14 & 128.32 & 136.05 & UN \\ 
  Uganda & ALL & 05-09 & 99.98 & 97.90 & 102.17 & IHME \\ 
  Uganda & ALL & 05-09 & 92.64 & 86.77 & 98.85 & RW2 \\ 
  Uganda & ALL & 05-09 & 92.82 & 89.49 & 96.21 & UN \\ 
  Uganda & ALL & 10-14 & 80.52 & 77.72 & 83.31 & IHME \\ 
  Uganda & ALL & 10-14 & 66.06 & 56.63 & 76.87 & RW2 \\ 
  Uganda & ALL & 10-14 & 65.56 & 59.96 & 71.10 & UN \\ 
  Uganda & CENTRAL & 80-84 & 199.34 & 216.78 & 182.98 & HT-Direct \\ 
  Uganda & CENTRAL & 80-84 & 205.05 & 188.71 & 222.59 & RW2 \\ 
  Uganda & CENTRAL & 85-89 & 160.57 & 174.67 & 147.41 & HT-Direct \\ 
  Uganda & CENTRAL & 85-89 & 179.52 & 167.05 & 192.61 & RW2 \\ 
  Uganda & CENTRAL & 90-94 & 141.10 & 153.63 & 129.44 & HT-Direct \\ 
  Uganda & CENTRAL & 90-94 & 159.40 & 147.61 & 171.34 & RW2 \\ 
  Uganda & CENTRAL & 95-99 & 130.62 & 146.02 & 116.62 & HT-Direct \\ 
  Uganda & CENTRAL & 95-99 & 143.28 & 132.09 & 154.87 & RW2 \\ 
  Uganda & CENTRAL & 00-04 & 121.10 & 134.73 & 108.68 & HT-Direct \\ 
  Uganda & CENTRAL & 00-04 & 116.86 & 107.73 & 127.25 & RW2 \\ 
  Uganda & CENTRAL & 05-09 & 99.55 & 116.96 & 84.49 & HT-Direct \\ 
  Uganda & CENTRAL & 05-09 & 80.79 & 72.01 & 90.69 & RW2 \\ 
  Uganda & CENTRAL & 10-14 & 49.81 & 76.54 & 32.10 & HT-Direct \\ 
  Uganda & CENTRAL & 10-14 & 54.13 & 41.62 & 68.35 & RW2 \\ 
  Uganda & CENTRAL & 15-19 & 36.07 & 13.96 & 87.93 & RW2 \\ 
  Uganda & EASTERN & 80-84 & 206.16 & 224.77 & 188.72 & HT-Direct \\ 
  Uganda & EASTERN & 80-84 & 219.33 & 201.13 & 238.52 & RW2 \\ 
  Uganda & EASTERN & 85-89 & 183.74 & 198.11 & 170.20 & HT-Direct \\ 
  Uganda & EASTERN & 85-89 & 201.02 & 187.34 & 216.11 & RW2 \\ 
  Uganda & EASTERN & 90-94 & 164.58 & 178.53 & 151.51 & HT-Direct \\ 
  Uganda & EASTERN & 90-94 & 175.65 & 163.41 & 189.44 & RW2 \\ 
  Uganda & EASTERN & 95-99 & 136.88 & 148.83 & 125.75 & HT-Direct \\ 
  Uganda & EASTERN & 95-99 & 145.83 & 135.58 & 156.33 & RW2 \\ 
  Uganda & EASTERN & 00-04 & 108.13 & 119.40 & 97.81 & HT-Direct \\ 
  Uganda & EASTERN & 00-04 & 112.33 & 103.32 & 121.50 & RW2 \\ 
  Uganda & EASTERN & 05-09 & 92.10 & 104.86 & 80.75 & HT-Direct \\ 
  Uganda & EASTERN & 05-09 & 81.19 & 73.51 & 89.75 & RW2 \\ 
  Uganda & EASTERN & 10-14 & 92.51 & 133.51 & 63.18 & HT-Direct \\ 
  Uganda & EASTERN & 10-14 & 61.75 & 48.99 & 79.03 & RW2 \\ 
  Uganda & EASTERN & 15-19 & 47.94 & 19.37 & 117.89 & RW2 \\ 
  Uganda & NORTHERN & 80-84 & 240.53 & 267.97 & 215.08 & HT-Direct \\ 
  Uganda & NORTHERN & 80-84 & 249.69 & 225.48 & 275.59 & RW2 \\ 
  Uganda & NORTHERN & 85-89 & 207.01 & 226.98 & 188.36 & HT-Direct \\ 
  Uganda & NORTHERN & 85-89 & 229.40 & 212.64 & 246.74 & RW2 \\ 
  Uganda & NORTHERN & 90-94 & 187.23 & 202.19 & 173.14 & HT-Direct \\ 
  Uganda & NORTHERN & 90-94 & 210.86 & 197.25 & 225.08 & RW2 \\ 
  Uganda & NORTHERN & 95-99 & 183.51 & 198.25 & 169.62 & HT-Direct \\ 
  Uganda & NORTHERN & 95-99 & 191.99 & 180.23 & 204.40 & RW2 \\ 
  Uganda & NORTHERN & 00-04 & 160.86 & 174.74 & 147.88 & HT-Direct \\ 
  Uganda & NORTHERN & 00-04 & 155.41 & 145.46 & 166.10 & RW2 \\ 
  Uganda & NORTHERN & 05-09 & 113.60 & 130.90 & 98.32 & HT-Direct \\ 
  Uganda & NORTHERN & 05-09 & 108.29 & 97.62 & 120.03 & RW2 \\ 
  Uganda & NORTHERN & 10-14 & 119.45 & 185.76 & 74.64 & HT-Direct \\ 
  Uganda & NORTHERN & 10-14 & 77.15 & 60.67 & 97.50 & RW2 \\ 
  Uganda & NORTHERN & 15-19 & 55.68 & 22.33 & 132.93 & RW2 \\ 
  Uganda & WESTERN & 80-84 & 180.34 & 196.02 & 165.64 & HT-Direct \\ 
  Uganda & WESTERN & 80-84 & 186.09 & 171.22 & 201.88 & RW2 \\ 
  Uganda & WESTERN & 85-89 & 157.84 & 171.68 & 144.92 & HT-Direct \\ 
  Uganda & WESTERN & 85-89 & 180.95 & 167.60 & 194.57 & RW2 \\ 
  Uganda & WESTERN & 90-94 & 160.98 & 175.76 & 147.22 & HT-Direct \\ 
  Uganda & WESTERN & 90-94 & 180.79 & 167.74 & 194.52 & RW2 \\ 
  Uganda & WESTERN & 95-99 & 178.83 & 195.93 & 162.91 & HT-Direct \\ 
  Uganda & WESTERN & 95-99 & 177.12 & 164.32 & 191.62 & RW2 \\ 
  Uganda & WESTERN & 00-04 & 147.38 & 162.37 & 133.56 & HT-Direct \\ 
  Uganda & WESTERN & 00-04 & 146.32 & 135.77 & 157.89 & RW2 \\ 
  Uganda & WESTERN & 05-09 & 106.91 & 122.47 & 93.13 & HT-Direct \\ 
  Uganda & WESTERN & 05-09 & 101.81 & 91.88 & 112.39 & RW2 \\ 
  Uganda & WESTERN & 10-14 & 103.80 & 135.27 & 78.98 & HT-Direct \\ 
  Uganda & WESTERN & 10-14 & 72.82 & 60.48 & 87.38 & RW2 \\ 
  Uganda & WESTERN & 15-19 & 52.94 & 22.33 & 121.53 & RW2 \\ 
  Zambia & ALL & 80-84 & 160.63 & 156.43 & 164.93 & IHME \\ 
  Zambia & ALL & 80-84 & 162.11 & 154.29 & 170.24 & RW2 \\ 
  Zambia & ALL & 80-84 & 162.09 & 157.42 & 167.22 & UN \\ 
  Zambia & ALL & 85-89 & 170.45 & 166.39 & 174.58 & IHME \\ 
  Zambia & ALL & 85-89 & 183.49 & 176.41 & 190.76 & RW2 \\ 
  Zambia & ALL & 85-89 & 183.57 & 178.67 & 188.63 & UN \\ 
  Zambia & ALL & 90-94 & 168.58 & 164.39 & 172.06 & IHME \\ 
  Zambia & ALL & 90-94 & 188.48 & 181.48 & 195.69 & RW2 \\ 
  Zambia & ALL & 90-94 & 188.27 & 183.24 & 193.10 & UN \\ 
  Zambia & ALL & 95-99 & 154.77 & 150.93 & 158.70 & IHME \\ 
  Zambia & ALL & 95-99 & 174.79 & 166.50 & 183.32 & RW2 \\ 
  Zambia & ALL & 95-99 & 175.17 & 170.41 & 180.08 & UN \\ 
  Zambia & ALL & 00-04 & 130.72 & 126.83 & 134.32 & IHME \\ 
  Zambia & ALL & 00-04 & 142.23 & 134.50 & 150.41 & RW2 \\ 
  Zambia & ALL & 00-04 & 141.89 & 137.34 & 147.01 & UN \\ 
  Zambia & ALL & 05-09 & 98.98 & 95.39 & 102.72 & IHME \\ 
  Zambia & ALL & 05-09 & 98.44 & 90.12 & 107.34 & RW2 \\ 
  Zambia & ALL & 05-09 & 98.69 & 95.12 & 102.47 & UN \\ 
  Zambia & ALL & 10-14 & 74.07 & 70.17 & 78.06 & IHME \\ 
  Zambia & ALL & 10-14 & 74.47 & 67.75 & 81.74 & RW2 \\ 
  Zambia & ALL & 10-14 & 74.30 & 70.13 & 78.89 & UN \\ 
  Zambia & CENTRAL & 80-84 & 139.03 & 166.84 & 115.21 & HT-Direct \\ 
  Zambia & CENTRAL & 80-84 & 141.44 & 123.97 & 160.98 & RW2 \\ 
  Zambia & CENTRAL & 85-89 & 159.22 & 180.17 & 140.30 & HT-Direct \\ 
  Zambia & CENTRAL & 85-89 & 162.15 & 148.34 & 176.89 & RW2 \\ 
  Zambia & CENTRAL & 90-94 & 155.41 & 177.30 & 135.77 & HT-Direct \\ 
  Zambia & CENTRAL & 90-94 & 170.36 & 157.73 & 183.94 & RW2 \\ 
  Zambia & CENTRAL & 95-99 & 154.71 & 175.45 & 136.01 & HT-Direct \\ 
  Zambia & CENTRAL & 95-99 & 160.88 & 147.78 & 174.88 & RW2 \\ 
  Zambia & CENTRAL & 00-04 & 115.39 & 134.57 & 98.63 & HT-Direct \\ 
  Zambia & CENTRAL & 00-04 & 131.47 & 118.55 & 145.74 & RW2 \\ 
  Zambia & CENTRAL & 05-09 & 89.09 & 113.77 & 69.35 & HT-Direct \\ 
  Zambia & CENTRAL & 05-09 & 94.23 & 80.87 & 109.38 & RW2 \\ 
  Zambia & CENTRAL & 10-14 & 62.44 & 82.39 & 47.07 & HT-Direct \\ 
  Zambia & CENTRAL & 10-14 & 71.10 & 58.09 & 86.68 & RW2 \\ 
  Zambia & CENTRAL & 15-19 & 55.20 & 23.65 & 124.47 & RW2 \\ 
  Zambia & COPPERBELT & 80-84 & 115.91 & 135.07 & 99.16 & HT-Direct \\ 
  Zambia & COPPERBELT & 80-84 & 126.88 & 111.37 & 143.62 & RW2 \\ 
  Zambia & COPPERBELT & 85-89 & 153.23 & 170.44 & 137.47 & HT-Direct \\ 
  Zambia & COPPERBELT & 85-89 & 152.56 & 140.97 & 165.04 & RW2 \\ 
  Zambia & COPPERBELT & 90-94 & 163.17 & 180.78 & 146.97 & HT-Direct \\ 
  Zambia & COPPERBELT & 90-94 & 163.91 & 151.80 & 177.13 & RW2 \\ 
  Zambia & COPPERBELT & 95-99 & 135.13 & 155.96 & 116.69 & HT-Direct \\ 
  Zambia & COPPERBELT & 95-99 & 153.78 & 139.90 & 169.01 & RW2 \\ 
  Zambia & COPPERBELT & 00-04 & 113.06 & 134.66 & 94.54 & HT-Direct \\ 
  Zambia & COPPERBELT & 00-04 & 123.56 & 109.89 & 138.65 & RW2 \\ 
  Zambia & COPPERBELT & 05-09 & 61.52 & 78.03 & 48.33 & HT-Direct \\ 
  Zambia & COPPERBELT & 05-09 & 86.87 & 73.52 & 102.11 & RW2 \\ 
  Zambia & COPPERBELT & 10-14 & 68.20 & 92.24 & 50.09 & HT-Direct \\ 
  Zambia & COPPERBELT & 10-14 & 65.05 & 51.13 & 82.02 & RW2 \\ 
  Zambia & COPPERBELT & 15-19 & 50.17 & 20.84 & 117.01 & RW2 \\ 
  Zambia & EASTERN & 80-84 & 225.13 & 250.32 & 201.80 & HT-Direct \\ 
  Zambia & EASTERN & 80-84 & 224.73 & 204.93 & 246.95 & RW2 \\ 
  Zambia & EASTERN & 85-89 & 237.44 & 256.59 & 219.29 & HT-Direct \\ 
  Zambia & EASTERN & 85-89 & 237.56 & 224.00 & 251.88 & RW2 \\ 
  Zambia & EASTERN & 90-94 & 223.14 & 239.77 & 207.36 & HT-Direct \\ 
  Zambia & EASTERN & 90-94 & 233.48 & 220.22 & 246.87 & RW2 \\ 
  Zambia & EASTERN & 95-99 & 174.24 & 192.45 & 157.42 & HT-Direct \\ 
  Zambia & EASTERN & 95-99 & 212.28 & 196.76 & 227.70 & RW2 \\ 
  Zambia & EASTERN & 00-04 & 155.98 & 175.99 & 137.86 & HT-Direct \\ 
  Zambia & EASTERN & 00-04 & 174.63 & 160.20 & 189.88 & RW2 \\ 
  Zambia & EASTERN & 05-09 & 122.48 & 140.91 & 106.16 & HT-Direct \\ 
  Zambia & EASTERN & 05-09 & 129.50 & 115.13 & 145.49 & RW2 \\ 
  Zambia & EASTERN & 10-14 & 102.80 & 125.31 & 83.94 & HT-Direct \\ 
  Zambia & EASTERN & 10-14 & 101.32 & 85.70 & 120.46 & RW2 \\ 
  Zambia & EASTERN & 15-19 & 81.46 & 35.55 & 177.18 & RW2 \\ 
  Zambia & LUAPULA & 80-84 & 209.28 & 232.85 & 187.51 & HT-Direct \\ 
  Zambia & LUAPULA & 80-84 & 217.36 & 196.60 & 239.18 & RW2 \\ 
  Zambia & LUAPULA & 85-89 & 245.51 & 270.35 & 222.27 & HT-Direct \\ 
  Zambia & LUAPULA & 85-89 & 244.74 & 227.99 & 262.60 & RW2 \\ 
  Zambia & LUAPULA & 90-94 & 238.09 & 261.43 & 216.22 & HT-Direct \\ 
  Zambia & LUAPULA & 90-94 & 248.00 & 230.90 & 266.23 & RW2 \\ 
  Zambia & LUAPULA & 95-99 & 208.33 & 232.79 & 185.81 & HT-Direct \\ 
  Zambia & LUAPULA & 95-99 & 222.94 & 204.56 & 242.41 & RW2 \\ 
  Zambia & LUAPULA & 00-04 & 136.74 & 160.11 & 116.31 & HT-Direct \\ 
  Zambia & LUAPULA & 00-04 & 170.91 & 152.08 & 190.78 & RW2 \\ 
  Zambia & LUAPULA & 05-09 & 103.77 & 132.77 & 80.51 & HT-Direct \\ 
  Zambia & LUAPULA & 05-09 & 115.40 & 97.04 & 136.61 & RW2 \\ 
  Zambia & LUAPULA & 10-14 & 85.84 & 119.46 & 61.03 & HT-Direct \\ 
  Zambia & LUAPULA & 10-14 & 81.73 & 62.58 & 106.53 & RW2 \\ 
  Zambia & LUAPULA & 15-19 & 59.06 & 23.86 & 141.61 & RW2 \\ 
  Zambia & LUSAKA & 80-84 & 120.34 & 139.56 & 103.45 & HT-Direct \\ 
  Zambia & LUSAKA & 80-84 & 122.99 & 108.67 & 138.45 & RW2 \\ 
  Zambia & LUSAKA & 85-89 & 133.64 & 150.01 & 118.82 & HT-Direct \\ 
  Zambia & LUSAKA & 85-89 & 145.29 & 133.75 & 157.32 & RW2 \\ 
  Zambia & LUSAKA & 90-94 & 170.62 & 191.12 & 151.91 & HT-Direct \\ 
  Zambia & LUSAKA & 90-94 & 156.96 & 145.08 & 169.90 & RW2 \\ 
  Zambia & LUSAKA & 95-99 & 111.96 & 131.64 & 94.91 & HT-Direct \\ 
  Zambia & LUSAKA & 95-99 & 150.31 & 136.84 & 165.11 & RW2 \\ 
  Zambia & LUSAKA & 00-04 & 128.59 & 149.42 & 110.29 & HT-Direct \\ 
  Zambia & LUSAKA & 00-04 & 125.63 & 112.60 & 140.08 & RW2 \\ 
  Zambia & LUSAKA & 05-09 & 60.99 & 80.81 & 45.79 & HT-Direct \\ 
  Zambia & LUSAKA & 05-09 & 92.13 & 78.62 & 107.55 & RW2 \\ 
  Zambia & LUSAKA & 10-14 & 67.92 & 87.77 & 52.31 & HT-Direct \\ 
  Zambia & LUSAKA & 10-14 & 71.54 & 57.96 & 87.67 & RW2 \\ 
  Zambia & LUSAKA & 15-19 & 56.98 & 24.38 & 129.54 & RW2 \\ 
  Zambia & NORTH-WESTERN & 80-84 & 151.29 & 171.09 & 133.40 & HT-Direct \\ 
  Zambia & NORTH-WESTERN & 80-84 & 150.24 & 134.84 & 167.14 & RW2 \\ 
  Zambia & NORTH-WESTERN & 85-89 & 155.72 & 176.06 & 137.34 & HT-Direct \\ 
  Zambia & NORTH-WESTERN & 85-89 & 162.18 & 149.90 & 175.08 & RW2 \\ 
  Zambia & NORTH-WESTERN & 90-94 & 156.39 & 171.39 & 142.47 & HT-Direct \\ 
  Zambia & NORTH-WESTERN & 90-94 & 161.43 & 150.37 & 172.97 & RW2 \\ 
  Zambia & NORTH-WESTERN & 95-99 & 120.63 & 137.15 & 105.86 & HT-Direct \\ 
  Zambia & NORTH-WESTERN & 95-99 & 144.85 & 132.62 & 157.72 & RW2 \\ 
  Zambia & NORTH-WESTERN & 00-04 & 109.08 & 128.30 & 92.43 & HT-Direct \\ 
  Zambia & NORTH-WESTERN & 00-04 & 113.42 & 101.48 & 126.59 & RW2 \\ 
  Zambia & NORTH-WESTERN & 05-09 & 63.32 & 86.99 & 45.76 & HT-Direct \\ 
  Zambia & NORTH-WESTERN & 05-09 & 77.99 & 66.34 & 91.71 & RW2 \\ 
  Zambia & NORTH-WESTERN & 10-14 & 56.94 & 72.97 & 44.26 & HT-Direct \\ 
  Zambia & NORTH-WESTERN & 10-14 & 56.98 & 46.30 & 70.27 & RW2 \\ 
  Zambia & NORTH-WESTERN & 15-19 & 42.85 & 18.19 & 99.74 & RW2 \\ 
  Zambia & NORTHERN & 80-84 & 187.68 & 210.16 & 167.10 & HT-Direct \\ 
  Zambia & NORTHERN & 80-84 & 190.68 & 172.76 & 209.89 & RW2 \\ 
  Zambia & NORTHERN & 85-89 & 210.22 & 231.13 & 190.73 & HT-Direct \\ 
  Zambia & NORTHERN & 85-89 & 215.98 & 201.95 & 230.74 & RW2 \\ 
  Zambia & NORTHERN & 90-94 & 204.84 & 223.59 & 187.29 & HT-Direct \\ 
  Zambia & NORTHERN & 90-94 & 224.00 & 210.66 & 238.15 & RW2 \\ 
  Zambia & NORTHERN & 95-99 & 202.68 & 221.12 & 185.41 & HT-Direct \\ 
  Zambia & NORTHERN & 95-99 & 209.18 & 195.22 & 224.38 & RW2 \\ 
  Zambia & NORTHERN & 00-04 & 154.61 & 173.16 & 137.71 & HT-Direct \\ 
  Zambia & NORTHERN & 00-04 & 166.47 & 153.05 & 181.21 & RW2 \\ 
  Zambia & NORTHERN & 05-09 & 97.49 & 113.13 & 83.80 & HT-Direct \\ 
  Zambia & NORTHERN & 05-09 & 113.94 & 100.90 & 128.16 & RW2 \\ 
  Zambia & NORTHERN & 10-14 & 68.49 & 85.18 & 54.88 & HT-Direct \\ 
  Zambia & NORTHERN & 10-14 & 80.42 & 66.77 & 95.55 & RW2 \\ 
  Zambia & NORTHERN & 15-19 & 57.84 & 24.84 & 129.04 & RW2 \\ 
  Zambia & SOUTHERN & 80-84 & 133.64 & 152.85 & 116.51 & HT-Direct \\ 
  Zambia & SOUTHERN & 80-84 & 132.61 & 118.27 & 148.39 & RW2 \\ 
  Zambia & SOUTHERN & 85-89 & 140.17 & 156.59 & 125.21 & HT-Direct \\ 
  Zambia & SOUTHERN & 85-89 & 147.88 & 136.65 & 159.79 & RW2 \\ 
  Zambia & SOUTHERN & 90-94 & 151.73 & 170.32 & 134.84 & HT-Direct \\ 
  Zambia & SOUTHERN & 90-94 & 152.38 & 140.89 & 164.43 & RW2 \\ 
  Zambia & SOUTHERN & 95-99 & 122.63 & 141.13 & 106.25 & HT-Direct \\ 
  Zambia & SOUTHERN & 95-99 & 141.17 & 128.88 & 154.25 & RW2 \\ 
  Zambia & SOUTHERN & 00-04 & 101.32 & 116.45 & 87.96 & HT-Direct \\ 
  Zambia & SOUTHERN & 00-04 & 114.03 & 102.84 & 126.39 & RW2 \\ 
  Zambia & SOUTHERN & 05-09 & 65.83 & 83.22 & 51.88 & HT-Direct \\ 
  Zambia & SOUTHERN & 05-09 & 81.51 & 70.28 & 94.23 & RW2 \\ 
  Zambia & SOUTHERN & 10-14 & 64.05 & 80.96 & 50.48 & HT-Direct \\ 
  Zambia & SOUTHERN & 10-14 & 62.11 & 51.18 & 75.36 & RW2 \\ 
  Zambia & SOUTHERN & 15-19 & 48.71 & 20.96 & 111.45 & RW2 \\ 
  Zambia & WESTERN & 80-84 & 209.67 & 234.78 & 186.59 & HT-Direct \\ 
  Zambia & WESTERN & 80-84 & 208.63 & 188.23 & 230.49 & RW2 \\ 
  Zambia & WESTERN & 85-89 & 207.78 & 235.99 & 182.14 & HT-Direct \\ 
  Zambia & WESTERN & 85-89 & 217.38 & 200.70 & 234.89 & RW2 \\ 
  Zambia & WESTERN & 90-94 & 187.88 & 210.81 & 166.92 & HT-Direct \\ 
  Zambia & WESTERN & 90-94 & 211.43 & 196.01 & 227.72 & RW2 \\ 
  Zambia & WESTERN & 95-99 & 190.19 & 213.64 & 168.77 & HT-Direct \\ 
  Zambia & WESTERN & 95-99 & 186.99 & 171.42 & 204.08 & RW2 \\ 
  Zambia & WESTERN & 00-04 & 128.18 & 152.08 & 107.57 & HT-Direct \\ 
  Zambia & WESTERN & 00-04 & 141.85 & 126.27 & 159.09 & RW2 \\ 
  Zambia & WESTERN & 05-09 & 76.02 & 104.75 & 54.69 & HT-Direct \\ 
  Zambia & WESTERN & 05-09 & 92.84 & 77.56 & 110.72 & RW2 \\ 
  Zambia & WESTERN & 10-14 & 55.93 & 79.49 & 39.05 & HT-Direct \\ 
  Zambia & WESTERN & 10-14 & 63.51 & 48.98 & 81.41 & RW2 \\ 
  Zambia & WESTERN & 15-19 & 44.33 & 18.33 & 104.42 & RW2 \\ 
  Zimbabwe & ALL & 80-84 & 83.19 & 81.66 & 84.71 & IHME \\ 
  Zimbabwe & ALL & 80-84 & 94.31 & 86.38 & 102.91 & RW2 \\ 
  Zimbabwe & ALL & 80-84 & 94.36 & 91.13 & 97.64 & UN \\ 
  Zimbabwe & ALL & 85-89 & 69.14 & 68.10 & 70.27 & IHME \\ 
  Zimbabwe & ALL & 85-89 & 75.82 & 69.42 & 82.69 & RW2 \\ 
  Zimbabwe & ALL & 85-89 & 75.76 & 73.21 & 78.66 & UN \\ 
  Zimbabwe & ALL & 90-94 & 66.99 & 65.87 & 68.08 & IHME \\ 
  Zimbabwe & ALL & 90-94 & 83.27 & 75.84 & 91.36 & RW2 \\ 
  Zimbabwe & ALL & 90-94 & 83.14 & 80.33 & 86.01 & UN \\ 
  Zimbabwe & ALL & 95-99 & 74.12 & 72.87 & 75.48 & IHME \\ 
  Zimbabwe & ALL & 95-99 & 100.70 & 89.87 & 112.71 & RW2 \\ 
  Zimbabwe & ALL & 95-99 & 101.16 & 97.31 & 105.10 & UN \\ 
  Zimbabwe & ALL & 00-04 & 81.03 & 79.51 & 82.40 & IHME \\ 
  Zimbabwe & ALL & 00-04 & 104.27 & 92.34 & 117.61 & RW2 \\ 
  Zimbabwe & ALL & 00-04 & 104.55 & 100.33 & 109.08 & UN \\ 
  Zimbabwe & ALL & 05-09 & 81.90 & 79.93 & 83.63 & IHME \\ 
  Zimbabwe & ALL & 05-09 & 97.79 & 89.06 & 107.28 & RW2 \\ 
  Zimbabwe & ALL & 05-09 & 97.60 & 93.32 & 102.22 & UN \\ 
  Zimbabwe & ALL & 10-14 & 68.31 & 65.57 & 71.12 & IHME \\ 
  Zimbabwe & ALL & 10-14 & 89.65 & 30.17 & 235.98 & RW2 \\ 
  Zimbabwe & ALL & 10-14 & 81.03 & 73.89 & 88.62 & UN \\ 
  Zimbabwe & BULAWAYO & 80-84 & 56.78 & 80.95 & 39.52 & HT-Direct \\ 
  Zimbabwe & BULAWAYO & 80-84 & 57.19 & 42.20 & 77.19 & RW2 \\ 
  Zimbabwe & BULAWAYO & 85-89 & 37.93 & 56.14 & 25.46 & HT-Direct \\ 
  Zimbabwe & BULAWAYO & 85-89 & 46.86 & 36.77 & 59.54 & RW2 \\ 
  Zimbabwe & BULAWAYO & 90-94 & 41.02 & 60.73 & 27.52 & HT-Direct \\ 
  Zimbabwe & BULAWAYO & 90-94 & 55.60 & 44.28 & 69.69 & RW2 \\ 
  Zimbabwe & BULAWAYO & 95-99 & 77.88 & 105.46 & 57.06 & HT-Direct \\ 
  Zimbabwe & BULAWAYO & 95-99 & 77.64 & 61.27 & 98.21 & RW2 \\ 
  Zimbabwe & BULAWAYO & 00-04 & 73.18 & 105.05 & 50.44 & HT-Direct \\ 
  Zimbabwe & BULAWAYO & 00-04 & 76.29 & 58.43 & 99.21 & RW2 \\ 
  Zimbabwe & BULAWAYO & 05-09 & 48.23 & 64.27 & 36.04 & HT-Direct \\ 
  Zimbabwe & BULAWAYO & 05-09 & 64.13 & 47.44 & 86.20 & RW2 \\ 
  Zimbabwe & BULAWAYO & 10-14 & 53.07 & 16.78 & 160.27 & RW2 \\ 
  Zimbabwe & BULAWAYO & 15-19 & 43.35 & 2.57 & 463.19 & RW2 \\ 
  Zimbabwe & HARARE & 80-84 & 61.37 & 87.32 & 42.76 & HT-Direct \\ 
  Zimbabwe & HARARE & 80-84 & 68.29 & 51.50 & 89.95 & RW2 \\ 
  Zimbabwe & HARARE & 85-89 & 55.35 & 75.34 & 40.43 & HT-Direct \\ 
  Zimbabwe & HARARE & 85-89 & 57.53 & 46.29 & 71.19 & RW2 \\ 
  Zimbabwe & HARARE & 90-94 & 55.23 & 76.41 & 39.66 & HT-Direct \\ 
  Zimbabwe & HARARE & 90-94 & 68.99 & 56.38 & 84.34 & RW2 \\ 
  Zimbabwe & HARARE & 95-99 & 80.53 & 113.45 & 56.56 & HT-Direct \\ 
  Zimbabwe & HARARE & 95-99 & 96.65 & 77.48 & 120.57 & RW2 \\ 
  Zimbabwe & HARARE & 00-04 & 77.85 & 108.96 & 55.08 & HT-Direct \\ 
  Zimbabwe & HARARE & 00-04 & 96.16 & 74.29 & 123.64 & RW2 \\ 
  Zimbabwe & HARARE & 05-09 & 63.85 & 85.08 & 47.64 & HT-Direct \\ 
  Zimbabwe & HARARE & 05-09 & 82.78 & 61.47 & 110.32 & RW2 \\ 
  Zimbabwe & HARARE & 10-14 & 70.20 & 22.76 & 202.06 & RW2 \\ 
  Zimbabwe & HARARE & 15-19 & 59.55 & 3.63 & 541.77 & RW2 \\ 
  Zimbabwe & MANICALAND & 80-84 & 115.70 & 141.13 & 94.34 & HT-Direct \\ 
  Zimbabwe & MANICALAND & 80-84 & 111.94 & 93.48 & 133.71 & RW2 \\ 
  Zimbabwe & MANICALAND & 85-89 & 75.61 & 95.13 & 59.83 & HT-Direct \\ 
  Zimbabwe & MANICALAND & 85-89 & 93.95 & 80.70 & 109.12 & RW2 \\ 
  Zimbabwe & MANICALAND & 90-94 & 96.11 & 115.87 & 79.41 & HT-Direct \\ 
  Zimbabwe & MANICALAND & 90-94 & 112.84 & 96.86 & 130.65 & RW2 \\ 
  Zimbabwe & MANICALAND & 95-99 & 122.78 & 156.90 & 95.24 & HT-Direct \\ 
  Zimbabwe & MANICALAND & 95-99 & 157.80 & 131.55 & 188.56 & RW2 \\ 
  Zimbabwe & MANICALAND & 00-04 & 125.63 & 157.98 & 99.12 & HT-Direct \\ 
  Zimbabwe & MANICALAND & 00-04 & 160.39 & 130.72 & 195.71 & RW2 \\ 
  Zimbabwe & MANICALAND & 05-09 & 122.91 & 153.24 & 97.89 & HT-Direct \\ 
  Zimbabwe & MANICALAND & 05-09 & 142.10 & 112.54 & 177.86 & RW2 \\ 
  Zimbabwe & MANICALAND & 10-14 & 123.37 & 41.90 & 319.44 & RW2 \\ 
  Zimbabwe & MANICALAND & 15-19 & 106.67 & 6.88 & 688.63 & RW2 \\ 
  Zimbabwe & MASHONALAND CENTRAL & 80-84 & 143.98 & 167.49 & 123.28 & HT-Direct \\ 
  Zimbabwe & MASHONALAND CENTRAL & 80-84 & 144.76 & 125.40 & 166.79 & RW2 \\ 
  Zimbabwe & MASHONALAND CENTRAL & 85-89 & 96.14 & 118.66 & 77.52 & HT-Direct \\ 
  Zimbabwe & MASHONALAND CENTRAL & 85-89 & 110.65 & 96.27 & 126.98 & RW2 \\ 
  Zimbabwe & MASHONALAND CENTRAL & 90-94 & 96.26 & 120.80 & 76.28 & HT-Direct \\ 
  Zimbabwe & MASHONALAND CENTRAL & 90-94 & 118.62 & 101.74 & 137.69 & RW2 \\ 
  Zimbabwe & MASHONALAND CENTRAL & 95-99 & 111.69 & 140.45 & 88.21 & HT-Direct \\ 
  Zimbabwe & MASHONALAND CENTRAL & 95-99 & 148.64 & 124.45 & 176.39 & RW2 \\ 
  Zimbabwe & MASHONALAND CENTRAL & 00-04 & 104.70 & 133.12 & 81.77 & HT-Direct \\ 
  Zimbabwe & MASHONALAND CENTRAL & 00-04 & 135.94 & 111.26 & 165.03 & RW2 \\ 
  Zimbabwe & MASHONALAND CENTRAL & 05-09 & 94.18 & 114.24 & 77.34 & HT-Direct \\ 
  Zimbabwe & MASHONALAND CENTRAL & 05-09 & 108.00 & 87.77 & 132.19 & RW2 \\ 
  Zimbabwe & MASHONALAND CENTRAL & 10-14 & 83.94 & 28.27 & 231.48 & RW2 \\ 
  Zimbabwe & MASHONALAND CENTRAL & 15-19 & 65.22 & 4.00 & 558.63 & RW2 \\ 
  Zimbabwe & MASHONALAND EAST & 80-84 & 72.89 & 99.69 & 52.87 & HT-Direct \\ 
  Zimbabwe & MASHONALAND EAST & 80-84 & 86.79 & 68.40 & 109.15 & RW2 \\ 
  Zimbabwe & MASHONALAND EAST & 85-89 & 75.65 & 95.97 & 59.35 & HT-Direct \\ 
  Zimbabwe & MASHONALAND EAST & 85-89 & 74.52 & 62.27 & 88.99 & RW2 \\ 
  Zimbabwe & MASHONALAND EAST & 90-94 & 72.74 & 97.42 & 53.94 & HT-Direct \\ 
  Zimbabwe & MASHONALAND EAST & 90-94 & 90.61 & 76.77 & 106.79 & RW2 \\ 
  Zimbabwe & MASHONALAND EAST & 95-99 & 114.25 & 140.56 & 92.34 & HT-Direct \\ 
  Zimbabwe & MASHONALAND EAST & 95-99 & 127.98 & 107.46 & 152.02 & RW2 \\ 
  Zimbabwe & MASHONALAND EAST & 00-04 & 68.65 & 95.90 & 48.73 & HT-Direct \\ 
  Zimbabwe & MASHONALAND EAST & 00-04 & 129.48 & 105.37 & 157.67 & RW2 \\ 
  Zimbabwe & MASHONALAND EAST & 05-09 & 96.47 & 116.86 & 79.32 & HT-Direct \\ 
  Zimbabwe & MASHONALAND EAST & 05-09 & 113.82 & 92.45 & 139.48 & RW2 \\ 
  Zimbabwe & MASHONALAND EAST & 10-14 & 98.29 & 33.97 & 263.36 & RW2 \\ 
  Zimbabwe & MASHONALAND EAST & 15-19 & 84.35 & 5.55 & 627.45 & RW2 \\ 
  Zimbabwe & MASHONALAND WEST & 80-84 & 90.50 & 116.39 & 69.92 & HT-Direct \\ 
  Zimbabwe & MASHONALAND WEST & 80-84 & 109.32 & 90.15 & 131.52 & RW2 \\ 
  Zimbabwe & MASHONALAND WEST & 85-89 & 98.02 & 115.83 & 82.70 & HT-Direct \\ 
  Zimbabwe & MASHONALAND WEST & 85-89 & 92.03 & 79.95 & 105.35 & RW2 \\ 
  Zimbabwe & MASHONALAND WEST & 90-94 & 90.27 & 109.62 & 74.05 & HT-Direct \\ 
  Zimbabwe & MASHONALAND WEST & 90-94 & 108.46 & 94.43 & 124.26 & RW2 \\ 
  Zimbabwe & MASHONALAND WEST & 95-99 & 90.24 & 114.00 & 71.03 & HT-Direct \\ 
  Zimbabwe & MASHONALAND WEST & 95-99 & 148.28 & 125.64 & 174.55 & RW2 \\ 
  Zimbabwe & MASHONALAND WEST & 00-04 & 93.71 & 113.10 & 77.35 & HT-Direct \\ 
  Zimbabwe & MASHONALAND WEST & 00-04 & 147.78 & 122.51 & 177.38 & RW2 \\ 
  Zimbabwe & MASHONALAND WEST & 05-09 & 124.64 & 154.78 & 99.68 & HT-Direct \\ 
  Zimbabwe & MASHONALAND WEST & 05-09 & 129.14 & 103.69 & 159.74 & RW2 \\ 
  Zimbabwe & MASHONALAND WEST & 10-14 & 110.83 & 37.94 & 289.14 & RW2 \\ 
  Zimbabwe & MASHONALAND WEST & 15-19 & 94.23 & 6.08 & 661.02 & RW2 \\ 
  Zimbabwe & MASVINGO & 80-84 & 95.18 & 119.74 & 75.23 & HT-Direct \\ 
  Zimbabwe & MASVINGO & 80-84 & 97.23 & 80.40 & 116.95 & RW2 \\ 
  Zimbabwe & MASVINGO & 85-89 & 67.72 & 82.28 & 55.57 & HT-Direct \\ 
  Zimbabwe & MASVINGO & 85-89 & 75.27 & 64.95 & 87.13 & RW2 \\ 
  Zimbabwe & MASVINGO & 90-94 & 69.50 & 85.02 & 56.64 & HT-Direct \\ 
  Zimbabwe & MASVINGO & 90-94 & 83.29 & 71.93 & 96.37 & RW2 \\ 
  Zimbabwe & MASVINGO & 95-99 & 80.97 & 107.94 & 60.29 & HT-Direct \\ 
  Zimbabwe & MASVINGO & 95-99 & 108.23 & 90.50 & 129.47 & RW2 \\ 
  Zimbabwe & MASVINGO & 00-04 & 87.49 & 116.94 & 64.91 & HT-Direct \\ 
  Zimbabwe & MASVINGO & 00-04 & 100.73 & 81.96 & 123.09 & RW2 \\ 
  Zimbabwe & MASVINGO & 05-09 & 67.26 & 80.95 & 55.74 & HT-Direct \\ 
  Zimbabwe & MASVINGO & 05-09 & 80.86 & 66.06 & 98.65 & RW2 \\ 
  Zimbabwe & MASVINGO & 10-14 & 63.58 & 21.33 & 180.15 & RW2 \\ 
  Zimbabwe & MASVINGO & 15-19 & 49.57 & 3.12 & 486.81 & RW2 \\ 
  Zimbabwe & MATABELELAND NORTH & 80-84 & 100.95 & 129.63 & 78.04 & HT-Direct \\ 
  Zimbabwe & MATABELELAND NORTH & 80-84 & 95.93 & 77.69 & 118.28 & RW2 \\ 
  Zimbabwe & MATABELELAND NORTH & 85-89 & 65.79 & 82.38 & 52.35 & HT-Direct \\ 
  Zimbabwe & MATABELELAND NORTH & 85-89 & 74.03 & 62.50 & 87.59 & RW2 \\ 
  Zimbabwe & MATABELELAND NORTH & 90-94 & 57.62 & 80.42 & 40.99 & HT-Direct \\ 
  Zimbabwe & MATABELELAND NORTH & 90-94 & 81.95 & 68.44 & 97.49 & RW2 \\ 
  Zimbabwe & MATABELELAND NORTH & 95-99 & 69.66 & 95.20 & 50.59 & HT-Direct \\ 
  Zimbabwe & MATABELELAND NORTH & 95-99 & 107.35 & 87.37 & 130.98 & RW2 \\ 
  Zimbabwe & MATABELELAND NORTH & 00-04 & 95.51 & 134.91 & 66.74 & HT-Direct \\ 
  Zimbabwe & MATABELELAND NORTH & 00-04 & 101.50 & 80.18 & 127.79 & RW2 \\ 
  Zimbabwe & MATABELELAND NORTH & 05-09 & 73.65 & 93.36 & 57.84 & HT-Direct \\ 
  Zimbabwe & MATABELELAND NORTH & 05-09 & 83.37 & 64.54 & 106.81 & RW2 \\ 
  Zimbabwe & MATABELELAND NORTH & 10-14 & 67.05 & 21.78 & 190.59 & RW2 \\ 
  Zimbabwe & MATABELELAND NORTH & 15-19 & 54.33 & 3.39 & 504.17 & RW2 \\ 
  Zimbabwe & MATABELELAND SOUTH & 80-84 & 65.10 & 81.82 & 51.60 & HT-Direct \\ 
  Zimbabwe & MATABELELAND SOUTH & 80-84 & 68.33 & 55.45 & 83.79 & RW2 \\ 
  Zimbabwe & MATABELELAND SOUTH & 85-89 & 49.28 & 67.95 & 35.55 & HT-Direct \\ 
  Zimbabwe & MATABELELAND SOUTH & 85-89 & 55.58 & 46.28 & 66.47 & RW2 \\ 
  Zimbabwe & MATABELELAND SOUTH & 90-94 & 60.62 & 79.08 & 46.26 & HT-Direct \\ 
  Zimbabwe & MATABELELAND SOUTH & 90-94 & 65.09 & 54.37 & 77.75 & RW2 \\ 
  Zimbabwe & MATABELELAND SOUTH & 95-99 & 70.33 & 96.97 & 50.61 & HT-Direct \\ 
  Zimbabwe & MATABELELAND SOUTH & 95-99 & 89.49 & 72.42 & 110.10 & RW2 \\ 
  Zimbabwe & MATABELELAND SOUTH & 00-04 & 39.95 & 57.61 & 27.54 & HT-Direct \\ 
  Zimbabwe & MATABELELAND SOUTH & 00-04 & 87.56 & 68.27 & 111.61 & RW2 \\ 
  Zimbabwe & MATABELELAND SOUTH & 05-09 & 66.41 & 87.27 & 50.27 & HT-Direct \\ 
  Zimbabwe & MATABELELAND SOUTH & 05-09 & 74.18 & 56.12 & 97.70 & RW2 \\ 
  Zimbabwe & MATABELELAND SOUTH & 10-14 & 61.83 & 20.22 & 179.94 & RW2 \\ 
  Zimbabwe & MATABELELAND SOUTH & 15-19 & 51.27 & 3.12 & 508.54 & RW2 \\ 
  Zimbabwe & MIDLANDS & 80-84 & 98.47 & 120.91 & 79.82 & HT-Direct \\ 
  Zimbabwe & MIDLANDS & 80-84 & 99.51 & 82.92 & 118.72 & RW2 \\ 
  Zimbabwe & MIDLANDS & 85-89 & 66.85 & 84.91 & 52.41 & HT-Direct \\ 
  Zimbabwe & MIDLANDS & 85-89 & 79.19 & 67.94 & 92.05 & RW2 \\ 
  Zimbabwe & MIDLANDS & 90-94 & 82.64 & 100.54 & 67.69 & HT-Direct \\ 
  Zimbabwe & MIDLANDS & 90-94 & 89.63 & 77.53 & 103.26 & RW2 \\ 
  Zimbabwe & MIDLANDS & 95-99 & 80.85 & 100.77 & 64.59 & HT-Direct \\ 
  Zimbabwe & MIDLANDS & 95-99 & 118.64 & 100.19 & 140.11 & RW2 \\ 
  Zimbabwe & MIDLANDS & 00-04 & 75.77 & 95.00 & 60.18 & HT-Direct \\ 
  Zimbabwe & MIDLANDS & 00-04 & 113.26 & 93.30 & 137.10 & RW2 \\ 
  Zimbabwe & MIDLANDS & 05-09 & 83.25 & 100.92 & 68.43 & HT-Direct \\ 
  Zimbabwe & MIDLANDS & 05-09 & 94.13 & 76.82 & 114.67 & RW2 \\ 
  Zimbabwe & MIDLANDS & 10-14 & 76.81 & 25.89 & 211.80 & RW2 \\ 
  Zimbabwe & MIDLANDS & 15-19 & 62.08 & 3.97 & 549.10 & RW2 \\ 
  \hline
\caption{Complete results.} 
\label{fulltable}
\end{longtable}

}

\end{document}


