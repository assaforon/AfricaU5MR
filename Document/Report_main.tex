\documentclass[12pt]{article}\usepackage[]{graphicx}\usepackage[]{color}
%% maxwidth is the original width if it is less than linewidth
%% otherwise use linewidth (to make sure the graphics do not exceed the margin)
\makeatletter
\def\maxwidth{ %
  \ifdim\Gin@nat@width>\linewidth
    \linewidth
  \else
    \Gin@nat@width
  \fi
}
\makeatother

\definecolor{fgcolor}{rgb}{0.345, 0.345, 0.345}
\newcommand{\hlnum}[1]{\textcolor[rgb]{0.686,0.059,0.569}{#1}}%
\newcommand{\hlstr}[1]{\textcolor[rgb]{0.192,0.494,0.8}{#1}}%
\newcommand{\hlcom}[1]{\textcolor[rgb]{0.678,0.584,0.686}{\textit{#1}}}%
\newcommand{\hlopt}[1]{\textcolor[rgb]{0,0,0}{#1}}%
\newcommand{\hlstd}[1]{\textcolor[rgb]{0.345,0.345,0.345}{#1}}%
\newcommand{\hlkwa}[1]{\textcolor[rgb]{0.161,0.373,0.58}{\textbf{#1}}}%
\newcommand{\hlkwb}[1]{\textcolor[rgb]{0.69,0.353,0.396}{#1}}%
\newcommand{\hlkwc}[1]{\textcolor[rgb]{0.333,0.667,0.333}{#1}}%
\newcommand{\hlkwd}[1]{\textcolor[rgb]{0.737,0.353,0.396}{\textbf{#1}}}%
\let\hlipl\hlkwb

\usepackage{framed}
\makeatletter
\newenvironment{kframe}{%
 \def\at@end@of@kframe{}%
 \ifinner\ifhmode%
  \def\at@end@of@kframe{\end{minipage}}%
  \begin{minipage}{\columnwidth}%
 \fi\fi%
 \def\FrameCommand##1{\hskip\@totalleftmargin \hskip-\fboxsep
 \colorbox{shadecolor}{##1}\hskip-\fboxsep
     % There is no \\@totalrightmargin, so:
     \hskip-\linewidth \hskip-\@totalleftmargin \hskip\columnwidth}%
 \MakeFramed {\advance\hsize-\width
   \@totalleftmargin\z@ \linewidth\hsize
   \@setminipage}}%
 {\par\unskip\endMakeFramed%
 \at@end@of@kframe}
\makeatother

\definecolor{shadecolor}{rgb}{.97, .97, .97}
\definecolor{messagecolor}{rgb}{0, 0, 0}
\definecolor{warningcolor}{rgb}{1, 0, 1}
\definecolor{errorcolor}{rgb}{1, 0, 0}
\newenvironment{knitrout}{}{} % an empty environment to be redefined in TeX

\usepackage{alltt}  
% \usepackage[sc]{mathpazo}
% \usepackage[T1]{fontenc}
\usepackage{pifont} 
\usepackage{url}
\usepackage{breakurl}
\usepackage[colorlinks = true,
            linkcolor = blue,
            urlcolor  = red!80!blue,
            citecolor = blue!80!black,
            anchorcolor = blue]{hyperref}
\usepackage{soul}
\usepackage{paralist}
\usepackage{bm}
\usepackage[round]{natbib}
\usepackage{graphicx}
\usepackage{amsmath, wrapfig,amssymb,multirow}
\usepackage[margin=1in]{geometry}
%%%Gives a continuous ordering of figures.  Comment out to get
%within section numbering of figures
\usepackage{chngcntr}
\counterwithout{figure}{section}
\usepackage[parfill]{parskip}
\usepackage{epsfig, subfigure}
\usepackage{amsfonts}
\usepackage{authblk}
\renewcommand\Affilfont{\footnotesize}
\usepackage{booktabs}
\usepackage{bigstrut}
\usepackage{tabularx}
\usepackage{threeparttable} 
\usepackage[format=hang,labelfont=bf]{caption}
\usepackage{float}
\floatstyle{boxed}
\usepackage{footnote}
\usepackage{booktabs}
\usepackage{longtable}
\usepackage{array}
\usepackage{multirow}
\usepackage[table]{xcolor}
\usepackage{wrapfig}
\usepackage{float}
\usepackage{colortbl}
\usepackage{pdflscape}
\usepackage{tabu}
\usepackage{threeparttable}
\makesavenoteenv{tabular}
\makesavenoteenv{table}
%%----------------------------------------- customized fonts
\newcommand\code{\bgroup\@makeother\_\@makeother\~\@makeother\$\@codex}
\def\@codex#1{{\normalfont\ttfamily\hyphenchar\font=-1 #1}\egroup}
\let\code=\texttt
\let\proglang=\textsf
\newcommand{\pkg}[1]{{\fontseries{b}\selectfont #1}}
\newcommand{\tm}[1]{\textcolor{blue}{\textit{Tyler: #1}}}
\newcommand{\sjc}[1]{\textcolor{red}{\textit{Sam: #1}}}
\definecolor{light-gray}{gray}{0.75}
\definecolor{orange}{RGB}{255,127,0}
\newcommand{\zl}[1]{\textcolor{orange}{\textit{Richard: #1}}}
\newcommand{\todo}[1]{\textbf{To-do list:} #1}
\newcommand{\blue}[1]{\textcolor{blue}{#1}}
\newcommand{\orange}[1]{\textcolor{orange}{#1}}
\newcommand{\ok}{\nonumber}
\usepackage{xspace} 
\newcommand{\vapkg}{\pkg{openVA}\xspace}
\newcommand{\tmark}{\text{\ding{51}}}
\newcommand{\cmark}{\text{\ding{55}}}
%%----------------------------------------- Increase the separation
\let\tempone\itemize
\let\temptwo\enditemize
\renewenvironment{itemize}{\tempone\addtolength{\itemsep}{0.5\baselineskip}}{\temptwo}


 
\usepackage{tabularx}
\IfFileExists{upquote.sty}{\usepackage{upquote}}{}
\begin{document}

%%----------------------------------------- 


% 
% Need to replace all 
% Cote\_dIvoire
% by
% C\^{o}te d'Ivoire
% 

\title{Supplement Material: Full results}
\author{authors}

\sloppy
\maketitle
\tableofcontents
\addtocontents{toc}{\protect\setcounter{tocdepth}{2}}
\clearpage
\section{Partitioning of variability}
Table~\ref{tab:var} presents the partitioning of variability among the random effect terms in the space-time model.
% latex table generated in R 3.4.3 by xtable 1.8-2 package
% Mon Mar 19 16:02:34 2018
\begin{table}[htb]
\centering\small
\begin{tabular}{lrrrrr}
  \hline
country & RW2 ($\sigma^2_{\gamma_t}$)& ICAR($\sigma^2_{\phi_i}$) & IID space ($\sigma^2_{\theta_i}$) & IID time ($\sigma^2_{\alpha_t}$) & IID space time ($\sigma^2_{\delta_{it}}$) \\ 
  \hline
  Burkina Faso & 56.9\% & 30.9\% & 3\% & 0.6\% & 8.7\% \\ 
  Benin & 65.5\% & 27.4\% & 2.7\% & 0.5\% & 3.8\% \\ 
  Angola & 6\% & 56.1\% & 1.4\% & 0.3\% & 36.3\% \\ 
  Burundi & 47.5\% & 35.6\% & 1.8\% & 0.4\% & 14.7\% \\ 
  Cameroon & 29.2\% & 65.3\% & 2.4\% & 0.4\% & 2.7\% \\ 
  Chad & 36.9\% & 41.4\% & 3.3\% & 0.6\% & 17.7\% \\ 
  Comoros & 64.9\% & 16\% & 1.6\% & 0.5\% & 17\% \\ 
  Congo & 72\% & 13.5\% & 2.4\% & 0.6\% & 11.4\% \\ 
  C\^{o}te d'Ivoire & 25.7\% & 54.8\% & 2.5\% & 0.4\% & 16.7\% \\ 
  DRC & 53.9\% & 28.1\% & 2.1\% & 0.4\% & 15.4\% \\ 
  Egypt & 80.2\% & 14.9\% & 0.9\% & 0.2\% & 3.7\% \\ 
  Ethiopia & 70.8\% & 24.6\% & 1\% & 0.2\% & 3.4\% \\ 
  Gabon & 51.4\% & 30.4\% & 3.7\% & 0.8\% & 13.6\% \\ 
  Gambia & 66.2\% & 18.6\% & 0.9\% & 0.2\% & 14\% \\ 
  Ghana & 56.5\% & 34.5\% & 2.6\% & 0.5\% & 5.9\% \\ 
  Guinea & 62.7\% & 31.5\% & 1.1\% & 0.2\% & 4.6\% \\ 
  Kenya & 31.5\% & 48.1\% & 1.7\% & 0.3\% & 18.4\% \\ 
  Lesotho & 28\% & 28.2\% & 4.6\% & 1.1\% & 38.1\% \\ 
  Liberia & 84.2\% & 6.9\% & 1.3\% & 0.3\% & 7.3\% \\ 
  Madagascar & 72.7\% & 16.2\% & 1.4\% & 0.3\% & 9.4\% \\ 
  Malawi & 87\% & 11.1\% & 1\% & 0.2\% & 0.6\% \\ 
  Mali & 42.8\% & 50.4\% & 1.2\% & 0.2\% & 5.4\% \\ 
  Morocco & 83\% & 8.8\% & 1\% & 0.2\% & 7\% \\ 
  Mozambique & 65.2\% & 21.8\% & 1\% & 0.2\% & 11.8\% \\ 
  Namibia & 44.5\% & 32.5\% & 2.2\% & 0.6\% & 20.2\% \\ 
  Niger & 57.1\% & 30.9\% & 1.4\% & 0.3\% & 10.3\% \\ 
  Nigeria & 26.7\% & 65.4\% & 1.9\% & 0.3\% & 5.7\% \\ 
  Rwanda & 83.7\% & 12.5\% & 1\% & 0.2\% & 2.5\% \\ 
  Senegal & 72.5\% & 23\% & 1.2\% & 0.2\% & 3\% \\ 
  Sierra Leone & 59.4\% & 24.7\% & 2.3\% & 0.5\% & 13.1\% \\ 
  Tanzania & 75.3\% & 18.1\% & 1.2\% & 0.2\% & 5.2\% \\ 
  Togo & 53.4\% & 39.9\% & 2.5\% & 0.5\% & 3.6\% \\ 
  Uganda & 87.1\% & 8.7\% & 1.3\% & 0.3\% & 2.7\% \\ 
  Zambia & 75\% & 19.2\% & 1.6\% & 0.3\% & 3.8\% \\ 
  Zimbabwe & 45.2\% & 44.3\% & 2.3\% & 0.5\% & 7.7\% \\ 
   \hline
\end{tabular}
\caption{Variance component proportions for each country.}
\label{tab:var}
\end{table}


\clearpage
\section{Summary of MDG goals}
Table~\ref{tab:mdg} summarizes the MDG4 status by country.
% latex table generated in R 3.4.3 by xtable 1.8-2 package
% Mon Mar 19 13:58:40 2018
\begin{table}[ht]
\centering
\begin{tabular}{lllrl}
  \hline
Country & MDG4 achieved & Percent achieved & Median deduction & [Min, Max] \\ 
   \hline
Angola & 1/18 & 5.56\% & 0.098 & [-0.63,0.88] \\ 
  Benin & 0/6 & 0\% & 0.465 & [ 0.38,0.57] \\ 
  Burkina Faso & 0/4 & 0\% & 0.599 & [ 0.41,0.65] \\ 
  Burundi & 1/5 & 20\% & 0.637 & [ 0.44,0.69] \\ 
  Cameroon & 0/5 & 0\% & 0.367 & [ 0.22,0.43] \\ 
  Chad & 0/8 & 0\% & 0.353 & [ 0.08,0.53] \\ 
  Comoros & 0/3 & 0\% & 0.335 & [ 0.10,0.41] \\ 
  Congo & 1/4 & 25\% & 0.621 & [ 0.39,0.69] \\ 
  C\^{o}te d'Ivoire & 0/11 & 0\% & 0.323 & [-0.01,0.59] \\ 
  DRC & 1/11 & 9.09\% & 0.501 & [ 0.27,0.78] \\ 
  Egypt & 2/4 & 50\% & 0.608 & [ 0.51,0.74] \\ 
  Ethiopia & 10/11 & 90.91\% & 0.711 & [ 0.58,0.80] \\ 
  Gabon & 0/5 & 0\% & 0.364 & [ 0.14,0.52] \\ 
  Gambia & 4/6 & 66.67\% & 0.673 & [ 0.36,0.82] \\ 
  Ghana & 0/8 & 0\% & 0.568 & [ 0.35,0.61] \\ 
  Guinea & 3/5 & 60\% & 0.699 & [ 0.40,0.73] \\ 
  Kenya & 3/8 & 37.5\% & 0.509 & [ 0.04,0.75] \\ 
  Lesotho & 0/10 & 0\% & 0.195 & [-0.44,0.46] \\ 
  Liberia & 4/5 & 80\% & 0.748 & [ 0.45,0.78] \\ 
  Madagascar & 6/6 & 100\% & 0.804 & [ 0.69,0.89] \\ 
  Malawi & 3/3 & 100\% & 0.717 & [ 0.71,0.73] \\ 
  Mali & 1/4 & 25\% & 0.617 & [ 0.46,0.74] \\ 
  Morocco & 4/7 & 57.14\% & 0.714 & [ 0.56,0.81] \\ 
  Mozambique & 6/11 & 54.55\% & 0.679 & [ 0.25,0.80] \\ 
  Namibia & 1/13 & 7.69\% & 0.443 & [ 0.03,0.68] \\ 
  Niger & 3/6 & 50\% & 0.706 & [ 0.47,0.84] \\ 
  Nigeria & 0/6 & 0\% & 0.466 & [ 0.22,0.58] \\ 
  Rwanda & 5/5 & 100\% & 0.789 & [ 0.71,0.80] \\ 
  Senegal & 6/11 & 54.55\% & 0.704 & [ 0.59,0.76] \\ 
  Sierra Leone & 0/4 & 0\% & 0.554 & [ 0.33,0.66] \\ 
  Tanzania & 16/20 & 80\% & 0.755 & [ 0.55,0.85] \\ 
  Togo & 0/6 & 0\% & 0.449 & [ 0.35,0.55] \\ 
  Uganda & 3/4 & 75\% & 0.711 & [ 0.67,0.74] \\ 
  Zambia & 4/9 & 44.44\% & 0.658 & [ 0.60,0.76] \\ 
  Zimbabwe & 0/10 & 0\% & 0.032 & [-0.14,0.34] \\ 
   \hline
\end{tabular}
\caption{MDG4 goal achievement status for subnational regions.}
\label{tab:mdg}
\end{table}


Figure \ref{fig:c1-1} to \ref{fig:c1-4} show the projected U5MR and the projected deduction of U5MR at year of 2015 and the time period of 2015-2019 compared to that of 1990 respectively. In addition to the subnational model results, we include the comparison to the RW2 only model fitted to the combined national data, i.e., without subnational spatial smoothing, after benchmarking with UN estimates.  We also compare our results with UN (B3) estimates described in You et al. (2015) and IHME estimates based on GBD 2015 Child Mortality Collaborators (2016) for the comparisons with 2015 estimates.  

\begin{figure}[htb]
\includegraphics[width = .49\textwidth]{../Main/Figures/Africa_reduction_2015.jpeg}
\includegraphics[width = .49\textwidth]{../Main/Figures/Africa_national_reduction_2015.jpeg}
\includegraphics[width = .49\textwidth]{../Main/Figures/Africa_UN_reduction_2015.jpeg}
\includegraphics[width = .49\textwidth]{../Main/Figures/Africa_IHME_reduction_2015.jpeg}
\caption{Deduction of U5MR from 1990 to 2015 estimated by different methods. Upper left: Subnational model. Upper right: National model. Lower left: UN B-3 estimates. Lower right: IHME GBD estimates}
\label{fig:c1-1}
\end{figure}

\begin{figure}[htb]
\includegraphics[width = .49\textwidth]{../Main/Figures/Africa_u5mr_2015.jpeg}
\includegraphics[width = .49\textwidth]{../Main/Figures/Africa_national_u5mr_2015.jpeg}
\includegraphics[width = .49\textwidth]{../Main/Figures/Africa_UN_u5mr_2015.jpeg}
\includegraphics[width = .49\textwidth]{../Main/Figures/Africa_IHME_u5mr_2015.jpeg}
\caption{Projection of U5MR for 2015 by different methods. Upper left: Subnational model. Upper right: National model. Lower left: UN B-3 estimates. Lower right: IHME GBD estimates}
\label{fig:c1-2}
\end{figure}


\begin{figure}[htb]
\includegraphics[width = .49\textwidth]{../Main/Figures/Africa_reduction_15-19.jpeg}
\includegraphics[width = .49\textwidth]{../Main/Figures/Africa_national_reduction_15-19.jpeg}
% \includegraphics[width = .49\textwidth]{../Main/Figures/Africa_UN_reduction_15-19.jpeg}
% \includegraphics[width = .49\textwidth]{../Main/Figures/Africa_IHME_reduction_15-19.jpeg}
\caption{Deduction of U5MR from 1990 to 2015-2019 period estimated by different methods. Left: Subnational model. Right: National model.}
\label{fig:c1-3}
\end{figure}

\begin{figure}[htb]
\includegraphics[width = .49\textwidth]{../Main/Figures/Africa_u5mr_15-19.jpeg}
\includegraphics[width = .49\textwidth]{../Main/Figures/Africa_national_u5mr_15-19.jpeg}
% \includegraphics[width = .49\textwidth]{../Main/Figures/Africa_UN_u5mr_15-19.jpeg}
% \includegraphics[width = .49\textwidth]{../Main/Figures/Africa_IHME_u5mr_15-19.jpeg}
\caption{Projection of U5MR for 2015-2019 period by different methods. Left: Subnational model. Right: National model.}
\label{fig:c1-4}
\end{figure}

\clearpage
\section{Cross validation summary}
Figure~\ref{fig:var} shows the distribution of the cross validation bias combined for all the regions in the study. Table~\ref{tab:cv} summarizes the cross validation results for each country.

\begin{figure}[htb]
\centering
\includegraphics[width = .8\textwidth]{../Main/Figures/CVbias.pdf}
\caption{Histogram and QQ-plot of the rescaled difference between the smoothed estimates and the direct estimates. The differences between the two estimates are scaled by the square root of the total variance of the two estimates.}
\label{fig:var}
\end{figure}


% latex table generated in R 3.4.3 by xtable 1.8-2 package
% Tue Mar 20 11:11:39 2018
\begin{table}[ht]
\centering
\begin{tabular}{lrrr}
  \hline
  % & \multicolumn{3}{c}{by time periods}   & \multicolumn{3}{c}{by time periods and regions}  \\
% \cmidrule{2-4} 
% \cmidrule{5-7}
Country & coverage & $Average bias$ & $sd(\mbox{bias})$ \\ 
  \midrule
Angola & 0.92 & 0.07 & 1.13 \\ 
  Benin & 0.97 & 0.01 & 0.90 \\ 
  Burkina Faso & 0.96 & 0.00 & 1.11 \\ 
  Burundi & 0.90 & 0.03 & 1.22 \\ 
  Cameroon & 0.97 & 0.02 & 0.89 \\ 
  Chad & 0.93 & -0.00 & 1.05 \\ 
  Comoros & 1.00 & -0.01 & 0.88 \\ 
  Congo & 0.86 & 0.00 & 1.38 \\ 
  C\^{o}te d'Ivoire & 0.92 & 0.02 & 1.07 \\ 
  DRC & 0.99 & 0.01 & 1.04 \\ 
  Egypt & 0.88 & -0.00 & 1.05 \\ 
  Ethiopia & 0.92 & 0.02 & 1.10 \\ 
  Gabon & 0.94 & -0.01 & 1.04 \\ 
  Gambia & 0.98 & -0.00 & 0.95 \\ 
  Ghana & 0.96 & 0.02 & 0.99 \\ 
  Guinea & 0.91 & 0.01 & 0.99 \\ 
  Kenya & 0.96 & 0.00 & 1.02 \\ 
  Lesotho & 1.00 & 0.02 & 0.83 \\ 
  Liberia & 1.00 & 0.00 & 0.85 \\ 
  Madagascar & 0.89 & -0.01 & 1.18 \\ 
  Malawi & 1.00 & 0.01 & 1.02 \\ 
  Mali & 0.92 & 0.03 & 1.07 \\ 
  Morocco & 0.97 & 0.00 & 1.03 \\ 
  Mozambique & 0.91 & 0.02 & 1.11 \\ 
  Namibia & 0.92 & 0.05 & 1.09 \\ 
  Niger & 0.89 & -0.01 & 1.39 \\ 
  Nigeria & 0.88 & 0.01 & 1.20 \\ 
  Rwanda & 0.69 & -0.01 & 1.58 \\ 
  Senegal & 0.88 & 0.01 & 1.22 \\ 
  Sierra Leone & 0.96 & 0.01 & 1.02 \\ 
  Tanzania & 0.92 & 0.02 & 1.11 \\ 
  Togo & 0.95 & 0.01 & 1.11 \\ 
  Uganda & 0.96 & 0.00 & 1.09 \\ 
  Zambia & 0.89 & 0.02 & 1.16 \\ 
  Zimbabwe & 0.95 & 0.02 & 0.94 \\ 
  \hline
  Average & 0.93 & 0.01 & 1.08 \\ 
   \bottomrule
\end{tabular}
\caption{Coverage of the $95\%$ posterior credible interval for the logit of the direct estimates, mean and standard deviation of the rescaled bias under two cross validation schemes. The biases are scaled by the estimated standard deviation of the difference between the smoothed estimates and the direct estimates.}
\label{tab:cv}
\end{table}

 

% %%%%%%%%%%%%%%%%%%%%%%%%%%%%%%%%%
% %%%%%%%%%%%%%%%%%%%%%%%%%%%%%%%%%
% \subsubsection{Within country relative risk by time}

% Figures \ref{fig:c3-1} and \ref{fig:c3-2}.

% \begin{figure}[htb]
% \includegraphics[width = \textwidth]{../Main/Figures/rr_Africa.jpeg}
% \caption{Africa: relative risk within each country by time period, i.e., $p_{ict} / min_i(p_{ict})$ for region $i$ in country $c$ and time $t$.}
% \label{fig:c3-1}
% \end{figure}

% \begin{figure}[htb]
% \includegraphics[width = \textwidth]{../Main/Figures/rr_Asia.jpeg}
% \caption{Asia: relative risk within each country by time period, i.e., $p_{ict} / min_i(p_{ict})$ for region $i$ in country $c$ and time $t$.}
% \label{fig:c3-2}
% \end{figure}

\clearpage
\section{Complete results}
% %%%%%%%%%%%%%%%%%%%%%%%%%%%%%%%%%%%%%%%%%%%%%%%%%%%%%%%%%%%%%%%%%%%%%%%%%%%%%%%%%%%%%%%%%%%%%%%%%%
% \subsection{Bangladesh}
% <<Bangladesh, echo=FALSE, results='hide'>>=
% countryname <- "Bangladesh"
% @
% <<Sexpr{paste0(countryname, "-run")}, messages=FALSE, child='single-country-combined.rnw'>>=
% @

% %%%%%%%%%%%%%%%%%%%%%%%%%%%%%%%%%%%%%%%%%%%%%%%%%%%%%%%%%%%%%%%%%%%%%%%%%%%%%%%%%%%%%%%%%%%%%%%%%%
% \clearpage
% \subsection{Cambodia}
% <<Cambodia, echo=FALSE, results='hide'>>=
% countryname <- "Cambodia"
% country_count <- country_count + 1
% @
% <<Sexpr{paste0(countryname, "-run")}, messages=FALSE, child='single-country-combined.rnw'>>=
% @


% %%%%%%%%%%%%%%%%%%%%%%%%%%%%%%%%%%%%%%%%%%%%%%%%%%%%%%%%%%%%%%%%%%%%%%%%%%%%%%%%%%%%%%%%%%%%%%%%%%
% \clearpage
% \subsection{Indonesia}
% <<Indonesia, echo=FALSE, results='hide'>>=
% countryname <- "Indonesia"
% country_count <- country_count + 1
% @
% <<Sexpr{paste0(countryname, "-run")}, messages=FALSE, child='single-country-combined.rnw'>>=
% @

% %%%%%%%%%%%%%%%%%%%%%%%%%%%%%%%%%%%%%%%%%%%%%%%%%%%%%%%%%%%%%%%%%%%%%%%%%%%%%%%%%%%%%%%%%%%%%%%%%%
% \clearpage
% \subsection{Philippines}
% <<Philippines, echo=FALSE, results='hide'>>=
% countryname <- "Philippines"
% country_count <- country_count + 1
% @
% <<Sexpr{paste0(countryname, "-run")}, messages=FALSE, child='single-country-combined.rnw'>>=
% @
%%%%%%%%%%%%%%%%%%%%%%%%%%%%%%%%%%%%%%%%%%%%%%%%%%%%%%%%%%%%%%%%%%%%%%%%%%%%%%%%%%%%%%%%%%%%%%%%%%
\subsection{Angola}


% \subsubsection{Summary of DHS surveys}

%%%%%%%%%%%%%%%%%%%%%%%%%%% Summary 


DHS surveys were conducted in Angola in 2015.
% years.out[1:(length(years.out)-1)], and years.out[length(years.out)]. 

We fit both the RW2 only model to the combined national data, and compare the time trend at national level with the estimates produced by the UN and IHME in Figure~\ref{fig:unnamed-chunk-2}. We then adjusted the combined national data to the UN estimates of U5MR, and refit the models on the benchmarked data. 

%%%%%%%%%%%%%%%%%%%%%%%%%% Plot5 
\begin{knitrout}
\definecolor{shadecolor}{rgb}{0.969, 0.969, 0.969}\color{fgcolor}\begin{figure}[bht]

{\centering \includegraphics[width=.9\textwidth]{../Main/Figures/Yearly_national_Angola} 

}

\caption[Temporal national trends along with UN (B3) estimates described in You et al]{Temporal national trends along with UN (B3) estimates described in You et al. (2015) and IHME estimates based on GBD 2015 Child Mortality Collaborators (2016). RW2 represents the smoothed national estimates using the original data before benchmarking with UN estimates. RW2-adj represents the smoothed national estimates using the benchmarked data.}\label{fig:unnamed-chunk-2}
\end{figure}


\end{knitrout}
 

We fit the RW2 model to the benchmarked data in each area. 
% The proportions of the explained variation is summarized in Table~\ref{tab:paste0(countryname, "-var")}. 
We compare the results in Figure~\ref{fig:unnamed-chunk-3} to \ref{fig:unnamed-chunk-7}.
Figure~\ref{fig:unnamed-chunk-3} compares the smoothed estimates against the direct estimates. Figure~\ref{fig:unnamed-chunk-4} and Figure~\ref{fig:unnamed-chunk-5} show the posterior median estimates of U5MR in each region over time and the reductions from 1990 period respectively.
Figure~\ref{fig:unnamed-chunk-6} shows the smoothed estimates by region over time and Figure~\ref{fig:unnamed-chunk-7} compares the smoothed estimates with direct estimates from each survey for each region over time.


% %%%%%%%%%%%%%%%%%%%%%%%%%%% Table1 
% <<echo=FALSE, results='asis'>>=
% load("rda/variance_tables.rda")
% countryname2 <- gsub(" ", "", countryname)
% variance <- tables.all[[countryname]]

% table_count <- table_count + 1

% names <- c("RW2 ($\\sigma^2_{\\gamma_{t}}$)", "ICAR ($\\sigma^2_{\\phi_{i}}$)", "IID space ($\\sigma^2_{\\theta_{i}}$)", "IID time ($\\sigma^2_{\\alpha_{t}}$)", "IID space time ($\\sigma^2_{\\delta_{it}}$)")

% variance$Proportion <- round(variance$Proportion*100, digits = 2)
% row.names(variance) <- names
% tab <- xtable(variance, digits = c(1, 3, 2),align = "l|ll",
%        label = paste0("tab:", countryname, "-var"),
%        caption = paste(country, ": summary of the variance components in the RW2 model", sep = ''))
% print(tab, comment = FALSE,sanitize.text.function = function(x) {x})
% @

%%%%%%%%%%%%%%%%%%%%%%%%%%% Plot1 
\begin{knitrout}
\definecolor{shadecolor}{rgb}{0.969, 0.969, 0.969}\color{fgcolor}\begin{figure}[bht]

{\centering \includegraphics[width=.9\textwidth]{../Main/Figures/SmoothvDirectAngola_meta} 

}

\caption[Smooth versus direct Admin 1 estimates]{Smooth versus direct Admin 1 estimates. Left: Combined (meta-analysis) survey estimate against combined direct estimates. Right: Combined (meta-analysis) survey estimate against direct estimates from each survey.}\label{fig:unnamed-chunk-3}
\end{figure}


\end{knitrout}

%%%%%%%%%%%%%%%%%%%%%%%%%%% Plot2 
\begin{knitrout}
\definecolor{shadecolor}{rgb}{0.969, 0.969, 0.969}\color{fgcolor}\begin{figure}[bht]

{\centering \includegraphics[width=.9\textwidth]{../Main/Figures/SmoothMedianAngola} 

}

\caption[Maps of posterior medians for Angola  over time]{Maps of posterior medians for Angola  over time.}\label{fig:unnamed-chunk-4}
\end{figure}


\end{knitrout}
%%%%%%%%%%%%%%%%%%%%%%%%%%% Plot2a
\begin{knitrout}
\definecolor{shadecolor}{rgb}{0.969, 0.969, 0.969}\color{fgcolor}\begin{figure}[bht]

{\centering \includegraphics[width=.9\textwidth]{../Main/Figures/ReductionMedianAngola} 

}

\caption[Maps of reduction of posterior median U5MR in each five-year period compared to 1990 in Angola over time]{Maps of reduction of posterior median U5MR in each five-year period compared to 1990 in Angola over time.}\label{fig:unnamed-chunk-5}
\end{figure}


\end{knitrout}
%%%%%%%%%%%%%%%%%%%%%%%%%%% Plot3 
\begin{knitrout}
\definecolor{shadecolor}{rgb}{0.969, 0.969, 0.969}\color{fgcolor}\begin{figure}[bht]

{\centering \includegraphics[width=.95\textwidth]{../Main/Figures/Yearly_v_Periods_Angola} 

}

\caption[Smoothed regional estimates over time]{Smoothed regional estimates over time. The line indicates yearly posterior median estimates and error bars indicate 95 \% posterior credible interval at each time period.}\label{fig:unnamed-chunk-6}
\end{figure}


\end{knitrout}

%%%%%%%%%%%%%%%%%%%%%%%%%%% Plot4 
\begin{knitrout}
\definecolor{shadecolor}{rgb}{0.969, 0.969, 0.969}\color{fgcolor}\begin{figure}[bht]

{\centering \includegraphics[width=.9\textwidth]{../Main/Figures/LineSubMedianAngola} 

}

\caption[Smoothed regional estimates over time compared to the direct estimates from each surveys]{Smoothed regional estimates over time compared to the direct estimates from each surveys. Direct estimates are not benchmarked with UN estimates. The line indicates posterior median and error bars indicate 95\% posterior credible interval.}\label{fig:unnamed-chunk-7}
\end{figure}


\end{knitrout}
% \subsubsection{National model results}
We further assess the RW2 model by holding out some observations, and compare the projections to the direct estimates in these holdout observations. Figure~\ref{fig:unnamed-chunk-8} compares the predicted estimates for the out-of-sample observations  with the direct estimates by holding out observations from each area in each time period.  Figure~\ref{fig:unnamed-chunk-9} compares the histogram of the bias rescaled by the total variance in the cross validation studies. Figure~\ref{fig:unnamed-chunk-10} compares the rescaled bias by region and time periods.



% %%%%%%%%%%%%%%%%%%%%%%%%%%% Plot6
% << echo=FALSE, out.width = ".9\\textwidth", fig.width = 12, fig.height = 6, fig.cap = "Out-of-sample predictions along with direct estimates in the cross validation study where all data from each time period is held out and predicted using the rest of the data.">>=
% fig_count <- fig_count + 1
% knitr::include_graphics(paste0("../Main/Figures/CV_byYear_withError_", countryname2, ".pdf")) 
% @
 
%%%%%%%%%%%%%%%%%%%%%%%%%%% Plot7
\begin{knitrout}
\definecolor{shadecolor}{rgb}{0.969, 0.969, 0.969}\color{fgcolor}\begin{figure}[bht]

{\centering \includegraphics[width=.9\textwidth]{../Main/Figures/CV_byYearRegion_withError_Angola} 

}

\caption[Out-of-sample predictions along with direct estimates in the cross validation study where data from one region in each time period is held out and predicted using the rest of the data]{Out-of-sample predictions along with direct estimates in the cross validation study where data from one region in each time period is held out and predicted using the rest of the data.}\label{fig:unnamed-chunk-8}
\end{figure}


\end{knitrout}

%%%%%%%%%%%%%%%%%%%%%%%%%%% Plot8
\begin{knitrout}
\definecolor{shadecolor}{rgb}{0.969, 0.969, 0.969}\color{fgcolor}\begin{figure}[bht]

{\centering \includegraphics[width=.9\textwidth]{../Main/Figures/CVbiasAngola} 

}

\caption[Histogram and QQ-plot of the rescaled difference between the smoothed estimates and the direct estimates in the cross validation study]{Histogram and QQ-plot of the rescaled difference between the smoothed estimates and the direct estimates in the cross validation study. The differences between the two estimates are rescaled by the square root of the total variance of the two estimates.}\label{fig:unnamed-chunk-9}
\end{figure}


\end{knitrout}

%%%%%%%%%%%%%%%%%%%%%%%%%%% Plot9
\begin{knitrout}
\definecolor{shadecolor}{rgb}{0.969, 0.969, 0.969}\color{fgcolor}\begin{figure}[bht]

{\centering \includegraphics[width=.7\textwidth]{../Main/Figures/CVbiasbyRegionAngola} 

}

\caption[Line plot of the difference between smoothed estimates and the direct estimates in the cross validation study]{Line plot of the difference between smoothed estimates and the direct estimates in the cross validation study. The differences between the two estimates are rescaled by the square root of the total variance of the two estimates.}\label{fig:unnamed-chunk-10}
\end{figure}


\end{knitrout}


%%%%%%%%%%%%%%%%%%%%%%%%%%%%%%%%%%%%%%%%%%%%%%%%%%%%%%%%%%%%%%%%%%%%%%%%%%%%%%%%%%%%%%%%%%%%%%%%%%
\clearpage
\subsection{Benin}


% \subsubsection{Summary of DHS surveys}

%%%%%%%%%%%%%%%%%%%%%%%%%%% Summary 


DHS surveys were conducted in Benin in 1996, 2001, and 2006.
% years.out[1:(length(years.out)-1)], and years.out[length(years.out)]. 

We fit both the RW2 only model to the combined national data, and compare the time trend at national level with the estimates produced by the UN and IHME in Figure~\ref{fig:unnamed-chunk-12}. We then adjusted the combined national data to the UN estimates of U5MR, and refit the models on the benchmarked data. 

%%%%%%%%%%%%%%%%%%%%%%%%%% Plot5 
\begin{knitrout}
\definecolor{shadecolor}{rgb}{0.969, 0.969, 0.969}\color{fgcolor}\begin{figure}[bht]

{\centering \includegraphics[width=.9\textwidth]{../Main/Figures/Yearly_national_Benin} 

}

\caption[Temporal national trends along with UN (B3) estimates described in You et al]{Temporal national trends along with UN (B3) estimates described in You et al. (2015) and IHME estimates based on GBD 2015 Child Mortality Collaborators (2016). RW2 represents the smoothed national estimates using the original data before benchmarking with UN estimates. RW2-adj represents the smoothed national estimates using the benchmarked data.}\label{fig:unnamed-chunk-12}
\end{figure}


\end{knitrout}
 

We fit the RW2 model to the benchmarked data in each area. 
% The proportions of the explained variation is summarized in Table~\ref{tab:paste0(countryname, "-var")}. 
We compare the results in Figure~\ref{fig:unnamed-chunk-13} to \ref{fig:unnamed-chunk-17}.
Figure~\ref{fig:unnamed-chunk-13} compares the smoothed estimates against the direct estimates. Figure~\ref{fig:unnamed-chunk-14} and Figure~\ref{fig:unnamed-chunk-15} show the posterior median estimates of U5MR in each region over time and the reductions from 1990 period respectively.
Figure~\ref{fig:unnamed-chunk-16} shows the smoothed estimates by region over time and Figure~\ref{fig:unnamed-chunk-17} compares the smoothed estimates with direct estimates from each survey for each region over time.


% %%%%%%%%%%%%%%%%%%%%%%%%%%% Table1 
% <<echo=FALSE, results='asis'>>=
% load("rda/variance_tables.rda")
% countryname2 <- gsub(" ", "", countryname)
% variance <- tables.all[[countryname]]

% table_count <- table_count + 1

% names <- c("RW2 ($\\sigma^2_{\\gamma_{t}}$)", "ICAR ($\\sigma^2_{\\phi_{i}}$)", "IID space ($\\sigma^2_{\\theta_{i}}$)", "IID time ($\\sigma^2_{\\alpha_{t}}$)", "IID space time ($\\sigma^2_{\\delta_{it}}$)")

% variance$Proportion <- round(variance$Proportion*100, digits = 2)
% row.names(variance) <- names
% tab <- xtable(variance, digits = c(1, 3, 2),align = "l|ll",
%        label = paste0("tab:", countryname, "-var"),
%        caption = paste(country, ": summary of the variance components in the RW2 model", sep = ''))
% print(tab, comment = FALSE,sanitize.text.function = function(x) {x})
% @

%%%%%%%%%%%%%%%%%%%%%%%%%%% Plot1 
\begin{knitrout}
\definecolor{shadecolor}{rgb}{0.969, 0.969, 0.969}\color{fgcolor}\begin{figure}[bht]

{\centering \includegraphics[width=.9\textwidth]{../Main/Figures/SmoothvDirectBenin_meta} 

}

\caption[Smooth versus direct Admin 1 estimates]{Smooth versus direct Admin 1 estimates. Left: Combined (meta-analysis) survey estimate against combined direct estimates. Right: Combined (meta-analysis) survey estimate against direct estimates from each survey.}\label{fig:unnamed-chunk-13}
\end{figure}


\end{knitrout}

%%%%%%%%%%%%%%%%%%%%%%%%%%% Plot2 
\begin{knitrout}
\definecolor{shadecolor}{rgb}{0.969, 0.969, 0.969}\color{fgcolor}\begin{figure}[bht]

{\centering \includegraphics[width=.9\textwidth]{../Main/Figures/SmoothMedianBenin} 

}

\caption[Maps of posterior medians for Benin  over time]{Maps of posterior medians for Benin  over time.}\label{fig:unnamed-chunk-14}
\end{figure}


\end{knitrout}
%%%%%%%%%%%%%%%%%%%%%%%%%%% Plot2a
\begin{knitrout}
\definecolor{shadecolor}{rgb}{0.969, 0.969, 0.969}\color{fgcolor}\begin{figure}[bht]

{\centering \includegraphics[width=.9\textwidth]{../Main/Figures/ReductionMedianBenin} 

}

\caption[Maps of reduction of posterior median U5MR in each five-year period compared to 1990 in Benin over time]{Maps of reduction of posterior median U5MR in each five-year period compared to 1990 in Benin over time.}\label{fig:unnamed-chunk-15}
\end{figure}


\end{knitrout}
%%%%%%%%%%%%%%%%%%%%%%%%%%% Plot3 
\begin{knitrout}
\definecolor{shadecolor}{rgb}{0.969, 0.969, 0.969}\color{fgcolor}\begin{figure}[bht]

{\centering \includegraphics[width=.95\textwidth]{../Main/Figures/Yearly_v_Periods_Benin} 

}

\caption[Smoothed regional estimates over time]{Smoothed regional estimates over time. The line indicates yearly posterior median estimates and error bars indicate 95 \% posterior credible interval at each time period.}\label{fig:unnamed-chunk-16}
\end{figure}


\end{knitrout}

%%%%%%%%%%%%%%%%%%%%%%%%%%% Plot4 
\begin{knitrout}
\definecolor{shadecolor}{rgb}{0.969, 0.969, 0.969}\color{fgcolor}\begin{figure}[bht]

{\centering \includegraphics[width=.9\textwidth]{../Main/Figures/LineSubMedianBenin} 

}

\caption[Smoothed regional estimates over time compared to the direct estimates from each surveys]{Smoothed regional estimates over time compared to the direct estimates from each surveys. Direct estimates are not benchmarked with UN estimates. The line indicates posterior median and error bars indicate 95\% posterior credible interval.}\label{fig:unnamed-chunk-17}
\end{figure}


\end{knitrout}
% \subsubsection{National model results}
We further assess the RW2 model by holding out some observations, and compare the projections to the direct estimates in these holdout observations. Figure~\ref{fig:unnamed-chunk-18} compares the predicted estimates for the out-of-sample observations  with the direct estimates by holding out observations from each area in each time period.  Figure~\ref{fig:unnamed-chunk-19} compares the histogram of the bias rescaled by the total variance in the cross validation studies. Figure~\ref{fig:unnamed-chunk-20} compares the rescaled bias by region and time periods.



% %%%%%%%%%%%%%%%%%%%%%%%%%%% Plot6
% << echo=FALSE, out.width = ".9\\textwidth", fig.width = 12, fig.height = 6, fig.cap = "Out-of-sample predictions along with direct estimates in the cross validation study where all data from each time period is held out and predicted using the rest of the data.">>=
% fig_count <- fig_count + 1
% knitr::include_graphics(paste0("../Main/Figures/CV_byYear_withError_", countryname2, ".pdf")) 
% @
 
%%%%%%%%%%%%%%%%%%%%%%%%%%% Plot7
\begin{knitrout}
\definecolor{shadecolor}{rgb}{0.969, 0.969, 0.969}\color{fgcolor}\begin{figure}[bht]

{\centering \includegraphics[width=.9\textwidth]{../Main/Figures/CV_byYearRegion_withError_Benin} 

}

\caption[Out-of-sample predictions along with direct estimates in the cross validation study where data from one region in each time period is held out and predicted using the rest of the data]{Out-of-sample predictions along with direct estimates in the cross validation study where data from one region in each time period is held out and predicted using the rest of the data.}\label{fig:unnamed-chunk-18}
\end{figure}


\end{knitrout}

%%%%%%%%%%%%%%%%%%%%%%%%%%% Plot8
\begin{knitrout}
\definecolor{shadecolor}{rgb}{0.969, 0.969, 0.969}\color{fgcolor}\begin{figure}[bht]

{\centering \includegraphics[width=.9\textwidth]{../Main/Figures/CVbiasBenin} 

}

\caption[Histogram and QQ-plot of the rescaled difference between the smoothed estimates and the direct estimates in the cross validation study]{Histogram and QQ-plot of the rescaled difference between the smoothed estimates and the direct estimates in the cross validation study. The differences between the two estimates are rescaled by the square root of the total variance of the two estimates.}\label{fig:unnamed-chunk-19}
\end{figure}


\end{knitrout}

%%%%%%%%%%%%%%%%%%%%%%%%%%% Plot9
\begin{knitrout}
\definecolor{shadecolor}{rgb}{0.969, 0.969, 0.969}\color{fgcolor}\begin{figure}[bht]

{\centering \includegraphics[width=.7\textwidth]{../Main/Figures/CVbiasbyRegionBenin} 

}

\caption[Line plot of the difference between smoothed estimates and the direct estimates in the cross validation study]{Line plot of the difference between smoothed estimates and the direct estimates in the cross validation study. The differences between the two estimates are rescaled by the square root of the total variance of the two estimates.}\label{fig:unnamed-chunk-20}
\end{figure}


\end{knitrout}

%%%%%%%%%%%%%%%%%%%%%%%%%%%%%%%%%%%%%%%%%%%%%%%%%%%%%%%%%%%%%%%%%%%%%%%%%%%%%%%%%%%%%%%%%%%%%%%%%%
\clearpage
\subsection{Burkina Faso}


% \subsubsection{Summary of DHS surveys}

%%%%%%%%%%%%%%%%%%%%%%%%%%% Summary 


DHS surveys were conducted in Burkina Faso in 1993, 1999, 2003, and 2010.
% years.out[1:(length(years.out)-1)], and years.out[length(years.out)]. 

We fit both the RW2 only model to the combined national data, and compare the time trend at national level with the estimates produced by the UN and IHME in Figure~\ref{fig:unnamed-chunk-22}. We then adjusted the combined national data to the UN estimates of U5MR, and refit the models on the benchmarked data. 

%%%%%%%%%%%%%%%%%%%%%%%%%% Plot5 
\begin{knitrout}
\definecolor{shadecolor}{rgb}{0.969, 0.969, 0.969}\color{fgcolor}\begin{figure}[bht]

{\centering \includegraphics[width=.9\textwidth]{../Main/Figures/Yearly_national_BurkinaFaso} 

}

\caption[Temporal national trends along with UN (B3) estimates described in You et al]{Temporal national trends along with UN (B3) estimates described in You et al. (2015) and IHME estimates based on GBD 2015 Child Mortality Collaborators (2016). RW2 represents the smoothed national estimates using the original data before benchmarking with UN estimates. RW2-adj represents the smoothed national estimates using the benchmarked data.}\label{fig:unnamed-chunk-22}
\end{figure}


\end{knitrout}
 

We fit the RW2 model to the benchmarked data in each area. 
% The proportions of the explained variation is summarized in Table~\ref{tab:paste0(countryname, "-var")}. 
We compare the results in Figure~\ref{fig:unnamed-chunk-23} to \ref{fig:unnamed-chunk-27}.
Figure~\ref{fig:unnamed-chunk-23} compares the smoothed estimates against the direct estimates. Figure~\ref{fig:unnamed-chunk-24} and Figure~\ref{fig:unnamed-chunk-25} show the posterior median estimates of U5MR in each region over time and the reductions from 1990 period respectively.
Figure~\ref{fig:unnamed-chunk-26} shows the smoothed estimates by region over time and Figure~\ref{fig:unnamed-chunk-27} compares the smoothed estimates with direct estimates from each survey for each region over time.


% %%%%%%%%%%%%%%%%%%%%%%%%%%% Table1 
% <<echo=FALSE, results='asis'>>=
% load("rda/variance_tables.rda")
% countryname2 <- gsub(" ", "", countryname)
% variance <- tables.all[[countryname]]

% table_count <- table_count + 1

% names <- c("RW2 ($\\sigma^2_{\\gamma_{t}}$)", "ICAR ($\\sigma^2_{\\phi_{i}}$)", "IID space ($\\sigma^2_{\\theta_{i}}$)", "IID time ($\\sigma^2_{\\alpha_{t}}$)", "IID space time ($\\sigma^2_{\\delta_{it}}$)")

% variance$Proportion <- round(variance$Proportion*100, digits = 2)
% row.names(variance) <- names
% tab <- xtable(variance, digits = c(1, 3, 2),align = "l|ll",
%        label = paste0("tab:", countryname, "-var"),
%        caption = paste(country, ": summary of the variance components in the RW2 model", sep = ''))
% print(tab, comment = FALSE,sanitize.text.function = function(x) {x})
% @

%%%%%%%%%%%%%%%%%%%%%%%%%%% Plot1 
\begin{knitrout}
\definecolor{shadecolor}{rgb}{0.969, 0.969, 0.969}\color{fgcolor}\begin{figure}[bht]

{\centering \includegraphics[width=.9\textwidth]{../Main/Figures/SmoothvDirectBurkinaFaso_meta} 

}

\caption[Smooth versus direct Admin 1 estimates]{Smooth versus direct Admin 1 estimates. Left: Combined (meta-analysis) survey estimate against combined direct estimates. Right: Combined (meta-analysis) survey estimate against direct estimates from each survey.}\label{fig:unnamed-chunk-23}
\end{figure}


\end{knitrout}

%%%%%%%%%%%%%%%%%%%%%%%%%%% Plot2 
\begin{knitrout}
\definecolor{shadecolor}{rgb}{0.969, 0.969, 0.969}\color{fgcolor}\begin{figure}[bht]

{\centering \includegraphics[width=.9\textwidth]{../Main/Figures/SmoothMedianBurkinaFaso} 

}

\caption[Maps of posterior medians for Burkina Faso  over time]{Maps of posterior medians for Burkina Faso  over time.}\label{fig:unnamed-chunk-24}
\end{figure}


\end{knitrout}
%%%%%%%%%%%%%%%%%%%%%%%%%%% Plot2a
\begin{knitrout}
\definecolor{shadecolor}{rgb}{0.969, 0.969, 0.969}\color{fgcolor}\begin{figure}[bht]

{\centering \includegraphics[width=.9\textwidth]{../Main/Figures/ReductionMedianBurkinaFaso} 

}

\caption[Maps of reduction of posterior median U5MR in each five-year period compared to 1990 in Burkina Faso over time]{Maps of reduction of posterior median U5MR in each five-year period compared to 1990 in Burkina Faso over time.}\label{fig:unnamed-chunk-25}
\end{figure}


\end{knitrout}
%%%%%%%%%%%%%%%%%%%%%%%%%%% Plot3 
\begin{knitrout}
\definecolor{shadecolor}{rgb}{0.969, 0.969, 0.969}\color{fgcolor}\begin{figure}[bht]

{\centering \includegraphics[width=.95\textwidth]{../Main/Figures/Yearly_v_Periods_BurkinaFaso} 

}

\caption[Smoothed regional estimates over time]{Smoothed regional estimates over time. The line indicates yearly posterior median estimates and error bars indicate 95 \% posterior credible interval at each time period.}\label{fig:unnamed-chunk-26}
\end{figure}


\end{knitrout}

%%%%%%%%%%%%%%%%%%%%%%%%%%% Plot4 
\begin{knitrout}
\definecolor{shadecolor}{rgb}{0.969, 0.969, 0.969}\color{fgcolor}\begin{figure}[bht]

{\centering \includegraphics[width=.9\textwidth]{../Main/Figures/LineSubMedianBurkinaFaso} 

}

\caption[Smoothed regional estimates over time compared to the direct estimates from each surveys]{Smoothed regional estimates over time compared to the direct estimates from each surveys. Direct estimates are not benchmarked with UN estimates. The line indicates posterior median and error bars indicate 95\% posterior credible interval.}\label{fig:unnamed-chunk-27}
\end{figure}


\end{knitrout}
% \subsubsection{National model results}
We further assess the RW2 model by holding out some observations, and compare the projections to the direct estimates in these holdout observations. Figure~\ref{fig:unnamed-chunk-28} compares the predicted estimates for the out-of-sample observations  with the direct estimates by holding out observations from each area in each time period.  Figure~\ref{fig:unnamed-chunk-29} compares the histogram of the bias rescaled by the total variance in the cross validation studies. Figure~\ref{fig:unnamed-chunk-30} compares the rescaled bias by region and time periods.



% %%%%%%%%%%%%%%%%%%%%%%%%%%% Plot6
% << echo=FALSE, out.width = ".9\\textwidth", fig.width = 12, fig.height = 6, fig.cap = "Out-of-sample predictions along with direct estimates in the cross validation study where all data from each time period is held out and predicted using the rest of the data.">>=
% fig_count <- fig_count + 1
% knitr::include_graphics(paste0("../Main/Figures/CV_byYear_withError_", countryname2, ".pdf")) 
% @
 
%%%%%%%%%%%%%%%%%%%%%%%%%%% Plot7
\begin{knitrout}
\definecolor{shadecolor}{rgb}{0.969, 0.969, 0.969}\color{fgcolor}\begin{figure}[bht]

{\centering \includegraphics[width=.9\textwidth]{../Main/Figures/CV_byYearRegion_withError_BurkinaFaso} 

}

\caption[Out-of-sample predictions along with direct estimates in the cross validation study where data from one region in each time period is held out and predicted using the rest of the data]{Out-of-sample predictions along with direct estimates in the cross validation study where data from one region in each time period is held out and predicted using the rest of the data.}\label{fig:unnamed-chunk-28}
\end{figure}


\end{knitrout}

%%%%%%%%%%%%%%%%%%%%%%%%%%% Plot8
\begin{knitrout}
\definecolor{shadecolor}{rgb}{0.969, 0.969, 0.969}\color{fgcolor}\begin{figure}[bht]

{\centering \includegraphics[width=.9\textwidth]{../Main/Figures/CVbiasBurkinaFaso} 

}

\caption[Histogram and QQ-plot of the rescaled difference between the smoothed estimates and the direct estimates in the cross validation study]{Histogram and QQ-plot of the rescaled difference between the smoothed estimates and the direct estimates in the cross validation study. The differences between the two estimates are rescaled by the square root of the total variance of the two estimates.}\label{fig:unnamed-chunk-29}
\end{figure}


\end{knitrout}

%%%%%%%%%%%%%%%%%%%%%%%%%%% Plot9
\begin{knitrout}
\definecolor{shadecolor}{rgb}{0.969, 0.969, 0.969}\color{fgcolor}\begin{figure}[bht]

{\centering \includegraphics[width=.7\textwidth]{../Main/Figures/CVbiasbyRegionBurkinaFaso} 

}

\caption[Line plot of the difference between smoothed estimates and the direct estimates in the cross validation study]{Line plot of the difference between smoothed estimates and the direct estimates in the cross validation study. The differences between the two estimates are rescaled by the square root of the total variance of the two estimates.}\label{fig:unnamed-chunk-30}
\end{figure}


\end{knitrout}

%%%%%%%%%%%%%%%%%%%%%%%%%%%%%%%%%%%%%%%%%%%%%%%%%%%%%%%%%%%%%%%%%%%%%%%%%%%%%%%%%%%%%%%%%%%%%%%%%%
\clearpage
\subsection{Burundi}


% \subsubsection{Summary of DHS surveys}

%%%%%%%%%%%%%%%%%%%%%%%%%%% Summary 


DHS surveys were conducted in Burundi in 2010.
% years.out[1:(length(years.out)-1)], and years.out[length(years.out)]. 

We fit both the RW2 only model to the combined national data, and compare the time trend at national level with the estimates produced by the UN and IHME in Figure~\ref{fig:unnamed-chunk-32}. We then adjusted the combined national data to the UN estimates of U5MR, and refit the models on the benchmarked data. 

%%%%%%%%%%%%%%%%%%%%%%%%%% Plot5 
\begin{knitrout}
\definecolor{shadecolor}{rgb}{0.969, 0.969, 0.969}\color{fgcolor}\begin{figure}[bht]

{\centering \includegraphics[width=.9\textwidth]{../Main/Figures/Yearly_national_Burundi} 

}

\caption[Temporal national trends along with UN (B3) estimates described in You et al]{Temporal national trends along with UN (B3) estimates described in You et al. (2015) and IHME estimates based on GBD 2015 Child Mortality Collaborators (2016). RW2 represents the smoothed national estimates using the original data before benchmarking with UN estimates. RW2-adj represents the smoothed national estimates using the benchmarked data.}\label{fig:unnamed-chunk-32}
\end{figure}


\end{knitrout}
 

We fit the RW2 model to the benchmarked data in each area. 
% The proportions of the explained variation is summarized in Table~\ref{tab:paste0(countryname, "-var")}. 
We compare the results in Figure~\ref{fig:unnamed-chunk-33} to \ref{fig:unnamed-chunk-37}.
Figure~\ref{fig:unnamed-chunk-33} compares the smoothed estimates against the direct estimates. Figure~\ref{fig:unnamed-chunk-34} and Figure~\ref{fig:unnamed-chunk-35} show the posterior median estimates of U5MR in each region over time and the reductions from 1990 period respectively.
Figure~\ref{fig:unnamed-chunk-36} shows the smoothed estimates by region over time and Figure~\ref{fig:unnamed-chunk-37} compares the smoothed estimates with direct estimates from each survey for each region over time.


% %%%%%%%%%%%%%%%%%%%%%%%%%%% Table1 
% <<echo=FALSE, results='asis'>>=
% load("rda/variance_tables.rda")
% countryname2 <- gsub(" ", "", countryname)
% variance <- tables.all[[countryname]]

% table_count <- table_count + 1

% names <- c("RW2 ($\\sigma^2_{\\gamma_{t}}$)", "ICAR ($\\sigma^2_{\\phi_{i}}$)", "IID space ($\\sigma^2_{\\theta_{i}}$)", "IID time ($\\sigma^2_{\\alpha_{t}}$)", "IID space time ($\\sigma^2_{\\delta_{it}}$)")

% variance$Proportion <- round(variance$Proportion*100, digits = 2)
% row.names(variance) <- names
% tab <- xtable(variance, digits = c(1, 3, 2),align = "l|ll",
%        label = paste0("tab:", countryname, "-var"),
%        caption = paste(country, ": summary of the variance components in the RW2 model", sep = ''))
% print(tab, comment = FALSE,sanitize.text.function = function(x) {x})
% @

%%%%%%%%%%%%%%%%%%%%%%%%%%% Plot1 
\begin{knitrout}
\definecolor{shadecolor}{rgb}{0.969, 0.969, 0.969}\color{fgcolor}\begin{figure}[bht]

{\centering \includegraphics[width=.9\textwidth]{../Main/Figures/SmoothvDirectBurundi_meta} 

}

\caption[Smooth versus direct Admin 1 estimates]{Smooth versus direct Admin 1 estimates. Left: Combined (meta-analysis) survey estimate against combined direct estimates. Right: Combined (meta-analysis) survey estimate against direct estimates from each survey.}\label{fig:unnamed-chunk-33}
\end{figure}


\end{knitrout}

%%%%%%%%%%%%%%%%%%%%%%%%%%% Plot2 
\begin{knitrout}
\definecolor{shadecolor}{rgb}{0.969, 0.969, 0.969}\color{fgcolor}\begin{figure}[bht]

{\centering \includegraphics[width=.9\textwidth]{../Main/Figures/SmoothMedianBurundi} 

}

\caption[Maps of posterior medians for Burundi  over time]{Maps of posterior medians for Burundi  over time.}\label{fig:unnamed-chunk-34}
\end{figure}


\end{knitrout}
%%%%%%%%%%%%%%%%%%%%%%%%%%% Plot2a
\begin{knitrout}
\definecolor{shadecolor}{rgb}{0.969, 0.969, 0.969}\color{fgcolor}\begin{figure}[bht]

{\centering \includegraphics[width=.9\textwidth]{../Main/Figures/ReductionMedianBurundi} 

}

\caption[Maps of reduction of posterior median U5MR in each five-year period compared to 1990 in Burundi over time]{Maps of reduction of posterior median U5MR in each five-year period compared to 1990 in Burundi over time.}\label{fig:unnamed-chunk-35}
\end{figure}


\end{knitrout}
%%%%%%%%%%%%%%%%%%%%%%%%%%% Plot3 
\begin{knitrout}
\definecolor{shadecolor}{rgb}{0.969, 0.969, 0.969}\color{fgcolor}\begin{figure}[bht]

{\centering \includegraphics[width=.95\textwidth]{../Main/Figures/Yearly_v_Periods_Burundi} 

}

\caption[Smoothed regional estimates over time]{Smoothed regional estimates over time. The line indicates yearly posterior median estimates and error bars indicate 95 \% posterior credible interval at each time period.}\label{fig:unnamed-chunk-36}
\end{figure}


\end{knitrout}

%%%%%%%%%%%%%%%%%%%%%%%%%%% Plot4 
\begin{knitrout}
\definecolor{shadecolor}{rgb}{0.969, 0.969, 0.969}\color{fgcolor}\begin{figure}[bht]

{\centering \includegraphics[width=.9\textwidth]{../Main/Figures/LineSubMedianBurundi} 

}

\caption[Smoothed regional estimates over time compared to the direct estimates from each surveys]{Smoothed regional estimates over time compared to the direct estimates from each surveys. Direct estimates are not benchmarked with UN estimates. The line indicates posterior median and error bars indicate 95\% posterior credible interval.}\label{fig:unnamed-chunk-37}
\end{figure}


\end{knitrout}
% \subsubsection{National model results}
We further assess the RW2 model by holding out some observations, and compare the projections to the direct estimates in these holdout observations. Figure~\ref{fig:unnamed-chunk-38} compares the predicted estimates for the out-of-sample observations  with the direct estimates by holding out observations from each area in each time period.  Figure~\ref{fig:unnamed-chunk-39} compares the histogram of the bias rescaled by the total variance in the cross validation studies. Figure~\ref{fig:unnamed-chunk-40} compares the rescaled bias by region and time periods.



% %%%%%%%%%%%%%%%%%%%%%%%%%%% Plot6
% << echo=FALSE, out.width = ".9\\textwidth", fig.width = 12, fig.height = 6, fig.cap = "Out-of-sample predictions along with direct estimates in the cross validation study where all data from each time period is held out and predicted using the rest of the data.">>=
% fig_count <- fig_count + 1
% knitr::include_graphics(paste0("../Main/Figures/CV_byYear_withError_", countryname2, ".pdf")) 
% @
 
%%%%%%%%%%%%%%%%%%%%%%%%%%% Plot7
\begin{knitrout}
\definecolor{shadecolor}{rgb}{0.969, 0.969, 0.969}\color{fgcolor}\begin{figure}[bht]

{\centering \includegraphics[width=.9\textwidth]{../Main/Figures/CV_byYearRegion_withError_Burundi} 

}

\caption[Out-of-sample predictions along with direct estimates in the cross validation study where data from one region in each time period is held out and predicted using the rest of the data]{Out-of-sample predictions along with direct estimates in the cross validation study where data from one region in each time period is held out and predicted using the rest of the data.}\label{fig:unnamed-chunk-38}
\end{figure}


\end{knitrout}

%%%%%%%%%%%%%%%%%%%%%%%%%%% Plot8
\begin{knitrout}
\definecolor{shadecolor}{rgb}{0.969, 0.969, 0.969}\color{fgcolor}\begin{figure}[bht]

{\centering \includegraphics[width=.9\textwidth]{../Main/Figures/CVbiasBurundi} 

}

\caption[Histogram and QQ-plot of the rescaled difference between the smoothed estimates and the direct estimates in the cross validation study]{Histogram and QQ-plot of the rescaled difference between the smoothed estimates and the direct estimates in the cross validation study. The differences between the two estimates are rescaled by the square root of the total variance of the two estimates.}\label{fig:unnamed-chunk-39}
\end{figure}


\end{knitrout}

%%%%%%%%%%%%%%%%%%%%%%%%%%% Plot9
\begin{knitrout}
\definecolor{shadecolor}{rgb}{0.969, 0.969, 0.969}\color{fgcolor}\begin{figure}[bht]

{\centering \includegraphics[width=.7\textwidth]{../Main/Figures/CVbiasbyRegionBurundi} 

}

\caption[Line plot of the difference between smoothed estimates and the direct estimates in the cross validation study]{Line plot of the difference between smoothed estimates and the direct estimates in the cross validation study. The differences between the two estimates are rescaled by the square root of the total variance of the two estimates.}\label{fig:unnamed-chunk-40}
\end{figure}


\end{knitrout}

%%%%%%%%%%%%%%%%%%%%%%%%%%%%%%%%%%%%%%%%%%%%%%%%%%%%%%%%%%%%%%%%%%%%%%%%%%%%%%%%%%%%%%%%%%%%%%%%%
\clearpage
\subsection{Cameroon}


% \subsubsection{Summary of DHS surveys}

%%%%%%%%%%%%%%%%%%%%%%%%%%% Summary 


DHS surveys were conducted in Cameroon in 1998, 2004, and 2011.
% years.out[1:(length(years.out)-1)], and years.out[length(years.out)]. 

We fit both the RW2 only model to the combined national data, and compare the time trend at national level with the estimates produced by the UN and IHME in Figure~\ref{fig:unnamed-chunk-42}. We then adjusted the combined national data to the UN estimates of U5MR, and refit the models on the benchmarked data. 

%%%%%%%%%%%%%%%%%%%%%%%%%% Plot5 
\begin{knitrout}
\definecolor{shadecolor}{rgb}{0.969, 0.969, 0.969}\color{fgcolor}\begin{figure}[bht]

{\centering \includegraphics[width=.9\textwidth]{../Main/Figures/Yearly_national_Cameroon} 

}

\caption[Temporal national trends along with UN (B3) estimates described in You et al]{Temporal national trends along with UN (B3) estimates described in You et al. (2015) and IHME estimates based on GBD 2015 Child Mortality Collaborators (2016). RW2 represents the smoothed national estimates using the original data before benchmarking with UN estimates. RW2-adj represents the smoothed national estimates using the benchmarked data.}\label{fig:unnamed-chunk-42}
\end{figure}


\end{knitrout}
 

We fit the RW2 model to the benchmarked data in each area. 
% The proportions of the explained variation is summarized in Table~\ref{tab:paste0(countryname, "-var")}. 
We compare the results in Figure~\ref{fig:unnamed-chunk-43} to \ref{fig:unnamed-chunk-47}.
Figure~\ref{fig:unnamed-chunk-43} compares the smoothed estimates against the direct estimates. Figure~\ref{fig:unnamed-chunk-44} and Figure~\ref{fig:unnamed-chunk-45} show the posterior median estimates of U5MR in each region over time and the reductions from 1990 period respectively.
Figure~\ref{fig:unnamed-chunk-46} shows the smoothed estimates by region over time and Figure~\ref{fig:unnamed-chunk-47} compares the smoothed estimates with direct estimates from each survey for each region over time.


% %%%%%%%%%%%%%%%%%%%%%%%%%%% Table1 
% <<echo=FALSE, results='asis'>>=
% load("rda/variance_tables.rda")
% countryname2 <- gsub(" ", "", countryname)
% variance <- tables.all[[countryname]]

% table_count <- table_count + 1

% names <- c("RW2 ($\\sigma^2_{\\gamma_{t}}$)", "ICAR ($\\sigma^2_{\\phi_{i}}$)", "IID space ($\\sigma^2_{\\theta_{i}}$)", "IID time ($\\sigma^2_{\\alpha_{t}}$)", "IID space time ($\\sigma^2_{\\delta_{it}}$)")

% variance$Proportion <- round(variance$Proportion*100, digits = 2)
% row.names(variance) <- names
% tab <- xtable(variance, digits = c(1, 3, 2),align = "l|ll",
%        label = paste0("tab:", countryname, "-var"),
%        caption = paste(country, ": summary of the variance components in the RW2 model", sep = ''))
% print(tab, comment = FALSE,sanitize.text.function = function(x) {x})
% @

%%%%%%%%%%%%%%%%%%%%%%%%%%% Plot1 
\begin{knitrout}
\definecolor{shadecolor}{rgb}{0.969, 0.969, 0.969}\color{fgcolor}\begin{figure}[bht]

{\centering \includegraphics[width=.9\textwidth]{../Main/Figures/SmoothvDirectCameroon_meta} 

}

\caption[Smooth versus direct Admin 1 estimates]{Smooth versus direct Admin 1 estimates. Left: Combined (meta-analysis) survey estimate against combined direct estimates. Right: Combined (meta-analysis) survey estimate against direct estimates from each survey.}\label{fig:unnamed-chunk-43}
\end{figure}


\end{knitrout}

%%%%%%%%%%%%%%%%%%%%%%%%%%% Plot2 
\begin{knitrout}
\definecolor{shadecolor}{rgb}{0.969, 0.969, 0.969}\color{fgcolor}\begin{figure}[bht]

{\centering \includegraphics[width=.9\textwidth]{../Main/Figures/SmoothMedianCameroon} 

}

\caption[Maps of posterior medians for Cameroon  over time]{Maps of posterior medians for Cameroon  over time.}\label{fig:unnamed-chunk-44}
\end{figure}


\end{knitrout}
%%%%%%%%%%%%%%%%%%%%%%%%%%% Plot2a
\begin{knitrout}
\definecolor{shadecolor}{rgb}{0.969, 0.969, 0.969}\color{fgcolor}\begin{figure}[bht]

{\centering \includegraphics[width=.9\textwidth]{../Main/Figures/ReductionMedianCameroon} 

}

\caption[Maps of reduction of posterior median U5MR in each five-year period compared to 1990 in Cameroon over time]{Maps of reduction of posterior median U5MR in each five-year period compared to 1990 in Cameroon over time.}\label{fig:unnamed-chunk-45}
\end{figure}


\end{knitrout}
%%%%%%%%%%%%%%%%%%%%%%%%%%% Plot3 
\begin{knitrout}
\definecolor{shadecolor}{rgb}{0.969, 0.969, 0.969}\color{fgcolor}\begin{figure}[bht]

{\centering \includegraphics[width=.95\textwidth]{../Main/Figures/Yearly_v_Periods_Cameroon} 

}

\caption[Smoothed regional estimates over time]{Smoothed regional estimates over time. The line indicates yearly posterior median estimates and error bars indicate 95 \% posterior credible interval at each time period.}\label{fig:unnamed-chunk-46}
\end{figure}


\end{knitrout}

%%%%%%%%%%%%%%%%%%%%%%%%%%% Plot4 
\begin{knitrout}
\definecolor{shadecolor}{rgb}{0.969, 0.969, 0.969}\color{fgcolor}\begin{figure}[bht]

{\centering \includegraphics[width=.9\textwidth]{../Main/Figures/LineSubMedianCameroon} 

}

\caption[Smoothed regional estimates over time compared to the direct estimates from each surveys]{Smoothed regional estimates over time compared to the direct estimates from each surveys. Direct estimates are not benchmarked with UN estimates. The line indicates posterior median and error bars indicate 95\% posterior credible interval.}\label{fig:unnamed-chunk-47}
\end{figure}


\end{knitrout}
% \subsubsection{National model results}
We further assess the RW2 model by holding out some observations, and compare the projections to the direct estimates in these holdout observations. Figure~\ref{fig:unnamed-chunk-48} compares the predicted estimates for the out-of-sample observations  with the direct estimates by holding out observations from each area in each time period.  Figure~\ref{fig:unnamed-chunk-49} compares the histogram of the bias rescaled by the total variance in the cross validation studies. Figure~\ref{fig:unnamed-chunk-50} compares the rescaled bias by region and time periods.



% %%%%%%%%%%%%%%%%%%%%%%%%%%% Plot6
% << echo=FALSE, out.width = ".9\\textwidth", fig.width = 12, fig.height = 6, fig.cap = "Out-of-sample predictions along with direct estimates in the cross validation study where all data from each time period is held out and predicted using the rest of the data.">>=
% fig_count <- fig_count + 1
% knitr::include_graphics(paste0("../Main/Figures/CV_byYear_withError_", countryname2, ".pdf")) 
% @
 
%%%%%%%%%%%%%%%%%%%%%%%%%%% Plot7
\begin{knitrout}
\definecolor{shadecolor}{rgb}{0.969, 0.969, 0.969}\color{fgcolor}\begin{figure}[bht]

{\centering \includegraphics[width=.9\textwidth]{../Main/Figures/CV_byYearRegion_withError_Cameroon} 

}

\caption[Out-of-sample predictions along with direct estimates in the cross validation study where data from one region in each time period is held out and predicted using the rest of the data]{Out-of-sample predictions along with direct estimates in the cross validation study where data from one region in each time period is held out and predicted using the rest of the data.}\label{fig:unnamed-chunk-48}
\end{figure}


\end{knitrout}

%%%%%%%%%%%%%%%%%%%%%%%%%%% Plot8
\begin{knitrout}
\definecolor{shadecolor}{rgb}{0.969, 0.969, 0.969}\color{fgcolor}\begin{figure}[bht]

{\centering \includegraphics[width=.9\textwidth]{../Main/Figures/CVbiasCameroon} 

}

\caption[Histogram and QQ-plot of the rescaled difference between the smoothed estimates and the direct estimates in the cross validation study]{Histogram and QQ-plot of the rescaled difference between the smoothed estimates and the direct estimates in the cross validation study. The differences between the two estimates are rescaled by the square root of the total variance of the two estimates.}\label{fig:unnamed-chunk-49}
\end{figure}


\end{knitrout}

%%%%%%%%%%%%%%%%%%%%%%%%%%% Plot9
\begin{knitrout}
\definecolor{shadecolor}{rgb}{0.969, 0.969, 0.969}\color{fgcolor}\begin{figure}[bht]

{\centering \includegraphics[width=.7\textwidth]{../Main/Figures/CVbiasbyRegionCameroon} 

}

\caption[Line plot of the difference between smoothed estimates and the direct estimates in the cross validation study]{Line plot of the difference between smoothed estimates and the direct estimates in the cross validation study. The differences between the two estimates are rescaled by the square root of the total variance of the two estimates.}\label{fig:unnamed-chunk-50}
\end{figure}


\end{knitrout}


%%%%%%%%%%%%%%%%%%%%%%%%%%%%%%%%%%%%%%%%%%%%%%%%%%%%%%%%%%%%%%%%%%%%%%%%%%%%%%%%%%%%%%%%%%%%%%%%%%
\clearpage
\subsection{Chad}


% \subsubsection{Summary of DHS surveys}

%%%%%%%%%%%%%%%%%%%%%%%%%%% Summary 


DHS surveys were conducted in Chad in 2004, and 2015.
% years.out[1:(length(years.out)-1)], and years.out[length(years.out)]. 

We fit both the RW2 only model to the combined national data, and compare the time trend at national level with the estimates produced by the UN and IHME in Figure~\ref{fig:unnamed-chunk-52}. We then adjusted the combined national data to the UN estimates of U5MR, and refit the models on the benchmarked data. 

%%%%%%%%%%%%%%%%%%%%%%%%%% Plot5 
\begin{knitrout}
\definecolor{shadecolor}{rgb}{0.969, 0.969, 0.969}\color{fgcolor}\begin{figure}[bht]

{\centering \includegraphics[width=.9\textwidth]{../Main/Figures/Yearly_national_Chad} 

}

\caption[Temporal national trends along with UN (B3) estimates described in You et al]{Temporal national trends along with UN (B3) estimates described in You et al. (2015) and IHME estimates based on GBD 2015 Child Mortality Collaborators (2016). RW2 represents the smoothed national estimates using the original data before benchmarking with UN estimates. RW2-adj represents the smoothed national estimates using the benchmarked data.}\label{fig:unnamed-chunk-52}
\end{figure}


\end{knitrout}
 

We fit the RW2 model to the benchmarked data in each area. 
% The proportions of the explained variation is summarized in Table~\ref{tab:paste0(countryname, "-var")}. 
We compare the results in Figure~\ref{fig:unnamed-chunk-53} to \ref{fig:unnamed-chunk-57}.
Figure~\ref{fig:unnamed-chunk-53} compares the smoothed estimates against the direct estimates. Figure~\ref{fig:unnamed-chunk-54} and Figure~\ref{fig:unnamed-chunk-55} show the posterior median estimates of U5MR in each region over time and the reductions from 1990 period respectively.
Figure~\ref{fig:unnamed-chunk-56} shows the smoothed estimates by region over time and Figure~\ref{fig:unnamed-chunk-57} compares the smoothed estimates with direct estimates from each survey for each region over time.


% %%%%%%%%%%%%%%%%%%%%%%%%%%% Table1 
% <<echo=FALSE, results='asis'>>=
% load("rda/variance_tables.rda")
% countryname2 <- gsub(" ", "", countryname)
% variance <- tables.all[[countryname]]

% table_count <- table_count + 1

% names <- c("RW2 ($\\sigma^2_{\\gamma_{t}}$)", "ICAR ($\\sigma^2_{\\phi_{i}}$)", "IID space ($\\sigma^2_{\\theta_{i}}$)", "IID time ($\\sigma^2_{\\alpha_{t}}$)", "IID space time ($\\sigma^2_{\\delta_{it}}$)")

% variance$Proportion <- round(variance$Proportion*100, digits = 2)
% row.names(variance) <- names
% tab <- xtable(variance, digits = c(1, 3, 2),align = "l|ll",
%        label = paste0("tab:", countryname, "-var"),
%        caption = paste(country, ": summary of the variance components in the RW2 model", sep = ''))
% print(tab, comment = FALSE,sanitize.text.function = function(x) {x})
% @

%%%%%%%%%%%%%%%%%%%%%%%%%%% Plot1 
\begin{knitrout}
\definecolor{shadecolor}{rgb}{0.969, 0.969, 0.969}\color{fgcolor}\begin{figure}[bht]

{\centering \includegraphics[width=.9\textwidth]{../Main/Figures/SmoothvDirectChad_meta} 

}

\caption[Smooth versus direct Admin 1 estimates]{Smooth versus direct Admin 1 estimates. Left: Combined (meta-analysis) survey estimate against combined direct estimates. Right: Combined (meta-analysis) survey estimate against direct estimates from each survey.}\label{fig:unnamed-chunk-53}
\end{figure}


\end{knitrout}

%%%%%%%%%%%%%%%%%%%%%%%%%%% Plot2 
\begin{knitrout}
\definecolor{shadecolor}{rgb}{0.969, 0.969, 0.969}\color{fgcolor}\begin{figure}[bht]

{\centering \includegraphics[width=.9\textwidth]{../Main/Figures/SmoothMedianChad} 

}

\caption[Maps of posterior medians for Chad  over time]{Maps of posterior medians for Chad  over time.}\label{fig:unnamed-chunk-54}
\end{figure}


\end{knitrout}
%%%%%%%%%%%%%%%%%%%%%%%%%%% Plot2a
\begin{knitrout}
\definecolor{shadecolor}{rgb}{0.969, 0.969, 0.969}\color{fgcolor}\begin{figure}[bht]

{\centering \includegraphics[width=.9\textwidth]{../Main/Figures/ReductionMedianChad} 

}

\caption[Maps of reduction of posterior median U5MR in each five-year period compared to 1990 in Chad over time]{Maps of reduction of posterior median U5MR in each five-year period compared to 1990 in Chad over time.}\label{fig:unnamed-chunk-55}
\end{figure}


\end{knitrout}
%%%%%%%%%%%%%%%%%%%%%%%%%%% Plot3 
\begin{knitrout}
\definecolor{shadecolor}{rgb}{0.969, 0.969, 0.969}\color{fgcolor}\begin{figure}[bht]

{\centering \includegraphics[width=.95\textwidth]{../Main/Figures/Yearly_v_Periods_Chad} 

}

\caption[Smoothed regional estimates over time]{Smoothed regional estimates over time. The line indicates yearly posterior median estimates and error bars indicate 95 \% posterior credible interval at each time period.}\label{fig:unnamed-chunk-56}
\end{figure}


\end{knitrout}

%%%%%%%%%%%%%%%%%%%%%%%%%%% Plot4 
\begin{knitrout}
\definecolor{shadecolor}{rgb}{0.969, 0.969, 0.969}\color{fgcolor}\begin{figure}[bht]

{\centering \includegraphics[width=.9\textwidth]{../Main/Figures/LineSubMedianChad} 

}

\caption[Smoothed regional estimates over time compared to the direct estimates from each surveys]{Smoothed regional estimates over time compared to the direct estimates from each surveys. Direct estimates are not benchmarked with UN estimates. The line indicates posterior median and error bars indicate 95\% posterior credible interval.}\label{fig:unnamed-chunk-57}
\end{figure}


\end{knitrout}
% \subsubsection{National model results}
We further assess the RW2 model by holding out some observations, and compare the projections to the direct estimates in these holdout observations. Figure~\ref{fig:unnamed-chunk-58} compares the predicted estimates for the out-of-sample observations  with the direct estimates by holding out observations from each area in each time period.  Figure~\ref{fig:unnamed-chunk-59} compares the histogram of the bias rescaled by the total variance in the cross validation studies. Figure~\ref{fig:unnamed-chunk-60} compares the rescaled bias by region and time periods.



% %%%%%%%%%%%%%%%%%%%%%%%%%%% Plot6
% << echo=FALSE, out.width = ".9\\textwidth", fig.width = 12, fig.height = 6, fig.cap = "Out-of-sample predictions along with direct estimates in the cross validation study where all data from each time period is held out and predicted using the rest of the data.">>=
% fig_count <- fig_count + 1
% knitr::include_graphics(paste0("../Main/Figures/CV_byYear_withError_", countryname2, ".pdf")) 
% @
 
%%%%%%%%%%%%%%%%%%%%%%%%%%% Plot7
\begin{knitrout}
\definecolor{shadecolor}{rgb}{0.969, 0.969, 0.969}\color{fgcolor}\begin{figure}[bht]

{\centering \includegraphics[width=.9\textwidth]{../Main/Figures/CV_byYearRegion_withError_Chad} 

}

\caption[Out-of-sample predictions along with direct estimates in the cross validation study where data from one region in each time period is held out and predicted using the rest of the data]{Out-of-sample predictions along with direct estimates in the cross validation study where data from one region in each time period is held out and predicted using the rest of the data.}\label{fig:unnamed-chunk-58}
\end{figure}


\end{knitrout}

%%%%%%%%%%%%%%%%%%%%%%%%%%% Plot8
\begin{knitrout}
\definecolor{shadecolor}{rgb}{0.969, 0.969, 0.969}\color{fgcolor}\begin{figure}[bht]

{\centering \includegraphics[width=.9\textwidth]{../Main/Figures/CVbiasChad} 

}

\caption[Histogram and QQ-plot of the rescaled difference between the smoothed estimates and the direct estimates in the cross validation study]{Histogram and QQ-plot of the rescaled difference between the smoothed estimates and the direct estimates in the cross validation study. The differences between the two estimates are rescaled by the square root of the total variance of the two estimates.}\label{fig:unnamed-chunk-59}
\end{figure}


\end{knitrout}

%%%%%%%%%%%%%%%%%%%%%%%%%%% Plot9
\begin{knitrout}
\definecolor{shadecolor}{rgb}{0.969, 0.969, 0.969}\color{fgcolor}\begin{figure}[bht]

{\centering \includegraphics[width=.7\textwidth]{../Main/Figures/CVbiasbyRegionChad} 

}

\caption[Line plot of the difference between smoothed estimates and the direct estimates in the cross validation study]{Line plot of the difference between smoothed estimates and the direct estimates in the cross validation study. The differences between the two estimates are rescaled by the square root of the total variance of the two estimates.}\label{fig:unnamed-chunk-60}
\end{figure}


\end{knitrout}

%%%%%%%%%%%%%%%%%%%%%%%%%%%%%%%%%%%%%%%%%%%%%%%%%%%%%%%%%%%%%%%%%%%%%%%%%%%%%%%%%%%%%%%%%%%%%%%%%%
\clearpage
\subsection{Comoros}


% \subsubsection{Summary of DHS surveys}

%%%%%%%%%%%%%%%%%%%%%%%%%%% Summary 


DHS surveys were conducted in Comoros in 1996, and 2012.
% years.out[1:(length(years.out)-1)], and years.out[length(years.out)]. 

We fit both the RW2 only model to the combined national data, and compare the time trend at national level with the estimates produced by the UN and IHME in Figure~\ref{fig:unnamed-chunk-62}. We then adjusted the combined national data to the UN estimates of U5MR, and refit the models on the benchmarked data. 

%%%%%%%%%%%%%%%%%%%%%%%%%% Plot5 
\begin{knitrout}
\definecolor{shadecolor}{rgb}{0.969, 0.969, 0.969}\color{fgcolor}\begin{figure}[bht]

{\centering \includegraphics[width=.9\textwidth]{../Main/Figures/Yearly_national_Comoros} 

}

\caption[Temporal national trends along with UN (B3) estimates described in You et al]{Temporal national trends along with UN (B3) estimates described in You et al. (2015) and IHME estimates based on GBD 2015 Child Mortality Collaborators (2016). RW2 represents the smoothed national estimates using the original data before benchmarking with UN estimates. RW2-adj represents the smoothed national estimates using the benchmarked data.}\label{fig:unnamed-chunk-62}
\end{figure}


\end{knitrout}
 

We fit the RW2 model to the benchmarked data in each area. 
% The proportions of the explained variation is summarized in Table~\ref{tab:paste0(countryname, "-var")}. 
We compare the results in Figure~\ref{fig:unnamed-chunk-63} to \ref{fig:unnamed-chunk-67}.
Figure~\ref{fig:unnamed-chunk-63} compares the smoothed estimates against the direct estimates. Figure~\ref{fig:unnamed-chunk-64} and Figure~\ref{fig:unnamed-chunk-65} show the posterior median estimates of U5MR in each region over time and the reductions from 1990 period respectively.
Figure~\ref{fig:unnamed-chunk-66} shows the smoothed estimates by region over time and Figure~\ref{fig:unnamed-chunk-67} compares the smoothed estimates with direct estimates from each survey for each region over time.


% %%%%%%%%%%%%%%%%%%%%%%%%%%% Table1 
% <<echo=FALSE, results='asis'>>=
% load("rda/variance_tables.rda")
% countryname2 <- gsub(" ", "", countryname)
% variance <- tables.all[[countryname]]

% table_count <- table_count + 1

% names <- c("RW2 ($\\sigma^2_{\\gamma_{t}}$)", "ICAR ($\\sigma^2_{\\phi_{i}}$)", "IID space ($\\sigma^2_{\\theta_{i}}$)", "IID time ($\\sigma^2_{\\alpha_{t}}$)", "IID space time ($\\sigma^2_{\\delta_{it}}$)")

% variance$Proportion <- round(variance$Proportion*100, digits = 2)
% row.names(variance) <- names
% tab <- xtable(variance, digits = c(1, 3, 2),align = "l|ll",
%        label = paste0("tab:", countryname, "-var"),
%        caption = paste(country, ": summary of the variance components in the RW2 model", sep = ''))
% print(tab, comment = FALSE,sanitize.text.function = function(x) {x})
% @

%%%%%%%%%%%%%%%%%%%%%%%%%%% Plot1 
\begin{knitrout}
\definecolor{shadecolor}{rgb}{0.969, 0.969, 0.969}\color{fgcolor}\begin{figure}[bht]

{\centering \includegraphics[width=.9\textwidth]{../Main/Figures/SmoothvDirectComoros_meta} 

}

\caption[Smooth versus direct Admin 1 estimates]{Smooth versus direct Admin 1 estimates. Left: Combined (meta-analysis) survey estimate against combined direct estimates. Right: Combined (meta-analysis) survey estimate against direct estimates from each survey.}\label{fig:unnamed-chunk-63}
\end{figure}


\end{knitrout}

%%%%%%%%%%%%%%%%%%%%%%%%%%% Plot2 
\begin{knitrout}
\definecolor{shadecolor}{rgb}{0.969, 0.969, 0.969}\color{fgcolor}\begin{figure}[bht]

{\centering \includegraphics[width=.9\textwidth]{../Main/Figures/SmoothMedianComoros} 

}

\caption[Maps of posterior medians for Comoros  over time]{Maps of posterior medians for Comoros  over time.}\label{fig:unnamed-chunk-64}
\end{figure}


\end{knitrout}
%%%%%%%%%%%%%%%%%%%%%%%%%%% Plot2a
\begin{knitrout}
\definecolor{shadecolor}{rgb}{0.969, 0.969, 0.969}\color{fgcolor}\begin{figure}[bht]

{\centering \includegraphics[width=.9\textwidth]{../Main/Figures/ReductionMedianComoros} 

}

\caption[Maps of reduction of posterior median U5MR in each five-year period compared to 1990 in Comoros over time]{Maps of reduction of posterior median U5MR in each five-year period compared to 1990 in Comoros over time.}\label{fig:unnamed-chunk-65}
\end{figure}


\end{knitrout}
%%%%%%%%%%%%%%%%%%%%%%%%%%% Plot3 
\begin{knitrout}
\definecolor{shadecolor}{rgb}{0.969, 0.969, 0.969}\color{fgcolor}\begin{figure}[bht]

{\centering \includegraphics[width=.95\textwidth]{../Main/Figures/Yearly_v_Periods_Comoros} 

}

\caption[Smoothed regional estimates over time]{Smoothed regional estimates over time. The line indicates yearly posterior median estimates and error bars indicate 95 \% posterior credible interval at each time period.}\label{fig:unnamed-chunk-66}
\end{figure}


\end{knitrout}

%%%%%%%%%%%%%%%%%%%%%%%%%%% Plot4 
\begin{knitrout}
\definecolor{shadecolor}{rgb}{0.969, 0.969, 0.969}\color{fgcolor}\begin{figure}[bht]

{\centering \includegraphics[width=.9\textwidth]{../Main/Figures/LineSubMedianComoros} 

}

\caption[Smoothed regional estimates over time compared to the direct estimates from each surveys]{Smoothed regional estimates over time compared to the direct estimates from each surveys. Direct estimates are not benchmarked with UN estimates. The line indicates posterior median and error bars indicate 95\% posterior credible interval.}\label{fig:unnamed-chunk-67}
\end{figure}


\end{knitrout}
% \subsubsection{National model results}
We further assess the RW2 model by holding out some observations, and compare the projections to the direct estimates in these holdout observations. Figure~\ref{fig:unnamed-chunk-68} compares the predicted estimates for the out-of-sample observations  with the direct estimates by holding out observations from each area in each time period.  Figure~\ref{fig:unnamed-chunk-69} compares the histogram of the bias rescaled by the total variance in the cross validation studies. Figure~\ref{fig:unnamed-chunk-70} compares the rescaled bias by region and time periods.



% %%%%%%%%%%%%%%%%%%%%%%%%%%% Plot6
% << echo=FALSE, out.width = ".9\\textwidth", fig.width = 12, fig.height = 6, fig.cap = "Out-of-sample predictions along with direct estimates in the cross validation study where all data from each time period is held out and predicted using the rest of the data.">>=
% fig_count <- fig_count + 1
% knitr::include_graphics(paste0("../Main/Figures/CV_byYear_withError_", countryname2, ".pdf")) 
% @
 
%%%%%%%%%%%%%%%%%%%%%%%%%%% Plot7
\begin{knitrout}
\definecolor{shadecolor}{rgb}{0.969, 0.969, 0.969}\color{fgcolor}\begin{figure}[bht]

{\centering \includegraphics[width=.9\textwidth]{../Main/Figures/CV_byYearRegion_withError_Comoros} 

}

\caption[Out-of-sample predictions along with direct estimates in the cross validation study where data from one region in each time period is held out and predicted using the rest of the data]{Out-of-sample predictions along with direct estimates in the cross validation study where data from one region in each time period is held out and predicted using the rest of the data.}\label{fig:unnamed-chunk-68}
\end{figure}


\end{knitrout}

%%%%%%%%%%%%%%%%%%%%%%%%%%% Plot8
\begin{knitrout}
\definecolor{shadecolor}{rgb}{0.969, 0.969, 0.969}\color{fgcolor}\begin{figure}[bht]

{\centering \includegraphics[width=.9\textwidth]{../Main/Figures/CVbiasComoros} 

}

\caption[Histogram and QQ-plot of the rescaled difference between the smoothed estimates and the direct estimates in the cross validation study]{Histogram and QQ-plot of the rescaled difference between the smoothed estimates and the direct estimates in the cross validation study. The differences between the two estimates are rescaled by the square root of the total variance of the two estimates.}\label{fig:unnamed-chunk-69}
\end{figure}


\end{knitrout}

%%%%%%%%%%%%%%%%%%%%%%%%%%% Plot9
\begin{knitrout}
\definecolor{shadecolor}{rgb}{0.969, 0.969, 0.969}\color{fgcolor}\begin{figure}[bht]

{\centering \includegraphics[width=.7\textwidth]{../Main/Figures/CVbiasbyRegionComoros} 

}

\caption[Line plot of the difference between smoothed estimates and the direct estimates in the cross validation study]{Line plot of the difference between smoothed estimates and the direct estimates in the cross validation study. The differences between the two estimates are rescaled by the square root of the total variance of the two estimates.}\label{fig:unnamed-chunk-70}
\end{figure}


\end{knitrout}

%%%%%%%%%%%%%%%%%%%%%%%%%%%%%%%%%%%%%%%%%%%%%%%%%%%%%%%%%%%%%%%%%%%%%%%%%%%%%%%%%%%%%%%%%%%%%%%%%%
\clearpage
\subsection{Congo}


% \subsubsection{Summary of DHS surveys}

%%%%%%%%%%%%%%%%%%%%%%%%%%% Summary 


DHS surveys were conducted in Congo in 2005, and 2012.
% years.out[1:(length(years.out)-1)], and years.out[length(years.out)]. 

We fit both the RW2 only model to the combined national data, and compare the time trend at national level with the estimates produced by the UN and IHME in Figure~\ref{fig:unnamed-chunk-72}. We then adjusted the combined national data to the UN estimates of U5MR, and refit the models on the benchmarked data. 

%%%%%%%%%%%%%%%%%%%%%%%%%% Plot5 
\begin{knitrout}
\definecolor{shadecolor}{rgb}{0.969, 0.969, 0.969}\color{fgcolor}\begin{figure}[bht]

{\centering \includegraphics[width=.9\textwidth]{../Main/Figures/Yearly_national_Congo} 

}

\caption[Temporal national trends along with UN (B3) estimates described in You et al]{Temporal national trends along with UN (B3) estimates described in You et al. (2015) and IHME estimates based on GBD 2015 Child Mortality Collaborators (2016). RW2 represents the smoothed national estimates using the original data before benchmarking with UN estimates. RW2-adj represents the smoothed national estimates using the benchmarked data.}\label{fig:unnamed-chunk-72}
\end{figure}


\end{knitrout}
 

We fit the RW2 model to the benchmarked data in each area. 
% The proportions of the explained variation is summarized in Table~\ref{tab:paste0(countryname, "-var")}. 
We compare the results in Figure~\ref{fig:unnamed-chunk-73} to \ref{fig:unnamed-chunk-77}.
Figure~\ref{fig:unnamed-chunk-73} compares the smoothed estimates against the direct estimates. Figure~\ref{fig:unnamed-chunk-74} and Figure~\ref{fig:unnamed-chunk-75} show the posterior median estimates of U5MR in each region over time and the reductions from 1990 period respectively.
Figure~\ref{fig:unnamed-chunk-76} shows the smoothed estimates by region over time and Figure~\ref{fig:unnamed-chunk-77} compares the smoothed estimates with direct estimates from each survey for each region over time.


% %%%%%%%%%%%%%%%%%%%%%%%%%%% Table1 
% <<echo=FALSE, results='asis'>>=
% load("rda/variance_tables.rda")
% countryname2 <- gsub(" ", "", countryname)
% variance <- tables.all[[countryname]]

% table_count <- table_count + 1

% names <- c("RW2 ($\\sigma^2_{\\gamma_{t}}$)", "ICAR ($\\sigma^2_{\\phi_{i}}$)", "IID space ($\\sigma^2_{\\theta_{i}}$)", "IID time ($\\sigma^2_{\\alpha_{t}}$)", "IID space time ($\\sigma^2_{\\delta_{it}}$)")

% variance$Proportion <- round(variance$Proportion*100, digits = 2)
% row.names(variance) <- names
% tab <- xtable(variance, digits = c(1, 3, 2),align = "l|ll",
%        label = paste0("tab:", countryname, "-var"),
%        caption = paste(country, ": summary of the variance components in the RW2 model", sep = ''))
% print(tab, comment = FALSE,sanitize.text.function = function(x) {x})
% @

%%%%%%%%%%%%%%%%%%%%%%%%%%% Plot1 
\begin{knitrout}
\definecolor{shadecolor}{rgb}{0.969, 0.969, 0.969}\color{fgcolor}\begin{figure}[bht]

{\centering \includegraphics[width=.9\textwidth]{../Main/Figures/SmoothvDirectCongo_meta} 

}

\caption[Smooth versus direct Admin 1 estimates]{Smooth versus direct Admin 1 estimates. Left: Combined (meta-analysis) survey estimate against combined direct estimates. Right: Combined (meta-analysis) survey estimate against direct estimates from each survey.}\label{fig:unnamed-chunk-73}
\end{figure}


\end{knitrout}

%%%%%%%%%%%%%%%%%%%%%%%%%%% Plot2 
\begin{knitrout}
\definecolor{shadecolor}{rgb}{0.969, 0.969, 0.969}\color{fgcolor}\begin{figure}[bht]

{\centering \includegraphics[width=.9\textwidth]{../Main/Figures/SmoothMedianCongo} 

}

\caption[Maps of posterior medians for Congo  over time]{Maps of posterior medians for Congo  over time.}\label{fig:unnamed-chunk-74}
\end{figure}


\end{knitrout}
%%%%%%%%%%%%%%%%%%%%%%%%%%% Plot2a
\begin{knitrout}
\definecolor{shadecolor}{rgb}{0.969, 0.969, 0.969}\color{fgcolor}\begin{figure}[bht]

{\centering \includegraphics[width=.9\textwidth]{../Main/Figures/ReductionMedianCongo} 

}

\caption[Maps of reduction of posterior median U5MR in each five-year period compared to 1990 in Congo over time]{Maps of reduction of posterior median U5MR in each five-year period compared to 1990 in Congo over time.}\label{fig:unnamed-chunk-75}
\end{figure}


\end{knitrout}
%%%%%%%%%%%%%%%%%%%%%%%%%%% Plot3 
\begin{knitrout}
\definecolor{shadecolor}{rgb}{0.969, 0.969, 0.969}\color{fgcolor}\begin{figure}[bht]

{\centering \includegraphics[width=.95\textwidth]{../Main/Figures/Yearly_v_Periods_Congo} 

}

\caption[Smoothed regional estimates over time]{Smoothed regional estimates over time. The line indicates yearly posterior median estimates and error bars indicate 95 \% posterior credible interval at each time period.}\label{fig:unnamed-chunk-76}
\end{figure}


\end{knitrout}

%%%%%%%%%%%%%%%%%%%%%%%%%%% Plot4 
\begin{knitrout}
\definecolor{shadecolor}{rgb}{0.969, 0.969, 0.969}\color{fgcolor}\begin{figure}[bht]

{\centering \includegraphics[width=.9\textwidth]{../Main/Figures/LineSubMedianCongo} 

}

\caption[Smoothed regional estimates over time compared to the direct estimates from each surveys]{Smoothed regional estimates over time compared to the direct estimates from each surveys. Direct estimates are not benchmarked with UN estimates. The line indicates posterior median and error bars indicate 95\% posterior credible interval.}\label{fig:unnamed-chunk-77}
\end{figure}


\end{knitrout}
% \subsubsection{National model results}
We further assess the RW2 model by holding out some observations, and compare the projections to the direct estimates in these holdout observations. Figure~\ref{fig:unnamed-chunk-78} compares the predicted estimates for the out-of-sample observations  with the direct estimates by holding out observations from each area in each time period.  Figure~\ref{fig:unnamed-chunk-79} compares the histogram of the bias rescaled by the total variance in the cross validation studies. Figure~\ref{fig:unnamed-chunk-80} compares the rescaled bias by region and time periods.



% %%%%%%%%%%%%%%%%%%%%%%%%%%% Plot6
% << echo=FALSE, out.width = ".9\\textwidth", fig.width = 12, fig.height = 6, fig.cap = "Out-of-sample predictions along with direct estimates in the cross validation study where all data from each time period is held out and predicted using the rest of the data.">>=
% fig_count <- fig_count + 1
% knitr::include_graphics(paste0("../Main/Figures/CV_byYear_withError_", countryname2, ".pdf")) 
% @
 
%%%%%%%%%%%%%%%%%%%%%%%%%%% Plot7
\begin{knitrout}
\definecolor{shadecolor}{rgb}{0.969, 0.969, 0.969}\color{fgcolor}\begin{figure}[bht]

{\centering \includegraphics[width=.9\textwidth]{../Main/Figures/CV_byYearRegion_withError_Congo} 

}

\caption[Out-of-sample predictions along with direct estimates in the cross validation study where data from one region in each time period is held out and predicted using the rest of the data]{Out-of-sample predictions along with direct estimates in the cross validation study where data from one region in each time period is held out and predicted using the rest of the data.}\label{fig:unnamed-chunk-78}
\end{figure}


\end{knitrout}

%%%%%%%%%%%%%%%%%%%%%%%%%%% Plot8
\begin{knitrout}
\definecolor{shadecolor}{rgb}{0.969, 0.969, 0.969}\color{fgcolor}\begin{figure}[bht]

{\centering \includegraphics[width=.9\textwidth]{../Main/Figures/CVbiasCongo} 

}

\caption[Histogram and QQ-plot of the rescaled difference between the smoothed estimates and the direct estimates in the cross validation study]{Histogram and QQ-plot of the rescaled difference between the smoothed estimates and the direct estimates in the cross validation study. The differences between the two estimates are rescaled by the square root of the total variance of the two estimates.}\label{fig:unnamed-chunk-79}
\end{figure}


\end{knitrout}

%%%%%%%%%%%%%%%%%%%%%%%%%%% Plot9
\begin{knitrout}
\definecolor{shadecolor}{rgb}{0.969, 0.969, 0.969}\color{fgcolor}\begin{figure}[bht]

{\centering \includegraphics[width=.7\textwidth]{../Main/Figures/CVbiasbyRegionCongo} 

}

\caption[Line plot of the difference between smoothed estimates and the direct estimates in the cross validation study]{Line plot of the difference between smoothed estimates and the direct estimates in the cross validation study. The differences between the two estimates are rescaled by the square root of the total variance of the two estimates.}\label{fig:unnamed-chunk-80}
\end{figure}


\end{knitrout}

%%%%%%%%%%%%%%%%%%%%%%%%%%%%%%%%%%%%%%%%%%%%%%%%%%%%%%%%%%%%%%%%%%%%%%%%%%%%%%%%%%%%%%%%%%%%%%%%%%
\clearpage
\subsection{C\^{o}te d'Ivoire}


% \subsubsection{Summary of DHS surveys}

%%%%%%%%%%%%%%%%%%%%%%%%%%% Summary 


DHS surveys were conducted in C<U+00F4>te d'Ivoire in 2011.
% years.out[1:(length(years.out)-1)], and years.out[length(years.out)]. 

We fit both the RW2 only model to the combined national data, and compare the time trend at national level with the estimates produced by the UN and IHME in Figure~\ref{fig:unnamed-chunk-82}. We then adjusted the combined national data to the UN estimates of U5MR, and refit the models on the benchmarked data. 

%%%%%%%%%%%%%%%%%%%%%%%%%% Plot5 
\begin{knitrout}
\definecolor{shadecolor}{rgb}{0.969, 0.969, 0.969}\color{fgcolor}\begin{figure}[bht]

{\centering \includegraphics[width=.9\textwidth]{../Main/Figures/Yearly_national_Cote_dIvoire} 

}

\caption[Temporal national trends along with UN (B3) estimates described in You et al]{Temporal national trends along with UN (B3) estimates described in You et al. (2015) and IHME estimates based on GBD 2015 Child Mortality Collaborators (2016). RW2 represents the smoothed national estimates using the original data before benchmarking with UN estimates. RW2-adj represents the smoothed national estimates using the benchmarked data.}\label{fig:unnamed-chunk-82}
\end{figure}


\end{knitrout}
 

We fit the RW2 model to the benchmarked data in each area. 
% The proportions of the explained variation is summarized in Table~\ref{tab:paste0(countryname, "-var")}. 
We compare the results in Figure~\ref{fig:unnamed-chunk-83} to \ref{fig:unnamed-chunk-87}.
Figure~\ref{fig:unnamed-chunk-83} compares the smoothed estimates against the direct estimates. Figure~\ref{fig:unnamed-chunk-84} and Figure~\ref{fig:unnamed-chunk-85} show the posterior median estimates of U5MR in each region over time and the reductions from 1990 period respectively.
Figure~\ref{fig:unnamed-chunk-86} shows the smoothed estimates by region over time and Figure~\ref{fig:unnamed-chunk-87} compares the smoothed estimates with direct estimates from each survey for each region over time.


% %%%%%%%%%%%%%%%%%%%%%%%%%%% Table1 
% <<echo=FALSE, results='asis'>>=
% load("rda/variance_tables.rda")
% countryname2 <- gsub(" ", "", countryname)
% variance <- tables.all[[countryname]]

% table_count <- table_count + 1

% names <- c("RW2 ($\\sigma^2_{\\gamma_{t}}$)", "ICAR ($\\sigma^2_{\\phi_{i}}$)", "IID space ($\\sigma^2_{\\theta_{i}}$)", "IID time ($\\sigma^2_{\\alpha_{t}}$)", "IID space time ($\\sigma^2_{\\delta_{it}}$)")

% variance$Proportion <- round(variance$Proportion*100, digits = 2)
% row.names(variance) <- names
% tab <- xtable(variance, digits = c(1, 3, 2),align = "l|ll",
%        label = paste0("tab:", countryname, "-var"),
%        caption = paste(country, ": summary of the variance components in the RW2 model", sep = ''))
% print(tab, comment = FALSE,sanitize.text.function = function(x) {x})
% @

%%%%%%%%%%%%%%%%%%%%%%%%%%% Plot1 
\begin{knitrout}
\definecolor{shadecolor}{rgb}{0.969, 0.969, 0.969}\color{fgcolor}\begin{figure}[bht]

{\centering \includegraphics[width=.9\textwidth]{../Main/Figures/SmoothvDirectCote_dIvoire_meta} 

}

\caption[Smooth versus direct Admin 1 estimates]{Smooth versus direct Admin 1 estimates. Left: Combined (meta-analysis) survey estimate against combined direct estimates. Right: Combined (meta-analysis) survey estimate against direct estimates from each survey.}\label{fig:unnamed-chunk-83}
\end{figure}


\end{knitrout}

%%%%%%%%%%%%%%%%%%%%%%%%%%% Plot2 
\begin{knitrout}
\definecolor{shadecolor}{rgb}{0.969, 0.969, 0.969}\color{fgcolor}\begin{figure}[bht]

{\centering \includegraphics[width=.9\textwidth]{../Main/Figures/SmoothMedianCote_dIvoire} 

}

\caption[Maps of posterior medians for C<c3><b4>te d'Ivoire  over time]{Maps of posterior medians for C<c3><b4>te d'Ivoire  over time.}\label{fig:unnamed-chunk-84}
\end{figure}


\end{knitrout}
%%%%%%%%%%%%%%%%%%%%%%%%%%% Plot2a
\begin{knitrout}
\definecolor{shadecolor}{rgb}{0.969, 0.969, 0.969}\color{fgcolor}\begin{figure}[bht]

{\centering \includegraphics[width=.9\textwidth]{../Main/Figures/ReductionMedianCote_dIvoire} 

}

\caption[Maps of reduction of posterior median U5MR in each five-year period compared to 1990 in C<c3><b4>te d'Ivoire over time]{Maps of reduction of posterior median U5MR in each five-year period compared to 1990 in C<c3><b4>te d'Ivoire over time.}\label{fig:unnamed-chunk-85}
\end{figure}


\end{knitrout}
%%%%%%%%%%%%%%%%%%%%%%%%%%% Plot3 
\begin{knitrout}
\definecolor{shadecolor}{rgb}{0.969, 0.969, 0.969}\color{fgcolor}\begin{figure}[bht]

{\centering \includegraphics[width=.95\textwidth]{../Main/Figures/Yearly_v_Periods_Cote_dIvoire} 

}

\caption[Smoothed regional estimates over time]{Smoothed regional estimates over time. The line indicates yearly posterior median estimates and error bars indicate 95 \% posterior credible interval at each time period.}\label{fig:unnamed-chunk-86}
\end{figure}


\end{knitrout}

%%%%%%%%%%%%%%%%%%%%%%%%%%% Plot4 
\begin{knitrout}
\definecolor{shadecolor}{rgb}{0.969, 0.969, 0.969}\color{fgcolor}\begin{figure}[bht]

{\centering \includegraphics[width=.9\textwidth]{../Main/Figures/LineSubMedianCote_dIvoire} 

}

\caption[Smoothed regional estimates over time compared to the direct estimates from each surveys]{Smoothed regional estimates over time compared to the direct estimates from each surveys. Direct estimates are not benchmarked with UN estimates. The line indicates posterior median and error bars indicate 95\% posterior credible interval.}\label{fig:unnamed-chunk-87}
\end{figure}


\end{knitrout}
% \subsubsection{National model results}
We further assess the RW2 model by holding out some observations, and compare the projections to the direct estimates in these holdout observations. Figure~\ref{fig:unnamed-chunk-88} compares the predicted estimates for the out-of-sample observations  with the direct estimates by holding out observations from each area in each time period.  Figure~\ref{fig:unnamed-chunk-89} compares the histogram of the bias rescaled by the total variance in the cross validation studies. Figure~\ref{fig:unnamed-chunk-90} compares the rescaled bias by region and time periods.



% %%%%%%%%%%%%%%%%%%%%%%%%%%% Plot6
% << echo=FALSE, out.width = ".9\\textwidth", fig.width = 12, fig.height = 6, fig.cap = "Out-of-sample predictions along with direct estimates in the cross validation study where all data from each time period is held out and predicted using the rest of the data.">>=
% fig_count <- fig_count + 1
% knitr::include_graphics(paste0("../Main/Figures/CV_byYear_withError_", countryname2, ".pdf")) 
% @
 
%%%%%%%%%%%%%%%%%%%%%%%%%%% Plot7
\begin{knitrout}
\definecolor{shadecolor}{rgb}{0.969, 0.969, 0.969}\color{fgcolor}\begin{figure}[bht]

{\centering \includegraphics[width=.9\textwidth]{../Main/Figures/CV_byYearRegion_withError_Cote_dIvoire} 

}

\caption[Out-of-sample predictions along with direct estimates in the cross validation study where data from one region in each time period is held out and predicted using the rest of the data]{Out-of-sample predictions along with direct estimates in the cross validation study where data from one region in each time period is held out and predicted using the rest of the data.}\label{fig:unnamed-chunk-88}
\end{figure}


\end{knitrout}

%%%%%%%%%%%%%%%%%%%%%%%%%%% Plot8
\begin{knitrout}
\definecolor{shadecolor}{rgb}{0.969, 0.969, 0.969}\color{fgcolor}\begin{figure}[bht]

{\centering \includegraphics[width=.9\textwidth]{../Main/Figures/CVbiasCote_dIvoire} 

}

\caption[Histogram and QQ-plot of the rescaled difference between the smoothed estimates and the direct estimates in the cross validation study]{Histogram and QQ-plot of the rescaled difference between the smoothed estimates and the direct estimates in the cross validation study. The differences between the two estimates are rescaled by the square root of the total variance of the two estimates.}\label{fig:unnamed-chunk-89}
\end{figure}


\end{knitrout}

%%%%%%%%%%%%%%%%%%%%%%%%%%% Plot9
\begin{knitrout}
\definecolor{shadecolor}{rgb}{0.969, 0.969, 0.969}\color{fgcolor}\begin{figure}[bht]

{\centering \includegraphics[width=.7\textwidth]{../Main/Figures/CVbiasbyRegionCote_dIvoire} 

}

\caption[Line plot of the difference between smoothed estimates and the direct estimates in the cross validation study]{Line plot of the difference between smoothed estimates and the direct estimates in the cross validation study. The differences between the two estimates are rescaled by the square root of the total variance of the two estimates.}\label{fig:unnamed-chunk-90}
\end{figure}


\end{knitrout}



%%%%%%%%%%%%%%%%%%%%%%%%%%%%%%%%%%%%%%%%%%%%%%%%%%%%%%%%%%%%%%%%%%%%%%%%%%%%%%%%%%%%%%%%%%%%%%%%%%
\clearpage
\subsection{DRC}


% \subsubsection{Summary of DHS surveys}

%%%%%%%%%%%%%%%%%%%%%%%%%%% Summary 


DHS surveys were conducted in DRC in 2007, and 2014.
% years.out[1:(length(years.out)-1)], and years.out[length(years.out)]. 

We fit both the RW2 only model to the combined national data, and compare the time trend at national level with the estimates produced by the UN and IHME in Figure~\ref{fig:unnamed-chunk-92}. We then adjusted the combined national data to the UN estimates of U5MR, and refit the models on the benchmarked data. 

%%%%%%%%%%%%%%%%%%%%%%%%%% Plot5 
\begin{knitrout}
\definecolor{shadecolor}{rgb}{0.969, 0.969, 0.969}\color{fgcolor}\begin{figure}[bht]

{\centering \includegraphics[width=.9\textwidth]{../Main/Figures/Yearly_national_DRC} 

}

\caption[Temporal national trends along with UN (B3) estimates described in You et al]{Temporal national trends along with UN (B3) estimates described in You et al. (2015) and IHME estimates based on GBD 2015 Child Mortality Collaborators (2016). RW2 represents the smoothed national estimates using the original data before benchmarking with UN estimates. RW2-adj represents the smoothed national estimates using the benchmarked data.}\label{fig:unnamed-chunk-92}
\end{figure}


\end{knitrout}
 

We fit the RW2 model to the benchmarked data in each area. 
% The proportions of the explained variation is summarized in Table~\ref{tab:paste0(countryname, "-var")}. 
We compare the results in Figure~\ref{fig:unnamed-chunk-93} to \ref{fig:unnamed-chunk-97}.
Figure~\ref{fig:unnamed-chunk-93} compares the smoothed estimates against the direct estimates. Figure~\ref{fig:unnamed-chunk-94} and Figure~\ref{fig:unnamed-chunk-95} show the posterior median estimates of U5MR in each region over time and the reductions from 1990 period respectively.
Figure~\ref{fig:unnamed-chunk-96} shows the smoothed estimates by region over time and Figure~\ref{fig:unnamed-chunk-97} compares the smoothed estimates with direct estimates from each survey for each region over time.


% %%%%%%%%%%%%%%%%%%%%%%%%%%% Table1 
% <<echo=FALSE, results='asis'>>=
% load("rda/variance_tables.rda")
% countryname2 <- gsub(" ", "", countryname)
% variance <- tables.all[[countryname]]

% table_count <- table_count + 1

% names <- c("RW2 ($\\sigma^2_{\\gamma_{t}}$)", "ICAR ($\\sigma^2_{\\phi_{i}}$)", "IID space ($\\sigma^2_{\\theta_{i}}$)", "IID time ($\\sigma^2_{\\alpha_{t}}$)", "IID space time ($\\sigma^2_{\\delta_{it}}$)")

% variance$Proportion <- round(variance$Proportion*100, digits = 2)
% row.names(variance) <- names
% tab <- xtable(variance, digits = c(1, 3, 2),align = "l|ll",
%        label = paste0("tab:", countryname, "-var"),
%        caption = paste(country, ": summary of the variance components in the RW2 model", sep = ''))
% print(tab, comment = FALSE,sanitize.text.function = function(x) {x})
% @

%%%%%%%%%%%%%%%%%%%%%%%%%%% Plot1 
\begin{knitrout}
\definecolor{shadecolor}{rgb}{0.969, 0.969, 0.969}\color{fgcolor}\begin{figure}[bht]

{\centering \includegraphics[width=.9\textwidth]{../Main/Figures/SmoothvDirectDRC_meta} 

}

\caption[Smooth versus direct Admin 1 estimates]{Smooth versus direct Admin 1 estimates. Left: Combined (meta-analysis) survey estimate against combined direct estimates. Right: Combined (meta-analysis) survey estimate against direct estimates from each survey.}\label{fig:unnamed-chunk-93}
\end{figure}


\end{knitrout}

%%%%%%%%%%%%%%%%%%%%%%%%%%% Plot2 
\begin{knitrout}
\definecolor{shadecolor}{rgb}{0.969, 0.969, 0.969}\color{fgcolor}\begin{figure}[bht]

{\centering \includegraphics[width=.9\textwidth]{../Main/Figures/SmoothMedianDRC} 

}

\caption[Maps of posterior medians for DRC  over time]{Maps of posterior medians for DRC  over time.}\label{fig:unnamed-chunk-94}
\end{figure}


\end{knitrout}
%%%%%%%%%%%%%%%%%%%%%%%%%%% Plot2a
\begin{knitrout}
\definecolor{shadecolor}{rgb}{0.969, 0.969, 0.969}\color{fgcolor}\begin{figure}[bht]

{\centering \includegraphics[width=.9\textwidth]{../Main/Figures/ReductionMedianDRC} 

}

\caption[Maps of reduction of posterior median U5MR in each five-year period compared to 1990 in DRC over time]{Maps of reduction of posterior median U5MR in each five-year period compared to 1990 in DRC over time.}\label{fig:unnamed-chunk-95}
\end{figure}


\end{knitrout}
%%%%%%%%%%%%%%%%%%%%%%%%%%% Plot3 
\begin{knitrout}
\definecolor{shadecolor}{rgb}{0.969, 0.969, 0.969}\color{fgcolor}\begin{figure}[bht]

{\centering \includegraphics[width=.95\textwidth]{../Main/Figures/Yearly_v_Periods_DRC} 

}

\caption[Smoothed regional estimates over time]{Smoothed regional estimates over time. The line indicates yearly posterior median estimates and error bars indicate 95 \% posterior credible interval at each time period.}\label{fig:unnamed-chunk-96}
\end{figure}


\end{knitrout}

%%%%%%%%%%%%%%%%%%%%%%%%%%% Plot4 
\begin{knitrout}
\definecolor{shadecolor}{rgb}{0.969, 0.969, 0.969}\color{fgcolor}\begin{figure}[bht]

{\centering \includegraphics[width=.9\textwidth]{../Main/Figures/LineSubMedianDRC} 

}

\caption[Smoothed regional estimates over time compared to the direct estimates from each surveys]{Smoothed regional estimates over time compared to the direct estimates from each surveys. Direct estimates are not benchmarked with UN estimates. The line indicates posterior median and error bars indicate 95\% posterior credible interval.}\label{fig:unnamed-chunk-97}
\end{figure}


\end{knitrout}
% \subsubsection{National model results}
We further assess the RW2 model by holding out some observations, and compare the projections to the direct estimates in these holdout observations. Figure~\ref{fig:unnamed-chunk-98} compares the predicted estimates for the out-of-sample observations  with the direct estimates by holding out observations from each area in each time period.  Figure~\ref{fig:unnamed-chunk-99} compares the histogram of the bias rescaled by the total variance in the cross validation studies. Figure~\ref{fig:unnamed-chunk-100} compares the rescaled bias by region and time periods.



% %%%%%%%%%%%%%%%%%%%%%%%%%%% Plot6
% << echo=FALSE, out.width = ".9\\textwidth", fig.width = 12, fig.height = 6, fig.cap = "Out-of-sample predictions along with direct estimates in the cross validation study where all data from each time period is held out and predicted using the rest of the data.">>=
% fig_count <- fig_count + 1
% knitr::include_graphics(paste0("../Main/Figures/CV_byYear_withError_", countryname2, ".pdf")) 
% @
 
%%%%%%%%%%%%%%%%%%%%%%%%%%% Plot7
\begin{knitrout}
\definecolor{shadecolor}{rgb}{0.969, 0.969, 0.969}\color{fgcolor}\begin{figure}[bht]

{\centering \includegraphics[width=.9\textwidth]{../Main/Figures/CV_byYearRegion_withError_DRC} 

}

\caption[Out-of-sample predictions along with direct estimates in the cross validation study where data from one region in each time period is held out and predicted using the rest of the data]{Out-of-sample predictions along with direct estimates in the cross validation study where data from one region in each time period is held out and predicted using the rest of the data.}\label{fig:unnamed-chunk-98}
\end{figure}


\end{knitrout}

%%%%%%%%%%%%%%%%%%%%%%%%%%% Plot8
\begin{knitrout}
\definecolor{shadecolor}{rgb}{0.969, 0.969, 0.969}\color{fgcolor}\begin{figure}[bht]

{\centering \includegraphics[width=.9\textwidth]{../Main/Figures/CVbiasDRC} 

}

\caption[Histogram and QQ-plot of the rescaled difference between the smoothed estimates and the direct estimates in the cross validation study]{Histogram and QQ-plot of the rescaled difference between the smoothed estimates and the direct estimates in the cross validation study. The differences between the two estimates are rescaled by the square root of the total variance of the two estimates.}\label{fig:unnamed-chunk-99}
\end{figure}


\end{knitrout}

%%%%%%%%%%%%%%%%%%%%%%%%%%% Plot9
\begin{knitrout}
\definecolor{shadecolor}{rgb}{0.969, 0.969, 0.969}\color{fgcolor}\begin{figure}[bht]

{\centering \includegraphics[width=.7\textwidth]{../Main/Figures/CVbiasbyRegionDRC} 

}

\caption[Line plot of the difference between smoothed estimates and the direct estimates in the cross validation study]{Line plot of the difference between smoothed estimates and the direct estimates in the cross validation study. The differences between the two estimates are rescaled by the square root of the total variance of the two estimates.}\label{fig:unnamed-chunk-100}
\end{figure}


\end{knitrout}


%%%%%%%%%%%%%%%%%%%%%%%%%%%%%%%%%%%%%%%%%%%%%%%%%%%%%%%%%%%%%%%%%%%%%%%%%%%%%%%%%%%%%%%%%%%%%%%%%%
\clearpage
\subsection{Egypt}


% \subsubsection{Summary of DHS surveys}

%%%%%%%%%%%%%%%%%%%%%%%%%%% Summary 


DHS surveys were conducted in Egypt in 1988, 1992, 1995, 2000, 2003, 2005, 2008, and 2014.
% years.out[1:(length(years.out)-1)], and years.out[length(years.out)]. 

We fit both the RW2 only model to the combined national data, and compare the time trend at national level with the estimates produced by the UN and IHME in Figure~\ref{fig:unnamed-chunk-102}. We then adjusted the combined national data to the UN estimates of U5MR, and refit the models on the benchmarked data. 

%%%%%%%%%%%%%%%%%%%%%%%%%% Plot5 
\begin{knitrout}
\definecolor{shadecolor}{rgb}{0.969, 0.969, 0.969}\color{fgcolor}\begin{figure}[bht]

{\centering \includegraphics[width=.9\textwidth]{../Main/Figures/Yearly_national_Egypt} 

}

\caption[Temporal national trends along with UN (B3) estimates described in You et al]{Temporal national trends along with UN (B3) estimates described in You et al. (2015) and IHME estimates based on GBD 2015 Child Mortality Collaborators (2016). RW2 represents the smoothed national estimates using the original data before benchmarking with UN estimates. RW2-adj represents the smoothed national estimates using the benchmarked data.}\label{fig:unnamed-chunk-102}
\end{figure}


\end{knitrout}
 

We fit the RW2 model to the benchmarked data in each area. 
% The proportions of the explained variation is summarized in Table~\ref{tab:paste0(countryname, "-var")}. 
We compare the results in Figure~\ref{fig:unnamed-chunk-103} to \ref{fig:unnamed-chunk-107}.
Figure~\ref{fig:unnamed-chunk-103} compares the smoothed estimates against the direct estimates. Figure~\ref{fig:unnamed-chunk-104} and Figure~\ref{fig:unnamed-chunk-105} show the posterior median estimates of U5MR in each region over time and the reductions from 1990 period respectively.
Figure~\ref{fig:unnamed-chunk-106} shows the smoothed estimates by region over time and Figure~\ref{fig:unnamed-chunk-107} compares the smoothed estimates with direct estimates from each survey for each region over time.


% %%%%%%%%%%%%%%%%%%%%%%%%%%% Table1 
% <<echo=FALSE, results='asis'>>=
% load("rda/variance_tables.rda")
% countryname2 <- gsub(" ", "", countryname)
% variance <- tables.all[[countryname]]

% table_count <- table_count + 1

% names <- c("RW2 ($\\sigma^2_{\\gamma_{t}}$)", "ICAR ($\\sigma^2_{\\phi_{i}}$)", "IID space ($\\sigma^2_{\\theta_{i}}$)", "IID time ($\\sigma^2_{\\alpha_{t}}$)", "IID space time ($\\sigma^2_{\\delta_{it}}$)")

% variance$Proportion <- round(variance$Proportion*100, digits = 2)
% row.names(variance) <- names
% tab <- xtable(variance, digits = c(1, 3, 2),align = "l|ll",
%        label = paste0("tab:", countryname, "-var"),
%        caption = paste(country, ": summary of the variance components in the RW2 model", sep = ''))
% print(tab, comment = FALSE,sanitize.text.function = function(x) {x})
% @

%%%%%%%%%%%%%%%%%%%%%%%%%%% Plot1 
\begin{knitrout}
\definecolor{shadecolor}{rgb}{0.969, 0.969, 0.969}\color{fgcolor}\begin{figure}[bht]

{\centering \includegraphics[width=.9\textwidth]{../Main/Figures/SmoothvDirectEgypt_meta} 

}

\caption[Smooth versus direct Admin 1 estimates]{Smooth versus direct Admin 1 estimates. Left: Combined (meta-analysis) survey estimate against combined direct estimates. Right: Combined (meta-analysis) survey estimate against direct estimates from each survey.}\label{fig:unnamed-chunk-103}
\end{figure}


\end{knitrout}

%%%%%%%%%%%%%%%%%%%%%%%%%%% Plot2 
\begin{knitrout}
\definecolor{shadecolor}{rgb}{0.969, 0.969, 0.969}\color{fgcolor}\begin{figure}[bht]

{\centering \includegraphics[width=.9\textwidth]{../Main/Figures/SmoothMedianEgypt} 

}

\caption[Maps of posterior medians for Egypt  over time]{Maps of posterior medians for Egypt  over time.}\label{fig:unnamed-chunk-104}
\end{figure}


\end{knitrout}
%%%%%%%%%%%%%%%%%%%%%%%%%%% Plot2a
\begin{knitrout}
\definecolor{shadecolor}{rgb}{0.969, 0.969, 0.969}\color{fgcolor}\begin{figure}[bht]

{\centering \includegraphics[width=.9\textwidth]{../Main/Figures/ReductionMedianEgypt} 

}

\caption[Maps of reduction of posterior median U5MR in each five-year period compared to 1990 in Egypt over time]{Maps of reduction of posterior median U5MR in each five-year period compared to 1990 in Egypt over time.}\label{fig:unnamed-chunk-105}
\end{figure}


\end{knitrout}
%%%%%%%%%%%%%%%%%%%%%%%%%%% Plot3 
\begin{knitrout}
\definecolor{shadecolor}{rgb}{0.969, 0.969, 0.969}\color{fgcolor}\begin{figure}[bht]

{\centering \includegraphics[width=.95\textwidth]{../Main/Figures/Yearly_v_Periods_Egypt} 

}

\caption[Smoothed regional estimates over time]{Smoothed regional estimates over time. The line indicates yearly posterior median estimates and error bars indicate 95 \% posterior credible interval at each time period.}\label{fig:unnamed-chunk-106}
\end{figure}


\end{knitrout}

%%%%%%%%%%%%%%%%%%%%%%%%%%% Plot4 
\begin{knitrout}
\definecolor{shadecolor}{rgb}{0.969, 0.969, 0.969}\color{fgcolor}\begin{figure}[bht]

{\centering \includegraphics[width=.9\textwidth]{../Main/Figures/LineSubMedianEgypt} 

}

\caption[Smoothed regional estimates over time compared to the direct estimates from each surveys]{Smoothed regional estimates over time compared to the direct estimates from each surveys. Direct estimates are not benchmarked with UN estimates. The line indicates posterior median and error bars indicate 95\% posterior credible interval.}\label{fig:unnamed-chunk-107}
\end{figure}


\end{knitrout}
% \subsubsection{National model results}
We further assess the RW2 model by holding out some observations, and compare the projections to the direct estimates in these holdout observations. Figure~\ref{fig:unnamed-chunk-108} compares the predicted estimates for the out-of-sample observations  with the direct estimates by holding out observations from each area in each time period.  Figure~\ref{fig:unnamed-chunk-109} compares the histogram of the bias rescaled by the total variance in the cross validation studies. Figure~\ref{fig:unnamed-chunk-110} compares the rescaled bias by region and time periods.



% %%%%%%%%%%%%%%%%%%%%%%%%%%% Plot6
% << echo=FALSE, out.width = ".9\\textwidth", fig.width = 12, fig.height = 6, fig.cap = "Out-of-sample predictions along with direct estimates in the cross validation study where all data from each time period is held out and predicted using the rest of the data.">>=
% fig_count <- fig_count + 1
% knitr::include_graphics(paste0("../Main/Figures/CV_byYear_withError_", countryname2, ".pdf")) 
% @
 
%%%%%%%%%%%%%%%%%%%%%%%%%%% Plot7
\begin{knitrout}
\definecolor{shadecolor}{rgb}{0.969, 0.969, 0.969}\color{fgcolor}\begin{figure}[bht]

{\centering \includegraphics[width=.9\textwidth]{../Main/Figures/CV_byYearRegion_withError_Egypt} 

}

\caption[Out-of-sample predictions along with direct estimates in the cross validation study where data from one region in each time period is held out and predicted using the rest of the data]{Out-of-sample predictions along with direct estimates in the cross validation study where data from one region in each time period is held out and predicted using the rest of the data.}\label{fig:unnamed-chunk-108}
\end{figure}


\end{knitrout}

%%%%%%%%%%%%%%%%%%%%%%%%%%% Plot8
\begin{knitrout}
\definecolor{shadecolor}{rgb}{0.969, 0.969, 0.969}\color{fgcolor}\begin{figure}[bht]

{\centering \includegraphics[width=.9\textwidth]{../Main/Figures/CVbiasEgypt} 

}

\caption[Histogram and QQ-plot of the rescaled difference between the smoothed estimates and the direct estimates in the cross validation study]{Histogram and QQ-plot of the rescaled difference between the smoothed estimates and the direct estimates in the cross validation study. The differences between the two estimates are rescaled by the square root of the total variance of the two estimates.}\label{fig:unnamed-chunk-109}
\end{figure}


\end{knitrout}

%%%%%%%%%%%%%%%%%%%%%%%%%%% Plot9
\begin{knitrout}
\definecolor{shadecolor}{rgb}{0.969, 0.969, 0.969}\color{fgcolor}\begin{figure}[bht]

{\centering \includegraphics[width=.7\textwidth]{../Main/Figures/CVbiasbyRegionEgypt} 

}

\caption[Line plot of the difference between smoothed estimates and the direct estimates in the cross validation study]{Line plot of the difference between smoothed estimates and the direct estimates in the cross validation study. The differences between the two estimates are rescaled by the square root of the total variance of the two estimates.}\label{fig:unnamed-chunk-110}
\end{figure}


\end{knitrout}


%%%%%%%%%%%%%%%%%%%%%%%%%%%%%%%%%%%%%%%%%%%%%%%%%%%%%%%%%%%%%%%%%%%%%%%%%%%%%%%%%%%%%%%%%%%%%%%%%%
\clearpage
\subsection{Ethiopia}


% \subsubsection{Summary of DHS surveys}

%%%%%%%%%%%%%%%%%%%%%%%%%%% Summary 


DHS surveys were conducted in Ethiopia in 2000, 2005, 2011, and 2016.
% years.out[1:(length(years.out)-1)], and years.out[length(years.out)]. 

We fit both the RW2 only model to the combined national data, and compare the time trend at national level with the estimates produced by the UN and IHME in Figure~\ref{fig:unnamed-chunk-112}. We then adjusted the combined national data to the UN estimates of U5MR, and refit the models on the benchmarked data. 

%%%%%%%%%%%%%%%%%%%%%%%%%% Plot5 
\begin{knitrout}
\definecolor{shadecolor}{rgb}{0.969, 0.969, 0.969}\color{fgcolor}\begin{figure}[bht]

{\centering \includegraphics[width=.9\textwidth]{../Main/Figures/Yearly_national_Ethiopia} 

}

\caption[Temporal national trends along with UN (B3) estimates described in You et al]{Temporal national trends along with UN (B3) estimates described in You et al. (2015) and IHME estimates based on GBD 2015 Child Mortality Collaborators (2016). RW2 represents the smoothed national estimates using the original data before benchmarking with UN estimates. RW2-adj represents the smoothed national estimates using the benchmarked data.}\label{fig:unnamed-chunk-112}
\end{figure}


\end{knitrout}
 

We fit the RW2 model to the benchmarked data in each area. 
% The proportions of the explained variation is summarized in Table~\ref{tab:paste0(countryname, "-var")}. 
We compare the results in Figure~\ref{fig:unnamed-chunk-113} to \ref{fig:unnamed-chunk-117}.
Figure~\ref{fig:unnamed-chunk-113} compares the smoothed estimates against the direct estimates. Figure~\ref{fig:unnamed-chunk-114} and Figure~\ref{fig:unnamed-chunk-115} show the posterior median estimates of U5MR in each region over time and the reductions from 1990 period respectively.
Figure~\ref{fig:unnamed-chunk-116} shows the smoothed estimates by region over time and Figure~\ref{fig:unnamed-chunk-117} compares the smoothed estimates with direct estimates from each survey for each region over time.


% %%%%%%%%%%%%%%%%%%%%%%%%%%% Table1 
% <<echo=FALSE, results='asis'>>=
% load("rda/variance_tables.rda")
% countryname2 <- gsub(" ", "", countryname)
% variance <- tables.all[[countryname]]

% table_count <- table_count + 1

% names <- c("RW2 ($\\sigma^2_{\\gamma_{t}}$)", "ICAR ($\\sigma^2_{\\phi_{i}}$)", "IID space ($\\sigma^2_{\\theta_{i}}$)", "IID time ($\\sigma^2_{\\alpha_{t}}$)", "IID space time ($\\sigma^2_{\\delta_{it}}$)")

% variance$Proportion <- round(variance$Proportion*100, digits = 2)
% row.names(variance) <- names
% tab <- xtable(variance, digits = c(1, 3, 2),align = "l|ll",
%        label = paste0("tab:", countryname, "-var"),
%        caption = paste(country, ": summary of the variance components in the RW2 model", sep = ''))
% print(tab, comment = FALSE,sanitize.text.function = function(x) {x})
% @

%%%%%%%%%%%%%%%%%%%%%%%%%%% Plot1 
\begin{knitrout}
\definecolor{shadecolor}{rgb}{0.969, 0.969, 0.969}\color{fgcolor}\begin{figure}[bht]

{\centering \includegraphics[width=.9\textwidth]{../Main/Figures/SmoothvDirectEthiopia_meta} 

}

\caption[Smooth versus direct Admin 1 estimates]{Smooth versus direct Admin 1 estimates. Left: Combined (meta-analysis) survey estimate against combined direct estimates. Right: Combined (meta-analysis) survey estimate against direct estimates from each survey.}\label{fig:unnamed-chunk-113}
\end{figure}


\end{knitrout}

%%%%%%%%%%%%%%%%%%%%%%%%%%% Plot2 
\begin{knitrout}
\definecolor{shadecolor}{rgb}{0.969, 0.969, 0.969}\color{fgcolor}\begin{figure}[bht]

{\centering \includegraphics[width=.9\textwidth]{../Main/Figures/SmoothMedianEthiopia} 

}

\caption[Maps of posterior medians for Ethiopia  over time]{Maps of posterior medians for Ethiopia  over time.}\label{fig:unnamed-chunk-114}
\end{figure}


\end{knitrout}
%%%%%%%%%%%%%%%%%%%%%%%%%%% Plot2a
\begin{knitrout}
\definecolor{shadecolor}{rgb}{0.969, 0.969, 0.969}\color{fgcolor}\begin{figure}[bht]

{\centering \includegraphics[width=.9\textwidth]{../Main/Figures/ReductionMedianEthiopia} 

}

\caption[Maps of reduction of posterior median U5MR in each five-year period compared to 1990 in Ethiopia over time]{Maps of reduction of posterior median U5MR in each five-year period compared to 1990 in Ethiopia over time.}\label{fig:unnamed-chunk-115}
\end{figure}


\end{knitrout}
%%%%%%%%%%%%%%%%%%%%%%%%%%% Plot3 
\begin{knitrout}
\definecolor{shadecolor}{rgb}{0.969, 0.969, 0.969}\color{fgcolor}\begin{figure}[bht]

{\centering \includegraphics[width=.95\textwidth]{../Main/Figures/Yearly_v_Periods_Ethiopia} 

}

\caption[Smoothed regional estimates over time]{Smoothed regional estimates over time. The line indicates yearly posterior median estimates and error bars indicate 95 \% posterior credible interval at each time period.}\label{fig:unnamed-chunk-116}
\end{figure}


\end{knitrout}

%%%%%%%%%%%%%%%%%%%%%%%%%%% Plot4 
\begin{knitrout}
\definecolor{shadecolor}{rgb}{0.969, 0.969, 0.969}\color{fgcolor}\begin{figure}[bht]

{\centering \includegraphics[width=.9\textwidth]{../Main/Figures/LineSubMedianEthiopia} 

}

\caption[Smoothed regional estimates over time compared to the direct estimates from each surveys]{Smoothed regional estimates over time compared to the direct estimates from each surveys. Direct estimates are not benchmarked with UN estimates. The line indicates posterior median and error bars indicate 95\% posterior credible interval.}\label{fig:unnamed-chunk-117}
\end{figure}


\end{knitrout}
% \subsubsection{National model results}
We further assess the RW2 model by holding out some observations, and compare the projections to the direct estimates in these holdout observations. Figure~\ref{fig:unnamed-chunk-118} compares the predicted estimates for the out-of-sample observations  with the direct estimates by holding out observations from each area in each time period.  Figure~\ref{fig:unnamed-chunk-119} compares the histogram of the bias rescaled by the total variance in the cross validation studies. Figure~\ref{fig:unnamed-chunk-120} compares the rescaled bias by region and time periods.



% %%%%%%%%%%%%%%%%%%%%%%%%%%% Plot6
% << echo=FALSE, out.width = ".9\\textwidth", fig.width = 12, fig.height = 6, fig.cap = "Out-of-sample predictions along with direct estimates in the cross validation study where all data from each time period is held out and predicted using the rest of the data.">>=
% fig_count <- fig_count + 1
% knitr::include_graphics(paste0("../Main/Figures/CV_byYear_withError_", countryname2, ".pdf")) 
% @
 
%%%%%%%%%%%%%%%%%%%%%%%%%%% Plot7
\begin{knitrout}
\definecolor{shadecolor}{rgb}{0.969, 0.969, 0.969}\color{fgcolor}\begin{figure}[bht]

{\centering \includegraphics[width=.9\textwidth]{../Main/Figures/CV_byYearRegion_withError_Ethiopia} 

}

\caption[Out-of-sample predictions along with direct estimates in the cross validation study where data from one region in each time period is held out and predicted using the rest of the data]{Out-of-sample predictions along with direct estimates in the cross validation study where data from one region in each time period is held out and predicted using the rest of the data.}\label{fig:unnamed-chunk-118}
\end{figure}


\end{knitrout}

%%%%%%%%%%%%%%%%%%%%%%%%%%% Plot8
\begin{knitrout}
\definecolor{shadecolor}{rgb}{0.969, 0.969, 0.969}\color{fgcolor}\begin{figure}[bht]

{\centering \includegraphics[width=.9\textwidth]{../Main/Figures/CVbiasEthiopia} 

}

\caption[Histogram and QQ-plot of the rescaled difference between the smoothed estimates and the direct estimates in the cross validation study]{Histogram and QQ-plot of the rescaled difference between the smoothed estimates and the direct estimates in the cross validation study. The differences between the two estimates are rescaled by the square root of the total variance of the two estimates.}\label{fig:unnamed-chunk-119}
\end{figure}


\end{knitrout}

%%%%%%%%%%%%%%%%%%%%%%%%%%% Plot9
\begin{knitrout}
\definecolor{shadecolor}{rgb}{0.969, 0.969, 0.969}\color{fgcolor}\begin{figure}[bht]

{\centering \includegraphics[width=.7\textwidth]{../Main/Figures/CVbiasbyRegionEthiopia} 

}

\caption[Line plot of the difference between smoothed estimates and the direct estimates in the cross validation study]{Line plot of the difference between smoothed estimates and the direct estimates in the cross validation study. The differences between the two estimates are rescaled by the square root of the total variance of the two estimates.}\label{fig:unnamed-chunk-120}
\end{figure}


\end{knitrout}


%%%%%%%%%%%%%%%%%%%%%%%%%%%%%%%%%%%%%%%%%%%%%%%%%%%%%%%%%%%%%%%%%%%%%%%%%%%%%%%%%%%%%%%%%%%%%%%%%%
\clearpage
\subsection{Gabon}


% \subsubsection{Summary of DHS surveys}

%%%%%%%%%%%%%%%%%%%%%%%%%%% Summary 


DHS surveys were conducted in Gabon in 2000, and 2012.
% years.out[1:(length(years.out)-1)], and years.out[length(years.out)]. 

We fit both the RW2 only model to the combined national data, and compare the time trend at national level with the estimates produced by the UN and IHME in Figure~\ref{fig:unnamed-chunk-122}. We then adjusted the combined national data to the UN estimates of U5MR, and refit the models on the benchmarked data. 

%%%%%%%%%%%%%%%%%%%%%%%%%% Plot5 
\begin{knitrout}
\definecolor{shadecolor}{rgb}{0.969, 0.969, 0.969}\color{fgcolor}\begin{figure}[bht]

{\centering \includegraphics[width=.9\textwidth]{../Main/Figures/Yearly_national_Gabon} 

}

\caption[Temporal national trends along with UN (B3) estimates described in You et al]{Temporal national trends along with UN (B3) estimates described in You et al. (2015) and IHME estimates based on GBD 2015 Child Mortality Collaborators (2016). RW2 represents the smoothed national estimates using the original data before benchmarking with UN estimates. RW2-adj represents the smoothed national estimates using the benchmarked data.}\label{fig:unnamed-chunk-122}
\end{figure}


\end{knitrout}
 

We fit the RW2 model to the benchmarked data in each area. 
% The proportions of the explained variation is summarized in Table~\ref{tab:paste0(countryname, "-var")}. 
We compare the results in Figure~\ref{fig:unnamed-chunk-123} to \ref{fig:unnamed-chunk-127}.
Figure~\ref{fig:unnamed-chunk-123} compares the smoothed estimates against the direct estimates. Figure~\ref{fig:unnamed-chunk-124} and Figure~\ref{fig:unnamed-chunk-125} show the posterior median estimates of U5MR in each region over time and the reductions from 1990 period respectively.
Figure~\ref{fig:unnamed-chunk-126} shows the smoothed estimates by region over time and Figure~\ref{fig:unnamed-chunk-127} compares the smoothed estimates with direct estimates from each survey for each region over time.


% %%%%%%%%%%%%%%%%%%%%%%%%%%% Table1 
% <<echo=FALSE, results='asis'>>=
% load("rda/variance_tables.rda")
% countryname2 <- gsub(" ", "", countryname)
% variance <- tables.all[[countryname]]

% table_count <- table_count + 1

% names <- c("RW2 ($\\sigma^2_{\\gamma_{t}}$)", "ICAR ($\\sigma^2_{\\phi_{i}}$)", "IID space ($\\sigma^2_{\\theta_{i}}$)", "IID time ($\\sigma^2_{\\alpha_{t}}$)", "IID space time ($\\sigma^2_{\\delta_{it}}$)")

% variance$Proportion <- round(variance$Proportion*100, digits = 2)
% row.names(variance) <- names
% tab <- xtable(variance, digits = c(1, 3, 2),align = "l|ll",
%        label = paste0("tab:", countryname, "-var"),
%        caption = paste(country, ": summary of the variance components in the RW2 model", sep = ''))
% print(tab, comment = FALSE,sanitize.text.function = function(x) {x})
% @

%%%%%%%%%%%%%%%%%%%%%%%%%%% Plot1 
\begin{knitrout}
\definecolor{shadecolor}{rgb}{0.969, 0.969, 0.969}\color{fgcolor}\begin{figure}[bht]

{\centering \includegraphics[width=.9\textwidth]{../Main/Figures/SmoothvDirectGabon_meta} 

}

\caption[Smooth versus direct Admin 1 estimates]{Smooth versus direct Admin 1 estimates. Left: Combined (meta-analysis) survey estimate against combined direct estimates. Right: Combined (meta-analysis) survey estimate against direct estimates from each survey.}\label{fig:unnamed-chunk-123}
\end{figure}


\end{knitrout}

%%%%%%%%%%%%%%%%%%%%%%%%%%% Plot2 
\begin{knitrout}
\definecolor{shadecolor}{rgb}{0.969, 0.969, 0.969}\color{fgcolor}\begin{figure}[bht]

{\centering \includegraphics[width=.9\textwidth]{../Main/Figures/SmoothMedianGabon} 

}

\caption[Maps of posterior medians for Gabon  over time]{Maps of posterior medians for Gabon  over time.}\label{fig:unnamed-chunk-124}
\end{figure}


\end{knitrout}
%%%%%%%%%%%%%%%%%%%%%%%%%%% Plot2a
\begin{knitrout}
\definecolor{shadecolor}{rgb}{0.969, 0.969, 0.969}\color{fgcolor}\begin{figure}[bht]

{\centering \includegraphics[width=.9\textwidth]{../Main/Figures/ReductionMedianGabon} 

}

\caption[Maps of reduction of posterior median U5MR in each five-year period compared to 1990 in Gabon over time]{Maps of reduction of posterior median U5MR in each five-year period compared to 1990 in Gabon over time.}\label{fig:unnamed-chunk-125}
\end{figure}


\end{knitrout}
%%%%%%%%%%%%%%%%%%%%%%%%%%% Plot3 
\begin{knitrout}
\definecolor{shadecolor}{rgb}{0.969, 0.969, 0.969}\color{fgcolor}\begin{figure}[bht]

{\centering \includegraphics[width=.95\textwidth]{../Main/Figures/Yearly_v_Periods_Gabon} 

}

\caption[Smoothed regional estimates over time]{Smoothed regional estimates over time. The line indicates yearly posterior median estimates and error bars indicate 95 \% posterior credible interval at each time period.}\label{fig:unnamed-chunk-126}
\end{figure}


\end{knitrout}

%%%%%%%%%%%%%%%%%%%%%%%%%%% Plot4 
\begin{knitrout}
\definecolor{shadecolor}{rgb}{0.969, 0.969, 0.969}\color{fgcolor}\begin{figure}[bht]

{\centering \includegraphics[width=.9\textwidth]{../Main/Figures/LineSubMedianGabon} 

}

\caption[Smoothed regional estimates over time compared to the direct estimates from each surveys]{Smoothed regional estimates over time compared to the direct estimates from each surveys. Direct estimates are not benchmarked with UN estimates. The line indicates posterior median and error bars indicate 95\% posterior credible interval.}\label{fig:unnamed-chunk-127}
\end{figure}


\end{knitrout}
% \subsubsection{National model results}
We further assess the RW2 model by holding out some observations, and compare the projections to the direct estimates in these holdout observations. Figure~\ref{fig:unnamed-chunk-128} compares the predicted estimates for the out-of-sample observations  with the direct estimates by holding out observations from each area in each time period.  Figure~\ref{fig:unnamed-chunk-129} compares the histogram of the bias rescaled by the total variance in the cross validation studies. Figure~\ref{fig:unnamed-chunk-130} compares the rescaled bias by region and time periods.



% %%%%%%%%%%%%%%%%%%%%%%%%%%% Plot6
% << echo=FALSE, out.width = ".9\\textwidth", fig.width = 12, fig.height = 6, fig.cap = "Out-of-sample predictions along with direct estimates in the cross validation study where all data from each time period is held out and predicted using the rest of the data.">>=
% fig_count <- fig_count + 1
% knitr::include_graphics(paste0("../Main/Figures/CV_byYear_withError_", countryname2, ".pdf")) 
% @
 
%%%%%%%%%%%%%%%%%%%%%%%%%%% Plot7
\begin{knitrout}
\definecolor{shadecolor}{rgb}{0.969, 0.969, 0.969}\color{fgcolor}\begin{figure}[bht]

{\centering \includegraphics[width=.9\textwidth]{../Main/Figures/CV_byYearRegion_withError_Gabon} 

}

\caption[Out-of-sample predictions along with direct estimates in the cross validation study where data from one region in each time period is held out and predicted using the rest of the data]{Out-of-sample predictions along with direct estimates in the cross validation study where data from one region in each time period is held out and predicted using the rest of the data.}\label{fig:unnamed-chunk-128}
\end{figure}


\end{knitrout}

%%%%%%%%%%%%%%%%%%%%%%%%%%% Plot8
\begin{knitrout}
\definecolor{shadecolor}{rgb}{0.969, 0.969, 0.969}\color{fgcolor}\begin{figure}[bht]

{\centering \includegraphics[width=.9\textwidth]{../Main/Figures/CVbiasGabon} 

}

\caption[Histogram and QQ-plot of the rescaled difference between the smoothed estimates and the direct estimates in the cross validation study]{Histogram and QQ-plot of the rescaled difference between the smoothed estimates and the direct estimates in the cross validation study. The differences between the two estimates are rescaled by the square root of the total variance of the two estimates.}\label{fig:unnamed-chunk-129}
\end{figure}


\end{knitrout}

%%%%%%%%%%%%%%%%%%%%%%%%%%% Plot9
\begin{knitrout}
\definecolor{shadecolor}{rgb}{0.969, 0.969, 0.969}\color{fgcolor}\begin{figure}[bht]

{\centering \includegraphics[width=.7\textwidth]{../Main/Figures/CVbiasbyRegionGabon} 

}

\caption[Line plot of the difference between smoothed estimates and the direct estimates in the cross validation study]{Line plot of the difference between smoothed estimates and the direct estimates in the cross validation study. The differences between the two estimates are rescaled by the square root of the total variance of the two estimates.}\label{fig:unnamed-chunk-130}
\end{figure}


\end{knitrout}

%%%%%%%%%%%%%%%%%%%%%%%%%%%%%%%%%%%%%%%%%%%%%%%%%%%%%%%%%%%%%%%%%%%%%%%%%%%%%%%%%%%%%%%%%%%%%%%%%%
\clearpage
\subsection{Gambia}


% \subsubsection{Summary of DHS surveys}

%%%%%%%%%%%%%%%%%%%%%%%%%%% Summary 


DHS surveys were conducted in Gambia in 2013.
% years.out[1:(length(years.out)-1)], and years.out[length(years.out)]. 

We fit both the RW2 only model to the combined national data, and compare the time trend at national level with the estimates produced by the UN and IHME in Figure~\ref{fig:unnamed-chunk-132}. We then adjusted the combined national data to the UN estimates of U5MR, and refit the models on the benchmarked data. 

%%%%%%%%%%%%%%%%%%%%%%%%%% Plot5 
\begin{knitrout}
\definecolor{shadecolor}{rgb}{0.969, 0.969, 0.969}\color{fgcolor}\begin{figure}[bht]

{\centering \includegraphics[width=.9\textwidth]{../Main/Figures/Yearly_national_Gambia} 

}

\caption[Temporal national trends along with UN (B3) estimates described in You et al]{Temporal national trends along with UN (B3) estimates described in You et al. (2015) and IHME estimates based on GBD 2015 Child Mortality Collaborators (2016). RW2 represents the smoothed national estimates using the original data before benchmarking with UN estimates. RW2-adj represents the smoothed national estimates using the benchmarked data.}\label{fig:unnamed-chunk-132}
\end{figure}


\end{knitrout}
 

We fit the RW2 model to the benchmarked data in each area. 
% The proportions of the explained variation is summarized in Table~\ref{tab:paste0(countryname, "-var")}. 
We compare the results in Figure~\ref{fig:unnamed-chunk-133} to \ref{fig:unnamed-chunk-137}.
Figure~\ref{fig:unnamed-chunk-133} compares the smoothed estimates against the direct estimates. Figure~\ref{fig:unnamed-chunk-134} and Figure~\ref{fig:unnamed-chunk-135} show the posterior median estimates of U5MR in each region over time and the reductions from 1990 period respectively.
Figure~\ref{fig:unnamed-chunk-136} shows the smoothed estimates by region over time and Figure~\ref{fig:unnamed-chunk-137} compares the smoothed estimates with direct estimates from each survey for each region over time.


% %%%%%%%%%%%%%%%%%%%%%%%%%%% Table1 
% <<echo=FALSE, results='asis'>>=
% load("rda/variance_tables.rda")
% countryname2 <- gsub(" ", "", countryname)
% variance <- tables.all[[countryname]]

% table_count <- table_count + 1

% names <- c("RW2 ($\\sigma^2_{\\gamma_{t}}$)", "ICAR ($\\sigma^2_{\\phi_{i}}$)", "IID space ($\\sigma^2_{\\theta_{i}}$)", "IID time ($\\sigma^2_{\\alpha_{t}}$)", "IID space time ($\\sigma^2_{\\delta_{it}}$)")

% variance$Proportion <- round(variance$Proportion*100, digits = 2)
% row.names(variance) <- names
% tab <- xtable(variance, digits = c(1, 3, 2),align = "l|ll",
%        label = paste0("tab:", countryname, "-var"),
%        caption = paste(country, ": summary of the variance components in the RW2 model", sep = ''))
% print(tab, comment = FALSE,sanitize.text.function = function(x) {x})
% @

%%%%%%%%%%%%%%%%%%%%%%%%%%% Plot1 
\begin{knitrout}
\definecolor{shadecolor}{rgb}{0.969, 0.969, 0.969}\color{fgcolor}\begin{figure}[bht]

{\centering \includegraphics[width=.9\textwidth]{../Main/Figures/SmoothvDirectGambia_meta} 

}

\caption[Smooth versus direct Admin 1 estimates]{Smooth versus direct Admin 1 estimates. Left: Combined (meta-analysis) survey estimate against combined direct estimates. Right: Combined (meta-analysis) survey estimate against direct estimates from each survey.}\label{fig:unnamed-chunk-133}
\end{figure}


\end{knitrout}

%%%%%%%%%%%%%%%%%%%%%%%%%%% Plot2 
\begin{knitrout}
\definecolor{shadecolor}{rgb}{0.969, 0.969, 0.969}\color{fgcolor}\begin{figure}[bht]

{\centering \includegraphics[width=.9\textwidth]{../Main/Figures/SmoothMedianGambia} 

}

\caption[Maps of posterior medians for Gambia  over time]{Maps of posterior medians for Gambia  over time.}\label{fig:unnamed-chunk-134}
\end{figure}


\end{knitrout}
%%%%%%%%%%%%%%%%%%%%%%%%%%% Plot2a
\begin{knitrout}
\definecolor{shadecolor}{rgb}{0.969, 0.969, 0.969}\color{fgcolor}\begin{figure}[bht]

{\centering \includegraphics[width=.9\textwidth]{../Main/Figures/ReductionMedianGambia} 

}

\caption[Maps of reduction of posterior median U5MR in each five-year period compared to 1990 in Gambia over time]{Maps of reduction of posterior median U5MR in each five-year period compared to 1990 in Gambia over time.}\label{fig:unnamed-chunk-135}
\end{figure}


\end{knitrout}
%%%%%%%%%%%%%%%%%%%%%%%%%%% Plot3 
\begin{knitrout}
\definecolor{shadecolor}{rgb}{0.969, 0.969, 0.969}\color{fgcolor}\begin{figure}[bht]

{\centering \includegraphics[width=.95\textwidth]{../Main/Figures/Yearly_v_Periods_Gambia} 

}

\caption[Smoothed regional estimates over time]{Smoothed regional estimates over time. The line indicates yearly posterior median estimates and error bars indicate 95 \% posterior credible interval at each time period.}\label{fig:unnamed-chunk-136}
\end{figure}


\end{knitrout}

%%%%%%%%%%%%%%%%%%%%%%%%%%% Plot4 
\begin{knitrout}
\definecolor{shadecolor}{rgb}{0.969, 0.969, 0.969}\color{fgcolor}\begin{figure}[bht]

{\centering \includegraphics[width=.9\textwidth]{../Main/Figures/LineSubMedianGambia} 

}

\caption[Smoothed regional estimates over time compared to the direct estimates from each surveys]{Smoothed regional estimates over time compared to the direct estimates from each surveys. Direct estimates are not benchmarked with UN estimates. The line indicates posterior median and error bars indicate 95\% posterior credible interval.}\label{fig:unnamed-chunk-137}
\end{figure}


\end{knitrout}
% \subsubsection{National model results}
We further assess the RW2 model by holding out some observations, and compare the projections to the direct estimates in these holdout observations. Figure~\ref{fig:unnamed-chunk-138} compares the predicted estimates for the out-of-sample observations  with the direct estimates by holding out observations from each area in each time period.  Figure~\ref{fig:unnamed-chunk-139} compares the histogram of the bias rescaled by the total variance in the cross validation studies. Figure~\ref{fig:unnamed-chunk-140} compares the rescaled bias by region and time periods.



% %%%%%%%%%%%%%%%%%%%%%%%%%%% Plot6
% << echo=FALSE, out.width = ".9\\textwidth", fig.width = 12, fig.height = 6, fig.cap = "Out-of-sample predictions along with direct estimates in the cross validation study where all data from each time period is held out and predicted using the rest of the data.">>=
% fig_count <- fig_count + 1
% knitr::include_graphics(paste0("../Main/Figures/CV_byYear_withError_", countryname2, ".pdf")) 
% @
 
%%%%%%%%%%%%%%%%%%%%%%%%%%% Plot7
\begin{knitrout}
\definecolor{shadecolor}{rgb}{0.969, 0.969, 0.969}\color{fgcolor}\begin{figure}[bht]

{\centering \includegraphics[width=.9\textwidth]{../Main/Figures/CV_byYearRegion_withError_Gambia} 

}

\caption[Out-of-sample predictions along with direct estimates in the cross validation study where data from one region in each time period is held out and predicted using the rest of the data]{Out-of-sample predictions along with direct estimates in the cross validation study where data from one region in each time period is held out and predicted using the rest of the data.}\label{fig:unnamed-chunk-138}
\end{figure}


\end{knitrout}

%%%%%%%%%%%%%%%%%%%%%%%%%%% Plot8
\begin{knitrout}
\definecolor{shadecolor}{rgb}{0.969, 0.969, 0.969}\color{fgcolor}\begin{figure}[bht]

{\centering \includegraphics[width=.9\textwidth]{../Main/Figures/CVbiasGambia} 

}

\caption[Histogram and QQ-plot of the rescaled difference between the smoothed estimates and the direct estimates in the cross validation study]{Histogram and QQ-plot of the rescaled difference between the smoothed estimates and the direct estimates in the cross validation study. The differences between the two estimates are rescaled by the square root of the total variance of the two estimates.}\label{fig:unnamed-chunk-139}
\end{figure}


\end{knitrout}

%%%%%%%%%%%%%%%%%%%%%%%%%%% Plot9
\begin{knitrout}
\definecolor{shadecolor}{rgb}{0.969, 0.969, 0.969}\color{fgcolor}\begin{figure}[bht]

{\centering \includegraphics[width=.7\textwidth]{../Main/Figures/CVbiasbyRegionGambia} 

}

\caption[Line plot of the difference between smoothed estimates and the direct estimates in the cross validation study]{Line plot of the difference between smoothed estimates and the direct estimates in the cross validation study. The differences between the two estimates are rescaled by the square root of the total variance of the two estimates.}\label{fig:unnamed-chunk-140}
\end{figure}


\end{knitrout}


%%%%%%%%%%%%%%%%%%%%%%%%%%%%%%%%%%%%%%%%%%%%%%%%%%%%%%%%%%%%%%%%%%%%%%%%%%%%%%%%%%%%%%%%%%%%%%%%%%
\clearpage
\subsection{Ghana}


% \subsubsection{Summary of DHS surveys}

%%%%%%%%%%%%%%%%%%%%%%%%%%% Summary 


DHS surveys were conducted in Ghana in 1989, 1993, 1998, 2003, 2008, and 2014.
% years.out[1:(length(years.out)-1)], and years.out[length(years.out)]. 

We fit both the RW2 only model to the combined national data, and compare the time trend at national level with the estimates produced by the UN and IHME in Figure~\ref{fig:unnamed-chunk-142}. We then adjusted the combined national data to the UN estimates of U5MR, and refit the models on the benchmarked data. 

%%%%%%%%%%%%%%%%%%%%%%%%%% Plot5 
\begin{knitrout}
\definecolor{shadecolor}{rgb}{0.969, 0.969, 0.969}\color{fgcolor}\begin{figure}[bht]

{\centering \includegraphics[width=.9\textwidth]{../Main/Figures/Yearly_national_Ghana} 

}

\caption[Temporal national trends along with UN (B3) estimates described in You et al]{Temporal national trends along with UN (B3) estimates described in You et al. (2015) and IHME estimates based on GBD 2015 Child Mortality Collaborators (2016). RW2 represents the smoothed national estimates using the original data before benchmarking with UN estimates. RW2-adj represents the smoothed national estimates using the benchmarked data.}\label{fig:unnamed-chunk-142}
\end{figure}


\end{knitrout}
 

We fit the RW2 model to the benchmarked data in each area. 
% The proportions of the explained variation is summarized in Table~\ref{tab:paste0(countryname, "-var")}. 
We compare the results in Figure~\ref{fig:unnamed-chunk-143} to \ref{fig:unnamed-chunk-147}.
Figure~\ref{fig:unnamed-chunk-143} compares the smoothed estimates against the direct estimates. Figure~\ref{fig:unnamed-chunk-144} and Figure~\ref{fig:unnamed-chunk-145} show the posterior median estimates of U5MR in each region over time and the reductions from 1990 period respectively.
Figure~\ref{fig:unnamed-chunk-146} shows the smoothed estimates by region over time and Figure~\ref{fig:unnamed-chunk-147} compares the smoothed estimates with direct estimates from each survey for each region over time.


% %%%%%%%%%%%%%%%%%%%%%%%%%%% Table1 
% <<echo=FALSE, results='asis'>>=
% load("rda/variance_tables.rda")
% countryname2 <- gsub(" ", "", countryname)
% variance <- tables.all[[countryname]]

% table_count <- table_count + 1

% names <- c("RW2 ($\\sigma^2_{\\gamma_{t}}$)", "ICAR ($\\sigma^2_{\\phi_{i}}$)", "IID space ($\\sigma^2_{\\theta_{i}}$)", "IID time ($\\sigma^2_{\\alpha_{t}}$)", "IID space time ($\\sigma^2_{\\delta_{it}}$)")

% variance$Proportion <- round(variance$Proportion*100, digits = 2)
% row.names(variance) <- names
% tab <- xtable(variance, digits = c(1, 3, 2),align = "l|ll",
%        label = paste0("tab:", countryname, "-var"),
%        caption = paste(country, ": summary of the variance components in the RW2 model", sep = ''))
% print(tab, comment = FALSE,sanitize.text.function = function(x) {x})
% @

%%%%%%%%%%%%%%%%%%%%%%%%%%% Plot1 
\begin{knitrout}
\definecolor{shadecolor}{rgb}{0.969, 0.969, 0.969}\color{fgcolor}\begin{figure}[bht]

{\centering \includegraphics[width=.9\textwidth]{../Main/Figures/SmoothvDirectGhana_meta} 

}

\caption[Smooth versus direct Admin 1 estimates]{Smooth versus direct Admin 1 estimates. Left: Combined (meta-analysis) survey estimate against combined direct estimates. Right: Combined (meta-analysis) survey estimate against direct estimates from each survey.}\label{fig:unnamed-chunk-143}
\end{figure}


\end{knitrout}

%%%%%%%%%%%%%%%%%%%%%%%%%%% Plot2 
\begin{knitrout}
\definecolor{shadecolor}{rgb}{0.969, 0.969, 0.969}\color{fgcolor}\begin{figure}[bht]

{\centering \includegraphics[width=.9\textwidth]{../Main/Figures/SmoothMedianGhana} 

}

\caption[Maps of posterior medians for Ghana  over time]{Maps of posterior medians for Ghana  over time.}\label{fig:unnamed-chunk-144}
\end{figure}


\end{knitrout}
%%%%%%%%%%%%%%%%%%%%%%%%%%% Plot2a
\begin{knitrout}
\definecolor{shadecolor}{rgb}{0.969, 0.969, 0.969}\color{fgcolor}\begin{figure}[bht]

{\centering \includegraphics[width=.9\textwidth]{../Main/Figures/ReductionMedianGhana} 

}

\caption[Maps of reduction of posterior median U5MR in each five-year period compared to 1990 in Ghana over time]{Maps of reduction of posterior median U5MR in each five-year period compared to 1990 in Ghana over time.}\label{fig:unnamed-chunk-145}
\end{figure}


\end{knitrout}
%%%%%%%%%%%%%%%%%%%%%%%%%%% Plot3 
\begin{knitrout}
\definecolor{shadecolor}{rgb}{0.969, 0.969, 0.969}\color{fgcolor}\begin{figure}[bht]

{\centering \includegraphics[width=.95\textwidth]{../Main/Figures/Yearly_v_Periods_Ghana} 

}

\caption[Smoothed regional estimates over time]{Smoothed regional estimates over time. The line indicates yearly posterior median estimates and error bars indicate 95 \% posterior credible interval at each time period.}\label{fig:unnamed-chunk-146}
\end{figure}


\end{knitrout}

%%%%%%%%%%%%%%%%%%%%%%%%%%% Plot4 
\begin{knitrout}
\definecolor{shadecolor}{rgb}{0.969, 0.969, 0.969}\color{fgcolor}\begin{figure}[bht]

{\centering \includegraphics[width=.9\textwidth]{../Main/Figures/LineSubMedianGhana} 

}

\caption[Smoothed regional estimates over time compared to the direct estimates from each surveys]{Smoothed regional estimates over time compared to the direct estimates from each surveys. Direct estimates are not benchmarked with UN estimates. The line indicates posterior median and error bars indicate 95\% posterior credible interval.}\label{fig:unnamed-chunk-147}
\end{figure}


\end{knitrout}
% \subsubsection{National model results}
We further assess the RW2 model by holding out some observations, and compare the projections to the direct estimates in these holdout observations. Figure~\ref{fig:unnamed-chunk-148} compares the predicted estimates for the out-of-sample observations  with the direct estimates by holding out observations from each area in each time period.  Figure~\ref{fig:unnamed-chunk-149} compares the histogram of the bias rescaled by the total variance in the cross validation studies. Figure~\ref{fig:unnamed-chunk-150} compares the rescaled bias by region and time periods.



% %%%%%%%%%%%%%%%%%%%%%%%%%%% Plot6
% << echo=FALSE, out.width = ".9\\textwidth", fig.width = 12, fig.height = 6, fig.cap = "Out-of-sample predictions along with direct estimates in the cross validation study where all data from each time period is held out and predicted using the rest of the data.">>=
% fig_count <- fig_count + 1
% knitr::include_graphics(paste0("../Main/Figures/CV_byYear_withError_", countryname2, ".pdf")) 
% @
 
%%%%%%%%%%%%%%%%%%%%%%%%%%% Plot7
\begin{knitrout}
\definecolor{shadecolor}{rgb}{0.969, 0.969, 0.969}\color{fgcolor}\begin{figure}[bht]

{\centering \includegraphics[width=.9\textwidth]{../Main/Figures/CV_byYearRegion_withError_Ghana} 

}

\caption[Out-of-sample predictions along with direct estimates in the cross validation study where data from one region in each time period is held out and predicted using the rest of the data]{Out-of-sample predictions along with direct estimates in the cross validation study where data from one region in each time period is held out and predicted using the rest of the data.}\label{fig:unnamed-chunk-148}
\end{figure}


\end{knitrout}

%%%%%%%%%%%%%%%%%%%%%%%%%%% Plot8
\begin{knitrout}
\definecolor{shadecolor}{rgb}{0.969, 0.969, 0.969}\color{fgcolor}\begin{figure}[bht]

{\centering \includegraphics[width=.9\textwidth]{../Main/Figures/CVbiasGhana} 

}

\caption[Histogram and QQ-plot of the rescaled difference between the smoothed estimates and the direct estimates in the cross validation study]{Histogram and QQ-plot of the rescaled difference between the smoothed estimates and the direct estimates in the cross validation study. The differences between the two estimates are rescaled by the square root of the total variance of the two estimates.}\label{fig:unnamed-chunk-149}
\end{figure}


\end{knitrout}

%%%%%%%%%%%%%%%%%%%%%%%%%%% Plot9
\begin{knitrout}
\definecolor{shadecolor}{rgb}{0.969, 0.969, 0.969}\color{fgcolor}\begin{figure}[bht]

{\centering \includegraphics[width=.7\textwidth]{../Main/Figures/CVbiasbyRegionGhana} 

}

\caption[Line plot of the difference between smoothed estimates and the direct estimates in the cross validation study]{Line plot of the difference between smoothed estimates and the direct estimates in the cross validation study. The differences between the two estimates are rescaled by the square root of the total variance of the two estimates.}\label{fig:unnamed-chunk-150}
\end{figure}


\end{knitrout}


%%%%%%%%%%%%%%%%%%%%%%%%%%%%%%%%%%%%%%%%%%%%%%%%%%%%%%%%%%%%%%%%%%%%%%%%%%%%%%%%%%%%%%%%%%%%%%%%%%
\clearpage
\subsection{Guinea}


% \subsubsection{Summary of DHS surveys}

%%%%%%%%%%%%%%%%%%%%%%%%%%% Summary 


DHS surveys were conducted in Guinea in 1999, 2005, and 2012.
% years.out[1:(length(years.out)-1)], and years.out[length(years.out)]. 

We fit both the RW2 only model to the combined national data, and compare the time trend at national level with the estimates produced by the UN and IHME in Figure~\ref{fig:unnamed-chunk-152}. We then adjusted the combined national data to the UN estimates of U5MR, and refit the models on the benchmarked data. 

%%%%%%%%%%%%%%%%%%%%%%%%%% Plot5 
\begin{knitrout}
\definecolor{shadecolor}{rgb}{0.969, 0.969, 0.969}\color{fgcolor}\begin{figure}[bht]

{\centering \includegraphics[width=.9\textwidth]{../Main/Figures/Yearly_national_Guinea} 

}

\caption[Temporal national trends along with UN (B3) estimates described in You et al]{Temporal national trends along with UN (B3) estimates described in You et al. (2015) and IHME estimates based on GBD 2015 Child Mortality Collaborators (2016). RW2 represents the smoothed national estimates using the original data before benchmarking with UN estimates. RW2-adj represents the smoothed national estimates using the benchmarked data.}\label{fig:unnamed-chunk-152}
\end{figure}


\end{knitrout}
 

We fit the RW2 model to the benchmarked data in each area. 
% The proportions of the explained variation is summarized in Table~\ref{tab:paste0(countryname, "-var")}. 
We compare the results in Figure~\ref{fig:unnamed-chunk-153} to \ref{fig:unnamed-chunk-157}.
Figure~\ref{fig:unnamed-chunk-153} compares the smoothed estimates against the direct estimates. Figure~\ref{fig:unnamed-chunk-154} and Figure~\ref{fig:unnamed-chunk-155} show the posterior median estimates of U5MR in each region over time and the reductions from 1990 period respectively.
Figure~\ref{fig:unnamed-chunk-156} shows the smoothed estimates by region over time and Figure~\ref{fig:unnamed-chunk-157} compares the smoothed estimates with direct estimates from each survey for each region over time.


% %%%%%%%%%%%%%%%%%%%%%%%%%%% Table1 
% <<echo=FALSE, results='asis'>>=
% load("rda/variance_tables.rda")
% countryname2 <- gsub(" ", "", countryname)
% variance <- tables.all[[countryname]]

% table_count <- table_count + 1

% names <- c("RW2 ($\\sigma^2_{\\gamma_{t}}$)", "ICAR ($\\sigma^2_{\\phi_{i}}$)", "IID space ($\\sigma^2_{\\theta_{i}}$)", "IID time ($\\sigma^2_{\\alpha_{t}}$)", "IID space time ($\\sigma^2_{\\delta_{it}}$)")

% variance$Proportion <- round(variance$Proportion*100, digits = 2)
% row.names(variance) <- names
% tab <- xtable(variance, digits = c(1, 3, 2),align = "l|ll",
%        label = paste0("tab:", countryname, "-var"),
%        caption = paste(country, ": summary of the variance components in the RW2 model", sep = ''))
% print(tab, comment = FALSE,sanitize.text.function = function(x) {x})
% @

%%%%%%%%%%%%%%%%%%%%%%%%%%% Plot1 
\begin{knitrout}
\definecolor{shadecolor}{rgb}{0.969, 0.969, 0.969}\color{fgcolor}\begin{figure}[bht]

{\centering \includegraphics[width=.9\textwidth]{../Main/Figures/SmoothvDirectGuinea_meta} 

}

\caption[Smooth versus direct Admin 1 estimates]{Smooth versus direct Admin 1 estimates. Left: Combined (meta-analysis) survey estimate against combined direct estimates. Right: Combined (meta-analysis) survey estimate against direct estimates from each survey.}\label{fig:unnamed-chunk-153}
\end{figure}


\end{knitrout}

%%%%%%%%%%%%%%%%%%%%%%%%%%% Plot2 
\begin{knitrout}
\definecolor{shadecolor}{rgb}{0.969, 0.969, 0.969}\color{fgcolor}\begin{figure}[bht]

{\centering \includegraphics[width=.9\textwidth]{../Main/Figures/SmoothMedianGuinea} 

}

\caption[Maps of posterior medians for Guinea  over time]{Maps of posterior medians for Guinea  over time.}\label{fig:unnamed-chunk-154}
\end{figure}


\end{knitrout}
%%%%%%%%%%%%%%%%%%%%%%%%%%% Plot2a
\begin{knitrout}
\definecolor{shadecolor}{rgb}{0.969, 0.969, 0.969}\color{fgcolor}\begin{figure}[bht]

{\centering \includegraphics[width=.9\textwidth]{../Main/Figures/ReductionMedianGuinea} 

}

\caption[Maps of reduction of posterior median U5MR in each five-year period compared to 1990 in Guinea over time]{Maps of reduction of posterior median U5MR in each five-year period compared to 1990 in Guinea over time.}\label{fig:unnamed-chunk-155}
\end{figure}


\end{knitrout}
%%%%%%%%%%%%%%%%%%%%%%%%%%% Plot3 
\begin{knitrout}
\definecolor{shadecolor}{rgb}{0.969, 0.969, 0.969}\color{fgcolor}\begin{figure}[bht]

{\centering \includegraphics[width=.95\textwidth]{../Main/Figures/Yearly_v_Periods_Guinea} 

}

\caption[Smoothed regional estimates over time]{Smoothed regional estimates over time. The line indicates yearly posterior median estimates and error bars indicate 95 \% posterior credible interval at each time period.}\label{fig:unnamed-chunk-156}
\end{figure}


\end{knitrout}

%%%%%%%%%%%%%%%%%%%%%%%%%%% Plot4 
\begin{knitrout}
\definecolor{shadecolor}{rgb}{0.969, 0.969, 0.969}\color{fgcolor}\begin{figure}[bht]

{\centering \includegraphics[width=.9\textwidth]{../Main/Figures/LineSubMedianGuinea} 

}

\caption[Smoothed regional estimates over time compared to the direct estimates from each surveys]{Smoothed regional estimates over time compared to the direct estimates from each surveys. Direct estimates are not benchmarked with UN estimates. The line indicates posterior median and error bars indicate 95\% posterior credible interval.}\label{fig:unnamed-chunk-157}
\end{figure}


\end{knitrout}
% \subsubsection{National model results}
We further assess the RW2 model by holding out some observations, and compare the projections to the direct estimates in these holdout observations. Figure~\ref{fig:unnamed-chunk-158} compares the predicted estimates for the out-of-sample observations  with the direct estimates by holding out observations from each area in each time period.  Figure~\ref{fig:unnamed-chunk-159} compares the histogram of the bias rescaled by the total variance in the cross validation studies. Figure~\ref{fig:unnamed-chunk-160} compares the rescaled bias by region and time periods.



% %%%%%%%%%%%%%%%%%%%%%%%%%%% Plot6
% << echo=FALSE, out.width = ".9\\textwidth", fig.width = 12, fig.height = 6, fig.cap = "Out-of-sample predictions along with direct estimates in the cross validation study where all data from each time period is held out and predicted using the rest of the data.">>=
% fig_count <- fig_count + 1
% knitr::include_graphics(paste0("../Main/Figures/CV_byYear_withError_", countryname2, ".pdf")) 
% @
 
%%%%%%%%%%%%%%%%%%%%%%%%%%% Plot7
\begin{knitrout}
\definecolor{shadecolor}{rgb}{0.969, 0.969, 0.969}\color{fgcolor}\begin{figure}[bht]

{\centering \includegraphics[width=.9\textwidth]{../Main/Figures/CV_byYearRegion_withError_Guinea} 

}

\caption[Out-of-sample predictions along with direct estimates in the cross validation study where data from one region in each time period is held out and predicted using the rest of the data]{Out-of-sample predictions along with direct estimates in the cross validation study where data from one region in each time period is held out and predicted using the rest of the data.}\label{fig:unnamed-chunk-158}
\end{figure}


\end{knitrout}

%%%%%%%%%%%%%%%%%%%%%%%%%%% Plot8
\begin{knitrout}
\definecolor{shadecolor}{rgb}{0.969, 0.969, 0.969}\color{fgcolor}\begin{figure}[bht]

{\centering \includegraphics[width=.9\textwidth]{../Main/Figures/CVbiasGuinea} 

}

\caption[Histogram and QQ-plot of the rescaled difference between the smoothed estimates and the direct estimates in the cross validation study]{Histogram and QQ-plot of the rescaled difference between the smoothed estimates and the direct estimates in the cross validation study. The differences between the two estimates are rescaled by the square root of the total variance of the two estimates.}\label{fig:unnamed-chunk-159}
\end{figure}


\end{knitrout}

%%%%%%%%%%%%%%%%%%%%%%%%%%% Plot9
\begin{knitrout}
\definecolor{shadecolor}{rgb}{0.969, 0.969, 0.969}\color{fgcolor}\begin{figure}[bht]

{\centering \includegraphics[width=.7\textwidth]{../Main/Figures/CVbiasbyRegionGuinea} 

}

\caption[Line plot of the difference between smoothed estimates and the direct estimates in the cross validation study]{Line plot of the difference between smoothed estimates and the direct estimates in the cross validation study. The differences between the two estimates are rescaled by the square root of the total variance of the two estimates.}\label{fig:unnamed-chunk-160}
\end{figure}


\end{knitrout}


%%%%%%%%%%%%%%%%%%%%%%%%%%%%%%%%%%%%%%%%%%%%%%%%%%%%%%%%%%%%%%%%%%%%%%%%%%%%%%%%%%%%%%%%%%%%%%%%%%
\clearpage
\subsection{Kenya}


% \subsubsection{Summary of DHS surveys}

%%%%%%%%%%%%%%%%%%%%%%%%%%% Summary 


DHS surveys were conducted in Kenya in 1993, 1998, 2003, 2008, and 2014.
% years.out[1:(length(years.out)-1)], and years.out[length(years.out)]. 

We fit both the RW2 only model to the combined national data, and compare the time trend at national level with the estimates produced by the UN and IHME in Figure~\ref{fig:unnamed-chunk-162}. We then adjusted the combined national data to the UN estimates of U5MR, and refit the models on the benchmarked data. 

%%%%%%%%%%%%%%%%%%%%%%%%%% Plot5 
\begin{knitrout}
\definecolor{shadecolor}{rgb}{0.969, 0.969, 0.969}\color{fgcolor}\begin{figure}[bht]

{\centering \includegraphics[width=.9\textwidth]{../Main/Figures/Yearly_national_Kenya} 

}

\caption[Temporal national trends along with UN (B3) estimates described in You et al]{Temporal national trends along with UN (B3) estimates described in You et al. (2015) and IHME estimates based on GBD 2015 Child Mortality Collaborators (2016). RW2 represents the smoothed national estimates using the original data before benchmarking with UN estimates. RW2-adj represents the smoothed national estimates using the benchmarked data.}\label{fig:unnamed-chunk-162}
\end{figure}


\end{knitrout}
 

We fit the RW2 model to the benchmarked data in each area. 
% The proportions of the explained variation is summarized in Table~\ref{tab:paste0(countryname, "-var")}. 
We compare the results in Figure~\ref{fig:unnamed-chunk-163} to \ref{fig:unnamed-chunk-167}.
Figure~\ref{fig:unnamed-chunk-163} compares the smoothed estimates against the direct estimates. Figure~\ref{fig:unnamed-chunk-164} and Figure~\ref{fig:unnamed-chunk-165} show the posterior median estimates of U5MR in each region over time and the reductions from 1990 period respectively.
Figure~\ref{fig:unnamed-chunk-166} shows the smoothed estimates by region over time and Figure~\ref{fig:unnamed-chunk-167} compares the smoothed estimates with direct estimates from each survey for each region over time.


% %%%%%%%%%%%%%%%%%%%%%%%%%%% Table1 
% <<echo=FALSE, results='asis'>>=
% load("rda/variance_tables.rda")
% countryname2 <- gsub(" ", "", countryname)
% variance <- tables.all[[countryname]]

% table_count <- table_count + 1

% names <- c("RW2 ($\\sigma^2_{\\gamma_{t}}$)", "ICAR ($\\sigma^2_{\\phi_{i}}$)", "IID space ($\\sigma^2_{\\theta_{i}}$)", "IID time ($\\sigma^2_{\\alpha_{t}}$)", "IID space time ($\\sigma^2_{\\delta_{it}}$)")

% variance$Proportion <- round(variance$Proportion*100, digits = 2)
% row.names(variance) <- names
% tab <- xtable(variance, digits = c(1, 3, 2),align = "l|ll",
%        label = paste0("tab:", countryname, "-var"),
%        caption = paste(country, ": summary of the variance components in the RW2 model", sep = ''))
% print(tab, comment = FALSE,sanitize.text.function = function(x) {x})
% @

%%%%%%%%%%%%%%%%%%%%%%%%%%% Plot1 
\begin{knitrout}
\definecolor{shadecolor}{rgb}{0.969, 0.969, 0.969}\color{fgcolor}\begin{figure}[bht]

{\centering \includegraphics[width=.9\textwidth]{../Main/Figures/SmoothvDirectKenya_meta} 

}

\caption[Smooth versus direct Admin 1 estimates]{Smooth versus direct Admin 1 estimates. Left: Combined (meta-analysis) survey estimate against combined direct estimates. Right: Combined (meta-analysis) survey estimate against direct estimates from each survey.}\label{fig:unnamed-chunk-163}
\end{figure}


\end{knitrout}

%%%%%%%%%%%%%%%%%%%%%%%%%%% Plot2 
\begin{knitrout}
\definecolor{shadecolor}{rgb}{0.969, 0.969, 0.969}\color{fgcolor}\begin{figure}[bht]

{\centering \includegraphics[width=.9\textwidth]{../Main/Figures/SmoothMedianKenya} 

}

\caption[Maps of posterior medians for Kenya  over time]{Maps of posterior medians for Kenya  over time.}\label{fig:unnamed-chunk-164}
\end{figure}


\end{knitrout}
%%%%%%%%%%%%%%%%%%%%%%%%%%% Plot2a
\begin{knitrout}
\definecolor{shadecolor}{rgb}{0.969, 0.969, 0.969}\color{fgcolor}\begin{figure}[bht]

{\centering \includegraphics[width=.9\textwidth]{../Main/Figures/ReductionMedianKenya} 

}

\caption[Maps of reduction of posterior median U5MR in each five-year period compared to 1990 in Kenya over time]{Maps of reduction of posterior median U5MR in each five-year period compared to 1990 in Kenya over time.}\label{fig:unnamed-chunk-165}
\end{figure}


\end{knitrout}
%%%%%%%%%%%%%%%%%%%%%%%%%%% Plot3 
\begin{knitrout}
\definecolor{shadecolor}{rgb}{0.969, 0.969, 0.969}\color{fgcolor}\begin{figure}[bht]

{\centering \includegraphics[width=.95\textwidth]{../Main/Figures/Yearly_v_Periods_Kenya} 

}

\caption[Smoothed regional estimates over time]{Smoothed regional estimates over time. The line indicates yearly posterior median estimates and error bars indicate 95 \% posterior credible interval at each time period.}\label{fig:unnamed-chunk-166}
\end{figure}


\end{knitrout}

%%%%%%%%%%%%%%%%%%%%%%%%%%% Plot4 
\begin{knitrout}
\definecolor{shadecolor}{rgb}{0.969, 0.969, 0.969}\color{fgcolor}\begin{figure}[bht]

{\centering \includegraphics[width=.9\textwidth]{../Main/Figures/LineSubMedianKenya} 

}

\caption[Smoothed regional estimates over time compared to the direct estimates from each surveys]{Smoothed regional estimates over time compared to the direct estimates from each surveys. Direct estimates are not benchmarked with UN estimates. The line indicates posterior median and error bars indicate 95\% posterior credible interval.}\label{fig:unnamed-chunk-167}
\end{figure}


\end{knitrout}
% \subsubsection{National model results}
We further assess the RW2 model by holding out some observations, and compare the projections to the direct estimates in these holdout observations. Figure~\ref{fig:unnamed-chunk-168} compares the predicted estimates for the out-of-sample observations  with the direct estimates by holding out observations from each area in each time period.  Figure~\ref{fig:unnamed-chunk-169} compares the histogram of the bias rescaled by the total variance in the cross validation studies. Figure~\ref{fig:unnamed-chunk-170} compares the rescaled bias by region and time periods.



% %%%%%%%%%%%%%%%%%%%%%%%%%%% Plot6
% << echo=FALSE, out.width = ".9\\textwidth", fig.width = 12, fig.height = 6, fig.cap = "Out-of-sample predictions along with direct estimates in the cross validation study where all data from each time period is held out and predicted using the rest of the data.">>=
% fig_count <- fig_count + 1
% knitr::include_graphics(paste0("../Main/Figures/CV_byYear_withError_", countryname2, ".pdf")) 
% @
 
%%%%%%%%%%%%%%%%%%%%%%%%%%% Plot7
\begin{knitrout}
\definecolor{shadecolor}{rgb}{0.969, 0.969, 0.969}\color{fgcolor}\begin{figure}[bht]

{\centering \includegraphics[width=.9\textwidth]{../Main/Figures/CV_byYearRegion_withError_Kenya} 

}

\caption[Out-of-sample predictions along with direct estimates in the cross validation study where data from one region in each time period is held out and predicted using the rest of the data]{Out-of-sample predictions along with direct estimates in the cross validation study where data from one region in each time period is held out and predicted using the rest of the data.}\label{fig:unnamed-chunk-168}
\end{figure}


\end{knitrout}

%%%%%%%%%%%%%%%%%%%%%%%%%%% Plot8
\begin{knitrout}
\definecolor{shadecolor}{rgb}{0.969, 0.969, 0.969}\color{fgcolor}\begin{figure}[bht]

{\centering \includegraphics[width=.9\textwidth]{../Main/Figures/CVbiasKenya} 

}

\caption[Histogram and QQ-plot of the rescaled difference between the smoothed estimates and the direct estimates in the cross validation study]{Histogram and QQ-plot of the rescaled difference between the smoothed estimates and the direct estimates in the cross validation study. The differences between the two estimates are rescaled by the square root of the total variance of the two estimates.}\label{fig:unnamed-chunk-169}
\end{figure}


\end{knitrout}

%%%%%%%%%%%%%%%%%%%%%%%%%%% Plot9
\begin{knitrout}
\definecolor{shadecolor}{rgb}{0.969, 0.969, 0.969}\color{fgcolor}\begin{figure}[bht]

{\centering \includegraphics[width=.7\textwidth]{../Main/Figures/CVbiasbyRegionKenya} 

}

\caption[Line plot of the difference between smoothed estimates and the direct estimates in the cross validation study]{Line plot of the difference between smoothed estimates and the direct estimates in the cross validation study. The differences between the two estimates are rescaled by the square root of the total variance of the two estimates.}\label{fig:unnamed-chunk-170}
\end{figure}


\end{knitrout}


%%%%%%%%%%%%%%%%%%%%%%%%%%%%%%%%%%%%%%%%%%%%%%%%%%%%%%%%%%%%%%%%%%%%%%%%%%%%%%%%%%%%%%%%%%%%%%%%%%
\clearpage
\subsection{Lesotho}


% \subsubsection{Summary of DHS surveys}

%%%%%%%%%%%%%%%%%%%%%%%%%%% Summary 


DHS surveys were conducted in Lesotho in 2005, 2010, and 2014.
% years.out[1:(length(years.out)-1)], and years.out[length(years.out)]. 

We fit both the RW2 only model to the combined national data, and compare the time trend at national level with the estimates produced by the UN and IHME in Figure~\ref{fig:unnamed-chunk-172}. We then adjusted the combined national data to the UN estimates of U5MR, and refit the models on the benchmarked data. 

%%%%%%%%%%%%%%%%%%%%%%%%%% Plot5 
\begin{knitrout}
\definecolor{shadecolor}{rgb}{0.969, 0.969, 0.969}\color{fgcolor}\begin{figure}[bht]

{\centering \includegraphics[width=.9\textwidth]{../Main/Figures/Yearly_national_Lesotho} 

}

\caption[Temporal national trends along with UN (B3) estimates described in You et al]{Temporal national trends along with UN (B3) estimates described in You et al. (2015) and IHME estimates based on GBD 2015 Child Mortality Collaborators (2016). RW2 represents the smoothed national estimates using the original data before benchmarking with UN estimates. RW2-adj represents the smoothed national estimates using the benchmarked data.}\label{fig:unnamed-chunk-172}
\end{figure}


\end{knitrout}
 

We fit the RW2 model to the benchmarked data in each area. 
% The proportions of the explained variation is summarized in Table~\ref{tab:paste0(countryname, "-var")}. 
We compare the results in Figure~\ref{fig:unnamed-chunk-173} to \ref{fig:unnamed-chunk-177}.
Figure~\ref{fig:unnamed-chunk-173} compares the smoothed estimates against the direct estimates. Figure~\ref{fig:unnamed-chunk-174} and Figure~\ref{fig:unnamed-chunk-175} show the posterior median estimates of U5MR in each region over time and the reductions from 1990 period respectively.
Figure~\ref{fig:unnamed-chunk-176} shows the smoothed estimates by region over time and Figure~\ref{fig:unnamed-chunk-177} compares the smoothed estimates with direct estimates from each survey for each region over time.


% %%%%%%%%%%%%%%%%%%%%%%%%%%% Table1 
% <<echo=FALSE, results='asis'>>=
% load("rda/variance_tables.rda")
% countryname2 <- gsub(" ", "", countryname)
% variance <- tables.all[[countryname]]

% table_count <- table_count + 1

% names <- c("RW2 ($\\sigma^2_{\\gamma_{t}}$)", "ICAR ($\\sigma^2_{\\phi_{i}}$)", "IID space ($\\sigma^2_{\\theta_{i}}$)", "IID time ($\\sigma^2_{\\alpha_{t}}$)", "IID space time ($\\sigma^2_{\\delta_{it}}$)")

% variance$Proportion <- round(variance$Proportion*100, digits = 2)
% row.names(variance) <- names
% tab <- xtable(variance, digits = c(1, 3, 2),align = "l|ll",
%        label = paste0("tab:", countryname, "-var"),
%        caption = paste(country, ": summary of the variance components in the RW2 model", sep = ''))
% print(tab, comment = FALSE,sanitize.text.function = function(x) {x})
% @

%%%%%%%%%%%%%%%%%%%%%%%%%%% Plot1 
\begin{knitrout}
\definecolor{shadecolor}{rgb}{0.969, 0.969, 0.969}\color{fgcolor}\begin{figure}[bht]

{\centering \includegraphics[width=.9\textwidth]{../Main/Figures/SmoothvDirectLesotho_meta} 

}

\caption[Smooth versus direct Admin 1 estimates]{Smooth versus direct Admin 1 estimates. Left: Combined (meta-analysis) survey estimate against combined direct estimates. Right: Combined (meta-analysis) survey estimate against direct estimates from each survey.}\label{fig:unnamed-chunk-173}
\end{figure}


\end{knitrout}

%%%%%%%%%%%%%%%%%%%%%%%%%%% Plot2 
\begin{knitrout}
\definecolor{shadecolor}{rgb}{0.969, 0.969, 0.969}\color{fgcolor}\begin{figure}[bht]

{\centering \includegraphics[width=.9\textwidth]{../Main/Figures/SmoothMedianLesotho} 

}

\caption[Maps of posterior medians for Lesotho  over time]{Maps of posterior medians for Lesotho  over time.}\label{fig:unnamed-chunk-174}
\end{figure}


\end{knitrout}
%%%%%%%%%%%%%%%%%%%%%%%%%%% Plot2a
\begin{knitrout}
\definecolor{shadecolor}{rgb}{0.969, 0.969, 0.969}\color{fgcolor}\begin{figure}[bht]

{\centering \includegraphics[width=.9\textwidth]{../Main/Figures/ReductionMedianLesotho} 

}

\caption[Maps of reduction of posterior median U5MR in each five-year period compared to 1990 in Lesotho over time]{Maps of reduction of posterior median U5MR in each five-year period compared to 1990 in Lesotho over time.}\label{fig:unnamed-chunk-175}
\end{figure}


\end{knitrout}
%%%%%%%%%%%%%%%%%%%%%%%%%%% Plot3 
\begin{knitrout}
\definecolor{shadecolor}{rgb}{0.969, 0.969, 0.969}\color{fgcolor}\begin{figure}[bht]

{\centering \includegraphics[width=.95\textwidth]{../Main/Figures/Yearly_v_Periods_Lesotho} 

}

\caption[Smoothed regional estimates over time]{Smoothed regional estimates over time. The line indicates yearly posterior median estimates and error bars indicate 95 \% posterior credible interval at each time period.}\label{fig:unnamed-chunk-176}
\end{figure}


\end{knitrout}

%%%%%%%%%%%%%%%%%%%%%%%%%%% Plot4 
\begin{knitrout}
\definecolor{shadecolor}{rgb}{0.969, 0.969, 0.969}\color{fgcolor}\begin{figure}[bht]

{\centering \includegraphics[width=.9\textwidth]{../Main/Figures/LineSubMedianLesotho} 

}

\caption[Smoothed regional estimates over time compared to the direct estimates from each surveys]{Smoothed regional estimates over time compared to the direct estimates from each surveys. Direct estimates are not benchmarked with UN estimates. The line indicates posterior median and error bars indicate 95\% posterior credible interval.}\label{fig:unnamed-chunk-177}
\end{figure}


\end{knitrout}
% \subsubsection{National model results}
We further assess the RW2 model by holding out some observations, and compare the projections to the direct estimates in these holdout observations. Figure~\ref{fig:unnamed-chunk-178} compares the predicted estimates for the out-of-sample observations  with the direct estimates by holding out observations from each area in each time period.  Figure~\ref{fig:unnamed-chunk-179} compares the histogram of the bias rescaled by the total variance in the cross validation studies. Figure~\ref{fig:unnamed-chunk-180} compares the rescaled bias by region and time periods.



% %%%%%%%%%%%%%%%%%%%%%%%%%%% Plot6
% << echo=FALSE, out.width = ".9\\textwidth", fig.width = 12, fig.height = 6, fig.cap = "Out-of-sample predictions along with direct estimates in the cross validation study where all data from each time period is held out and predicted using the rest of the data.">>=
% fig_count <- fig_count + 1
% knitr::include_graphics(paste0("../Main/Figures/CV_byYear_withError_", countryname2, ".pdf")) 
% @
 
%%%%%%%%%%%%%%%%%%%%%%%%%%% Plot7
\begin{knitrout}
\definecolor{shadecolor}{rgb}{0.969, 0.969, 0.969}\color{fgcolor}\begin{figure}[bht]

{\centering \includegraphics[width=.9\textwidth]{../Main/Figures/CV_byYearRegion_withError_Lesotho} 

}

\caption[Out-of-sample predictions along with direct estimates in the cross validation study where data from one region in each time period is held out and predicted using the rest of the data]{Out-of-sample predictions along with direct estimates in the cross validation study where data from one region in each time period is held out and predicted using the rest of the data.}\label{fig:unnamed-chunk-178}
\end{figure}


\end{knitrout}

%%%%%%%%%%%%%%%%%%%%%%%%%%% Plot8
\begin{knitrout}
\definecolor{shadecolor}{rgb}{0.969, 0.969, 0.969}\color{fgcolor}\begin{figure}[bht]

{\centering \includegraphics[width=.9\textwidth]{../Main/Figures/CVbiasLesotho} 

}

\caption[Histogram and QQ-plot of the rescaled difference between the smoothed estimates and the direct estimates in the cross validation study]{Histogram and QQ-plot of the rescaled difference between the smoothed estimates and the direct estimates in the cross validation study. The differences between the two estimates are rescaled by the square root of the total variance of the two estimates.}\label{fig:unnamed-chunk-179}
\end{figure}


\end{knitrout}

%%%%%%%%%%%%%%%%%%%%%%%%%%% Plot9
\begin{knitrout}
\definecolor{shadecolor}{rgb}{0.969, 0.969, 0.969}\color{fgcolor}\begin{figure}[bht]

{\centering \includegraphics[width=.7\textwidth]{../Main/Figures/CVbiasbyRegionLesotho} 

}

\caption[Line plot of the difference between smoothed estimates and the direct estimates in the cross validation study]{Line plot of the difference between smoothed estimates and the direct estimates in the cross validation study. The differences between the two estimates are rescaled by the square root of the total variance of the two estimates.}\label{fig:unnamed-chunk-180}
\end{figure}


\end{knitrout}


%%%%%%%%%%%%%%%%%%%%%%%%%%%%%%%%%%%%%%%%%%%%%%%%%%%%%%%%%%%%%%%%%%%%%%%%%%%%%%%%%%%%%%%%%%%%%%%%%%
\clearpage
\subsection{Liberia}


% \subsubsection{Summary of DHS surveys}

%%%%%%%%%%%%%%%%%%%%%%%%%%% Summary 


DHS surveys were conducted in Liberia in 2007, and 2013.
% years.out[1:(length(years.out)-1)], and years.out[length(years.out)]. 

We fit both the RW2 only model to the combined national data, and compare the time trend at national level with the estimates produced by the UN and IHME in Figure~\ref{fig:unnamed-chunk-182}. We then adjusted the combined national data to the UN estimates of U5MR, and refit the models on the benchmarked data. 

%%%%%%%%%%%%%%%%%%%%%%%%%% Plot5 
\begin{knitrout}
\definecolor{shadecolor}{rgb}{0.969, 0.969, 0.969}\color{fgcolor}\begin{figure}[bht]

{\centering \includegraphics[width=.9\textwidth]{../Main/Figures/Yearly_national_Liberia} 

}

\caption[Temporal national trends along with UN (B3) estimates described in You et al]{Temporal national trends along with UN (B3) estimates described in You et al. (2015) and IHME estimates based on GBD 2015 Child Mortality Collaborators (2016). RW2 represents the smoothed national estimates using the original data before benchmarking with UN estimates. RW2-adj represents the smoothed national estimates using the benchmarked data.}\label{fig:unnamed-chunk-182}
\end{figure}


\end{knitrout}
 

We fit the RW2 model to the benchmarked data in each area. 
% The proportions of the explained variation is summarized in Table~\ref{tab:paste0(countryname, "-var")}. 
We compare the results in Figure~\ref{fig:unnamed-chunk-183} to \ref{fig:unnamed-chunk-187}.
Figure~\ref{fig:unnamed-chunk-183} compares the smoothed estimates against the direct estimates. Figure~\ref{fig:unnamed-chunk-184} and Figure~\ref{fig:unnamed-chunk-185} show the posterior median estimates of U5MR in each region over time and the reductions from 1990 period respectively.
Figure~\ref{fig:unnamed-chunk-186} shows the smoothed estimates by region over time and Figure~\ref{fig:unnamed-chunk-187} compares the smoothed estimates with direct estimates from each survey for each region over time.


% %%%%%%%%%%%%%%%%%%%%%%%%%%% Table1 
% <<echo=FALSE, results='asis'>>=
% load("rda/variance_tables.rda")
% countryname2 <- gsub(" ", "", countryname)
% variance <- tables.all[[countryname]]

% table_count <- table_count + 1

% names <- c("RW2 ($\\sigma^2_{\\gamma_{t}}$)", "ICAR ($\\sigma^2_{\\phi_{i}}$)", "IID space ($\\sigma^2_{\\theta_{i}}$)", "IID time ($\\sigma^2_{\\alpha_{t}}$)", "IID space time ($\\sigma^2_{\\delta_{it}}$)")

% variance$Proportion <- round(variance$Proportion*100, digits = 2)
% row.names(variance) <- names
% tab <- xtable(variance, digits = c(1, 3, 2),align = "l|ll",
%        label = paste0("tab:", countryname, "-var"),
%        caption = paste(country, ": summary of the variance components in the RW2 model", sep = ''))
% print(tab, comment = FALSE,sanitize.text.function = function(x) {x})
% @

%%%%%%%%%%%%%%%%%%%%%%%%%%% Plot1 
\begin{knitrout}
\definecolor{shadecolor}{rgb}{0.969, 0.969, 0.969}\color{fgcolor}\begin{figure}[bht]

{\centering \includegraphics[width=.9\textwidth]{../Main/Figures/SmoothvDirectLiberia_meta} 

}

\caption[Smooth versus direct Admin 1 estimates]{Smooth versus direct Admin 1 estimates. Left: Combined (meta-analysis) survey estimate against combined direct estimates. Right: Combined (meta-analysis) survey estimate against direct estimates from each survey.}\label{fig:unnamed-chunk-183}
\end{figure}


\end{knitrout}

%%%%%%%%%%%%%%%%%%%%%%%%%%% Plot2 
\begin{knitrout}
\definecolor{shadecolor}{rgb}{0.969, 0.969, 0.969}\color{fgcolor}\begin{figure}[bht]

{\centering \includegraphics[width=.9\textwidth]{../Main/Figures/SmoothMedianLiberia} 

}

\caption[Maps of posterior medians for Liberia  over time]{Maps of posterior medians for Liberia  over time.}\label{fig:unnamed-chunk-184}
\end{figure}


\end{knitrout}
%%%%%%%%%%%%%%%%%%%%%%%%%%% Plot2a
\begin{knitrout}
\definecolor{shadecolor}{rgb}{0.969, 0.969, 0.969}\color{fgcolor}\begin{figure}[bht]

{\centering \includegraphics[width=.9\textwidth]{../Main/Figures/ReductionMedianLiberia} 

}

\caption[Maps of reduction of posterior median U5MR in each five-year period compared to 1990 in Liberia over time]{Maps of reduction of posterior median U5MR in each five-year period compared to 1990 in Liberia over time.}\label{fig:unnamed-chunk-185}
\end{figure}


\end{knitrout}
%%%%%%%%%%%%%%%%%%%%%%%%%%% Plot3 
\begin{knitrout}
\definecolor{shadecolor}{rgb}{0.969, 0.969, 0.969}\color{fgcolor}\begin{figure}[bht]

{\centering \includegraphics[width=.95\textwidth]{../Main/Figures/Yearly_v_Periods_Liberia} 

}

\caption[Smoothed regional estimates over time]{Smoothed regional estimates over time. The line indicates yearly posterior median estimates and error bars indicate 95 \% posterior credible interval at each time period.}\label{fig:unnamed-chunk-186}
\end{figure}


\end{knitrout}

%%%%%%%%%%%%%%%%%%%%%%%%%%% Plot4 
\begin{knitrout}
\definecolor{shadecolor}{rgb}{0.969, 0.969, 0.969}\color{fgcolor}\begin{figure}[bht]

{\centering \includegraphics[width=.9\textwidth]{../Main/Figures/LineSubMedianLiberia} 

}

\caption[Smoothed regional estimates over time compared to the direct estimates from each surveys]{Smoothed regional estimates over time compared to the direct estimates from each surveys. Direct estimates are not benchmarked with UN estimates. The line indicates posterior median and error bars indicate 95\% posterior credible interval.}\label{fig:unnamed-chunk-187}
\end{figure}


\end{knitrout}
% \subsubsection{National model results}
We further assess the RW2 model by holding out some observations, and compare the projections to the direct estimates in these holdout observations. Figure~\ref{fig:unnamed-chunk-188} compares the predicted estimates for the out-of-sample observations  with the direct estimates by holding out observations from each area in each time period.  Figure~\ref{fig:unnamed-chunk-189} compares the histogram of the bias rescaled by the total variance in the cross validation studies. Figure~\ref{fig:unnamed-chunk-190} compares the rescaled bias by region and time periods.



% %%%%%%%%%%%%%%%%%%%%%%%%%%% Plot6
% << echo=FALSE, out.width = ".9\\textwidth", fig.width = 12, fig.height = 6, fig.cap = "Out-of-sample predictions along with direct estimates in the cross validation study where all data from each time period is held out and predicted using the rest of the data.">>=
% fig_count <- fig_count + 1
% knitr::include_graphics(paste0("../Main/Figures/CV_byYear_withError_", countryname2, ".pdf")) 
% @
 
%%%%%%%%%%%%%%%%%%%%%%%%%%% Plot7
\begin{knitrout}
\definecolor{shadecolor}{rgb}{0.969, 0.969, 0.969}\color{fgcolor}\begin{figure}[bht]

{\centering \includegraphics[width=.9\textwidth]{../Main/Figures/CV_byYearRegion_withError_Liberia} 

}

\caption[Out-of-sample predictions along with direct estimates in the cross validation study where data from one region in each time period is held out and predicted using the rest of the data]{Out-of-sample predictions along with direct estimates in the cross validation study where data from one region in each time period is held out and predicted using the rest of the data.}\label{fig:unnamed-chunk-188}
\end{figure}


\end{knitrout}

%%%%%%%%%%%%%%%%%%%%%%%%%%% Plot8
\begin{knitrout}
\definecolor{shadecolor}{rgb}{0.969, 0.969, 0.969}\color{fgcolor}\begin{figure}[bht]

{\centering \includegraphics[width=.9\textwidth]{../Main/Figures/CVbiasLiberia} 

}

\caption[Histogram and QQ-plot of the rescaled difference between the smoothed estimates and the direct estimates in the cross validation study]{Histogram and QQ-plot of the rescaled difference between the smoothed estimates and the direct estimates in the cross validation study. The differences between the two estimates are rescaled by the square root of the total variance of the two estimates.}\label{fig:unnamed-chunk-189}
\end{figure}


\end{knitrout}

%%%%%%%%%%%%%%%%%%%%%%%%%%% Plot9
\begin{knitrout}
\definecolor{shadecolor}{rgb}{0.969, 0.969, 0.969}\color{fgcolor}\begin{figure}[bht]

{\centering \includegraphics[width=.7\textwidth]{../Main/Figures/CVbiasbyRegionLiberia} 

}

\caption[Line plot of the difference between smoothed estimates and the direct estimates in the cross validation study]{Line plot of the difference between smoothed estimates and the direct estimates in the cross validation study. The differences between the two estimates are rescaled by the square root of the total variance of the two estimates.}\label{fig:unnamed-chunk-190}
\end{figure}


\end{knitrout}


%%%%%%%%%%%%%%%%%%%%%%%%%%%%%%%%%%%%%%%%%%%%%%%%%%%%%%%%%%%%%%%%%%%%%%%%%%%%%%%%%%%%%%%%%%%%%%%%%%
\clearpage
\subsection{Madagascar}


% \subsubsection{Summary of DHS surveys}

%%%%%%%%%%%%%%%%%%%%%%%%%%% Summary 


DHS surveys were conducted in Madagascar in 1992, 1997, 2004, and 2009.
% years.out[1:(length(years.out)-1)], and years.out[length(years.out)]. 

We fit both the RW2 only model to the combined national data, and compare the time trend at national level with the estimates produced by the UN and IHME in Figure~\ref{fig:unnamed-chunk-192}. We then adjusted the combined national data to the UN estimates of U5MR, and refit the models on the benchmarked data. 

%%%%%%%%%%%%%%%%%%%%%%%%%% Plot5 
\begin{knitrout}
\definecolor{shadecolor}{rgb}{0.969, 0.969, 0.969}\color{fgcolor}\begin{figure}[bht]

{\centering \includegraphics[width=.9\textwidth]{../Main/Figures/Yearly_national_Madagascar} 

}

\caption[Temporal national trends along with UN (B3) estimates described in You et al]{Temporal national trends along with UN (B3) estimates described in You et al. (2015) and IHME estimates based on GBD 2015 Child Mortality Collaborators (2016). RW2 represents the smoothed national estimates using the original data before benchmarking with UN estimates. RW2-adj represents the smoothed national estimates using the benchmarked data.}\label{fig:unnamed-chunk-192}
\end{figure}


\end{knitrout}
 

We fit the RW2 model to the benchmarked data in each area. 
% The proportions of the explained variation is summarized in Table~\ref{tab:paste0(countryname, "-var")}. 
We compare the results in Figure~\ref{fig:unnamed-chunk-193} to \ref{fig:unnamed-chunk-197}.
Figure~\ref{fig:unnamed-chunk-193} compares the smoothed estimates against the direct estimates. Figure~\ref{fig:unnamed-chunk-194} and Figure~\ref{fig:unnamed-chunk-195} show the posterior median estimates of U5MR in each region over time and the reductions from 1990 period respectively.
Figure~\ref{fig:unnamed-chunk-196} shows the smoothed estimates by region over time and Figure~\ref{fig:unnamed-chunk-197} compares the smoothed estimates with direct estimates from each survey for each region over time.


% %%%%%%%%%%%%%%%%%%%%%%%%%%% Table1 
% <<echo=FALSE, results='asis'>>=
% load("rda/variance_tables.rda")
% countryname2 <- gsub(" ", "", countryname)
% variance <- tables.all[[countryname]]

% table_count <- table_count + 1

% names <- c("RW2 ($\\sigma^2_{\\gamma_{t}}$)", "ICAR ($\\sigma^2_{\\phi_{i}}$)", "IID space ($\\sigma^2_{\\theta_{i}}$)", "IID time ($\\sigma^2_{\\alpha_{t}}$)", "IID space time ($\\sigma^2_{\\delta_{it}}$)")

% variance$Proportion <- round(variance$Proportion*100, digits = 2)
% row.names(variance) <- names
% tab <- xtable(variance, digits = c(1, 3, 2),align = "l|ll",
%        label = paste0("tab:", countryname, "-var"),
%        caption = paste(country, ": summary of the variance components in the RW2 model", sep = ''))
% print(tab, comment = FALSE,sanitize.text.function = function(x) {x})
% @

%%%%%%%%%%%%%%%%%%%%%%%%%%% Plot1 
\begin{knitrout}
\definecolor{shadecolor}{rgb}{0.969, 0.969, 0.969}\color{fgcolor}\begin{figure}[bht]

{\centering \includegraphics[width=.9\textwidth]{../Main/Figures/SmoothvDirectMadagascar_meta} 

}

\caption[Smooth versus direct Admin 1 estimates]{Smooth versus direct Admin 1 estimates. Left: Combined (meta-analysis) survey estimate against combined direct estimates. Right: Combined (meta-analysis) survey estimate against direct estimates from each survey.}\label{fig:unnamed-chunk-193}
\end{figure}


\end{knitrout}

%%%%%%%%%%%%%%%%%%%%%%%%%%% Plot2 
\begin{knitrout}
\definecolor{shadecolor}{rgb}{0.969, 0.969, 0.969}\color{fgcolor}\begin{figure}[bht]

{\centering \includegraphics[width=.9\textwidth]{../Main/Figures/SmoothMedianMadagascar} 

}

\caption[Maps of posterior medians for Madagascar  over time]{Maps of posterior medians for Madagascar  over time.}\label{fig:unnamed-chunk-194}
\end{figure}


\end{knitrout}
%%%%%%%%%%%%%%%%%%%%%%%%%%% Plot2a
\begin{knitrout}
\definecolor{shadecolor}{rgb}{0.969, 0.969, 0.969}\color{fgcolor}\begin{figure}[bht]

{\centering \includegraphics[width=.9\textwidth]{../Main/Figures/ReductionMedianMadagascar} 

}

\caption[Maps of reduction of posterior median U5MR in each five-year period compared to 1990 in Madagascar over time]{Maps of reduction of posterior median U5MR in each five-year period compared to 1990 in Madagascar over time.}\label{fig:unnamed-chunk-195}
\end{figure}


\end{knitrout}
%%%%%%%%%%%%%%%%%%%%%%%%%%% Plot3 
\begin{knitrout}
\definecolor{shadecolor}{rgb}{0.969, 0.969, 0.969}\color{fgcolor}\begin{figure}[bht]

{\centering \includegraphics[width=.95\textwidth]{../Main/Figures/Yearly_v_Periods_Madagascar} 

}

\caption[Smoothed regional estimates over time]{Smoothed regional estimates over time. The line indicates yearly posterior median estimates and error bars indicate 95 \% posterior credible interval at each time period.}\label{fig:unnamed-chunk-196}
\end{figure}


\end{knitrout}

%%%%%%%%%%%%%%%%%%%%%%%%%%% Plot4 
\begin{knitrout}
\definecolor{shadecolor}{rgb}{0.969, 0.969, 0.969}\color{fgcolor}\begin{figure}[bht]

{\centering \includegraphics[width=.9\textwidth]{../Main/Figures/LineSubMedianMadagascar} 

}

\caption[Smoothed regional estimates over time compared to the direct estimates from each surveys]{Smoothed regional estimates over time compared to the direct estimates from each surveys. Direct estimates are not benchmarked with UN estimates. The line indicates posterior median and error bars indicate 95\% posterior credible interval.}\label{fig:unnamed-chunk-197}
\end{figure}


\end{knitrout}
% \subsubsection{National model results}
We further assess the RW2 model by holding out some observations, and compare the projections to the direct estimates in these holdout observations. Figure~\ref{fig:unnamed-chunk-198} compares the predicted estimates for the out-of-sample observations  with the direct estimates by holding out observations from each area in each time period.  Figure~\ref{fig:unnamed-chunk-199} compares the histogram of the bias rescaled by the total variance in the cross validation studies. Figure~\ref{fig:unnamed-chunk-200} compares the rescaled bias by region and time periods.



% %%%%%%%%%%%%%%%%%%%%%%%%%%% Plot6
% << echo=FALSE, out.width = ".9\\textwidth", fig.width = 12, fig.height = 6, fig.cap = "Out-of-sample predictions along with direct estimates in the cross validation study where all data from each time period is held out and predicted using the rest of the data.">>=
% fig_count <- fig_count + 1
% knitr::include_graphics(paste0("../Main/Figures/CV_byYear_withError_", countryname2, ".pdf")) 
% @
 
%%%%%%%%%%%%%%%%%%%%%%%%%%% Plot7
\begin{knitrout}
\definecolor{shadecolor}{rgb}{0.969, 0.969, 0.969}\color{fgcolor}\begin{figure}[bht]

{\centering \includegraphics[width=.9\textwidth]{../Main/Figures/CV_byYearRegion_withError_Madagascar} 

}

\caption[Out-of-sample predictions along with direct estimates in the cross validation study where data from one region in each time period is held out and predicted using the rest of the data]{Out-of-sample predictions along with direct estimates in the cross validation study where data from one region in each time period is held out and predicted using the rest of the data.}\label{fig:unnamed-chunk-198}
\end{figure}


\end{knitrout}

%%%%%%%%%%%%%%%%%%%%%%%%%%% Plot8
\begin{knitrout}
\definecolor{shadecolor}{rgb}{0.969, 0.969, 0.969}\color{fgcolor}\begin{figure}[bht]

{\centering \includegraphics[width=.9\textwidth]{../Main/Figures/CVbiasMadagascar} 

}

\caption[Histogram and QQ-plot of the rescaled difference between the smoothed estimates and the direct estimates in the cross validation study]{Histogram and QQ-plot of the rescaled difference between the smoothed estimates and the direct estimates in the cross validation study. The differences between the two estimates are rescaled by the square root of the total variance of the two estimates.}\label{fig:unnamed-chunk-199}
\end{figure}


\end{knitrout}

%%%%%%%%%%%%%%%%%%%%%%%%%%% Plot9
\begin{knitrout}
\definecolor{shadecolor}{rgb}{0.969, 0.969, 0.969}\color{fgcolor}\begin{figure}[bht]

{\centering \includegraphics[width=.7\textwidth]{../Main/Figures/CVbiasbyRegionMadagascar} 

}

\caption[Line plot of the difference between smoothed estimates and the direct estimates in the cross validation study]{Line plot of the difference between smoothed estimates and the direct estimates in the cross validation study. The differences between the two estimates are rescaled by the square root of the total variance of the two estimates.}\label{fig:unnamed-chunk-200}
\end{figure}


\end{knitrout}


%%%%%%%%%%%%%%%%%%%%%%%%%%%%%%%%%%%%%%%%%%%%%%%%%%%%%%%%%%%%%%%%%%%%%%%%%%%%%%%%%%%%%%%%%%%%%%%%%%
\clearpage
\subsection{Malawi}


% \subsubsection{Summary of DHS surveys}

%%%%%%%%%%%%%%%%%%%%%%%%%%% Summary 


DHS surveys were conducted in Malawi in 1992, 2000, 2004, 2010, and 2015.
% years.out[1:(length(years.out)-1)], and years.out[length(years.out)]. 

We fit both the RW2 only model to the combined national data, and compare the time trend at national level with the estimates produced by the UN and IHME in Figure~\ref{fig:unnamed-chunk-202}. We then adjusted the combined national data to the UN estimates of U5MR, and refit the models on the benchmarked data. 

%%%%%%%%%%%%%%%%%%%%%%%%%% Plot5 
\begin{knitrout}
\definecolor{shadecolor}{rgb}{0.969, 0.969, 0.969}\color{fgcolor}\begin{figure}[bht]

{\centering \includegraphics[width=.9\textwidth]{../Main/Figures/Yearly_national_Malawi} 

}

\caption[Temporal national trends along with UN (B3) estimates described in You et al]{Temporal national trends along with UN (B3) estimates described in You et al. (2015) and IHME estimates based on GBD 2015 Child Mortality Collaborators (2016). RW2 represents the smoothed national estimates using the original data before benchmarking with UN estimates. RW2-adj represents the smoothed national estimates using the benchmarked data.}\label{fig:unnamed-chunk-202}
\end{figure}


\end{knitrout}
 

We fit the RW2 model to the benchmarked data in each area. 
% The proportions of the explained variation is summarized in Table~\ref{tab:paste0(countryname, "-var")}. 
We compare the results in Figure~\ref{fig:unnamed-chunk-203} to \ref{fig:unnamed-chunk-207}.
Figure~\ref{fig:unnamed-chunk-203} compares the smoothed estimates against the direct estimates. Figure~\ref{fig:unnamed-chunk-204} and Figure~\ref{fig:unnamed-chunk-205} show the posterior median estimates of U5MR in each region over time and the reductions from 1990 period respectively.
Figure~\ref{fig:unnamed-chunk-206} shows the smoothed estimates by region over time and Figure~\ref{fig:unnamed-chunk-207} compares the smoothed estimates with direct estimates from each survey for each region over time.


% %%%%%%%%%%%%%%%%%%%%%%%%%%% Table1 
% <<echo=FALSE, results='asis'>>=
% load("rda/variance_tables.rda")
% countryname2 <- gsub(" ", "", countryname)
% variance <- tables.all[[countryname]]

% table_count <- table_count + 1

% names <- c("RW2 ($\\sigma^2_{\\gamma_{t}}$)", "ICAR ($\\sigma^2_{\\phi_{i}}$)", "IID space ($\\sigma^2_{\\theta_{i}}$)", "IID time ($\\sigma^2_{\\alpha_{t}}$)", "IID space time ($\\sigma^2_{\\delta_{it}}$)")

% variance$Proportion <- round(variance$Proportion*100, digits = 2)
% row.names(variance) <- names
% tab <- xtable(variance, digits = c(1, 3, 2),align = "l|ll",
%        label = paste0("tab:", countryname, "-var"),
%        caption = paste(country, ": summary of the variance components in the RW2 model", sep = ''))
% print(tab, comment = FALSE,sanitize.text.function = function(x) {x})
% @

%%%%%%%%%%%%%%%%%%%%%%%%%%% Plot1 
\begin{knitrout}
\definecolor{shadecolor}{rgb}{0.969, 0.969, 0.969}\color{fgcolor}\begin{figure}[bht]

{\centering \includegraphics[width=.9\textwidth]{../Main/Figures/SmoothvDirectMalawi_meta} 

}

\caption[Smooth versus direct Admin 1 estimates]{Smooth versus direct Admin 1 estimates. Left: Combined (meta-analysis) survey estimate against combined direct estimates. Right: Combined (meta-analysis) survey estimate against direct estimates from each survey.}\label{fig:unnamed-chunk-203}
\end{figure}


\end{knitrout}

%%%%%%%%%%%%%%%%%%%%%%%%%%% Plot2 
\begin{knitrout}
\definecolor{shadecolor}{rgb}{0.969, 0.969, 0.969}\color{fgcolor}\begin{figure}[bht]

{\centering \includegraphics[width=.9\textwidth]{../Main/Figures/SmoothMedianMalawi} 

}

\caption[Maps of posterior medians for Malawi  over time]{Maps of posterior medians for Malawi  over time.}\label{fig:unnamed-chunk-204}
\end{figure}


\end{knitrout}
%%%%%%%%%%%%%%%%%%%%%%%%%%% Plot2a
\begin{knitrout}
\definecolor{shadecolor}{rgb}{0.969, 0.969, 0.969}\color{fgcolor}\begin{figure}[bht]

{\centering \includegraphics[width=.9\textwidth]{../Main/Figures/ReductionMedianMalawi} 

}

\caption[Maps of reduction of posterior median U5MR in each five-year period compared to 1990 in Malawi over time]{Maps of reduction of posterior median U5MR in each five-year period compared to 1990 in Malawi over time.}\label{fig:unnamed-chunk-205}
\end{figure}


\end{knitrout}
%%%%%%%%%%%%%%%%%%%%%%%%%%% Plot3 
\begin{knitrout}
\definecolor{shadecolor}{rgb}{0.969, 0.969, 0.969}\color{fgcolor}\begin{figure}[bht]

{\centering \includegraphics[width=.95\textwidth]{../Main/Figures/Yearly_v_Periods_Malawi} 

}

\caption[Smoothed regional estimates over time]{Smoothed regional estimates over time. The line indicates yearly posterior median estimates and error bars indicate 95 \% posterior credible interval at each time period.}\label{fig:unnamed-chunk-206}
\end{figure}


\end{knitrout}

%%%%%%%%%%%%%%%%%%%%%%%%%%% Plot4 
\begin{knitrout}
\definecolor{shadecolor}{rgb}{0.969, 0.969, 0.969}\color{fgcolor}\begin{figure}[bht]

{\centering \includegraphics[width=.9\textwidth]{../Main/Figures/LineSubMedianMalawi} 

}

\caption[Smoothed regional estimates over time compared to the direct estimates from each surveys]{Smoothed regional estimates over time compared to the direct estimates from each surveys. Direct estimates are not benchmarked with UN estimates. The line indicates posterior median and error bars indicate 95\% posterior credible interval.}\label{fig:unnamed-chunk-207}
\end{figure}


\end{knitrout}
% \subsubsection{National model results}
We further assess the RW2 model by holding out some observations, and compare the projections to the direct estimates in these holdout observations. Figure~\ref{fig:unnamed-chunk-208} compares the predicted estimates for the out-of-sample observations  with the direct estimates by holding out observations from each area in each time period.  Figure~\ref{fig:unnamed-chunk-209} compares the histogram of the bias rescaled by the total variance in the cross validation studies. Figure~\ref{fig:unnamed-chunk-210} compares the rescaled bias by region and time periods.



% %%%%%%%%%%%%%%%%%%%%%%%%%%% Plot6
% << echo=FALSE, out.width = ".9\\textwidth", fig.width = 12, fig.height = 6, fig.cap = "Out-of-sample predictions along with direct estimates in the cross validation study where all data from each time period is held out and predicted using the rest of the data.">>=
% fig_count <- fig_count + 1
% knitr::include_graphics(paste0("../Main/Figures/CV_byYear_withError_", countryname2, ".pdf")) 
% @
 
%%%%%%%%%%%%%%%%%%%%%%%%%%% Plot7
\begin{knitrout}
\definecolor{shadecolor}{rgb}{0.969, 0.969, 0.969}\color{fgcolor}\begin{figure}[bht]

{\centering \includegraphics[width=.9\textwidth]{../Main/Figures/CV_byYearRegion_withError_Malawi} 

}

\caption[Out-of-sample predictions along with direct estimates in the cross validation study where data from one region in each time period is held out and predicted using the rest of the data]{Out-of-sample predictions along with direct estimates in the cross validation study where data from one region in each time period is held out and predicted using the rest of the data.}\label{fig:unnamed-chunk-208}
\end{figure}


\end{knitrout}

%%%%%%%%%%%%%%%%%%%%%%%%%%% Plot8
\begin{knitrout}
\definecolor{shadecolor}{rgb}{0.969, 0.969, 0.969}\color{fgcolor}\begin{figure}[bht]

{\centering \includegraphics[width=.9\textwidth]{../Main/Figures/CVbiasMalawi} 

}

\caption[Histogram and QQ-plot of the rescaled difference between the smoothed estimates and the direct estimates in the cross validation study]{Histogram and QQ-plot of the rescaled difference between the smoothed estimates and the direct estimates in the cross validation study. The differences between the two estimates are rescaled by the square root of the total variance of the two estimates.}\label{fig:unnamed-chunk-209}
\end{figure}


\end{knitrout}

%%%%%%%%%%%%%%%%%%%%%%%%%%% Plot9
\begin{knitrout}
\definecolor{shadecolor}{rgb}{0.969, 0.969, 0.969}\color{fgcolor}\begin{figure}[bht]

{\centering \includegraphics[width=.7\textwidth]{../Main/Figures/CVbiasbyRegionMalawi} 

}

\caption[Line plot of the difference between smoothed estimates and the direct estimates in the cross validation study]{Line plot of the difference between smoothed estimates and the direct estimates in the cross validation study. The differences between the two estimates are rescaled by the square root of the total variance of the two estimates.}\label{fig:unnamed-chunk-210}
\end{figure}


\end{knitrout}


%%%%%%%%%%%%%%%%%%%%%%%%%%%%%%%%%%%%%%%%%%%%%%%%%%%%%%%%%%%%%%%%%%%%%%%%%%%%%%%%%%%%%%%%%%%%%%%%%%
\clearpage
\subsection{Mali}


% \subsubsection{Summary of DHS surveys}

%%%%%%%%%%%%%%%%%%%%%%%%%%% Summary 


DHS surveys were conducted in Mali in 1987, 1995, 2001, and 2006.
% years.out[1:(length(years.out)-1)], and years.out[length(years.out)]. 

We fit both the RW2 only model to the combined national data, and compare the time trend at national level with the estimates produced by the UN and IHME in Figure~\ref{fig:unnamed-chunk-212}. We then adjusted the combined national data to the UN estimates of U5MR, and refit the models on the benchmarked data. 

%%%%%%%%%%%%%%%%%%%%%%%%%% Plot5 
\begin{knitrout}
\definecolor{shadecolor}{rgb}{0.969, 0.969, 0.969}\color{fgcolor}\begin{figure}[bht]

{\centering \includegraphics[width=.9\textwidth]{../Main/Figures/Yearly_national_Mali} 

}

\caption[Temporal national trends along with UN (B3) estimates described in You et al]{Temporal national trends along with UN (B3) estimates described in You et al. (2015) and IHME estimates based on GBD 2015 Child Mortality Collaborators (2016). RW2 represents the smoothed national estimates using the original data before benchmarking with UN estimates. RW2-adj represents the smoothed national estimates using the benchmarked data.}\label{fig:unnamed-chunk-212}
\end{figure}


\end{knitrout}
 

We fit the RW2 model to the benchmarked data in each area. 
% The proportions of the explained variation is summarized in Table~\ref{tab:paste0(countryname, "-var")}. 
We compare the results in Figure~\ref{fig:unnamed-chunk-213} to \ref{fig:unnamed-chunk-217}.
Figure~\ref{fig:unnamed-chunk-213} compares the smoothed estimates against the direct estimates. Figure~\ref{fig:unnamed-chunk-214} and Figure~\ref{fig:unnamed-chunk-215} show the posterior median estimates of U5MR in each region over time and the reductions from 1990 period respectively.
Figure~\ref{fig:unnamed-chunk-216} shows the smoothed estimates by region over time and Figure~\ref{fig:unnamed-chunk-217} compares the smoothed estimates with direct estimates from each survey for each region over time.


% %%%%%%%%%%%%%%%%%%%%%%%%%%% Table1 
% <<echo=FALSE, results='asis'>>=
% load("rda/variance_tables.rda")
% countryname2 <- gsub(" ", "", countryname)
% variance <- tables.all[[countryname]]

% table_count <- table_count + 1

% names <- c("RW2 ($\\sigma^2_{\\gamma_{t}}$)", "ICAR ($\\sigma^2_{\\phi_{i}}$)", "IID space ($\\sigma^2_{\\theta_{i}}$)", "IID time ($\\sigma^2_{\\alpha_{t}}$)", "IID space time ($\\sigma^2_{\\delta_{it}}$)")

% variance$Proportion <- round(variance$Proportion*100, digits = 2)
% row.names(variance) <- names
% tab <- xtable(variance, digits = c(1, 3, 2),align = "l|ll",
%        label = paste0("tab:", countryname, "-var"),
%        caption = paste(country, ": summary of the variance components in the RW2 model", sep = ''))
% print(tab, comment = FALSE,sanitize.text.function = function(x) {x})
% @

%%%%%%%%%%%%%%%%%%%%%%%%%%% Plot1 
\begin{knitrout}
\definecolor{shadecolor}{rgb}{0.969, 0.969, 0.969}\color{fgcolor}\begin{figure}[bht]

{\centering \includegraphics[width=.9\textwidth]{../Main/Figures/SmoothvDirectMali_meta} 

}

\caption[Smooth versus direct Admin 1 estimates]{Smooth versus direct Admin 1 estimates. Left: Combined (meta-analysis) survey estimate against combined direct estimates. Right: Combined (meta-analysis) survey estimate against direct estimates from each survey.}\label{fig:unnamed-chunk-213}
\end{figure}


\end{knitrout}

%%%%%%%%%%%%%%%%%%%%%%%%%%% Plot2 
\begin{knitrout}
\definecolor{shadecolor}{rgb}{0.969, 0.969, 0.969}\color{fgcolor}\begin{figure}[bht]

{\centering \includegraphics[width=.9\textwidth]{../Main/Figures/SmoothMedianMali} 

}

\caption[Maps of posterior medians for Mali  over time]{Maps of posterior medians for Mali  over time.}\label{fig:unnamed-chunk-214}
\end{figure}


\end{knitrout}
%%%%%%%%%%%%%%%%%%%%%%%%%%% Plot2a
\begin{knitrout}
\definecolor{shadecolor}{rgb}{0.969, 0.969, 0.969}\color{fgcolor}\begin{figure}[bht]

{\centering \includegraphics[width=.9\textwidth]{../Main/Figures/ReductionMedianMali} 

}

\caption[Maps of reduction of posterior median U5MR in each five-year period compared to 1990 in Mali over time]{Maps of reduction of posterior median U5MR in each five-year period compared to 1990 in Mali over time.}\label{fig:unnamed-chunk-215}
\end{figure}


\end{knitrout}
%%%%%%%%%%%%%%%%%%%%%%%%%%% Plot3 
\begin{knitrout}
\definecolor{shadecolor}{rgb}{0.969, 0.969, 0.969}\color{fgcolor}\begin{figure}[bht]

{\centering \includegraphics[width=.95\textwidth]{../Main/Figures/Yearly_v_Periods_Mali} 

}

\caption[Smoothed regional estimates over time]{Smoothed regional estimates over time. The line indicates yearly posterior median estimates and error bars indicate 95 \% posterior credible interval at each time period.}\label{fig:unnamed-chunk-216}
\end{figure}


\end{knitrout}

%%%%%%%%%%%%%%%%%%%%%%%%%%% Plot4 
\begin{knitrout}
\definecolor{shadecolor}{rgb}{0.969, 0.969, 0.969}\color{fgcolor}\begin{figure}[bht]

{\centering \includegraphics[width=.9\textwidth]{../Main/Figures/LineSubMedianMali} 

}

\caption[Smoothed regional estimates over time compared to the direct estimates from each surveys]{Smoothed regional estimates over time compared to the direct estimates from each surveys. Direct estimates are not benchmarked with UN estimates. The line indicates posterior median and error bars indicate 95\% posterior credible interval.}\label{fig:unnamed-chunk-217}
\end{figure}


\end{knitrout}
% \subsubsection{National model results}
We further assess the RW2 model by holding out some observations, and compare the projections to the direct estimates in these holdout observations. Figure~\ref{fig:unnamed-chunk-218} compares the predicted estimates for the out-of-sample observations  with the direct estimates by holding out observations from each area in each time period.  Figure~\ref{fig:unnamed-chunk-219} compares the histogram of the bias rescaled by the total variance in the cross validation studies. Figure~\ref{fig:unnamed-chunk-220} compares the rescaled bias by region and time periods.



% %%%%%%%%%%%%%%%%%%%%%%%%%%% Plot6
% << echo=FALSE, out.width = ".9\\textwidth", fig.width = 12, fig.height = 6, fig.cap = "Out-of-sample predictions along with direct estimates in the cross validation study where all data from each time period is held out and predicted using the rest of the data.">>=
% fig_count <- fig_count + 1
% knitr::include_graphics(paste0("../Main/Figures/CV_byYear_withError_", countryname2, ".pdf")) 
% @
 
%%%%%%%%%%%%%%%%%%%%%%%%%%% Plot7
\begin{knitrout}
\definecolor{shadecolor}{rgb}{0.969, 0.969, 0.969}\color{fgcolor}\begin{figure}[bht]

{\centering \includegraphics[width=.9\textwidth]{../Main/Figures/CV_byYearRegion_withError_Mali} 

}

\caption[Out-of-sample predictions along with direct estimates in the cross validation study where data from one region in each time period is held out and predicted using the rest of the data]{Out-of-sample predictions along with direct estimates in the cross validation study where data from one region in each time period is held out and predicted using the rest of the data.}\label{fig:unnamed-chunk-218}
\end{figure}


\end{knitrout}

%%%%%%%%%%%%%%%%%%%%%%%%%%% Plot8
\begin{knitrout}
\definecolor{shadecolor}{rgb}{0.969, 0.969, 0.969}\color{fgcolor}\begin{figure}[bht]

{\centering \includegraphics[width=.9\textwidth]{../Main/Figures/CVbiasMali} 

}

\caption[Histogram and QQ-plot of the rescaled difference between the smoothed estimates and the direct estimates in the cross validation study]{Histogram and QQ-plot of the rescaled difference between the smoothed estimates and the direct estimates in the cross validation study. The differences between the two estimates are rescaled by the square root of the total variance of the two estimates.}\label{fig:unnamed-chunk-219}
\end{figure}


\end{knitrout}

%%%%%%%%%%%%%%%%%%%%%%%%%%% Plot9
\begin{knitrout}
\definecolor{shadecolor}{rgb}{0.969, 0.969, 0.969}\color{fgcolor}\begin{figure}[bht]

{\centering \includegraphics[width=.7\textwidth]{../Main/Figures/CVbiasbyRegionMali} 

}

\caption[Line plot of the difference between smoothed estimates and the direct estimates in the cross validation study]{Line plot of the difference between smoothed estimates and the direct estimates in the cross validation study. The differences between the two estimates are rescaled by the square root of the total variance of the two estimates.}\label{fig:unnamed-chunk-220}
\end{figure}


\end{knitrout}


%%%%%%%%%%%%%%%%%%%%%%%%%%%%%%%%%%%%%%%%%%%%%%%%%%%%%%%%%%%%%%%%%%%%%%%%%%%%%%%%%%%%%%%%%%%%%%%%%%
\clearpage
\subsection{Morocco}


% \subsubsection{Summary of DHS surveys}

%%%%%%%%%%%%%%%%%%%%%%%%%%% Summary 


DHS surveys were conducted in Morocco in 1987, 1992, and 2003.
% years.out[1:(length(years.out)-1)], and years.out[length(years.out)]. 

We fit both the RW2 only model to the combined national data, and compare the time trend at national level with the estimates produced by the UN and IHME in Figure~\ref{fig:unnamed-chunk-222}. We then adjusted the combined national data to the UN estimates of U5MR, and refit the models on the benchmarked data. 

%%%%%%%%%%%%%%%%%%%%%%%%%% Plot5 
\begin{knitrout}
\definecolor{shadecolor}{rgb}{0.969, 0.969, 0.969}\color{fgcolor}\begin{figure}[bht]

{\centering \includegraphics[width=.9\textwidth]{../Main/Figures/Yearly_national_Morocco} 

}

\caption[Temporal national trends along with UN (B3) estimates described in You et al]{Temporal national trends along with UN (B3) estimates described in You et al. (2015) and IHME estimates based on GBD 2015 Child Mortality Collaborators (2016). RW2 represents the smoothed national estimates using the original data before benchmarking with UN estimates. RW2-adj represents the smoothed national estimates using the benchmarked data.}\label{fig:unnamed-chunk-222}
\end{figure}


\end{knitrout}
 

We fit the RW2 model to the benchmarked data in each area. 
% The proportions of the explained variation is summarized in Table~\ref{tab:paste0(countryname, "-var")}. 
We compare the results in Figure~\ref{fig:unnamed-chunk-223} to \ref{fig:unnamed-chunk-227}.
Figure~\ref{fig:unnamed-chunk-223} compares the smoothed estimates against the direct estimates. Figure~\ref{fig:unnamed-chunk-224} and Figure~\ref{fig:unnamed-chunk-225} show the posterior median estimates of U5MR in each region over time and the reductions from 1990 period respectively.
Figure~\ref{fig:unnamed-chunk-226} shows the smoothed estimates by region over time and Figure~\ref{fig:unnamed-chunk-227} compares the smoothed estimates with direct estimates from each survey for each region over time.


% %%%%%%%%%%%%%%%%%%%%%%%%%%% Table1 
% <<echo=FALSE, results='asis'>>=
% load("rda/variance_tables.rda")
% countryname2 <- gsub(" ", "", countryname)
% variance <- tables.all[[countryname]]

% table_count <- table_count + 1

% names <- c("RW2 ($\\sigma^2_{\\gamma_{t}}$)", "ICAR ($\\sigma^2_{\\phi_{i}}$)", "IID space ($\\sigma^2_{\\theta_{i}}$)", "IID time ($\\sigma^2_{\\alpha_{t}}$)", "IID space time ($\\sigma^2_{\\delta_{it}}$)")

% variance$Proportion <- round(variance$Proportion*100, digits = 2)
% row.names(variance) <- names
% tab <- xtable(variance, digits = c(1, 3, 2),align = "l|ll",
%        label = paste0("tab:", countryname, "-var"),
%        caption = paste(country, ": summary of the variance components in the RW2 model", sep = ''))
% print(tab, comment = FALSE,sanitize.text.function = function(x) {x})
% @

%%%%%%%%%%%%%%%%%%%%%%%%%%% Plot1 
\begin{knitrout}
\definecolor{shadecolor}{rgb}{0.969, 0.969, 0.969}\color{fgcolor}\begin{figure}[bht]

{\centering \includegraphics[width=.9\textwidth]{../Main/Figures/SmoothvDirectMorocco_meta} 

}

\caption[Smooth versus direct Admin 1 estimates]{Smooth versus direct Admin 1 estimates. Left: Combined (meta-analysis) survey estimate against combined direct estimates. Right: Combined (meta-analysis) survey estimate against direct estimates from each survey.}\label{fig:unnamed-chunk-223}
\end{figure}


\end{knitrout}

%%%%%%%%%%%%%%%%%%%%%%%%%%% Plot2 
\begin{knitrout}
\definecolor{shadecolor}{rgb}{0.969, 0.969, 0.969}\color{fgcolor}\begin{figure}[bht]

{\centering \includegraphics[width=.9\textwidth]{../Main/Figures/SmoothMedianMorocco} 

}

\caption[Maps of posterior medians for Morocco  over time]{Maps of posterior medians for Morocco  over time.}\label{fig:unnamed-chunk-224}
\end{figure}


\end{knitrout}
%%%%%%%%%%%%%%%%%%%%%%%%%%% Plot2a
\begin{knitrout}
\definecolor{shadecolor}{rgb}{0.969, 0.969, 0.969}\color{fgcolor}\begin{figure}[bht]

{\centering \includegraphics[width=.9\textwidth]{../Main/Figures/ReductionMedianMorocco} 

}

\caption[Maps of reduction of posterior median U5MR in each five-year period compared to 1990 in Morocco over time]{Maps of reduction of posterior median U5MR in each five-year period compared to 1990 in Morocco over time.}\label{fig:unnamed-chunk-225}
\end{figure}


\end{knitrout}
%%%%%%%%%%%%%%%%%%%%%%%%%%% Plot3 
\begin{knitrout}
\definecolor{shadecolor}{rgb}{0.969, 0.969, 0.969}\color{fgcolor}\begin{figure}[bht]

{\centering \includegraphics[width=.95\textwidth]{../Main/Figures/Yearly_v_Periods_Morocco} 

}

\caption[Smoothed regional estimates over time]{Smoothed regional estimates over time. The line indicates yearly posterior median estimates and error bars indicate 95 \% posterior credible interval at each time period.}\label{fig:unnamed-chunk-226}
\end{figure}


\end{knitrout}

%%%%%%%%%%%%%%%%%%%%%%%%%%% Plot4 
\begin{knitrout}
\definecolor{shadecolor}{rgb}{0.969, 0.969, 0.969}\color{fgcolor}\begin{figure}[bht]

{\centering \includegraphics[width=.9\textwidth]{../Main/Figures/LineSubMedianMorocco} 

}

\caption[Smoothed regional estimates over time compared to the direct estimates from each surveys]{Smoothed regional estimates over time compared to the direct estimates from each surveys. Direct estimates are not benchmarked with UN estimates. The line indicates posterior median and error bars indicate 95\% posterior credible interval.}\label{fig:unnamed-chunk-227}
\end{figure}


\end{knitrout}
% \subsubsection{National model results}
We further assess the RW2 model by holding out some observations, and compare the projections to the direct estimates in these holdout observations. Figure~\ref{fig:unnamed-chunk-228} compares the predicted estimates for the out-of-sample observations  with the direct estimates by holding out observations from each area in each time period.  Figure~\ref{fig:unnamed-chunk-229} compares the histogram of the bias rescaled by the total variance in the cross validation studies. Figure~\ref{fig:unnamed-chunk-230} compares the rescaled bias by region and time periods.



% %%%%%%%%%%%%%%%%%%%%%%%%%%% Plot6
% << echo=FALSE, out.width = ".9\\textwidth", fig.width = 12, fig.height = 6, fig.cap = "Out-of-sample predictions along with direct estimates in the cross validation study where all data from each time period is held out and predicted using the rest of the data.">>=
% fig_count <- fig_count + 1
% knitr::include_graphics(paste0("../Main/Figures/CV_byYear_withError_", countryname2, ".pdf")) 
% @
 
%%%%%%%%%%%%%%%%%%%%%%%%%%% Plot7
\begin{knitrout}
\definecolor{shadecolor}{rgb}{0.969, 0.969, 0.969}\color{fgcolor}\begin{figure}[bht]

{\centering \includegraphics[width=.9\textwidth]{../Main/Figures/CV_byYearRegion_withError_Morocco} 

}

\caption[Out-of-sample predictions along with direct estimates in the cross validation study where data from one region in each time period is held out and predicted using the rest of the data]{Out-of-sample predictions along with direct estimates in the cross validation study where data from one region in each time period is held out and predicted using the rest of the data.}\label{fig:unnamed-chunk-228}
\end{figure}


\end{knitrout}

%%%%%%%%%%%%%%%%%%%%%%%%%%% Plot8
\begin{knitrout}
\definecolor{shadecolor}{rgb}{0.969, 0.969, 0.969}\color{fgcolor}\begin{figure}[bht]

{\centering \includegraphics[width=.9\textwidth]{../Main/Figures/CVbiasMorocco} 

}

\caption[Histogram and QQ-plot of the rescaled difference between the smoothed estimates and the direct estimates in the cross validation study]{Histogram and QQ-plot of the rescaled difference between the smoothed estimates and the direct estimates in the cross validation study. The differences between the two estimates are rescaled by the square root of the total variance of the two estimates.}\label{fig:unnamed-chunk-229}
\end{figure}


\end{knitrout}

%%%%%%%%%%%%%%%%%%%%%%%%%%% Plot9
\begin{knitrout}
\definecolor{shadecolor}{rgb}{0.969, 0.969, 0.969}\color{fgcolor}\begin{figure}[bht]

{\centering \includegraphics[width=.7\textwidth]{../Main/Figures/CVbiasbyRegionMorocco} 

}

\caption[Line plot of the difference between smoothed estimates and the direct estimates in the cross validation study]{Line plot of the difference between smoothed estimates and the direct estimates in the cross validation study. The differences between the two estimates are rescaled by the square root of the total variance of the two estimates.}\label{fig:unnamed-chunk-230}
\end{figure}


\end{knitrout}


%%%%%%%%%%%%%%%%%%%%%%%%%%%%%%%%%%%%%%%%%%%%%%%%%%%%%%%%%%%%%%%%%%%%%%%%%%%%%%%%%%%%%%%%%%%%%%%%%%
\clearpage
\subsection{Mozambique}


% \subsubsection{Summary of DHS surveys}

%%%%%%%%%%%%%%%%%%%%%%%%%%% Summary 


DHS surveys were conducted in Mozambique in 2003, and 2011.
% years.out[1:(length(years.out)-1)], and years.out[length(years.out)]. 

We fit both the RW2 only model to the combined national data, and compare the time trend at national level with the estimates produced by the UN and IHME in Figure~\ref{fig:unnamed-chunk-232}. We then adjusted the combined national data to the UN estimates of U5MR, and refit the models on the benchmarked data. 

%%%%%%%%%%%%%%%%%%%%%%%%%% Plot5 
\begin{knitrout}
\definecolor{shadecolor}{rgb}{0.969, 0.969, 0.969}\color{fgcolor}\begin{figure}[bht]

{\centering \includegraphics[width=.9\textwidth]{../Main/Figures/Yearly_national_Mozambique} 

}

\caption[Temporal national trends along with UN (B3) estimates described in You et al]{Temporal national trends along with UN (B3) estimates described in You et al. (2015) and IHME estimates based on GBD 2015 Child Mortality Collaborators (2016). RW2 represents the smoothed national estimates using the original data before benchmarking with UN estimates. RW2-adj represents the smoothed national estimates using the benchmarked data.}\label{fig:unnamed-chunk-232}
\end{figure}


\end{knitrout}
 

We fit the RW2 model to the benchmarked data in each area. 
% The proportions of the explained variation is summarized in Table~\ref{tab:paste0(countryname, "-var")}. 
We compare the results in Figure~\ref{fig:unnamed-chunk-233} to \ref{fig:unnamed-chunk-237}.
Figure~\ref{fig:unnamed-chunk-233} compares the smoothed estimates against the direct estimates. Figure~\ref{fig:unnamed-chunk-234} and Figure~\ref{fig:unnamed-chunk-235} show the posterior median estimates of U5MR in each region over time and the reductions from 1990 period respectively.
Figure~\ref{fig:unnamed-chunk-236} shows the smoothed estimates by region over time and Figure~\ref{fig:unnamed-chunk-237} compares the smoothed estimates with direct estimates from each survey for each region over time.


% %%%%%%%%%%%%%%%%%%%%%%%%%%% Table1 
% <<echo=FALSE, results='asis'>>=
% load("rda/variance_tables.rda")
% countryname2 <- gsub(" ", "", countryname)
% variance <- tables.all[[countryname]]

% table_count <- table_count + 1

% names <- c("RW2 ($\\sigma^2_{\\gamma_{t}}$)", "ICAR ($\\sigma^2_{\\phi_{i}}$)", "IID space ($\\sigma^2_{\\theta_{i}}$)", "IID time ($\\sigma^2_{\\alpha_{t}}$)", "IID space time ($\\sigma^2_{\\delta_{it}}$)")

% variance$Proportion <- round(variance$Proportion*100, digits = 2)
% row.names(variance) <- names
% tab <- xtable(variance, digits = c(1, 3, 2),align = "l|ll",
%        label = paste0("tab:", countryname, "-var"),
%        caption = paste(country, ": summary of the variance components in the RW2 model", sep = ''))
% print(tab, comment = FALSE,sanitize.text.function = function(x) {x})
% @

%%%%%%%%%%%%%%%%%%%%%%%%%%% Plot1 
\begin{knitrout}
\definecolor{shadecolor}{rgb}{0.969, 0.969, 0.969}\color{fgcolor}\begin{figure}[bht]

{\centering \includegraphics[width=.9\textwidth]{../Main/Figures/SmoothvDirectMozambique_meta} 

}

\caption[Smooth versus direct Admin 1 estimates]{Smooth versus direct Admin 1 estimates. Left: Combined (meta-analysis) survey estimate against combined direct estimates. Right: Combined (meta-analysis) survey estimate against direct estimates from each survey.}\label{fig:unnamed-chunk-233}
\end{figure}


\end{knitrout}

%%%%%%%%%%%%%%%%%%%%%%%%%%% Plot2 
\begin{knitrout}
\definecolor{shadecolor}{rgb}{0.969, 0.969, 0.969}\color{fgcolor}\begin{figure}[bht]

{\centering \includegraphics[width=.9\textwidth]{../Main/Figures/SmoothMedianMozambique} 

}

\caption[Maps of posterior medians for Mozambique  over time]{Maps of posterior medians for Mozambique  over time.}\label{fig:unnamed-chunk-234}
\end{figure}


\end{knitrout}
%%%%%%%%%%%%%%%%%%%%%%%%%%% Plot2a
\begin{knitrout}
\definecolor{shadecolor}{rgb}{0.969, 0.969, 0.969}\color{fgcolor}\begin{figure}[bht]

{\centering \includegraphics[width=.9\textwidth]{../Main/Figures/ReductionMedianMozambique} 

}

\caption[Maps of reduction of posterior median U5MR in each five-year period compared to 1990 in Mozambique over time]{Maps of reduction of posterior median U5MR in each five-year period compared to 1990 in Mozambique over time.}\label{fig:unnamed-chunk-235}
\end{figure}


\end{knitrout}
%%%%%%%%%%%%%%%%%%%%%%%%%%% Plot3 
\begin{knitrout}
\definecolor{shadecolor}{rgb}{0.969, 0.969, 0.969}\color{fgcolor}\begin{figure}[bht]

{\centering \includegraphics[width=.95\textwidth]{../Main/Figures/Yearly_v_Periods_Mozambique} 

}

\caption[Smoothed regional estimates over time]{Smoothed regional estimates over time. The line indicates yearly posterior median estimates and error bars indicate 95 \% posterior credible interval at each time period.}\label{fig:unnamed-chunk-236}
\end{figure}


\end{knitrout}

%%%%%%%%%%%%%%%%%%%%%%%%%%% Plot4 
\begin{knitrout}
\definecolor{shadecolor}{rgb}{0.969, 0.969, 0.969}\color{fgcolor}\begin{figure}[bht]

{\centering \includegraphics[width=.9\textwidth]{../Main/Figures/LineSubMedianMozambique} 

}

\caption[Smoothed regional estimates over time compared to the direct estimates from each surveys]{Smoothed regional estimates over time compared to the direct estimates from each surveys. Direct estimates are not benchmarked with UN estimates. The line indicates posterior median and error bars indicate 95\% posterior credible interval.}\label{fig:unnamed-chunk-237}
\end{figure}


\end{knitrout}
% \subsubsection{National model results}
We further assess the RW2 model by holding out some observations, and compare the projections to the direct estimates in these holdout observations. Figure~\ref{fig:unnamed-chunk-238} compares the predicted estimates for the out-of-sample observations  with the direct estimates by holding out observations from each area in each time period.  Figure~\ref{fig:unnamed-chunk-239} compares the histogram of the bias rescaled by the total variance in the cross validation studies. Figure~\ref{fig:unnamed-chunk-240} compares the rescaled bias by region and time periods.



% %%%%%%%%%%%%%%%%%%%%%%%%%%% Plot6
% << echo=FALSE, out.width = ".9\\textwidth", fig.width = 12, fig.height = 6, fig.cap = "Out-of-sample predictions along with direct estimates in the cross validation study where all data from each time period is held out and predicted using the rest of the data.">>=
% fig_count <- fig_count + 1
% knitr::include_graphics(paste0("../Main/Figures/CV_byYear_withError_", countryname2, ".pdf")) 
% @
 
%%%%%%%%%%%%%%%%%%%%%%%%%%% Plot7
\begin{knitrout}
\definecolor{shadecolor}{rgb}{0.969, 0.969, 0.969}\color{fgcolor}\begin{figure}[bht]

{\centering \includegraphics[width=.9\textwidth]{../Main/Figures/CV_byYearRegion_withError_Mozambique} 

}

\caption[Out-of-sample predictions along with direct estimates in the cross validation study where data from one region in each time period is held out and predicted using the rest of the data]{Out-of-sample predictions along with direct estimates in the cross validation study where data from one region in each time period is held out and predicted using the rest of the data.}\label{fig:unnamed-chunk-238}
\end{figure}


\end{knitrout}

%%%%%%%%%%%%%%%%%%%%%%%%%%% Plot8
\begin{knitrout}
\definecolor{shadecolor}{rgb}{0.969, 0.969, 0.969}\color{fgcolor}\begin{figure}[bht]

{\centering \includegraphics[width=.9\textwidth]{../Main/Figures/CVbiasMozambique} 

}

\caption[Histogram and QQ-plot of the rescaled difference between the smoothed estimates and the direct estimates in the cross validation study]{Histogram and QQ-plot of the rescaled difference between the smoothed estimates and the direct estimates in the cross validation study. The differences between the two estimates are rescaled by the square root of the total variance of the two estimates.}\label{fig:unnamed-chunk-239}
\end{figure}


\end{knitrout}

%%%%%%%%%%%%%%%%%%%%%%%%%%% Plot9
\begin{knitrout}
\definecolor{shadecolor}{rgb}{0.969, 0.969, 0.969}\color{fgcolor}\begin{figure}[bht]

{\centering \includegraphics[width=.7\textwidth]{../Main/Figures/CVbiasbyRegionMozambique} 

}

\caption[Line plot of the difference between smoothed estimates and the direct estimates in the cross validation study]{Line plot of the difference between smoothed estimates and the direct estimates in the cross validation study. The differences between the two estimates are rescaled by the square root of the total variance of the two estimates.}\label{fig:unnamed-chunk-240}
\end{figure}


\end{knitrout}


%%%%%%%%%%%%%%%%%%%%%%%%%%%%%%%%%%%%%%%%%%%%%%%%%%%%%%%%%%%%%%%%%%%%%%%%%%%%%%%%%%%%%%%%%%%%%%%%%%
\clearpage
\subsection{Namibia}


% \subsubsection{Summary of DHS surveys}

%%%%%%%%%%%%%%%%%%%%%%%%%%% Summary 


DHS surveys were conducted in Namibia in 2000, 2007, and 2013.
% years.out[1:(length(years.out)-1)], and years.out[length(years.out)]. 

We fit both the RW2 only model to the combined national data, and compare the time trend at national level with the estimates produced by the UN and IHME in Figure~\ref{fig:unnamed-chunk-242}. We then adjusted the combined national data to the UN estimates of U5MR, and refit the models on the benchmarked data. 

%%%%%%%%%%%%%%%%%%%%%%%%%% Plot5 
\begin{knitrout}
\definecolor{shadecolor}{rgb}{0.969, 0.969, 0.969}\color{fgcolor}\begin{figure}[bht]

{\centering \includegraphics[width=.9\textwidth]{../Main/Figures/Yearly_national_Namibia} 

}

\caption[Temporal national trends along with UN (B3) estimates described in You et al]{Temporal national trends along with UN (B3) estimates described in You et al. (2015) and IHME estimates based on GBD 2015 Child Mortality Collaborators (2016). RW2 represents the smoothed national estimates using the original data before benchmarking with UN estimates. RW2-adj represents the smoothed national estimates using the benchmarked data.}\label{fig:unnamed-chunk-242}
\end{figure}


\end{knitrout}
 

We fit the RW2 model to the benchmarked data in each area. 
% The proportions of the explained variation is summarized in Table~\ref{tab:paste0(countryname, "-var")}. 
We compare the results in Figure~\ref{fig:unnamed-chunk-243} to \ref{fig:unnamed-chunk-247}.
Figure~\ref{fig:unnamed-chunk-243} compares the smoothed estimates against the direct estimates. Figure~\ref{fig:unnamed-chunk-244} and Figure~\ref{fig:unnamed-chunk-245} show the posterior median estimates of U5MR in each region over time and the reductions from 1990 period respectively.
Figure~\ref{fig:unnamed-chunk-246} shows the smoothed estimates by region over time and Figure~\ref{fig:unnamed-chunk-247} compares the smoothed estimates with direct estimates from each survey for each region over time.


% %%%%%%%%%%%%%%%%%%%%%%%%%%% Table1 
% <<echo=FALSE, results='asis'>>=
% load("rda/variance_tables.rda")
% countryname2 <- gsub(" ", "", countryname)
% variance <- tables.all[[countryname]]

% table_count <- table_count + 1

% names <- c("RW2 ($\\sigma^2_{\\gamma_{t}}$)", "ICAR ($\\sigma^2_{\\phi_{i}}$)", "IID space ($\\sigma^2_{\\theta_{i}}$)", "IID time ($\\sigma^2_{\\alpha_{t}}$)", "IID space time ($\\sigma^2_{\\delta_{it}}$)")

% variance$Proportion <- round(variance$Proportion*100, digits = 2)
% row.names(variance) <- names
% tab <- xtable(variance, digits = c(1, 3, 2),align = "l|ll",
%        label = paste0("tab:", countryname, "-var"),
%        caption = paste(country, ": summary of the variance components in the RW2 model", sep = ''))
% print(tab, comment = FALSE,sanitize.text.function = function(x) {x})
% @

%%%%%%%%%%%%%%%%%%%%%%%%%%% Plot1 
\begin{knitrout}
\definecolor{shadecolor}{rgb}{0.969, 0.969, 0.969}\color{fgcolor}\begin{figure}[bht]

{\centering \includegraphics[width=.9\textwidth]{../Main/Figures/SmoothvDirectNamibia_meta} 

}

\caption[Smooth versus direct Admin 1 estimates]{Smooth versus direct Admin 1 estimates. Left: Combined (meta-analysis) survey estimate against combined direct estimates. Right: Combined (meta-analysis) survey estimate against direct estimates from each survey.}\label{fig:unnamed-chunk-243}
\end{figure}


\end{knitrout}

%%%%%%%%%%%%%%%%%%%%%%%%%%% Plot2 
\begin{knitrout}
\definecolor{shadecolor}{rgb}{0.969, 0.969, 0.969}\color{fgcolor}\begin{figure}[bht]

{\centering \includegraphics[width=.9\textwidth]{../Main/Figures/SmoothMedianNamibia} 

}

\caption[Maps of posterior medians for Namibia  over time]{Maps of posterior medians for Namibia  over time.}\label{fig:unnamed-chunk-244}
\end{figure}


\end{knitrout}
%%%%%%%%%%%%%%%%%%%%%%%%%%% Plot2a
\begin{knitrout}
\definecolor{shadecolor}{rgb}{0.969, 0.969, 0.969}\color{fgcolor}\begin{figure}[bht]

{\centering \includegraphics[width=.9\textwidth]{../Main/Figures/ReductionMedianNamibia} 

}

\caption[Maps of reduction of posterior median U5MR in each five-year period compared to 1990 in Namibia over time]{Maps of reduction of posterior median U5MR in each five-year period compared to 1990 in Namibia over time.}\label{fig:unnamed-chunk-245}
\end{figure}


\end{knitrout}
%%%%%%%%%%%%%%%%%%%%%%%%%%% Plot3 
\begin{knitrout}
\definecolor{shadecolor}{rgb}{0.969, 0.969, 0.969}\color{fgcolor}\begin{figure}[bht]

{\centering \includegraphics[width=.95\textwidth]{../Main/Figures/Yearly_v_Periods_Namibia} 

}

\caption[Smoothed regional estimates over time]{Smoothed regional estimates over time. The line indicates yearly posterior median estimates and error bars indicate 95 \% posterior credible interval at each time period.}\label{fig:unnamed-chunk-246}
\end{figure}


\end{knitrout}

%%%%%%%%%%%%%%%%%%%%%%%%%%% Plot4 
\begin{knitrout}
\definecolor{shadecolor}{rgb}{0.969, 0.969, 0.969}\color{fgcolor}\begin{figure}[bht]

{\centering \includegraphics[width=.9\textwidth]{../Main/Figures/LineSubMedianNamibia} 

}

\caption[Smoothed regional estimates over time compared to the direct estimates from each surveys]{Smoothed regional estimates over time compared to the direct estimates from each surveys. Direct estimates are not benchmarked with UN estimates. The line indicates posterior median and error bars indicate 95\% posterior credible interval.}\label{fig:unnamed-chunk-247}
\end{figure}


\end{knitrout}
% \subsubsection{National model results}
We further assess the RW2 model by holding out some observations, and compare the projections to the direct estimates in these holdout observations. Figure~\ref{fig:unnamed-chunk-248} compares the predicted estimates for the out-of-sample observations  with the direct estimates by holding out observations from each area in each time period.  Figure~\ref{fig:unnamed-chunk-249} compares the histogram of the bias rescaled by the total variance in the cross validation studies. Figure~\ref{fig:unnamed-chunk-250} compares the rescaled bias by region and time periods.



% %%%%%%%%%%%%%%%%%%%%%%%%%%% Plot6
% << echo=FALSE, out.width = ".9\\textwidth", fig.width = 12, fig.height = 6, fig.cap = "Out-of-sample predictions along with direct estimates in the cross validation study where all data from each time period is held out and predicted using the rest of the data.">>=
% fig_count <- fig_count + 1
% knitr::include_graphics(paste0("../Main/Figures/CV_byYear_withError_", countryname2, ".pdf")) 
% @
 
%%%%%%%%%%%%%%%%%%%%%%%%%%% Plot7
\begin{knitrout}
\definecolor{shadecolor}{rgb}{0.969, 0.969, 0.969}\color{fgcolor}\begin{figure}[bht]

{\centering \includegraphics[width=.9\textwidth]{../Main/Figures/CV_byYearRegion_withError_Namibia} 

}

\caption[Out-of-sample predictions along with direct estimates in the cross validation study where data from one region in each time period is held out and predicted using the rest of the data]{Out-of-sample predictions along with direct estimates in the cross validation study where data from one region in each time period is held out and predicted using the rest of the data.}\label{fig:unnamed-chunk-248}
\end{figure}


\end{knitrout}

%%%%%%%%%%%%%%%%%%%%%%%%%%% Plot8
\begin{knitrout}
\definecolor{shadecolor}{rgb}{0.969, 0.969, 0.969}\color{fgcolor}\begin{figure}[bht]

{\centering \includegraphics[width=.9\textwidth]{../Main/Figures/CVbiasNamibia} 

}

\caption[Histogram and QQ-plot of the rescaled difference between the smoothed estimates and the direct estimates in the cross validation study]{Histogram and QQ-plot of the rescaled difference between the smoothed estimates and the direct estimates in the cross validation study. The differences between the two estimates are rescaled by the square root of the total variance of the two estimates.}\label{fig:unnamed-chunk-249}
\end{figure}


\end{knitrout}

%%%%%%%%%%%%%%%%%%%%%%%%%%% Plot9
\begin{knitrout}
\definecolor{shadecolor}{rgb}{0.969, 0.969, 0.969}\color{fgcolor}\begin{figure}[bht]

{\centering \includegraphics[width=.7\textwidth]{../Main/Figures/CVbiasbyRegionNamibia} 

}

\caption[Line plot of the difference between smoothed estimates and the direct estimates in the cross validation study]{Line plot of the difference between smoothed estimates and the direct estimates in the cross validation study. The differences between the two estimates are rescaled by the square root of the total variance of the two estimates.}\label{fig:unnamed-chunk-250}
\end{figure}


\end{knitrout}


%%%%%%%%%%%%%%%%%%%%%%%%%%%%%%%%%%%%%%%%%%%%%%%%%%%%%%%%%%%%%%%%%%%%%%%%%%%%%%%%%%%%%%%%%%%%%%%%%%
\clearpage
\subsection{Niger}


% \subsubsection{Summary of DHS surveys}

%%%%%%%%%%%%%%%%%%%%%%%%%%% Summary 


DHS surveys were conducted in Niger in 1992, 1998, 2006, and 2012.
% years.out[1:(length(years.out)-1)], and years.out[length(years.out)]. 

We fit both the RW2 only model to the combined national data, and compare the time trend at national level with the estimates produced by the UN and IHME in Figure~\ref{fig:unnamed-chunk-252}. We then adjusted the combined national data to the UN estimates of U5MR, and refit the models on the benchmarked data. 

%%%%%%%%%%%%%%%%%%%%%%%%%% Plot5 
\begin{knitrout}
\definecolor{shadecolor}{rgb}{0.969, 0.969, 0.969}\color{fgcolor}\begin{figure}[bht]

{\centering \includegraphics[width=.9\textwidth]{../Main/Figures/Yearly_national_Niger} 

}

\caption[Temporal national trends along with UN (B3) estimates described in You et al]{Temporal national trends along with UN (B3) estimates described in You et al. (2015) and IHME estimates based on GBD 2015 Child Mortality Collaborators (2016). RW2 represents the smoothed national estimates using the original data before benchmarking with UN estimates. RW2-adj represents the smoothed national estimates using the benchmarked data.}\label{fig:unnamed-chunk-252}
\end{figure}


\end{knitrout}
 

We fit the RW2 model to the benchmarked data in each area. 
% The proportions of the explained variation is summarized in Table~\ref{tab:paste0(countryname, "-var")}. 
We compare the results in Figure~\ref{fig:unnamed-chunk-253} to \ref{fig:unnamed-chunk-257}.
Figure~\ref{fig:unnamed-chunk-253} compares the smoothed estimates against the direct estimates. Figure~\ref{fig:unnamed-chunk-254} and Figure~\ref{fig:unnamed-chunk-255} show the posterior median estimates of U5MR in each region over time and the reductions from 1990 period respectively.
Figure~\ref{fig:unnamed-chunk-256} shows the smoothed estimates by region over time and Figure~\ref{fig:unnamed-chunk-257} compares the smoothed estimates with direct estimates from each survey for each region over time.


% %%%%%%%%%%%%%%%%%%%%%%%%%%% Table1 
% <<echo=FALSE, results='asis'>>=
% load("rda/variance_tables.rda")
% countryname2 <- gsub(" ", "", countryname)
% variance <- tables.all[[countryname]]

% table_count <- table_count + 1

% names <- c("RW2 ($\\sigma^2_{\\gamma_{t}}$)", "ICAR ($\\sigma^2_{\\phi_{i}}$)", "IID space ($\\sigma^2_{\\theta_{i}}$)", "IID time ($\\sigma^2_{\\alpha_{t}}$)", "IID space time ($\\sigma^2_{\\delta_{it}}$)")

% variance$Proportion <- round(variance$Proportion*100, digits = 2)
% row.names(variance) <- names
% tab <- xtable(variance, digits = c(1, 3, 2),align = "l|ll",
%        label = paste0("tab:", countryname, "-var"),
%        caption = paste(country, ": summary of the variance components in the RW2 model", sep = ''))
% print(tab, comment = FALSE,sanitize.text.function = function(x) {x})
% @

%%%%%%%%%%%%%%%%%%%%%%%%%%% Plot1 
\begin{knitrout}
\definecolor{shadecolor}{rgb}{0.969, 0.969, 0.969}\color{fgcolor}\begin{figure}[bht]

{\centering \includegraphics[width=.9\textwidth]{../Main/Figures/SmoothvDirectNiger_meta} 

}

\caption[Smooth versus direct Admin 1 estimates]{Smooth versus direct Admin 1 estimates. Left: Combined (meta-analysis) survey estimate against combined direct estimates. Right: Combined (meta-analysis) survey estimate against direct estimates from each survey.}\label{fig:unnamed-chunk-253}
\end{figure}


\end{knitrout}

%%%%%%%%%%%%%%%%%%%%%%%%%%% Plot2 
\begin{knitrout}
\definecolor{shadecolor}{rgb}{0.969, 0.969, 0.969}\color{fgcolor}\begin{figure}[bht]

{\centering \includegraphics[width=.9\textwidth]{../Main/Figures/SmoothMedianNiger} 

}

\caption[Maps of posterior medians for Niger  over time]{Maps of posterior medians for Niger  over time.}\label{fig:unnamed-chunk-254}
\end{figure}


\end{knitrout}
%%%%%%%%%%%%%%%%%%%%%%%%%%% Plot2a
\begin{knitrout}
\definecolor{shadecolor}{rgb}{0.969, 0.969, 0.969}\color{fgcolor}\begin{figure}[bht]

{\centering \includegraphics[width=.9\textwidth]{../Main/Figures/ReductionMedianNiger} 

}

\caption[Maps of reduction of posterior median U5MR in each five-year period compared to 1990 in Niger over time]{Maps of reduction of posterior median U5MR in each five-year period compared to 1990 in Niger over time.}\label{fig:unnamed-chunk-255}
\end{figure}


\end{knitrout}
%%%%%%%%%%%%%%%%%%%%%%%%%%% Plot3 
\begin{knitrout}
\definecolor{shadecolor}{rgb}{0.969, 0.969, 0.969}\color{fgcolor}\begin{figure}[bht]

{\centering \includegraphics[width=.95\textwidth]{../Main/Figures/Yearly_v_Periods_Niger} 

}

\caption[Smoothed regional estimates over time]{Smoothed regional estimates over time. The line indicates yearly posterior median estimates and error bars indicate 95 \% posterior credible interval at each time period.}\label{fig:unnamed-chunk-256}
\end{figure}


\end{knitrout}

%%%%%%%%%%%%%%%%%%%%%%%%%%% Plot4 
\begin{knitrout}
\definecolor{shadecolor}{rgb}{0.969, 0.969, 0.969}\color{fgcolor}\begin{figure}[bht]

{\centering \includegraphics[width=.9\textwidth]{../Main/Figures/LineSubMedianNiger} 

}

\caption[Smoothed regional estimates over time compared to the direct estimates from each surveys]{Smoothed regional estimates over time compared to the direct estimates from each surveys. Direct estimates are not benchmarked with UN estimates. The line indicates posterior median and error bars indicate 95\% posterior credible interval.}\label{fig:unnamed-chunk-257}
\end{figure}


\end{knitrout}
% \subsubsection{National model results}
We further assess the RW2 model by holding out some observations, and compare the projections to the direct estimates in these holdout observations. Figure~\ref{fig:unnamed-chunk-258} compares the predicted estimates for the out-of-sample observations  with the direct estimates by holding out observations from each area in each time period.  Figure~\ref{fig:unnamed-chunk-259} compares the histogram of the bias rescaled by the total variance in the cross validation studies. Figure~\ref{fig:unnamed-chunk-260} compares the rescaled bias by region and time periods.



% %%%%%%%%%%%%%%%%%%%%%%%%%%% Plot6
% << echo=FALSE, out.width = ".9\\textwidth", fig.width = 12, fig.height = 6, fig.cap = "Out-of-sample predictions along with direct estimates in the cross validation study where all data from each time period is held out and predicted using the rest of the data.">>=
% fig_count <- fig_count + 1
% knitr::include_graphics(paste0("../Main/Figures/CV_byYear_withError_", countryname2, ".pdf")) 
% @
 
%%%%%%%%%%%%%%%%%%%%%%%%%%% Plot7
\begin{knitrout}
\definecolor{shadecolor}{rgb}{0.969, 0.969, 0.969}\color{fgcolor}\begin{figure}[bht]

{\centering \includegraphics[width=.9\textwidth]{../Main/Figures/CV_byYearRegion_withError_Niger} 

}

\caption[Out-of-sample predictions along with direct estimates in the cross validation study where data from one region in each time period is held out and predicted using the rest of the data]{Out-of-sample predictions along with direct estimates in the cross validation study where data from one region in each time period is held out and predicted using the rest of the data.}\label{fig:unnamed-chunk-258}
\end{figure}


\end{knitrout}

%%%%%%%%%%%%%%%%%%%%%%%%%%% Plot8
\begin{knitrout}
\definecolor{shadecolor}{rgb}{0.969, 0.969, 0.969}\color{fgcolor}\begin{figure}[bht]

{\centering \includegraphics[width=.9\textwidth]{../Main/Figures/CVbiasNiger} 

}

\caption[Histogram and QQ-plot of the rescaled difference between the smoothed estimates and the direct estimates in the cross validation study]{Histogram and QQ-plot of the rescaled difference between the smoothed estimates and the direct estimates in the cross validation study. The differences between the two estimates are rescaled by the square root of the total variance of the two estimates.}\label{fig:unnamed-chunk-259}
\end{figure}


\end{knitrout}

%%%%%%%%%%%%%%%%%%%%%%%%%%% Plot9
\begin{knitrout}
\definecolor{shadecolor}{rgb}{0.969, 0.969, 0.969}\color{fgcolor}\begin{figure}[bht]

{\centering \includegraphics[width=.7\textwidth]{../Main/Figures/CVbiasbyRegionNiger} 

}

\caption[Line plot of the difference between smoothed estimates and the direct estimates in the cross validation study]{Line plot of the difference between smoothed estimates and the direct estimates in the cross validation study. The differences between the two estimates are rescaled by the square root of the total variance of the two estimates.}\label{fig:unnamed-chunk-260}
\end{figure}


\end{knitrout}


%%%%%%%%%%%%%%%%%%%%%%%%%%%%%%%%%%%%%%%%%%%%%%%%%%%%%%%%%%%%%%%%%%%%%%%%%%%%%%%%%%%%%%%%%%%%%%%%%%
\clearpage
\subsection{Nigeria}


% \subsubsection{Summary of DHS surveys}

%%%%%%%%%%%%%%%%%%%%%%%%%%% Summary 


DHS surveys were conducted in Nigeria in 1990, 2003, 2008, and 2013.
% years.out[1:(length(years.out)-1)], and years.out[length(years.out)]. 

We fit both the RW2 only model to the combined national data, and compare the time trend at national level with the estimates produced by the UN and IHME in Figure~\ref{fig:unnamed-chunk-262}. We then adjusted the combined national data to the UN estimates of U5MR, and refit the models on the benchmarked data. 

%%%%%%%%%%%%%%%%%%%%%%%%%% Plot5 
\begin{knitrout}
\definecolor{shadecolor}{rgb}{0.969, 0.969, 0.969}\color{fgcolor}\begin{figure}[bht]

{\centering \includegraphics[width=.9\textwidth]{../Main/Figures/Yearly_national_Nigeria} 

}

\caption[Temporal national trends along with UN (B3) estimates described in You et al]{Temporal national trends along with UN (B3) estimates described in You et al. (2015) and IHME estimates based on GBD 2015 Child Mortality Collaborators (2016). RW2 represents the smoothed national estimates using the original data before benchmarking with UN estimates. RW2-adj represents the smoothed national estimates using the benchmarked data.}\label{fig:unnamed-chunk-262}
\end{figure}


\end{knitrout}
 

We fit the RW2 model to the benchmarked data in each area. 
% The proportions of the explained variation is summarized in Table~\ref{tab:paste0(countryname, "-var")}. 
We compare the results in Figure~\ref{fig:unnamed-chunk-263} to \ref{fig:unnamed-chunk-267}.
Figure~\ref{fig:unnamed-chunk-263} compares the smoothed estimates against the direct estimates. Figure~\ref{fig:unnamed-chunk-264} and Figure~\ref{fig:unnamed-chunk-265} show the posterior median estimates of U5MR in each region over time and the reductions from 1990 period respectively.
Figure~\ref{fig:unnamed-chunk-266} shows the smoothed estimates by region over time and Figure~\ref{fig:unnamed-chunk-267} compares the smoothed estimates with direct estimates from each survey for each region over time.


% %%%%%%%%%%%%%%%%%%%%%%%%%%% Table1 
% <<echo=FALSE, results='asis'>>=
% load("rda/variance_tables.rda")
% countryname2 <- gsub(" ", "", countryname)
% variance <- tables.all[[countryname]]

% table_count <- table_count + 1

% names <- c("RW2 ($\\sigma^2_{\\gamma_{t}}$)", "ICAR ($\\sigma^2_{\\phi_{i}}$)", "IID space ($\\sigma^2_{\\theta_{i}}$)", "IID time ($\\sigma^2_{\\alpha_{t}}$)", "IID space time ($\\sigma^2_{\\delta_{it}}$)")

% variance$Proportion <- round(variance$Proportion*100, digits = 2)
% row.names(variance) <- names
% tab <- xtable(variance, digits = c(1, 3, 2),align = "l|ll",
%        label = paste0("tab:", countryname, "-var"),
%        caption = paste(country, ": summary of the variance components in the RW2 model", sep = ''))
% print(tab, comment = FALSE,sanitize.text.function = function(x) {x})
% @

%%%%%%%%%%%%%%%%%%%%%%%%%%% Plot1 
\begin{knitrout}
\definecolor{shadecolor}{rgb}{0.969, 0.969, 0.969}\color{fgcolor}\begin{figure}[bht]

{\centering \includegraphics[width=.9\textwidth]{../Main/Figures/SmoothvDirectNigeria_meta} 

}

\caption[Smooth versus direct Admin 1 estimates]{Smooth versus direct Admin 1 estimates. Left: Combined (meta-analysis) survey estimate against combined direct estimates. Right: Combined (meta-analysis) survey estimate against direct estimates from each survey.}\label{fig:unnamed-chunk-263}
\end{figure}


\end{knitrout}

%%%%%%%%%%%%%%%%%%%%%%%%%%% Plot2 
\begin{knitrout}
\definecolor{shadecolor}{rgb}{0.969, 0.969, 0.969}\color{fgcolor}\begin{figure}[bht]

{\centering \includegraphics[width=.9\textwidth]{../Main/Figures/SmoothMedianNigeria} 

}

\caption[Maps of posterior medians for Nigeria  over time]{Maps of posterior medians for Nigeria  over time.}\label{fig:unnamed-chunk-264}
\end{figure}


\end{knitrout}
%%%%%%%%%%%%%%%%%%%%%%%%%%% Plot2a
\begin{knitrout}
\definecolor{shadecolor}{rgb}{0.969, 0.969, 0.969}\color{fgcolor}\begin{figure}[bht]

{\centering \includegraphics[width=.9\textwidth]{../Main/Figures/ReductionMedianNigeria} 

}

\caption[Maps of reduction of posterior median U5MR in each five-year period compared to 1990 in Nigeria over time]{Maps of reduction of posterior median U5MR in each five-year period compared to 1990 in Nigeria over time.}\label{fig:unnamed-chunk-265}
\end{figure}


\end{knitrout}
%%%%%%%%%%%%%%%%%%%%%%%%%%% Plot3 
\begin{knitrout}
\definecolor{shadecolor}{rgb}{0.969, 0.969, 0.969}\color{fgcolor}\begin{figure}[bht]

{\centering \includegraphics[width=.95\textwidth]{../Main/Figures/Yearly_v_Periods_Nigeria} 

}

\caption[Smoothed regional estimates over time]{Smoothed regional estimates over time. The line indicates yearly posterior median estimates and error bars indicate 95 \% posterior credible interval at each time period.}\label{fig:unnamed-chunk-266}
\end{figure}


\end{knitrout}

%%%%%%%%%%%%%%%%%%%%%%%%%%% Plot4 
\begin{knitrout}
\definecolor{shadecolor}{rgb}{0.969, 0.969, 0.969}\color{fgcolor}\begin{figure}[bht]

{\centering \includegraphics[width=.9\textwidth]{../Main/Figures/LineSubMedianNigeria} 

}

\caption[Smoothed regional estimates over time compared to the direct estimates from each surveys]{Smoothed regional estimates over time compared to the direct estimates from each surveys. Direct estimates are not benchmarked with UN estimates. The line indicates posterior median and error bars indicate 95\% posterior credible interval.}\label{fig:unnamed-chunk-267}
\end{figure}


\end{knitrout}
% \subsubsection{National model results}
We further assess the RW2 model by holding out some observations, and compare the projections to the direct estimates in these holdout observations. Figure~\ref{fig:unnamed-chunk-268} compares the predicted estimates for the out-of-sample observations  with the direct estimates by holding out observations from each area in each time period.  Figure~\ref{fig:unnamed-chunk-269} compares the histogram of the bias rescaled by the total variance in the cross validation studies. Figure~\ref{fig:unnamed-chunk-270} compares the rescaled bias by region and time periods.



% %%%%%%%%%%%%%%%%%%%%%%%%%%% Plot6
% << echo=FALSE, out.width = ".9\\textwidth", fig.width = 12, fig.height = 6, fig.cap = "Out-of-sample predictions along with direct estimates in the cross validation study where all data from each time period is held out and predicted using the rest of the data.">>=
% fig_count <- fig_count + 1
% knitr::include_graphics(paste0("../Main/Figures/CV_byYear_withError_", countryname2, ".pdf")) 
% @
 
%%%%%%%%%%%%%%%%%%%%%%%%%%% Plot7
\begin{knitrout}
\definecolor{shadecolor}{rgb}{0.969, 0.969, 0.969}\color{fgcolor}\begin{figure}[bht]

{\centering \includegraphics[width=.9\textwidth]{../Main/Figures/CV_byYearRegion_withError_Nigeria} 

}

\caption[Out-of-sample predictions along with direct estimates in the cross validation study where data from one region in each time period is held out and predicted using the rest of the data]{Out-of-sample predictions along with direct estimates in the cross validation study where data from one region in each time period is held out and predicted using the rest of the data.}\label{fig:unnamed-chunk-268}
\end{figure}


\end{knitrout}

%%%%%%%%%%%%%%%%%%%%%%%%%%% Plot8
\begin{knitrout}
\definecolor{shadecolor}{rgb}{0.969, 0.969, 0.969}\color{fgcolor}\begin{figure}[bht]

{\centering \includegraphics[width=.9\textwidth]{../Main/Figures/CVbiasNigeria} 

}

\caption[Histogram and QQ-plot of the rescaled difference between the smoothed estimates and the direct estimates in the cross validation study]{Histogram and QQ-plot of the rescaled difference between the smoothed estimates and the direct estimates in the cross validation study. The differences between the two estimates are rescaled by the square root of the total variance of the two estimates.}\label{fig:unnamed-chunk-269}
\end{figure}


\end{knitrout}

%%%%%%%%%%%%%%%%%%%%%%%%%%% Plot9
\begin{knitrout}
\definecolor{shadecolor}{rgb}{0.969, 0.969, 0.969}\color{fgcolor}\begin{figure}[bht]

{\centering \includegraphics[width=.7\textwidth]{../Main/Figures/CVbiasbyRegionNigeria} 

}

\caption[Line plot of the difference between smoothed estimates and the direct estimates in the cross validation study]{Line plot of the difference between smoothed estimates and the direct estimates in the cross validation study. The differences between the two estimates are rescaled by the square root of the total variance of the two estimates.}\label{fig:unnamed-chunk-270}
\end{figure}


\end{knitrout}


%%%%%%%%%%%%%%%%%%%%%%%%%%%%%%%%%%%%%%%%%%%%%%%%%%%%%%%%%%%%%%%%%%%%%%%%%%%%%%%%%%%%%%%%%%%%%%%%%%
\clearpage
\subsection{Rwanda}


% \subsubsection{Summary of DHS surveys}

%%%%%%%%%%%%%%%%%%%%%%%%%%% Summary 


DHS surveys were conducted in Rwanda in 2000, 2005, 2008, 2010, and 2015.
% years.out[1:(length(years.out)-1)], and years.out[length(years.out)]. 

We fit both the RW2 only model to the combined national data, and compare the time trend at national level with the estimates produced by the UN and IHME in Figure~\ref{fig:unnamed-chunk-272}. We then adjusted the combined national data to the UN estimates of U5MR, and refit the models on the benchmarked data. 

%%%%%%%%%%%%%%%%%%%%%%%%%% Plot5 
\begin{knitrout}
\definecolor{shadecolor}{rgb}{0.969, 0.969, 0.969}\color{fgcolor}\begin{figure}[bht]

{\centering \includegraphics[width=.9\textwidth]{../Main/Figures/Yearly_national_Rwanda} 

}

\caption[Temporal national trends along with UN (B3) estimates described in You et al]{Temporal national trends along with UN (B3) estimates described in You et al. (2015) and IHME estimates based on GBD 2015 Child Mortality Collaborators (2016). RW2 represents the smoothed national estimates using the original data before benchmarking with UN estimates. RW2-adj represents the smoothed national estimates using the benchmarked data.}\label{fig:unnamed-chunk-272}
\end{figure}


\end{knitrout}
 

We fit the RW2 model to the benchmarked data in each area. 
% The proportions of the explained variation is summarized in Table~\ref{tab:paste0(countryname, "-var")}. 
We compare the results in Figure~\ref{fig:unnamed-chunk-273} to \ref{fig:unnamed-chunk-277}.
Figure~\ref{fig:unnamed-chunk-273} compares the smoothed estimates against the direct estimates. Figure~\ref{fig:unnamed-chunk-274} and Figure~\ref{fig:unnamed-chunk-275} show the posterior median estimates of U5MR in each region over time and the reductions from 1990 period respectively.
Figure~\ref{fig:unnamed-chunk-276} shows the smoothed estimates by region over time and Figure~\ref{fig:unnamed-chunk-277} compares the smoothed estimates with direct estimates from each survey for each region over time.


% %%%%%%%%%%%%%%%%%%%%%%%%%%% Table1 
% <<echo=FALSE, results='asis'>>=
% load("rda/variance_tables.rda")
% countryname2 <- gsub(" ", "", countryname)
% variance <- tables.all[[countryname]]

% table_count <- table_count + 1

% names <- c("RW2 ($\\sigma^2_{\\gamma_{t}}$)", "ICAR ($\\sigma^2_{\\phi_{i}}$)", "IID space ($\\sigma^2_{\\theta_{i}}$)", "IID time ($\\sigma^2_{\\alpha_{t}}$)", "IID space time ($\\sigma^2_{\\delta_{it}}$)")

% variance$Proportion <- round(variance$Proportion*100, digits = 2)
% row.names(variance) <- names
% tab <- xtable(variance, digits = c(1, 3, 2),align = "l|ll",
%        label = paste0("tab:", countryname, "-var"),
%        caption = paste(country, ": summary of the variance components in the RW2 model", sep = ''))
% print(tab, comment = FALSE,sanitize.text.function = function(x) {x})
% @

%%%%%%%%%%%%%%%%%%%%%%%%%%% Plot1 
\begin{knitrout}
\definecolor{shadecolor}{rgb}{0.969, 0.969, 0.969}\color{fgcolor}\begin{figure}[bht]

{\centering \includegraphics[width=.9\textwidth]{../Main/Figures/SmoothvDirectRwanda_meta} 

}

\caption[Smooth versus direct Admin 1 estimates]{Smooth versus direct Admin 1 estimates. Left: Combined (meta-analysis) survey estimate against combined direct estimates. Right: Combined (meta-analysis) survey estimate against direct estimates from each survey.}\label{fig:unnamed-chunk-273}
\end{figure}


\end{knitrout}

%%%%%%%%%%%%%%%%%%%%%%%%%%% Plot2 
\begin{knitrout}
\definecolor{shadecolor}{rgb}{0.969, 0.969, 0.969}\color{fgcolor}\begin{figure}[bht]

{\centering \includegraphics[width=.9\textwidth]{../Main/Figures/SmoothMedianRwanda} 

}

\caption[Maps of posterior medians for Rwanda  over time]{Maps of posterior medians for Rwanda  over time.}\label{fig:unnamed-chunk-274}
\end{figure}


\end{knitrout}
%%%%%%%%%%%%%%%%%%%%%%%%%%% Plot2a
\begin{knitrout}
\definecolor{shadecolor}{rgb}{0.969, 0.969, 0.969}\color{fgcolor}\begin{figure}[bht]

{\centering \includegraphics[width=.9\textwidth]{../Main/Figures/ReductionMedianRwanda} 

}

\caption[Maps of reduction of posterior median U5MR in each five-year period compared to 1990 in Rwanda over time]{Maps of reduction of posterior median U5MR in each five-year period compared to 1990 in Rwanda over time.}\label{fig:unnamed-chunk-275}
\end{figure}


\end{knitrout}
%%%%%%%%%%%%%%%%%%%%%%%%%%% Plot3 
\begin{knitrout}
\definecolor{shadecolor}{rgb}{0.969, 0.969, 0.969}\color{fgcolor}\begin{figure}[bht]

{\centering \includegraphics[width=.95\textwidth]{../Main/Figures/Yearly_v_Periods_Rwanda} 

}

\caption[Smoothed regional estimates over time]{Smoothed regional estimates over time. The line indicates yearly posterior median estimates and error bars indicate 95 \% posterior credible interval at each time period.}\label{fig:unnamed-chunk-276}
\end{figure}


\end{knitrout}

%%%%%%%%%%%%%%%%%%%%%%%%%%% Plot4 
\begin{knitrout}
\definecolor{shadecolor}{rgb}{0.969, 0.969, 0.969}\color{fgcolor}\begin{figure}[bht]

{\centering \includegraphics[width=.9\textwidth]{../Main/Figures/LineSubMedianRwanda} 

}

\caption[Smoothed regional estimates over time compared to the direct estimates from each surveys]{Smoothed regional estimates over time compared to the direct estimates from each surveys. Direct estimates are not benchmarked with UN estimates. The line indicates posterior median and error bars indicate 95\% posterior credible interval.}\label{fig:unnamed-chunk-277}
\end{figure}


\end{knitrout}
% \subsubsection{National model results}
We further assess the RW2 model by holding out some observations, and compare the projections to the direct estimates in these holdout observations. Figure~\ref{fig:unnamed-chunk-278} compares the predicted estimates for the out-of-sample observations  with the direct estimates by holding out observations from each area in each time period.  Figure~\ref{fig:unnamed-chunk-279} compares the histogram of the bias rescaled by the total variance in the cross validation studies. Figure~\ref{fig:unnamed-chunk-280} compares the rescaled bias by region and time periods.



% %%%%%%%%%%%%%%%%%%%%%%%%%%% Plot6
% << echo=FALSE, out.width = ".9\\textwidth", fig.width = 12, fig.height = 6, fig.cap = "Out-of-sample predictions along with direct estimates in the cross validation study where all data from each time period is held out and predicted using the rest of the data.">>=
% fig_count <- fig_count + 1
% knitr::include_graphics(paste0("../Main/Figures/CV_byYear_withError_", countryname2, ".pdf")) 
% @
 
%%%%%%%%%%%%%%%%%%%%%%%%%%% Plot7
\begin{knitrout}
\definecolor{shadecolor}{rgb}{0.969, 0.969, 0.969}\color{fgcolor}\begin{figure}[bht]

{\centering \includegraphics[width=.9\textwidth]{../Main/Figures/CV_byYearRegion_withError_Rwanda} 

}

\caption[Out-of-sample predictions along with direct estimates in the cross validation study where data from one region in each time period is held out and predicted using the rest of the data]{Out-of-sample predictions along with direct estimates in the cross validation study where data from one region in each time period is held out and predicted using the rest of the data.}\label{fig:unnamed-chunk-278}
\end{figure}


\end{knitrout}

%%%%%%%%%%%%%%%%%%%%%%%%%%% Plot8
\begin{knitrout}
\definecolor{shadecolor}{rgb}{0.969, 0.969, 0.969}\color{fgcolor}\begin{figure}[bht]

{\centering \includegraphics[width=.9\textwidth]{../Main/Figures/CVbiasRwanda} 

}

\caption[Histogram and QQ-plot of the rescaled difference between the smoothed estimates and the direct estimates in the cross validation study]{Histogram and QQ-plot of the rescaled difference between the smoothed estimates and the direct estimates in the cross validation study. The differences between the two estimates are rescaled by the square root of the total variance of the two estimates.}\label{fig:unnamed-chunk-279}
\end{figure}


\end{knitrout}

%%%%%%%%%%%%%%%%%%%%%%%%%%% Plot9
\begin{knitrout}
\definecolor{shadecolor}{rgb}{0.969, 0.969, 0.969}\color{fgcolor}\begin{figure}[bht]

{\centering \includegraphics[width=.7\textwidth]{../Main/Figures/CVbiasbyRegionRwanda} 

}

\caption[Line plot of the difference between smoothed estimates and the direct estimates in the cross validation study]{Line plot of the difference between smoothed estimates and the direct estimates in the cross validation study. The differences between the two estimates are rescaled by the square root of the total variance of the two estimates.}\label{fig:unnamed-chunk-280}
\end{figure}


\end{knitrout}


%%%%%%%%%%%%%%%%%%%%%%%%%%%%%%%%%%%%%%%%%%%%%%%%%%%%%%%%%%%%%%%%%%%%%%%%%%%%%%%%%%%%%%%%%%%%%%%%%%
\clearpage
\subsection{Senegal}


% \subsubsection{Summary of DHS surveys}

%%%%%%%%%%%%%%%%%%%%%%%%%%% Summary 


DHS surveys were conducted in Senegal in 1992, 1997, 2005, 2010, 2012, 2014, 2015, and 2016.
% years.out[1:(length(years.out)-1)], and years.out[length(years.out)]. 

We fit both the RW2 only model to the combined national data, and compare the time trend at national level with the estimates produced by the UN and IHME in Figure~\ref{fig:unnamed-chunk-282}. We then adjusted the combined national data to the UN estimates of U5MR, and refit the models on the benchmarked data. 

%%%%%%%%%%%%%%%%%%%%%%%%%% Plot5 
\begin{knitrout}
\definecolor{shadecolor}{rgb}{0.969, 0.969, 0.969}\color{fgcolor}\begin{figure}[bht]

{\centering \includegraphics[width=.9\textwidth]{../Main/Figures/Yearly_national_Senegal} 

}

\caption[Temporal national trends along with UN (B3) estimates described in You et al]{Temporal national trends along with UN (B3) estimates described in You et al. (2015) and IHME estimates based on GBD 2015 Child Mortality Collaborators (2016). RW2 represents the smoothed national estimates using the original data before benchmarking with UN estimates. RW2-adj represents the smoothed national estimates using the benchmarked data.}\label{fig:unnamed-chunk-282}
\end{figure}


\end{knitrout}
 

We fit the RW2 model to the benchmarked data in each area. 
% The proportions of the explained variation is summarized in Table~\ref{tab:paste0(countryname, "-var")}. 
We compare the results in Figure~\ref{fig:unnamed-chunk-283} to \ref{fig:unnamed-chunk-287}.
Figure~\ref{fig:unnamed-chunk-283} compares the smoothed estimates against the direct estimates. Figure~\ref{fig:unnamed-chunk-284} and Figure~\ref{fig:unnamed-chunk-285} show the posterior median estimates of U5MR in each region over time and the reductions from 1990 period respectively.
Figure~\ref{fig:unnamed-chunk-286} shows the smoothed estimates by region over time and Figure~\ref{fig:unnamed-chunk-287} compares the smoothed estimates with direct estimates from each survey for each region over time.


% %%%%%%%%%%%%%%%%%%%%%%%%%%% Table1 
% <<echo=FALSE, results='asis'>>=
% load("rda/variance_tables.rda")
% countryname2 <- gsub(" ", "", countryname)
% variance <- tables.all[[countryname]]

% table_count <- table_count + 1

% names <- c("RW2 ($\\sigma^2_{\\gamma_{t}}$)", "ICAR ($\\sigma^2_{\\phi_{i}}$)", "IID space ($\\sigma^2_{\\theta_{i}}$)", "IID time ($\\sigma^2_{\\alpha_{t}}$)", "IID space time ($\\sigma^2_{\\delta_{it}}$)")

% variance$Proportion <- round(variance$Proportion*100, digits = 2)
% row.names(variance) <- names
% tab <- xtable(variance, digits = c(1, 3, 2),align = "l|ll",
%        label = paste0("tab:", countryname, "-var"),
%        caption = paste(country, ": summary of the variance components in the RW2 model", sep = ''))
% print(tab, comment = FALSE,sanitize.text.function = function(x) {x})
% @

%%%%%%%%%%%%%%%%%%%%%%%%%%% Plot1 
\begin{knitrout}
\definecolor{shadecolor}{rgb}{0.969, 0.969, 0.969}\color{fgcolor}\begin{figure}[bht]

{\centering \includegraphics[width=.9\textwidth]{../Main/Figures/SmoothvDirectSenegal_meta} 

}

\caption[Smooth versus direct Admin 1 estimates]{Smooth versus direct Admin 1 estimates. Left: Combined (meta-analysis) survey estimate against combined direct estimates. Right: Combined (meta-analysis) survey estimate against direct estimates from each survey.}\label{fig:unnamed-chunk-283}
\end{figure}


\end{knitrout}

%%%%%%%%%%%%%%%%%%%%%%%%%%% Plot2 
\begin{knitrout}
\definecolor{shadecolor}{rgb}{0.969, 0.969, 0.969}\color{fgcolor}\begin{figure}[bht]

{\centering \includegraphics[width=.9\textwidth]{../Main/Figures/SmoothMedianSenegal} 

}

\caption[Maps of posterior medians for Senegal  over time]{Maps of posterior medians for Senegal  over time.}\label{fig:unnamed-chunk-284}
\end{figure}


\end{knitrout}
%%%%%%%%%%%%%%%%%%%%%%%%%%% Plot2a
\begin{knitrout}
\definecolor{shadecolor}{rgb}{0.969, 0.969, 0.969}\color{fgcolor}\begin{figure}[bht]

{\centering \includegraphics[width=.9\textwidth]{../Main/Figures/ReductionMedianSenegal} 

}

\caption[Maps of reduction of posterior median U5MR in each five-year period compared to 1990 in Senegal over time]{Maps of reduction of posterior median U5MR in each five-year period compared to 1990 in Senegal over time.}\label{fig:unnamed-chunk-285}
\end{figure}


\end{knitrout}
%%%%%%%%%%%%%%%%%%%%%%%%%%% Plot3 
\begin{knitrout}
\definecolor{shadecolor}{rgb}{0.969, 0.969, 0.969}\color{fgcolor}\begin{figure}[bht]

{\centering \includegraphics[width=.95\textwidth]{../Main/Figures/Yearly_v_Periods_Senegal} 

}

\caption[Smoothed regional estimates over time]{Smoothed regional estimates over time. The line indicates yearly posterior median estimates and error bars indicate 95 \% posterior credible interval at each time period.}\label{fig:unnamed-chunk-286}
\end{figure}


\end{knitrout}

%%%%%%%%%%%%%%%%%%%%%%%%%%% Plot4 
\begin{knitrout}
\definecolor{shadecolor}{rgb}{0.969, 0.969, 0.969}\color{fgcolor}\begin{figure}[bht]

{\centering \includegraphics[width=.9\textwidth]{../Main/Figures/LineSubMedianSenegal} 

}

\caption[Smoothed regional estimates over time compared to the direct estimates from each surveys]{Smoothed regional estimates over time compared to the direct estimates from each surveys. Direct estimates are not benchmarked with UN estimates. The line indicates posterior median and error bars indicate 95\% posterior credible interval.}\label{fig:unnamed-chunk-287}
\end{figure}


\end{knitrout}
% \subsubsection{National model results}
We further assess the RW2 model by holding out some observations, and compare the projections to the direct estimates in these holdout observations. Figure~\ref{fig:unnamed-chunk-288} compares the predicted estimates for the out-of-sample observations  with the direct estimates by holding out observations from each area in each time period.  Figure~\ref{fig:unnamed-chunk-289} compares the histogram of the bias rescaled by the total variance in the cross validation studies. Figure~\ref{fig:unnamed-chunk-290} compares the rescaled bias by region and time periods.



% %%%%%%%%%%%%%%%%%%%%%%%%%%% Plot6
% << echo=FALSE, out.width = ".9\\textwidth", fig.width = 12, fig.height = 6, fig.cap = "Out-of-sample predictions along with direct estimates in the cross validation study where all data from each time period is held out and predicted using the rest of the data.">>=
% fig_count <- fig_count + 1
% knitr::include_graphics(paste0("../Main/Figures/CV_byYear_withError_", countryname2, ".pdf")) 
% @
 
%%%%%%%%%%%%%%%%%%%%%%%%%%% Plot7
\begin{knitrout}
\definecolor{shadecolor}{rgb}{0.969, 0.969, 0.969}\color{fgcolor}\begin{figure}[bht]

{\centering \includegraphics[width=.9\textwidth]{../Main/Figures/CV_byYearRegion_withError_Senegal} 

}

\caption[Out-of-sample predictions along with direct estimates in the cross validation study where data from one region in each time period is held out and predicted using the rest of the data]{Out-of-sample predictions along with direct estimates in the cross validation study where data from one region in each time period is held out and predicted using the rest of the data.}\label{fig:unnamed-chunk-288}
\end{figure}


\end{knitrout}

%%%%%%%%%%%%%%%%%%%%%%%%%%% Plot8
\begin{knitrout}
\definecolor{shadecolor}{rgb}{0.969, 0.969, 0.969}\color{fgcolor}\begin{figure}[bht]

{\centering \includegraphics[width=.9\textwidth]{../Main/Figures/CVbiasSenegal} 

}

\caption[Histogram and QQ-plot of the rescaled difference between the smoothed estimates and the direct estimates in the cross validation study]{Histogram and QQ-plot of the rescaled difference between the smoothed estimates and the direct estimates in the cross validation study. The differences between the two estimates are rescaled by the square root of the total variance of the two estimates.}\label{fig:unnamed-chunk-289}
\end{figure}


\end{knitrout}

%%%%%%%%%%%%%%%%%%%%%%%%%%% Plot9
\begin{knitrout}
\definecolor{shadecolor}{rgb}{0.969, 0.969, 0.969}\color{fgcolor}\begin{figure}[bht]

{\centering \includegraphics[width=.7\textwidth]{../Main/Figures/CVbiasbyRegionSenegal} 

}

\caption[Line plot of the difference between smoothed estimates and the direct estimates in the cross validation study]{Line plot of the difference between smoothed estimates and the direct estimates in the cross validation study. The differences between the two estimates are rescaled by the square root of the total variance of the two estimates.}\label{fig:unnamed-chunk-290}
\end{figure}


\end{knitrout}


%%%%%%%%%%%%%%%%%%%%%%%%%%%%%%%%%%%%%%%%%%%%%%%%%%%%%%%%%%%%%%%%%%%%%%%%%%%%%%%%%%%%%%%%%%%%%%%%%%
\clearpage
\subsection{Sierra Leone}


% \subsubsection{Summary of DHS surveys}

%%%%%%%%%%%%%%%%%%%%%%%%%%% Summary 


DHS surveys were conducted in Sierra Leone in 2013.
% years.out[1:(length(years.out)-1)], and years.out[length(years.out)]. 

We fit both the RW2 only model to the combined national data, and compare the time trend at national level with the estimates produced by the UN and IHME in Figure~\ref{fig:unnamed-chunk-292}. We then adjusted the combined national data to the UN estimates of U5MR, and refit the models on the benchmarked data. 

%%%%%%%%%%%%%%%%%%%%%%%%%% Plot5 
\begin{knitrout}
\definecolor{shadecolor}{rgb}{0.969, 0.969, 0.969}\color{fgcolor}\begin{figure}[bht]

{\centering \includegraphics[width=.9\textwidth]{../Main/Figures/Yearly_national_SierraLeone} 

}

\caption[Temporal national trends along with UN (B3) estimates described in You et al]{Temporal national trends along with UN (B3) estimates described in You et al. (2015) and IHME estimates based on GBD 2015 Child Mortality Collaborators (2016). RW2 represents the smoothed national estimates using the original data before benchmarking with UN estimates. RW2-adj represents the smoothed national estimates using the benchmarked data.}\label{fig:unnamed-chunk-292}
\end{figure}


\end{knitrout}
 

We fit the RW2 model to the benchmarked data in each area. 
% The proportions of the explained variation is summarized in Table~\ref{tab:paste0(countryname, "-var")}. 
We compare the results in Figure~\ref{fig:unnamed-chunk-293} to \ref{fig:unnamed-chunk-297}.
Figure~\ref{fig:unnamed-chunk-293} compares the smoothed estimates against the direct estimates. Figure~\ref{fig:unnamed-chunk-294} and Figure~\ref{fig:unnamed-chunk-295} show the posterior median estimates of U5MR in each region over time and the reductions from 1990 period respectively.
Figure~\ref{fig:unnamed-chunk-296} shows the smoothed estimates by region over time and Figure~\ref{fig:unnamed-chunk-297} compares the smoothed estimates with direct estimates from each survey for each region over time.


% %%%%%%%%%%%%%%%%%%%%%%%%%%% Table1 
% <<echo=FALSE, results='asis'>>=
% load("rda/variance_tables.rda")
% countryname2 <- gsub(" ", "", countryname)
% variance <- tables.all[[countryname]]

% table_count <- table_count + 1

% names <- c("RW2 ($\\sigma^2_{\\gamma_{t}}$)", "ICAR ($\\sigma^2_{\\phi_{i}}$)", "IID space ($\\sigma^2_{\\theta_{i}}$)", "IID time ($\\sigma^2_{\\alpha_{t}}$)", "IID space time ($\\sigma^2_{\\delta_{it}}$)")

% variance$Proportion <- round(variance$Proportion*100, digits = 2)
% row.names(variance) <- names
% tab <- xtable(variance, digits = c(1, 3, 2),align = "l|ll",
%        label = paste0("tab:", countryname, "-var"),
%        caption = paste(country, ": summary of the variance components in the RW2 model", sep = ''))
% print(tab, comment = FALSE,sanitize.text.function = function(x) {x})
% @

%%%%%%%%%%%%%%%%%%%%%%%%%%% Plot1 
\begin{knitrout}
\definecolor{shadecolor}{rgb}{0.969, 0.969, 0.969}\color{fgcolor}\begin{figure}[bht]

{\centering \includegraphics[width=.9\textwidth]{../Main/Figures/SmoothvDirectSierraLeone_meta} 

}

\caption[Smooth versus direct Admin 1 estimates]{Smooth versus direct Admin 1 estimates. Left: Combined (meta-analysis) survey estimate against combined direct estimates. Right: Combined (meta-analysis) survey estimate against direct estimates from each survey.}\label{fig:unnamed-chunk-293}
\end{figure}


\end{knitrout}

%%%%%%%%%%%%%%%%%%%%%%%%%%% Plot2 
\begin{knitrout}
\definecolor{shadecolor}{rgb}{0.969, 0.969, 0.969}\color{fgcolor}\begin{figure}[bht]

{\centering \includegraphics[width=.9\textwidth]{../Main/Figures/SmoothMedianSierraLeone} 

}

\caption[Maps of posterior medians for Sierra Leone  over time]{Maps of posterior medians for Sierra Leone  over time.}\label{fig:unnamed-chunk-294}
\end{figure}


\end{knitrout}
%%%%%%%%%%%%%%%%%%%%%%%%%%% Plot2a
\begin{knitrout}
\definecolor{shadecolor}{rgb}{0.969, 0.969, 0.969}\color{fgcolor}\begin{figure}[bht]

{\centering \includegraphics[width=.9\textwidth]{../Main/Figures/ReductionMedianSierraLeone} 

}

\caption[Maps of reduction of posterior median U5MR in each five-year period compared to 1990 in Sierra Leone over time]{Maps of reduction of posterior median U5MR in each five-year period compared to 1990 in Sierra Leone over time.}\label{fig:unnamed-chunk-295}
\end{figure}


\end{knitrout}
%%%%%%%%%%%%%%%%%%%%%%%%%%% Plot3 
\begin{knitrout}
\definecolor{shadecolor}{rgb}{0.969, 0.969, 0.969}\color{fgcolor}\begin{figure}[bht]

{\centering \includegraphics[width=.95\textwidth]{../Main/Figures/Yearly_v_Periods_SierraLeone} 

}

\caption[Smoothed regional estimates over time]{Smoothed regional estimates over time. The line indicates yearly posterior median estimates and error bars indicate 95 \% posterior credible interval at each time period.}\label{fig:unnamed-chunk-296}
\end{figure}


\end{knitrout}

%%%%%%%%%%%%%%%%%%%%%%%%%%% Plot4 
\begin{knitrout}
\definecolor{shadecolor}{rgb}{0.969, 0.969, 0.969}\color{fgcolor}\begin{figure}[bht]

{\centering \includegraphics[width=.9\textwidth]{../Main/Figures/LineSubMedianSierraLeone} 

}

\caption[Smoothed regional estimates over time compared to the direct estimates from each surveys]{Smoothed regional estimates over time compared to the direct estimates from each surveys. Direct estimates are not benchmarked with UN estimates. The line indicates posterior median and error bars indicate 95\% posterior credible interval.}\label{fig:unnamed-chunk-297}
\end{figure}


\end{knitrout}
% \subsubsection{National model results}
We further assess the RW2 model by holding out some observations, and compare the projections to the direct estimates in these holdout observations. Figure~\ref{fig:unnamed-chunk-298} compares the predicted estimates for the out-of-sample observations  with the direct estimates by holding out observations from each area in each time period.  Figure~\ref{fig:unnamed-chunk-299} compares the histogram of the bias rescaled by the total variance in the cross validation studies. Figure~\ref{fig:unnamed-chunk-300} compares the rescaled bias by region and time periods.



% %%%%%%%%%%%%%%%%%%%%%%%%%%% Plot6
% << echo=FALSE, out.width = ".9\\textwidth", fig.width = 12, fig.height = 6, fig.cap = "Out-of-sample predictions along with direct estimates in the cross validation study where all data from each time period is held out and predicted using the rest of the data.">>=
% fig_count <- fig_count + 1
% knitr::include_graphics(paste0("../Main/Figures/CV_byYear_withError_", countryname2, ".pdf")) 
% @
 
%%%%%%%%%%%%%%%%%%%%%%%%%%% Plot7
\begin{knitrout}
\definecolor{shadecolor}{rgb}{0.969, 0.969, 0.969}\color{fgcolor}\begin{figure}[bht]

{\centering \includegraphics[width=.9\textwidth]{../Main/Figures/CV_byYearRegion_withError_SierraLeone} 

}

\caption[Out-of-sample predictions along with direct estimates in the cross validation study where data from one region in each time period is held out and predicted using the rest of the data]{Out-of-sample predictions along with direct estimates in the cross validation study where data from one region in each time period is held out and predicted using the rest of the data.}\label{fig:unnamed-chunk-298}
\end{figure}


\end{knitrout}

%%%%%%%%%%%%%%%%%%%%%%%%%%% Plot8
\begin{knitrout}
\definecolor{shadecolor}{rgb}{0.969, 0.969, 0.969}\color{fgcolor}\begin{figure}[bht]

{\centering \includegraphics[width=.9\textwidth]{../Main/Figures/CVbiasSierraLeone} 

}

\caption[Histogram and QQ-plot of the rescaled difference between the smoothed estimates and the direct estimates in the cross validation study]{Histogram and QQ-plot of the rescaled difference between the smoothed estimates and the direct estimates in the cross validation study. The differences between the two estimates are rescaled by the square root of the total variance of the two estimates.}\label{fig:unnamed-chunk-299}
\end{figure}


\end{knitrout}

%%%%%%%%%%%%%%%%%%%%%%%%%%% Plot9
\begin{knitrout}
\definecolor{shadecolor}{rgb}{0.969, 0.969, 0.969}\color{fgcolor}\begin{figure}[bht]

{\centering \includegraphics[width=.7\textwidth]{../Main/Figures/CVbiasbyRegionSierraLeone} 

}

\caption[Line plot of the difference between smoothed estimates and the direct estimates in the cross validation study]{Line plot of the difference between smoothed estimates and the direct estimates in the cross validation study. The differences between the two estimates are rescaled by the square root of the total variance of the two estimates.}\label{fig:unnamed-chunk-300}
\end{figure}


\end{knitrout}


%%%%%%%%%%%%%%%%%%%%%%%%%%%%%%%%%%%%%%%%%%%%%%%%%%%%%%%%%%%%%%%%%%%%%%%%%%%%%%%%%%%%%%%%%%%%%%%%%%
\clearpage
\subsection{Tanzania}


% \subsubsection{Summary of DHS surveys}

%%%%%%%%%%%%%%%%%%%%%%%%%%% Summary 


DHS surveys were conducted in Tanzania in 1996, 1999, 2005, 2008, 2010, 2012, and 2015.
% years.out[1:(length(years.out)-1)], and years.out[length(years.out)]. 

We fit both the RW2 only model to the combined national data, and compare the time trend at national level with the estimates produced by the UN and IHME in Figure~\ref{fig:unnamed-chunk-302}. We then adjusted the combined national data to the UN estimates of U5MR, and refit the models on the benchmarked data. 

%%%%%%%%%%%%%%%%%%%%%%%%%% Plot5 
\begin{knitrout}
\definecolor{shadecolor}{rgb}{0.969, 0.969, 0.969}\color{fgcolor}\begin{figure}[bht]

{\centering \includegraphics[width=.9\textwidth]{../Main/Figures/Yearly_national_Tanzania} 

}

\caption[Temporal national trends along with UN (B3) estimates described in You et al]{Temporal national trends along with UN (B3) estimates described in You et al. (2015) and IHME estimates based on GBD 2015 Child Mortality Collaborators (2016). RW2 represents the smoothed national estimates using the original data before benchmarking with UN estimates. RW2-adj represents the smoothed national estimates using the benchmarked data.}\label{fig:unnamed-chunk-302}
\end{figure}


\end{knitrout}
 

We fit the RW2 model to the benchmarked data in each area. 
% The proportions of the explained variation is summarized in Table~\ref{tab:paste0(countryname, "-var")}. 
We compare the results in Figure~\ref{fig:unnamed-chunk-303} to \ref{fig:unnamed-chunk-307}.
Figure~\ref{fig:unnamed-chunk-303} compares the smoothed estimates against the direct estimates. Figure~\ref{fig:unnamed-chunk-304} and Figure~\ref{fig:unnamed-chunk-305} show the posterior median estimates of U5MR in each region over time and the reductions from 1990 period respectively.
Figure~\ref{fig:unnamed-chunk-306} shows the smoothed estimates by region over time and Figure~\ref{fig:unnamed-chunk-307} compares the smoothed estimates with direct estimates from each survey for each region over time.


% %%%%%%%%%%%%%%%%%%%%%%%%%%% Table1 
% <<echo=FALSE, results='asis'>>=
% load("rda/variance_tables.rda")
% countryname2 <- gsub(" ", "", countryname)
% variance <- tables.all[[countryname]]

% table_count <- table_count + 1

% names <- c("RW2 ($\\sigma^2_{\\gamma_{t}}$)", "ICAR ($\\sigma^2_{\\phi_{i}}$)", "IID space ($\\sigma^2_{\\theta_{i}}$)", "IID time ($\\sigma^2_{\\alpha_{t}}$)", "IID space time ($\\sigma^2_{\\delta_{it}}$)")

% variance$Proportion <- round(variance$Proportion*100, digits = 2)
% row.names(variance) <- names
% tab <- xtable(variance, digits = c(1, 3, 2),align = "l|ll",
%        label = paste0("tab:", countryname, "-var"),
%        caption = paste(country, ": summary of the variance components in the RW2 model", sep = ''))
% print(tab, comment = FALSE,sanitize.text.function = function(x) {x})
% @

%%%%%%%%%%%%%%%%%%%%%%%%%%% Plot1 
\begin{knitrout}
\definecolor{shadecolor}{rgb}{0.969, 0.969, 0.969}\color{fgcolor}\begin{figure}[bht]

{\centering \includegraphics[width=.9\textwidth]{../Main/Figures/SmoothvDirectTanzania_meta} 

}

\caption[Smooth versus direct Admin 1 estimates]{Smooth versus direct Admin 1 estimates. Left: Combined (meta-analysis) survey estimate against combined direct estimates. Right: Combined (meta-analysis) survey estimate against direct estimates from each survey.}\label{fig:unnamed-chunk-303}
\end{figure}


\end{knitrout}

%%%%%%%%%%%%%%%%%%%%%%%%%%% Plot2 
\begin{knitrout}
\definecolor{shadecolor}{rgb}{0.969, 0.969, 0.969}\color{fgcolor}\begin{figure}[bht]

{\centering \includegraphics[width=.9\textwidth]{../Main/Figures/SmoothMedianTanzania} 

}

\caption[Maps of posterior medians for Tanzania  over time]{Maps of posterior medians for Tanzania  over time.}\label{fig:unnamed-chunk-304}
\end{figure}


\end{knitrout}
%%%%%%%%%%%%%%%%%%%%%%%%%%% Plot2a
\begin{knitrout}
\definecolor{shadecolor}{rgb}{0.969, 0.969, 0.969}\color{fgcolor}\begin{figure}[bht]

{\centering \includegraphics[width=.9\textwidth]{../Main/Figures/ReductionMedianTanzania} 

}

\caption[Maps of reduction of posterior median U5MR in each five-year period compared to 1990 in Tanzania over time]{Maps of reduction of posterior median U5MR in each five-year period compared to 1990 in Tanzania over time.}\label{fig:unnamed-chunk-305}
\end{figure}


\end{knitrout}
%%%%%%%%%%%%%%%%%%%%%%%%%%% Plot3 
\begin{knitrout}
\definecolor{shadecolor}{rgb}{0.969, 0.969, 0.969}\color{fgcolor}\begin{figure}[bht]

{\centering \includegraphics[width=.95\textwidth]{../Main/Figures/Yearly_v_Periods_Tanzania} 

}

\caption[Smoothed regional estimates over time]{Smoothed regional estimates over time. The line indicates yearly posterior median estimates and error bars indicate 95 \% posterior credible interval at each time period.}\label{fig:unnamed-chunk-306}
\end{figure}


\end{knitrout}

%%%%%%%%%%%%%%%%%%%%%%%%%%% Plot4 
\begin{knitrout}
\definecolor{shadecolor}{rgb}{0.969, 0.969, 0.969}\color{fgcolor}\begin{figure}[bht]

{\centering \includegraphics[width=.9\textwidth]{../Main/Figures/LineSubMedianTanzania} 

}

\caption[Smoothed regional estimates over time compared to the direct estimates from each surveys]{Smoothed regional estimates over time compared to the direct estimates from each surveys. Direct estimates are not benchmarked with UN estimates. The line indicates posterior median and error bars indicate 95\% posterior credible interval.}\label{fig:unnamed-chunk-307}
\end{figure}


\end{knitrout}
% \subsubsection{National model results}
We further assess the RW2 model by holding out some observations, and compare the projections to the direct estimates in these holdout observations. Figure~\ref{fig:unnamed-chunk-308} compares the predicted estimates for the out-of-sample observations  with the direct estimates by holding out observations from each area in each time period.  Figure~\ref{fig:unnamed-chunk-309} compares the histogram of the bias rescaled by the total variance in the cross validation studies. Figure~\ref{fig:unnamed-chunk-310} compares the rescaled bias by region and time periods.



% %%%%%%%%%%%%%%%%%%%%%%%%%%% Plot6
% << echo=FALSE, out.width = ".9\\textwidth", fig.width = 12, fig.height = 6, fig.cap = "Out-of-sample predictions along with direct estimates in the cross validation study where all data from each time period is held out and predicted using the rest of the data.">>=
% fig_count <- fig_count + 1
% knitr::include_graphics(paste0("../Main/Figures/CV_byYear_withError_", countryname2, ".pdf")) 
% @
 
%%%%%%%%%%%%%%%%%%%%%%%%%%% Plot7
\begin{knitrout}
\definecolor{shadecolor}{rgb}{0.969, 0.969, 0.969}\color{fgcolor}\begin{figure}[bht]

{\centering \includegraphics[width=.9\textwidth]{../Main/Figures/CV_byYearRegion_withError_Tanzania} 

}

\caption[Out-of-sample predictions along with direct estimates in the cross validation study where data from one region in each time period is held out and predicted using the rest of the data]{Out-of-sample predictions along with direct estimates in the cross validation study where data from one region in each time period is held out and predicted using the rest of the data.}\label{fig:unnamed-chunk-308}
\end{figure}


\end{knitrout}

%%%%%%%%%%%%%%%%%%%%%%%%%%% Plot8
\begin{knitrout}
\definecolor{shadecolor}{rgb}{0.969, 0.969, 0.969}\color{fgcolor}\begin{figure}[bht]

{\centering \includegraphics[width=.9\textwidth]{../Main/Figures/CVbiasTanzania} 

}

\caption[Histogram and QQ-plot of the rescaled difference between the smoothed estimates and the direct estimates in the cross validation study]{Histogram and QQ-plot of the rescaled difference between the smoothed estimates and the direct estimates in the cross validation study. The differences between the two estimates are rescaled by the square root of the total variance of the two estimates.}\label{fig:unnamed-chunk-309}
\end{figure}


\end{knitrout}

%%%%%%%%%%%%%%%%%%%%%%%%%%% Plot9
\begin{knitrout}
\definecolor{shadecolor}{rgb}{0.969, 0.969, 0.969}\color{fgcolor}\begin{figure}[bht]

{\centering \includegraphics[width=.7\textwidth]{../Main/Figures/CVbiasbyRegionTanzania} 

}

\caption[Line plot of the difference between smoothed estimates and the direct estimates in the cross validation study]{Line plot of the difference between smoothed estimates and the direct estimates in the cross validation study. The differences between the two estimates are rescaled by the square root of the total variance of the two estimates.}\label{fig:unnamed-chunk-310}
\end{figure}


\end{knitrout}


%%%%%%%%%%%%%%%%%%%%%%%%%%%%%%%%%%%%%%%%%%%%%%%%%%%%%%%%%%%%%%%%%%%%%%%%%%%%%%%%%%%%%%%%%%%%%%%%%%
\clearpage
\subsection{Togo}


% \subsubsection{Summary of DHS surveys}

%%%%%%%%%%%%%%%%%%%%%%%%%%% Summary 


DHS surveys were conducted in Togo in 1998, and 2013.
% years.out[1:(length(years.out)-1)], and years.out[length(years.out)]. 

We fit both the RW2 only model to the combined national data, and compare the time trend at national level with the estimates produced by the UN and IHME in Figure~\ref{fig:unnamed-chunk-312}. We then adjusted the combined national data to the UN estimates of U5MR, and refit the models on the benchmarked data. 

%%%%%%%%%%%%%%%%%%%%%%%%%% Plot5 
\begin{knitrout}
\definecolor{shadecolor}{rgb}{0.969, 0.969, 0.969}\color{fgcolor}\begin{figure}[bht]

{\centering \includegraphics[width=.9\textwidth]{../Main/Figures/Yearly_national_Togo} 

}

\caption[Temporal national trends along with UN (B3) estimates described in You et al]{Temporal national trends along with UN (B3) estimates described in You et al. (2015) and IHME estimates based on GBD 2015 Child Mortality Collaborators (2016). RW2 represents the smoothed national estimates using the original data before benchmarking with UN estimates. RW2-adj represents the smoothed national estimates using the benchmarked data.}\label{fig:unnamed-chunk-312}
\end{figure}


\end{knitrout}
 

We fit the RW2 model to the benchmarked data in each area. 
% The proportions of the explained variation is summarized in Table~\ref{tab:paste0(countryname, "-var")}. 
We compare the results in Figure~\ref{fig:unnamed-chunk-313} to \ref{fig:unnamed-chunk-317}.
Figure~\ref{fig:unnamed-chunk-313} compares the smoothed estimates against the direct estimates. Figure~\ref{fig:unnamed-chunk-314} and Figure~\ref{fig:unnamed-chunk-315} show the posterior median estimates of U5MR in each region over time and the reductions from 1990 period respectively.
Figure~\ref{fig:unnamed-chunk-316} shows the smoothed estimates by region over time and Figure~\ref{fig:unnamed-chunk-317} compares the smoothed estimates with direct estimates from each survey for each region over time.


% %%%%%%%%%%%%%%%%%%%%%%%%%%% Table1 
% <<echo=FALSE, results='asis'>>=
% load("rda/variance_tables.rda")
% countryname2 <- gsub(" ", "", countryname)
% variance <- tables.all[[countryname]]

% table_count <- table_count + 1

% names <- c("RW2 ($\\sigma^2_{\\gamma_{t}}$)", "ICAR ($\\sigma^2_{\\phi_{i}}$)", "IID space ($\\sigma^2_{\\theta_{i}}$)", "IID time ($\\sigma^2_{\\alpha_{t}}$)", "IID space time ($\\sigma^2_{\\delta_{it}}$)")

% variance$Proportion <- round(variance$Proportion*100, digits = 2)
% row.names(variance) <- names
% tab <- xtable(variance, digits = c(1, 3, 2),align = "l|ll",
%        label = paste0("tab:", countryname, "-var"),
%        caption = paste(country, ": summary of the variance components in the RW2 model", sep = ''))
% print(tab, comment = FALSE,sanitize.text.function = function(x) {x})
% @

%%%%%%%%%%%%%%%%%%%%%%%%%%% Plot1 
\begin{knitrout}
\definecolor{shadecolor}{rgb}{0.969, 0.969, 0.969}\color{fgcolor}\begin{figure}[bht]

{\centering \includegraphics[width=.9\textwidth]{../Main/Figures/SmoothvDirectTogo_meta} 

}

\caption[Smooth versus direct Admin 1 estimates]{Smooth versus direct Admin 1 estimates. Left: Combined (meta-analysis) survey estimate against combined direct estimates. Right: Combined (meta-analysis) survey estimate against direct estimates from each survey.}\label{fig:unnamed-chunk-313}
\end{figure}


\end{knitrout}

%%%%%%%%%%%%%%%%%%%%%%%%%%% Plot2 
\begin{knitrout}
\definecolor{shadecolor}{rgb}{0.969, 0.969, 0.969}\color{fgcolor}\begin{figure}[bht]

{\centering \includegraphics[width=.9\textwidth]{../Main/Figures/SmoothMedianTogo} 

}

\caption[Maps of posterior medians for Togo  over time]{Maps of posterior medians for Togo  over time.}\label{fig:unnamed-chunk-314}
\end{figure}


\end{knitrout}
%%%%%%%%%%%%%%%%%%%%%%%%%%% Plot2a
\begin{knitrout}
\definecolor{shadecolor}{rgb}{0.969, 0.969, 0.969}\color{fgcolor}\begin{figure}[bht]

{\centering \includegraphics[width=.9\textwidth]{../Main/Figures/ReductionMedianTogo} 

}

\caption[Maps of reduction of posterior median U5MR in each five-year period compared to 1990 in Togo over time]{Maps of reduction of posterior median U5MR in each five-year period compared to 1990 in Togo over time.}\label{fig:unnamed-chunk-315}
\end{figure}


\end{knitrout}
%%%%%%%%%%%%%%%%%%%%%%%%%%% Plot3 
\begin{knitrout}
\definecolor{shadecolor}{rgb}{0.969, 0.969, 0.969}\color{fgcolor}\begin{figure}[bht]

{\centering \includegraphics[width=.95\textwidth]{../Main/Figures/Yearly_v_Periods_Togo} 

}

\caption[Smoothed regional estimates over time]{Smoothed regional estimates over time. The line indicates yearly posterior median estimates and error bars indicate 95 \% posterior credible interval at each time period.}\label{fig:unnamed-chunk-316}
\end{figure}


\end{knitrout}

%%%%%%%%%%%%%%%%%%%%%%%%%%% Plot4 
\begin{knitrout}
\definecolor{shadecolor}{rgb}{0.969, 0.969, 0.969}\color{fgcolor}\begin{figure}[bht]

{\centering \includegraphics[width=.9\textwidth]{../Main/Figures/LineSubMedianTogo} 

}

\caption[Smoothed regional estimates over time compared to the direct estimates from each surveys]{Smoothed regional estimates over time compared to the direct estimates from each surveys. Direct estimates are not benchmarked with UN estimates. The line indicates posterior median and error bars indicate 95\% posterior credible interval.}\label{fig:unnamed-chunk-317}
\end{figure}


\end{knitrout}
% \subsubsection{National model results}
We further assess the RW2 model by holding out some observations, and compare the projections to the direct estimates in these holdout observations. Figure~\ref{fig:unnamed-chunk-318} compares the predicted estimates for the out-of-sample observations  with the direct estimates by holding out observations from each area in each time period.  Figure~\ref{fig:unnamed-chunk-319} compares the histogram of the bias rescaled by the total variance in the cross validation studies. Figure~\ref{fig:unnamed-chunk-320} compares the rescaled bias by region and time periods.



% %%%%%%%%%%%%%%%%%%%%%%%%%%% Plot6
% << echo=FALSE, out.width = ".9\\textwidth", fig.width = 12, fig.height = 6, fig.cap = "Out-of-sample predictions along with direct estimates in the cross validation study where all data from each time period is held out and predicted using the rest of the data.">>=
% fig_count <- fig_count + 1
% knitr::include_graphics(paste0("../Main/Figures/CV_byYear_withError_", countryname2, ".pdf")) 
% @
 
%%%%%%%%%%%%%%%%%%%%%%%%%%% Plot7
\begin{knitrout}
\definecolor{shadecolor}{rgb}{0.969, 0.969, 0.969}\color{fgcolor}\begin{figure}[bht]

{\centering \includegraphics[width=.9\textwidth]{../Main/Figures/CV_byYearRegion_withError_Togo} 

}

\caption[Out-of-sample predictions along with direct estimates in the cross validation study where data from one region in each time period is held out and predicted using the rest of the data]{Out-of-sample predictions along with direct estimates in the cross validation study where data from one region in each time period is held out and predicted using the rest of the data.}\label{fig:unnamed-chunk-318}
\end{figure}


\end{knitrout}

%%%%%%%%%%%%%%%%%%%%%%%%%%% Plot8
\begin{knitrout}
\definecolor{shadecolor}{rgb}{0.969, 0.969, 0.969}\color{fgcolor}\begin{figure}[bht]

{\centering \includegraphics[width=.9\textwidth]{../Main/Figures/CVbiasTogo} 

}

\caption[Histogram and QQ-plot of the rescaled difference between the smoothed estimates and the direct estimates in the cross validation study]{Histogram and QQ-plot of the rescaled difference between the smoothed estimates and the direct estimates in the cross validation study. The differences between the two estimates are rescaled by the square root of the total variance of the two estimates.}\label{fig:unnamed-chunk-319}
\end{figure}


\end{knitrout}

%%%%%%%%%%%%%%%%%%%%%%%%%%% Plot9
\begin{knitrout}
\definecolor{shadecolor}{rgb}{0.969, 0.969, 0.969}\color{fgcolor}\begin{figure}[bht]

{\centering \includegraphics[width=.7\textwidth]{../Main/Figures/CVbiasbyRegionTogo} 

}

\caption[Line plot of the difference between smoothed estimates and the direct estimates in the cross validation study]{Line plot of the difference between smoothed estimates and the direct estimates in the cross validation study. The differences between the two estimates are rescaled by the square root of the total variance of the two estimates.}\label{fig:unnamed-chunk-320}
\end{figure}


\end{knitrout}


%%%%%%%%%%%%%%%%%%%%%%%%%%%%%%%%%%%%%%%%%%%%%%%%%%%%%%%%%%%%%%%%%%%%%%%%%%%%%%%%%%%%%%%%%%%%%%%%%%
\clearpage
\subsection{Uganda}


% \subsubsection{Summary of DHS surveys}

%%%%%%%%%%%%%%%%%%%%%%%%%%% Summary 


DHS surveys were conducted in Uganda in 1989, 1995, 2001, 2006, and 2011.
% years.out[1:(length(years.out)-1)], and years.out[length(years.out)]. 

We fit both the RW2 only model to the combined national data, and compare the time trend at national level with the estimates produced by the UN and IHME in Figure~\ref{fig:unnamed-chunk-322}. We then adjusted the combined national data to the UN estimates of U5MR, and refit the models on the benchmarked data. 

%%%%%%%%%%%%%%%%%%%%%%%%%% Plot5 
\begin{knitrout}
\definecolor{shadecolor}{rgb}{0.969, 0.969, 0.969}\color{fgcolor}\begin{figure}[bht]

{\centering \includegraphics[width=.9\textwidth]{../Main/Figures/Yearly_national_Uganda} 

}

\caption[Temporal national trends along with UN (B3) estimates described in You et al]{Temporal national trends along with UN (B3) estimates described in You et al. (2015) and IHME estimates based on GBD 2015 Child Mortality Collaborators (2016). RW2 represents the smoothed national estimates using the original data before benchmarking with UN estimates. RW2-adj represents the smoothed national estimates using the benchmarked data.}\label{fig:unnamed-chunk-322}
\end{figure}


\end{knitrout}
 

We fit the RW2 model to the benchmarked data in each area. 
% The proportions of the explained variation is summarized in Table~\ref{tab:paste0(countryname, "-var")}. 
We compare the results in Figure~\ref{fig:unnamed-chunk-323} to \ref{fig:unnamed-chunk-327}.
Figure~\ref{fig:unnamed-chunk-323} compares the smoothed estimates against the direct estimates. Figure~\ref{fig:unnamed-chunk-324} and Figure~\ref{fig:unnamed-chunk-325} show the posterior median estimates of U5MR in each region over time and the reductions from 1990 period respectively.
Figure~\ref{fig:unnamed-chunk-326} shows the smoothed estimates by region over time and Figure~\ref{fig:unnamed-chunk-327} compares the smoothed estimates with direct estimates from each survey for each region over time.


% %%%%%%%%%%%%%%%%%%%%%%%%%%% Table1 
% <<echo=FALSE, results='asis'>>=
% load("rda/variance_tables.rda")
% countryname2 <- gsub(" ", "", countryname)
% variance <- tables.all[[countryname]]

% table_count <- table_count + 1

% names <- c("RW2 ($\\sigma^2_{\\gamma_{t}}$)", "ICAR ($\\sigma^2_{\\phi_{i}}$)", "IID space ($\\sigma^2_{\\theta_{i}}$)", "IID time ($\\sigma^2_{\\alpha_{t}}$)", "IID space time ($\\sigma^2_{\\delta_{it}}$)")

% variance$Proportion <- round(variance$Proportion*100, digits = 2)
% row.names(variance) <- names
% tab <- xtable(variance, digits = c(1, 3, 2),align = "l|ll",
%        label = paste0("tab:", countryname, "-var"),
%        caption = paste(country, ": summary of the variance components in the RW2 model", sep = ''))
% print(tab, comment = FALSE,sanitize.text.function = function(x) {x})
% @

%%%%%%%%%%%%%%%%%%%%%%%%%%% Plot1 
\begin{knitrout}
\definecolor{shadecolor}{rgb}{0.969, 0.969, 0.969}\color{fgcolor}\begin{figure}[bht]

{\centering \includegraphics[width=.9\textwidth]{../Main/Figures/SmoothvDirectUganda_meta} 

}

\caption[Smooth versus direct Admin 1 estimates]{Smooth versus direct Admin 1 estimates. Left: Combined (meta-analysis) survey estimate against combined direct estimates. Right: Combined (meta-analysis) survey estimate against direct estimates from each survey.}\label{fig:unnamed-chunk-323}
\end{figure}


\end{knitrout}

%%%%%%%%%%%%%%%%%%%%%%%%%%% Plot2 
\begin{knitrout}
\definecolor{shadecolor}{rgb}{0.969, 0.969, 0.969}\color{fgcolor}\begin{figure}[bht]

{\centering \includegraphics[width=.9\textwidth]{../Main/Figures/SmoothMedianUganda} 

}

\caption[Maps of posterior medians for Uganda  over time]{Maps of posterior medians for Uganda  over time.}\label{fig:unnamed-chunk-324}
\end{figure}


\end{knitrout}
%%%%%%%%%%%%%%%%%%%%%%%%%%% Plot2a
\begin{knitrout}
\definecolor{shadecolor}{rgb}{0.969, 0.969, 0.969}\color{fgcolor}\begin{figure}[bht]

{\centering \includegraphics[width=.9\textwidth]{../Main/Figures/ReductionMedianUganda} 

}

\caption[Maps of reduction of posterior median U5MR in each five-year period compared to 1990 in Uganda over time]{Maps of reduction of posterior median U5MR in each five-year period compared to 1990 in Uganda over time.}\label{fig:unnamed-chunk-325}
\end{figure}


\end{knitrout}
%%%%%%%%%%%%%%%%%%%%%%%%%%% Plot3 
\begin{knitrout}
\definecolor{shadecolor}{rgb}{0.969, 0.969, 0.969}\color{fgcolor}\begin{figure}[bht]

{\centering \includegraphics[width=.95\textwidth]{../Main/Figures/Yearly_v_Periods_Uganda} 

}

\caption[Smoothed regional estimates over time]{Smoothed regional estimates over time. The line indicates yearly posterior median estimates and error bars indicate 95 \% posterior credible interval at each time period.}\label{fig:unnamed-chunk-326}
\end{figure}


\end{knitrout}

%%%%%%%%%%%%%%%%%%%%%%%%%%% Plot4 
\begin{knitrout}
\definecolor{shadecolor}{rgb}{0.969, 0.969, 0.969}\color{fgcolor}\begin{figure}[bht]

{\centering \includegraphics[width=.9\textwidth]{../Main/Figures/LineSubMedianUganda} 

}

\caption[Smoothed regional estimates over time compared to the direct estimates from each surveys]{Smoothed regional estimates over time compared to the direct estimates from each surveys. Direct estimates are not benchmarked with UN estimates. The line indicates posterior median and error bars indicate 95\% posterior credible interval.}\label{fig:unnamed-chunk-327}
\end{figure}


\end{knitrout}
% \subsubsection{National model results}
We further assess the RW2 model by holding out some observations, and compare the projections to the direct estimates in these holdout observations. Figure~\ref{fig:unnamed-chunk-328} compares the predicted estimates for the out-of-sample observations  with the direct estimates by holding out observations from each area in each time period.  Figure~\ref{fig:unnamed-chunk-329} compares the histogram of the bias rescaled by the total variance in the cross validation studies. Figure~\ref{fig:unnamed-chunk-330} compares the rescaled bias by region and time periods.



% %%%%%%%%%%%%%%%%%%%%%%%%%%% Plot6
% << echo=FALSE, out.width = ".9\\textwidth", fig.width = 12, fig.height = 6, fig.cap = "Out-of-sample predictions along with direct estimates in the cross validation study where all data from each time period is held out and predicted using the rest of the data.">>=
% fig_count <- fig_count + 1
% knitr::include_graphics(paste0("../Main/Figures/CV_byYear_withError_", countryname2, ".pdf")) 
% @
 
%%%%%%%%%%%%%%%%%%%%%%%%%%% Plot7
\begin{knitrout}
\definecolor{shadecolor}{rgb}{0.969, 0.969, 0.969}\color{fgcolor}\begin{figure}[bht]

{\centering \includegraphics[width=.9\textwidth]{../Main/Figures/CV_byYearRegion_withError_Uganda} 

}

\caption[Out-of-sample predictions along with direct estimates in the cross validation study where data from one region in each time period is held out and predicted using the rest of the data]{Out-of-sample predictions along with direct estimates in the cross validation study where data from one region in each time period is held out and predicted using the rest of the data.}\label{fig:unnamed-chunk-328}
\end{figure}


\end{knitrout}

%%%%%%%%%%%%%%%%%%%%%%%%%%% Plot8
\begin{knitrout}
\definecolor{shadecolor}{rgb}{0.969, 0.969, 0.969}\color{fgcolor}\begin{figure}[bht]

{\centering \includegraphics[width=.9\textwidth]{../Main/Figures/CVbiasUganda} 

}

\caption[Histogram and QQ-plot of the rescaled difference between the smoothed estimates and the direct estimates in the cross validation study]{Histogram and QQ-plot of the rescaled difference between the smoothed estimates and the direct estimates in the cross validation study. The differences between the two estimates are rescaled by the square root of the total variance of the two estimates.}\label{fig:unnamed-chunk-329}
\end{figure}


\end{knitrout}

%%%%%%%%%%%%%%%%%%%%%%%%%%% Plot9
\begin{knitrout}
\definecolor{shadecolor}{rgb}{0.969, 0.969, 0.969}\color{fgcolor}\begin{figure}[bht]

{\centering \includegraphics[width=.7\textwidth]{../Main/Figures/CVbiasbyRegionUganda} 

}

\caption[Line plot of the difference between smoothed estimates and the direct estimates in the cross validation study]{Line plot of the difference between smoothed estimates and the direct estimates in the cross validation study. The differences between the two estimates are rescaled by the square root of the total variance of the two estimates.}\label{fig:unnamed-chunk-330}
\end{figure}


\end{knitrout}



%%%%%%%%%%%%%%%%%%%%%%%%%%%%%%%%%%%%%%%%%%%%%%%%%%%%%%%%%%%%%%%%%%%%%%%%%%%%%%%%%%%%%%%%%%%%%%%%%%
\clearpage
\subsection{Zambia}


% \subsubsection{Summary of DHS surveys}

%%%%%%%%%%%%%%%%%%%%%%%%%%% Summary 


DHS surveys were conducted in Zambia in 1992, 1996, 2007, and 2014.
% years.out[1:(length(years.out)-1)], and years.out[length(years.out)]. 

We fit both the RW2 only model to the combined national data, and compare the time trend at national level with the estimates produced by the UN and IHME in Figure~\ref{fig:unnamed-chunk-332}. We then adjusted the combined national data to the UN estimates of U5MR, and refit the models on the benchmarked data. 

%%%%%%%%%%%%%%%%%%%%%%%%%% Plot5 
\begin{knitrout}
\definecolor{shadecolor}{rgb}{0.969, 0.969, 0.969}\color{fgcolor}\begin{figure}[bht]

{\centering \includegraphics[width=.9\textwidth]{../Main/Figures/Yearly_national_Zambia} 

}

\caption[Temporal national trends along with UN (B3) estimates described in You et al]{Temporal national trends along with UN (B3) estimates described in You et al. (2015) and IHME estimates based on GBD 2015 Child Mortality Collaborators (2016). RW2 represents the smoothed national estimates using the original data before benchmarking with UN estimates. RW2-adj represents the smoothed national estimates using the benchmarked data.}\label{fig:unnamed-chunk-332}
\end{figure}


\end{knitrout}
 

We fit the RW2 model to the benchmarked data in each area. 
% The proportions of the explained variation is summarized in Table~\ref{tab:paste0(countryname, "-var")}. 
We compare the results in Figure~\ref{fig:unnamed-chunk-333} to \ref{fig:unnamed-chunk-337}.
Figure~\ref{fig:unnamed-chunk-333} compares the smoothed estimates against the direct estimates. Figure~\ref{fig:unnamed-chunk-334} and Figure~\ref{fig:unnamed-chunk-335} show the posterior median estimates of U5MR in each region over time and the reductions from 1990 period respectively.
Figure~\ref{fig:unnamed-chunk-336} shows the smoothed estimates by region over time and Figure~\ref{fig:unnamed-chunk-337} compares the smoothed estimates with direct estimates from each survey for each region over time.


% %%%%%%%%%%%%%%%%%%%%%%%%%%% Table1 
% <<echo=FALSE, results='asis'>>=
% load("rda/variance_tables.rda")
% countryname2 <- gsub(" ", "", countryname)
% variance <- tables.all[[countryname]]

% table_count <- table_count + 1

% names <- c("RW2 ($\\sigma^2_{\\gamma_{t}}$)", "ICAR ($\\sigma^2_{\\phi_{i}}$)", "IID space ($\\sigma^2_{\\theta_{i}}$)", "IID time ($\\sigma^2_{\\alpha_{t}}$)", "IID space time ($\\sigma^2_{\\delta_{it}}$)")

% variance$Proportion <- round(variance$Proportion*100, digits = 2)
% row.names(variance) <- names
% tab <- xtable(variance, digits = c(1, 3, 2),align = "l|ll",
%        label = paste0("tab:", countryname, "-var"),
%        caption = paste(country, ": summary of the variance components in the RW2 model", sep = ''))
% print(tab, comment = FALSE,sanitize.text.function = function(x) {x})
% @

%%%%%%%%%%%%%%%%%%%%%%%%%%% Plot1 
\begin{knitrout}
\definecolor{shadecolor}{rgb}{0.969, 0.969, 0.969}\color{fgcolor}\begin{figure}[bht]

{\centering \includegraphics[width=.9\textwidth]{../Main/Figures/SmoothvDirectZambia_meta} 

}

\caption[Smooth versus direct Admin 1 estimates]{Smooth versus direct Admin 1 estimates. Left: Combined (meta-analysis) survey estimate against combined direct estimates. Right: Combined (meta-analysis) survey estimate against direct estimates from each survey.}\label{fig:unnamed-chunk-333}
\end{figure}


\end{knitrout}

%%%%%%%%%%%%%%%%%%%%%%%%%%% Plot2 
\begin{knitrout}
\definecolor{shadecolor}{rgb}{0.969, 0.969, 0.969}\color{fgcolor}\begin{figure}[bht]

{\centering \includegraphics[width=.9\textwidth]{../Main/Figures/SmoothMedianZambia} 

}

\caption[Maps of posterior medians for Zambia  over time]{Maps of posterior medians for Zambia  over time.}\label{fig:unnamed-chunk-334}
\end{figure}


\end{knitrout}
%%%%%%%%%%%%%%%%%%%%%%%%%%% Plot2a
\begin{knitrout}
\definecolor{shadecolor}{rgb}{0.969, 0.969, 0.969}\color{fgcolor}\begin{figure}[bht]

{\centering \includegraphics[width=.9\textwidth]{../Main/Figures/ReductionMedianZambia} 

}

\caption[Maps of reduction of posterior median U5MR in each five-year period compared to 1990 in Zambia over time]{Maps of reduction of posterior median U5MR in each five-year period compared to 1990 in Zambia over time.}\label{fig:unnamed-chunk-335}
\end{figure}


\end{knitrout}
%%%%%%%%%%%%%%%%%%%%%%%%%%% Plot3 
\begin{knitrout}
\definecolor{shadecolor}{rgb}{0.969, 0.969, 0.969}\color{fgcolor}\begin{figure}[bht]

{\centering \includegraphics[width=.95\textwidth]{../Main/Figures/Yearly_v_Periods_Zambia} 

}

\caption[Smoothed regional estimates over time]{Smoothed regional estimates over time. The line indicates yearly posterior median estimates and error bars indicate 95 \% posterior credible interval at each time period.}\label{fig:unnamed-chunk-336}
\end{figure}


\end{knitrout}

%%%%%%%%%%%%%%%%%%%%%%%%%%% Plot4 
\begin{knitrout}
\definecolor{shadecolor}{rgb}{0.969, 0.969, 0.969}\color{fgcolor}\begin{figure}[bht]

{\centering \includegraphics[width=.9\textwidth]{../Main/Figures/LineSubMedianZambia} 

}

\caption[Smoothed regional estimates over time compared to the direct estimates from each surveys]{Smoothed regional estimates over time compared to the direct estimates from each surveys. Direct estimates are not benchmarked with UN estimates. The line indicates posterior median and error bars indicate 95\% posterior credible interval.}\label{fig:unnamed-chunk-337}
\end{figure}


\end{knitrout}
% \subsubsection{National model results}
We further assess the RW2 model by holding out some observations, and compare the projections to the direct estimates in these holdout observations. Figure~\ref{fig:unnamed-chunk-338} compares the predicted estimates for the out-of-sample observations  with the direct estimates by holding out observations from each area in each time period.  Figure~\ref{fig:unnamed-chunk-339} compares the histogram of the bias rescaled by the total variance in the cross validation studies. Figure~\ref{fig:unnamed-chunk-340} compares the rescaled bias by region and time periods.



% %%%%%%%%%%%%%%%%%%%%%%%%%%% Plot6
% << echo=FALSE, out.width = ".9\\textwidth", fig.width = 12, fig.height = 6, fig.cap = "Out-of-sample predictions along with direct estimates in the cross validation study where all data from each time period is held out and predicted using the rest of the data.">>=
% fig_count <- fig_count + 1
% knitr::include_graphics(paste0("../Main/Figures/CV_byYear_withError_", countryname2, ".pdf")) 
% @
 
%%%%%%%%%%%%%%%%%%%%%%%%%%% Plot7
\begin{knitrout}
\definecolor{shadecolor}{rgb}{0.969, 0.969, 0.969}\color{fgcolor}\begin{figure}[bht]

{\centering \includegraphics[width=.9\textwidth]{../Main/Figures/CV_byYearRegion_withError_Zambia} 

}

\caption[Out-of-sample predictions along with direct estimates in the cross validation study where data from one region in each time period is held out and predicted using the rest of the data]{Out-of-sample predictions along with direct estimates in the cross validation study where data from one region in each time period is held out and predicted using the rest of the data.}\label{fig:unnamed-chunk-338}
\end{figure}


\end{knitrout}

%%%%%%%%%%%%%%%%%%%%%%%%%%% Plot8
\begin{knitrout}
\definecolor{shadecolor}{rgb}{0.969, 0.969, 0.969}\color{fgcolor}\begin{figure}[bht]

{\centering \includegraphics[width=.9\textwidth]{../Main/Figures/CVbiasZambia} 

}

\caption[Histogram and QQ-plot of the rescaled difference between the smoothed estimates and the direct estimates in the cross validation study]{Histogram and QQ-plot of the rescaled difference between the smoothed estimates and the direct estimates in the cross validation study. The differences between the two estimates are rescaled by the square root of the total variance of the two estimates.}\label{fig:unnamed-chunk-339}
\end{figure}


\end{knitrout}

%%%%%%%%%%%%%%%%%%%%%%%%%%% Plot9
\begin{knitrout}
\definecolor{shadecolor}{rgb}{0.969, 0.969, 0.969}\color{fgcolor}\begin{figure}[bht]

{\centering \includegraphics[width=.7\textwidth]{../Main/Figures/CVbiasbyRegionZambia} 

}

\caption[Line plot of the difference between smoothed estimates and the direct estimates in the cross validation study]{Line plot of the difference between smoothed estimates and the direct estimates in the cross validation study. The differences between the two estimates are rescaled by the square root of the total variance of the two estimates.}\label{fig:unnamed-chunk-340}
\end{figure}


\end{knitrout}



%%%%%%%%%%%%%%%%%%%%%%%%%%%%%%%%%%%%%%%%%%%%%%%%%%%%%%%%%%%%%%%%%%%%%%%%%%%%%%%%%%%%%%%%%%%%%%%%%%
\clearpage
\subsection{Zimbabwe}


% \subsubsection{Summary of DHS surveys}

%%%%%%%%%%%%%%%%%%%%%%%%%%% Summary 


DHS surveys were conducted in Zimbabwe in 1994, 1999, 2006, and 2015.
% years.out[1:(length(years.out)-1)], and years.out[length(years.out)]. 

We fit both the RW2 only model to the combined national data, and compare the time trend at national level with the estimates produced by the UN and IHME in Figure~\ref{fig:unnamed-chunk-342}. We then adjusted the combined national data to the UN estimates of U5MR, and refit the models on the benchmarked data. 

%%%%%%%%%%%%%%%%%%%%%%%%%% Plot5 
\begin{knitrout}
\definecolor{shadecolor}{rgb}{0.969, 0.969, 0.969}\color{fgcolor}\begin{figure}[bht]

{\centering \includegraphics[width=.9\textwidth]{../Main/Figures/Yearly_national_Zimbabwe} 

}

\caption[Temporal national trends along with UN (B3) estimates described in You et al]{Temporal national trends along with UN (B3) estimates described in You et al. (2015) and IHME estimates based on GBD 2015 Child Mortality Collaborators (2016). RW2 represents the smoothed national estimates using the original data before benchmarking with UN estimates. RW2-adj represents the smoothed national estimates using the benchmarked data.}\label{fig:unnamed-chunk-342}
\end{figure}


\end{knitrout}
 

We fit the RW2 model to the benchmarked data in each area. 
% The proportions of the explained variation is summarized in Table~\ref{tab:paste0(countryname, "-var")}. 
We compare the results in Figure~\ref{fig:unnamed-chunk-343} to \ref{fig:unnamed-chunk-347}.
Figure~\ref{fig:unnamed-chunk-343} compares the smoothed estimates against the direct estimates. Figure~\ref{fig:unnamed-chunk-344} and Figure~\ref{fig:unnamed-chunk-345} show the posterior median estimates of U5MR in each region over time and the reductions from 1990 period respectively.
Figure~\ref{fig:unnamed-chunk-346} shows the smoothed estimates by region over time and Figure~\ref{fig:unnamed-chunk-347} compares the smoothed estimates with direct estimates from each survey for each region over time.


% %%%%%%%%%%%%%%%%%%%%%%%%%%% Table1 
% <<echo=FALSE, results='asis'>>=
% load("rda/variance_tables.rda")
% countryname2 <- gsub(" ", "", countryname)
% variance <- tables.all[[countryname]]

% table_count <- table_count + 1

% names <- c("RW2 ($\\sigma^2_{\\gamma_{t}}$)", "ICAR ($\\sigma^2_{\\phi_{i}}$)", "IID space ($\\sigma^2_{\\theta_{i}}$)", "IID time ($\\sigma^2_{\\alpha_{t}}$)", "IID space time ($\\sigma^2_{\\delta_{it}}$)")

% variance$Proportion <- round(variance$Proportion*100, digits = 2)
% row.names(variance) <- names
% tab <- xtable(variance, digits = c(1, 3, 2),align = "l|ll",
%        label = paste0("tab:", countryname, "-var"),
%        caption = paste(country, ": summary of the variance components in the RW2 model", sep = ''))
% print(tab, comment = FALSE,sanitize.text.function = function(x) {x})
% @

%%%%%%%%%%%%%%%%%%%%%%%%%%% Plot1 
\begin{knitrout}
\definecolor{shadecolor}{rgb}{0.969, 0.969, 0.969}\color{fgcolor}\begin{figure}[bht]

{\centering \includegraphics[width=.9\textwidth]{../Main/Figures/SmoothvDirectZimbabwe_meta} 

}

\caption[Smooth versus direct Admin 1 estimates]{Smooth versus direct Admin 1 estimates. Left: Combined (meta-analysis) survey estimate against combined direct estimates. Right: Combined (meta-analysis) survey estimate against direct estimates from each survey.}\label{fig:unnamed-chunk-343}
\end{figure}


\end{knitrout}

%%%%%%%%%%%%%%%%%%%%%%%%%%% Plot2 
\begin{knitrout}
\definecolor{shadecolor}{rgb}{0.969, 0.969, 0.969}\color{fgcolor}\begin{figure}[bht]

{\centering \includegraphics[width=.9\textwidth]{../Main/Figures/SmoothMedianZimbabwe} 

}

\caption[Maps of posterior medians for Zimbabwe  over time]{Maps of posterior medians for Zimbabwe  over time.}\label{fig:unnamed-chunk-344}
\end{figure}


\end{knitrout}
%%%%%%%%%%%%%%%%%%%%%%%%%%% Plot2a
\begin{knitrout}
\definecolor{shadecolor}{rgb}{0.969, 0.969, 0.969}\color{fgcolor}\begin{figure}[bht]

{\centering \includegraphics[width=.9\textwidth]{../Main/Figures/ReductionMedianZimbabwe} 

}

\caption[Maps of reduction of posterior median U5MR in each five-year period compared to 1990 in Zimbabwe over time]{Maps of reduction of posterior median U5MR in each five-year period compared to 1990 in Zimbabwe over time.}\label{fig:unnamed-chunk-345}
\end{figure}


\end{knitrout}
%%%%%%%%%%%%%%%%%%%%%%%%%%% Plot3 
\begin{knitrout}
\definecolor{shadecolor}{rgb}{0.969, 0.969, 0.969}\color{fgcolor}\begin{figure}[bht]

{\centering \includegraphics[width=.95\textwidth]{../Main/Figures/Yearly_v_Periods_Zimbabwe} 

}

\caption[Smoothed regional estimates over time]{Smoothed regional estimates over time. The line indicates yearly posterior median estimates and error bars indicate 95 \% posterior credible interval at each time period.}\label{fig:unnamed-chunk-346}
\end{figure}


\end{knitrout}

%%%%%%%%%%%%%%%%%%%%%%%%%%% Plot4 
\begin{knitrout}
\definecolor{shadecolor}{rgb}{0.969, 0.969, 0.969}\color{fgcolor}\begin{figure}[bht]

{\centering \includegraphics[width=.9\textwidth]{../Main/Figures/LineSubMedianZimbabwe} 

}

\caption[Smoothed regional estimates over time compared to the direct estimates from each surveys]{Smoothed regional estimates over time compared to the direct estimates from each surveys. Direct estimates are not benchmarked with UN estimates. The line indicates posterior median and error bars indicate 95\% posterior credible interval.}\label{fig:unnamed-chunk-347}
\end{figure}


\end{knitrout}
% \subsubsection{National model results}
We further assess the RW2 model by holding out some observations, and compare the projections to the direct estimates in these holdout observations. Figure~\ref{fig:unnamed-chunk-348} compares the predicted estimates for the out-of-sample observations  with the direct estimates by holding out observations from each area in each time period.  Figure~\ref{fig:unnamed-chunk-349} compares the histogram of the bias rescaled by the total variance in the cross validation studies. Figure~\ref{fig:unnamed-chunk-350} compares the rescaled bias by region and time periods.



% %%%%%%%%%%%%%%%%%%%%%%%%%%% Plot6
% << echo=FALSE, out.width = ".9\\textwidth", fig.width = 12, fig.height = 6, fig.cap = "Out-of-sample predictions along with direct estimates in the cross validation study where all data from each time period is held out and predicted using the rest of the data.">>=
% fig_count <- fig_count + 1
% knitr::include_graphics(paste0("../Main/Figures/CV_byYear_withError_", countryname2, ".pdf")) 
% @
 
%%%%%%%%%%%%%%%%%%%%%%%%%%% Plot7
\begin{knitrout}
\definecolor{shadecolor}{rgb}{0.969, 0.969, 0.969}\color{fgcolor}\begin{figure}[bht]

{\centering \includegraphics[width=.9\textwidth]{../Main/Figures/CV_byYearRegion_withError_Zimbabwe} 

}

\caption[Out-of-sample predictions along with direct estimates in the cross validation study where data from one region in each time period is held out and predicted using the rest of the data]{Out-of-sample predictions along with direct estimates in the cross validation study where data from one region in each time period is held out and predicted using the rest of the data.}\label{fig:unnamed-chunk-348}
\end{figure}


\end{knitrout}

%%%%%%%%%%%%%%%%%%%%%%%%%%% Plot8
\begin{knitrout}
\definecolor{shadecolor}{rgb}{0.969, 0.969, 0.969}\color{fgcolor}\begin{figure}[bht]

{\centering \includegraphics[width=.9\textwidth]{../Main/Figures/CVbiasZimbabwe} 

}

\caption[Histogram and QQ-plot of the rescaled difference between the smoothed estimates and the direct estimates in the cross validation study]{Histogram and QQ-plot of the rescaled difference between the smoothed estimates and the direct estimates in the cross validation study. The differences between the two estimates are rescaled by the square root of the total variance of the two estimates.}\label{fig:unnamed-chunk-349}
\end{figure}


\end{knitrout}

%%%%%%%%%%%%%%%%%%%%%%%%%%% Plot9
\begin{knitrout}
\definecolor{shadecolor}{rgb}{0.969, 0.969, 0.969}\color{fgcolor}\begin{figure}[bht]

{\centering \includegraphics[width=.7\textwidth]{../Main/Figures/CVbiasbyRegionZimbabwe} 

}

\caption[Line plot of the difference between smoothed estimates and the direct estimates in the cross validation study]{Line plot of the difference between smoothed estimates and the direct estimates in the cross validation study. The differences between the two estimates are rescaled by the square root of the total variance of the two estimates.}\label{fig:unnamed-chunk-350}
\end{figure}


\end{knitrout}


% %%%%%%%%%%%%%%%%%%%%%%%%%%%%%%%%%%%%%%%%%%%%%%%%%%%%%%%%%%%%%%%%%%%%%%%%%%%%%%%%%%%%%%%%%%%%%%%%%%
% 
\clearpage
\section{Full table of the complete results: 5-year period}
{\scriptsize
% latex table generated in R 3.4.3 by xtable 1.8-2 package
% Mon Apr  2 21:00:33 2018
\begin{longtable}{lllrrrl}
  \hline
Country & Region & Year & Median & Lower & Upper & Method \\ 
  \hline 
\endhead 
\hline 
{\footnotesize Continued on next page} 
\endfoot 
\endlastfoot 
Angola & ALL & 80-84 & 189.53 & 179.33 & 200.89 & IHME \\ 
  Angola & ALL & 80-84 & 251.28 & 192.05 & 322.16 & RW2 \\ 
  Angola & ALL & 80-84 & 232.95 & 214.36 & 255.00 & UN \\ 
  Angola & ALL & 85-89 & 187.44 & 179.24 & 196.32 & IHME \\ 
  Angola & ALL & 85-89 & 222.56 & 189.53 & 258.44 & RW2 \\ 
  Angola & ALL & 85-89 & 227.72 & 214.73 & 241.69 & UN \\ 
  Angola & ALL & 90-94 & 182.79 & 175.12 & 191.15 & IHME \\ 
  Angola & ALL & 90-94 & 226.02 & 200.25 & 254.31 & RW2 \\ 
  Angola & ALL & 90-94 & 227.00 & 215.43 & 238.75 & UN \\ 
  Angola & ALL & 95-99 & 161.43 & 154.52 & 168.39 & IHME \\ 
  Angola & ALL & 95-99 & 224.40 & 202.98 & 247.22 & RW2 \\ 
  Angola & ALL & 95-99 & 223.29 & 211.97 & 235.65 & UN \\ 
  Angola & ALL & 00-04 & 144.35 & 138.13 & 151.24 & IHME \\ 
  Angola & ALL & 00-04 & 212.56 & 185.32 & 242.76 & RW2 \\ 
  Angola & ALL & 00-04 & 212.39 & 198.39 & 226.68 & UN \\ 
  Angola & ALL & 05-09 & 113.54 & 107.89 & 119.83 & IHME \\ 
  Angola & ALL & 05-09 & 194.23 & 133.86 & 273.39 & RW2 \\ 
  Angola & ALL & 05-09 & 195.23 & 177.10 & 218.26 & UN \\ 
  Angola & ALL & 10-14 & 91.11 & 83.25 & 99.61 & IHME \\ 
  Angola & ALL & 10-14 & 173.94 & 150.46 & 200.03 & RW2 \\ 
  Angola & ALL & 10-14 & 173.83 & 144.34 & 207.56 & UN \\ 
  Angola & BENGO & 80-84 & 109.42 & 28.82 & 341.04 & RW2 \\ 
  Angola & BENGO & 85-89 & 98.46 & 430.20 & 15.55 & HT-Direct \\ 
  Angola & BENGO & 85-89 & 98.18 & 37.18 & 232.16 & RW2 \\ 
  Angola & BENGO & 90-94 & 37.81 & 145.36 & 9.00 & HT-Direct \\ 
  Angola & BENGO & 90-94 & 107.18 & 54.88 & 196.73 & RW2 \\ 
  Angola & BENGO & 95-99 & 99.96 & 213.76 & 43.40 & HT-Direct \\ 
  Angola & BENGO & 95-99 & 110.33 & 68.70 & 170.86 & RW2 \\ 
  Angola & BENGO & 00-04 & 131.53 & 220.77 & 74.90 & HT-Direct \\ 
  Angola & BENGO & 00-04 & 121.38 & 77.40 & 184.04 & RW2 \\ 
  Angola & BENGO & 05-09 & 66.52 & 113.31 & 38.22 & HT-Direct \\ 
  Angola & BENGO & 05-09 & 120.34 & 61.94 & 221.01 & RW2 \\ 
  Angola & BENGO & 10-14 & 28.58 & 52.50 & 15.39 & HT-Direct \\ 
  Angola & BENGO & 10-14 & 109.96 & 48.26 & 231.45 & RW2 \\ 
  Angola & BENGO & 15-19 & 99.26 & 19.43 & 382.74 & RW2 \\ 
  Angola & BENGUELA & 80-84 & 528.44 & 701.44 & 348.32 & HT-Direct \\ 
  Angola & BENGUELA & 80-84 & 382.03 & 295.80 & 472.34 & RW2 \\ 
  Angola & BENGUELA & 85-89 & 472.84 & 609.98 & 339.67 & HT-Direct \\ 
  Angola & BENGUELA & 85-89 & 371.05 & 308.05 & 437.98 & RW2 \\ 
  Angola & BENGUELA & 90-94 & 382.04 & 456.77 & 312.51 & HT-Direct \\ 
  Angola & BENGUELA & 90-94 & 399.95 & 350.79 & 452.10 & RW2 \\ 
  Angola & BENGUELA & 95-99 & 396.29 & 459.02 & 336.80 & HT-Direct \\ 
  Angola & BENGUELA & 95-99 & 391.60 & 345.92 & 441.97 & RW2 \\ 
  Angola & BENGUELA & 00-04 & 284.90 & 347.64 & 229.50 & HT-Direct \\ 
  Angola & BENGUELA & 00-04 & 388.81 & 331.42 & 450.89 & RW2 \\ 
  Angola & BENGUELA & 05-09 & 173.61 & 214.51 & 139.13 & HT-Direct \\ 
  Angola & BENGUELA & 05-09 & 356.35 & 259.79 & 466.03 & RW2 \\ 
  Angola & BENGUELA & 10-14 & 117.95 & 149.51 & 92.34 & HT-Direct \\ 
  Angola & BENGUELA & 10-14 & 306.23 & 239.98 & 380.60 & RW2 \\ 
  Angola & BENGUELA & 15-19 & 256.79 & 87.07 & 554.63 & RW2 \\ 
  Angola & BIÉ & 80-84 & 530.35 & 747.96 & 300.55 & HT-Direct \\ 
  Angola & BIÉ & 80-84 & 207.78 & 140.72 & 302.01 & RW2 \\ 
  Angola & BIÉ & 85-89 & 104.44 & 232.42 & 42.98 & HT-Direct \\ 
  Angola & BIÉ & 85-89 & 179.37 & 131.64 & 239.28 & RW2 \\ 
  Angola & BIÉ & 90-94 & 145.09 & 247.55 & 80.50 & HT-Direct \\ 
  Angola & BIÉ & 90-94 & 185.19 & 147.45 & 229.35 & RW2 \\ 
  Angola & BIÉ & 95-99 & 150.46 & 183.96 & 122.14 & HT-Direct \\ 
  Angola & BIÉ & 95-99 & 179.43 & 150.51 & 211.60 & RW2 \\ 
  Angola & BIÉ & 00-04 & 171.01 & 217.76 & 132.60 & HT-Direct \\ 
  Angola & BIÉ & 00-04 & 188.75 & 153.86 & 228.77 & RW2 \\ 
  Angola & BIÉ & 05-09 & 88.68 & 116.56 & 66.97 & HT-Direct \\ 
  Angola & BIÉ & 05-09 & 187.08 & 127.22 & 266.30 & RW2 \\ 
  Angola & BIÉ & 10-14 & 76.05 & 98.32 & 58.50 & HT-Direct \\ 
  Angola & BIÉ & 10-14 & 176.26 & 129.69 & 236.48 & RW2 \\ 
  Angola & BIÉ & 15-19 & 164.50 & 51.31 & 419.47 & RW2 \\ 
  Angola & CABINDA & 80-84 & 99.11 & 32.41 & 280.32 & RW2 \\ 
  Angola & CABINDA & 85-89 & 38.54 & 228.39 & 5.40 & HT-Direct \\ 
  Angola & CABINDA & 85-89 & 87.35 & 39.54 & 185.43 & RW2 \\ 
  Angola & CABINDA & 90-94 & 150.85 & 295.21 & 70.06 & HT-Direct \\ 
  Angola & CABINDA & 90-94 & 92.79 & 53.96 & 154.23 & RW2 \\ 
  Angola & CABINDA & 95-99 & 49.46 & 96.83 & 24.64 & HT-Direct \\ 
  Angola & CABINDA & 95-99 & 89.74 & 59.41 & 132.96 & RW2 \\ 
  Angola & CABINDA & 00-04 & 54.12 & 94.76 & 30.32 & HT-Direct \\ 
  Angola & CABINDA & 00-04 & 94.76 & 62.66 & 138.82 & RW2 \\ 
  Angola & CABINDA & 05-09 & 65.27 & 112.63 & 37.00 & HT-Direct \\ 
  Angola & CABINDA & 05-09 & 92.97 & 51.48 & 161.67 & RW2 \\ 
  Angola & CABINDA & 10-14 & 34.72 & 54.72 & 21.85 & HT-Direct \\ 
  Angola & CABINDA & 10-14 & 86.45 & 44.88 & 159.87 & RW2 \\ 
  Angola & CABINDA & 15-19 & 79.89 & 18.84 & 287.17 & RW2 \\ 
  Angola & CUANDO CUBANGO & 80-84 & 117.10 & 34.29 & 333.57 & RW2 \\ 
  Angola & CUANDO CUBANGO & 85-89 & 114.77 & 46.62 & 256.40 & RW2 \\ 
  Angola & CUANDO CUBANGO & 90-94 & 79.09 & 200.42 & 28.58 & HT-Direct \\ 
  Angola & CUANDO CUBANGO & 90-94 & 134.36 & 73.13 & 232.00 & RW2 \\ 
  Angola & CUANDO CUBANGO & 95-99 & 160.05 & 274.70 & 87.48 & HT-Direct \\ 
  Angola & CUANDO CUBANGO & 95-99 & 142.79 & 92.50 & 211.75 & RW2 \\ 
  Angola & CUANDO CUBANGO & 00-04 & 127.93 & 227.75 & 68.00 & HT-Direct \\ 
  Angola & CUANDO CUBANGO & 00-04 & 160.56 & 111.27 & 225.42 & RW2 \\ 
  Angola & CUANDO CUBANGO & 05-09 & 93.82 & 141.57 & 61.03 & HT-Direct \\ 
  Angola & CUANDO CUBANGO & 05-09 & 166.30 & 100.17 & 262.85 & RW2 \\ 
  Angola & CUANDO CUBANGO & 10-14 & 58.76 & 85.86 & 39.84 & HT-Direct \\ 
  Angola & CUANDO CUBANGO & 10-14 & 161.09 & 96.22 & 258.20 & RW2 \\ 
  Angola & CUANDO CUBANGO & 15-19 & 154.35 & 41.23 & 437.49 & RW2 \\ 
  Angola & CUANZA NORTE & 80-84 & 192.10 & 79.26 & 390.69 & RW2 \\ 
  Angola & CUANZA NORTE & 85-89 & 34.74 & 198.72 & 5.20 & HT-Direct \\ 
  Angola & CUANZA NORTE & 85-89 & 178.63 & 97.94 & 303.21 & RW2 \\ 
  Angola & CUANZA NORTE & 90-94 & 208.84 & 329.17 & 124.34 & HT-Direct \\ 
  Angola & CUANZA NORTE & 90-94 & 199.01 & 134.75 & 283.23 & RW2 \\ 
  Angola & CUANZA NORTE & 95-99 & 243.31 & 376.80 & 146.04 & HT-Direct \\ 
  Angola & CUANZA NORTE & 95-99 & 203.92 & 151.92 & 269.34 & RW2 \\ 
  Angola & CUANZA NORTE & 00-04 & 210.17 & 304.41 & 139.26 & HT-Direct \\ 
  Angola & CUANZA NORTE & 00-04 & 218.26 & 161.94 & 290.73 & RW2 \\ 
  Angola & CUANZA NORTE & 05-09 & 115.65 & 147.23 & 90.12 & HT-Direct \\ 
  Angola & CUANZA NORTE & 05-09 & 212.26 & 132.84 & 324.83 & RW2 \\ 
  Angola & CUANZA NORTE & 10-14 & 64.68 & 96.87 & 42.69 & HT-Direct \\ 
  Angola & CUANZA NORTE & 10-14 & 192.49 & 116.23 & 300.44 & RW2 \\ 
  Angola & CUANZA NORTE & 15-19 & 170.96 & 46.61 & 466.61 & RW2 \\ 
  Angola & CUANZA SUL & 80-84 & 238.79 & 689.39 & 42.46 & HT-Direct \\ 
  Angola & CUANZA SUL & 80-84 & 223.42 & 139.32 & 338.52 & RW2 \\ 
  Angola & CUANZA SUL & 85-89 & 263.21 & 392.14 & 165.15 & HT-Direct \\ 
  Angola & CUANZA SUL & 85-89 & 217.14 & 160.52 & 286.99 & RW2 \\ 
  Angola & CUANZA SUL & 90-94 & 268.58 & 376.47 & 182.56 & HT-Direct \\ 
  Angola & CUANZA SUL & 90-94 & 244.80 & 197.98 & 298.81 & RW2 \\ 
  Angola & CUANZA SUL & 95-99 & 225.53 & 285.73 & 174.91 & HT-Direct \\ 
  Angola & CUANZA SUL & 95-99 & 253.65 & 211.66 & 300.30 & RW2 \\ 
  Angola & CUANZA SUL & 00-04 & 201.75 & 284.21 & 138.59 & HT-Direct \\ 
  Angola & CUANZA SUL & 00-04 & 275.07 & 220.81 & 337.17 & RW2 \\ 
  Angola & CUANZA SUL & 05-09 & 173.43 & 207.23 & 144.15 & HT-Direct \\ 
  Angola & CUANZA SUL & 05-09 & 277.19 & 188.37 & 388.35 & RW2 \\ 
  Angola & CUANZA SUL & 10-14 & 111.11 & 154.93 & 78.53 & HT-Direct \\ 
  Angola & CUANZA SUL & 10-14 & 264.54 & 187.38 & 359.78 & RW2 \\ 
  Angola & CUANZA SUL & 15-19 & 248.55 & 80.73 & 557.06 & RW2 \\ 
  Angola & CUNENE & 80-84 & 114.44 & 46.21 & 244.93 & RW2 \\ 
  Angola & CUNENE & 85-89 & 26.46 & 158.80 & 3.90 & HT-Direct \\ 
  Angola & CUNENE & 85-89 & 119.82 & 68.96 & 196.42 & RW2 \\ 
  Angola & CUNENE & 90-94 & 162.50 & 228.99 & 112.50 & HT-Direct \\ 
  Angola & CUNENE & 90-94 & 147.94 & 110.37 & 195.19 & RW2 \\ 
  Angola & CUNENE & 95-99 & 139.97 & 182.02 & 106.36 & HT-Direct \\ 
  Angola & CUNENE & 95-99 & 161.94 & 131.41 & 197.84 & RW2 \\ 
  Angola & CUNENE & 00-04 & 178.73 & 242.58 & 128.83 & HT-Direct \\ 
  Angola & CUNENE & 00-04 & 183.52 & 140.49 & 237.25 & RW2 \\ 
  Angola & CUNENE & 05-09 & 100.54 & 153.70 & 64.36 & HT-Direct \\ 
  Angola & CUNENE & 05-09 & 188.07 & 117.10 & 288.73 & RW2 \\ 
  Angola & CUNENE & 10-14 & 59.55 & 89.76 & 39.07 & HT-Direct \\ 
  Angola & CUNENE & 10-14 & 179.08 & 108.03 & 282.03 & RW2 \\ 
  Angola & CUNENE & 15-19 & 167.24 & 44.99 & 460.88 & RW2 \\ 
  Angola & HUAMBO & 80-84 & 420.65 & 626.69 & 238.98 & HT-Direct \\ 
  Angola & HUAMBO & 80-84 & 301.64 & 221.49 & 394.39 & RW2 \\ 
  Angola & HUAMBO & 85-89 & 297.44 & 430.74 & 191.52 & HT-Direct \\ 
  Angola & HUAMBO & 85-89 & 276.24 & 218.44 & 342.88 & RW2 \\ 
  Angola & HUAMBO & 90-94 & 338.47 & 441.04 & 249.11 & HT-Direct \\ 
  Angola & HUAMBO & 90-94 & 287.52 & 238.45 & 342.94 & RW2 \\ 
  Angola & HUAMBO & 95-99 & 229.27 & 301.63 & 170.05 & HT-Direct \\ 
  Angola & HUAMBO & 95-99 & 270.72 & 225.77 & 322.69 & RW2 \\ 
  Angola & HUAMBO & 00-04 & 225.22 & 296.42 & 167.07 & HT-Direct \\ 
  Angola & HUAMBO & 00-04 & 263.53 & 211.78 & 323.65 & RW2 \\ 
  Angola & HUAMBO & 05-09 & 132.93 & 179.51 & 97.01 & HT-Direct \\ 
  Angola & HUAMBO & 05-09 & 236.06 & 160.68 & 333.28 & RW2 \\ 
  Angola & HUAMBO & 10-14 & 73.48 & 96.21 & 55.78 & HT-Direct \\ 
  Angola & HUAMBO & 10-14 & 197.53 & 143.12 & 265.31 & RW2 \\ 
  Angola & HUAMBO & 15-19 & 161.33 & 49.39 & 412.12 & RW2 \\ 
  Angola & HUÍLA & 80-84 & 489.05 & 924.26 & 69.83 & HT-Direct \\ 
  Angola & HUÍLA & 80-84 & 246.63 & 156.09 & 369.85 & RW2 \\ 
  Angola & HUÍLA & 85-89 & 268.10 & 401.89 & 166.45 & HT-Direct \\ 
  Angola & HUÍLA & 85-89 & 233.35 & 174.04 & 305.37 & RW2 \\ 
  Angola & HUÍLA & 90-94 & 248.32 & 328.09 & 182.68 & HT-Direct \\ 
  Angola & HUÍLA & 90-94 & 255.43 & 209.47 & 307.91 & RW2 \\ 
  Angola & HUÍLA & 95-99 & 247.66 & 322.29 & 185.58 & HT-Direct \\ 
  Angola & HUÍLA & 95-99 & 254.78 & 214.39 & 299.72 & RW2 \\ 
  Angola & HUÍLA & 00-04 & 185.56 & 228.04 & 149.46 & HT-Direct \\ 
  Angola & HUÍLA & 00-04 & 265.88 & 219.65 & 317.35 & RW2 \\ 
  Angola & HUÍLA & 05-09 & 140.70 & 182.53 & 107.20 & HT-Direct \\ 
  Angola & HUÍLA & 05-09 & 257.80 & 179.81 & 356.02 & RW2 \\ 
  Angola & HUÍLA & 10-14 & 99.42 & 129.59 & 75.66 & HT-Direct \\ 
  Angola & HUÍLA & 10-14 & 236.71 & 177.01 & 309.04 & RW2 \\ 
  Angola & HUÍLA & 15-19 & 214.04 & 69.37 & 498.84 & RW2 \\ 
  Angola & LUANDA & 80-84 & 336.21 & 625.39 & 133.20 & HT-Direct \\ 
  Angola & LUANDA & 80-84 & 191.71 & 117.26 & 294.76 & RW2 \\ 
  Angola & LUANDA & 85-89 & 129.35 & 262.95 & 58.27 & HT-Direct \\ 
  Angola & LUANDA & 85-89 & 164.54 & 115.98 & 226.65 & RW2 \\ 
  Angola & LUANDA & 90-94 & 145.78 & 219.95 & 93.62 & HT-Direct \\ 
  Angola & LUANDA & 90-94 & 169.20 & 132.42 & 213.09 & RW2 \\ 
  Angola & LUANDA & 95-99 & 170.36 & 217.74 & 131.55 & HT-Direct \\ 
  Angola & LUANDA & 95-99 & 160.13 & 131.39 & 194.23 & RW2 \\ 
  Angola & LUANDA & 00-04 & 111.93 & 144.81 & 85.76 & HT-Direct \\ 
  Angola & LUANDA & 00-04 & 157.62 & 123.93 & 199.09 & RW2 \\ 
  Angola & LUANDA & 05-09 & 73.93 & 103.88 & 52.11 & HT-Direct \\ 
  Angola & LUANDA & 05-09 & 139.66 & 88.33 & 213.93 & RW2 \\ 
  Angola & LUANDA & 10-14 & 40.39 & 59.59 & 27.20 & HT-Direct \\ 
  Angola & LUANDA & 10-14 & 114.17 & 70.24 & 179.05 & RW2 \\ 
  Angola & LUANDA & 15-19 & 91.90 & 23.73 & 293.55 & RW2 \\ 
  Angola & LUNDA NORTE & 80-84 & 193.46 & 83.79 & 387.91 & RW2 \\ 
  Angola & LUNDA NORTE & 85-89 & 171.61 & 351.28 & 73.44 & HT-Direct \\ 
  Angola & LUNDA NORTE & 85-89 & 173.83 & 103.34 & 278.07 & RW2 \\ 
  Angola & LUNDA NORTE & 90-94 & 211.78 & 348.48 & 118.92 & HT-Direct \\ 
  Angola & LUNDA NORTE & 90-94 & 183.91 & 129.87 & 253.73 & RW2 \\ 
  Angola & LUNDA NORTE & 95-99 & 167.23 & 244.71 & 110.68 & HT-Direct \\ 
  Angola & LUNDA NORTE & 95-99 & 176.81 & 132.62 & 231.84 & RW2 \\ 
  Angola & LUNDA NORTE & 00-04 & 115.43 & 185.27 & 69.66 & HT-Direct \\ 
  Angola & LUNDA NORTE & 00-04 & 180.85 & 129.36 & 245.83 & RW2 \\ 
  Angola & LUNDA NORTE & 05-09 & 62.90 & 107.24 & 36.15 & HT-Direct \\ 
  Angola & LUNDA NORTE & 05-09 & 173.75 & 104.84 & 273.15 & RW2 \\ 
  Angola & LUNDA NORTE & 10-14 & 66.14 & 99.90 & 43.25 & HT-Direct \\ 
  Angola & LUNDA NORTE & 10-14 & 158.73 & 93.56 & 256.28 & RW2 \\ 
  Angola & LUNDA NORTE & 15-19 & 143.19 & 37.82 & 423.32 & RW2 \\ 
  Angola & LUNDA SUL & 80-84 & 109.20 & 353.84 & 26.71 & HT-Direct \\ 
  Angola & LUNDA SUL & 80-84 & 108.90 & 53.92 & 204.54 & RW2 \\ 
  Angola & LUNDA SUL & 85-89 & 15.23 & 105.87 & 2.02 & HT-Direct \\ 
  Angola & LUNDA SUL & 85-89 & 105.99 & 63.36 & 171.73 & RW2 \\ 
  Angola & LUNDA SUL & 90-94 & 193.83 & 321.41 & 108.77 & HT-Direct \\ 
  Angola & LUNDA SUL & 90-94 & 123.01 & 84.78 & 174.77 & RW2 \\ 
  Angola & LUNDA SUL & 95-99 & 143.73 & 241.71 & 81.21 & HT-Direct \\ 
  Angola & LUNDA SUL & 95-99 & 127.56 & 93.35 & 172.49 & RW2 \\ 
  Angola & LUNDA SUL & 00-04 & 82.35 & 123.86 & 53.90 & HT-Direct \\ 
  Angola & LUNDA SUL & 00-04 & 139.75 & 101.44 & 189.40 & RW2 \\ 
  Angola & LUNDA SUL & 05-09 & 46.97 & 70.00 & 31.26 & HT-Direct \\ 
  Angola & LUNDA SUL & 05-09 & 141.73 & 86.66 & 222.70 & RW2 \\ 
  Angola & LUNDA SUL & 10-14 & 48.19 & 70.85 & 32.53 & HT-Direct \\ 
  Angola & LUNDA SUL & 10-14 & 136.08 & 82.95 & 215.43 & RW2 \\ 
  Angola & LUNDA SUL & 15-19 & 129.21 & 34.99 & 379.73 & RW2 \\ 
  Angola & MALANJE & 80-84 & 579.93 & 986.73 & 25.00 & HT-Direct \\ 
  Angola & MALANJE & 80-84 & 264.83 & 131.24 & 468.08 & RW2 \\ 
  Angola & MALANJE & 85-89 & 274.02 & 660.60 & 68.20 & HT-Direct \\ 
  Angola & MALANJE & 85-89 & 216.82 & 129.29 & 342.15 & RW2 \\ 
  Angola & MALANJE & 90-94 & 198.51 & 336.01 & 108.12 & HT-Direct \\ 
  Angola & MALANJE & 90-94 & 208.45 & 144.93 & 289.43 & RW2 \\ 
  Angola & MALANJE & 95-99 & 168.71 & 266.33 & 101.90 & HT-Direct \\ 
  Angola & MALANJE & 95-99 & 185.66 & 139.92 & 241.23 & RW2 \\ 
  Angola & MALANJE & 00-04 & 125.26 & 171.96 & 89.86 & HT-Direct \\ 
  Angola & MALANJE & 00-04 & 177.25 & 135.58 & 227.22 & RW2 \\ 
  Angola & MALANJE & 05-09 & 64.34 & 102.49 & 39.76 & HT-Direct \\ 
  Angola & MALANJE & 05-09 & 157.73 & 101.88 & 236.56 & RW2 \\ 
  Angola & MALANJE & 10-14 & 54.22 & 73.86 & 39.57 & HT-Direct \\ 
  Angola & MALANJE & 10-14 & 132.66 & 88.06 & 195.04 & RW2 \\ 
  Angola & MALANJE & 15-19 & 109.37 & 30.90 & 320.51 & RW2 \\ 
  Angola & MOXICO & 80-84 & 420.82 & 22.26 & 951.40 & RW2 \\ 
  Angola & MOXICO & 85-89 & 284.96 & 25.53 & 835.53 & RW2 \\ 
  Angola & MOXICO & 90-94 & 214.87 & 35.74 & 631.55 & RW2 \\ 
  Angola & MOXICO & 95-99 & 144.37 & 41.52 & 361.23 & RW2 \\ 
  Angola & MOXICO & 00-04 & 8.73 & 56.86 & 1.29 & HT-Direct \\ 
  Angola & MOXICO & 00-04 & 100.98 & 41.93 & 210.40 & RW2 \\ 
  Angola & MOXICO & 05-09 & 1.33 & 9.89 & 0.18 & HT-Direct \\ 
  Angola & MOXICO & 05-09 & 64.80 & 22.69 & 170.38 & RW2 \\ 
  Angola & MOXICO & 10-14 & 18.10 & 44.66 & 7.22 & HT-Direct \\ 
  Angola & MOXICO & 10-14 & 39.12 & 8.51 & 163.55 & RW2 \\ 
  Angola & MOXICO & 15-19 & 23.17 & 1.88 & 228.51 & RW2 \\ 
  Angola & NAMIBE & 80-84 & 291.47 & 586.81 & 106.47 & HT-Direct \\ 
  Angola & NAMIBE & 80-84 & 187.91 & 112.15 & 293.95 & RW2 \\ 
  Angola & NAMIBE & 85-89 & 150.73 & 320.17 & 62.69 & HT-Direct \\ 
  Angola & NAMIBE & 85-89 & 187.73 & 131.24 & 261.06 & RW2 \\ 
  Angola & NAMIBE & 90-94 & 223.44 & 324.81 & 146.82 & HT-Direct \\ 
  Angola & NAMIBE & 90-94 & 217.18 & 168.81 & 275.41 & RW2 \\ 
  Angola & NAMIBE & 95-99 & 218.53 & 294.91 & 157.51 & HT-Direct \\ 
  Angola & NAMIBE & 95-99 & 223.17 & 178.55 & 276.85 & RW2 \\ 
  Angola & NAMIBE & 00-04 & 183.89 & 259.15 & 126.74 & HT-Direct \\ 
  Angola & NAMIBE & 00-04 & 234.44 & 180.41 & 299.39 & RW2 \\ 
  Angola & NAMIBE & 05-09 & 132.11 & 187.89 & 91.03 & HT-Direct \\ 
  Angola & NAMIBE & 05-09 & 222.75 & 145.23 & 325.79 & RW2 \\ 
  Angola & NAMIBE & 10-14 & 71.44 & 100.25 & 50.44 & HT-Direct \\ 
  Angola & NAMIBE & 10-14 & 197.56 & 131.06 & 284.00 & RW2 \\ 
  Angola & NAMIBE & 15-19 & 170.59 & 49.46 & 448.96 & RW2 \\ 
  Angola & UÍGE & 80-84 & 694.11 & 878.49 & 415.95 & HT-Direct \\ 
  Angola & UÍGE & 80-84 & 390.28 & 277.44 & 524.09 & RW2 \\ 
  Angola & UÍGE & 85-89 & 431.39 & 609.64 & 269.31 & HT-Direct \\ 
  Angola & UÍGE & 85-89 & 293.74 & 222.37 & 376.10 & RW2 \\ 
  Angola & UÍGE & 90-94 & 182.55 & 285.37 & 111.02 & HT-Direct \\ 
  Angola & UÍGE & 90-94 & 254.00 & 195.77 & 318.83 & RW2 \\ 
  Angola & UÍGE & 95-99 & 157.17 & 237.29 & 100.54 & HT-Direct \\ 
  Angola & UÍGE & 95-99 & 210.63 & 162.59 & 263.11 & RW2 \\ 
  Angola & UÍGE & 00-04 & 155.05 & 205.94 & 114.91 & HT-Direct \\ 
  Angola & UÍGE & 00-04 & 190.29 & 146.06 & 242.81 & RW2 \\ 
  Angola & UÍGE & 05-09 & 100.91 & 150.54 & 66.36 & HT-Direct \\ 
  Angola & UÍGE & 05-09 & 160.43 & 103.89 & 238.66 & RW2 \\ 
  Angola & UÍGE & 10-14 & 58.50 & 78.62 & 43.29 & HT-Direct \\ 
  Angola & UÍGE & 10-14 & 126.76 & 86.46 & 183.65 & RW2 \\ 
  Angola & UÍGE & 15-19 & 97.68 & 27.61 & 292.41 & RW2 \\ 
  Angola & ZAIRE & 80-84 & 73.36 & 27.47 & 191.95 & RW2 \\ 
  Angola & ZAIRE & 85-89 & 76.72 & 204.35 & 26.18 & HT-Direct \\ 
  Angola & ZAIRE & 85-89 & 70.64 & 37.55 & 130.03 & RW2 \\ 
  Angola & ZAIRE & 90-94 & 67.85 & 134.50 & 32.97 & HT-Direct \\ 
  Angola & ZAIRE & 90-94 & 83.32 & 54.91 & 124.25 & RW2 \\ 
  Angola & ZAIRE & 95-99 & 85.15 & 130.15 & 54.73 & HT-Direct \\ 
  Angola & ZAIRE & 95-99 & 93.36 & 67.63 & 127.21 & RW2 \\ 
  Angola & ZAIRE & 00-04 & 111.82 & 175.54 & 69.29 & HT-Direct \\ 
  Angola & ZAIRE & 00-04 & 113.69 & 80.46 & 159.50 & RW2 \\ 
  Angola & ZAIRE & 05-09 & 78.01 & 128.45 & 46.32 & HT-Direct \\ 
  Angola & ZAIRE & 05-09 & 124.74 & 74.57 & 202.46 & RW2 \\ 
  Angola & ZAIRE & 10-14 & 44.31 & 64.82 & 30.08 & HT-Direct \\ 
  Angola & ZAIRE & 10-14 & 126.81 & 75.20 & 205.41 & RW2 \\ 
  Angola & ZAIRE & 15-19 & 126.76 & 32.44 & 383.78 & RW2 \\ 
  Benin & ALL & 80-84 & 213.87 & 210.96 & 216.86 & IHME \\ 
  Benin & ALL & 80-84 & 210.20 & 200.64 & 220.10 & RW2 \\ 
  Benin & ALL & 80-84 & 210.24 & 204.33 & 216.73 & UN \\ 
  Benin & ALL & 85-89 & 191.79 & 189.49 & 194.29 & IHME \\ 
  Benin & ALL & 85-89 & 192.37 & 184.32 & 200.65 & RW2 \\ 
  Benin & ALL & 85-89 & 192.27 & 186.66 & 198.04 & UN \\ 
  Benin & ALL & 90-94 & 172.46 & 170.20 & 174.63 & IHME \\ 
  Benin & ALL & 90-94 & 170.36 & 163.93 & 177.00 & RW2 \\ 
  Benin & ALL & 90-94 & 170.47 & 165.82 & 175.17 & UN \\ 
  Benin & ALL & 95-99 & 152.56 & 150.34 & 154.78 & IHME \\ 
  Benin & ALL & 95-99 & 153.61 & 147.78 & 159.58 & RW2 \\ 
  Benin & ALL & 95-99 & 153.58 & 149.23 & 158.09 & UN \\ 
  Benin & ALL & 00-04 & 130.49 & 128.21 & 132.69 & IHME \\ 
  Benin & ALL & 00-04 & 136.99 & 130.31 & 143.99 & RW2 \\ 
  Benin & ALL & 00-04 & 136.92 & 132.18 & 141.28 & UN \\ 
  Benin & ALL & 05-09 & 107.75 & 105.25 & 110.37 & IHME \\ 
  Benin & ALL & 05-09 & 118.12 & 103.83 & 134.08 & RW2 \\ 
  Benin & ALL & 05-09 & 118.35 & 113.03 & 124.29 & UN \\ 
  Benin & ALL & 10-14 & 88.79 & 85.44 & 92.52 & IHME \\ 
  Benin & ALL & 10-14 & 100.31 & 38.17 & 235.56 & RW2 \\ 
  Benin & ALL & 10-14 & 106.29 & 98.52 & 115.37 & UN \\ 
  Benin & ATACORA & 80-84 & 290.00 & 315.66 & 265.61 & HT-Direct \\ 
  Benin & ATACORA & 80-84 & 269.43 & 250.07 & 289.97 & RW2 \\ 
  Benin & ATACORA & 85-89 & 237.50 & 256.40 & 219.59 & HT-Direct \\ 
  Benin & ATACORA & 85-89 & 239.27 & 225.75 & 252.98 & RW2 \\ 
  Benin & ATACORA & 90-94 & 205.36 & 222.70 & 189.04 & HT-Direct \\ 
  Benin & ATACORA & 90-94 & 211.46 & 199.56 & 223.69 & RW2 \\ 
  Benin & ATACORA & 95-99 & 202.99 & 219.10 & 187.77 & HT-Direct \\ 
  Benin & ATACORA & 95-99 & 192.08 & 181.34 & 203.35 & RW2 \\ 
  Benin & ATACORA & 00-04 & 170.83 & 190.14 & 153.11 & HT-Direct \\ 
  Benin & ATACORA & 00-04 & 173.59 & 159.91 & 188.37 & RW2 \\ 
  Benin & ATACORA & 05-09 & 139.52 & 173.80 & 111.09 & HT-Direct \\ 
  Benin & ATACORA & 05-09 & 150.79 & 126.74 & 178.95 & RW2 \\ 
  Benin & ATACORA & 10-14 & 128.39 & 54.04 & 274.00 & RW2 \\ 
  Benin & ATACORA & 15-19 & 108.71 & 11.83 & 546.08 & RW2 \\ 
  Benin & ATLANTIQUE & 80-84 & 179.55 & 201.53 & 159.48 & HT-Direct \\ 
  Benin & ATLANTIQUE & 80-84 & 171.63 & 155.85 & 188.66 & RW2 \\ 
  Benin & ATLANTIQUE & 85-89 & 163.21 & 183.00 & 145.19 & HT-Direct \\ 
  Benin & ATLANTIQUE & 85-89 & 153.80 & 142.83 & 165.58 & RW2 \\ 
  Benin & ATLANTIQUE & 90-94 & 126.24 & 139.82 & 113.80 & HT-Direct \\ 
  Benin & ATLANTIQUE & 90-94 & 134.31 & 125.47 & 143.42 & RW2 \\ 
  Benin & ATLANTIQUE & 95-99 & 124.64 & 136.40 & 113.77 & HT-Direct \\ 
  Benin & ATLANTIQUE & 95-99 & 121.24 & 113.57 & 129.31 & RW2 \\ 
  Benin & ATLANTIQUE & 00-04 & 115.02 & 128.93 & 102.43 & HT-Direct \\ 
  Benin & ATLANTIQUE & 00-04 & 111.03 & 101.64 & 121.36 & RW2 \\ 
  Benin & ATLANTIQUE & 05-09 & 80.50 & 104.68 & 61.52 & HT-Direct \\ 
  Benin & ATLANTIQUE & 05-09 & 98.12 & 81.56 & 118.03 & RW2 \\ 
  Benin & ATLANTIQUE & 10-14 & 85.53 & 35.34 & 193.84 & RW2 \\ 
  Benin & ATLANTIQUE & 15-19 & 73.90 & 8.03 & 435.52 & RW2 \\ 
  Benin & BORGOU & 80-84 & 213.43 & 239.38 & 189.59 & HT-Direct \\ 
  Benin & BORGOU & 80-84 & 207.22 & 187.95 & 227.75 & RW2 \\ 
  Benin & BORGOU & 85-89 & 194.27 & 216.02 & 174.23 & HT-Direct \\ 
  Benin & BORGOU & 85-89 & 195.00 & 181.62 & 209.18 & RW2 \\ 
  Benin & BORGOU & 90-94 & 184.89 & 203.83 & 167.33 & HT-Direct \\ 
  Benin & BORGOU & 90-94 & 179.74 & 168.23 & 192.27 & RW2 \\ 
  Benin & BORGOU & 95-99 & 174.96 & 191.97 & 159.17 & HT-Direct \\ 
  Benin & BORGOU & 95-99 & 166.02 & 155.09 & 177.78 & RW2 \\ 
  Benin & BORGOU & 00-04 & 141.96 & 163.68 & 122.70 & HT-Direct \\ 
  Benin & BORGOU & 00-04 & 150.40 & 135.61 & 166.29 & RW2 \\ 
  Benin & BORGOU & 05-09 & 110.61 & 145.48 & 83.29 & HT-Direct \\ 
  Benin & BORGOU & 05-09 & 130.48 & 105.74 & 158.21 & RW2 \\ 
  Benin & BORGOU & 10-14 & 111.02 & 45.50 & 243.60 & RW2 \\ 
  Benin & BORGOU & 15-19 & 93.29 & 10.08 & 505.17 & RW2 \\ 
  Benin & MONO & 80-84 & 204.65 & 231.64 & 180.07 & HT-Direct \\ 
  Benin & MONO & 80-84 & 200.29 & 180.72 & 221.16 & RW2 \\ 
  Benin & MONO & 85-89 & 182.97 & 203.22 & 164.33 & HT-Direct \\ 
  Benin & MONO & 85-89 & 178.87 & 166.30 & 192.17 & RW2 \\ 
  Benin & MONO & 90-94 & 155.61 & 170.84 & 141.52 & HT-Direct \\ 
  Benin & MONO & 90-94 & 154.41 & 144.56 & 165.12 & RW2 \\ 
  Benin & MONO & 95-99 & 141.13 & 157.00 & 126.62 & HT-Direct \\ 
  Benin & MONO & 95-99 & 134.49 & 125.12 & 144.42 & RW2 \\ 
  Benin & MONO & 00-04 & 112.33 & 127.30 & 98.92 & HT-Direct \\ 
  Benin & MONO & 00-04 & 116.72 & 105.78 & 128.66 & RW2 \\ 
  Benin & MONO & 05-09 & 89.04 & 121.94 & 64.37 & HT-Direct \\ 
  Benin & MONO & 05-09 & 97.56 & 78.95 & 119.27 & RW2 \\ 
  Benin & MONO & 10-14 & 79.99 & 32.10 & 185.68 & RW2 \\ 
  Benin & MONO & 15-19 & 65.76 & 6.86 & 410.35 & RW2 \\ 
  Benin & OUEME & 80-84 & 212.06 & 240.44 & 186.20 & HT-Direct \\ 
  Benin & OUEME & 80-84 & 204.85 & 185.25 & 226.06 & RW2 \\ 
  Benin & OUEME & 85-89 & 194.26 & 213.37 & 176.48 & HT-Direct \\ 
  Benin & OUEME & 85-89 & 187.77 & 175.45 & 200.89 & RW2 \\ 
  Benin & OUEME & 90-94 & 165.29 & 179.90 & 151.64 & HT-Direct \\ 
  Benin & OUEME & 90-94 & 167.77 & 157.66 & 178.20 & RW2 \\ 
  Benin & OUEME & 95-99 & 160.48 & 176.84 & 145.37 & HT-Direct \\ 
  Benin & OUEME & 95-99 & 154.00 & 143.99 & 164.48 & RW2 \\ 
  Benin & OUEME & 00-04 & 138.31 & 154.71 & 123.40 & HT-Direct \\ 
  Benin & OUEME & 00-04 & 143.47 & 131.56 & 156.42 & RW2 \\ 
  Benin & OUEME & 05-09 & 130.33 & 162.18 & 103.96 & HT-Direct \\ 
  Benin & OUEME & 05-09 & 130.33 & 109.21 & 155.32 & RW2 \\ 
  Benin & OUEME & 10-14 & 116.77 & 48.78 & 253.84 & RW2 \\ 
  Benin & OUEME & 15-19 & 104.58 & 11.75 & 530.27 & RW2 \\ 
  Benin & ZOU & 80-84 & 233.74 & 260.97 & 208.55 & HT-Direct \\ 
  Benin & ZOU & 80-84 & 224.20 & 205.09 & 244.77 & RW2 \\ 
  Benin & ZOU & 85-89 & 201.71 & 224.15 & 181.00 & HT-Direct \\ 
  Benin & ZOU & 85-89 & 198.81 & 185.78 & 212.65 & RW2 \\ 
  Benin & ZOU & 90-94 & 179.20 & 194.89 & 164.52 & HT-Direct \\ 
  Benin & ZOU & 90-94 & 172.31 & 162.59 & 182.48 & RW2 \\ 
  Benin & ZOU & 95-99 & 144.26 & 156.87 & 132.50 & HT-Direct \\ 
  Benin & ZOU & 95-99 & 151.86 & 143.11 & 160.93 & RW2 \\ 
  Benin & ZOU & 00-04 & 147.49 & 163.21 & 133.04 & HT-Direct \\ 
  Benin & ZOU & 00-04 & 135.44 & 124.76 & 146.93 & RW2 \\ 
  Benin & ZOU & 05-09 & 82.36 & 113.91 & 58.97 & HT-Direct \\ 
  Benin & ZOU & 05-09 & 115.79 & 96.90 & 137.97 & RW2 \\ 
  Benin & ZOU & 10-14 & 97.41 & 40.36 & 216.81 & RW2 \\ 
  Benin & ZOU & 15-19 & 82.14 & 8.90 & 460.26 & RW2 \\ 
  Burkina Faso & ALL & 80-84 & 242.41 & 239.49 & 245.15 & IHME \\ 
  Burkina Faso & ALL & 80-84 & 232.08 & 222.75 & 241.67 & RW2 \\ 
  Burkina Faso & ALL & 80-84 & 232.03 & 224.90 & 238.70 & UN \\ 
  Burkina Faso & ALL & 85-89 & 218.13 & 215.86 & 220.79 & IHME \\ 
  Burkina Faso & ALL & 85-89 & 209.98 & 202.20 & 217.93 & RW2 \\ 
  Burkina Faso & ALL & 85-89 & 210.07 & 204.38 & 216.27 & UN \\ 
  Burkina Faso & ALL & 90-94 & 202.65 & 200.30 & 204.84 & IHME \\ 
  Burkina Faso & ALL & 90-94 & 201.84 & 195.42 & 208.43 & RW2 \\ 
  Burkina Faso & ALL & 90-94 & 201.77 & 196.36 & 207.54 & UN \\ 
  Burkina Faso & ALL & 95-99 & 188.88 & 186.70 & 191.42 & IHME \\ 
  Burkina Faso & ALL & 95-99 & 193.74 & 187.24 & 200.36 & RW2 \\ 
  Burkina Faso & ALL & 95-99 & 193.79 & 188.27 & 200.09 & UN \\ 
  Burkina Faso & ALL & 00-04 & 168.26 & 165.68 & 170.79 & IHME \\ 
  Burkina Faso & ALL & 00-04 & 176.96 & 169.51 & 184.71 & RW2 \\ 
  Burkina Faso & ALL & 00-04 & 176.96 & 170.60 & 183.88 & UN \\ 
  Burkina Faso & ALL & 05-09 & 144.04 & 140.79 & 147.39 & IHME \\ 
  Burkina Faso & ALL & 05-09 & 139.53 & 131.41 & 148.06 & RW2 \\ 
  Burkina Faso & ALL & 05-09 & 139.54 & 133.63 & 145.72 & UN \\ 
  Burkina Faso & ALL & 10-14 & 122.58 & 118.41 & 127.04 & IHME \\ 
  Burkina Faso & ALL & 10-14 & 103.58 & 40.39 & 236.89 & RW2 \\ 
  Burkina Faso & ALL & 10-14 & 102.06 & 93.11 & 111.53 & UN \\ 
  Burkina Faso & CENTRAL/SOUTH & 80-84 & 235.28 & 252.06 & 219.28 & HT-Direct \\ 
  Burkina Faso & CENTRAL/SOUTH & 80-84 & 216.67 & 202.70 & 231.13 & RW2 \\ 
  Burkina Faso & CENTRAL/SOUTH & 85-89 & 187.11 & 199.35 & 175.46 & HT-Direct \\ 
  Burkina Faso & CENTRAL/SOUTH & 85-89 & 203.32 & 193.13 & 214.06 & RW2 \\ 
  Burkina Faso & CENTRAL/SOUTH & 90-94 & 180.67 & 191.09 & 170.69 & HT-Direct \\ 
  Burkina Faso & CENTRAL/SOUTH & 90-94 & 201.46 & 191.83 & 211.50 & RW2 \\ 
  Burkina Faso & CENTRAL/SOUTH & 95-99 & 177.16 & 187.85 & 166.95 & HT-Direct \\ 
  Burkina Faso & CENTRAL/SOUTH & 95-99 & 195.31 & 185.52 & 205.52 & RW2 \\ 
  Burkina Faso & CENTRAL/SOUTH & 00-04 & 145.59 & 156.55 & 135.28 & HT-Direct \\ 
  Burkina Faso & CENTRAL/SOUTH & 00-04 & 179.66 & 168.84 & 191.14 & RW2 \\ 
  Burkina Faso & CENTRAL/SOUTH & 05-09 & 109.25 & 121.11 & 98.43 & HT-Direct \\ 
  Burkina Faso & CENTRAL/SOUTH & 05-09 & 142.91 & 129.38 & 157.46 & RW2 \\ 
  Burkina Faso & CENTRAL/SOUTH & 10-14 & 107.21 & 47.01 & 226.55 & RW2 \\ 
  Burkina Faso & CENTRAL/SOUTH & 15-19 & 78.90 & 8.82 & 443.68 & RW2 \\ 
  Burkina Faso & EAST & 80-84 & 263.51 & 287.21 & 241.11 & HT-Direct \\ 
  Burkina Faso & EAST & 80-84 & 257.92 & 239.19 & 277.88 & RW2 \\ 
  Burkina Faso & EAST & 85-89 & 222.35 & 242.60 & 203.34 & HT-Direct \\ 
  Burkina Faso & EAST & 85-89 & 226.38 & 213.11 & 240.16 & RW2 \\ 
  Burkina Faso & EAST & 90-94 & 213.19 & 228.46 & 198.68 & HT-Direct \\ 
  Burkina Faso & EAST & 90-94 & 210.81 & 200.04 & 221.98 & RW2 \\ 
  Burkina Faso & EAST & 95-99 & 193.41 & 207.86 & 179.75 & HT-Direct \\ 
  Burkina Faso & EAST & 95-99 & 194.16 & 183.89 & 204.83 & RW2 \\ 
  Burkina Faso & EAST & 00-04 & 168.41 & 187.29 & 151.09 & HT-Direct \\ 
  Burkina Faso & EAST & 00-04 & 170.34 & 158.43 & 183.07 & RW2 \\ 
  Burkina Faso & EAST & 05-09 & 125.02 & 142.50 & 109.40 & HT-Direct \\ 
  Burkina Faso & EAST & 05-09 & 129.97 & 115.93 & 145.58 & RW2 \\ 
  Burkina Faso & EAST & 10-14 & 93.32 & 39.90 & 201.19 & RW2 \\ 
  Burkina Faso & EAST & 15-19 & 66.11 & 7.28 & 403.03 & RW2 \\ 
  Burkina Faso & NORTH & 80-84 & 233.39 & 257.81 & 210.63 & HT-Direct \\ 
  Burkina Faso & NORTH & 80-84 & 224.74 & 211.26 & 238.88 & RW2 \\ 
  Burkina Faso & NORTH & 85-89 & 229.12 & 246.29 & 212.81 & HT-Direct \\ 
  Burkina Faso & NORTH & 85-89 & 195.25 & 185.86 & 204.99 & RW2 \\ 
  Burkina Faso & NORTH & 90-94 & 231.48 & 248.44 & 215.35 & HT-Direct \\ 
  Burkina Faso & NORTH & 90-94 & 182.83 & 174.85 & 191.00 & RW2 \\ 
  Burkina Faso & NORTH & 95-99 & 234.46 & 251.92 & 217.86 & HT-Direct \\ 
  Burkina Faso & NORTH & 95-99 & 170.71 & 163.02 & 178.72 & RW2 \\ 
  Burkina Faso & NORTH & 00-04 & 228.79 & 248.25 & 210.42 & HT-Direct \\ 
  Burkina Faso & NORTH & 00-04 & 150.03 & 141.69 & 158.90 & RW2 \\ 
  Burkina Faso & NORTH & 05-09 & 183.59 & 203.35 & 165.35 & HT-Direct \\ 
  Burkina Faso & NORTH & 05-09 & 114.15 & 104.13 & 125.00 & RW2 \\ 
  Burkina Faso & NORTH & 10-14 & 81.85 & 35.12 & 177.19 & RW2 \\ 
  Burkina Faso & NORTH & 15-19 & 57.96 & 6.36 & 367.23 & RW2 \\ 
  Burkina Faso & WEST & 80-84 & 219.42 & 236.24 & 203.49 & HT-Direct \\ 
  Burkina Faso & WEST & 80-84 & 233.50 & 215.16 & 252.69 & RW2 \\ 
  Burkina Faso & WEST & 85-89 & 200.83 & 214.64 & 187.69 & HT-Direct \\ 
  Burkina Faso & WEST & 85-89 & 226.80 & 214.52 & 239.47 & RW2 \\ 
  Burkina Faso & WEST & 90-94 & 205.41 & 218.87 & 192.57 & HT-Direct \\ 
  Burkina Faso & WEST & 90-94 & 232.52 & 221.20 & 244.37 & RW2 \\ 
  Burkina Faso & WEST & 95-99 & 191.78 & 205.57 & 178.70 & HT-Direct \\ 
  Burkina Faso & WEST & 95-99 & 235.74 & 224.02 & 247.81 & RW2 \\ 
  Burkina Faso & WEST & 00-04 & 181.98 & 197.44 & 167.48 & HT-Direct \\ 
  Burkina Faso & WEST & 00-04 & 228.14 & 214.80 & 242.14 & RW2 \\ 
  Burkina Faso & WEST & 05-09 & 133.38 & 148.88 & 119.26 & HT-Direct \\ 
  Burkina Faso & WEST & 05-09 & 193.66 & 176.90 & 211.83 & RW2 \\ 
  Burkina Faso & WEST & 10-14 & 155.58 & 70.09 & 308.32 & RW2 \\ 
  Burkina Faso & WEST & 15-19 & 123.44 & 14.09 & 570.02 & RW2 \\ 
  Burundi & ALL & 80-84 & 202.90 & 196.35 & 208.65 & IHME \\ 
  Burundi & ALL & 80-84 & 202.14 & 163.91 & 246.94 & RW2 \\ 
  Burundi & ALL & 80-84 & 201.57 & 193.32 & 211.06 & UN \\ 
  Burundi & ALL & 85-89 & 174.26 & 169.47 & 179.73 & IHME \\ 
  Burundi & ALL & 85-89 & 172.52 & 152.02 & 194.58 & RW2 \\ 
  Burundi & ALL & 85-89 & 172.35 & 164.06 & 180.82 & UN \\ 
  Burundi & ALL & 90-94 & 174.19 & 169.16 & 179.53 & IHME \\ 
  Burundi & ALL & 90-94 & 173.31 & 158.82 & 188.95 & RW2 \\ 
  Burundi & ALL & 90-94 & 173.74 & 165.42 & 182.23 & UN \\ 
  Burundi & ALL & 95-99 & 172.33 & 167.23 & 177.79 & IHME \\ 
  Burundi & ALL & 95-99 & 164.52 & 153.42 & 176.14 & RW2 \\ 
  Burundi & ALL & 95-99 & 164.77 & 156.21 & 173.93 & UN \\ 
  Burundi & ALL & 00-04 & 149.22 & 144.17 & 154.66 & IHME \\ 
  Burundi & ALL & 00-04 & 143.59 & 135.13 & 152.53 & RW2 \\ 
  Burundi & ALL & 00-04 & 142.94 & 134.94 & 151.10 & UN \\ 
  Burundi & ALL & 05-09 & 99.50 & 95.67 & 103.48 & IHME \\ 
  Burundi & ALL & 05-09 & 114.07 & 103.87 & 125.14 & RW2 \\ 
  Burundi & ALL & 05-09 & 114.34 & 104.29 & 124.91 & UN \\ 
  Burundi & ALL & 10-14 & 82.36 & 74.46 & 90.07 & IHME \\ 
  Burundi & ALL & 10-14 & 87.32 & 31.01 & 218.25 & RW2 \\ 
  Burundi & ALL & 10-14 & 92.23 & 78.38 & 109.08 & UN \\ 
  Burundi & BUJUMBURA & 80-84 & 237.56 & 389.25 & 132.18 & HT-Direct \\ 
  Burundi & BUJUMBURA & 80-84 & 186.14 & 118.57 & 288.34 & RW2 \\ 
  Burundi & BUJUMBURA & 85-89 & 138.40 & 240.16 & 75.47 & HT-Direct \\ 
  Burundi & BUJUMBURA & 85-89 & 133.40 & 96.33 & 181.29 & RW2 \\ 
  Burundi & BUJUMBURA & 90-94 & 101.43 & 154.75 & 65.07 & HT-Direct \\ 
  Burundi & BUJUMBURA & 90-94 & 114.59 & 87.59 & 146.46 & RW2 \\ 
  Burundi & BUJUMBURA & 95-99 & 148.69 & 218.95 & 98.14 & HT-Direct \\ 
  Burundi & BUJUMBURA & 95-99 & 98.61 & 78.39 & 122.21 & RW2 \\ 
  Burundi & BUJUMBURA & 00-04 & 86.48 & 122.71 & 60.22 & HT-Direct \\ 
  Burundi & BUJUMBURA & 00-04 & 77.61 & 62.65 & 95.46 & RW2 \\ 
  Burundi & BUJUMBURA & 05-09 & 56.89 & 77.28 & 41.63 & HT-Direct \\ 
  Burundi & BUJUMBURA & 05-09 & 58.74 & 43.17 & 80.34 & RW2 \\ 
  Burundi & BUJUMBURA & 10-14 & 44.27 & 14.32 & 134.58 & RW2 \\ 
  Burundi & BUJUMBURA & 15-19 & 33.49 & 2.36 & 358.63 & RW2 \\ 
  Burundi & CENTRE-EAST & 80-84 & 313.83 & 457.10 & 199.00 & HT-Direct \\ 
  Burundi & CENTRE-EAST & 80-84 & 203.27 & 146.40 & 285.79 & RW2 \\ 
  Burundi & CENTRE-EAST & 85-89 & 152.09 & 198.98 & 114.67 & HT-Direct \\ 
  Burundi & CENTRE-EAST & 85-89 & 161.60 & 134.30 & 193.81 & RW2 \\ 
  Burundi & CENTRE-EAST & 90-94 & 168.81 & 200.59 & 141.17 & HT-Direct \\ 
  Burundi & CENTRE-EAST & 90-94 & 154.14 & 135.59 & 173.90 & RW2 \\ 
  Burundi & CENTRE-EAST & 95-99 & 180.82 & 209.82 & 155.04 & HT-Direct \\ 
  Burundi & CENTRE-EAST & 95-99 & 148.11 & 133.88 & 163.11 & RW2 \\ 
  Burundi & CENTRE-EAST & 00-04 & 179.53 & 207.30 & 154.75 & HT-Direct \\ 
  Burundi & CENTRE-EAST & 00-04 & 134.06 & 121.55 & 148.22 & RW2 \\ 
  Burundi & CENTRE-EAST & 05-09 & 92.76 & 108.54 & 79.08 & HT-Direct \\ 
  Burundi & CENTRE-EAST & 05-09 & 109.01 & 93.01 & 127.46 & RW2 \\ 
  Burundi & CENTRE-EAST & 10-14 & 84.85 & 32.72 & 202.48 & RW2 \\ 
  Burundi & CENTRE-EAST & 15-19 & 65.02 & 5.63 & 451.28 & RW2 \\ 
  Burundi & NORTH & 80-84 & 207.97 & 314.69 & 130.54 & HT-Direct \\ 
  Burundi & NORTH & 80-84 & 184.52 & 130.46 & 250.53 & RW2 \\ 
  Burundi & NORTH & 85-89 & 160.34 & 213.88 & 118.18 & HT-Direct \\ 
  Burundi & NORTH & 85-89 & 174.22 & 142.68 & 209.40 & RW2 \\ 
  Burundi & NORTH & 90-94 & 223.90 & 265.46 & 187.18 & HT-Direct \\ 
  Burundi & NORTH & 90-94 & 191.07 & 168.67 & 216.47 & RW2 \\ 
  Burundi & NORTH & 95-99 & 250.90 & 295.09 & 211.34 & HT-Direct \\ 
  Burundi & NORTH & 95-99 & 192.42 & 173.37 & 213.87 & RW2 \\ 
  Burundi & NORTH & 00-04 & 210.64 & 239.95 & 184.04 & HT-Direct \\ 
  Burundi & NORTH & 00-04 & 172.62 & 157.31 & 188.95 & RW2 \\ 
  Burundi & NORTH & 05-09 & 130.72 & 148.74 & 114.59 & HT-Direct \\ 
  Burundi & NORTH & 05-09 & 144.22 & 125.62 & 165.16 & RW2 \\ 
  Burundi & NORTH & 10-14 & 116.80 & 45.84 & 265.72 & RW2 \\ 
  Burundi & NORTH & 15-19 & 93.55 & 8.04 & 559.64 & RW2 \\ 
  Burundi & SOUTH & 80-84 & 203.48 & 291.21 & 137.07 & HT-Direct \\ 
  Burundi & SOUTH & 80-84 & 218.88 & 164.37 & 287.38 & RW2 \\ 
  Burundi & SOUTH & 85-89 & 228.74 & 292.34 & 175.55 & HT-Direct \\ 
  Burundi & SOUTH & 85-89 & 173.87 & 143.34 & 210.32 & RW2 \\ 
  Burundi & SOUTH & 90-94 & 151.94 & 198.30 & 114.86 & HT-Direct \\ 
  Burundi & SOUTH & 90-94 & 152.59 & 130.89 & 177.25 & RW2 \\ 
  Burundi & SOUTH & 95-99 & 164.09 & 193.32 & 138.52 & HT-Direct \\ 
  Burundi & SOUTH & 95-99 & 132.11 & 117.45 & 147.28 & RW2 \\ 
  Burundi & SOUTH & 00-04 & 138.80 & 158.69 & 121.04 & HT-Direct \\ 
  Burundi & SOUTH & 00-04 & 110.82 & 100.65 & 121.93 & RW2 \\ 
  Burundi & SOUTH & 05-09 & 81.50 & 97.90 & 67.64 & HT-Direct \\ 
  Burundi & SOUTH & 05-09 & 87.93 & 73.18 & 106.22 & RW2 \\ 
  Burundi & SOUTH & 10-14 & 68.02 & 25.20 & 173.35 & RW2 \\ 
  Burundi & SOUTH & 15-19 & 52.71 & 4.31 & 413.21 & RW2 \\ 
  Burundi & WEST & 80-84 & 166.54 & 258.40 & 102.80 & HT-Direct \\ 
  Burundi & WEST & 80-84 & 195.53 & 135.00 & 266.75 & RW2 \\ 
  Burundi & WEST & 85-89 & 180.68 & 239.49 & 133.78 & HT-Direct \\ 
  Burundi & WEST & 85-89 & 187.12 & 153.64 & 225.21 & RW2 \\ 
  Burundi & WEST & 90-94 & 253.72 & 309.47 & 205.03 & HT-Direct \\ 
  Burundi & WEST & 90-94 & 200.39 & 174.87 & 229.88 & RW2 \\ 
  Burundi & WEST & 95-99 & 245.65 & 292.03 & 204.50 & HT-Direct \\ 
  Burundi & WEST & 95-99 & 193.27 & 172.41 & 217.85 & RW2 \\ 
  Burundi & WEST & 00-04 & 208.71 & 246.05 & 175.72 & HT-Direct \\ 
  Burundi & WEST & 00-04 & 161.80 & 144.84 & 180.47 & RW2 \\ 
  Burundi & WEST & 05-09 & 96.47 & 118.86 & 77.93 & HT-Direct \\ 
  Burundi & WEST & 05-09 & 120.06 & 96.81 & 146.00 & RW2 \\ 
  Burundi & WEST & 10-14 & 84.90 & 31.35 & 205.07 & RW2 \\ 
  Burundi & WEST & 15-19 & 58.55 & 4.90 & 430.22 & RW2 \\ 
  Cameroon & ADAM/NORD/EXT-NORD & 80-84 & 219.24 & 245.15 & 195.36 & HT-Direct \\ 
  Cameroon & ADAM/NORD/EXT-NORD & 80-84 & 242.28 & 218.86 & 267.71 & RW2 \\ 
  Cameroon & ADAM/NORD/EXT-NORD & 85-89 & 197.13 & 214.59 & 180.76 & HT-Direct \\ 
  Cameroon & ADAM/NORD/EXT-NORD & 85-89 & 198.95 & 185.03 & 213.61 & RW2 \\ 
  Cameroon & ADAM/NORD/EXT-NORD & 90-94 & 196.96 & 211.34 & 183.33 & HT-Direct \\ 
  Cameroon & ADAM/NORD/EXT-NORD & 90-94 & 195.57 & 184.65 & 207.00 & RW2 \\ 
  Cameroon & ADAM/NORD/EXT-NORD & 95-99 & 198.76 & 214.58 & 183.83 & HT-Direct \\ 
  Cameroon & ADAM/NORD/EXT-NORD & 95-99 & 202.45 & 190.76 & 214.52 & RW2 \\ 
  Cameroon & ADAM/NORD/EXT-NORD & 00-04 & 182.22 & 195.51 & 169.64 & HT-Direct \\ 
  Cameroon & ADAM/NORD/EXT-NORD & 00-04 & 185.29 & 175.08 & 195.88 & RW2 \\ 
  Cameroon & ADAM/NORD/EXT-NORD & 05-09 & 169.03 & 184.41 & 154.68 & HT-Direct \\ 
  Cameroon & ADAM/NORD/EXT-NORD & 05-09 & 152.49 & 142.35 & 163.26 & RW2 \\ 
  Cameroon & ADAM/NORD/EXT-NORD & 10-14 & 149.39 & 175.90 & 126.26 & HT-Direct \\ 
  Cameroon & ADAM/NORD/EXT-NORD & 10-14 & 134.00 & 117.11 & 153.30 & RW2 \\ 
  Cameroon & ADAM/NORD/EXT-NORD & 15-19 & 121.25 & 53.91 & 254.20 & RW2 \\ 
  Cameroon & ALL & 80-84 & 158.55 & 156.21 & 160.95 & IHME \\ 
  Cameroon & ALL & 80-84 & 169.36 & 155.99 & 183.63 & RW2 \\ 
  Cameroon & ALL & 80-84 & 169.28 & 163.27 & 175.03 & UN \\ 
  Cameroon & ALL & 85-89 & 140.84 & 138.75 & 142.90 & IHME \\ 
  Cameroon & ALL & 85-89 & 141.89 & 133.22 & 150.96 & RW2 \\ 
  Cameroon & ALL & 85-89 & 142.18 & 137.56 & 146.54 & UN \\ 
  Cameroon & ALL & 90-94 & 138.19 & 136.10 & 140.09 & IHME \\ 
  Cameroon & ALL & 90-94 & 143.56 & 136.57 & 150.85 & RW2 \\ 
  Cameroon & ALL & 90-94 & 143.13 & 138.46 & 148.01 & UN \\ 
  Cameroon & ALL & 95-99 & 140.21 & 137.84 & 142.57 & IHME \\ 
  Cameroon & ALL & 95-99 & 153.74 & 145.96 & 161.82 & RW2 \\ 
  Cameroon & ALL & 95-99 & 154.14 & 148.71 & 159.73 & UN \\ 
  Cameroon & ALL & 00-04 & 132.62 & 130.14 & 135.27 & IHME \\ 
  Cameroon & ALL & 00-04 & 139.93 & 133.35 & 146.81 & RW2 \\ 
  Cameroon & ALL & 00-04 & 139.95 & 133.56 & 146.00 & UN \\ 
  Cameroon & ALL & 05-09 & 117.53 & 114.49 & 120.78 & IHME \\ 
  Cameroon & ALL & 05-09 & 116.82 & 110.32 & 123.65 & RW2 \\ 
  Cameroon & ALL & 05-09 & 116.63 & 105.17 & 128.48 & UN \\ 
  Cameroon & ALL & 10-14 & 99.64 & 95.07 & 104.52 & IHME \\ 
  Cameroon & ALL & 10-14 & 98.48 & 87.41 & 110.70 & RW2 \\ 
  Cameroon & ALL & 10-14 & 98.85 & 80.91 & 120.50 & UN \\ 
  Cameroon & CENTRE/SUD/EST & 80-84 & 148.34 & 173.72 & 126.10 & HT-Direct \\ 
  Cameroon & CENTRE/SUD/EST & 80-84 & 159.13 & 138.62 & 181.61 & RW2 \\ 
  Cameroon & CENTRE/SUD/EST & 85-89 & 123.21 & 138.17 & 109.67 & HT-Direct \\ 
  Cameroon & CENTRE/SUD/EST & 85-89 & 131.98 & 120.48 & 144.19 & RW2 \\ 
  Cameroon & CENTRE/SUD/EST & 90-94 & 138.00 & 153.23 & 124.06 & HT-Direct \\ 
  Cameroon & CENTRE/SUD/EST & 90-94 & 134.73 & 125.08 & 145.12 & RW2 \\ 
  Cameroon & CENTRE/SUD/EST & 95-99 & 137.79 & 151.57 & 125.09 & HT-Direct \\ 
  Cameroon & CENTRE/SUD/EST & 95-99 & 144.35 & 134.58 & 154.87 & RW2 \\ 
  Cameroon & CENTRE/SUD/EST & 00-04 & 143.06 & 158.49 & 128.90 & HT-Direct \\ 
  Cameroon & CENTRE/SUD/EST & 00-04 & 132.44 & 123.17 & 142.55 & RW2 \\ 
  Cameroon & CENTRE/SUD/EST & 05-09 & 108.10 & 122.82 & 94.95 & HT-Direct \\ 
  Cameroon & CENTRE/SUD/EST & 05-09 & 103.32 & 94.02 & 113.50 & RW2 \\ 
  Cameroon & CENTRE/SUD/EST & 10-14 & 88.00 & 120.02 & 63.90 & HT-Direct \\ 
  Cameroon & CENTRE/SUD/EST & 10-14 & 84.67 & 69.13 & 102.25 & RW2 \\ 
  Cameroon & CENTRE/SUD/EST & 15-19 & 71.07 & 29.18 & 162.97 & RW2 \\ 
  Cameroon & NORD-OUEST/SUD-OUEST & 80-84 & 121.50 & 145.99 & 100.64 & HT-Direct \\ 
  Cameroon & NORD-OUEST/SUD-OUEST & 80-84 & 122.46 & 104.66 & 142.71 & RW2 \\ 
  Cameroon & NORD-OUEST/SUD-OUEST & 85-89 & 86.39 & 102.10 & 72.89 & HT-Direct \\ 
  Cameroon & NORD-OUEST/SUD-OUEST & 85-89 & 99.98 & 89.33 & 111.77 & RW2 \\ 
  Cameroon & NORD-OUEST/SUD-OUEST & 90-94 & 104.27 & 119.32 & 90.92 & HT-Direct \\ 
  Cameroon & NORD-OUEST/SUD-OUEST & 90-94 & 102.30 & 93.43 & 111.77 & RW2 \\ 
  Cameroon & NORD-OUEST/SUD-OUEST & 95-99 & 107.73 & 122.55 & 94.51 & HT-Direct \\ 
  Cameroon & NORD-OUEST/SUD-OUEST & 95-99 & 110.95 & 101.92 & 120.83 & RW2 \\ 
  Cameroon & NORD-OUEST/SUD-OUEST & 00-04 & 111.34 & 128.37 & 96.32 & HT-Direct \\ 
  Cameroon & NORD-OUEST/SUD-OUEST & 00-04 & 102.91 & 93.94 & 112.88 & RW2 \\ 
  Cameroon & NORD-OUEST/SUD-OUEST & 05-09 & 85.67 & 102.81 & 71.17 & HT-Direct \\ 
  Cameroon & NORD-OUEST/SUD-OUEST & 05-09 & 81.60 & 72.15 & 92.13 & RW2 \\ 
  Cameroon & NORD-OUEST/SUD-OUEST & 10-14 & 67.56 & 98.00 & 46.10 & HT-Direct \\ 
  Cameroon & NORD-OUEST/SUD-OUEST & 10-14 & 68.23 & 54.58 & 84.37 & RW2 \\ 
  Cameroon & NORD-OUEST/SUD-OUEST & 15-19 & 58.52 & 23.98 & 136.32 & RW2 \\ 
  Cameroon & OUEST/LITTORAL & 80-84 & 120.23 & 141.80 & 101.54 & HT-Direct \\ 
  Cameroon & OUEST/LITTORAL & 80-84 & 125.26 & 107.90 & 145.22 & RW2 \\ 
  Cameroon & OUEST/LITTORAL & 85-89 & 97.43 & 115.77 & 81.73 & HT-Direct \\ 
  Cameroon & OUEST/LITTORAL & 85-89 & 104.67 & 93.42 & 117.10 & RW2 \\ 
  Cameroon & OUEST/LITTORAL & 90-94 & 104.84 & 122.48 & 89.49 & HT-Direct \\ 
  Cameroon & OUEST/LITTORAL & 90-94 & 109.46 & 99.64 & 119.99 & RW2 \\ 
  Cameroon & OUEST/LITTORAL & 95-99 & 123.76 & 137.69 & 111.06 & HT-Direct \\ 
  Cameroon & OUEST/LITTORAL & 95-99 & 122.01 & 112.60 & 132.17 & RW2 \\ 
  Cameroon & OUEST/LITTORAL & 00-04 & 116.38 & 133.09 & 101.53 & HT-Direct \\ 
  Cameroon & OUEST/LITTORAL & 00-04 & 116.87 & 106.81 & 127.84 & RW2 \\ 
  Cameroon & OUEST/LITTORAL & 05-09 & 107.68 & 132.79 & 86.85 & HT-Direct \\ 
  Cameroon & OUEST/LITTORAL & 05-09 & 97.29 & 85.49 & 110.54 & RW2 \\ 
  Cameroon & OUEST/LITTORAL & 10-14 & 88.84 & 126.66 & 61.51 & HT-Direct \\ 
  Cameroon & OUEST/LITTORAL & 10-14 & 85.77 & 68.41 & 106.58 & RW2 \\ 
  Cameroon & OUEST/LITTORAL & 15-19 & 77.59 & 32.01 & 177.60 & RW2 \\ 
  Cameroon & YAOUNDE/DOUALA & 80-84 & 119.33 & 156.26 & 90.19 & HT-Direct \\ 
  Cameroon & YAOUNDE/DOUALA & 80-84 & 119.33 & 96.39 & 147.62 & RW2 \\ 
  Cameroon & YAOUNDE/DOUALA & 85-89 & 86.58 & 109.53 & 68.07 & HT-Direct \\ 
  Cameroon & YAOUNDE/DOUALA & 85-89 & 93.92 & 80.71 & 108.93 & RW2 \\ 
  Cameroon & YAOUNDE/DOUALA & 90-94 & 93.75 & 113.08 & 77.44 & HT-Direct \\ 
  Cameroon & YAOUNDE/DOUALA & 90-94 & 92.88 & 82.55 & 104.28 & RW2 \\ 
  Cameroon & YAOUNDE/DOUALA & 95-99 & 89.72 & 106.92 & 75.06 & HT-Direct \\ 
  Cameroon & YAOUNDE/DOUALA & 95-99 & 98.27 & 88.27 & 109.05 & RW2 \\ 
  Cameroon & YAOUNDE/DOUALA & 00-04 & 94.35 & 111.87 & 79.33 & HT-Direct \\ 
  Cameroon & YAOUNDE/DOUALA & 00-04 & 89.53 & 80.58 & 99.55 & RW2 \\ 
  Cameroon & YAOUNDE/DOUALA & 05-09 & 78.76 & 94.65 & 65.34 & HT-Direct \\ 
  Cameroon & YAOUNDE/DOUALA & 05-09 & 69.76 & 60.89 & 79.76 & RW2 \\ 
  Cameroon & YAOUNDE/DOUALA & 10-14 & 48.81 & 84.64 & 27.69 & HT-Direct \\ 
  Cameroon & YAOUNDE/DOUALA & 10-14 & 57.03 & 43.91 & 73.39 & RW2 \\ 
  Cameroon & YAOUNDE/DOUALA & 15-19 & 48.07 & 18.83 & 117.04 & RW2 \\ 
  Chad & ALL & 80-84 & 229.89 & 226.29 & 233.50 & IHME \\ 
  Chad & ALL & 80-84 & 237.13 & 209.02 & 267.75 & RW2 \\ 
  Chad & ALL & 80-84 & 236.47 & 227.15 & 245.73 & UN \\ 
  Chad & ALL & 85-89 & 212.02 & 208.56 & 215.04 & IHME \\ 
  Chad & ALL & 85-89 & 222.45 & 200.68 & 245.80 & RW2 \\ 
  Chad & ALL & 85-89 & 223.02 & 215.48 & 230.64 & UN \\ 
  Chad & ALL & 90-94 & 195.25 & 192.48 & 198.26 & IHME \\ 
  Chad & ALL & 90-94 & 209.21 & 193.84 & 225.43 & RW2 \\ 
  Chad & ALL & 90-94 & 209.68 & 202.88 & 216.80 & UN \\ 
  Chad & ALL & 95-99 & 186.83 & 183.80 & 190.07 & IHME \\ 
  Chad & ALL & 95-99 & 199.03 & 187.71 & 210.78 & RW2 \\ 
  Chad & ALL & 95-99 & 198.22 & 191.97 & 205.09 & UN \\ 
  Chad & ALL & 00-04 & 178.32 & 174.29 & 182.45 & IHME \\ 
  Chad & ALL & 00-04 & 183.50 & 172.37 & 195.20 & RW2 \\ 
  Chad & ALL & 00-04 & 184.23 & 176.95 & 191.97 & UN \\ 
  Chad & ALL & 05-09 & 159.79 & 154.39 & 165.02 & IHME \\ 
  Chad & ALL & 05-09 & 172.08 & 157.73 & 187.54 & RW2 \\ 
  Chad & ALL & 05-09 & 171.26 & 160.47 & 182.07 & UN \\ 
  Chad & ALL & 10-14 & 141.83 & 135.05 & 148.74 & IHME \\ 
  Chad & ALL & 10-14 & 152.93 & 142.70 & 163.68 & RW2 \\ 
  Chad & ALL & 10-14 & 153.04 & 134.67 & 175.78 & UN \\ 
  Chad & ZONE 1 & 80-84 & 167.45 & 224.83 & 122.40 & HT-Direct \\ 
  Chad & ZONE 1 & 80-84 & 190.73 & 148.14 & 242.34 & RW2 \\ 
  Chad & ZONE 1 & 85-89 & 148.35 & 193.06 & 112.54 & HT-Direct \\ 
  Chad & ZONE 1 & 85-89 & 189.30 & 159.13 & 224.28 & RW2 \\ 
  Chad & ZONE 1 & 90-94 & 193.20 & 226.56 & 163.72 & HT-Direct \\ 
  Chad & ZONE 1 & 90-94 & 183.50 & 162.50 & 207.00 & RW2 \\ 
  Chad & ZONE 1 & 95-99 & 164.39 & 187.35 & 143.75 & HT-Direct \\ 
  Chad & ZONE 1 & 95-99 & 171.58 & 156.07 & 188.15 & RW2 \\ 
  Chad & ZONE 1 & 00-04 & 156.96 & 177.17 & 138.67 & HT-Direct \\ 
  Chad & ZONE 1 & 00-04 & 165.34 & 150.90 & 180.71 & RW2 \\ 
  Chad & ZONE 1 & 05-09 & 157.94 & 185.07 & 134.13 & HT-Direct \\ 
  Chad & ZONE 1 & 05-09 & 164.45 & 146.88 & 183.76 & RW2 \\ 
  Chad & ZONE 1 & 10-14 & 141.00 & 171.24 & 115.35 & HT-Direct \\ 
  Chad & ZONE 1 & 10-14 & 158.05 & 131.74 & 189.07 & RW2 \\ 
  Chad & ZONE 1 & 15-19 & 149.13 & 67.40 & 298.47 & RW2 \\ 
  Chad & ZONE 2 & 80-84 & 242.74 & 306.55 & 188.60 & HT-Direct \\ 
  Chad & ZONE 2 & 80-84 & 254.60 & 206.28 & 310.88 & RW2 \\ 
  Chad & ZONE 2 & 85-89 & 184.89 & 234.51 & 143.80 & HT-Direct \\ 
  Chad & ZONE 2 & 85-89 & 222.74 & 188.95 & 260.52 & RW2 \\ 
  Chad & ZONE 2 & 90-94 & 171.45 & 208.28 & 139.98 & HT-Direct \\ 
  Chad & ZONE 2 & 90-94 & 192.84 & 170.04 & 218.15 & RW2 \\ 
  Chad & ZONE 2 & 95-99 & 170.44 & 196.51 & 147.19 & HT-Direct \\ 
  Chad & ZONE 2 & 95-99 & 163.08 & 146.99 & 180.35 & RW2 \\ 
  Chad & ZONE 2 & 00-04 & 125.39 & 148.89 & 105.13 & HT-Direct \\ 
  Chad & ZONE 2 & 00-04 & 138.55 & 124.19 & 154.08 & RW2 \\ 
  Chad & ZONE 2 & 05-09 & 110.69 & 135.88 & 89.69 & HT-Direct \\ 
  Chad & ZONE 2 & 05-09 & 121.67 & 106.58 & 138.26 & RW2 \\ 
  Chad & ZONE 2 & 10-14 & 98.12 & 119.27 & 80.38 & HT-Direct \\ 
  Chad & ZONE 2 & 10-14 & 105.82 & 87.59 & 128.10 & RW2 \\ 
  Chad & ZONE 2 & 15-19 & 91.60 & 40.21 & 195.05 & RW2 \\ 
  Chad & ZONE 3 & 80-84 & 299.22 & 372.66 & 234.84 & HT-Direct \\ 
  Chad & ZONE 3 & 80-84 & 296.11 & 243.13 & 357.48 & RW2 \\ 
  Chad & ZONE 3 & 85-89 & 236.85 & 299.84 & 183.63 & HT-Direct \\ 
  Chad & ZONE 3 & 85-89 & 262.23 & 224.99 & 303.66 & RW2 \\ 
  Chad & ZONE 3 & 90-94 & 191.19 & 222.18 & 163.62 & HT-Direct \\ 
  Chad & ZONE 3 & 90-94 & 230.30 & 206.03 & 256.07 & RW2 \\ 
  Chad & ZONE 3 & 95-99 & 206.20 & 235.94 & 179.34 & HT-Direct \\ 
  Chad & ZONE 3 & 95-99 & 199.03 & 181.51 & 217.60 & RW2 \\ 
  Chad & ZONE 3 & 00-04 & 164.77 & 187.62 & 144.21 & HT-Direct \\ 
  Chad & ZONE 3 & 00-04 & 173.39 & 158.75 & 189.05 & RW2 \\ 
  Chad & ZONE 3 & 05-09 & 146.73 & 167.58 & 128.07 & HT-Direct \\ 
  Chad & ZONE 3 & 05-09 & 154.59 & 139.69 & 170.70 & RW2 \\ 
  Chad & ZONE 3 & 10-14 & 122.50 & 145.93 & 102.38 & HT-Direct \\ 
  Chad & ZONE 3 & 10-14 & 134.96 & 115.01 & 158.55 & RW2 \\ 
  Chad & ZONE 3 & 15-19 & 116.53 & 52.95 & 237.10 & RW2 \\ 
  Chad & ZONE 4 & 80-84 & 203.05 & 272.46 & 147.73 & HT-Direct \\ 
  Chad & ZONE 4 & 80-84 & 211.81 & 162.77 & 271.71 & RW2 \\ 
  Chad & ZONE 4 & 85-89 & 140.08 & 192.51 & 100.16 & HT-Direct \\ 
  Chad & ZONE 4 & 85-89 & 186.80 & 153.48 & 225.67 & RW2 \\ 
  Chad & ZONE 4 & 90-94 & 147.21 & 183.84 & 116.83 & HT-Direct \\ 
  Chad & ZONE 4 & 90-94 & 164.02 & 141.50 & 189.42 & RW2 \\ 
  Chad & ZONE 4 & 95-99 & 145.52 & 173.16 & 121.64 & HT-Direct \\ 
  Chad & ZONE 4 & 95-99 & 140.78 & 125.15 & 158.13 & RW2 \\ 
  Chad & ZONE 4 & 00-04 & 120.58 & 141.55 & 102.35 & HT-Direct \\ 
  Chad & ZONE 4 & 00-04 & 120.64 & 108.09 & 134.36 & RW2 \\ 
  Chad & ZONE 4 & 05-09 & 93.41 & 109.41 & 79.54 & HT-Direct \\ 
  Chad & ZONE 4 & 05-09 & 106.11 & 94.24 & 119.04 & RW2 \\ 
  Chad & ZONE 4 & 10-14 & 85.28 & 102.31 & 70.86 & HT-Direct \\ 
  Chad & ZONE 4 & 10-14 & 93.23 & 78.00 & 111.58 & RW2 \\ 
  Chad & ZONE 4 & 15-19 & 81.80 & 35.57 & 176.12 & RW2 \\ 
  Chad & ZONE 5 & 80-84 & 144.19 & 207.32 & 97.90 & HT-Direct \\ 
  Chad & ZONE 5 & 80-84 & 189.37 & 145.88 & 241.28 & RW2 \\ 
  Chad & ZONE 5 & 85-89 & 189.24 & 248.24 & 141.62 & HT-Direct \\ 
  Chad & ZONE 5 & 85-89 & 184.43 & 152.60 & 221.26 & RW2 \\ 
  Chad & ZONE 5 & 90-94 & 162.12 & 197.39 & 132.12 & HT-Direct \\ 
  Chad & ZONE 5 & 90-94 & 176.27 & 153.12 & 202.03 & RW2 \\ 
  Chad & ZONE 5 & 95-99 & 152.60 & 192.95 & 119.43 & HT-Direct \\ 
  Chad & ZONE 5 & 95-99 & 163.89 & 145.64 & 183.76 & RW2 \\ 
  Chad & ZONE 5 & 00-04 & 156.70 & 184.85 & 132.14 & HT-Direct \\ 
  Chad & ZONE 5 & 00-04 & 153.98 & 138.06 & 171.11 & RW2 \\ 
  Chad & ZONE 5 & 05-09 & 123.34 & 153.27 & 98.58 & HT-Direct \\ 
  Chad & ZONE 5 & 05-09 & 146.67 & 129.40 & 165.87 & RW2 \\ 
  Chad & ZONE 5 & 10-14 & 119.85 & 141.31 & 101.26 & HT-Direct \\ 
  Chad & ZONE 5 & 10-14 & 134.83 & 114.50 & 158.49 & RW2 \\ 
  Chad & ZONE 5 & 15-19 & 121.95 & 56.40 & 244.40 & RW2 \\ 
  Chad & ZONE 6 & 80-84 & 190.20 & 266.92 & 131.57 & HT-Direct \\ 
  Chad & ZONE 6 & 80-84 & 175.28 & 133.38 & 230.10 & RW2 \\ 
  Chad & ZONE 6 & 85-89 & 159.45 & 202.65 & 124.03 & HT-Direct \\ 
  Chad & ZONE 6 & 85-89 & 171.25 & 141.66 & 205.18 & RW2 \\ 
  Chad & ZONE 6 & 90-94 & 153.84 & 190.64 & 123.07 & HT-Direct \\ 
  Chad & ZONE 6 & 90-94 & 165.42 & 143.77 & 189.27 & RW2 \\ 
  Chad & ZONE 6 & 95-99 & 125.35 & 151.68 & 103.04 & HT-Direct \\ 
  Chad & ZONE 6 & 95-99 & 158.68 & 140.89 & 176.98 & RW2 \\ 
  Chad & ZONE 6 & 00-04 & 158.13 & 178.92 & 139.35 & HT-Direct \\ 
  Chad & ZONE 6 & 00-04 & 159.16 & 145.27 & 173.91 & RW2 \\ 
  Chad & ZONE 6 & 05-09 & 165.67 & 186.22 & 146.98 & HT-Direct \\ 
  Chad & ZONE 6 & 05-09 & 163.04 & 148.40 & 178.91 & RW2 \\ 
  Chad & ZONE 6 & 10-14 & 129.62 & 151.71 & 110.33 & HT-Direct \\ 
  Chad & ZONE 6 & 10-14 & 158.47 & 136.73 & 183.30 & RW2 \\ 
  Chad & ZONE 6 & 15-19 & 150.81 & 70.01 & 295.52 & RW2 \\ 
  Chad & ZONE 7 & 80-84 & 210.70 & 276.54 & 157.13 & HT-Direct \\ 
  Chad & ZONE 7 & 80-84 & 238.72 & 187.39 & 295.14 & RW2 \\ 
  Chad & ZONE 7 & 85-89 & 202.04 & 249.50 & 161.66 & HT-Direct \\ 
  Chad & ZONE 7 & 85-89 & 250.18 & 214.16 & 289.37 & RW2 \\ 
  Chad & ZONE 7 & 90-94 & 221.09 & 255.93 & 189.78 & HT-Direct \\ 
  Chad & ZONE 7 & 90-94 & 257.58 & 232.41 & 284.47 & RW2 \\ 
  Chad & ZONE 7 & 95-99 & 268.87 & 294.52 & 244.67 & HT-Direct \\ 
  Chad & ZONE 7 & 95-99 & 255.67 & 237.31 & 275.40 & RW2 \\ 
  Chad & ZONE 7 & 00-04 & 236.21 & 265.94 & 208.86 & HT-Direct \\ 
  Chad & ZONE 7 & 00-04 & 248.96 & 229.45 & 270.85 & RW2 \\ 
  Chad & ZONE 7 & 05-09 & 247.06 & 291.14 & 207.71 & HT-Direct \\ 
  Chad & ZONE 7 & 05-09 & 238.38 & 216.39 & 261.89 & RW2 \\ 
  Chad & ZONE 7 & 10-14 & 175.32 & 195.23 & 157.05 & HT-Direct \\ 
  Chad & ZONE 7 & 10-14 & 214.12 & 189.32 & 240.23 & RW2 \\ 
  Chad & ZONE 7 & 15-19 & 187.90 & 90.63 & 345.34 & RW2 \\ 
  Chad & ZONE 8 & 80-84 & 224.78 & 300.63 & 163.59 & HT-Direct \\ 
  Chad & ZONE 8 & 80-84 & 261.49 & 207.35 & 321.46 & RW2 \\ 
  Chad & ZONE 8 & 85-89 & 226.53 & 271.48 & 187.10 & HT-Direct \\ 
  Chad & ZONE 8 & 85-89 & 253.54 & 217.09 & 293.07 & RW2 \\ 
  Chad & ZONE 8 & 90-94 & 216.01 & 265.49 & 173.56 & HT-Direct \\ 
  Chad & ZONE 8 & 90-94 & 241.89 & 214.23 & 271.82 & RW2 \\ 
  Chad & ZONE 8 & 95-99 & 220.46 & 252.68 & 191.30 & HT-Direct \\ 
  Chad & ZONE 8 & 95-99 & 223.75 & 203.33 & 245.79 & RW2 \\ 
  Chad & ZONE 8 & 00-04 & 202.47 & 235.33 & 173.15 & HT-Direct \\ 
  Chad & ZONE 8 & 00-04 & 205.60 & 187.58 & 225.59 & RW2 \\ 
  Chad & ZONE 8 & 05-09 & 186.19 & 209.14 & 165.24 & HT-Direct \\ 
  Chad & ZONE 8 & 05-09 & 187.45 & 170.95 & 205.41 & RW2 \\ 
  Chad & ZONE 8 & 10-14 & 130.85 & 150.99 & 113.04 & HT-Direct \\ 
  Chad & ZONE 8 & 10-14 & 161.61 & 140.20 & 185.09 & RW2 \\ 
  Chad & ZONE 8 & 15-19 & 135.92 & 62.65 & 269.76 & RW2 \\ 
  Comoros & ALL & 80-84 & 154.72 & 150.37 & 159.08 & IHME \\ 
  Comoros & ALL & 80-84 & 164.34 & 141.77 & 189.72 & RW2 \\ 
  Comoros & ALL & 80-84 & 164.08 & 156.29 & 172.55 & UN \\ 
  Comoros & ALL & 85-89 & 131.45 & 127.70 & 135.50 & IHME \\ 
  Comoros & ALL & 85-89 & 136.56 & 116.60 & 159.15 & RW2 \\ 
  Comoros & ALL & 85-89 & 137.66 & 130.88 & 145.53 & UN \\ 
  Comoros & ALL & 90-94 & 113.33 & 109.47 & 116.99 & IHME \\ 
  Comoros & ALL & 90-94 & 117.71 & 101.57 & 136.09 & RW2 \\ 
  Comoros & ALL & 90-94 & 116.85 & 110.74 & 123.03 & UN \\ 
  Comoros & ALL & 95-99 & 95.92 & 90.91 & 101.18 & IHME \\ 
  Comoros & ALL & 95-99 & 102.93 & 73.27 & 142.05 & RW2 \\ 
  Comoros & ALL & 95-99 & 101.90 & 94.26 & 109.80 & UN \\ 
  Comoros & ALL & 00-04 & 77.83 & 71.47 & 84.91 & IHME \\ 
  Comoros & ALL & 00-04 & 94.29 & 42.54 & 195.48 & RW2 \\ 
  Comoros & ALL & 00-04 & 97.73 & 87.11 & 108.71 & UN \\ 
  Comoros & ALL & 05-09 & 62.98 & 55.76 & 71.17 & IHME \\ 
  Comoros & ALL & 05-09 & 89.76 & 35.41 & 208.83 & RW2 \\ 
  Comoros & ALL & 05-09 & 95.07 & 79.54 & 113.51 & UN \\ 
  Comoros & ALL & 10-14 & 50.62 & 43.93 & 57.81 & IHME \\ 
  Comoros & ALL & 10-14 & 87.45 & 44.68 & 163.48 & RW2 \\ 
  Comoros & ALL & 10-14 & 85.93 & 66.84 & 110.09 & UN \\ 
  Comoros & MOHELI & 80-84 & 164.52 & 248.69 & 104.87 & HT-Direct \\ 
  Comoros & MOHELI & 80-84 & 168.02 & 110.21 & 247.31 & RW2 \\ 
  Comoros & MOHELI & 85-89 & 122.96 & 207.33 & 69.89 & HT-Direct \\ 
  Comoros & MOHELI & 85-89 & 129.48 & 92.23 & 179.12 & RW2 \\ 
  Comoros & MOHELI & 90-94 & 72.72 & 113.18 & 45.97 & HT-Direct \\ 
  Comoros & MOHELI & 90-94 & 107.54 & 75.37 & 150.51 & RW2 \\ 
  Comoros & MOHELI & 95-99 & 66.63 & 98.47 & 44.58 & HT-Direct \\ 
  Comoros & MOHELI & 95-99 & 92.97 & 56.36 & 149.07 & RW2 \\ 
  Comoros & MOHELI & 00-04 & 41.40 & 62.21 & 27.35 & HT-Direct \\ 
  Comoros & MOHELI & 00-04 & 84.28 & 35.11 & 189.88 & RW2 \\ 
  Comoros & MOHELI & 05-09 & 27.35 & 50.73 & 14.58 & HT-Direct \\ 
  Comoros & MOHELI & 05-09 & 79.90 & 26.20 & 219.10 & RW2 \\ 
  Comoros & MOHELI & 10-14 & 70.35 & 146.95 & 32.18 & HT-Direct \\ 
  Comoros & MOHELI & 10-14 & 77.37 & 22.68 & 234.68 & RW2 \\ 
  Comoros & MOHELI & 15-19 & 75.07 & 8.29 & 451.39 & RW2 \\ 
  Comoros & NDZOUANI & 80-84 & 174.50 & 221.34 & 135.85 & HT-Direct \\ 
  Comoros & NDZOUANI & 80-84 & 185.78 & 147.35 & 232.28 & RW2 \\ 
  Comoros & NDZOUANI & 85-89 & 122.95 & 149.11 & 100.83 & HT-Direct \\ 
  Comoros & NDZOUANI & 85-89 & 144.22 & 120.36 & 171.61 & RW2 \\ 
  Comoros & NDZOUANI & 90-94 & 101.84 & 122.39 & 84.41 & HT-Direct \\ 
  Comoros & NDZOUANI & 90-94 & 119.33 & 98.13 & 144.33 & RW2 \\ 
  Comoros & NDZOUANI & 95-99 & 58.10 & 81.23 & 41.27 & HT-Direct \\ 
  Comoros & NDZOUANI & 95-99 & 101.40 & 69.26 & 146.00 & RW2 \\ 
  Comoros & NDZOUANI & 00-04 & 52.66 & 76.39 & 36.01 & HT-Direct \\ 
  Comoros & NDZOUANI & 00-04 & 89.86 & 41.92 & 182.45 & RW2 \\ 
  Comoros & NDZOUANI & 05-09 & 42.27 & 63.31 & 28.01 & HT-Direct \\ 
  Comoros & NDZOUANI & 05-09 & 82.89 & 33.10 & 192.37 & RW2 \\ 
  Comoros & NDZOUANI & 10-14 & 55.60 & 88.31 & 34.55 & HT-Direct \\ 
  Comoros & NDZOUANI & 10-14 & 78.26 & 32.99 & 174.79 & RW2 \\ 
  Comoros & NDZOUANI & 15-19 & 73.86 & 11.22 & 365.63 & RW2 \\ 
  Comoros & NGAZIDJA & 80-84 & 137.42 & 159.67 & 117.84 & HT-Direct \\ 
  Comoros & NGAZIDJA & 80-84 & 146.63 & 125.37 & 170.73 & RW2 \\ 
  Comoros & NGAZIDJA & 85-89 & 105.57 & 129.30 & 85.77 & HT-Direct \\ 
  Comoros & NGAZIDJA & 85-89 & 124.42 & 103.95 & 148.14 & RW2 \\ 
  Comoros & NGAZIDJA & 90-94 & 94.66 & 112.25 & 79.58 & HT-Direct \\ 
  Comoros & NGAZIDJA & 90-94 & 113.53 & 94.68 & 135.93 & RW2 \\ 
  Comoros & NGAZIDJA & 95-99 & 80.42 & 105.43 & 60.93 & HT-Direct \\ 
  Comoros & NGAZIDJA & 95-99 & 106.48 & 74.67 & 149.49 & RW2 \\ 
  Comoros & NGAZIDJA & 00-04 & 51.51 & 69.91 & 37.75 & HT-Direct \\ 
  Comoros & NGAZIDJA & 00-04 & 102.91 & 49.45 & 201.94 & RW2 \\ 
  Comoros & NGAZIDJA & 05-09 & 57.07 & 76.20 & 42.52 & HT-Direct \\ 
  Comoros & NGAZIDJA & 05-09 & 102.84 & 42.91 & 225.84 & RW2 \\ 
  Comoros & NGAZIDJA & 10-14 & 59.22 & 90.94 & 38.10 & HT-Direct \\ 
  Comoros & NGAZIDJA & 10-14 & 104.23 & 46.46 & 215.53 & RW2 \\ 
  Comoros & NGAZIDJA & 15-19 & 105.93 & 16.93 & 452.55 & RW2 \\ 
  Congo & ALL & 80-84 & 112.06 & 107.16 & 117.11 & IHME \\ 
  Congo & ALL & 80-84 & 106.72 & 87.03 & 130.28 & RW2 \\ 
  Congo & ALL & 80-84 & 107.21 & 97.73 & 116.73 & UN \\ 
  Congo & ALL & 85-89 & 96.31 & 93.24 & 99.30 & IHME \\ 
  Congo & ALL & 85-89 & 95.22 & 83.68 & 108.10 & RW2 \\ 
  Congo & ALL & 85-89 & 94.68 & 88.56 & 101.05 & UN \\ 
  Congo & ALL & 90-94 & 95.27 & 92.60 & 97.85 & IHME \\ 
  Congo & ALL & 90-94 & 97.68 & 87.66 & 108.59 & RW2 \\ 
  Congo & ALL & 90-94 & 98.32 & 93.04 & 103.57 & UN \\ 
  Congo & ALL & 95-99 & 108.34 & 104.24 & 112.75 & IHME \\ 
  Congo & ALL & 95-99 & 116.21 & 107.24 & 125.76 & RW2 \\ 
  Congo & ALL & 95-99 & 115.58 & 110.27 & 120.96 & UN \\ 
  Congo & ALL & 00-04 & 103.83 & 100.74 & 106.98 & IHME \\ 
  Congo & ALL & 00-04 & 113.89 & 105.82 & 122.58 & RW2 \\ 
  Congo & ALL & 00-04 & 113.93 & 108.87 & 118.74 & UN \\ 
  Congo & ALL & 05-09 & 79.42 & 76.79 & 81.99 & IHME \\ 
  Congo & ALL & 05-09 & 79.17 & 70.08 & 89.16 & RW2 \\ 
  Congo & ALL & 05-09 & 79.66 & 75.41 & 84.77 & UN \\ 
  Congo & ALL & 10-14 & 62.66 & 58.56 & 67.24 & IHME \\ 
  Congo & ALL & 10-14 & 54.21 & 45.46 & 64.47 & RW2 \\ 
  Congo & ALL & 10-14 & 53.17 & 47.00 & 60.03 & UN \\ 
  Congo & BRAZZAVILLE & 80-84 & 80.37 & 121.51 & 52.32 & HT-Direct \\ 
  Congo & BRAZZAVILLE & 80-84 & 75.19 & 54.51 & 102.75 & RW2 \\ 
  Congo & BRAZZAVILLE & 85-89 & 64.78 & 86.74 & 48.08 & HT-Direct \\ 
  Congo & BRAZZAVILLE & 85-89 & 68.99 & 56.20 & 84.46 & RW2 \\ 
  Congo & BRAZZAVILLE & 90-94 & 61.65 & 79.69 & 47.48 & HT-Direct \\ 
  Congo & BRAZZAVILLE & 90-94 & 74.61 & 63.83 & 86.99 & RW2 \\ 
  Congo & BRAZZAVILLE & 95-99 & 128.71 & 151.23 & 109.11 & HT-Direct \\ 
  Congo & BRAZZAVILLE & 95-99 & 103.59 & 92.41 & 116.20 & RW2 \\ 
  Congo & BRAZZAVILLE & 00-04 & 93.21 & 110.92 & 78.08 & HT-Direct \\ 
  Congo & BRAZZAVILLE & 00-04 & 105.74 & 93.25 & 119.55 & RW2 \\ 
  Congo & BRAZZAVILLE & 05-09 & 68.38 & 90.44 & 51.40 & HT-Direct \\ 
  Congo & BRAZZAVILLE & 05-09 & 83.21 & 68.79 & 99.98 & RW2 \\ 
  Congo & BRAZZAVILLE & 10-14 & 80.14 & 127.51 & 49.37 & HT-Direct \\ 
  Congo & BRAZZAVILLE & 10-14 & 55.01 & 41.03 & 74.01 & RW2 \\ 
  Congo & BRAZZAVILLE & 15-19 & 34.40 & 11.71 & 97.51 & RW2 \\ 
  Congo & NORD & 80-84 & 119.86 & 164.48 & 86.10 & HT-Direct \\ 
  Congo & NORD & 80-84 & 123.68 & 95.56 & 159.28 & RW2 \\ 
  Congo & NORD & 85-89 & 108.94 & 133.69 & 88.30 & HT-Direct \\ 
  Congo & NORD & 85-89 & 104.64 & 89.35 & 122.42 & RW2 \\ 
  Congo & NORD & 90-94 & 94.11 & 111.34 & 79.30 & HT-Direct \\ 
  Congo & NORD & 90-94 & 102.75 & 90.80 & 115.70 & RW2 \\ 
  Congo & NORD & 95-99 & 142.46 & 160.16 & 126.42 & HT-Direct \\ 
  Congo & NORD & 95-99 & 130.48 & 119.58 & 142.10 & RW2 \\ 
  Congo & NORD & 00-04 & 122.69 & 133.67 & 112.49 & HT-Direct \\ 
  Congo & NORD & 00-04 & 126.53 & 117.51 & 136.33 & RW2 \\ 
  Congo & NORD & 05-09 & 92.19 & 104.62 & 81.09 & HT-Direct \\ 
  Congo & NORD & 05-09 & 93.22 & 83.15 & 104.51 & RW2 \\ 
  Congo & NORD & 10-14 & 60.70 & 77.28 & 47.50 & HT-Direct \\ 
  Congo & NORD & 10-14 & 54.59 & 45.70 & 64.95 & RW2 \\ 
  Congo & NORD & 15-19 & 29.42 & 10.58 & 77.40 & RW2 \\ 
  Congo & POINTE NOIRE & 80-84 & 93.52 & 142.88 & 60.02 & HT-Direct \\ 
  Congo & POINTE NOIRE & 80-84 & 110.51 & 80.47 & 151.23 & RW2 \\ 
  Congo & POINTE NOIRE & 85-89 & 105.98 & 142.81 & 77.79 & HT-Direct \\ 
  Congo & POINTE NOIRE & 85-89 & 90.97 & 73.50 & 112.29 & RW2 \\ 
  Congo & POINTE NOIRE & 90-94 & 79.97 & 103.85 & 61.21 & HT-Direct \\ 
  Congo & POINTE NOIRE & 90-94 & 85.81 & 72.95 & 100.87 & RW2 \\ 
  Congo & POINTE NOIRE & 95-99 & 115.66 & 140.18 & 94.96 & HT-Direct \\ 
  Congo & POINTE NOIRE & 95-99 & 103.98 & 91.35 & 117.86 & RW2 \\ 
  Congo & POINTE NOIRE & 00-04 & 88.65 & 108.93 & 71.84 & HT-Direct \\ 
  Congo & POINTE NOIRE & 00-04 & 96.18 & 83.30 & 110.19 & RW2 \\ 
  Congo & POINTE NOIRE & 05-09 & 52.13 & 69.27 & 39.05 & HT-Direct \\ 
  Congo & POINTE NOIRE & 05-09 & 69.52 & 57.36 & 83.91 & RW2 \\ 
  Congo & POINTE NOIRE & 10-14 & 68.53 & 105.75 & 43.78 & HT-Direct \\ 
  Congo & POINTE NOIRE & 10-14 & 42.53 & 31.87 & 58.01 & RW2 \\ 
  Congo & POINTE NOIRE & 15-19 & 24.68 & 8.39 & 72.50 & RW2 \\ 
  Congo & SUD & 80-84 & 131.98 & 175.90 & 97.72 & HT-Direct \\ 
  Congo & SUD & 80-84 & 139.28 & 109.73 & 174.23 & RW2 \\ 
  Congo & SUD & 85-89 & 100.64 & 126.63 & 79.50 & HT-Direct \\ 
  Congo & SUD & 85-89 & 115.62 & 98.43 & 135.04 & RW2 \\ 
  Congo & SUD & 90-94 & 120.06 & 139.19 & 103.25 & HT-Direct \\ 
  Congo & SUD & 90-94 & 110.69 & 98.74 & 123.93 & RW2 \\ 
  Congo & SUD & 95-99 & 133.53 & 152.73 & 116.40 & HT-Direct \\ 
  Congo & SUD & 95-99 & 134.12 & 122.67 & 146.66 & RW2 \\ 
  Congo & SUD & 00-04 & 133.71 & 150.67 & 118.39 & HT-Direct \\ 
  Congo & SUD & 00-04 & 122.96 & 112.37 & 134.60 & RW2 \\ 
  Congo & SUD & 05-09 & 72.61 & 85.61 & 61.46 & HT-Direct \\ 
  Congo & SUD & 05-09 & 85.04 & 74.82 & 96.63 & RW2 \\ 
  Congo & SUD & 10-14 & 55.61 & 71.04 & 43.37 & HT-Direct \\ 
  Congo & SUD & 10-14 & 47.91 & 40.19 & 56.91 & RW2 \\ 
  Congo & SUD & 15-19 & 25.11 & 9.06 & 66.07 & RW2 \\ 
  C\^{o}te d'Ivoire & ALL & 80-84 & 155.44 & 153.04 & 157.92 & IHME \\ 
  C\^{o}te d'Ivoire & ALL & 80-84 & 158.39 & 114.30 & 215.37 & RW2 \\ 
  C\^{o}te d'Ivoire & ALL & 80-84 & 160.36 & 154.83 & 165.77 & UN \\ 
  C\^{o}te d'Ivoire & ALL & 85-89 & 149.63 & 147.56 & 151.83 & IHME \\ 
  C\^{o}te d'Ivoire & ALL & 85-89 & 155.10 & 130.26 & 183.60 & RW2 \\ 
  C\^{o}te d'Ivoire & ALL & 85-89 & 152.92 & 148.31 & 157.65 & UN \\ 
  C\^{o}te d'Ivoire & ALL & 90-94 & 148.94 & 146.74 & 151.09 & IHME \\ 
  C\^{o}te d'Ivoire & ALL & 90-94 & 152.43 & 135.08 & 171.50 & RW2 \\ 
  C\^{o}te d'Ivoire & ALL & 90-94 & 153.01 & 148.26 & 157.66 & UN \\ 
  C\^{o}te d'Ivoire & ALL & 95-99 & 145.95 & 143.33 & 148.31 & IHME \\ 
  C\^{o}te d'Ivoire & ALL & 95-99 & 150.77 & 136.70 & 165.94 & RW2 \\ 
  C\^{o}te d'Ivoire & ALL & 95-99 & 150.89 & 145.62 & 156.27 & UN \\ 
  C\^{o}te d'Ivoire & ALL & 00-04 & 135.89 & 133.26 & 138.59 & IHME \\ 
  C\^{o}te d'Ivoire & ALL & 00-04 & 139.70 & 127.98 & 152.38 & RW2 \\ 
  C\^{o}te d'Ivoire & ALL & 00-04 & 139.57 & 133.91 & 145.15 & UN \\ 
  C\^{o}te d'Ivoire & ALL & 05-09 & 121.06 & 117.85 & 124.21 & IHME \\ 
  C\^{o}te d'Ivoire & ALL & 05-09 & 121.06 & 111.56 & 131.24 & RW2 \\ 
  C\^{o}te d'Ivoire & ALL & 05-09 & 121.06 & 115.65 & 126.46 & UN \\ 
  C\^{o}te d'Ivoire & ALL & 10-14 & 104.54 & 100.31 & 108.75 & IHME \\ 
  C\^{o}te d'Ivoire & ALL & 10-14 & 102.70 & 85.19 & 123.18 & RW2 \\ 
  C\^{o}te d'Ivoire & ALL & 10-14 & 102.89 & 94.95 & 110.57 & UN \\ 
  C\^{o}te d'Ivoire & CENTRE & 80-84 & 160.59 & 265.44 & 91.97 & HT-Direct \\ 
  C\^{o}te d'Ivoire & CENTRE & 80-84 & 168.41 & 106.67 & 257.22 & RW2 \\ 
  C\^{o}te d'Ivoire & CENTRE & 85-89 & 119.37 & 202.67 & 67.42 & HT-Direct \\ 
  C\^{o}te d'Ivoire & CENTRE & 85-89 & 140.18 & 100.26 & 193.04 & RW2 \\ 
  C\^{o}te d'Ivoire & CENTRE & 90-94 & 137.82 & 210.27 & 87.56 & HT-Direct \\ 
  C\^{o}te d'Ivoire & CENTRE & 90-94 & 139.94 & 109.68 & 176.96 & RW2 \\ 
  C\^{o}te d'Ivoire & CENTRE & 95-99 & 125.98 & 163.04 & 96.38 & HT-Direct \\ 
  C\^{o}te d'Ivoire & CENTRE & 95-99 & 148.69 & 125.26 & 175.23 & RW2 \\ 
  C\^{o}te d'Ivoire & CENTRE & 00-04 & 158.78 & 199.53 & 125.06 & HT-Direct \\ 
  C\^{o}te d'Ivoire & CENTRE & 00-04 & 153.45 & 131.88 & 177.89 & RW2 \\ 
  C\^{o}te d'Ivoire & CENTRE & 05-09 & 152.11 & 193.31 & 118.41 & HT-Direct \\ 
  C\^{o}te d'Ivoire & CENTRE & 05-09 & 141.61 & 115.94 & 171.82 & RW2 \\ 
  C\^{o}te d'Ivoire & CENTRE & 10-14 & 61.13 & 139.39 & 25.50 & HT-Direct \\ 
  C\^{o}te d'Ivoire & CENTRE & 10-14 & 131.33 & 92.27 & 183.38 & RW2 \\ 
  C\^{o}te d'Ivoire & CENTRE & 15-19 & 123.63 & 45.03 & 295.00 & RW2 \\ 
  C\^{o}te d'Ivoire & CENTRE-EST & 80-84 & 162.46 & 280.28 & 88.10 & HT-Direct \\ 
  C\^{o}te d'Ivoire & CENTRE-EST & 80-84 & 152.28 & 92.10 & 243.73 & RW2 \\ 
  C\^{o}te d'Ivoire & CENTRE-EST & 85-89 & 106.01 & 220.16 & 47.45 & HT-Direct \\ 
  C\^{o}te d'Ivoire & CENTRE-EST & 85-89 & 125.17 & 87.82 & 175.79 & RW2 \\ 
  C\^{o}te d'Ivoire & CENTRE-EST & 90-94 & 120.90 & 162.24 & 88.97 & HT-Direct \\ 
  C\^{o}te d'Ivoire & CENTRE-EST & 90-94 & 123.85 & 97.65 & 155.72 & RW2 \\ 
  C\^{o}te d'Ivoire & CENTRE-EST & 95-99 & 101.71 & 138.75 & 73.71 & HT-Direct \\ 
  C\^{o}te d'Ivoire & CENTRE-EST & 95-99 & 132.01 & 109.50 & 157.48 & RW2 \\ 
  C\^{o}te d'Ivoire & CENTRE-EST & 00-04 & 130.38 & 169.98 & 98.91 & HT-Direct \\ 
  C\^{o}te d'Ivoire & CENTRE-EST & 00-04 & 138.68 & 116.19 & 164.19 & RW2 \\ 
  C\^{o}te d'Ivoire & CENTRE-EST & 05-09 & 160.01 & 214.63 & 117.21 & HT-Direct \\ 
  C\^{o}te d'Ivoire & CENTRE-EST & 05-09 & 131.66 & 103.54 & 166.37 & RW2 \\ 
  C\^{o}te d'Ivoire & CENTRE-EST & 10-14 & 67.70 & 155.70 & 27.80 & HT-Direct \\ 
  C\^{o}te d'Ivoire & CENTRE-EST & 10-14 & 125.42 & 84.30 & 184.69 & RW2 \\ 
  C\^{o}te d'Ivoire & CENTRE-EST & 15-19 & 120.97 & 42.98 & 301.44 & RW2 \\ 
  C\^{o}te d'Ivoire & CENTRE-NORD & 80-84 & 127.06 & 189.30 & 83.19 & HT-Direct \\ 
  C\^{o}te d'Ivoire & CENTRE-NORD & 80-84 & 162.55 & 109.68 & 233.27 & RW2 \\ 
  C\^{o}te d'Ivoire & CENTRE-NORD & 85-89 & 116.88 & 176.47 & 75.57 & HT-Direct \\ 
  C\^{o}te d'Ivoire & CENTRE-NORD & 85-89 & 122.43 & 89.54 & 163.77 & RW2 \\ 
  C\^{o}te d'Ivoire & CENTRE-NORD & 90-94 & 82.78 & 137.49 & 48.62 & HT-Direct \\ 
  C\^{o}te d'Ivoire & CENTRE-NORD & 90-94 & 110.27 & 85.08 & 141.54 & RW2 \\ 
  C\^{o}te d'Ivoire & CENTRE-NORD & 95-99 & 132.57 & 218.07 & 77.27 & HT-Direct \\ 
  C\^{o}te d'Ivoire & CENTRE-NORD & 95-99 & 105.61 & 84.23 & 131.41 & RW2 \\ 
  C\^{o}te d'Ivoire & CENTRE-NORD & 00-04 & 71.74 & 109.00 & 46.55 & HT-Direct \\ 
  C\^{o}te d'Ivoire & CENTRE-NORD & 00-04 & 97.29 & 77.87 & 120.85 & RW2 \\ 
  C\^{o}te d'Ivoire & CENTRE-NORD & 05-09 & 81.70 & 117.29 & 56.22 & HT-Direct \\ 
  C\^{o}te d'Ivoire & CENTRE-NORD & 05-09 & 79.81 & 60.68 & 104.11 & RW2 \\ 
  C\^{o}te d'Ivoire & CENTRE-NORD & 10-14 & 57.26 & 112.87 & 28.17 & HT-Direct \\ 
  C\^{o}te d'Ivoire & CENTRE-NORD & 10-14 & 66.34 & 44.38 & 98.12 & RW2 \\ 
  C\^{o}te d'Ivoire & CENTRE-NORD & 15-19 & 55.93 & 19.71 & 151.53 & RW2 \\ 
  C\^{o}te d'Ivoire & CENTRE-OUEST & 80-84 & 53.46 & 189.78 & 13.44 & HT-Direct \\ 
  C\^{o}te d'Ivoire & CENTRE-OUEST & 80-84 & 174.79 & 99.11 & 288.65 & RW2 \\ 
  C\^{o}te d'Ivoire & CENTRE-OUEST & 85-89 & 95.55 & 203.46 & 41.86 & HT-Direct \\ 
  C\^{o}te d'Ivoire & CENTRE-OUEST & 85-89 & 135.56 & 91.25 & 197.01 & RW2 \\ 
  C\^{o}te d'Ivoire & CENTRE-OUEST & 90-94 & 143.56 & 197.48 & 102.48 & HT-Direct \\ 
  C\^{o}te d'Ivoire & CENTRE-OUEST & 90-94 & 124.87 & 95.59 & 161.67 & RW2 \\ 
  C\^{o}te d'Ivoire & CENTRE-OUEST & 95-99 & 92.43 & 129.04 & 65.42 & HT-Direct \\ 
  C\^{o}te d'Ivoire & CENTRE-OUEST & 95-99 & 120.62 & 99.04 & 146.20 & RW2 \\ 
  C\^{o}te d'Ivoire & CENTRE-OUEST & 00-04 & 119.58 & 168.48 & 83.46 & HT-Direct \\ 
  C\^{o}te d'Ivoire & CENTRE-OUEST & 00-04 & 111.23 & 93.04 & 132.57 & RW2 \\ 
  C\^{o}te d'Ivoire & CENTRE-OUEST & 05-09 & 84.60 & 110.52 & 64.33 & HT-Direct \\ 
  C\^{o}te d'Ivoire & CENTRE-OUEST & 05-09 & 90.84 & 72.46 & 113.01 & RW2 \\ 
  C\^{o}te d'Ivoire & CENTRE-OUEST & 10-14 & 70.92 & 139.11 & 34.81 & HT-Direct \\ 
  C\^{o}te d'Ivoire & CENTRE-OUEST & 10-14 & 75.03 & 51.53 & 107.52 & RW2 \\ 
  C\^{o}te d'Ivoire & CENTRE-OUEST & 15-19 & 63.23 & 22.53 & 165.02 & RW2 \\ 
  C\^{o}te d'Ivoire & NORD & 80-84 & 234.48 & 394.16 & 126.03 & HT-Direct \\ 
  C\^{o}te d'Ivoire & NORD & 80-84 & 247.92 & 158.70 & 364.52 & RW2 \\ 
  C\^{o}te d'Ivoire & NORD & 85-89 & 185.93 & 296.60 & 110.09 & HT-Direct \\ 
  C\^{o}te d'Ivoire & NORD & 85-89 & 213.44 & 156.04 & 283.80 & RW2 \\ 
  C\^{o}te d'Ivoire & NORD & 90-94 & 192.67 & 277.88 & 128.93 & HT-Direct \\ 
  C\^{o}te d'Ivoire & NORD & 90-94 & 217.05 & 174.60 & 266.40 & RW2 \\ 
  C\^{o}te d'Ivoire & NORD & 95-99 & 223.74 & 273.84 & 180.52 & HT-Direct \\ 
  C\^{o}te d'Ivoire & NORD & 95-99 & 232.72 & 200.91 & 267.63 & RW2 \\ 
  C\^{o}te d'Ivoire & NORD & 00-04 & 244.98 & 289.95 & 204.97 & HT-Direct \\ 
  C\^{o}te d'Ivoire & NORD & 00-04 & 240.39 & 214.03 & 268.99 & RW2 \\ 
  C\^{o}te d'Ivoire & NORD & 05-09 & 211.03 & 246.95 & 179.10 & HT-Direct \\ 
  C\^{o}te d'Ivoire & NORD & 05-09 & 223.92 & 197.19 & 253.48 & RW2 \\ 
  C\^{o}te d'Ivoire & NORD & 10-14 & 211.81 & 274.95 & 159.96 & HT-Direct \\ 
  C\^{o}te d'Ivoire & NORD & 10-14 & 211.57 & 170.01 & 260.34 & RW2 \\ 
  C\^{o}te d'Ivoire & NORD & 15-19 & 202.53 & 85.46 & 413.40 & RW2 \\ 
  C\^{o}te d'Ivoire & NORD-EST & 80-84 & 274.70 & 461.90 & 143.19 & HT-Direct \\ 
  C\^{o}te d'Ivoire & NORD-EST & 80-84 & 216.17 & 129.74 & 340.78 & RW2 \\ 
  C\^{o}te d'Ivoire & NORD-EST & 85-89 & 117.48 & 224.25 & 57.76 & HT-Direct \\ 
  C\^{o}te d'Ivoire & NORD-EST & 85-89 & 167.35 & 115.35 & 236.99 & RW2 \\ 
  C\^{o}te d'Ivoire & NORD-EST & 90-94 & 182.93 & 269.00 & 119.89 & HT-Direct \\ 
  C\^{o}te d'Ivoire & NORD-EST & 90-94 & 153.76 & 118.15 & 197.33 & RW2 \\ 
  C\^{o}te d'Ivoire & NORD-EST & 95-99 & 118.32 & 157.40 & 87.93 & HT-Direct \\ 
  C\^{o}te d'Ivoire & NORD-EST & 95-99 & 151.15 & 124.25 & 182.14 & RW2 \\ 
  C\^{o}te d'Ivoire & NORD-EST & 00-04 & 145.72 & 199.23 & 104.71 & HT-Direct \\ 
  C\^{o}te d'Ivoire & NORD-EST & 00-04 & 145.94 & 120.96 & 175.15 & RW2 \\ 
  C\^{o}te d'Ivoire & NORD-EST & 05-09 & 129.59 & 191.22 & 85.72 & HT-Direct \\ 
  C\^{o}te d'Ivoire & NORD-EST & 05-09 & 127.42 & 99.84 & 160.99 & RW2 \\ 
  C\^{o}te d'Ivoire & NORD-EST & 10-14 & 116.29 & 179.33 & 73.42 & HT-Direct \\ 
  C\^{o}te d'Ivoire & NORD-EST & 10-14 & 112.60 & 78.96 & 158.41 & RW2 \\ 
  C\^{o}te d'Ivoire & NORD-EST & 15-19 & 100.67 & 37.67 & 246.80 & RW2 \\ 
  C\^{o}te d'Ivoire & NORD-OUEST & 80-84 & 224.51 & 321.45 & 150.33 & HT-Direct \\ 
  C\^{o}te d'Ivoire & NORD-OUEST & 80-84 & 226.71 & 158.30 & 311.51 & RW2 \\ 
  C\^{o}te d'Ivoire & NORD-OUEST & 85-89 & 168.60 & 235.20 & 117.95 & HT-Direct \\ 
  C\^{o}te d'Ivoire & NORD-OUEST & 85-89 & 195.99 & 153.51 & 246.58 & RW2 \\ 
  C\^{o}te d'Ivoire & NORD-OUEST & 90-94 & 144.36 & 194.88 & 105.22 & HT-Direct \\ 
  C\^{o}te d'Ivoire & NORD-OUEST & 90-94 & 200.70 & 167.86 & 238.36 & RW2 \\ 
  C\^{o}te d'Ivoire & NORD-OUEST & 95-99 & 226.21 & 272.04 & 186.12 & HT-Direct \\ 
  C\^{o}te d'Ivoire & NORD-OUEST & 95-99 & 215.87 & 190.17 & 244.69 & RW2 \\ 
  C\^{o}te d'Ivoire & NORD-OUEST & 00-04 & 223.28 & 254.77 & 194.67 & HT-Direct \\ 
  C\^{o}te d'Ivoire & NORD-OUEST & 00-04 & 220.26 & 198.75 & 243.46 & RW2 \\ 
  C\^{o}te d'Ivoire & NORD-OUEST & 05-09 & 184.22 & 219.80 & 153.28 & HT-Direct \\ 
  C\^{o}te d'Ivoire & NORD-OUEST & 05-09 & 200.06 & 174.52 & 228.26 & RW2 \\ 
  C\^{o}te d'Ivoire & NORD-OUEST & 10-14 & 183.99 & 260.85 & 125.91 & HT-Direct \\ 
  C\^{o}te d'Ivoire & NORD-OUEST & 10-14 & 184.08 & 142.92 & 232.85 & RW2 \\ 
  C\^{o}te d'Ivoire & NORD-OUEST & 15-19 & 171.47 & 69.78 & 366.07 & RW2 \\ 
  C\^{o}te d'Ivoire & OUEST & 80-84 & 135.65 & 295.00 & 55.59 & HT-Direct \\ 
  C\^{o}te d'Ivoire & OUEST & 80-84 & 319.33 & 201.30 & 448.68 & RW2 \\ 
  C\^{o}te d'Ivoire & OUEST & 85-89 & 226.42 & 334.19 & 145.79 & HT-Direct \\ 
  C\^{o}te d'Ivoire & OUEST & 85-89 & 253.74 & 191.66 & 323.72 & RW2 \\ 
  C\^{o}te d'Ivoire & OUEST & 90-94 & 207.04 & 257.76 & 164.09 & HT-Direct \\ 
  C\^{o}te d'Ivoire & OUEST & 90-94 & 231.59 & 195.31 & 272.27 & RW2 \\ 
  C\^{o}te d'Ivoire & OUEST & 95-99 & 220.44 & 256.47 & 188.19 & HT-Direct \\ 
  C\^{o}te d'Ivoire & OUEST & 95-99 & 217.31 & 192.95 & 244.81 & RW2 \\ 
  C\^{o}te d'Ivoire & OUEST & 00-04 & 193.03 & 224.33 & 165.17 & HT-Direct \\ 
  C\^{o}te d'Ivoire & OUEST & 00-04 & 190.30 & 170.03 & 212.67 & RW2 \\ 
  C\^{o}te d'Ivoire & OUEST & 05-09 & 123.79 & 153.40 & 99.22 & HT-Direct \\ 
  C\^{o}te d'Ivoire & OUEST & 05-09 & 146.19 & 123.99 & 171.27 & RW2 \\ 
  C\^{o}te d'Ivoire & OUEST & 10-14 & 132.87 & 208.99 & 81.61 & HT-Direct \\ 
  C\^{o}te d'Ivoire & OUEST & 10-14 & 113.42 & 83.08 & 151.76 & RW2 \\ 
  C\^{o}te d'Ivoire & OUEST & 15-19 & 88.99 & 32.88 & 219.30 & RW2 \\ 
  C\^{o}te d'Ivoire & SUD SANS ABIDJAN & 80-84 & 93.77 & 285.60 & 26.08 & HT-Direct \\ 
  C\^{o}te d'Ivoire & SUD SANS ABIDJAN & 80-84 & 192.62 & 112.24 & 309.48 & RW2 \\ 
  C\^{o}te d'Ivoire & SUD SANS ABIDJAN & 85-89 & 138.81 & 247.04 & 73.37 & HT-Direct \\ 
  C\^{o}te d'Ivoire & SUD SANS ABIDJAN & 85-89 & 152.46 & 105.12 & 215.84 & RW2 \\ 
  C\^{o}te d'Ivoire & SUD SANS ABIDJAN & 90-94 & 126.90 & 181.23 & 87.13 & HT-Direct \\ 
  C\^{o}te d'Ivoire & SUD SANS ABIDJAN & 90-94 & 142.81 & 110.56 & 182.44 & RW2 \\ 
  C\^{o}te d'Ivoire & SUD SANS ABIDJAN & 95-99 & 137.66 & 181.86 & 102.86 & HT-Direct \\ 
  C\^{o}te d'Ivoire & SUD SANS ABIDJAN & 95-99 & 141.15 & 117.83 & 168.01 & RW2 \\ 
  C\^{o}te d'Ivoire & SUD SANS ABIDJAN & 00-04 & 139.43 & 180.57 & 106.45 & HT-Direct \\ 
  C\^{o}te d'Ivoire & SUD SANS ABIDJAN & 00-04 & 134.73 & 115.82 & 156.35 & RW2 \\ 
  C\^{o}te d'Ivoire & SUD SANS ABIDJAN & 05-09 & 113.15 & 138.47 & 91.96 & HT-Direct \\ 
  C\^{o}te d'Ivoire & SUD SANS ABIDJAN & 05-09 & 114.67 & 96.08 & 136.44 & RW2 \\ 
  C\^{o}te d'Ivoire & SUD SANS ABIDJAN & 10-14 & 70.40 & 130.08 & 36.94 & HT-Direct \\ 
  C\^{o}te d'Ivoire & SUD SANS ABIDJAN & 10-14 & 98.13 & 70.65 & 134.37 & RW2 \\ 
  C\^{o}te d'Ivoire & SUD SANS ABIDJAN & 15-19 & 84.60 & 30.86 & 212.76 & RW2 \\ 
  C\^{o}te d'Ivoire & SUD-OUEST & 80-84 & 102.08 & 326.44 & 25.97 & HT-Direct \\ 
  C\^{o}te d'Ivoire & SUD-OUEST & 80-84 & 196.26 & 107.84 & 329.72 & RW2 \\ 
  C\^{o}te d'Ivoire & SUD-OUEST & 85-89 & 152.51 & 255.09 & 86.40 & HT-Direct \\ 
  C\^{o}te d'Ivoire & SUD-OUEST & 85-89 & 145.75 & 95.60 & 216.84 & RW2 \\ 
  C\^{o}te d'Ivoire & SUD-OUEST & 90-94 & 122.21 & 189.03 & 76.77 & HT-Direct \\ 
  C\^{o}te d'Ivoire & SUD-OUEST & 90-94 & 127.13 & 92.40 & 172.24 & RW2 \\ 
  C\^{o}te d'Ivoire & SUD-OUEST & 95-99 & 123.36 & 234.63 & 60.68 & HT-Direct \\ 
  C\^{o}te d'Ivoire & SUD-OUEST & 95-99 & 116.32 & 89.99 & 148.95 & RW2 \\ 
  C\^{o}te d'Ivoire & SUD-OUEST & 00-04 & 65.35 & 107.39 & 39.05 & HT-Direct \\ 
  C\^{o}te d'Ivoire & SUD-OUEST & 00-04 & 101.58 & 81.65 & 125.22 & RW2 \\ 
  C\^{o}te d'Ivoire & SUD-OUEST & 05-09 & 78.68 & 101.25 & 60.80 & HT-Direct \\ 
  C\^{o}te d'Ivoire & SUD-OUEST & 05-09 & 79.32 & 63.23 & 99.19 & RW2 \\ 
  C\^{o}te d'Ivoire & SUD-OUEST & 10-14 & 78.02 & 156.78 & 37.09 & HT-Direct \\ 
  C\^{o}te d'Ivoire & SUD-OUEST & 10-14 & 63.23 & 42.98 & 92.49 & RW2 \\ 
  C\^{o}te d'Ivoire & SUD-OUEST & 15-19 & 51.19 & 17.60 & 141.52 & RW2 \\ 
  C\^{o}te d'Ivoire & VILLE D'ABIDJAN & 80-84 & 125.46 & 286.42 & 48.77 & HT-Direct \\ 
  C\^{o}te d'Ivoire & VILLE D'ABIDJAN & 80-84 & 151.11 & 82.69 & 268.15 & RW2 \\ 
  C\^{o}te d'Ivoire & VILLE D'ABIDJAN & 85-89 & 133.37 & 225.15 & 75.36 & HT-Direct \\ 
  C\^{o}te d'Ivoire & VILLE D'ABIDJAN & 85-89 & 116.44 & 78.37 & 170.86 & RW2 \\ 
  C\^{o}te d'Ivoire & VILLE D'ABIDJAN & 90-94 & 88.03 & 135.03 & 56.33 & HT-Direct \\ 
  C\^{o}te d'Ivoire & VILLE D'ABIDJAN & 90-94 & 107.24 & 81.50 & 140.09 & RW2 \\ 
  C\^{o}te d'Ivoire & VILLE D'ABIDJAN & 95-99 & 103.45 & 138.84 & 76.28 & HT-Direct \\ 
  C\^{o}te d'Ivoire & VILLE D'ABIDJAN & 95-99 & 106.28 & 85.81 & 130.26 & RW2 \\ 
  C\^{o}te d'Ivoire & VILLE D'ABIDJAN & 00-04 & 83.74 & 117.33 & 59.13 & HT-Direct \\ 
  C\^{o}te d'Ivoire & VILLE D'ABIDJAN & 00-04 & 104.21 & 85.62 & 125.35 & RW2 \\ 
  C\^{o}te d'Ivoire & VILLE D'ABIDJAN & 05-09 & 99.29 & 128.09 & 76.39 & HT-Direct \\ 
  C\^{o}te d'Ivoire & VILLE D'ABIDJAN & 05-09 & 93.50 & 75.83 & 115.18 & RW2 \\ 
  C\^{o}te d'Ivoire & VILLE D'ABIDJAN & 10-14 & 80.05 & 132.59 & 47.20 & HT-Direct \\ 
  C\^{o}te d'Ivoire & VILLE D'ABIDJAN & 10-14 & 85.32 & 59.22 & 122.63 & RW2 \\ 
  C\^{o}te d'Ivoire & VILLE D'ABIDJAN & 15-19 & 78.87 & 27.29 & 211.63 & RW2 \\ 
  DRC & ALL & 80-84 & 177.53 & 170.43 & 184.58 & IHME \\ 
  DRC & ALL & 80-84 & 206.56 & 169.75 & 249.04 & RW2 \\ 
  DRC & ALL & 80-84 & 205.54 & 194.81 & 217.49 & UN \\ 
  DRC & ALL & 85-89 & 161.02 & 156.18 & 166.64 & IHME \\ 
  DRC & ALL & 85-89 & 190.70 & 167.27 & 216.23 & RW2 \\ 
  DRC & ALL & 85-89 & 192.63 & 184.31 & 202.27 & UN \\ 
  DRC & ALL & 90-94 & 165.89 & 160.54 & 171.62 & IHME \\ 
  DRC & ALL & 90-94 & 183.69 & 167.32 & 201.32 & RW2 \\ 
  DRC & ALL & 90-94 & 182.13 & 174.46 & 190.42 & UN \\ 
  DRC & ALL & 95-99 & 162.41 & 157.27 & 167.95 & IHME \\ 
  DRC & ALL & 95-99 & 171.81 & 159.01 & 185.31 & RW2 \\ 
  DRC & ALL & 95-99 & 171.66 & 164.49 & 179.34 & UN \\ 
  DRC & ALL & 00-04 & 142.43 & 137.67 & 147.57 & IHME \\ 
  DRC & ALL & 00-04 & 151.93 & 142.57 & 161.84 & RW2 \\ 
  DRC & ALL & 00-04 & 152.22 & 146.01 & 158.37 & UN \\ 
  DRC & ALL & 05-09 & 120.74 & 116.11 & 125.82 & IHME \\ 
  DRC & ALL & 05-09 & 129.30 & 120.20 & 138.96 & RW2 \\ 
  DRC & ALL & 05-09 & 129.24 & 122.26 & 136.71 & UN \\ 
  DRC & ALL & 10-14 & 99.73 & 94.09 & 106.03 & IHME \\ 
  DRC & ALL & 10-14 & 107.92 & 98.21 & 118.39 & RW2 \\ 
  DRC & ALL & 10-14 & 107.92 & 97.62 & 119.22 & UN \\ 
  DRC & BANDUNDU & 80-84 & 151.31 & 208.25 & 107.82 & HT-Direct \\ 
  DRC & BANDUNDU & 80-84 & 238.22 & 178.63 & 309.96 & RW2 \\ 
  DRC & BANDUNDU & 85-89 & 153.26 & 238.72 & 94.59 & HT-Direct \\ 
  DRC & BANDUNDU & 85-89 & 207.61 & 166.61 & 256.08 & RW2 \\ 
  DRC & BANDUNDU & 90-94 & 204.52 & 253.73 & 162.77 & HT-Direct \\ 
  DRC & BANDUNDU & 90-94 & 177.66 & 151.75 & 207.07 & RW2 \\ 
  DRC & BANDUNDU & 95-99 & 162.80 & 210.22 & 124.39 & HT-Direct \\ 
  DRC & BANDUNDU & 95-99 & 161.33 & 141.94 & 182.78 & RW2 \\ 
  DRC & BANDUNDU & 00-04 & 117.68 & 141.98 & 97.07 & HT-Direct \\ 
  DRC & BANDUNDU & 00-04 & 133.39 & 118.42 & 149.94 & RW2 \\ 
  DRC & BANDUNDU & 05-09 & 100.27 & 122.57 & 81.65 & HT-Direct \\ 
  DRC & BANDUNDU & 05-09 & 101.42 & 87.91 & 116.68 & RW2 \\ 
  DRC & BANDUNDU & 10-14 & 72.77 & 95.81 & 54.93 & HT-Direct \\ 
  DRC & BANDUNDU & 10-14 & 79.50 & 64.53 & 97.65 & RW2 \\ 
  DRC & BANDUNDU & 15-19 & 63.28 & 26.72 & 141.66 & RW2 \\ 
  DRC & BAS-CONGO & 80-84 & 183.53 & 273.26 & 118.47 & HT-Direct \\ 
  DRC & BAS-CONGO & 80-84 & 258.13 & 184.46 & 345.98 & RW2 \\ 
  DRC & BAS-CONGO & 85-89 & 186.01 & 251.34 & 134.61 & HT-Direct \\ 
  DRC & BAS-CONGO & 85-89 & 235.01 & 187.22 & 290.60 & RW2 \\ 
  DRC & BAS-CONGO & 90-94 & 226.00 & 295.06 & 169.23 & HT-Direct \\ 
  DRC & BAS-CONGO & 90-94 & 210.71 & 179.20 & 246.14 & RW2 \\ 
  DRC & BAS-CONGO & 95-99 & 210.85 & 253.40 & 173.78 & HT-Direct \\ 
  DRC & BAS-CONGO & 95-99 & 200.69 & 178.06 & 225.47 & RW2 \\ 
  DRC & BAS-CONGO & 00-04 & 164.02 & 190.52 & 140.56 & HT-Direct \\ 
  DRC & BAS-CONGO & 00-04 & 175.05 & 156.16 & 195.41 & RW2 \\ 
  DRC & BAS-CONGO & 05-09 & 131.12 & 168.66 & 100.92 & HT-Direct \\ 
  DRC & BAS-CONGO & 05-09 & 141.51 & 120.41 & 165.66 & RW2 \\ 
  DRC & BAS-CONGO & 10-14 & 117.10 & 163.64 & 82.50 & HT-Direct \\ 
  DRC & BAS-CONGO & 10-14 & 118.36 & 92.14 & 151.30 & RW2 \\ 
  DRC & BAS-CONGO & 15-19 & 100.38 & 42.18 & 222.50 & RW2 \\ 
  DRC & EQUATEUR & 80-84 & 141.17 & 208.45 & 93.05 & HT-Direct \\ 
  DRC & EQUATEUR & 80-84 & 190.05 & 139.32 & 253.49 & RW2 \\ 
  DRC & EQUATEUR & 85-89 & 160.67 & 206.67 & 123.31 & HT-Direct \\ 
  DRC & EQUATEUR & 85-89 & 184.06 & 147.56 & 226.12 & RW2 \\ 
  DRC & EQUATEUR & 90-94 & 217.44 & 293.70 & 156.59 & HT-Direct \\ 
  DRC & EQUATEUR & 90-94 & 175.86 & 148.75 & 206.62 & RW2 \\ 
  DRC & EQUATEUR & 95-99 & 166.05 & 217.22 & 125.01 & HT-Direct \\ 
  DRC & EQUATEUR & 95-99 & 180.37 & 158.03 & 204.95 & RW2 \\ 
  DRC & EQUATEUR & 00-04 & 148.06 & 178.74 & 121.86 & HT-Direct \\ 
  DRC & EQUATEUR & 00-04 & 170.23 & 151.24 & 190.91 & RW2 \\ 
  DRC & EQUATEUR & 05-09 & 141.63 & 172.73 & 115.36 & HT-Direct \\ 
  DRC & EQUATEUR & 05-09 & 149.23 & 130.25 & 170.35 & RW2 \\ 
  DRC & EQUATEUR & 10-14 & 141.25 & 178.11 & 110.99 & HT-Direct \\ 
  DRC & EQUATEUR & 10-14 & 135.07 & 111.87 & 162.41 & RW2 \\ 
  DRC & EQUATEUR & 15-19 & 123.79 & 55.20 & 257.08 & RW2 \\ 
  DRC & KASAI-OCCIDENTAL & 80-84 & 251.38 & 359.52 & 167.27 & HT-Direct \\ 
  DRC & KASAI-OCCIDENTAL & 80-84 & 258.25 & 193.80 & 336.20 & RW2 \\ 
  DRC & KASAI-OCCIDENTAL & 85-89 & 197.53 & 254.22 & 150.93 & HT-Direct \\ 
  DRC & KASAI-OCCIDENTAL & 85-89 & 236.00 & 194.24 & 284.10 & RW2 \\ 
  DRC & KASAI-OCCIDENTAL & 90-94 & 189.70 & 237.17 & 149.85 & HT-Direct \\ 
  DRC & KASAI-OCCIDENTAL & 90-94 & 212.96 & 186.35 & 242.36 & RW2 \\ 
  DRC & KASAI-OCCIDENTAL & 95-99 & 210.95 & 239.83 & 184.69 & HT-Direct \\ 
  DRC & KASAI-OCCIDENTAL & 95-99 & 205.70 & 187.10 & 225.68 & RW2 \\ 
  DRC & KASAI-OCCIDENTAL & 00-04 & 177.00 & 210.87 & 147.55 & HT-Direct \\ 
  DRC & KASAI-OCCIDENTAL & 00-04 & 181.81 & 163.43 & 201.93 & RW2 \\ 
  DRC & KASAI-OCCIDENTAL & 05-09 & 145.26 & 176.94 & 118.44 & HT-Direct \\ 
  DRC & KASAI-OCCIDENTAL & 05-09 & 147.30 & 126.76 & 170.26 & RW2 \\ 
  DRC & KASAI-OCCIDENTAL & 10-14 & 105.02 & 152.58 & 71.04 & HT-Direct \\ 
  DRC & KASAI-OCCIDENTAL & 10-14 & 121.73 & 96.17 & 152.45 & RW2 \\ 
  DRC & KASAI-OCCIDENTAL & 15-19 & 101.96 & 43.76 & 218.88 & RW2 \\ 
  DRC & KASAI-ORIENTAL & 80-84 & 129.05 & 195.97 & 82.63 & HT-Direct \\ 
  DRC & KASAI-ORIENTAL & 80-84 & 204.83 & 155.91 & 264.46 & RW2 \\ 
  DRC & KASAI-ORIENTAL & 85-89 & 194.34 & 237.70 & 157.25 & HT-Direct \\ 
  DRC & KASAI-ORIENTAL & 85-89 & 192.61 & 161.34 & 228.02 & RW2 \\ 
  DRC & KASAI-ORIENTAL & 90-94 & 174.12 & 209.68 & 143.49 & HT-Direct \\ 
  DRC & KASAI-ORIENTAL & 90-94 & 178.70 & 158.42 & 200.88 & RW2 \\ 
  DRC & KASAI-ORIENTAL & 95-99 & 170.51 & 192.33 & 150.70 & HT-Direct \\ 
  DRC & KASAI-ORIENTAL & 95-99 & 178.31 & 163.47 & 194.01 & RW2 \\ 
  DRC & KASAI-ORIENTAL & 00-04 & 161.36 & 186.33 & 139.17 & HT-Direct \\ 
  DRC & KASAI-ORIENTAL & 00-04 & 163.00 & 148.17 & 179.05 & RW2 \\ 
  DRC & KASAI-ORIENTAL & 05-09 & 131.50 & 158.56 & 108.46 & HT-Direct \\ 
  DRC & KASAI-ORIENTAL & 05-09 & 136.37 & 119.80 & 155.07 & RW2 \\ 
  DRC & KASAI-ORIENTAL & 10-14 & 111.25 & 145.55 & 84.23 & HT-Direct \\ 
  DRC & KASAI-ORIENTAL & 10-14 & 116.76 & 96.26 & 141.12 & RW2 \\ 
  DRC & KASAI-ORIENTAL & 15-19 & 101.05 & 44.28 & 216.24 & RW2 \\ 
  DRC & KATANGA & 80-84 & 144.81 & 215.69 & 94.42 & HT-Direct \\ 
  DRC & KATANGA & 80-84 & 209.83 & 157.40 & 272.74 & RW2 \\ 
  DRC & KATANGA & 85-89 & 180.11 & 231.00 & 138.41 & HT-Direct \\ 
  DRC & KATANGA & 85-89 & 200.51 & 165.63 & 240.44 & RW2 \\ 
  DRC & KATANGA & 90-94 & 193.08 & 235.59 & 156.68 & HT-Direct \\ 
  DRC & KATANGA & 90-94 & 189.09 & 165.59 & 214.82 & RW2 \\ 
  DRC & KATANGA & 95-99 & 184.51 & 217.86 & 155.26 & HT-Direct \\ 
  DRC & KATANGA & 95-99 & 190.95 & 172.51 & 210.88 & RW2 \\ 
  DRC & KATANGA & 00-04 & 173.01 & 199.14 & 149.67 & HT-Direct \\ 
  DRC & KATANGA & 00-04 & 175.55 & 159.87 & 192.77 & RW2 \\ 
  DRC & KATANGA & 05-09 & 130.85 & 154.29 & 110.51 & HT-Direct \\ 
  DRC & KATANGA & 05-09 & 147.41 & 132.73 & 163.24 & RW2 \\ 
  DRC & KATANGA & 10-14 & 124.20 & 140.95 & 109.19 & HT-Direct \\ 
  DRC & KATANGA & 10-14 & 126.71 & 112.48 & 142.38 & RW2 \\ 
  DRC & KATANGA & 15-19 & 109.94 & 50.36 & 225.34 & RW2 \\ 
  DRC & KINSHASA & 80-84 & 147.77 & 205.11 & 104.36 & HT-Direct \\ 
  DRC & KINSHASA & 80-84 & 129.09 & 94.08 & 176.80 & RW2 \\ 
  DRC & KINSHASA & 85-89 & 87.95 & 118.44 & 64.72 & HT-Direct \\ 
  DRC & KINSHASA & 85-89 & 119.38 & 96.71 & 147.12 & RW2 \\ 
  DRC & KINSHASA & 90-94 & 105.20 & 129.03 & 85.34 & HT-Direct \\ 
  DRC & KINSHASA & 90-94 & 110.22 & 95.95 & 126.42 & RW2 \\ 
  DRC & KINSHASA & 95-99 & 106.92 & 125.85 & 90.55 & HT-Direct \\ 
  DRC & KINSHASA & 95-99 & 111.32 & 99.79 & 123.89 & RW2 \\ 
  DRC & KINSHASA & 00-04 & 103.03 & 118.45 & 89.40 & HT-Direct \\ 
  DRC & KINSHASA & 00-04 & 104.33 & 93.79 & 115.66 & RW2 \\ 
  DRC & KINSHASA & 05-09 & 78.25 & 98.19 & 62.07 & HT-Direct \\ 
  DRC & KINSHASA & 05-09 & 91.42 & 79.68 & 104.73 & RW2 \\ 
  DRC & KINSHASA & 10-14 & 86.87 & 109.00 & 68.88 & HT-Direct \\ 
  DRC & KINSHASA & 10-14 & 84.18 & 69.27 & 102.46 & RW2 \\ 
  DRC & KINSHASA & 15-19 & 79.04 & 34.00 & 175.08 & RW2 \\ 
  DRC & MANIEMA & 80-84 & 131.24 & 212.84 & 77.83 & HT-Direct \\ 
  DRC & MANIEMA & 80-84 & 187.13 & 134.22 & 251.84 & RW2 \\ 
  DRC & MANIEMA & 85-89 & 163.64 & 228.40 & 114.52 & HT-Direct \\ 
  DRC & MANIEMA & 85-89 & 184.65 & 147.46 & 227.57 & RW2 \\ 
  DRC & MANIEMA & 90-94 & 140.57 & 174.98 & 112.01 & HT-Direct \\ 
  DRC & MANIEMA & 90-94 & 180.41 & 155.86 & 207.78 & RW2 \\ 
  DRC & MANIEMA & 95-99 & 223.91 & 262.36 & 189.64 & HT-Direct \\ 
  DRC & MANIEMA & 95-99 & 188.86 & 169.30 & 210.80 & RW2 \\ 
  DRC & MANIEMA & 00-04 & 196.16 & 249.20 & 152.13 & HT-Direct \\ 
  DRC & MANIEMA & 00-04 & 177.16 & 157.92 & 199.00 & RW2 \\ 
  DRC & MANIEMA & 05-09 & 135.61 & 162.29 & 112.73 & HT-Direct \\ 
  DRC & MANIEMA & 05-09 & 150.01 & 131.02 & 171.13 & RW2 \\ 
  DRC & MANIEMA & 10-14 & 110.82 & 144.08 & 84.49 & HT-Direct \\ 
  DRC & MANIEMA & 10-14 & 128.89 & 105.46 & 156.15 & RW2 \\ 
  DRC & MANIEMA & 15-19 & 111.89 & 48.97 & 235.38 & RW2 \\ 
  DRC & NORD-KIVU & 80-84 & 239.97 & 330.11 & 168.25 & HT-Direct \\ 
  DRC & NORD-KIVU & 80-84 & 266.50 & 194.62 & 353.65 & RW2 \\ 
  DRC & NORD-KIVU & 85-89 & 127.40 & 189.88 & 83.36 & HT-Direct \\ 
  DRC & NORD-KIVU & 85-89 & 217.27 & 170.83 & 272.13 & RW2 \\ 
  DRC & NORD-KIVU & 90-94 & 186.16 & 246.78 & 137.70 & HT-Direct \\ 
  DRC & NORD-KIVU & 90-94 & 173.61 & 145.03 & 206.27 & RW2 \\ 
  DRC & NORD-KIVU & 95-99 & 144.53 & 183.51 & 112.68 & HT-Direct \\ 
  DRC & NORD-KIVU & 95-99 & 147.51 & 127.73 & 169.66 & RW2 \\ 
  DRC & NORD-KIVU & 00-04 & 90.39 & 124.57 & 64.90 & HT-Direct \\ 
  DRC & NORD-KIVU & 00-04 & 112.25 & 96.80 & 130.03 & RW2 \\ 
  DRC & NORD-KIVU & 05-09 & 80.82 & 99.85 & 65.15 & HT-Direct \\ 
  DRC & NORD-KIVU & 05-09 & 76.40 & 63.83 & 91.27 & RW2 \\ 
  DRC & NORD-KIVU & 10-14 & 36.08 & 59.54 & 21.65 & HT-Direct \\ 
  DRC & NORD-KIVU & 10-14 & 52.47 & 39.46 & 69.28 & RW2 \\ 
  DRC & NORD-KIVU & 15-19 & 36.14 & 14.40 & 87.93 & RW2 \\ 
  DRC & ORIENTALE & 80-84 & 193.65 & 312.05 & 112.80 & HT-Direct \\ 
  DRC & ORIENTALE & 80-84 & 227.06 & 163.58 & 306.55 & RW2 \\ 
  DRC & ORIENTALE & 85-89 & 189.78 & 250.12 & 141.26 & HT-Direct \\ 
  DRC & ORIENTALE & 85-89 & 208.56 & 166.13 & 259.10 & RW2 \\ 
  DRC & ORIENTALE & 90-94 & 162.38 & 219.21 & 118.06 & HT-Direct \\ 
  DRC & ORIENTALE & 90-94 & 189.17 & 160.27 & 221.73 & RW2 \\ 
  DRC & ORIENTALE & 95-99 & 192.78 & 234.29 & 157.11 & HT-Direct \\ 
  DRC & ORIENTALE & 95-99 & 184.36 & 163.06 & 207.83 & RW2 \\ 
  DRC & ORIENTALE & 00-04 & 154.71 & 187.77 & 126.56 & HT-Direct \\ 
  DRC & ORIENTALE & 00-04 & 163.19 & 145.35 & 182.85 & RW2 \\ 
  DRC & ORIENTALE & 05-09 & 124.07 & 150.55 & 101.69 & HT-Direct \\ 
  DRC & ORIENTALE & 05-09 & 131.66 & 114.85 & 150.63 & RW2 \\ 
  DRC & ORIENTALE & 10-14 & 105.48 & 135.52 & 81.47 & HT-Direct \\ 
  DRC & ORIENTALE & 10-14 & 108.41 & 88.78 & 131.56 & RW2 \\ 
  DRC & ORIENTALE & 15-19 & 90.03 & 39.07 & 193.50 & RW2 \\ 
  DRC & SUD-KIVU & 80-84 & 252.32 & 331.34 & 186.88 & HT-Direct \\ 
  DRC & SUD-KIVU & 80-84 & 267.83 & 200.97 & 347.48 & RW2 \\ 
  DRC & SUD-KIVU & 85-89 & 137.93 & 208.20 & 88.72 & HT-Direct \\ 
  DRC & SUD-KIVU & 85-89 & 247.32 & 198.86 & 302.94 & RW2 \\ 
  DRC & SUD-KIVU & 90-94 & 226.61 & 290.05 & 173.66 & HT-Direct \\ 
  DRC & SUD-KIVU & 90-94 & 226.50 & 193.71 & 263.07 & RW2 \\ 
  DRC & SUD-KIVU & 95-99 & 242.19 & 295.45 & 195.86 & HT-Direct \\ 
  DRC & SUD-KIVU & 95-99 & 221.20 & 195.76 & 249.20 & RW2 \\ 
  DRC & SUD-KIVU & 00-04 & 178.09 & 217.00 & 144.87 & HT-Direct \\ 
  DRC & SUD-KIVU & 00-04 & 195.81 & 173.74 & 220.14 & RW2 \\ 
  DRC & SUD-KIVU & 05-09 & 145.94 & 181.41 & 116.42 & HT-Direct \\ 
  DRC & SUD-KIVU & 05-09 & 157.36 & 136.20 & 181.48 & RW2 \\ 
  DRC & SUD-KIVU & 10-14 & 129.41 & 166.65 & 99.49 & HT-Direct \\ 
  DRC & SUD-KIVU & 10-14 & 128.61 & 104.39 & 157.25 & RW2 \\ 
  DRC & SUD-KIVU & 15-19 & 105.81 & 46.06 & 225.51 & RW2 \\ 
  Egypt & ALL & 80-84 & 141.99 & 138.72 & 145.18 & IHME \\ 
  Egypt & ALL & 80-84 & 147.64 & 143.17 & 152.23 & RW2 \\ 
  Egypt & ALL & 80-84 & 147.62 & 144.54 & 150.81 & UN \\ 
  Egypt & ALL & 85-89 & 97.53 & 95.24 & 99.97 & IHME \\ 
  Egypt & ALL & 85-89 & 103.18 & 99.96 & 106.47 & RW2 \\ 
  Egypt & ALL & 85-89 & 103.23 & 101.14 & 105.40 & UN \\ 
  Egypt & ALL & 90-94 & 74.53 & 72.62 & 76.55 & IHME \\ 
  Egypt & ALL & 90-94 & 77.72 & 75.04 & 80.49 & RW2 \\ 
  Egypt & ALL & 90-94 & 77.64 & 76.01 & 79.24 & UN \\ 
  Egypt & ALL & 95-99 & 52.51 & 50.86 & 54.20 & IHME \\ 
  Egypt & ALL & 95-99 & 56.79 & 54.17 & 59.51 & RW2 \\ 
  Egypt & ALL & 95-99 & 56.85 & 55.60 & 58.34 & UN \\ 
  Egypt & ALL & 00-04 & 38.05 & 36.67 & 39.38 & IHME \\ 
  Egypt & ALL & 00-04 & 41.53 & 38.97 & 44.26 & RW2 \\ 
  Egypt & ALL & 00-04 & 41.53 & 40.30 & 42.88 & UN \\ 
  Egypt & ALL & 05-09 & 31.32 & 29.78 & 32.92 & IHME \\ 
  Egypt & ALL & 05-09 & 32.82 & 29.91 & 36.01 & RW2 \\ 
  Egypt & ALL & 05-09 & 32.82 & 31.32 & 34.10 & UN \\ 
  Egypt & ALL & 10-14 & 25.79 & 24.14 & 27.47 & IHME \\ 
  Egypt & ALL & 10-14 & 26.43 & 9.81 & 68.75 & RW2 \\ 
  Egypt & ALL & 10-14 & 26.84 & 24.98 & 28.86 & UN \\ 
  Egypt & FRONTIER GOVERNORATES & 80-84 & 98.02 & 117.07 & 81.78 & HT-Direct \\ 
  Egypt & FRONTIER GOVERNORATES & 80-84 & 99.33 & 85.54 & 115.16 & RW2 \\ 
  Egypt & FRONTIER GOVERNORATES & 85-89 & 70.59 & 82.61 & 60.20 & HT-Direct \\ 
  Egypt & FRONTIER GOVERNORATES & 85-89 & 72.36 & 65.10 & 80.44 & RW2 \\ 
  Egypt & FRONTIER GOVERNORATES & 90-94 & 53.94 & 64.01 & 45.38 & HT-Direct \\ 
  Egypt & FRONTIER GOVERNORATES & 90-94 & 56.98 & 51.09 & 63.27 & RW2 \\ 
  Egypt & FRONTIER GOVERNORATES & 95-99 & 44.93 & 55.25 & 36.47 & HT-Direct \\ 
  Egypt & FRONTIER GOVERNORATES & 95-99 & 45.38 & 39.93 & 51.45 & RW2 \\ 
  Egypt & FRONTIER GOVERNORATES & 00-04 & 32.98 & 42.89 & 25.30 & HT-Direct \\ 
  Egypt & FRONTIER GOVERNORATES & 00-04 & 36.73 & 31.05 & 43.51 & RW2 \\ 
  Egypt & FRONTIER GOVERNORATES & 05-09 & 35.12 & 49.99 & 24.56 & HT-Direct \\ 
  Egypt & FRONTIER GOVERNORATES & 05-09 & 33.46 & 25.97 & 43.34 & RW2 \\ 
  Egypt & FRONTIER GOVERNORATES & 10-14 & 31.67 & 12.53 & 79.63 & RW2 \\ 
  Egypt & FRONTIER GOVERNORATES & 15-19 & 29.83 & 3.12 & 236.19 & RW2 \\ 
  Egypt & LOWER EGYPT & 80-84 & 122.40 & 128.61 & 116.45 & HT-Direct \\ 
  Egypt & LOWER EGYPT & 80-84 & 125.83 & 119.93 & 132.05 & RW2 \\ 
  Egypt & LOWER EGYPT & 85-89 & 83.34 & 87.71 & 79.18 & HT-Direct \\ 
  Egypt & LOWER EGYPT & 85-89 & 86.65 & 82.87 & 90.64 & RW2 \\ 
  Egypt & LOWER EGYPT & 90-94 & 63.23 & 67.30 & 59.40 & HT-Direct \\ 
  Egypt & LOWER EGYPT & 90-94 & 63.14 & 60.01 & 66.40 & RW2 \\ 
  Egypt & LOWER EGYPT & 95-99 & 42.66 & 46.44 & 39.18 & HT-Direct \\ 
  Egypt & LOWER EGYPT & 95-99 & 45.84 & 42.86 & 48.97 & RW2 \\ 
  Egypt & LOWER EGYPT & 00-04 & 32.65 & 36.53 & 29.17 & HT-Direct \\ 
  Egypt & LOWER EGYPT & 00-04 & 33.76 & 30.90 & 36.90 & RW2 \\ 
  Egypt & LOWER EGYPT & 05-09 & 26.74 & 31.53 & 22.66 & HT-Direct \\ 
  Egypt & LOWER EGYPT & 05-09 & 27.74 & 24.13 & 31.92 & RW2 \\ 
  Egypt & LOWER EGYPT & 10-14 & 23.57 & 9.64 & 56.81 & RW2 \\ 
  Egypt & LOWER EGYPT & 15-19 & 20.07 & 2.16 & 167.70 & RW2 \\ 
  Egypt & UPPER EGYPT & 80-84 & 195.32 & 202.99 & 187.86 & HT-Direct \\ 
  Egypt & UPPER EGYPT & 80-84 & 198.20 & 190.74 & 205.86 & RW2 \\ 
  Egypt & UPPER EGYPT & 85-89 & 127.49 & 132.51 & 122.64 & HT-Direct \\ 
  Egypt & UPPER EGYPT & 85-89 & 136.80 & 131.86 & 141.74 & RW2 \\ 
  Egypt & UPPER EGYPT & 90-94 & 103.65 & 108.25 & 99.23 & HT-Direct \\ 
  Egypt & UPPER EGYPT & 90-94 & 102.00 & 98.14 & 106.06 & RW2 \\ 
  Egypt & UPPER EGYPT & 95-99 & 71.69 & 75.77 & 67.82 & HT-Direct \\ 
  Egypt & UPPER EGYPT & 95-99 & 74.22 & 70.57 & 78.09 & RW2 \\ 
  Egypt & UPPER EGYPT & 00-04 & 48.99 & 52.85 & 45.40 & HT-Direct \\ 
  Egypt & UPPER EGYPT & 00-04 & 52.71 & 49.20 & 56.46 & RW2 \\ 
  Egypt & UPPER EGYPT & 05-09 & 39.49 & 44.37 & 35.14 & HT-Direct \\ 
  Egypt & UPPER EGYPT & 05-09 & 41.13 & 36.90 & 45.82 & RW2 \\ 
  Egypt & UPPER EGYPT & 10-14 & 33.01 & 13.79 & 77.36 & RW2 \\ 
  Egypt & UPPER EGYPT & 15-19 & 26.46 & 2.81 & 208.44 & RW2 \\ 
  Egypt & URBAN GOVERNORATES & 80-84 & 86.39 & 94.19 & 79.19 & HT-Direct \\ 
  Egypt & URBAN GOVERNORATES & 80-84 & 90.03 & 83.04 & 97.51 & RW2 \\ 
  Egypt & URBAN GOVERNORATES & 85-89 & 66.50 & 72.91 & 60.62 & HT-Direct \\ 
  Egypt & URBAN GOVERNORATES & 85-89 & 65.27 & 61.02 & 69.98 & RW2 \\ 
  Egypt & URBAN GOVERNORATES & 90-94 & 46.74 & 52.60 & 41.50 & HT-Direct \\ 
  Egypt & URBAN GOVERNORATES & 90-94 & 50.40 & 46.41 & 54.53 & RW2 \\ 
  Egypt & URBAN GOVERNORATES & 95-99 & 37.91 & 44.53 & 32.24 & HT-Direct \\ 
  Egypt & URBAN GOVERNORATES & 95-99 & 39.56 & 35.63 & 43.72 & RW2 \\ 
  Egypt & URBAN GOVERNORATES & 00-04 & 29.24 & 35.49 & 24.07 & HT-Direct \\ 
  Egypt & URBAN GOVERNORATES & 00-04 & 31.84 & 27.90 & 36.38 & RW2 \\ 
  Egypt & URBAN GOVERNORATES & 05-09 & 30.64 & 40.19 & 23.30 & HT-Direct \\ 
  Egypt & URBAN GOVERNORATES & 05-09 & 28.99 & 23.64 & 35.86 & RW2 \\ 
  Egypt & URBAN GOVERNORATES & 10-14 & 27.49 & 10.98 & 67.89 & RW2 \\ 
  Egypt & URBAN GOVERNORATES & 15-19 & 26.12 & 2.75 & 213.27 & RW2 \\ 
  Ethiopia & ADDIS ABABA & 80-84 & 90.19 & 117.00 & 69.04 & HT-Direct \\ 
  Ethiopia & ADDIS ABABA & 80-84 & 89.89 & 72.59 & 110.43 & RW2 \\ 
  Ethiopia & ADDIS ABABA & 85-89 & 85.55 & 107.51 & 67.73 & HT-Direct \\ 
  Ethiopia & ADDIS ABABA & 85-89 & 89.29 & 77.43 & 102.73 & RW2 \\ 
  Ethiopia & ADDIS ABABA & 90-94 & 85.66 & 105.45 & 69.30 & HT-Direct \\ 
  Ethiopia & ADDIS ABABA & 90-94 & 90.05 & 79.75 & 101.70 & RW2 \\ 
  Ethiopia & ADDIS ABABA & 95-99 & 93.37 & 111.53 & 77.90 & HT-Direct \\ 
  Ethiopia & ADDIS ABABA & 95-99 & 79.49 & 70.09 & 90.62 & RW2 \\ 
  Ethiopia & ADDIS ABABA & 00-04 & 52.84 & 72.01 & 38.56 & HT-Direct \\ 
  Ethiopia & ADDIS ABABA & 00-04 & 64.17 & 54.90 & 74.96 & RW2 \\ 
  Ethiopia & ADDIS ABABA & 05-09 & 46.19 & 65.19 & 32.53 & HT-Direct \\ 
  Ethiopia & ADDIS ABABA & 05-09 & 47.08 & 37.85 & 58.23 & RW2 \\ 
  Ethiopia & ADDIS ABABA & 10-14 & 35.14 & 63.26 & 19.27 & HT-Direct \\ 
  Ethiopia & ADDIS ABABA & 10-14 & 34.81 & 24.26 & 48.59 & RW2 \\ 
  Ethiopia & ADDIS ABABA & 15-19 & 25.95 & 9.91 & 64.50 & RW2 \\ 
  Ethiopia & AFFAR & 80-84 & 328.98 & 380.06 & 281.64 & HT-Direct \\ 
  Ethiopia & AFFAR & 80-84 & 318.85 & 284.16 & 357.80 & RW2 \\ 
  Ethiopia & AFFAR & 85-89 & 292.96 & 327.95 & 260.26 & HT-Direct \\ 
  Ethiopia & AFFAR & 85-89 & 275.80 & 254.91 & 297.97 & RW2 \\ 
  Ethiopia & AFFAR & 90-94 & 240.69 & 271.11 & 212.68 & HT-Direct \\ 
  Ethiopia & AFFAR & 90-94 & 240.11 & 223.02 & 257.30 & RW2 \\ 
  Ethiopia & AFFAR & 95-99 & 179.55 & 202.24 & 158.90 & HT-Direct \\ 
  Ethiopia & AFFAR & 95-99 & 193.45 & 178.22 & 208.24 & RW2 \\ 
  Ethiopia & AFFAR & 00-04 & 154.54 & 178.86 & 132.99 & HT-Direct \\ 
  Ethiopia & AFFAR & 00-04 & 155.07 & 142.26 & 168.21 & RW2 \\ 
  Ethiopia & AFFAR & 05-09 & 122.21 & 141.98 & 104.86 & HT-Direct \\ 
  Ethiopia & AFFAR & 05-09 & 117.85 & 106.35 & 130.61 & RW2 \\ 
  Ethiopia & AFFAR & 10-14 & 114.45 & 145.45 & 89.36 & HT-Direct \\ 
  Ethiopia & AFFAR & 10-14 & 89.92 & 76.67 & 106.47 & RW2 \\ 
  Ethiopia & AFFAR & 15-19 & 68.89 & 31.38 & 146.58 & RW2 \\ 
  Ethiopia & ALL & 80-84 & 254.72 & 235.08 & 275.78 & IHME \\ 
  Ethiopia & ALL & 80-84 & 235.97 & 222.85 & 249.61 & RW2 \\ 
  Ethiopia & ALL & 80-84 & 235.81 & 227.46 & 244.60 & UN \\ 
  Ethiopia & ALL & 85-89 & 211.56 & 208.67 & 214.26 & IHME \\ 
  Ethiopia & ALL & 85-89 & 215.41 & 206.20 & 224.88 & RW2 \\ 
  Ethiopia & ALL & 85-89 & 215.75 & 208.65 & 222.99 & UN \\ 
  Ethiopia & ALL & 90-94 & 191.46 & 188.83 & 193.83 & IHME \\ 
  Ethiopia & ALL & 90-94 & 194.59 & 187.18 & 202.25 & RW2 \\ 
  Ethiopia & ALL & 90-94 & 194.17 & 188.18 & 201.32 & UN \\ 
  Ethiopia & ALL & 95-99 & 162.49 & 160.11 & 164.87 & IHME \\ 
  Ethiopia & ALL & 95-99 & 161.77 & 155.47 & 168.22 & RW2 \\ 
  Ethiopia & ALL & 95-99 & 162.00 & 156.75 & 167.62 & UN \\ 
  Ethiopia & ALL & 00-04 & 129.27 & 127.17 & 131.38 & IHME \\ 
  Ethiopia & ALL & 00-04 & 131.82 & 125.31 & 138.67 & RW2 \\ 
  Ethiopia & ALL & 00-04 & 131.72 & 126.68 & 136.42 & UN \\ 
  Ethiopia & ALL & 05-09 & 97.86 & 95.47 & 100.57 & IHME \\ 
  Ethiopia & ALL & 05-09 & 94.28 & 88.06 & 100.87 & RW2 \\ 
  Ethiopia & ALL & 05-09 & 94.39 & 89.20 & 99.54 & UN \\ 
  Ethiopia & ALL & 10-14 & 72.87 & 69.72 & 75.85 & IHME \\ 
  Ethiopia & ALL & 10-14 & 67.46 & 60.49 & 75.11 & RW2 \\ 
  Ethiopia & ALL & 10-14 & 67.36 & 60.19 & 75.41 & UN \\ 
  Ethiopia & AMHARA & 80-84 & 236.38 & 261.23 & 213.21 & HT-Direct \\ 
  Ethiopia & AMHARA & 80-84 & 234.19 & 215.15 & 254.36 & RW2 \\ 
  Ethiopia & AMHARA & 85-89 & 216.11 & 235.03 & 198.31 & HT-Direct \\ 
  Ethiopia & AMHARA & 85-89 & 212.94 & 200.38 & 225.65 & RW2 \\ 
  Ethiopia & AMHARA & 90-94 & 196.04 & 210.96 & 181.93 & HT-Direct \\ 
  Ethiopia & AMHARA & 90-94 & 196.03 & 186.01 & 206.13 & RW2 \\ 
  Ethiopia & AMHARA & 95-99 & 163.13 & 175.63 & 151.36 & HT-Direct \\ 
  Ethiopia & AMHARA & 95-99 & 166.19 & 157.47 & 175.42 & RW2 \\ 
  Ethiopia & AMHARA & 00-04 & 151.21 & 167.39 & 136.35 & HT-Direct \\ 
  Ethiopia & AMHARA & 00-04 & 136.62 & 127.32 & 147.04 & RW2 \\ 
  Ethiopia & AMHARA & 05-09 & 103.66 & 122.62 & 87.34 & HT-Direct \\ 
  Ethiopia & AMHARA & 05-09 & 101.01 & 91.17 & 111.79 & RW2 \\ 
  Ethiopia & AMHARA & 10-14 & 66.28 & 87.36 & 50.01 & HT-Direct \\ 
  Ethiopia & AMHARA & 10-14 & 72.58 & 60.36 & 85.57 & RW2 \\ 
  Ethiopia & AMHARA & 15-19 & 51.66 & 22.79 & 112.68 & RW2 \\ 
  Ethiopia & BENISHANGUL-GUMUZ & 80-84 & 273.63 & 323.95 & 228.49 & HT-Direct \\ 
  Ethiopia & BENISHANGUL-GUMUZ & 80-84 & 270.40 & 236.35 & 307.28 & RW2 \\ 
  Ethiopia & BENISHANGUL-GUMUZ & 85-89 & 255.95 & 291.44 & 223.42 & HT-Direct \\ 
  Ethiopia & BENISHANGUL-GUMUZ & 85-89 & 251.52 & 230.69 & 273.31 & RW2 \\ 
  Ethiopia & BENISHANGUL-GUMUZ & 90-94 & 240.45 & 265.79 & 216.81 & HT-Direct \\ 
  Ethiopia & BENISHANGUL-GUMUZ & 90-94 & 237.28 & 220.99 & 253.73 & RW2 \\ 
  Ethiopia & BENISHANGUL-GUMUZ & 95-99 & 194.10 & 219.21 & 171.24 & HT-Direct \\ 
  Ethiopia & BENISHANGUL-GUMUZ & 95-99 & 206.17 & 191.52 & 221.93 & RW2 \\ 
  Ethiopia & BENISHANGUL-GUMUZ & 00-04 & 190.56 & 214.51 & 168.70 & HT-Direct \\ 
  Ethiopia & BENISHANGUL-GUMUZ & 00-04 & 172.43 & 158.29 & 188.83 & RW2 \\ 
  Ethiopia & BENISHANGUL-GUMUZ & 05-09 & 135.41 & 165.59 & 110.01 & HT-Direct \\ 
  Ethiopia & BENISHANGUL-GUMUZ & 05-09 & 128.00 & 114.08 & 143.37 & RW2 \\ 
  Ethiopia & BENISHANGUL-GUMUZ & 10-14 & 90.34 & 116.33 & 69.70 & HT-Direct \\ 
  Ethiopia & BENISHANGUL-GUMUZ & 10-14 & 90.93 & 74.90 & 108.51 & RW2 \\ 
  Ethiopia & BENISHANGUL-GUMUZ & 15-19 & 63.66 & 27.95 & 137.99 & RW2 \\ 
  Ethiopia & DIRE DAWA & 80-84 & 204.66 & 256.04 & 161.36 & HT-Direct \\ 
  Ethiopia & DIRE DAWA & 80-84 & 223.81 & 189.76 & 260.90 & RW2 \\ 
  Ethiopia & DIRE DAWA & 85-89 & 219.10 & 252.79 & 188.77 & HT-Direct \\ 
  Ethiopia & DIRE DAWA & 85-89 & 210.13 & 189.76 & 231.98 & RW2 \\ 
  Ethiopia & DIRE DAWA & 90-94 & 203.24 & 233.63 & 175.89 & HT-Direct \\ 
  Ethiopia & DIRE DAWA & 90-94 & 197.41 & 181.20 & 215.11 & RW2 \\ 
  Ethiopia & DIRE DAWA & 95-99 & 169.82 & 197.15 & 145.59 & HT-Direct \\ 
  Ethiopia & DIRE DAWA & 95-99 & 165.01 & 151.11 & 180.46 & RW2 \\ 
  Ethiopia & DIRE DAWA & 00-04 & 129.26 & 150.21 & 110.85 & HT-Direct \\ 
  Ethiopia & DIRE DAWA & 00-04 & 131.23 & 119.30 & 144.20 & RW2 \\ 
  Ethiopia & DIRE DAWA & 05-09 & 96.29 & 115.74 & 79.82 & HT-Direct \\ 
  Ethiopia & DIRE DAWA & 05-09 & 96.36 & 85.68 & 108.33 & RW2 \\ 
  Ethiopia & DIRE DAWA & 10-14 & 81.13 & 105.84 & 61.79 & HT-Direct \\ 
  Ethiopia & DIRE DAWA & 10-14 & 71.09 & 59.05 & 85.10 & RW2 \\ 
  Ethiopia & DIRE DAWA & 15-19 & 52.69 & 23.29 & 115.28 & RW2 \\ 
  Ethiopia & GAMBELA & 80-84 & 265.93 & 312.11 & 224.35 & HT-Direct \\ 
  Ethiopia & GAMBELA & 80-84 & 273.14 & 238.32 & 309.68 & RW2 \\ 
  Ethiopia & GAMBELA & 85-89 & 245.68 & 285.64 & 209.67 & HT-Direct \\ 
  Ethiopia & GAMBELA & 85-89 & 254.73 & 232.31 & 278.42 & RW2 \\ 
  Ethiopia & GAMBELA & 90-94 & 261.14 & 294.62 & 230.23 & HT-Direct \\ 
  Ethiopia & GAMBELA & 90-94 & 235.05 & 217.17 & 254.78 & RW2 \\ 
  Ethiopia & GAMBELA & 95-99 & 193.44 & 219.53 & 169.77 & HT-Direct \\ 
  Ethiopia & GAMBELA & 95-99 & 189.37 & 174.51 & 205.72 & RW2 \\ 
  Ethiopia & GAMBELA & 00-04 & 126.24 & 149.32 & 106.29 & HT-Direct \\ 
  Ethiopia & GAMBELA & 00-04 & 143.74 & 130.05 & 158.09 & RW2 \\ 
  Ethiopia & GAMBELA & 05-09 & 106.75 & 130.40 & 86.96 & HT-Direct \\ 
  Ethiopia & GAMBELA & 05-09 & 102.65 & 90.83 & 115.57 & RW2 \\ 
  Ethiopia & GAMBELA & 10-14 & 87.32 & 111.89 & 67.73 & HT-Direct \\ 
  Ethiopia & GAMBELA & 10-14 & 74.36 & 62.20 & 88.78 & RW2 \\ 
  Ethiopia & GAMBELA & 15-19 & 54.28 & 24.06 & 118.42 & RW2 \\ 
  Ethiopia & HARARI & 80-84 & 265.94 & 335.32 & 206.46 & HT-Direct \\ 
  Ethiopia & HARARI & 80-84 & 249.58 & 208.16 & 295.88 & RW2 \\ 
  Ethiopia & HARARI & 85-89 & 194.98 & 228.33 & 165.45 & HT-Direct \\ 
  Ethiopia & HARARI & 85-89 & 219.81 & 197.45 & 243.88 & RW2 \\ 
  Ethiopia & HARARI & 90-94 & 222.73 & 251.20 & 196.63 & HT-Direct \\ 
  Ethiopia & HARARI & 90-94 & 195.38 & 179.24 & 213.19 & RW2 \\ 
  Ethiopia & HARARI & 95-99 & 151.24 & 175.09 & 130.13 & HT-Direct \\ 
  Ethiopia & HARARI & 95-99 & 150.92 & 137.31 & 165.39 & RW2 \\ 
  Ethiopia & HARARI & 00-04 & 98.46 & 117.28 & 82.38 & HT-Direct \\ 
  Ethiopia & HARARI & 00-04 & 111.37 & 98.88 & 124.19 & RW2 \\ 
  Ethiopia & HARARI & 05-09 & 77.26 & 96.34 & 61.71 & HT-Direct \\ 
  Ethiopia & HARARI & 05-09 & 79.87 & 69.60 & 91.36 & RW2 \\ 
  Ethiopia & HARARI & 10-14 & 78.87 & 105.46 & 58.54 & HT-Direct \\ 
  Ethiopia & HARARI & 10-14 & 60.24 & 48.71 & 75.24 & RW2 \\ 
  Ethiopia & HARARI & 15-19 & 46.20 & 19.78 & 106.68 & RW2 \\ 
  Ethiopia & OROMIYA & 80-84 & 244.25 & 268.72 & 221.34 & HT-Direct \\ 
  Ethiopia & OROMIYA & 80-84 & 235.24 & 217.63 & 254.13 & RW2 \\ 
  Ethiopia & OROMIYA & 85-89 & 204.02 & 220.58 & 188.40 & HT-Direct \\ 
  Ethiopia & OROMIYA & 85-89 & 213.14 & 201.51 & 225.33 & RW2 \\ 
  Ethiopia & OROMIYA & 90-94 & 207.72 & 224.96 & 191.48 & HT-Direct \\ 
  Ethiopia & OROMIYA & 90-94 & 195.54 & 185.93 & 205.55 & RW2 \\ 
  Ethiopia & OROMIYA & 95-99 & 158.25 & 170.20 & 146.99 & HT-Direct \\ 
  Ethiopia & OROMIYA & 95-99 & 160.71 & 152.72 & 169.09 & RW2 \\ 
  Ethiopia & OROMIYA & 00-04 & 127.96 & 140.86 & 116.08 & HT-Direct \\ 
  Ethiopia & OROMIYA & 00-04 & 126.69 & 119.10 & 134.74 & RW2 \\ 
  Ethiopia & OROMIYA & 05-09 & 93.37 & 105.08 & 82.85 & HT-Direct \\ 
  Ethiopia & OROMIYA & 05-09 & 92.57 & 85.31 & 100.29 & RW2 \\ 
  Ethiopia & OROMIYA & 10-14 & 78.51 & 96.58 & 63.58 & HT-Direct \\ 
  Ethiopia & OROMIYA & 10-14 & 67.84 & 59.62 & 77.01 & RW2 \\ 
  Ethiopia & OROMIYA & 15-19 & 50.09 & 22.95 & 105.51 & RW2 \\ 
  Ethiopia & SNNP & 80-84 & 237.15 & 270.34 & 206.88 & HT-Direct \\ 
  Ethiopia & SNNP & 80-84 & 245.36 & 219.90 & 271.79 & RW2 \\ 
  Ethiopia & SNNP & 85-89 & 237.31 & 259.20 & 216.72 & HT-Direct \\ 
  Ethiopia & SNNP & 85-89 & 229.81 & 215.19 & 245.26 & RW2 \\ 
  Ethiopia & SNNP & 90-94 & 218.67 & 235.95 & 202.33 & HT-Direct \\ 
  Ethiopia & SNNP & 90-94 & 212.86 & 201.59 & 224.88 & RW2 \\ 
  Ethiopia & SNNP & 95-99 & 173.44 & 187.58 & 160.15 & HT-Direct \\ 
  Ethiopia & SNNP & 95-99 & 174.54 & 164.85 & 184.81 & RW2 \\ 
  Ethiopia & SNNP & 00-04 & 135.26 & 148.19 & 123.30 & HT-Direct \\ 
  Ethiopia & SNNP & 00-04 & 136.37 & 127.38 & 145.70 & RW2 \\ 
  Ethiopia & SNNP & 05-09 & 102.83 & 118.89 & 88.72 & HT-Direct \\ 
  Ethiopia & SNNP & 05-09 & 99.72 & 90.51 & 109.75 & RW2 \\ 
  Ethiopia & SNNP & 10-14 & 83.34 & 106.93 & 64.59 & HT-Direct \\ 
  Ethiopia & SNNP & 10-14 & 73.60 & 62.29 & 87.04 & RW2 \\ 
  Ethiopia & SNNP & 15-19 & 54.61 & 24.32 & 119.06 & RW2 \\ 
  Ethiopia & SOMALI & 80-84 & 202.00 & 260.21 & 154.09 & HT-Direct \\ 
  Ethiopia & SOMALI & 80-84 & 178.96 & 148.71 & 214.11 & RW2 \\ 
  Ethiopia & SOMALI & 85-89 & 147.16 & 180.15 & 119.32 & HT-Direct \\ 
  Ethiopia & SOMALI & 85-89 & 171.42 & 151.18 & 193.44 & RW2 \\ 
  Ethiopia & SOMALI & 90-94 & 188.28 & 227.20 & 154.69 & HT-Direct \\ 
  Ethiopia & SOMALI & 90-94 & 167.32 & 151.57 & 184.22 & RW2 \\ 
  Ethiopia & SOMALI & 95-99 & 142.12 & 162.79 & 123.68 & HT-Direct \\ 
  Ethiopia & SOMALI & 95-99 & 147.61 & 135.33 & 160.61 & RW2 \\ 
  Ethiopia & SOMALI & 00-04 & 129.97 & 148.62 & 113.35 & HT-Direct \\ 
  Ethiopia & SOMALI & 00-04 & 126.73 & 116.43 & 137.83 & RW2 \\ 
  Ethiopia & SOMALI & 05-09 & 103.87 & 120.31 & 89.46 & HT-Direct \\ 
  Ethiopia & SOMALI & 05-09 & 101.56 & 92.28 & 111.81 & RW2 \\ 
  Ethiopia & SOMALI & 10-14 & 91.15 & 110.38 & 74.98 & HT-Direct \\ 
  Ethiopia & SOMALI & 10-14 & 81.86 & 71.20 & 94.02 & RW2 \\ 
  Ethiopia & SOMALI & 15-19 & 66.41 & 30.28 & 140.75 & RW2 \\ 
  Ethiopia & TIGRAY & 80-84 & 252.69 & 285.29 & 222.67 & HT-Direct \\ 
  Ethiopia & TIGRAY & 80-84 & 259.46 & 234.14 & 286.53 & RW2 \\ 
  Ethiopia & TIGRAY & 85-89 & 220.93 & 243.98 & 199.48 & HT-Direct \\ 
  Ethiopia & TIGRAY & 85-89 & 219.60 & 204.72 & 235.61 & RW2 \\ 
  Ethiopia & TIGRAY & 90-94 & 197.68 & 217.23 & 179.50 & HT-Direct \\ 
  Ethiopia & TIGRAY & 90-94 & 184.43 & 172.97 & 196.68 & RW2 \\ 
  Ethiopia & TIGRAY & 95-99 & 138.63 & 152.48 & 125.84 & HT-Direct \\ 
  Ethiopia & TIGRAY & 95-99 & 138.70 & 129.32 & 148.35 & RW2 \\ 
  Ethiopia & TIGRAY & 00-04 & 89.39 & 102.43 & 77.87 & HT-Direct \\ 
  Ethiopia & TIGRAY & 00-04 & 101.89 & 93.53 & 110.65 & RW2 \\ 
  Ethiopia & TIGRAY & 05-09 & 77.96 & 91.18 & 66.52 & HT-Direct \\ 
  Ethiopia & TIGRAY & 05-09 & 70.68 & 63.73 & 78.34 & RW2 \\ 
  Ethiopia & TIGRAY & 10-14 & 55.22 & 69.42 & 43.78 & HT-Direct \\ 
  Ethiopia & TIGRAY & 10-14 & 48.88 & 41.56 & 57.62 & RW2 \\ 
  Ethiopia & TIGRAY & 15-19 & 33.96 & 14.90 & 75.20 & RW2 \\ 
  Gabon & ALL & 80-84 & 105.56 & 102.43 & 108.52 & IHME \\ 
  Gabon & ALL & 80-84 & 110.48 & 95.37 & 127.67 & RW2 \\ 
  Gabon & ALL & 80-84 & 109.94 & 102.25 & 119.10 & UN \\ 
  Gabon & ALL & 85-89 & 89.66 & 87.53 & 91.86 & IHME \\ 
  Gabon & ALL & 85-89 & 96.12 & 84.72 & 108.69 & RW2 \\ 
  Gabon & ALL & 85-89 & 96.92 & 90.85 & 103.92 & UN \\ 
  Gabon & ALL & 90-94 & 79.90 & 78.23 & 81.80 & IHME \\ 
  Gabon & ALL & 90-94 & 92.83 & 83.69 & 102.90 & RW2 \\ 
  Gabon & ALL & 90-94 & 91.93 & 86.14 & 97.99 & UN \\ 
  Gabon & ALL & 95-99 & 73.53 & 71.86 & 75.24 & IHME \\ 
  Gabon & ALL & 95-99 & 88.84 & 78.51 & 100.29 & RW2 \\ 
  Gabon & ALL & 95-99 & 88.83 & 83.31 & 94.85 & UN \\ 
  Gabon & ALL & 00-04 & 69.99 & 68.33 & 71.76 & IHME \\ 
  Gabon & ALL & 00-04 & 82.35 & 66.09 & 102.83 & RW2 \\ 
  Gabon & ALL & 00-04 & 83.16 & 77.53 & 89.24 & UN \\ 
  Gabon & ALL & 05-09 & 63.23 & 61.13 & 64.97 & IHME \\ 
  Gabon & ALL & 05-09 & 70.22 & 57.95 & 84.82 & RW2 \\ 
  Gabon & ALL & 05-09 & 71.64 & 66.79 & 76.89 & UN \\ 
  Gabon & ALL & 10-14 & 52.38 & 49.61 & 55.02 & IHME \\ 
  Gabon & ALL & 10-14 & 58.55 & 50.24 & 68.07 & RW2 \\ 
  Gabon & ALL & 10-14 & 57.25 & 51.45 & 64.37 & UN \\ 
  Gabon & EAST & 80-84 & 71.07 & 99.51 & 50.30 & HT-Direct \\ 
  Gabon & EAST & 80-84 & 75.06 & 56.14 & 99.98 & RW2 \\ 
  Gabon & EAST & 85-89 & 59.77 & 81.58 & 43.52 & HT-Direct \\ 
  Gabon & EAST & 85-89 & 63.53 & 51.34 & 78.16 & RW2 \\ 
  Gabon & EAST & 90-94 & 53.77 & 66.88 & 43.10 & HT-Direct \\ 
  Gabon & EAST & 90-94 & 59.36 & 50.27 & 69.84 & RW2 \\ 
  Gabon & EAST & 95-99 & 62.94 & 78.44 & 50.34 & HT-Direct \\ 
  Gabon & EAST & 95-99 & 70.10 & 59.01 & 83.26 & RW2 \\ 
  Gabon & EAST & 00-04 & 60.95 & 81.52 & 45.32 & HT-Direct \\ 
  Gabon & EAST & 00-04 & 78.85 & 62.45 & 100.02 & RW2 \\ 
  Gabon & EAST & 05-09 & 53.92 & 75.62 & 38.19 & HT-Direct \\ 
  Gabon & EAST & 05-09 & 69.63 & 54.29 & 89.38 & RW2 \\ 
  Gabon & EAST & 10-14 & 77.36 & 118.42 & 49.74 & HT-Direct \\ 
  Gabon & EAST & 10-14 & 56.87 & 42.26 & 76.01 & RW2 \\ 
  Gabon & EAST & 15-19 & 45.98 & 16.17 & 124.15 & RW2 \\ 
  Gabon & LIBREVILLE,PORT-GENTIL & 80-84 & 98.38 & 125.53 & 76.59 & HT-Direct \\ 
  Gabon & LIBREVILLE,PORT-GENTIL & 80-84 & 113.60 & 88.89 & 142.62 & RW2 \\ 
  Gabon & LIBREVILLE,PORT-GENTIL & 85-89 & 83.11 & 100.47 & 68.52 & HT-Direct \\ 
  Gabon & LIBREVILLE,PORT-GENTIL & 85-89 & 101.20 & 86.08 & 118.69 & RW2 \\ 
  Gabon & LIBREVILLE,PORT-GENTIL & 90-94 & 104.24 & 122.98 & 88.07 & HT-Direct \\ 
  Gabon & LIBREVILLE,PORT-GENTIL & 90-94 & 92.58 & 80.02 & 107.94 & RW2 \\ 
  Gabon & LIBREVILLE,PORT-GENTIL & 95-99 & 73.79 & 91.46 & 59.31 & HT-Direct \\ 
  Gabon & LIBREVILLE,PORT-GENTIL & 95-99 & 93.51 & 78.65 & 110.96 & RW2 \\ 
  Gabon & LIBREVILLE,PORT-GENTIL & 00-04 & 54.34 & 79.61 & 36.78 & HT-Direct \\ 
  Gabon & LIBREVILLE,PORT-GENTIL & 00-04 & 88.14 & 66.74 & 113.27 & RW2 \\ 
  Gabon & LIBREVILLE,PORT-GENTIL & 05-09 & 47.17 & 63.14 & 35.10 & HT-Direct \\ 
  Gabon & LIBREVILLE,PORT-GENTIL & 05-09 & 69.07 & 52.97 & 87.86 & RW2 \\ 
  Gabon & LIBREVILLE,PORT-GENTIL & 10-14 & 75.53 & 107.89 & 52.30 & HT-Direct \\ 
  Gabon & LIBREVILLE,PORT-GENTIL & 10-14 & 53.11 & 40.84 & 68.68 & RW2 \\ 
  Gabon & LIBREVILLE,PORT-GENTIL & 15-19 & 41.09 & 14.67 & 113.80 & RW2 \\ 
  Gabon & NORTH & 80-84 & 125.59 & 166.00 & 93.90 & HT-Direct \\ 
  Gabon & NORTH & 80-84 & 136.81 & 107.53 & 172.54 & RW2 \\ 
  Gabon & NORTH & 85-89 & 101.25 & 123.63 & 82.54 & HT-Direct \\ 
  Gabon & NORTH & 85-89 & 110.70 & 93.78 & 130.39 & RW2 \\ 
  Gabon & NORTH & 90-94 & 89.93 & 108.09 & 74.56 & HT-Direct \\ 
  Gabon & NORTH & 90-94 & 97.69 & 84.75 & 112.04 & RW2 \\ 
  Gabon & NORTH & 95-99 & 97.87 & 117.82 & 80.98 & HT-Direct \\ 
  Gabon & NORTH & 95-99 & 107.45 & 92.36 & 125.18 & RW2 \\ 
  Gabon & NORTH & 00-04 & 90.92 & 123.90 & 66.05 & HT-Direct \\ 
  Gabon & NORTH & 00-04 & 112.83 & 91.41 & 140.24 & RW2 \\ 
  Gabon & NORTH & 05-09 & 75.44 & 90.90 & 62.44 & HT-Direct \\ 
  Gabon & NORTH & 05-09 & 94.59 & 78.57 & 113.70 & RW2 \\ 
  Gabon & NORTH & 10-14 & 100.25 & 126.73 & 78.81 & HT-Direct \\ 
  Gabon & NORTH & 10-14 & 73.60 & 61.84 & 87.44 & RW2 \\ 
  Gabon & NORTH & 15-19 & 56.53 & 21.68 & 141.29 & RW2 \\ 
  Gabon & SOUTH & 80-84 & 120.68 & 156.22 & 92.35 & HT-Direct \\ 
  Gabon & SOUTH & 80-84 & 125.02 & 98.72 & 156.98 & RW2 \\ 
  Gabon & SOUTH & 85-89 & 75.71 & 97.39 & 58.54 & HT-Direct \\ 
  Gabon & SOUTH & 85-89 & 98.76 & 82.51 & 117.67 & RW2 \\ 
  Gabon & SOUTH & 90-94 & 86.45 & 103.33 & 72.11 & HT-Direct \\ 
  Gabon & SOUTH & 90-94 & 85.95 & 74.62 & 98.70 & RW2 \\ 
  Gabon & SOUTH & 95-99 & 78.64 & 95.31 & 64.68 & HT-Direct \\ 
  Gabon & SOUTH & 95-99 & 92.22 & 79.03 & 108.19 & RW2 \\ 
  Gabon & SOUTH & 00-04 & 82.52 & 107.82 & 62.74 & HT-Direct \\ 
  Gabon & SOUTH & 00-04 & 93.57 & 74.92 & 117.84 & RW2 \\ 
  Gabon & SOUTH & 05-09 & 66.64 & 89.87 & 49.10 & HT-Direct \\ 
  Gabon & SOUTH & 05-09 & 74.15 & 59.11 & 93.09 & RW2 \\ 
  Gabon & SOUTH & 10-14 & 68.52 & 96.32 & 48.31 & HT-Direct \\ 
  Gabon & SOUTH & 10-14 & 53.66 & 41.87 & 68.19 & RW2 \\ 
  Gabon & SOUTH & 15-19 & 38.09 & 13.77 & 101.42 & RW2 \\ 
  Gabon & WEST & 80-84 & 154.47 & 195.61 & 120.68 & HT-Direct \\ 
  Gabon & WEST & 80-84 & 142.34 & 114.10 & 179.28 & RW2 \\ 
  Gabon & WEST & 85-89 & 97.53 & 118.17 & 80.17 & HT-Direct \\ 
  Gabon & WEST & 85-89 & 106.86 & 90.89 & 125.18 & RW2 \\ 
  Gabon & WEST & 90-94 & 67.05 & 83.30 & 53.79 & HT-Direct \\ 
  Gabon & WEST & 90-94 & 88.55 & 75.56 & 102.65 & RW2 \\ 
  Gabon & WEST & 95-99 & 88.25 & 106.38 & 72.95 & HT-Direct \\ 
  Gabon & WEST & 95-99 & 93.22 & 78.36 & 109.05 & RW2 \\ 
  Gabon & WEST & 00-04 & 64.81 & 106.86 & 38.59 & HT-Direct \\ 
  Gabon & WEST & 00-04 & 96.37 & 75.87 & 121.26 & RW2 \\ 
  Gabon & WEST & 05-09 & 57.56 & 76.43 & 43.13 & HT-Direct \\ 
  Gabon & WEST & 05-09 & 82.71 & 66.57 & 101.99 & RW2 \\ 
  Gabon & WEST & 10-14 & 101.09 & 132.91 & 76.21 & HT-Direct \\ 
  Gabon & WEST & 10-14 & 68.30 & 55.55 & 84.13 & RW2 \\ 
  Gabon & WEST & 15-19 & 56.61 & 21.55 & 143.38 & RW2 \\ 
  Gambia & ALL & 80-84 & 135.82 & 131.90 & 139.80 & IHME \\ 
  Gambia & ALL & 80-84 & 231.78 & 145.93 & 347.78 & RW2 \\ 
  Gambia & ALL & 80-84 & 228.47 & 214.24 & 242.39 & UN \\ 
  Gambia & ALL & 85-89 & 120.48 & 117.40 & 123.37 & IHME \\ 
  Gambia & ALL & 85-89 & 190.37 & 137.48 & 257.00 & RW2 \\ 
  Gambia & ALL & 85-89 & 191.14 & 180.53 & 202.28 & UN \\ 
  Gambia & ALL & 90-94 & 108.04 & 105.88 & 110.37 & IHME \\ 
  Gambia & ALL & 90-94 & 158.19 & 120.92 & 204.28 & RW2 \\ 
  Gambia & ALL & 90-94 & 158.24 & 150.73 & 166.37 & UN \\ 
  Gambia & ALL & 95-99 & 94.18 & 92.22 & 96.12 & IHME \\ 
  Gambia & ALL & 95-99 & 132.33 & 106.91 & 162.34 & RW2 \\ 
  Gambia & ALL & 95-99 & 132.66 & 126.03 & 139.90 & UN \\ 
  Gambia & ALL & 00-04 & 79.35 & 77.59 & 81.14 & IHME \\ 
  Gambia & ALL & 00-04 & 111.24 & 93.35 & 132.28 & RW2 \\ 
  Gambia & ALL & 00-04 & 110.82 & 104.31 & 117.45 & UN \\ 
  Gambia & ALL & 05-09 & 65.01 & 63.14 & 66.77 & IHME \\ 
  Gambia & ALL & 05-09 & 91.10 & 75.52 & 109.44 & RW2 \\ 
  Gambia & ALL & 05-09 & 90.98 & 83.48 & 100.36 & UN \\ 
  Gambia & ALL & 10-14 & 51.98 & 49.99 & 54.01 & IHME \\ 
  Gambia & ALL & 10-14 & 75.81 & 54.80 & 103.78 & RW2 \\ 
  Gambia & ALL & 10-14 & 76.27 & 66.30 & 88.83 & UN \\ 
  Gambia & BANJUL & 80-84 & 112.43 & 300.72 & 35.97 & HT-Direct \\ 
  Gambia & BANJUL & 80-84 & 128.94 & 50.54 & 297.19 & RW2 \\ 
  Gambia & BANJUL & 85-89 & 86.72 & 154.53 & 47.01 & HT-Direct \\ 
  Gambia & BANJUL & 85-89 & 110.04 & 55.28 & 206.32 & RW2 \\ 
  Gambia & BANJUL & 90-94 & 81.83 & 130.12 & 50.42 & HT-Direct \\ 
  Gambia & BANJUL & 90-94 & 97.03 & 58.20 & 157.85 & RW2 \\ 
  Gambia & BANJUL & 95-99 & 49.26 & 71.85 & 33.51 & HT-Direct \\ 
  Gambia & BANJUL & 95-99 & 88.03 & 59.66 & 127.26 & RW2 \\ 
  Gambia & BANJUL & 00-04 & 53.64 & 80.84 & 35.24 & HT-Direct \\ 
  Gambia & BANJUL & 00-04 & 82.55 & 59.45 & 112.99 & RW2 \\ 
  Gambia & BANJUL & 05-09 & 56.07 & 75.97 & 41.15 & HT-Direct \\ 
  Gambia & BANJUL & 05-09 & 75.02 & 53.23 & 105.09 & RW2 \\ 
  Gambia & BANJUL & 10-14 & 50.91 & 70.81 & 36.39 & HT-Direct \\ 
  Gambia & BANJUL & 10-14 & 68.14 & 40.94 & 112.05 & RW2 \\ 
  Gambia & BANJUL & 15-19 & 62.04 & 16.83 & 206.48 & RW2 \\ 
  Gambia & CENTRAL RIVER & 80-84 & 145.98 & 299.41 & 63.99 & HT-Direct \\ 
  Gambia & CENTRAL RIVER & 80-84 & 223.76 & 109.53 & 403.47 & RW2 \\ 
  Gambia & CENTRAL RIVER & 85-89 & 111.39 & 194.41 & 61.13 & HT-Direct \\ 
  Gambia & CENTRAL RIVER & 85-89 & 182.43 & 105.68 & 296.38 & RW2 \\ 
  Gambia & CENTRAL RIVER & 90-94 & 84.32 & 120.50 & 58.28 & HT-Direct \\ 
  Gambia & CENTRAL RIVER & 90-94 & 152.46 & 101.06 & 222.95 & RW2 \\ 
  Gambia & CENTRAL RIVER & 95-99 & 107.74 & 145.43 & 78.92 & HT-Direct \\ 
  Gambia & CENTRAL RIVER & 95-99 & 127.64 & 94.47 & 170.17 & RW2 \\ 
  Gambia & CENTRAL RIVER & 00-04 & 78.10 & 99.95 & 60.70 & HT-Direct \\ 
  Gambia & CENTRAL RIVER & 00-04 & 106.88 & 83.35 & 136.17 & RW2 \\ 
  Gambia & CENTRAL RIVER & 05-09 & 57.94 & 77.17 & 43.27 & HT-Direct \\ 
  Gambia & CENTRAL RIVER & 05-09 & 83.95 & 61.95 & 112.82 & RW2 \\ 
  Gambia & CENTRAL RIVER & 10-14 & 41.43 & 57.43 & 29.75 & HT-Direct \\ 
  Gambia & CENTRAL RIVER & 10-14 & 64.01 & 39.87 & 100.87 & RW2 \\ 
  Gambia & CENTRAL RIVER & 15-19 & 48.15 & 13.62 & 156.26 & RW2 \\ 
  Gambia & LOWER RIVER & 80-84 & 380.22 & 540.49 & 242.40 & HT-Direct \\ 
  Gambia & LOWER RIVER & 80-84 & 390.18 & 225.43 & 584.05 & RW2 \\ 
  Gambia & LOWER RIVER & 85-89 & 167.40 & 247.12 & 109.66 & HT-Direct \\ 
  Gambia & LOWER RIVER & 85-89 & 294.69 & 187.72 & 431.66 & RW2 \\ 
  Gambia & LOWER RIVER & 90-94 & 141.20 & 190.88 & 102.81 & HT-Direct \\ 
  Gambia & LOWER RIVER & 90-94 & 221.75 & 151.52 & 312.25 & RW2 \\ 
  Gambia & LOWER RIVER & 95-99 & 120.08 & 176.79 & 79.80 & HT-Direct \\ 
  Gambia & LOWER RIVER & 95-99 & 164.80 & 119.71 & 223.19 & RW2 \\ 
  Gambia & LOWER RIVER & 00-04 & 83.50 & 129.15 & 53.01 & HT-Direct \\ 
  Gambia & LOWER RIVER & 00-04 & 121.84 & 90.10 & 163.94 & RW2 \\ 
  Gambia & LOWER RIVER & 05-09 & 72.16 & 96.98 & 53.32 & HT-Direct \\ 
  Gambia & LOWER RIVER & 05-09 & 84.36 & 58.61 & 120.19 & RW2 \\ 
  Gambia & LOWER RIVER & 10-14 & 34.17 & 55.32 & 20.93 & HT-Direct \\ 
  Gambia & LOWER RIVER & 10-14 & 56.39 & 32.22 & 97.25 & RW2 \\ 
  Gambia & LOWER RIVER & 15-19 & 37.11 & 9.73 & 131.09 & RW2 \\ 
  Gambia & NORTH BANK & 80-84 & 206.13 & 350.18 & 111.20 & HT-Direct \\ 
  Gambia & NORTH BANK & 80-84 & 279.84 & 156.63 & 449.03 & RW2 \\ 
  Gambia & NORTH BANK & 85-89 & 148.72 & 201.16 & 108.10 & HT-Direct \\ 
  Gambia & NORTH BANK & 85-89 & 219.04 & 142.40 & 320.55 & RW2 \\ 
  Gambia & NORTH BANK & 90-94 & 101.70 & 136.95 & 74.75 & HT-Direct \\ 
  Gambia & NORTH BANK & 90-94 & 174.28 & 123.85 & 240.21 & RW2 \\ 
  Gambia & NORTH BANK & 95-99 & 96.34 & 125.47 & 73.40 & HT-Direct \\ 
  Gambia & NORTH BANK & 95-99 & 138.42 & 104.91 & 180.22 & RW2 \\ 
  Gambia & NORTH BANK & 00-04 & 93.08 & 119.46 & 72.04 & HT-Direct \\ 
  Gambia & NORTH BANK & 00-04 & 109.37 & 84.74 & 140.51 & RW2 \\ 
  Gambia & NORTH BANK & 05-09 & 55.23 & 75.04 & 40.43 & HT-Direct \\ 
  Gambia & NORTH BANK & 05-09 & 80.69 & 56.96 & 112.79 & RW2 \\ 
  Gambia & NORTH BANK & 10-14 & 28.97 & 51.08 & 16.27 & HT-Direct \\ 
  Gambia & NORTH BANK & 10-14 & 57.46 & 32.18 & 100.13 & RW2 \\ 
  Gambia & NORTH BANK & 15-19 & 40.74 & 10.24 & 145.43 & RW2 \\ 
  Gambia & UPPER RIVER & 80-84 & 240.89 & 356.14 & 154.02 & HT-Direct \\ 
  Gambia & UPPER RIVER & 80-84 & 334.64 & 199.47 & 502.82 & RW2 \\ 
  Gambia & UPPER RIVER & 85-89 & 166.20 & 274.60 & 94.99 & HT-Direct \\ 
  Gambia & UPPER RIVER & 85-89 & 279.53 & 182.41 & 402.97 & RW2 \\ 
  Gambia & UPPER RIVER & 90-94 & 212.51 & 265.72 & 167.53 & HT-Direct \\ 
  Gambia & UPPER RIVER & 90-94 & 237.08 & 170.44 & 318.85 & RW2 \\ 
  Gambia & UPPER RIVER & 95-99 & 131.26 & 157.74 & 108.65 & HT-Direct \\ 
  Gambia & UPPER RIVER & 95-99 & 199.11 & 155.37 & 251.94 & RW2 \\ 
  Gambia & UPPER RIVER & 00-04 & 121.49 & 164.61 & 88.48 & HT-Direct \\ 
  Gambia & UPPER RIVER & 00-04 & 166.42 & 129.07 & 213.01 & RW2 \\ 
  Gambia & UPPER RIVER & 05-09 & 117.91 & 158.34 & 86.74 & HT-Direct \\ 
  Gambia & UPPER RIVER & 05-09 & 130.30 & 95.51 & 175.26 & RW2 \\ 
  Gambia & UPPER RIVER & 10-14 & 58.91 & 76.75 & 45.01 & HT-Direct \\ 
  Gambia & UPPER RIVER & 10-14 & 97.83 & 63.05 & 148.30 & RW2 \\ 
  Gambia & UPPER RIVER & 15-19 & 72.49 & 21.34 & 216.61 & RW2 \\ 
  Gambia & WESTERN & 80-84 & 149.61 & 286.40 & 71.60 & HT-Direct \\ 
  Gambia & WESTERN & 80-84 & 198.99 & 99.28 & 362.69 & RW2 \\ 
  Gambia & WESTERN & 85-89 & 123.19 & 192.34 & 76.54 & HT-Direct \\ 
  Gambia & WESTERN & 85-89 & 163.43 & 96.63 & 263.97 & RW2 \\ 
  Gambia & WESTERN & 90-94 & 83.51 & 121.71 & 56.53 & HT-Direct \\ 
  Gambia & WESTERN & 90-94 & 138.66 & 92.07 & 203.67 & RW2 \\ 
  Gambia & WESTERN & 95-99 & 86.41 & 118.10 & 62.61 & HT-Direct \\ 
  Gambia & WESTERN & 95-99 & 119.59 & 87.74 & 160.30 & RW2 \\ 
  Gambia & WESTERN & 00-04 & 85.77 & 109.98 & 66.49 & HT-Direct \\ 
  Gambia & WESTERN & 00-04 & 105.00 & 82.19 & 133.14 & RW2 \\ 
  Gambia & WESTERN & 05-09 & 57.43 & 71.08 & 46.27 & HT-Direct \\ 
  Gambia & WESTERN & 05-09 & 88.24 & 67.94 & 113.79 & RW2 \\ 
  Gambia & WESTERN & 10-14 & 61.64 & 84.07 & 44.90 & HT-Direct \\ 
  Gambia & WESTERN & 10-14 & 73.53 & 47.19 & 113.42 & RW2 \\ 
  Gambia & WESTERN & 15-19 & 61.51 & 17.46 & 193.58 & RW2 \\ 
  Ghana & ALL & 80-84 & 148.92 & 146.86 & 151.03 & IHME \\ 
  Ghana & ALL & 80-84 & 162.85 & 154.80 & 171.24 & RW2 \\ 
  Ghana & ALL & 80-84 & 162.85 & 158.95 & 166.77 & UN \\ 
  Ghana & ALL & 85-89 & 134.21 & 132.48 & 135.87 & IHME \\ 
  Ghana & ALL & 85-89 & 143.76 & 136.58 & 151.25 & RW2 \\ 
  Ghana & ALL & 85-89 & 143.80 & 140.49 & 147.51 & UN \\ 
  Ghana & ALL & 90-94 & 116.65 & 115.07 & 118.43 & IHME \\ 
  Ghana & ALL & 90-94 & 120.47 & 113.87 & 127.37 & RW2 \\ 
  Ghana & ALL & 90-94 & 120.40 & 117.32 & 123.76 & UN \\ 
  Ghana & ALL & 95-99 & 105.14 & 103.63 & 106.60 & IHME \\ 
  Ghana & ALL & 95-99 & 108.99 & 101.93 & 116.45 & RW2 \\ 
  Ghana & ALL & 95-99 & 109.02 & 106.25 & 112.08 & UN \\ 
  Ghana & ALL & 00-04 & 93.30 & 91.80 & 94.80 & IHME \\ 
  Ghana & ALL & 00-04 & 94.18 & 87.62 & 101.19 & RW2 \\ 
  Ghana & ALL & 00-04 & 94.36 & 91.46 & 97.37 & UN \\ 
  Ghana & ALL & 05-09 & 79.49 & 77.85 & 81.08 & IHME \\ 
  Ghana & ALL & 05-09 & 82.60 & 75.16 & 90.73 & RW2 \\ 
  Ghana & ALL & 05-09 & 82.20 & 79.04 & 85.53 & UN \\ 
  Ghana & ALL & 10-14 & 64.38 & 62.12 & 66.82 & IHME \\ 
  Ghana & ALL & 10-14 & 68.64 & 58.60 & 80.17 & RW2 \\ 
  Ghana & ALL & 10-14 & 68.91 & 64.12 & 74.35 & UN \\ 
  Ghana & ASHANTI & 80-84 & 132.46 & 151.34 & 115.61 & HT-Direct \\ 
  Ghana & ASHANTI & 80-84 & 133.61 & 118.91 & 149.69 & RW2 \\ 
  Ghana & ASHANTI & 85-89 & 122.88 & 140.18 & 107.44 & HT-Direct \\ 
  Ghana & ASHANTI & 85-89 & 122.43 & 111.55 & 133.86 & RW2 \\ 
  Ghana & ASHANTI & 90-94 & 84.64 & 98.88 & 72.28 & HT-Direct \\ 
  Ghana & ASHANTI & 90-94 & 109.15 & 99.36 & 119.57 & RW2 \\ 
  Ghana & ASHANTI & 95-99 & 115.12 & 133.15 & 99.26 & HT-Direct \\ 
  Ghana & ASHANTI & 95-99 & 104.72 & 94.82 & 115.65 & RW2 \\ 
  Ghana & ASHANTI & 00-04 & 94.63 & 112.05 & 79.68 & HT-Direct \\ 
  Ghana & ASHANTI & 00-04 & 98.33 & 87.72 & 110.19 & RW2 \\ 
  Ghana & ASHANTI & 05-09 & 81.03 & 104.38 & 62.53 & HT-Direct \\ 
  Ghana & ASHANTI & 05-09 & 91.21 & 77.97 & 106.40 & RW2 \\ 
  Ghana & ASHANTI & 10-14 & 74.94 & 99.86 & 55.86 & HT-Direct \\ 
  Ghana & ASHANTI & 10-14 & 80.01 & 63.42 & 100.48 & RW2 \\ 
  Ghana & ASHANTI & 15-19 & 68.57 & 28.58 & 154.82 & RW2 \\ 
  Ghana & BRONG AHAFO & 80-84 & 132.03 & 156.16 & 111.14 & HT-Direct \\ 
  Ghana & BRONG AHAFO & 80-84 & 139.44 & 121.29 & 159.89 & RW2 \\ 
  Ghana & BRONG AHAFO & 85-89 & 122.97 & 144.36 & 104.36 & HT-Direct \\ 
  Ghana & BRONG AHAFO & 85-89 & 122.38 & 109.82 & 136.30 & RW2 \\ 
  Ghana & BRONG AHAFO & 90-94 & 90.98 & 112.15 & 73.47 & HT-Direct \\ 
  Ghana & BRONG AHAFO & 90-94 & 103.67 & 93.05 & 115.57 & RW2 \\ 
  Ghana & BRONG AHAFO & 95-99 & 90.91 & 111.74 & 73.64 & HT-Direct \\ 
  Ghana & BRONG AHAFO & 95-99 & 93.10 & 82.81 & 104.77 & RW2 \\ 
  Ghana & BRONG AHAFO & 00-04 & 84.62 & 103.83 & 68.69 & HT-Direct \\ 
  Ghana & BRONG AHAFO & 00-04 & 81.72 & 71.46 & 93.41 & RW2 \\ 
  Ghana & BRONG AHAFO & 05-09 & 65.47 & 89.43 & 47.60 & HT-Direct \\ 
  Ghana & BRONG AHAFO & 05-09 & 70.45 & 58.75 & 84.07 & RW2 \\ 
  Ghana & BRONG AHAFO & 10-14 & 44.70 & 61.67 & 32.24 & HT-Direct \\ 
  Ghana & BRONG AHAFO & 10-14 & 56.90 & 43.99 & 72.81 & RW2 \\ 
  Ghana & BRONG AHAFO & 15-19 & 44.99 & 18.46 & 103.33 & RW2 \\ 
  Ghana & CENTRAL & 80-84 & 184.65 & 211.69 & 160.37 & HT-Direct \\ 
  Ghana & CENTRAL & 80-84 & 184.82 & 164.09 & 207.91 & RW2 \\ 
  Ghana & CENTRAL & 85-89 & 154.43 & 177.15 & 134.15 & HT-Direct \\ 
  Ghana & CENTRAL & 85-89 & 155.30 & 141.18 & 170.43 & RW2 \\ 
  Ghana & CENTRAL & 90-94 & 111.73 & 130.22 & 95.57 & HT-Direct \\ 
  Ghana & CENTRAL & 90-94 & 126.57 & 114.43 & 139.52 & RW2 \\ 
  Ghana & CENTRAL & 95-99 & 102.00 & 127.72 & 80.98 & HT-Direct \\ 
  Ghana & CENTRAL & 95-99 & 110.55 & 97.96 & 124.16 & RW2 \\ 
  Ghana & CENTRAL & 00-04 & 90.56 & 111.93 & 72.94 & HT-Direct \\ 
  Ghana & CENTRAL & 00-04 & 95.88 & 83.48 & 110.00 & RW2 \\ 
  Ghana & CENTRAL & 05-09 & 89.72 & 118.57 & 67.35 & HT-Direct \\ 
  Ghana & CENTRAL & 05-09 & 83.13 & 68.75 & 100.24 & RW2 \\ 
  Ghana & CENTRAL & 10-14 & 68.17 & 110.51 & 41.30 & HT-Direct \\ 
  Ghana & CENTRAL & 10-14 & 68.04 & 50.71 & 92.22 & RW2 \\ 
  Ghana & CENTRAL & 15-19 & 54.70 & 21.51 & 131.46 & RW2 \\ 
  Ghana & EASTERN & 80-84 & 119.77 & 136.97 & 104.47 & HT-Direct \\ 
  Ghana & EASTERN & 80-84 & 124.93 & 110.83 & 140.59 & RW2 \\ 
  Ghana & EASTERN & 85-89 & 101.64 & 119.90 & 85.90 & HT-Direct \\ 
  Ghana & EASTERN & 85-89 & 114.66 & 103.77 & 126.69 & RW2 \\ 
  Ghana & EASTERN & 90-94 & 117.69 & 137.92 & 100.08 & HT-Direct \\ 
  Ghana & EASTERN & 90-94 & 101.75 & 91.87 & 112.89 & RW2 \\ 
  Ghana & EASTERN & 95-99 & 88.75 & 110.14 & 71.18 & HT-Direct \\ 
  Ghana & EASTERN & 95-99 & 94.95 & 84.27 & 106.88 & RW2 \\ 
  Ghana & EASTERN & 00-04 & 69.93 & 91.94 & 52.88 & HT-Direct \\ 
  Ghana & EASTERN & 00-04 & 86.89 & 75.37 & 99.90 & RW2 \\ 
  Ghana & EASTERN & 05-09 & 69.93 & 89.68 & 54.27 & HT-Direct \\ 
  Ghana & EASTERN & 05-09 & 79.20 & 66.45 & 94.04 & RW2 \\ 
  Ghana & EASTERN & 10-14 & 75.23 & 107.86 & 51.89 & HT-Direct \\ 
  Ghana & EASTERN & 10-14 & 68.56 & 52.76 & 88.71 & RW2 \\ 
  Ghana & EASTERN & 15-19 & 58.26 & 23.54 & 134.20 & RW2 \\ 
  Ghana & GREATER ACCRA & 80-84 & 123.51 & 145.58 & 104.37 & HT-Direct \\ 
  Ghana & GREATER ACCRA & 80-84 & 128.66 & 111.40 & 148.28 & RW2 \\ 
  Ghana & GREATER ACCRA & 85-89 & 109.33 & 131.46 & 90.54 & HT-Direct \\ 
  Ghana & GREATER ACCRA & 85-89 & 108.66 & 96.61 & 122.11 & RW2 \\ 
  Ghana & GREATER ACCRA & 90-94 & 83.73 & 104.17 & 67.00 & HT-Direct \\ 
  Ghana & GREATER ACCRA & 90-94 & 88.61 & 78.16 & 100.10 & RW2 \\ 
  Ghana & GREATER ACCRA & 95-99 & 73.28 & 94.28 & 56.67 & HT-Direct \\ 
  Ghana & GREATER ACCRA & 95-99 & 76.87 & 66.42 & 88.68 & RW2 \\ 
  Ghana & GREATER ACCRA & 00-04 & 48.57 & 67.82 & 34.58 & HT-Direct \\ 
  Ghana & GREATER ACCRA & 00-04 & 66.14 & 55.53 & 78.68 & RW2 \\ 
  Ghana & GREATER ACCRA & 05-09 & 68.10 & 93.02 & 49.50 & HT-Direct \\ 
  Ghana & GREATER ACCRA & 05-09 & 57.18 & 45.41 & 71.77 & RW2 \\ 
  Ghana & GREATER ACCRA & 10-14 & 30.81 & 62.30 & 14.98 & HT-Direct \\ 
  Ghana & GREATER ACCRA & 10-14 & 46.62 & 32.70 & 66.76 & RW2 \\ 
  Ghana & GREATER ACCRA & 15-19 & 37.24 & 14.03 & 95.01 & RW2 \\ 
  Ghana & UPPER W,E \& NORTHERN & 80-84 & 248.49 & 270.79 & 227.45 & HT-Direct \\ 
  Ghana & UPPER W,E \& NORTHERN & 80-84 & 253.95 & 234.53 & 274.30 & RW2 \\ 
  Ghana & UPPER W,E \& NORTHERN & 85-89 & 206.91 & 223.23 & 191.49 & HT-Direct \\ 
  Ghana & UPPER W,E \& NORTHERN & 85-89 & 215.39 & 202.40 & 228.85 & RW2 \\ 
  Ghana & UPPER W,E \& NORTHERN & 90-94 & 171.04 & 185.98 & 157.07 & HT-Direct \\ 
  Ghana & UPPER W,E \& NORTHERN & 90-94 & 176.59 & 165.09 & 188.64 & RW2 \\ 
  Ghana & UPPER W,E \& NORTHERN & 95-99 & 140.55 & 154.72 & 127.48 & HT-Direct \\ 
  Ghana & UPPER W,E \& NORTHERN & 95-99 & 152.88 & 141.69 & 164.63 & RW2 \\ 
  Ghana & UPPER W,E \& NORTHERN & 00-04 & 126.60 & 140.37 & 114.00 & HT-Direct \\ 
  Ghana & UPPER W,E \& NORTHERN & 00-04 & 131.91 & 121.36 & 143.22 & RW2 \\ 
  Ghana & UPPER W,E \& NORTHERN & 05-09 & 117.70 & 136.86 & 100.90 & HT-Direct \\ 
  Ghana & UPPER W,E \& NORTHERN & 05-09 & 112.93 & 100.60 & 126.79 & RW2 \\ 
  Ghana & UPPER W,E \& NORTHERN & 10-14 & 75.79 & 93.35 & 61.31 & HT-Direct \\ 
  Ghana & UPPER W,E \& NORTHERN & 10-14 & 90.32 & 74.71 & 109.01 & RW2 \\ 
  Ghana & UPPER W,E \& NORTHERN & 15-19 & 70.15 & 29.55 & 156.69 & RW2 \\ 
  Ghana & VOLTA & 80-84 & 155.46 & 177.78 & 135.48 & HT-Direct \\ 
  Ghana & VOLTA & 80-84 & 153.91 & 136.94 & 172.42 & RW2 \\ 
  Ghana & VOLTA & 85-89 & 118.64 & 134.55 & 104.39 & HT-Direct \\ 
  Ghana & VOLTA & 85-89 & 130.51 & 119.13 & 142.80 & RW2 \\ 
  Ghana & VOLTA & 90-94 & 107.58 & 128.13 & 89.99 & HT-Direct \\ 
  Ghana & VOLTA & 90-94 & 107.46 & 96.72 & 119.17 & RW2 \\ 
  Ghana & VOLTA & 95-99 & 88.21 & 114.48 & 67.51 & HT-Direct \\ 
  Ghana & VOLTA & 95-99 & 93.99 & 82.52 & 106.76 & RW2 \\ 
  Ghana & VOLTA & 00-04 & 80.22 & 104.66 & 61.09 & HT-Direct \\ 
  Ghana & VOLTA & 00-04 & 81.15 & 69.24 & 94.69 & RW2 \\ 
  Ghana & VOLTA & 05-09 & 64.31 & 87.12 & 47.16 & HT-Direct \\ 
  Ghana & VOLTA & 05-09 & 69.67 & 56.53 & 85.68 & RW2 \\ 
  Ghana & VOLTA & 10-14 & 55.65 & 97.16 & 31.26 & HT-Direct \\ 
  Ghana & VOLTA & 10-14 & 56.31 & 41.16 & 77.19 & RW2 \\ 
  Ghana & VOLTA & 15-19 & 44.45 & 17.45 & 108.42 & RW2 \\ 
  Ghana & WESTERN & 80-84 & 147.64 & 175.97 & 123.19 & HT-Direct \\ 
  Ghana & WESTERN & 80-84 & 155.85 & 134.45 & 179.72 & RW2 \\ 
  Ghana & WESTERN & 85-89 & 126.72 & 148.82 & 107.49 & HT-Direct \\ 
  Ghana & WESTERN & 85-89 & 135.99 & 122.12 & 151.25 & RW2 \\ 
  Ghana & WESTERN & 90-94 & 117.50 & 138.05 & 99.66 & HT-Direct \\ 
  Ghana & WESTERN & 90-94 & 114.39 & 102.93 & 127.24 & RW2 \\ 
  Ghana & WESTERN & 95-99 & 92.48 & 114.40 & 74.40 & HT-Direct \\ 
  Ghana & WESTERN & 95-99 & 101.06 & 89.55 & 114.34 & RW2 \\ 
  Ghana & WESTERN & 00-04 & 93.60 & 117.08 & 74.43 & HT-Direct \\ 
  Ghana & WESTERN & 00-04 & 86.59 & 75.05 & 99.85 & RW2 \\ 
  Ghana & WESTERN & 05-09 & 61.65 & 83.35 & 45.32 & HT-Direct \\ 
  Ghana & WESTERN & 05-09 & 72.27 & 59.51 & 87.38 & RW2 \\ 
  Ghana & WESTERN & 10-14 & 44.56 & 67.36 & 29.23 & HT-Direct \\ 
  Ghana & WESTERN & 10-14 & 56.55 & 41.79 & 75.19 & RW2 \\ 
  Ghana & WESTERN & 15-19 & 42.90 & 16.77 & 103.57 & RW2 \\ 
  Guinea & ALL & 80-84 & 266.35 & 262.09 & 270.46 & IHME \\ 
  Guinea & ALL & 80-84 & 277.54 & 261.89 & 293.76 & RW2 \\ 
  Guinea & ALL & 80-84 & 277.32 & 268.92 & 287.27 & UN \\ 
  Guinea & ALL & 85-89 & 243.38 & 239.92 & 247.15 & IHME \\ 
  Guinea & ALL & 85-89 & 252.93 & 240.41 & 265.82 & RW2 \\ 
  Guinea & ALL & 85-89 & 253.38 & 246.20 & 260.64 & UN \\ 
  Guinea & ALL & 90-94 & 218.40 & 215.40 & 221.44 & IHME \\ 
  Guinea & ALL & 90-94 & 225.75 & 216.66 & 235.12 & RW2 \\ 
  Guinea & ALL & 90-94 & 225.40 & 219.01 & 232.15 & UN \\ 
  Guinea & ALL & 95-99 & 191.12 & 188.41 & 193.91 & IHME \\ 
  Guinea & ALL & 95-99 & 191.68 & 183.93 & 199.61 & RW2 \\ 
  Guinea & ALL & 95-99 & 191.83 & 185.91 & 197.59 & UN \\ 
  Guinea & ALL & 00-04 & 164.93 & 162.06 & 167.72 & IHME \\ 
  Guinea & ALL & 00-04 & 155.90 & 148.49 & 163.64 & RW2 \\ 
  Guinea & ALL & 00-04 & 156.13 & 150.88 & 161.29 & UN \\ 
  Guinea & ALL & 05-09 & 141.98 & 138.53 & 145.73 & IHME \\ 
  Guinea & ALL & 05-09 & 127.16 & 117.49 & 137.51 & RW2 \\ 
  Guinea & ALL & 05-09 & 126.22 & 120.11 & 132.37 & UN \\ 
  Guinea & ALL & 10-14 & 120.45 & 116.00 & 125.13 & IHME \\ 
  Guinea & ALL & 10-14 & 103.98 & 91.81 & 117.45 & RW2 \\ 
  Guinea & ALL & 10-14 & 104.64 & 96.21 & 113.99 & UN \\ 
  Guinea & CENTRAL GUINEA & 80-84 & 263.88 & 294.91 & 235.03 & HT-Direct \\ 
  Guinea & CENTRAL GUINEA & 80-84 & 266.06 & 239.46 & 294.61 & RW2 \\ 
  Guinea & CENTRAL GUINEA & 85-89 & 220.23 & 243.53 & 198.57 & HT-Direct \\ 
  Guinea & CENTRAL GUINEA & 85-89 & 237.83 & 219.77 & 256.43 & RW2 \\ 
  Guinea & CENTRAL GUINEA & 90-94 & 206.07 & 223.56 & 189.62 & HT-Direct \\ 
  Guinea & CENTRAL GUINEA & 90-94 & 213.22 & 199.49 & 227.47 & RW2 \\ 
  Guinea & CENTRAL GUINEA & 95-99 & 182.20 & 198.39 & 167.07 & HT-Direct \\ 
  Guinea & CENTRAL GUINEA & 95-99 & 186.64 & 174.83 & 199.18 & RW2 \\ 
  Guinea & CENTRAL GUINEA & 00-04 & 163.77 & 176.99 & 151.35 & HT-Direct \\ 
  Guinea & CENTRAL GUINEA & 00-04 & 159.14 & 149.08 & 169.95 & RW2 \\ 
  Guinea & CENTRAL GUINEA & 05-09 & 139.19 & 165.35 & 116.58 & HT-Direct \\ 
  Guinea & CENTRAL GUINEA & 05-09 & 129.50 & 115.52 & 145.02 & RW2 \\ 
  Guinea & CENTRAL GUINEA & 10-14 & 98.27 & 131.01 & 73.02 & HT-Direct \\ 
  Guinea & CENTRAL GUINEA & 10-14 & 101.78 & 81.02 & 126.00 & RW2 \\ 
  Guinea & CENTRAL GUINEA & 15-19 & 78.69 & 32.32 & 176.51 & RW2 \\ 
  Guinea & CONAKRY & 80-84 & 209.49 & 248.24 & 175.38 & HT-Direct \\ 
  Guinea & CONAKRY & 80-84 & 215.12 & 183.40 & 250.92 & RW2 \\ 
  Guinea & CONAKRY & 85-89 & 162.95 & 190.09 & 139.02 & HT-Direct \\ 
  Guinea & CONAKRY & 85-89 & 176.29 & 156.85 & 197.57 & RW2 \\ 
  Guinea & CONAKRY & 90-94 & 147.80 & 170.04 & 128.03 & HT-Direct \\ 
  Guinea & CONAKRY & 90-94 & 144.29 & 129.70 & 160.50 & RW2 \\ 
  Guinea & CONAKRY & 95-99 & 109.77 & 126.99 & 94.63 & HT-Direct \\ 
  Guinea & CONAKRY & 95-99 & 112.83 & 100.18 & 126.48 & RW2 \\ 
  Guinea & CONAKRY & 00-04 & 63.38 & 86.06 & 46.38 & HT-Direct \\ 
  Guinea & CONAKRY & 00-04 & 87.36 & 73.91 & 102.42 & RW2 \\ 
  Guinea & CONAKRY & 05-09 & 88.62 & 123.66 & 62.80 & HT-Direct \\ 
  Guinea & CONAKRY & 05-09 & 68.52 & 54.75 & 85.52 & RW2 \\ 
  Guinea & CONAKRY & 10-14 & 56.44 & 100.38 & 31.07 & HT-Direct \\ 
  Guinea & CONAKRY & 10-14 & 51.52 & 34.64 & 77.16 & RW2 \\ 
  Guinea & CONAKRY & 15-19 & 38.09 & 12.90 & 108.01 & RW2 \\ 
  Guinea & FOREST GUINEA & 80-84 & 275.95 & 305.42 & 248.31 & HT-Direct \\ 
  Guinea & FOREST GUINEA & 80-84 & 288.06 & 259.96 & 317.83 & RW2 \\ 
  Guinea & FOREST GUINEA & 85-89 & 271.89 & 298.40 & 246.92 & HT-Direct \\ 
  Guinea & FOREST GUINEA & 85-89 & 290.70 & 269.42 & 313.23 & RW2 \\ 
  Guinea & FOREST GUINEA & 90-94 & 270.30 & 292.17 & 249.49 & HT-Direct \\ 
  Guinea & FOREST GUINEA & 90-94 & 270.41 & 252.94 & 289.05 & RW2 \\ 
  Guinea & FOREST GUINEA & 95-99 & 211.77 & 229.46 & 195.10 & HT-Direct \\ 
  Guinea & FOREST GUINEA & 95-99 & 217.86 & 203.69 & 233.09 & RW2 \\ 
  Guinea & FOREST GUINEA & 00-04 & 167.40 & 185.54 & 150.70 & HT-Direct \\ 
  Guinea & FOREST GUINEA & 00-04 & 162.06 & 148.87 & 175.98 & RW2 \\ 
  Guinea & FOREST GUINEA & 05-09 & 112.01 & 139.83 & 89.15 & HT-Direct \\ 
  Guinea & FOREST GUINEA & 05-09 & 119.51 & 103.97 & 136.41 & RW2 \\ 
  Guinea & FOREST GUINEA & 10-14 & 101.43 & 128.31 & 79.66 & HT-Direct \\ 
  Guinea & FOREST GUINEA & 10-14 & 93.99 & 76.40 & 114.98 & RW2 \\ 
  Guinea & FOREST GUINEA & 15-19 & 75.30 & 31.31 & 171.68 & RW2 \\ 
  Guinea & LOWER GUINEA & 80-84 & 272.29 & 305.73 & 241.25 & HT-Direct \\ 
  Guinea & LOWER GUINEA & 80-84 & 276.23 & 248.00 & 307.04 & RW2 \\ 
  Guinea & LOWER GUINEA & 85-89 & 229.17 & 250.90 & 208.79 & HT-Direct \\ 
  Guinea & LOWER GUINEA & 85-89 & 243.17 & 225.06 & 261.86 & RW2 \\ 
  Guinea & LOWER GUINEA & 90-94 & 206.08 & 224.88 & 188.47 & HT-Direct \\ 
  Guinea & LOWER GUINEA & 90-94 & 216.37 & 201.61 & 231.60 & RW2 \\ 
  Guinea & LOWER GUINEA & 95-99 & 184.50 & 201.52 & 168.62 & HT-Direct \\ 
  Guinea & LOWER GUINEA & 95-99 & 188.51 & 175.72 & 202.25 & RW2 \\ 
  Guinea & LOWER GUINEA & 00-04 & 164.27 & 181.69 & 148.21 & HT-Direct \\ 
  Guinea & LOWER GUINEA & 00-04 & 157.21 & 145.01 & 170.62 & RW2 \\ 
  Guinea & LOWER GUINEA & 05-09 & 135.99 & 167.08 & 109.93 & HT-Direct \\ 
  Guinea & LOWER GUINEA & 05-09 & 121.07 & 105.83 & 138.14 & RW2 \\ 
  Guinea & LOWER GUINEA & 10-14 & 72.30 & 102.61 & 50.44 & HT-Direct \\ 
  Guinea & LOWER GUINEA & 10-14 & 85.64 & 64.66 & 110.86 & RW2 \\ 
  Guinea & LOWER GUINEA & 15-19 & 58.40 & 22.40 & 140.62 & RW2 \\ 
  Guinea & UPPER GUINEA & 80-84 & 308.94 & 344.20 & 275.77 & HT-Direct \\ 
  Guinea & UPPER GUINEA & 80-84 & 314.90 & 284.19 & 347.63 & RW2 \\ 
  Guinea & UPPER GUINEA & 85-89 & 277.30 & 303.72 & 252.34 & HT-Direct \\ 
  Guinea & UPPER GUINEA & 85-89 & 286.40 & 266.33 & 307.90 & RW2 \\ 
  Guinea & UPPER GUINEA & 90-94 & 245.07 & 264.09 & 226.99 & HT-Direct \\ 
  Guinea & UPPER GUINEA & 90-94 & 251.76 & 236.81 & 267.15 & RW2 \\ 
  Guinea & UPPER GUINEA & 95-99 & 204.03 & 218.50 & 190.29 & HT-Direct \\ 
  Guinea & UPPER GUINEA & 95-99 & 212.02 & 199.72 & 224.68 & RW2 \\ 
  Guinea & UPPER GUINEA & 00-04 & 180.16 & 197.54 & 163.99 & HT-Direct \\ 
  Guinea & UPPER GUINEA & 00-04 & 183.84 & 171.19 & 197.10 & RW2 \\ 
  Guinea & UPPER GUINEA & 05-09 & 194.77 & 222.05 & 170.11 & HT-Direct \\ 
  Guinea & UPPER GUINEA & 05-09 & 169.26 & 153.44 & 186.47 & RW2 \\ 
  Guinea & UPPER GUINEA & 10-14 & 174.31 & 212.79 & 141.55 & HT-Direct \\ 
  Guinea & UPPER GUINEA & 10-14 & 163.51 & 137.31 & 194.68 & RW2 \\ 
  Guinea & UPPER GUINEA & 15-19 & 159.61 & 72.09 & 318.82 & RW2 \\ 
  Kenya & ALL & 80-84 & 101.94 & 100.33 & 103.62 & IHME \\ 
  Kenya & ALL & 80-84 & 102.59 & 95.32 & 110.36 & RW2 \\ 
  Kenya & ALL & 80-84 & 102.57 & 99.81 & 105.46 & UN \\ 
  Kenya & ALL & 85-89 & 93.67 & 92.16 & 95.14 & IHME \\ 
  Kenya & ALL & 85-89 & 97.40 & 91.44 & 103.65 & RW2 \\ 
  Kenya & ALL & 85-89 & 97.47 & 94.70 & 100.12 & UN \\ 
  Kenya & ALL & 90-94 & 93.48 & 91.93 & 95.02 & IHME \\ 
  Kenya & ALL & 90-94 & 108.35 & 102.29 & 114.75 & RW2 \\ 
  Kenya & ALL & 90-94 & 108.12 & 105.07 & 111.37 & UN \\ 
  Kenya & ALL & 95-99 & 93.71 & 92.03 & 95.27 & IHME \\ 
  Kenya & ALL & 95-99 & 113.65 & 106.96 & 120.65 & RW2 \\ 
  Kenya & ALL & 95-99 & 113.99 & 110.54 & 117.42 & UN \\ 
  Kenya & ALL & 00-04 & 81.88 & 80.14 & 83.73 & IHME \\ 
  Kenya & ALL & 00-04 & 100.00 & 93.65 & 106.79 & RW2 \\ 
  Kenya & ALL & 00-04 & 99.71 & 96.14 & 103.53 & UN \\ 
  Kenya & ALL & 05-09 & 63.75 & 62.16 & 65.45 & IHME \\ 
  Kenya & ALL & 05-09 & 75.16 & 68.68 & 82.16 & RW2 \\ 
  Kenya & ALL & 05-09 & 75.42 & 71.54 & 79.43 & UN \\ 
  Kenya & ALL & 10-14 & 52.82 & 50.68 & 54.98 & IHME \\ 
  Kenya & ALL & 10-14 & 56.44 & 51.21 & 62.13 & RW2 \\ 
  Kenya & ALL & 10-14 & 56.31 & 51.59 & 61.06 & UN \\ 
  Kenya & CENTRAL & 80-84 & 48.15 & 61.80 & 37.40 & HT-Direct \\ 
  Kenya & CENTRAL & 80-84 & 49.10 & 39.56 & 60.98 & RW2 \\ 
  Kenya & CENTRAL & 85-89 & 42.31 & 53.90 & 33.13 & HT-Direct \\ 
  Kenya & CENTRAL & 85-89 & 44.01 & 37.59 & 51.41 & RW2 \\ 
  Kenya & CENTRAL & 90-94 & 49.57 & 60.73 & 40.38 & HT-Direct \\ 
  Kenya & CENTRAL & 90-94 & 53.85 & 46.56 & 62.17 & RW2 \\ 
  Kenya & CENTRAL & 95-99 & 56.26 & 71.49 & 44.12 & HT-Direct \\ 
  Kenya & CENTRAL & 95-99 & 68.02 & 57.99 & 79.43 & RW2 \\ 
  Kenya & CENTRAL & 00-04 & 60.07 & 76.93 & 46.71 & HT-Direct \\ 
  Kenya & CENTRAL & 00-04 & 67.01 & 56.52 & 79.36 & RW2 \\ 
  Kenya & CENTRAL & 05-09 & 52.10 & 68.95 & 39.20 & HT-Direct \\ 
  Kenya & CENTRAL & 05-09 & 56.51 & 45.88 & 69.35 & RW2 \\ 
  Kenya & CENTRAL & 10-14 & 40.79 & 57.56 & 28.75 & HT-Direct \\ 
  Kenya & CENTRAL & 10-14 & 47.71 & 35.53 & 64.07 & RW2 \\ 
  Kenya & CENTRAL & 15-19 & 40.82 & 14.93 & 106.86 & RW2 \\ 
  Kenya & COAST & 80-84 & 156.99 & 181.03 & 135.62 & HT-Direct \\ 
  Kenya & COAST & 80-84 & 154.42 & 135.21 & 176.20 & RW2 \\ 
  Kenya & COAST & 85-89 & 105.60 & 120.65 & 92.24 & HT-Direct \\ 
  Kenya & COAST & 85-89 & 116.38 & 104.89 & 128.81 & RW2 \\ 
  Kenya & COAST & 90-94 & 107.92 & 122.93 & 94.54 & HT-Direct \\ 
  Kenya & COAST & 90-94 & 120.41 & 108.61 & 133.04 & RW2 \\ 
  Kenya & COAST & 95-99 & 137.79 & 163.20 & 115.79 & HT-Direct \\ 
  Kenya & COAST & 95-99 & 128.56 & 113.73 & 145.22 & RW2 \\ 
  Kenya & COAST & 00-04 & 80.76 & 97.76 & 66.49 & HT-Direct \\ 
  Kenya & COAST & 00-04 & 106.07 & 92.52 & 121.30 & RW2 \\ 
  Kenya & COAST & 05-09 & 66.19 & 79.86 & 54.73 & HT-Direct \\ 
  Kenya & COAST & 05-09 & 76.81 & 65.06 & 90.40 & RW2 \\ 
  Kenya & COAST & 10-14 & 54.96 & 74.98 & 40.06 & HT-Direct \\ 
  Kenya & COAST & 10-14 & 56.99 & 43.99 & 74.00 & RW2 \\ 
  Kenya & COAST & 15-19 & 42.81 & 16.01 & 110.65 & RW2 \\ 
  Kenya & EASTERN & 80-84 & 75.07 & 88.74 & 63.37 & HT-Direct \\ 
  Kenya & EASTERN & 80-84 & 76.36 & 65.74 & 88.73 & RW2 \\ 
  Kenya & EASTERN & 85-89 & 62.33 & 73.67 & 52.64 & HT-Direct \\ 
  Kenya & EASTERN & 85-89 & 64.77 & 57.59 & 72.76 & RW2 \\ 
  Kenya & EASTERN & 90-94 & 64.04 & 74.95 & 54.62 & HT-Direct \\ 
  Kenya & EASTERN & 90-94 & 75.33 & 67.43 & 84.03 & RW2 \\ 
  Kenya & EASTERN & 95-99 & 85.41 & 99.59 & 73.09 & HT-Direct \\ 
  Kenya & EASTERN & 95-99 & 89.90 & 79.98 & 100.96 & RW2 \\ 
  Kenya & EASTERN & 00-04 & 76.93 & 91.16 & 64.77 & HT-Direct \\ 
  Kenya & EASTERN & 00-04 & 81.91 & 71.94 & 93.13 & RW2 \\ 
  Kenya & EASTERN & 05-09 & 45.79 & 58.05 & 36.01 & HT-Direct \\ 
  Kenya & EASTERN & 05-09 & 63.68 & 53.82 & 75.12 & RW2 \\ 
  Kenya & EASTERN & 10-14 & 46.60 & 60.62 & 35.69 & HT-Direct \\ 
  Kenya & EASTERN & 10-14 & 50.32 & 39.94 & 63.33 & RW2 \\ 
  Kenya & EASTERN & 15-19 & 40.38 & 15.48 & 100.64 & RW2 \\ 
  Kenya & NAIROBI & 80-84 & 66.08 & 97.47 & 44.31 & HT-Direct \\ 
  Kenya & NAIROBI & 80-84 & 68.51 & 50.57 & 92.56 & RW2 \\ 
  Kenya & NAIROBI & 85-89 & 60.97 & 83.52 & 44.22 & HT-Direct \\ 
  Kenya & NAIROBI & 85-89 & 63.12 & 51.27 & 77.58 & RW2 \\ 
  Kenya & NAIROBI & 90-94 & 72.91 & 97.05 & 54.41 & HT-Direct \\ 
  Kenya & NAIROBI & 90-94 & 78.40 & 65.23 & 93.90 & RW2 \\ 
  Kenya & NAIROBI & 95-99 & 102.28 & 140.88 & 73.35 & HT-Direct \\ 
  Kenya & NAIROBI & 95-99 & 99.07 & 81.91 & 119.29 & RW2 \\ 
  Kenya & NAIROBI & 00-04 & 72.78 & 97.93 & 53.70 & HT-Direct \\ 
  Kenya & NAIROBI & 00-04 & 97.52 & 79.84 & 118.40 & RW2 \\ 
  Kenya & NAIROBI & 05-09 & 71.39 & 102.08 & 49.42 & HT-Direct \\ 
  Kenya & NAIROBI & 05-09 & 83.28 & 66.06 & 104.46 & RW2 \\ 
  Kenya & NAIROBI & 10-14 & 70.98 & 99.76 & 50.04 & HT-Direct \\ 
  Kenya & NAIROBI & 10-14 & 72.15 & 52.98 & 97.66 & RW2 \\ 
  Kenya & NAIROBI & 15-19 & 62.98 & 23.32 & 160.58 & RW2 \\ 
  Kenya & NORTHEASTERN & 80-84 & 219.17 & 312.84 & 147.53 & HT-Direct \\ 
  Kenya & NORTHEASTERN & 80-84 & 209.80 & 157.73 & 271.51 & RW2 \\ 
  Kenya & NORTHEASTERN & 85-89 & 117.52 & 150.44 & 91.04 & HT-Direct \\ 
  Kenya & NORTHEASTERN & 85-89 & 155.88 & 130.77 & 184.21 & RW2 \\ 
  Kenya & NORTHEASTERN & 90-94 & 169.72 & 200.82 & 142.57 & HT-Direct \\ 
  Kenya & NORTHEASTERN & 90-94 & 153.08 & 133.77 & 175.08 & RW2 \\ 
  Kenya & NORTHEASTERN & 95-99 & 133.24 & 162.55 & 108.54 & HT-Direct \\ 
  Kenya & NORTHEASTERN & 95-99 & 146.68 & 127.01 & 169.07 & RW2 \\ 
  Kenya & NORTHEASTERN & 00-04 & 77.82 & 97.41 & 61.90 & HT-Direct \\ 
  Kenya & NORTHEASTERN & 00-04 & 106.89 & 90.18 & 125.51 & RW2 \\ 
  Kenya & NORTHEASTERN & 05-09 & 56.48 & 74.33 & 42.71 & HT-Direct \\ 
  Kenya & NORTHEASTERN & 05-09 & 69.20 & 56.38 & 84.57 & RW2 \\ 
  Kenya & NORTHEASTERN & 10-14 & 47.82 & 65.18 & 34.90 & HT-Direct \\ 
  Kenya & NORTHEASTERN & 10-14 & 46.84 & 35.28 & 62.26 & RW2 \\ 
  Kenya & NORTHEASTERN & 15-19 & 32.16 & 11.68 & 87.41 & RW2 \\ 
  Kenya & NYANZA & 80-84 & 151.29 & 173.65 & 131.35 & HT-Direct \\ 
  Kenya & NYANZA & 80-84 & 163.28 & 142.92 & 185.21 & RW2 \\ 
  Kenya & NYANZA & 85-89 & 169.65 & 187.37 & 153.30 & HT-Direct \\ 
  Kenya & NYANZA & 85-89 & 174.04 & 159.89 & 189.36 & RW2 \\ 
  Kenya & NYANZA & 90-94 & 203.88 & 226.12 & 183.32 & HT-Direct \\ 
  Kenya & NYANZA & 90-94 & 217.41 & 199.74 & 236.37 & RW2 \\ 
  Kenya & NYANZA & 95-99 & 216.33 & 244.19 & 190.84 & HT-Direct \\ 
  Kenya & NYANZA & 95-99 & 250.29 & 227.06 & 275.19 & RW2 \\ 
  Kenya & NYANZA & 00-04 & 205.36 & 236.89 & 177.06 & HT-Direct \\ 
  Kenya & NYANZA & 00-04 & 215.27 & 191.98 & 241.26 & RW2 \\ 
  Kenya & NYANZA & 05-09 & 116.34 & 135.57 & 99.53 & HT-Direct \\ 
  Kenya & NYANZA & 05-09 & 139.38 & 121.77 & 159.33 & RW2 \\ 
  Kenya & NYANZA & 10-14 & 69.00 & 83.34 & 56.97 & HT-Direct \\ 
  Kenya & NYANZA & 10-14 & 79.02 & 65.33 & 95.20 & RW2 \\ 
  Kenya & NYANZA & 15-19 & 42.77 & 16.47 & 106.57 & RW2 \\ 
  Kenya & RIFT VALLEY & 80-84 & 79.34 & 93.04 & 67.50 & HT-Direct \\ 
  Kenya & RIFT VALLEY & 80-84 & 79.03 & 68.62 & 90.88 & RW2 \\ 
  Kenya & RIFT VALLEY & 85-89 & 61.90 & 71.65 & 53.40 & HT-Direct \\ 
  Kenya & RIFT VALLEY & 85-89 & 67.94 & 61.09 & 75.30 & RW2 \\ 
  Kenya & RIFT VALLEY & 90-94 & 75.98 & 86.61 & 66.56 & HT-Direct \\ 
  Kenya & RIFT VALLEY & 90-94 & 77.85 & 70.61 & 85.62 & RW2 \\ 
  Kenya & RIFT VALLEY & 95-99 & 73.32 & 85.01 & 63.13 & HT-Direct \\ 
  Kenya & RIFT VALLEY & 95-99 & 90.23 & 80.93 & 100.31 & RW2 \\ 
  Kenya & RIFT VALLEY & 00-04 & 71.86 & 82.30 & 62.65 & HT-Direct \\ 
  Kenya & RIFT VALLEY & 00-04 & 81.50 & 73.09 & 90.70 & RW2 \\ 
  Kenya & RIFT VALLEY & 05-09 & 49.53 & 57.60 & 42.54 & HT-Direct \\ 
  Kenya & RIFT VALLEY & 05-09 & 62.18 & 54.74 & 70.53 & RW2 \\ 
  Kenya & RIFT VALLEY & 10-14 & 45.68 & 53.97 & 38.61 & HT-Direct \\ 
  Kenya & RIFT VALLEY & 10-14 & 47.21 & 40.30 & 55.36 & RW2 \\ 
  Kenya & RIFT VALLEY & 15-19 & 35.99 & 14.43 & 88.01 & RW2 \\ 
  Kenya & WESTERN & 80-84 & 119.11 & 139.54 & 101.32 & HT-Direct \\ 
  Kenya & WESTERN & 80-84 & 124.21 & 107.17 & 143.54 & RW2 \\ 
  Kenya & WESTERN & 85-89 & 118.37 & 135.87 & 102.85 & HT-Direct \\ 
  Kenya & WESTERN & 85-89 & 124.33 & 111.99 & 138.08 & RW2 \\ 
  Kenya & WESTERN & 90-94 & 145.57 & 163.39 & 129.40 & HT-Direct \\ 
  Kenya & WESTERN & 90-94 & 152.43 & 138.58 & 167.49 & RW2 \\ 
  Kenya & WESTERN & 95-99 & 143.68 & 166.33 & 123.66 & HT-Direct \\ 
  Kenya & WESTERN & 95-99 & 177.16 & 158.68 & 197.25 & RW2 \\ 
  Kenya & WESTERN & 00-04 & 148.67 & 171.88 & 128.11 & HT-Direct \\ 
  Kenya & WESTERN & 00-04 & 155.27 & 137.37 & 175.84 & RW2 \\ 
  Kenya & WESTERN & 05-09 & 90.07 & 112.17 & 71.97 & HT-Direct \\ 
  Kenya & WESTERN & 05-09 & 103.70 & 88.49 & 121.27 & RW2 \\ 
  Kenya & WESTERN & 10-14 & 53.89 & 66.20 & 43.76 & HT-Direct \\ 
  Kenya & WESTERN & 10-14 & 61.85 & 50.28 & 75.48 & RW2 \\ 
  Kenya & WESTERN & 15-19 & 35.84 & 13.69 & 89.26 & RW2 \\ 
  Lesotho & ALL & 80-84 & 98.56 & 96.09 & 101.23 & IHME \\ 
  Lesotho & ALL & 80-84 & 109.45 & 88.41 & 134.86 & RW2 \\ 
  Lesotho & ALL & 80-84 & 109.69 & 105.00 & 115.30 & UN \\ 
  Lesotho & ALL & 85-89 & 94.41 & 92.09 & 96.83 & IHME \\ 
  Lesotho & ALL & 85-89 & 91.81 & 80.26 & 104.69 & RW2 \\ 
  Lesotho & ALL & 85-89 & 92.68 & 88.60 & 96.91 & UN \\ 
  Lesotho & ALL & 90-94 & 90.80 & 88.68 & 92.82 & IHME \\ 
  Lesotho & ALL & 90-94 & 91.73 & 83.05 & 101.22 & RW2 \\ 
  Lesotho & ALL & 90-94 & 90.01 & 86.45 & 93.80 & UN \\ 
  Lesotho & ALL & 95-99 & 97.19 & 95.08 & 99.45 & IHME \\ 
  Lesotho & ALL & 95-99 & 104.95 & 92.86 & 118.22 & RW2 \\ 
  Lesotho & ALL & 95-99 & 107.25 & 103.63 & 111.67 & UN \\ 
  Lesotho & ALL & 00-04 & 112.34 & 109.67 & 115.04 & IHME \\ 
  Lesotho & ALL & 00-04 & 121.07 & 111.35 & 131.59 & RW2 \\ 
  Lesotho & ALL & 00-04 & 120.24 & 115.87 & 124.57 & UN \\ 
  Lesotho & ALL & 05-09 & 116.61 & 112.84 & 120.42 & IHME \\ 
  Lesotho & ALL & 05-09 & 117.67 & 106.51 & 129.85 & RW2 \\ 
  Lesotho & ALL & 05-09 & 117.97 & 113.03 & 123.21 & UN \\ 
  Lesotho & ALL & 10-14 & 99.72 & 94.23 & 106.12 & IHME \\ 
  Lesotho & ALL & 10-14 & 95.68 & 81.15 & 112.32 & RW2 \\ 
  Lesotho & ALL & 10-14 & 95.61 & 88.80 & 103.23 & UN \\ 
  Lesotho & BEREA & 80-84 & 156.27 & 216.69 & 110.33 & HT-Direct \\ 
  Lesotho & BEREA & 80-84 & 153.18 & 110.39 & 208.38 & RW2 \\ 
  Lesotho & BEREA & 85-89 & 78.06 & 112.50 & 53.53 & HT-Direct \\ 
  Lesotho & BEREA & 85-89 & 113.26 & 88.62 & 143.50 & RW2 \\ 
  Lesotho & BEREA & 90-94 & 102.41 & 135.41 & 76.74 & HT-Direct \\ 
  Lesotho & BEREA & 90-94 & 103.95 & 86.38 & 124.33 & RW2 \\ 
  Lesotho & BEREA & 95-99 & 84.42 & 114.54 & 61.67 & HT-Direct \\ 
  Lesotho & BEREA & 95-99 & 107.93 & 90.61 & 127.98 & RW2 \\ 
  Lesotho & BEREA & 00-04 & 106.50 & 133.82 & 84.21 & HT-Direct \\ 
  Lesotho & BEREA & 00-04 & 109.00 & 93.36 & 127.13 & RW2 \\ 
  Lesotho & BEREA & 05-09 & 92.54 & 122.03 & 69.61 & HT-Direct \\ 
  Lesotho & BEREA & 05-09 & 100.89 & 83.03 & 121.89 & RW2 \\ 
  Lesotho & BEREA & 10-14 & 75.48 & 111.98 & 50.20 & HT-Direct \\ 
  Lesotho & BEREA & 10-14 & 75.59 & 56.37 & 100.65 & RW2 \\ 
  Lesotho & BEREA & 15-19 & 52.49 & 19.26 & 132.79 & RW2 \\ 
  Lesotho & BUTHA-BUTHE & 80-84 & 57.90 & 104.19 & 31.46 & HT-Direct \\ 
  Lesotho & BUTHA-BUTHE & 80-84 & 91.15 & 59.08 & 138.08 & RW2 \\ 
  Lesotho & BUTHA-BUTHE & 85-89 & 70.92 & 105.47 & 47.09 & HT-Direct \\ 
  Lesotho & BUTHA-BUTHE & 85-89 & 73.80 & 55.05 & 98.40 & RW2 \\ 
  Lesotho & BUTHA-BUTHE & 90-94 & 71.81 & 96.96 & 52.81 & HT-Direct \\ 
  Lesotho & BUTHA-BUTHE & 90-94 & 74.65 & 60.29 & 91.88 & RW2 \\ 
  Lesotho & BUTHA-BUTHE & 95-99 & 75.18 & 110.59 & 50.46 & HT-Direct \\ 
  Lesotho & BUTHA-BUTHE & 95-99 & 84.79 & 69.42 & 103.23 & RW2 \\ 
  Lesotho & BUTHA-BUTHE & 00-04 & 84.13 & 110.09 & 63.86 & HT-Direct \\ 
  Lesotho & BUTHA-BUTHE & 00-04 & 93.41 & 78.32 & 111.15 & RW2 \\ 
  Lesotho & BUTHA-BUTHE & 05-09 & 90.93 & 121.32 & 67.57 & HT-Direct \\ 
  Lesotho & BUTHA-BUTHE & 05-09 & 94.77 & 76.29 & 117.09 & RW2 \\ 
  Lesotho & BUTHA-BUTHE & 10-14 & 70.91 & 116.07 & 42.48 & HT-Direct \\ 
  Lesotho & BUTHA-BUTHE & 10-14 & 78.31 & 55.74 & 109.18 & RW2 \\ 
  Lesotho & BUTHA-BUTHE & 15-19 & 60.50 & 21.31 & 156.43 & RW2 \\ 
  Lesotho & LERIBE & 80-84 & 99.07 & 168.98 & 56.13 & HT-Direct \\ 
  Lesotho & LERIBE & 80-84 & 102.20 & 70.07 & 146.76 & RW2 \\ 
  Lesotho & LERIBE & 85-89 & 86.06 & 117.01 & 62.72 & HT-Direct \\ 
  Lesotho & LERIBE & 85-89 & 85.57 & 66.59 & 109.38 & RW2 \\ 
  Lesotho & LERIBE & 90-94 & 73.28 & 97.22 & 54.87 & HT-Direct \\ 
  Lesotho & LERIBE & 90-94 & 89.64 & 74.32 & 107.65 & RW2 \\ 
  Lesotho & LERIBE & 95-99 & 86.06 & 119.76 & 61.19 & HT-Direct \\ 
  Lesotho & LERIBE & 95-99 & 106.45 & 89.35 & 126.16 & RW2 \\ 
  Lesotho & LERIBE & 00-04 & 121.32 & 152.78 & 95.62 & HT-Direct \\ 
  Lesotho & LERIBE & 00-04 & 122.65 & 105.59 & 141.72 & RW2 \\ 
  Lesotho & LERIBE & 05-09 & 130.16 & 164.10 & 102.38 & HT-Direct \\ 
  Lesotho & LERIBE & 05-09 & 129.72 & 108.88 & 154.00 & RW2 \\ 
  Lesotho & LERIBE & 10-14 & 97.98 & 138.70 & 68.27 & HT-Direct \\ 
  Lesotho & LERIBE & 10-14 & 112.14 & 86.24 & 145.19 & RW2 \\ 
  Lesotho & LERIBE & 15-19 & 90.67 & 34.51 & 213.69 & RW2 \\ 
  Lesotho & MAFETENG & 80-84 & 59.65 & 107.92 & 32.19 & HT-Direct \\ 
  Lesotho & MAFETENG & 80-84 & 97.74 & 64.03 & 147.47 & RW2 \\ 
  Lesotho & MAFETENG & 85-89 & 90.78 & 128.73 & 63.21 & HT-Direct \\ 
  Lesotho & MAFETENG & 85-89 & 81.04 & 60.70 & 107.68 & RW2 \\ 
  Lesotho & MAFETENG & 90-94 & 78.96 & 110.97 & 55.60 & HT-Direct \\ 
  Lesotho & MAFETENG & 90-94 & 83.98 & 67.45 & 104.10 & RW2 \\ 
  Lesotho & MAFETENG & 95-99 & 68.47 & 103.28 & 44.80 & HT-Direct \\ 
  Lesotho & MAFETENG & 95-99 & 98.47 & 80.59 & 119.35 & RW2 \\ 
  Lesotho & MAFETENG & 00-04 & 96.49 & 125.32 & 73.74 & HT-Direct \\ 
  Lesotho & MAFETENG & 00-04 & 112.44 & 95.13 & 132.30 & RW2 \\ 
  Lesotho & MAFETENG & 05-09 & 121.57 & 151.84 & 96.64 & HT-Direct \\ 
  Lesotho & MAFETENG & 05-09 & 117.97 & 97.54 & 141.97 & RW2 \\ 
  Lesotho & MAFETENG & 10-14 & 89.59 & 155.58 & 49.93 & HT-Direct \\ 
  Lesotho & MAFETENG & 10-14 & 100.54 & 72.73 & 137.91 & RW2 \\ 
  Lesotho & MAFETENG & 15-19 & 80.22 & 28.23 & 200.48 & RW2 \\ 
  Lesotho & MASERU & 80-84 & 50.39 & 82.60 & 30.32 & HT-Direct \\ 
  Lesotho & MASERU & 80-84 & 76.38 & 52.27 & 109.44 & RW2 \\ 
  Lesotho & MASERU & 85-89 & 52.49 & 77.12 & 35.42 & HT-Direct \\ 
  Lesotho & MASERU & 85-89 & 68.27 & 52.47 & 88.01 & RW2 \\ 
  Lesotho & MASERU & 90-94 & 81.30 & 105.52 & 62.25 & HT-Direct \\ 
  Lesotho & MASERU & 90-94 & 76.65 & 63.90 & 91.68 & RW2 \\ 
  Lesotho & MASERU & 95-99 & 74.79 & 96.62 & 57.57 & HT-Direct \\ 
  Lesotho & MASERU & 95-99 & 96.44 & 81.99 & 113.17 & RW2 \\ 
  Lesotho & MASERU & 00-04 & 119.31 & 144.32 & 98.14 & HT-Direct \\ 
  Lesotho & MASERU & 00-04 & 116.47 & 101.39 & 133.61 & RW2 \\ 
  Lesotho & MASERU & 05-09 & 111.96 & 147.47 & 84.16 & HT-Direct \\ 
  Lesotho & MASERU & 05-09 & 127.45 & 106.63 & 151.96 & RW2 \\ 
  Lesotho & MASERU & 10-14 & 96.50 & 135.63 & 67.78 & HT-Direct \\ 
  Lesotho & MASERU & 10-14 & 113.38 & 86.83 & 146.47 & RW2 \\ 
  Lesotho & MASERU & 15-19 & 94.18 & 35.71 & 221.44 & RW2 \\ 
  Lesotho & MOHALE'S HOEK & 80-84 & 126.37 & 190.23 & 81.78 & HT-Direct \\ 
  Lesotho & MOHALE'S HOEK & 80-84 & 132.29 & 94.49 & 181.80 & RW2 \\ 
  Lesotho & MOHALE'S HOEK & 85-89 & 96.17 & 128.87 & 71.09 & HT-Direct \\ 
  Lesotho & MOHALE'S HOEK & 85-89 & 107.06 & 84.85 & 134.33 & RW2 \\ 
  Lesotho & MOHALE'S HOEK & 90-94 & 87.98 & 114.58 & 67.09 & HT-Direct \\ 
  Lesotho & MOHALE'S HOEK & 90-94 & 108.00 & 91.02 & 127.84 & RW2 \\ 
  Lesotho & MOHALE'S HOEK & 95-99 & 120.89 & 157.57 & 91.82 & HT-Direct \\ 
  Lesotho & MOHALE'S HOEK & 95-99 & 122.92 & 104.65 & 144.21 & RW2 \\ 
  Lesotho & MOHALE'S HOEK & 00-04 & 137.37 & 173.17 & 108.00 & HT-Direct \\ 
  Lesotho & MOHALE'S HOEK & 00-04 & 135.22 & 116.74 & 156.12 & RW2 \\ 
  Lesotho & MOHALE'S HOEK & 05-09 & 133.38 & 169.68 & 103.88 & HT-Direct \\ 
  Lesotho & MOHALE'S HOEK & 05-09 & 135.86 & 113.37 & 162.11 & RW2 \\ 
  Lesotho & MOHALE'S HOEK & 10-14 & 89.62 & 134.68 & 58.61 & HT-Direct \\ 
  Lesotho & MOHALE'S HOEK & 10-14 & 110.85 & 83.63 & 145.34 & RW2 \\ 
  Lesotho & MOHALE'S HOEK & 15-19 & 84.01 & 31.38 & 201.11 & RW2 \\ 
  Lesotho & MOKHOTLONG & 80-84 & 104.16 & 180.34 & 57.89 & HT-Direct \\ 
  Lesotho & MOKHOTLONG & 80-84 & 148.78 & 101.63 & 210.85 & RW2 \\ 
  Lesotho & MOKHOTLONG & 85-89 & 87.20 & 124.00 & 60.56 & HT-Direct \\ 
  Lesotho & MOKHOTLONG & 85-89 & 115.33 & 89.37 & 146.62 & RW2 \\ 
  Lesotho & MOKHOTLONG & 90-94 & 123.46 & 154.42 & 97.98 & HT-Direct \\ 
  Lesotho & MOKHOTLONG & 90-94 & 110.26 & 92.72 & 130.62 & RW2 \\ 
  Lesotho & MOKHOTLONG & 95-99 & 95.33 & 128.94 & 69.78 & HT-Direct \\ 
  Lesotho & MOKHOTLONG & 95-99 & 117.63 & 99.48 & 138.82 & RW2 \\ 
  Lesotho & MOKHOTLONG & 00-04 & 108.55 & 134.87 & 86.86 & HT-Direct \\ 
  Lesotho & MOKHOTLONG & 00-04 & 121.69 & 104.87 & 140.83 & RW2 \\ 
  Lesotho & MOKHOTLONG & 05-09 & 103.77 & 137.66 & 77.47 & HT-Direct \\ 
  Lesotho & MOKHOTLONG & 05-09 & 116.21 & 96.26 & 139.59 & RW2 \\ 
  Lesotho & MOKHOTLONG & 10-14 & 96.92 & 136.33 & 68.00 & HT-Direct \\ 
  Lesotho & MOKHOTLONG & 10-14 & 90.82 & 68.76 & 119.13 & RW2 \\ 
  Lesotho & MOKHOTLONG & 15-19 & 65.98 & 24.51 & 162.84 & RW2 \\ 
  Lesotho & QASHA'S NEK & 80-84 & 154.47 & 243.34 & 94.02 & HT-Direct \\ 
  Lesotho & QASHA'S NEK & 80-84 & 156.29 & 109.21 & 220.38 & RW2 \\ 
  Lesotho & QASHA'S NEK & 85-89 & 111.82 & 155.81 & 79.09 & HT-Direct \\ 
  Lesotho & QASHA'S NEK & 85-89 & 118.77 & 92.62 & 150.84 & RW2 \\ 
  Lesotho & QASHA'S NEK & 90-94 & 106.19 & 134.42 & 83.32 & HT-Direct \\ 
  Lesotho & QASHA'S NEK & 90-94 & 111.92 & 93.96 & 132.73 & RW2 \\ 
  Lesotho & QASHA'S NEK & 95-99 & 93.77 & 122.96 & 70.95 & HT-Direct \\ 
  Lesotho & QASHA'S NEK & 95-99 & 119.49 & 100.94 & 140.96 & RW2 \\ 
  Lesotho & QASHA'S NEK & 00-04 & 103.49 & 136.54 & 77.73 & HT-Direct \\ 
  Lesotho & QASHA'S NEK & 00-04 & 124.69 & 105.91 & 146.14 & RW2 \\ 
  Lesotho & QASHA'S NEK & 05-09 & 125.92 & 162.66 & 96.53 & HT-Direct \\ 
  Lesotho & QASHA'S NEK & 05-09 & 120.43 & 98.49 & 146.36 & RW2 \\ 
  Lesotho & QASHA'S NEK & 10-14 & 97.20 & 154.55 & 59.64 & HT-Direct \\ 
  Lesotho & QASHA'S NEK & 10-14 & 94.62 & 69.07 & 128.79 & RW2 \\ 
  Lesotho & QASHA'S NEK & 15-19 & 69.23 & 24.92 & 173.64 & RW2 \\ 
  Lesotho & QUTHING & 80-84 & 122.26 & 180.04 & 81.18 & HT-Direct \\ 
  Lesotho & QUTHING & 80-84 & 150.26 & 105.49 & 209.39 & RW2 \\ 
  Lesotho & QUTHING & 85-89 & 76.67 & 113.37 & 51.17 & HT-Direct \\ 
  Lesotho & QUTHING & 85-89 & 117.51 & 91.51 & 149.57 & RW2 \\ 
  Lesotho & QUTHING & 90-94 & 123.26 & 155.31 & 97.07 & HT-Direct \\ 
  Lesotho & QUTHING & 90-94 & 113.68 & 95.17 & 135.25 & RW2 \\ 
  Lesotho & QUTHING & 95-99 & 110.08 & 148.19 & 80.84 & HT-Direct \\ 
  Lesotho & QUTHING & 95-99 & 122.71 & 103.01 & 145.60 & RW2 \\ 
  Lesotho & QUTHING & 00-04 & 108.16 & 141.09 & 82.19 & HT-Direct \\ 
  Lesotho & QUTHING & 00-04 & 128.08 & 108.45 & 150.82 & RW2 \\ 
  Lesotho & QUTHING & 05-09 & 125.70 & 163.64 & 95.56 & HT-Direct \\ 
  Lesotho & QUTHING & 05-09 & 122.71 & 99.82 & 150.12 & RW2 \\ 
  Lesotho & QUTHING & 10-14 & 77.80 & 129.69 & 45.58 & HT-Direct \\ 
  Lesotho & QUTHING & 10-14 & 95.25 & 68.05 & 131.58 & RW2 \\ 
  Lesotho & QUTHING & 15-19 & 68.33 & 24.02 & 174.77 & RW2 \\ 
  Lesotho & THABA-TSEKA & 80-84 & 134.26 & 218.11 & 79.38 & HT-Direct \\ 
  Lesotho & THABA-TSEKA & 80-84 & 166.46 & 115.93 & 233.67 & RW2 \\ 
  Lesotho & THABA-TSEKA & 85-89 & 95.81 & 135.18 & 67.02 & HT-Direct \\ 
  Lesotho & THABA-TSEKA & 85-89 & 123.94 & 95.59 & 159.51 & RW2 \\ 
  Lesotho & THABA-TSEKA & 90-94 & 105.50 & 142.43 & 77.29 & HT-Direct \\ 
  Lesotho & THABA-TSEKA & 90-94 & 113.84 & 94.30 & 136.96 & RW2 \\ 
  Lesotho & THABA-TSEKA & 95-99 & 108.90 & 140.64 & 83.62 & HT-Direct \\ 
  Lesotho & THABA-TSEKA & 95-99 & 117.34 & 99.73 & 137.79 & RW2 \\ 
  Lesotho & THABA-TSEKA & 00-04 & 121.14 & 148.94 & 97.92 & HT-Direct \\ 
  Lesotho & THABA-TSEKA & 00-04 & 116.76 & 101.10 & 134.66 & RW2 \\ 
  Lesotho & THABA-TSEKA & 05-09 & 94.25 & 121.04 & 72.90 & HT-Direct \\ 
  Lesotho & THABA-TSEKA & 05-09 & 105.85 & 87.15 & 127.73 & RW2 \\ 
  Lesotho & THABA-TSEKA & 10-14 & 55.74 & 105.03 & 28.84 & HT-Direct \\ 
  Lesotho & THABA-TSEKA & 10-14 & 77.37 & 56.11 & 105.10 & RW2 \\ 
  Lesotho & THABA-TSEKA & 15-19 & 52.63 & 18.87 & 132.93 & RW2 \\ 
  Liberia & ALL & 80-84 & 236.32 & 232.47 & 239.96 & IHME \\ 
  Liberia & ALL & 80-84 & 237.19 & 207.24 & 270.04 & RW2 \\ 
  Liberia & ALL & 80-84 & 236.87 & 227.22 & 245.59 & UN \\ 
  Liberia & ALL & 85-89 & 227.05 & 223.85 & 230.33 & IHME \\ 
  Liberia & ALL & 85-89 & 242.19 & 222.41 & 262.82 & RW2 \\ 
  Liberia & ALL & 85-89 & 242.85 & 234.47 & 252.21 & UN \\ 
  Liberia & ALL & 90-94 & 220.60 & 216.62 & 224.48 & IHME \\ 
  Liberia & ALL & 90-94 & 254.77 & 238.92 & 271.47 & RW2 \\ 
  Liberia & ALL & 90-94 & 253.53 & 244.13 & 264.03 & UN \\ 
  Liberia & ALL & 95-99 & 195.34 & 192.46 & 198.27 & IHME \\ 
  Liberia & ALL & 95-99 & 214.35 & 200.55 & 228.66 & RW2 \\ 
  Liberia & ALL & 95-99 & 215.18 & 207.46 & 223.06 & UN \\ 
  Liberia & ALL & 00-04 & 146.95 & 144.51 & 149.52 & IHME \\ 
  Liberia & ALL & 00-04 & 157.99 & 147.43 & 169.26 & RW2 \\ 
  Liberia & ALL & 00-04 & 157.48 & 151.80 & 163.44 & UN \\ 
  Liberia & ALL & 05-09 & 106.28 & 104.22 & 108.45 & IHME \\ 
  Liberia & ALL & 05-09 & 107.67 & 96.48 & 119.83 & RW2 \\ 
  Liberia & ALL & 05-09 & 108.36 & 103.63 & 113.29 & UN \\ 
  Liberia & ALL & 10-14 & 83.09 & 80.24 & 85.99 & IHME \\ 
  Liberia & ALL & 10-14 & 81.53 & 71.56 & 92.67 & RW2 \\ 
  Liberia & ALL & 10-14 & 80.99 & 75.01 & 87.22 & UN \\ 
  Liberia & NORTH CENTRAL & 80-84 & 232.00 & 291.44 & 181.58 & HT-Direct \\ 
  Liberia & NORTH CENTRAL & 80-84 & 230.35 & 191.17 & 275.51 & RW2 \\ 
  Liberia & NORTH CENTRAL & 85-89 & 227.02 & 266.10 & 192.18 & HT-Direct \\ 
  Liberia & NORTH CENTRAL & 85-89 & 238.25 & 211.52 & 266.82 & RW2 \\ 
  Liberia & NORTH CENTRAL & 90-94 & 246.92 & 280.07 & 216.51 & HT-Direct \\ 
  Liberia & NORTH CENTRAL & 90-94 & 252.52 & 231.31 & 274.87 & RW2 \\ 
  Liberia & NORTH CENTRAL & 95-99 & 207.51 & 233.97 & 183.32 & HT-Direct \\ 
  Liberia & NORTH CENTRAL & 95-99 & 205.36 & 188.49 & 223.55 & RW2 \\ 
  Liberia & NORTH CENTRAL & 00-04 & 139.79 & 157.90 & 123.44 & HT-Direct \\ 
  Liberia & NORTH CENTRAL & 00-04 & 144.84 & 132.22 & 158.69 & RW2 \\ 
  Liberia & NORTH CENTRAL & 05-09 & 96.03 & 112.78 & 81.53 & HT-Direct \\ 
  Liberia & NORTH CENTRAL & 05-09 & 101.24 & 89.51 & 114.33 & RW2 \\ 
  Liberia & NORTH CENTRAL & 10-14 & 75.31 & 102.44 & 54.92 & HT-Direct \\ 
  Liberia & NORTH CENTRAL & 10-14 & 70.03 & 57.15 & 85.12 & RW2 \\ 
  Liberia & NORTH CENTRAL & 15-19 & 47.77 & 19.01 & 114.85 & RW2 \\ 
  Liberia & NORTH WESTERN & 80-84 & 213.26 & 300.87 & 145.84 & HT-Direct \\ 
  Liberia & NORTH WESTERN & 80-84 & 262.90 & 206.31 & 324.32 & RW2 \\ 
  Liberia & NORTH WESTERN & 85-89 & 273.61 & 323.02 & 229.21 & HT-Direct \\ 
  Liberia & NORTH WESTERN & 85-89 & 284.82 & 249.31 & 322.32 & RW2 \\ 
  Liberia & NORTH WESTERN & 90-94 & 328.52 & 380.27 & 280.61 & HT-Direct \\ 
  Liberia & NORTH WESTERN & 90-94 & 311.89 & 283.41 & 342.48 & RW2 \\ 
  Liberia & NORTH WESTERN & 95-99 & 271.35 & 308.30 & 237.31 & HT-Direct \\ 
  Liberia & NORTH WESTERN & 95-99 & 262.51 & 239.34 & 287.91 & RW2 \\ 
  Liberia & NORTH WESTERN & 00-04 & 172.95 & 199.87 & 148.99 & HT-Direct \\ 
  Liberia & NORTH WESTERN & 00-04 & 190.17 & 172.02 & 210.08 & RW2 \\ 
  Liberia & NORTH WESTERN & 05-09 & 125.63 & 151.83 & 103.40 & HT-Direct \\ 
  Liberia & NORTH WESTERN & 05-09 & 135.92 & 119.87 & 153.85 & RW2 \\ 
  Liberia & NORTH WESTERN & 10-14 & 110.16 & 134.53 & 89.74 & HT-Direct \\ 
  Liberia & NORTH WESTERN & 10-14 & 96.36 & 82.05 & 112.70 & RW2 \\ 
  Liberia & NORTH WESTERN & 15-19 & 67.29 & 27.42 & 154.38 & RW2 \\ 
  Liberia & SOUTH CENTRAL & 80-84 & 242.85 & 286.82 & 203.70 & HT-Direct \\ 
  Liberia & SOUTH CENTRAL & 80-84 & 245.77 & 211.24 & 284.01 & RW2 \\ 
  Liberia & SOUTH CENTRAL & 85-89 & 240.36 & 272.63 & 210.81 & HT-Direct \\ 
  Liberia & SOUTH CENTRAL & 85-89 & 253.02 & 229.26 & 278.46 & RW2 \\ 
  Liberia & SOUTH CENTRAL & 90-94 & 275.74 & 305.83 & 247.56 & HT-Direct \\ 
  Liberia & SOUTH CENTRAL & 90-94 & 267.05 & 247.24 & 287.64 & RW2 \\ 
  Liberia & SOUTH CENTRAL & 95-99 & 202.88 & 225.24 & 182.22 & HT-Direct \\ 
  Liberia & SOUTH CENTRAL & 95-99 & 217.32 & 201.04 & 234.54 & RW2 \\ 
  Liberia & SOUTH CENTRAL & 00-04 & 156.36 & 173.81 & 140.37 & HT-Direct \\ 
  Liberia & SOUTH CENTRAL & 00-04 & 155.26 & 142.59 & 169.02 & RW2 \\ 
  Liberia & SOUTH CENTRAL & 05-09 & 93.08 & 113.93 & 75.72 & HT-Direct \\ 
  Liberia & SOUTH CENTRAL & 05-09 & 110.59 & 97.25 & 125.51 & RW2 \\ 
  Liberia & SOUTH CENTRAL & 10-14 & 93.64 & 121.50 & 71.64 & HT-Direct \\ 
  Liberia & SOUTH CENTRAL & 10-14 & 78.53 & 65.24 & 94.38 & RW2 \\ 
  Liberia & SOUTH CENTRAL & 15-19 & 55.07 & 22.33 & 128.93 & RW2 \\ 
  Liberia & SOUTH EASTERN A & 80-84 & 184.28 & 270.36 & 121.06 & HT-Direct \\ 
  Liberia & SOUTH EASTERN A & 80-84 & 206.47 & 160.95 & 259.81 & RW2 \\ 
  Liberia & SOUTH EASTERN A & 85-89 & 204.74 & 248.18 & 167.21 & HT-Direct \\ 
  Liberia & SOUTH EASTERN A & 85-89 & 220.12 & 190.23 & 253.28 & RW2 \\ 
  Liberia & SOUTH EASTERN A & 90-94 & 258.51 & 305.74 & 216.30 & HT-Direct \\ 
  Liberia & SOUTH EASTERN A & 90-94 & 239.23 & 214.50 & 265.94 & RW2 \\ 
  Liberia & SOUTH EASTERN A & 95-99 & 185.08 & 221.54 & 153.43 & HT-Direct \\ 
  Liberia & SOUTH EASTERN A & 95-99 & 197.69 & 177.75 & 219.61 & RW2 \\ 
  Liberia & SOUTH EASTERN A & 00-04 & 139.61 & 163.62 & 118.62 & HT-Direct \\ 
  Liberia & SOUTH EASTERN A & 00-04 & 144.60 & 129.40 & 161.49 & RW2 \\ 
  Liberia & SOUTH EASTERN A & 05-09 & 111.05 & 135.66 & 90.44 & HT-Direct \\ 
  Liberia & SOUTH EASTERN A & 05-09 & 107.00 & 93.03 & 122.84 & RW2 \\ 
  Liberia & SOUTH EASTERN A & 10-14 & 78.28 & 105.72 & 57.51 & HT-Direct \\ 
  Liberia & SOUTH EASTERN A & 10-14 & 78.64 & 63.86 & 96.05 & RW2 \\ 
  Liberia & SOUTH EASTERN A & 15-19 & 56.95 & 22.77 & 134.56 & RW2 \\ 
  Liberia & SOUTH EASTERN B & 80-84 & 164.40 & 225.27 & 117.49 & HT-Direct \\ 
  Liberia & SOUTH EASTERN B & 80-84 & 174.08 & 135.86 & 222.32 & RW2 \\ 
  Liberia & SOUTH EASTERN B & 85-89 & 191.63 & 227.89 & 159.94 & HT-Direct \\ 
  Liberia & SOUTH EASTERN B & 85-89 & 194.06 & 169.35 & 221.75 & RW2 \\ 
  Liberia & SOUTH EASTERN B & 90-94 & 233.96 & 263.50 & 206.80 & HT-Direct \\ 
  Liberia & SOUTH EASTERN B & 90-94 & 219.46 & 199.96 & 240.57 & RW2 \\ 
  Liberia & SOUTH EASTERN B & 95-99 & 171.96 & 199.60 & 147.45 & HT-Direct \\ 
  Liberia & SOUTH EASTERN B & 95-99 & 190.45 & 170.01 & 211.17 & RW2 \\ 
  Liberia & SOUTH EASTERN B & 00-04 & 136.74 & 165.64 & 112.21 & HT-Direct \\ 
  Liberia & SOUTH EASTERN B & 00-04 & 154.88 & 136.91 & 173.20 & RW2 \\ 
  Liberia & SOUTH EASTERN B & 05-09 & 129.05 & 142.09 & 117.05 & HT-Direct \\ 
  Liberia & SOUTH EASTERN B & 05-09 & 136.46 & 124.59 & 149.16 & RW2 \\ 
  Liberia & SOUTH EASTERN B & 10-14 & 149.05 & 185.19 & 118.93 & HT-Direct \\ 
  Liberia & SOUTH EASTERN B & 10-14 & 124.34 & 105.06 & 147.78 & RW2 \\ 
  Liberia & SOUTH EASTERN B & 15-19 & 113.91 & 46.90 & 251.44 & RW2 \\ 
  Madagascar & ALL & 80-84 & 165.73 & 161.52 & 170.18 & IHME \\ 
  Madagascar & ALL & 80-84 & 180.00 & 170.61 & 189.79 & RW2 \\ 
  Madagascar & ALL & 80-84 & 180.07 & 174.50 & 186.07 & UN \\ 
  Madagascar & ALL & 85-89 & 161.67 & 157.73 & 165.74 & IHME \\ 
  Madagascar & ALL & 85-89 & 175.66 & 167.31 & 184.31 & RW2 \\ 
  Madagascar & ALL & 85-89 & 175.51 & 170.54 & 181.16 & UN \\ 
  Madagascar & ALL & 90-94 & 139.44 & 135.98 & 142.88 & IHME \\ 
  Madagascar & ALL & 90-94 & 151.48 & 143.57 & 159.73 & RW2 \\ 
  Madagascar & ALL & 90-94 & 151.62 & 146.90 & 156.19 & UN \\ 
  Madagascar & ALL & 95-99 & 119.85 & 116.48 & 123.52 & IHME \\ 
  Madagascar & ALL & 95-99 & 127.32 & 119.66 & 135.35 & RW2 \\ 
  Madagascar & ALL & 95-99 & 127.22 & 123.04 & 131.47 & UN \\ 
  Madagascar & ALL & 00-04 & 98.55 & 95.14 & 102.11 & IHME \\ 
  Madagascar & ALL & 00-04 & 97.00 & 90.08 & 104.43 & RW2 \\ 
  Madagascar & ALL & 00-04 & 97.07 & 92.82 & 101.48 & UN \\ 
  Madagascar & ALL & 05-09 & 78.84 & 75.52 & 82.79 & IHME \\ 
  Madagascar & ALL & 05-09 & 71.38 & 64.15 & 79.35 & RW2 \\ 
  Madagascar & ALL & 05-09 & 71.36 & 66.41 & 76.61 & UN \\ 
  Madagascar & ALL & 10-14 & 71.77 & 66.33 & 78.09 & IHME \\ 
  Madagascar & ALL & 10-14 & 51.97 & 18.99 & 133.66 & RW2 \\ 
  Madagascar & ALL & 10-14 & 55.71 & 48.63 & 64.70 & UN \\ 
  Madagascar & ANTANANARIVO & 80-84 & 141.96 & 158.62 & 126.78 & HT-Direct \\ 
  Madagascar & ANTANANARIVO & 80-84 & 140.62 & 126.67 & 155.59 & RW2 \\ 
  Madagascar & ANTANANARIVO & 85-89 & 156.52 & 174.76 & 139.86 & HT-Direct \\ 
  Madagascar & ANTANANARIVO & 85-89 & 143.21 & 130.49 & 157.71 & RW2 \\ 
  Madagascar & ANTANANARIVO & 90-94 & 113.62 & 129.92 & 99.14 & HT-Direct \\ 
  Madagascar & ANTANANARIVO & 90-94 & 120.51 & 108.82 & 133.49 & RW2 \\ 
  Madagascar & ANTANANARIVO & 95-99 & 88.48 & 102.74 & 76.03 & HT-Direct \\ 
  Madagascar & ANTANANARIVO & 95-99 & 92.06 & 81.19 & 103.57 & RW2 \\ 
  Madagascar & ANTANANARIVO & 00-04 & 61.07 & 72.39 & 51.42 & HT-Direct \\ 
  Madagascar & ANTANANARIVO & 00-04 & 74.98 & 64.97 & 85.89 & RW2 \\ 
  Madagascar & ANTANANARIVO & 05-09 & 65.24 & 81.38 & 52.13 & HT-Direct \\ 
  Madagascar & ANTANANARIVO & 05-09 & 59.98 & 48.16 & 75.34 & RW2 \\ 
  Madagascar & ANTANANARIVO & 10-14 & 47.35 & 17.26 & 125.17 & RW2 \\ 
  Madagascar & ANTANANARIVO & 15-19 & 37.36 & 3.13 & 320.63 & RW2 \\ 
  Madagascar & ANTSIRANANA & 80-84 & 151.14 & 176.79 & 128.63 & HT-Direct \\ 
  Madagascar & ANTSIRANANA & 80-84 & 142.51 & 123.14 & 164.36 & RW2 \\ 
  Madagascar & ANTSIRANANA & 85-89 & 134.70 & 156.36 & 115.63 & HT-Direct \\ 
  Madagascar & ANTSIRANANA & 85-89 & 137.92 & 122.33 & 154.87 & RW2 \\ 
  Madagascar & ANTSIRANANA & 90-94 & 134.46 & 158.92 & 113.26 & HT-Direct \\ 
  Madagascar & ANTSIRANANA & 90-94 & 129.46 & 113.44 & 147.54 & RW2 \\ 
  Madagascar & ANTSIRANANA & 95-99 & 95.35 & 115.91 & 78.12 & HT-Direct \\ 
  Madagascar & ANTSIRANANA & 95-99 & 108.69 & 93.26 & 126.17 & RW2 \\ 
  Madagascar & ANTSIRANANA & 00-04 & 98.12 & 126.83 & 75.35 & HT-Direct \\ 
  Madagascar & ANTSIRANANA & 00-04 & 91.89 & 74.81 & 112.86 & RW2 \\ 
  Madagascar & ANTSIRANANA & 05-09 & 51.31 & 84.34 & 30.79 & HT-Direct \\ 
  Madagascar & ANTSIRANANA & 05-09 & 64.33 & 42.73 & 95.70 & RW2 \\ 
  Madagascar & ANTSIRANANA & 10-14 & 41.33 & 12.09 & 129.80 & RW2 \\ 
  Madagascar & ANTSIRANANA & 15-19 & 26.11 & 1.69 & 281.80 & RW2 \\ 
  Madagascar & FIANARANTSOA & 80-84 & 250.64 & 278.62 & 224.60 & HT-Direct \\ 
  Madagascar & FIANARANTSOA & 80-84 & 234.90 & 212.39 & 259.50 & RW2 \\ 
  Madagascar & FIANARANTSOA & 85-89 & 214.89 & 236.24 & 194.97 & HT-Direct \\ 
  Madagascar & FIANARANTSOA & 85-89 & 215.22 & 198.66 & 232.31 & RW2 \\ 
  Madagascar & FIANARANTSOA & 90-94 & 184.59 & 202.88 & 167.59 & HT-Direct \\ 
  Madagascar & FIANARANTSOA & 90-94 & 194.09 & 178.69 & 210.00 & RW2 \\ 
  Madagascar & FIANARANTSOA & 95-99 & 168.89 & 188.07 & 151.31 & HT-Direct \\ 
  Madagascar & FIANARANTSOA & 95-99 & 172.13 & 157.34 & 188.49 & RW2 \\ 
  Madagascar & FIANARANTSOA & 00-04 & 139.18 & 155.72 & 124.14 & HT-Direct \\ 
  Madagascar & FIANARANTSOA & 00-04 & 144.43 & 130.68 & 159.77 & RW2 \\ 
  Madagascar & FIANARANTSOA & 05-09 & 89.04 & 102.63 & 77.09 & HT-Direct \\ 
  Madagascar & FIANARANTSOA & 05-09 & 97.96 & 84.68 & 112.99 & RW2 \\ 
  Madagascar & FIANARANTSOA & 10-14 & 60.10 & 22.99 & 144.54 & RW2 \\ 
  Madagascar & FIANARANTSOA & 15-19 & 35.53 & 3.08 & 292.81 & RW2 \\ 
  Madagascar & MAHAJANGA & 80-84 & 250.23 & 278.59 & 223.87 & HT-Direct \\ 
  Madagascar & MAHAJANGA & 80-84 & 233.52 & 210.75 & 258.38 & RW2 \\ 
  Madagascar & MAHAJANGA & 85-89 & 188.21 & 209.77 & 168.40 & HT-Direct \\ 
  Madagascar & MAHAJANGA & 85-89 & 189.91 & 173.92 & 206.31 & RW2 \\ 
  Madagascar & MAHAJANGA & 90-94 & 146.44 & 162.32 & 131.86 & HT-Direct \\ 
  Madagascar & MAHAJANGA & 90-94 & 152.58 & 139.45 & 166.50 & RW2 \\ 
  Madagascar & MAHAJANGA & 95-99 & 112.95 & 131.23 & 96.93 & HT-Direct \\ 
  Madagascar & MAHAJANGA & 95-99 & 121.72 & 108.61 & 136.06 & RW2 \\ 
  Madagascar & MAHAJANGA & 00-04 & 106.10 & 126.68 & 88.53 & HT-Direct \\ 
  Madagascar & MAHAJANGA & 00-04 & 99.07 & 86.26 & 114.32 & RW2 \\ 
  Madagascar & MAHAJANGA & 05-09 & 57.51 & 71.56 & 46.08 & HT-Direct \\ 
  Madagascar & MAHAJANGA & 05-09 & 65.98 & 53.45 & 81.19 & RW2 \\ 
  Madagascar & MAHAJANGA & 10-14 & 39.96 & 14.63 & 103.01 & RW2 \\ 
  Madagascar & MAHAJANGA & 15-19 & 24.02 & 2.02 & 216.61 & RW2 \\ 
  Madagascar & TOAMASINA & 80-84 & 176.87 & 203.15 & 153.35 & HT-Direct \\ 
  Madagascar & TOAMASINA & 80-84 & 173.46 & 152.35 & 196.62 & RW2 \\ 
  Madagascar & TOAMASINA & 85-89 & 185.37 & 206.17 & 166.22 & HT-Direct \\ 
  Madagascar & TOAMASINA & 85-89 & 184.68 & 169.63 & 201.04 & RW2 \\ 
  Madagascar & TOAMASINA & 90-94 & 175.92 & 196.20 & 157.32 & HT-Direct \\ 
  Madagascar & TOAMASINA & 90-94 & 170.81 & 155.19 & 187.99 & RW2 \\ 
  Madagascar & TOAMASINA & 95-99 & 121.03 & 137.76 & 106.08 & HT-Direct \\ 
  Madagascar & TOAMASINA & 95-99 & 126.56 & 113.75 & 140.57 & RW2 \\ 
  Madagascar & TOAMASINA & 00-04 & 72.86 & 89.03 & 59.44 & HT-Direct \\ 
  Madagascar & TOAMASINA & 00-04 & 85.90 & 73.16 & 100.29 & RW2 \\ 
  Madagascar & TOAMASINA & 05-09 & 48.62 & 67.71 & 34.71 & HT-Direct \\ 
  Madagascar & TOAMASINA & 05-09 & 50.33 & 37.58 & 66.58 & RW2 \\ 
  Madagascar & TOAMASINA & 10-14 & 27.76 & 9.50 & 76.87 & RW2 \\ 
  Madagascar & TOAMASINA & 15-19 & 15.22 & 1.24 & 153.91 & RW2 \\ 
  Madagascar & TOLIARY & 80-84 & 181.89 & 212.46 & 154.86 & HT-Direct \\ 
  Madagascar & TOLIARY & 80-84 & 175.79 & 152.72 & 201.63 & RW2 \\ 
  Madagascar & TOLIARY & 85-89 & 180.74 & 201.71 & 161.52 & HT-Direct \\ 
  Madagascar & TOLIARY & 85-89 & 176.15 & 160.94 & 192.55 & RW2 \\ 
  Madagascar & TOLIARY & 90-94 & 153.11 & 172.19 & 135.81 & HT-Direct \\ 
  Madagascar & TOLIARY & 90-94 & 162.98 & 148.07 & 178.96 & RW2 \\ 
  Madagascar & TOLIARY & 95-99 & 142.51 & 162.21 & 124.85 & HT-Direct \\ 
  Madagascar & TOLIARY & 95-99 & 142.32 & 128.18 & 158.18 & RW2 \\ 
  Madagascar & TOLIARY & 00-04 & 107.44 & 121.35 & 94.96 & HT-Direct \\ 
  Madagascar & TOLIARY & 00-04 & 113.11 & 101.43 & 126.13 & RW2 \\ 
  Madagascar & TOLIARY & 05-09 & 61.91 & 75.59 & 50.57 & HT-Direct \\ 
  Madagascar & TOLIARY & 05-09 & 70.46 & 57.76 & 85.25 & RW2 \\ 
  Madagascar & TOLIARY & 10-14 & 39.80 & 14.42 & 102.53 & RW2 \\ 
  Madagascar & TOLIARY & 15-19 & 22.26 & 1.82 & 204.12 & RW2 \\ 
  Malawi & ALL & 80-84 & 249.54 & 244.17 & 254.87 & IHME \\ 
  Malawi & ALL & 80-84 & 247.37 & 236.82 & 258.23 & RW2 \\ 
  Malawi & ALL & 80-84 & 247.38 & 240.91 & 254.48 & UN \\ 
  Malawi & ALL & 85-89 & 235.41 & 230.64 & 240.45 & IHME \\ 
  Malawi & ALL & 85-89 & 251.17 & 242.98 & 259.51 & RW2 \\ 
  Malawi & ALL & 85-89 & 251.19 & 244.39 & 258.13 & UN \\ 
  Malawi & ALL & 90-94 & 212.05 & 207.95 & 216.48 & IHME \\ 
  Malawi & ALL & 90-94 & 227.16 & 219.79 & 234.70 & RW2 \\ 
  Malawi & ALL & 90-94 & 227.09 & 221.44 & 233.44 & UN \\ 
  Malawi & ALL & 95-99 & 184.45 & 180.19 & 188.56 & IHME \\ 
  Malawi & ALL & 95-99 & 197.02 & 190.60 & 203.56 & RW2 \\ 
  Malawi & ALL & 95-99 & 197.10 & 192.43 & 202.12 & UN \\ 
  Malawi & ALL & 00-04 & 147.19 & 143.21 & 151.06 & IHME \\ 
  Malawi & ALL & 00-04 & 148.13 & 142.64 & 153.83 & RW2 \\ 
  Malawi & ALL & 00-04 & 148.18 & 143.93 & 152.99 & UN \\ 
  Malawi & ALL & 05-09 & 112.49 & 109.06 & 115.99 & IHME \\ 
  Malawi & ALL & 05-09 & 105.34 & 100.66 & 110.22 & RW2 \\ 
  Malawi & ALL & 05-09 & 105.24 & 101.01 & 109.83 & UN \\ 
  Malawi & ALL & 10-14 & 91.52 & 87.03 & 96.06 & IHME \\ 
  Malawi & ALL & 10-14 & 79.49 & 72.83 & 86.66 & RW2 \\ 
  Malawi & ALL & 10-14 & 79.65 & 72.89 & 87.10 & UN \\ 
  Malawi & CENTRAL REGION & 80-84 & 271.78 & 290.18 & 254.14 & HT-Direct \\ 
  Malawi & CENTRAL REGION & 80-84 & 278.09 & 261.82 & 295.12 & RW2 \\ 
  Malawi & CENTRAL REGION & 85-89 & 268.83 & 281.51 & 256.51 & HT-Direct \\ 
  Malawi & CENTRAL REGION & 85-89 & 273.04 & 262.22 & 284.25 & RW2 \\ 
  Malawi & CENTRAL REGION & 90-94 & 223.18 & 234.62 & 212.15 & HT-Direct \\ 
  Malawi & CENTRAL REGION & 90-94 & 239.50 & 229.67 & 249.57 & RW2 \\ 
  Malawi & CENTRAL REGION & 95-99 & 191.53 & 201.24 & 182.17 & HT-Direct \\ 
  Malawi & CENTRAL REGION & 95-99 & 202.53 & 193.94 & 211.21 & RW2 \\ 
  Malawi & CENTRAL REGION & 00-04 & 149.62 & 158.84 & 140.84 & HT-Direct \\ 
  Malawi & CENTRAL REGION & 00-04 & 152.67 & 145.41 & 160.30 & RW2 \\ 
  Malawi & CENTRAL REGION & 05-09 & 104.93 & 112.68 & 97.66 & HT-Direct \\ 
  Malawi & CENTRAL REGION & 05-09 & 110.62 & 104.09 & 117.53 & RW2 \\ 
  Malawi & CENTRAL REGION & 10-14 & 76.30 & 85.29 & 68.19 & HT-Direct \\ 
  Malawi & CENTRAL REGION & 10-14 & 84.62 & 75.72 & 94.60 & RW2 \\ 
  Malawi & CENTRAL REGION & 15-19 & 65.68 & 30.19 & 137.92 & RW2 \\ 
  Malawi & NORTHERN REGION & 80-84 & 193.59 & 216.41 & 172.65 & HT-Direct \\ 
  Malawi & NORTHERN REGION & 80-84 & 199.91 & 180.89 & 220.10 & RW2 \\ 
  Malawi & NORTHERN REGION & 85-89 & 196.25 & 212.94 & 180.57 & HT-Direct \\ 
  Malawi & NORTHERN REGION & 85-89 & 200.85 & 188.41 & 214.20 & RW2 \\ 
  Malawi & NORTHERN REGION & 90-94 & 171.04 & 186.48 & 156.63 & HT-Direct \\ 
  Malawi & NORTHERN REGION & 90-94 & 177.60 & 166.31 & 189.34 & RW2 \\ 
  Malawi & NORTHERN REGION & 95-99 & 135.85 & 149.65 & 123.13 & HT-Direct \\ 
  Malawi & NORTHERN REGION & 95-99 & 150.60 & 140.24 & 161.30 & RW2 \\ 
  Malawi & NORTHERN REGION & 00-04 & 116.03 & 129.02 & 104.20 & HT-Direct \\ 
  Malawi & NORTHERN REGION & 00-04 & 114.50 & 105.97 & 123.97 & RW2 \\ 
  Malawi & NORTHERN REGION & 05-09 & 79.94 & 92.20 & 69.18 & HT-Direct \\ 
  Malawi & NORTHERN REGION & 05-09 & 82.78 & 74.87 & 91.48 & RW2 \\ 
  Malawi & NORTHERN REGION & 10-14 & 52.18 & 61.64 & 44.11 & HT-Direct \\ 
  Malawi & NORTHERN REGION & 10-14 & 62.47 & 53.01 & 73.25 & RW2 \\ 
  Malawi & NORTHERN REGION & 15-19 & 47.77 & 21.03 & 104.96 & RW2 \\ 
  Malawi & SOUTHERN REGION & 80-84 & 229.31 & 243.00 & 216.17 & HT-Direct \\ 
  Malawi & SOUTHERN REGION & 80-84 & 232.05 & 219.40 & 245.33 & RW2 \\ 
  Malawi & SOUTHERN REGION & 85-89 & 229.98 & 241.28 & 219.05 & HT-Direct \\ 
  Malawi & SOUTHERN REGION & 85-89 & 243.97 & 233.64 & 254.33 & RW2 \\ 
  Malawi & SOUTHERN REGION & 90-94 & 223.45 & 233.16 & 214.04 & HT-Direct \\ 
  Malawi & SOUTHERN REGION & 90-94 & 230.84 & 222.01 & 240.01 & RW2 \\ 
  Malawi & SOUTHERN REGION & 95-99 & 193.89 & 201.82 & 186.20 & HT-Direct \\ 
  Malawi & SOUTHERN REGION & 95-99 & 201.87 & 194.41 & 209.67 & RW2 \\ 
  Malawi & SOUTHERN REGION & 00-04 & 142.00 & 149.15 & 135.14 & HT-Direct \\ 
  Malawi & SOUTHERN REGION & 00-04 & 149.94 & 143.53 & 156.61 & RW2 \\ 
  Malawi & SOUTHERN REGION & 05-09 & 103.86 & 110.13 & 97.91 & HT-Direct \\ 
  Malawi & SOUTHERN REGION & 05-09 & 106.17 & 100.59 & 112.00 & RW2 \\ 
  Malawi & SOUTHERN REGION & 10-14 & 65.79 & 73.50 & 58.83 & HT-Direct \\ 
  Malawi & SOUTHERN REGION & 10-14 & 78.05 & 69.58 & 87.34 & RW2 \\ 
  Malawi & SOUTHERN REGION & 15-19 & 57.78 & 26.21 & 123.23 & RW2 \\ 
  Mali & ALL & 80-84 & 264.91 & 261.52 & 267.92 & IHME \\ 
  Mali & ALL & 80-84 & 308.58 & 297.83 & 319.53 & RW2 \\ 
  Mali & ALL & 80-84 & 308.66 & 299.76 & 318.13 & UN \\ 
  Mali & ALL & 85-89 & 240.25 & 237.66 & 243.28 & IHME \\ 
  Mali & ALL & 85-89 & 271.57 & 263.33 & 279.94 & RW2 \\ 
  Mali & ALL & 85-89 & 271.40 & 263.51 & 279.20 & UN \\ 
  Mali & ALL & 90-94 & 222.21 & 219.55 & 225.14 & IHME \\ 
  Mali & ALL & 90-94 & 247.13 & 240.27 & 254.12 & RW2 \\ 
  Mali & ALL & 90-94 & 247.33 & 239.50 & 254.41 & UN \\ 
  Mali & ALL & 95-99 & 205.62 & 202.49 & 208.44 & IHME \\ 
  Mali & ALL & 95-99 & 234.72 & 226.41 & 243.18 & RW2 \\ 
  Mali & ALL & 95-99 & 234.26 & 227.42 & 242.23 & UN \\ 
  Mali & ALL & 00-04 & 182.24 & 179.26 & 185.52 & IHME \\ 
  Mali & ALL & 00-04 & 200.55 & 192.30 & 209.11 & RW2 \\ 
  Mali & ALL & 00-04 & 200.96 & 193.90 & 208.55 & UN \\ 
  Mali & ALL & 05-09 & 158.74 & 155.00 & 162.52 & IHME \\ 
  Mali & ALL & 05-09 & 155.74 & 142.41 & 170.07 & RW2 \\ 
  Mali & ALL & 05-09 & 155.29 & 143.82 & 166.77 & UN \\ 
  Mali & ALL & 10-14 & 140.35 & 134.80 & 145.97 & IHME \\ 
  Mali & ALL & 10-14 & 116.87 & 45.58 & 265.70 & RW2 \\ 
  Mali & ALL & 10-14 & 128.09 & 110.80 & 148.02 & UN \\ 
  Mali & BAMAKO & 80-84 & 173.88 & 192.18 & 156.97 & HT-Direct \\ 
  Mali & BAMAKO & 80-84 & 181.48 & 165.17 & 198.80 & RW2 \\ 
  Mali & BAMAKO & 85-89 & 148.58 & 166.26 & 132.47 & HT-Direct \\ 
  Mali & BAMAKO & 85-89 & 154.74 & 143.71 & 166.51 & RW2 \\ 
  Mali & BAMAKO & 90-94 & 144.98 & 161.06 & 130.25 & HT-Direct \\ 
  Mali & BAMAKO & 90-94 & 138.34 & 128.64 & 148.89 & RW2 \\ 
  Mali & BAMAKO & 95-99 & 130.87 & 149.49 & 114.26 & HT-Direct \\ 
  Mali & BAMAKO & 95-99 & 130.56 & 120.49 & 141.40 & RW2 \\ 
  Mali & BAMAKO & 00-04 & 109.22 & 124.45 & 95.66 & HT-Direct \\ 
  Mali & BAMAKO & 00-04 & 106.92 & 96.40 & 118.53 & RW2 \\ 
  Mali & BAMAKO & 05-09 & 77.40 & 113.98 & 51.88 & HT-Direct \\ 
  Mali & BAMAKO & 05-09 & 80.49 & 64.93 & 98.49 & RW2 \\ 
  Mali & BAMAKO & 10-14 & 59.45 & 23.40 & 141.68 & RW2 \\ 
  Mali & BAMAKO & 15-19 & 43.63 & 4.43 & 320.46 & RW2 \\ 
  Mali & KAYES, KOULIKORO & 80-84 & 294.57 & 311.15 & 278.52 & HT-Direct \\ 
  Mali & KAYES, KOULIKORO & 80-84 & 302.71 & 287.34 & 318.98 & RW2 \\ 
  Mali & KAYES, KOULIKORO & 85-89 & 257.13 & 271.77 & 243.01 & HT-Direct \\ 
  Mali & KAYES, KOULIKORO & 85-89 & 261.92 & 250.73 & 273.23 & RW2 \\ 
  Mali & KAYES, KOULIKORO & 90-94 & 232.98 & 244.69 & 221.66 & HT-Direct \\ 
  Mali & KAYES, KOULIKORO & 90-94 & 236.60 & 227.30 & 246.00 & RW2 \\ 
  Mali & KAYES, KOULIKORO & 95-99 & 238.16 & 250.52 & 226.22 & HT-Direct \\ 
  Mali & KAYES, KOULIKORO & 95-99 & 227.74 & 218.69 & 237.07 & RW2 \\ 
  Mali & KAYES, KOULIKORO & 00-04 & 197.55 & 212.32 & 183.58 & HT-Direct \\ 
  Mali & KAYES, KOULIKORO & 00-04 & 194.15 & 183.07 & 205.95 & RW2 \\ 
  Mali & KAYES, KOULIKORO & 05-09 & 165.21 & 201.03 & 134.70 & HT-Direct \\ 
  Mali & KAYES, KOULIKORO & 05-09 & 153.29 & 134.58 & 174.20 & RW2 \\ 
  Mali & KAYES, KOULIKORO & 10-14 & 118.41 & 50.67 & 252.50 & RW2 \\ 
  Mali & KAYES, KOULIKORO & 15-19 & 90.55 & 10.11 & 497.98 & RW2 \\ 
  Mali & MOPTI, TOMBOUCTOU, GAO, KIDAL & 80-84 & 383.04 & 406.82 & 359.81 & HT-Direct \\ 
  Mali & MOPTI, TOMBOUCTOU, GAO, KIDAL & 80-84 & 406.69 & 383.81 & 429.34 & RW2 \\ 
  Mali & MOPTI, TOMBOUCTOU, GAO, KIDAL & 85-89 & 351.97 & 372.29 & 332.18 & HT-Direct \\ 
  Mali & MOPTI, TOMBOUCTOU, GAO, KIDAL & 85-89 & 348.51 & 333.65 & 363.77 & RW2 \\ 
  Mali & MOPTI, TOMBOUCTOU, GAO, KIDAL & 90-94 & 296.41 & 311.90 & 281.37 & HT-Direct \\ 
  Mali & MOPTI, TOMBOUCTOU, GAO, KIDAL & 90-94 & 298.62 & 286.69 & 311.20 & RW2 \\ 
  Mali & MOPTI, TOMBOUCTOU, GAO, KIDAL & 95-99 & 279.53 & 297.16 & 262.56 & HT-Direct \\ 
  Mali & MOPTI, TOMBOUCTOU, GAO, KIDAL & 95-99 & 262.82 & 250.97 & 274.99 & RW2 \\ 
  Mali & MOPTI, TOMBOUCTOU, GAO, KIDAL & 00-04 & 191.87 & 207.24 & 177.38 & HT-Direct \\ 
  Mali & MOPTI, TOMBOUCTOU, GAO, KIDAL & 00-04 & 201.70 & 189.43 & 214.48 & RW2 \\ 
  Mali & MOPTI, TOMBOUCTOU, GAO, KIDAL & 05-09 & 173.69 & 207.09 & 144.70 & HT-Direct \\ 
  Mali & MOPTI, TOMBOUCTOU, GAO, KIDAL & 05-09 & 145.11 & 127.56 & 165.12 & RW2 \\ 
  Mali & MOPTI, TOMBOUCTOU, GAO, KIDAL & 10-14 & 101.98 & 43.37 & 222.30 & RW2 \\ 
  Mali & MOPTI, TOMBOUCTOU, GAO, KIDAL & 15-19 & 70.54 & 7.48 & 431.61 & RW2 \\ 
  Mali & SIKASSO, SEGOU & 80-84 & 279.83 & 295.21 & 264.96 & HT-Direct \\ 
  Mali & SIKASSO, SEGOU & 80-84 & 290.18 & 275.81 & 304.97 & RW2 \\ 
  Mali & SIKASSO, SEGOU & 85-89 & 264.39 & 277.07 & 252.09 & HT-Direct \\ 
  Mali & SIKASSO, SEGOU & 85-89 & 266.79 & 256.76 & 277.15 & RW2 \\ 
  Mali & SIKASSO, SEGOU & 90-94 & 249.90 & 261.24 & 238.90 & HT-Direct \\ 
  Mali & SIKASSO, SEGOU & 90-94 & 253.64 & 244.43 & 262.94 & RW2 \\ 
  Mali & SIKASSO, SEGOU & 95-99 & 265.37 & 283.18 & 248.29 & HT-Direct \\ 
  Mali & SIKASSO, SEGOU & 95-99 & 254.85 & 243.68 & 266.42 & RW2 \\ 
  Mali & SIKASSO, SEGOU & 00-04 & 234.67 & 251.49 & 218.65 & HT-Direct \\ 
  Mali & SIKASSO, SEGOU & 00-04 & 227.15 & 214.84 & 240.15 & RW2 \\ 
  Mali & SIKASSO, SEGOU & 05-09 & 197.36 & 224.43 & 172.82 & HT-Direct \\ 
  Mali & SIKASSO, SEGOU & 05-09 & 189.10 & 171.01 & 208.57 & RW2 \\ 
  Mali & SIKASSO, SEGOU & 10-14 & 155.15 & 69.90 & 313.00 & RW2 \\ 
  Mali & SIKASSO, SEGOU & 15-19 & 125.41 & 14.71 & 580.00 & RW2 \\ 
  Morocco & ALL & 80-84 & 110.92 & 109.27 & 112.75 & IHME \\ 
  Morocco & ALL & 80-84 & 120.67 & 113.75 & 127.96 & RW2 \\ 
  Morocco & ALL & 80-84 & 120.68 & 117.85 & 123.93 & UN \\ 
  Morocco & ALL & 85-89 & 83.10 & 81.61 & 84.46 & IHME \\ 
  Morocco & ALL & 85-89 & 93.08 & 86.59 & 99.98 & RW2 \\ 
  Morocco & ALL & 85-89 & 93.06 & 90.49 & 95.68 & UN \\ 
  Morocco & ALL & 90-94 & 63.58 & 62.39 & 64.93 & IHME \\ 
  Morocco & ALL & 90-94 & 73.04 & 66.93 & 79.66 & RW2 \\ 
  Morocco & ALL & 90-94 & 73.15 & 70.89 & 75.36 & UN \\ 
  Morocco & ALL & 95-99 & 50.42 & 49.10 & 51.85 & IHME \\ 
  Morocco & ALL & 95-99 & 57.95 & 51.95 & 64.54 & RW2 \\ 
  Morocco & ALL & 95-99 & 57.73 & 55.73 & 59.79 & UN \\ 
  Morocco & ALL & 00-04 & 40.28 & 38.49 & 42.03 & IHME \\ 
  Morocco & ALL & 00-04 & 45.74 & 39.40 & 53.09 & RW2 \\ 
  Morocco & ALL & 00-04 & 45.88 & 44.01 & 47.82 & UN \\ 
  Morocco & ALL & 05-09 & 32.25 & 30.20 & 34.41 & IHME \\ 
  Morocco & ALL & 05-09 & 35.82 & 10.21 & 117.54 & RW2 \\ 
  Morocco & ALL & 05-09 & 37.16 & 35.23 & 39.25 & UN \\ 
  Morocco & ALL & 10-14 & 25.87 & 23.78 & 28.26 & IHME \\ 
  Morocco & ALL & 10-14 & 27.81 & 1.22 & 388.29 & RW2 \\ 
  Morocco & ALL & 10-14 & 30.51 & 27.87 & 33.48 & UN \\ 
  Morocco & CENTRE & 80-84 & 85.05 & 95.98 & 75.27 & HT-Direct \\ 
  Morocco & CENTRE & 80-84 & 84.67 & 75.28 & 95.27 & RW2 \\ 
  Morocco & CENTRE & 85-89 & 56.63 & 65.84 & 48.64 & HT-Direct \\ 
  Morocco & CENTRE & 85-89 & 65.41 & 58.52 & 72.74 & RW2 \\ 
  Morocco & CENTRE & 90-94 & 49.56 & 60.37 & 40.61 & HT-Direct \\ 
  Morocco & CENTRE & 90-94 & 52.48 & 45.91 & 59.71 & RW2 \\ 
  Morocco & CENTRE & 95-99 & 45.35 & 58.21 & 35.22 & HT-Direct \\ 
  Morocco & CENTRE & 95-99 & 44.62 & 37.74 & 52.62 & RW2 \\ 
  Morocco & CENTRE & 00-04 & 39.73 & 53.39 & 29.46 & HT-Direct \\ 
  Morocco & CENTRE & 00-04 & 37.71 & 29.85 & 47.84 & RW2 \\ 
  Morocco & CENTRE & 05-09 & 31.80 & 10.52 & 91.14 & RW2 \\ 
  Morocco & CENTRE & 10-14 & 26.66 & 1.78 & 297.88 & RW2 \\ 
  Morocco & CENTRE & 15-19 & 22.69 & 0.19 & 728.81 & RW2 \\ 
  Morocco & CENTRE-NORD & 80-84 & 121.85 & 140.94 & 105.02 & HT-Direct \\ 
  Morocco & CENTRE-NORD & 80-84 & 131.00 & 114.04 & 150.03 & RW2 \\ 
  Morocco & CENTRE-NORD & 85-89 & 100.57 & 117.58 & 85.78 & HT-Direct \\ 
  Morocco & CENTRE-NORD & 85-89 & 104.53 & 93.39 & 117.13 & RW2 \\ 
  Morocco & CENTRE-NORD & 90-94 & 86.48 & 106.10 & 70.20 & HT-Direct \\ 
  Morocco & CENTRE-NORD & 90-94 & 82.99 & 72.41 & 95.24 & RW2 \\ 
  Morocco & CENTRE-NORD & 95-99 & 67.16 & 89.49 & 50.10 & HT-Direct \\ 
  Morocco & CENTRE-NORD & 95-99 & 66.87 & 55.71 & 80.17 & RW2 \\ 
  Morocco & CENTRE-NORD & 00-04 & 47.07 & 66.85 & 32.94 & HT-Direct \\ 
  Morocco & CENTRE-NORD & 00-04 & 52.17 & 39.68 & 68.06 & RW2 \\ 
  Morocco & CENTRE-NORD & 05-09 & 39.99 & 13.07 & 114.55 & RW2 \\ 
  Morocco & CENTRE-NORD & 10-14 & 30.55 & 2.01 & 320.97 & RW2 \\ 
  Morocco & CENTRE-NORD & 15-19 & 22.85 & 0.19 & 734.55 & RW2 \\ 
  Morocco & CENTRE-SUD & 80-84 & 95.88 & 114.94 & 79.70 & HT-Direct \\ 
  Morocco & CENTRE-SUD & 80-84 & 108.39 & 91.22 & 128.48 & RW2 \\ 
  Morocco & CENTRE-SUD & 85-89 & 106.44 & 134.79 & 83.47 & HT-Direct \\ 
  Morocco & CENTRE-SUD & 85-89 & 91.78 & 79.64 & 105.88 & RW2 \\ 
  Morocco & CENTRE-SUD & 90-94 & 75.23 & 97.13 & 57.95 & HT-Direct \\ 
  Morocco & CENTRE-SUD & 90-94 & 77.93 & 66.34 & 91.61 & RW2 \\ 
  Morocco & CENTRE-SUD & 95-99 & 64.17 & 94.64 & 43.05 & HT-Direct \\ 
  Morocco & CENTRE-SUD & 95-99 & 68.04 & 54.50 & 84.70 & RW2 \\ 
  Morocco & CENTRE-SUD & 00-04 & 58.27 & 92.24 & 36.31 & HT-Direct \\ 
  Morocco & CENTRE-SUD & 00-04 & 57.98 & 42.22 & 79.31 & RW2 \\ 
  Morocco & CENTRE-SUD & 05-09 & 48.60 & 15.72 & 138.48 & RW2 \\ 
  Morocco & CENTRE-SUD & 10-14 & 40.60 & 2.67 & 397.51 & RW2 \\ 
  Morocco & CENTRE-SUD & 15-19 & 34.38 & 0.28 & 800.25 & RW2 \\ 
  Morocco & NORD-OUEST & 80-84 & 115.10 & 127.84 & 103.49 & HT-Direct \\ 
  Morocco & NORD-OUEST & 80-84 & 123.57 & 111.44 & 136.72 & RW2 \\ 
  Morocco & NORD-OUEST & 85-89 & 97.96 & 111.66 & 85.77 & HT-Direct \\ 
  Morocco & NORD-OUEST & 85-89 & 94.30 & 85.47 & 104.19 & RW2 \\ 
  Morocco & NORD-OUEST & 90-94 & 61.40 & 74.85 & 50.24 & HT-Direct \\ 
  Morocco & NORD-OUEST & 90-94 & 71.65 & 62.92 & 81.34 & RW2 \\ 
  Morocco & NORD-OUEST & 95-99 & 57.30 & 72.67 & 45.02 & HT-Direct \\ 
  Morocco & NORD-OUEST & 95-99 & 56.48 & 47.81 & 66.62 & RW2 \\ 
  Morocco & NORD-OUEST & 00-04 & 45.13 & 63.68 & 31.80 & HT-Direct \\ 
  Morocco & NORD-OUEST & 00-04 & 43.89 & 34.09 & 56.53 & RW2 \\ 
  Morocco & NORD-OUEST & 05-09 & 33.84 & 11.04 & 98.48 & RW2 \\ 
  Morocco & NORD-OUEST & 10-14 & 25.95 & 1.75 & 290.68 & RW2 \\ 
  Morocco & NORD-OUEST & 15-19 & 19.60 & 0.17 & 695.69 & RW2 \\ 
  Morocco & ORIENTAL & 80-84 & 102.99 & 123.46 & 85.59 & HT-Direct \\ 
  Morocco & ORIENTAL & 80-84 & 103.01 & 86.56 & 122.28 & RW2 \\ 
  Morocco & ORIENTAL & 85-89 & 70.21 & 87.18 & 56.34 & HT-Direct \\ 
  Morocco & ORIENTAL & 85-89 & 82.79 & 71.20 & 96.04 & RW2 \\ 
  Morocco & ORIENTAL & 90-94 & 69.58 & 104.13 & 45.90 & HT-Direct \\ 
  Morocco & ORIENTAL & 90-94 & 67.78 & 55.69 & 82.07 & RW2 \\ 
  Morocco & ORIENTAL & 95-99 & 66.29 & 95.98 & 45.32 & HT-Direct \\ 
  Morocco & ORIENTAL & 95-99 & 57.16 & 44.22 & 73.62 & RW2 \\ 
  Morocco & ORIENTAL & 00-04 & 38.95 & 69.89 & 21.40 & HT-Direct \\ 
  Morocco & ORIENTAL & 00-04 & 46.80 & 31.97 & 68.14 & RW2 \\ 
  Morocco & ORIENTAL & 05-09 & 37.85 & 11.45 & 116.21 & RW2 \\ 
  Morocco & ORIENTAL & 10-14 & 30.37 & 1.87 & 337.87 & RW2 \\ 
  Morocco & ORIENTAL & 15-19 & 24.49 & 0.20 & 751.42 & RW2 \\ 
  Morocco & SUD & 80-84 & 159.28 & 183.91 & 137.39 & HT-Direct \\ 
  Morocco & SUD & 80-84 & 170.06 & 148.23 & 194.02 & RW2 \\ 
  Morocco & SUD & 85-89 & 120.16 & 140.38 & 102.50 & HT-Direct \\ 
  Morocco & SUD & 85-89 & 121.56 & 108.25 & 135.86 & RW2 \\ 
  Morocco & SUD & 90-94 & 80.53 & 100.37 & 64.33 & HT-Direct \\ 
  Morocco & SUD & 90-94 & 87.27 & 75.63 & 100.50 & RW2 \\ 
  Morocco & SUD & 95-99 & 70.57 & 97.65 & 50.58 & HT-Direct \\ 
  Morocco & SUD & 95-99 & 64.50 & 52.58 & 79.00 & RW2 \\ 
  Morocco & SUD & 00-04 & 42.53 & 66.38 & 27.00 & HT-Direct \\ 
  Morocco & SUD & 00-04 & 46.40 & 34.17 & 62.81 & RW2 \\ 
  Morocco & SUD & 05-09 & 32.86 & 10.47 & 97.05 & RW2 \\ 
  Morocco & SUD & 10-14 & 23.07 & 1.51 & 263.13 & RW2 \\ 
  Morocco & SUD & 15-19 & 16.08 & 0.13 & 662.37 & RW2 \\ 
  Morocco & TENSIFT & 80-84 & 138.66 & 158.16 & 121.23 & HT-Direct \\ 
  Morocco & TENSIFT & 80-84 & 143.08 & 125.55 & 162.42 & RW2 \\ 
  Morocco & TENSIFT & 85-89 & 91.41 & 110.61 & 75.27 & HT-Direct \\ 
  Morocco & TENSIFT & 85-89 & 108.09 & 95.41 & 122.02 & RW2 \\ 
  Morocco & TENSIFT & 90-94 & 95.43 & 118.56 & 76.43 & HT-Direct \\ 
  Morocco & TENSIFT & 90-94 & 83.72 & 72.15 & 97.08 & RW2 \\ 
  Morocco & TENSIFT & 95-99 & 63.52 & 89.90 & 44.51 & HT-Direct \\ 
  Morocco & TENSIFT & 95-99 & 66.76 & 54.76 & 81.18 & RW2 \\ 
  Morocco & TENSIFT & 00-04 & 49.43 & 68.51 & 35.46 & HT-Direct \\ 
  Morocco & TENSIFT & 00-04 & 51.81 & 39.45 & 67.50 & RW2 \\ 
  Morocco & TENSIFT & 05-09 & 39.64 & 12.88 & 115.30 & RW2 \\ 
  Morocco & TENSIFT & 10-14 & 30.01 & 2.00 & 329.13 & RW2 \\ 
  Morocco & TENSIFT & 15-19 & 22.69 & 0.20 & 732.27 & RW2 \\ 
  Mozambique & ALL & 80-84 & 250.42 & 237.09 & 264.07 & IHME \\ 
  Mozambique & ALL & 80-84 & 260.00 & 242.14 & 278.69 & RW2 \\ 
  Mozambique & ALL & 80-84 & 259.85 & 250.08 & 271.14 & UN \\ 
  Mozambique & ALL & 85-89 & 230.62 & 223.88 & 238.24 & IHME \\ 
  Mozambique & ALL & 85-89 & 248.51 & 232.21 & 265.48 & RW2 \\ 
  Mozambique & ALL & 85-89 & 248.72 & 238.91 & 257.88 & UN \\ 
  Mozambique & ALL & 90-94 & 205.17 & 202.81 & 207.71 & IHME \\ 
  Mozambique & ALL & 90-94 & 231.83 & 218.06 & 246.23 & RW2 \\ 
  Mozambique & ALL & 90-94 & 231.85 & 224.04 & 239.63 & UN \\ 
  Mozambique & ALL & 95-99 & 176.97 & 174.80 & 179.20 & IHME \\ 
  Mozambique & ALL & 95-99 & 198.19 & 188.48 & 208.19 & RW2 \\ 
  Mozambique & ALL & 95-99 & 198.20 & 192.02 & 205.08 & UN \\ 
  Mozambique & ALL & 00-04 & 144.16 & 141.98 & 146.21 & IHME \\ 
  Mozambique & ALL & 00-04 & 155.15 & 145.29 & 165.59 & RW2 \\ 
  Mozambique & ALL & 00-04 & 154.80 & 149.67 & 160.17 & UN \\ 
  Mozambique & ALL & 05-09 & 114.91 & 112.42 & 117.29 & IHME \\ 
  Mozambique & ALL & 05-09 & 119.68 & 109.50 & 130.67 & RW2 \\ 
  Mozambique & ALL & 05-09 & 120.29 & 115.63 & 125.70 & UN \\ 
  Mozambique & ALL & 10-14 & 90.78 & 87.44 & 93.88 & IHME \\ 
  Mozambique & ALL & 10-14 & 92.91 & 78.59 & 109.41 & RW2 \\ 
  Mozambique & ALL & 10-14 & 92.10 & 85.76 & 99.25 & UN \\ 
  Mozambique & CABO DELGADO & 80-84 & 301.69 & 372.89 & 238.91 & HT-Direct \\ 
  Mozambique & CABO DELGADO & 80-84 & 317.66 & 266.98 & 371.12 & RW2 \\ 
  Mozambique & CABO DELGADO & 85-89 & 287.41 & 334.14 & 244.81 & HT-Direct \\ 
  Mozambique & CABO DELGADO & 85-89 & 308.80 & 275.68 & 343.69 & RW2 \\ 
  Mozambique & CABO DELGADO & 90-94 & 272.10 & 311.44 & 236.02 & HT-Direct \\ 
  Mozambique & CABO DELGADO & 90-94 & 292.29 & 266.24 & 320.04 & RW2 \\ 
  Mozambique & CABO DELGADO & 95-99 & 242.70 & 277.86 & 210.68 & HT-Direct \\ 
  Mozambique & CABO DELGADO & 95-99 & 251.89 & 229.06 & 277.31 & RW2 \\ 
  Mozambique & CABO DELGADO & 00-04 & 194.02 & 226.87 & 164.92 & HT-Direct \\ 
  Mozambique & CABO DELGADO & 00-04 & 189.58 & 167.91 & 213.29 & RW2 \\ 
  Mozambique & CABO DELGADO & 05-09 & 100.98 & 128.53 & 78.80 & HT-Direct \\ 
  Mozambique & CABO DELGADO & 05-09 & 131.71 & 108.26 & 158.67 & RW2 \\ 
  Mozambique & CABO DELGADO & 10-14 & 60.13 & 106.56 & 33.18 & HT-Direct \\ 
  Mozambique & CABO DELGADO & 10-14 & 89.90 & 62.39 & 124.35 & RW2 \\ 
  Mozambique & CABO DELGADO & 15-19 & 60.37 & 22.09 & 150.69 & RW2 \\ 
  Mozambique & GAZA & 80-84 & 218.86 & 265.84 & 178.16 & HT-Direct \\ 
  Mozambique & GAZA & 80-84 & 219.87 & 186.79 & 257.49 & RW2 \\ 
  Mozambique & GAZA & 85-89 & 202.12 & 240.52 & 168.49 & HT-Direct \\ 
  Mozambique & GAZA & 85-89 & 212.13 & 188.00 & 238.55 & RW2 \\ 
  Mozambique & GAZA & 90-94 & 185.32 & 219.61 & 155.32 & HT-Direct \\ 
  Mozambique & GAZA & 90-94 & 197.90 & 178.18 & 218.99 & RW2 \\ 
  Mozambique & GAZA & 95-99 & 161.56 & 188.50 & 137.82 & HT-Direct \\ 
  Mozambique & GAZA & 95-99 & 172.77 & 155.41 & 191.08 & RW2 \\ 
  Mozambique & GAZA & 00-04 & 126.17 & 153.72 & 102.96 & HT-Direct \\ 
  Mozambique & GAZA & 00-04 & 142.66 & 125.36 & 161.99 & RW2 \\ 
  Mozambique & GAZA & 05-09 & 113.40 & 146.20 & 87.21 & HT-Direct \\ 
  Mozambique & GAZA & 05-09 & 117.72 & 98.16 & 140.87 & RW2 \\ 
  Mozambique & GAZA & 10-14 & 93.54 & 135.74 & 63.50 & HT-Direct \\ 
  Mozambique & GAZA & 10-14 & 98.35 & 74.83 & 128.94 & RW2 \\ 
  Mozambique & GAZA & 15-19 & 81.98 & 33.43 & 188.36 & RW2 \\ 
  Mozambique & INHAMBANE & 80-84 & 224.34 & 273.20 & 182.04 & HT-Direct \\ 
  Mozambique & INHAMBANE & 80-84 & 237.78 & 200.79 & 278.55 & RW2 \\ 
  Mozambique & INHAMBANE & 85-89 & 178.42 & 220.09 & 143.19 & HT-Direct \\ 
  Mozambique & INHAMBANE & 85-89 & 219.21 & 193.75 & 246.76 & RW2 \\ 
  Mozambique & INHAMBANE & 90-94 & 194.09 & 221.90 & 169.02 & HT-Direct \\ 
  Mozambique & INHAMBANE & 90-94 & 193.83 & 176.46 & 212.83 & RW2 \\ 
  Mozambique & INHAMBANE & 95-99 & 154.14 & 172.51 & 137.40 & HT-Direct \\ 
  Mozambique & INHAMBANE & 95-99 & 155.92 & 142.42 & 170.73 & RW2 \\ 
  Mozambique & INHAMBANE & 00-04 & 95.51 & 119.31 & 76.05 & HT-Direct \\ 
  Mozambique & INHAMBANE & 00-04 & 114.22 & 98.92 & 131.45 & RW2 \\ 
  Mozambique & INHAMBANE & 05-09 & 58.48 & 85.67 & 39.54 & HT-Direct \\ 
  Mozambique & INHAMBANE & 05-09 & 82.36 & 64.85 & 103.23 & RW2 \\ 
  Mozambique & INHAMBANE & 10-14 & 34.10 & 80.20 & 14.09 & HT-Direct \\ 
  Mozambique & INHAMBANE & 10-14 & 60.11 & 41.02 & 85.63 & RW2 \\ 
  Mozambique & INHAMBANE & 15-19 & 43.79 & 16.23 & 111.15 & RW2 \\ 
  Mozambique & MANICA & 80-84 & 270.94 & 338.61 & 212.45 & HT-Direct \\ 
  Mozambique & MANICA & 80-84 & 279.87 & 235.33 & 328.85 & RW2 \\ 
  Mozambique & MANICA & 85-89 & 265.20 & 319.32 & 217.33 & HT-Direct \\ 
  Mozambique & MANICA & 85-89 & 261.16 & 230.89 & 293.85 & RW2 \\ 
  Mozambique & MANICA & 90-94 & 214.34 & 244.31 & 187.12 & HT-Direct \\ 
  Mozambique & MANICA & 90-94 & 234.74 & 213.89 & 256.79 & RW2 \\ 
  Mozambique & MANICA & 95-99 & 193.56 & 222.21 & 167.81 & HT-Direct \\ 
  Mozambique & MANICA & 95-99 & 195.46 & 177.96 & 213.95 & RW2 \\ 
  Mozambique & MANICA & 00-04 & 123.43 & 146.75 & 103.37 & HT-Direct \\ 
  Mozambique & MANICA & 00-04 & 151.16 & 134.12 & 170.13 & RW2 \\ 
  Mozambique & MANICA & 05-09 & 104.27 & 132.37 & 81.57 & HT-Direct \\ 
  Mozambique & MANICA & 05-09 & 116.81 & 97.63 & 139.10 & RW2 \\ 
  Mozambique & MANICA & 10-14 & 113.10 & 176.77 & 70.41 & HT-Direct \\ 
  Mozambique & MANICA & 10-14 & 91.76 & 69.23 & 121.55 & RW2 \\ 
  Mozambique & MANICA & 15-19 & 72.01 & 29.18 & 169.07 & RW2 \\ 
  Mozambique & MAPUTO CIDADE & 80-84 & 95.34 & 126.69 & 71.12 & HT-Direct \\ 
  Mozambique & MAPUTO CIDADE & 80-84 & 98.61 & 77.76 & 124.95 & RW2 \\ 
  Mozambique & MAPUTO CIDADE & 85-89 & 99.97 & 127.17 & 78.07 & HT-Direct \\ 
  Mozambique & MAPUTO CIDADE & 85-89 & 102.89 & 87.17 & 121.15 & RW2 \\ 
  Mozambique & MAPUTO CIDADE & 90-94 & 95.91 & 119.13 & 76.83 & HT-Direct \\ 
  Mozambique & MAPUTO CIDADE & 90-94 & 104.07 & 89.90 & 119.77 & RW2 \\ 
  Mozambique & MAPUTO CIDADE & 95-99 & 90.57 & 111.96 & 72.94 & HT-Direct \\ 
  Mozambique & MAPUTO CIDADE & 95-99 & 100.77 & 87.45 & 115.35 & RW2 \\ 
  Mozambique & MAPUTO CIDADE & 00-04 & 92.35 & 112.34 & 75.62 & HT-Direct \\ 
  Mozambique & MAPUTO CIDADE & 00-04 & 94.80 & 81.68 & 109.95 & RW2 \\ 
  Mozambique & MAPUTO CIDADE & 05-09 & 75.06 & 96.89 & 57.83 & HT-Direct \\ 
  Mozambique & MAPUTO CIDADE & 05-09 & 88.22 & 71.52 & 108.91 & RW2 \\ 
  Mozambique & MAPUTO CIDADE & 10-14 & 66.54 & 116.45 & 37.12 & HT-Direct \\ 
  Mozambique & MAPUTO CIDADE & 10-14 & 82.25 & 56.65 & 117.58 & RW2 \\ 
  Mozambique & MAPUTO CIDADE & 15-19 & 76.67 & 27.51 & 194.23 & RW2 \\ 
  Mozambique & MAPUTO PROVINCIA & 80-84 & 134.79 & 190.48 & 93.51 & HT-Direct \\ 
  Mozambique & MAPUTO PROVINCIA & 80-84 & 130.37 & 100.97 & 168.02 & RW2 \\ 
  Mozambique & MAPUTO PROVINCIA & 85-89 & 129.04 & 164.42 & 100.37 & HT-Direct \\ 
  Mozambique & MAPUTO PROVINCIA & 85-89 & 131.48 & 110.64 & 155.98 & RW2 \\ 
  Mozambique & MAPUTO PROVINCIA & 90-94 & 118.37 & 146.97 & 94.73 & HT-Direct \\ 
  Mozambique & MAPUTO PROVINCIA & 90-94 & 128.76 & 111.65 & 147.50 & RW2 \\ 
  Mozambique & MAPUTO PROVINCIA & 95-99 & 93.94 & 118.88 & 73.80 & HT-Direct \\ 
  Mozambique & MAPUTO PROVINCIA & 95-99 & 120.75 & 105.07 & 137.57 & RW2 \\ 
  Mozambique & MAPUTO PROVINCIA & 00-04 & 114.01 & 137.97 & 93.77 & HT-Direct \\ 
  Mozambique & MAPUTO PROVINCIA & 00-04 & 109.85 & 95.19 & 126.51 & RW2 \\ 
  Mozambique & MAPUTO PROVINCIA & 05-09 & 85.20 & 107.47 & 67.21 & HT-Direct \\ 
  Mozambique & MAPUTO PROVINCIA & 05-09 & 99.65 & 81.92 & 121.03 & RW2 \\ 
  Mozambique & MAPUTO PROVINCIA & 10-14 & 76.66 & 133.78 & 42.72 & HT-Direct \\ 
  Mozambique & MAPUTO PROVINCIA & 10-14 & 90.96 & 65.27 & 125.72 & RW2 \\ 
  Mozambique & MAPUTO PROVINCIA & 15-19 & 83.08 & 31.64 & 199.73 & RW2 \\ 
  Mozambique & NAMPULA & 80-84 & 303.38 & 350.91 & 259.70 & HT-Direct \\ 
  Mozambique & NAMPULA & 80-84 & 311.24 & 271.62 & 352.85 & RW2 \\ 
  Mozambique & NAMPULA & 85-89 & 239.61 & 287.83 & 197.23 & HT-Direct \\ 
  Mozambique & NAMPULA & 85-89 & 292.09 & 260.89 & 324.15 & RW2 \\ 
  Mozambique & NAMPULA & 90-94 & 259.15 & 310.41 & 213.73 & HT-Direct \\ 
  Mozambique & NAMPULA & 90-94 & 266.36 & 241.83 & 292.92 & RW2 \\ 
  Mozambique & NAMPULA & 95-99 & 231.42 & 258.29 & 206.56 & HT-Direct \\ 
  Mozambique & NAMPULA & 95-99 & 219.73 & 200.64 & 240.83 & RW2 \\ 
  Mozambique & NAMPULA & 00-04 & 129.86 & 156.74 & 107.00 & HT-Direct \\ 
  Mozambique & NAMPULA & 00-04 & 157.77 & 139.25 & 178.36 & RW2 \\ 
  Mozambique & NAMPULA & 05-09 & 62.71 & 86.48 & 45.14 & HT-Direct \\ 
  Mozambique & NAMPULA & 05-09 & 106.48 & 86.77 & 129.10 & RW2 \\ 
  Mozambique & NAMPULA & 10-14 & 78.29 & 124.54 & 48.27 & HT-Direct \\ 
  Mozambique & NAMPULA & 10-14 & 71.55 & 51.13 & 97.12 & RW2 \\ 
  Mozambique & NAMPULA & 15-19 & 47.50 & 18.09 & 117.77 & RW2 \\ 
  Mozambique & NIASSA & 80-84 & 331.59 & 415.89 & 256.86 & HT-Direct \\ 
  Mozambique & NIASSA & 80-84 & 295.70 & 242.64 & 356.17 & RW2 \\ 
  Mozambique & NIASSA & 85-89 & 230.32 & 292.84 & 177.80 & HT-Direct \\ 
  Mozambique & NIASSA & 85-89 & 271.51 & 234.68 & 311.92 & RW2 \\ 
  Mozambique & NIASSA & 90-94 & 211.33 & 257.42 & 171.59 & HT-Direct \\ 
  Mozambique & NIASSA & 90-94 & 243.93 & 217.22 & 272.44 & RW2 \\ 
  Mozambique & NIASSA & 95-99 & 189.65 & 217.87 & 164.31 & HT-Direct \\ 
  Mozambique & NIASSA & 95-99 & 201.25 & 181.60 & 222.34 & RW2 \\ 
  Mozambique & NIASSA & 00-04 & 152.09 & 188.48 & 121.68 & HT-Direct \\ 
  Mozambique & NIASSA & 00-04 & 147.78 & 129.48 & 168.25 & RW2 \\ 
  Mozambique & NIASSA & 05-09 & 79.42 & 98.47 & 63.79 & HT-Direct \\ 
  Mozambique & NIASSA & 05-09 & 102.99 & 85.43 & 123.57 & RW2 \\ 
  Mozambique & NIASSA & 10-14 & 83.02 & 148.36 & 44.94 & HT-Direct \\ 
  Mozambique & NIASSA & 10-14 & 71.77 & 52.01 & 98.18 & RW2 \\ 
  Mozambique & NIASSA & 15-19 & 49.89 & 18.87 & 123.87 & RW2 \\ 
  Mozambique & SOFALA & 80-84 & 316.07 & 367.96 & 268.39 & HT-Direct \\ 
  Mozambique & SOFALA & 80-84 & 329.17 & 289.94 & 370.76 & RW2 \\ 
  Mozambique & SOFALA & 85-89 & 295.54 & 335.22 & 258.73 & HT-Direct \\ 
  Mozambique & SOFALA & 85-89 & 295.24 & 268.13 & 324.06 & RW2 \\ 
  Mozambique & SOFALA & 90-94 & 230.78 & 266.29 & 198.72 & HT-Direct \\ 
  Mozambique & SOFALA & 90-94 & 254.72 & 233.37 & 277.37 & RW2 \\ 
  Mozambique & SOFALA & 95-99 & 178.11 & 207.25 & 152.28 & HT-Direct \\ 
  Mozambique & SOFALA & 95-99 & 202.55 & 184.46 & 221.79 & RW2 \\ 
  Mozambique & SOFALA & 00-04 & 143.18 & 167.54 & 121.84 & HT-Direct \\ 
  Mozambique & SOFALA & 00-04 & 148.27 & 131.94 & 166.09 & RW2 \\ 
  Mozambique & SOFALA & 05-09 & 97.25 & 128.19 & 73.15 & HT-Direct \\ 
  Mozambique & SOFALA & 05-09 & 106.68 & 89.72 & 126.52 & RW2 \\ 
  Mozambique & SOFALA & 10-14 & 70.63 & 100.77 & 49.02 & HT-Direct \\ 
  Mozambique & SOFALA & 10-14 & 77.38 & 59.86 & 100.01 & RW2 \\ 
  Mozambique & SOFALA & 15-19 & 55.88 & 22.95 & 130.68 & RW2 \\ 
  Mozambique & TETE & 80-84 & 293.29 & 357.31 & 236.51 & HT-Direct \\ 
  Mozambique & TETE & 80-84 & 302.06 & 258.83 & 348.35 & RW2 \\ 
  Mozambique & TETE & 85-89 & 268.28 & 312.54 & 228.20 & HT-Direct \\ 
  Mozambique & TETE & 85-89 & 292.94 & 263.51 & 323.85 & RW2 \\ 
  Mozambique & TETE & 90-94 & 261.51 & 291.39 & 233.70 & HT-Direct \\ 
  Mozambique & TETE & 90-94 & 275.00 & 253.75 & 297.43 & RW2 \\ 
  Mozambique & TETE & 95-99 & 237.47 & 267.11 & 210.18 & HT-Direct \\ 
  Mozambique & TETE & 95-99 & 238.56 & 219.88 & 258.15 & RW2 \\ 
  Mozambique & TETE & 00-04 & 170.39 & 196.52 & 147.11 & HT-Direct \\ 
  Mozambique & TETE & 00-04 & 190.70 & 172.17 & 210.67 & RW2 \\ 
  Mozambique & TETE & 05-09 & 116.42 & 139.45 & 96.77 & HT-Direct \\ 
  Mozambique & TETE & 05-09 & 150.66 & 129.83 & 174.48 & RW2 \\ 
  Mozambique & TETE & 10-14 & 146.31 & 214.37 & 97.18 & HT-Direct \\ 
  Mozambique & TETE & 10-14 & 120.86 & 94.00 & 154.84 & RW2 \\ 
  Mozambique & TETE & 15-19 & 96.78 & 39.92 & 218.44 & RW2 \\ 
  Mozambique & ZAMBEZIA & 80-84 & 271.78 & 326.99 & 222.81 & HT-Direct \\ 
  Mozambique & ZAMBEZIA & 80-84 & 275.33 & 237.04 & 318.43 & RW2 \\ 
  Mozambique & ZAMBEZIA & 85-89 & 263.78 & 311.27 & 221.20 & HT-Direct \\ 
  Mozambique & ZAMBEZIA & 85-89 & 267.44 & 239.63 & 297.54 & RW2 \\ 
  Mozambique & ZAMBEZIA & 90-94 & 246.76 & 279.44 & 216.75 & HT-Direct \\ 
  Mozambique & ZAMBEZIA & 90-94 & 252.59 & 231.52 & 274.83 & RW2 \\ 
  Mozambique & ZAMBEZIA & 95-99 & 192.54 & 218.65 & 168.87 & HT-Direct \\ 
  Mozambique & ZAMBEZIA & 95-99 & 220.71 & 202.74 & 239.46 & RW2 \\ 
  Mozambique & ZAMBEZIA & 00-04 & 150.44 & 175.90 & 128.09 & HT-Direct \\ 
  Mozambique & ZAMBEZIA & 00-04 & 177.47 & 160.21 & 196.21 & RW2 \\ 
  Mozambique & ZAMBEZIA & 05-09 & 143.37 & 166.09 & 123.30 & HT-Direct \\ 
  Mozambique & ZAMBEZIA & 05-09 & 140.24 & 122.06 & 160.49 & RW2 \\ 
  Mozambique & ZAMBEZIA & 10-14 & 88.24 & 129.42 & 59.27 & HT-Direct \\ 
  Mozambique & ZAMBEZIA & 10-14 & 111.34 & 87.59 & 140.93 & RW2 \\ 
  Mozambique & ZAMBEZIA & 15-19 & 88.62 & 36.80 & 197.67 & RW2 \\ 
  Namibia & ALL & 80-84 & 92.76 & 89.63 & 95.98 & IHME \\ 
  Namibia & ALL & 80-84 & 94.40 & 77.46 & 114.60 & RW2 \\ 
  Namibia & ALL & 80-84 & 94.25 & 90.45 & 98.04 & UN \\ 
  Namibia & ALL & 85-89 & 77.18 & 74.67 & 79.77 & IHME \\ 
  Namibia & ALL & 85-89 & 80.38 & 69.75 & 92.45 & RW2 \\ 
  Namibia & ALL & 85-89 & 81.00 & 77.97 & 84.32 & UN \\ 
  Namibia & ALL & 90-94 & 66.03 & 64.06 & 68.03 & IHME \\ 
  Namibia & ALL & 90-94 & 71.83 & 64.15 & 80.34 & RW2 \\ 
  Namibia & ALL & 90-94 & 70.97 & 68.09 & 74.11 & UN \\ 
  Namibia & ALL & 95-99 & 61.14 & 59.16 & 63.31 & IHME \\ 
  Namibia & ALL & 95-99 & 71.72 & 62.53 & 81.83 & RW2 \\ 
  Namibia & ALL & 95-99 & 72.65 & 69.21 & 75.92 & UN \\ 
  Namibia & ALL & 00-04 & 64.68 & 62.33 & 67.00 & IHME \\ 
  Namibia & ALL & 00-04 & 76.05 & 68.56 & 84.46 & RW2 \\ 
  Namibia & ALL & 00-04 & 75.58 & 72.49 & 78.81 & UN \\ 
  Namibia & ALL & 05-09 & 61.58 & 58.75 & 64.51 & IHME \\ 
  Namibia & ALL & 05-09 & 64.42 & 58.34 & 71.06 & RW2 \\ 
  Namibia & ALL & 05-09 & 64.51 & 61.04 & 68.51 & UN \\ 
  Namibia & ALL & 10-14 & 48.89 & 45.65 & 52.49 & IHME \\ 
  Namibia & ALL & 10-14 & 50.33 & 42.12 & 59.97 & RW2 \\ 
  Namibia & ALL & 10-14 & 50.39 & 45.14 & 56.48 & UN \\ 
  Namibia & CAPRIVI & 80-84 & 101.08 & 190.45 & 51.00 & HT-Direct \\ 
  Namibia & CAPRIVI & 80-84 & 108.11 & 66.66 & 170.86 & RW2 \\ 
  Namibia & CAPRIVI & 85-89 & 80.31 & 125.07 & 50.64 & HT-Direct \\ 
  Namibia & CAPRIVI & 85-89 & 91.98 & 66.08 & 125.71 & RW2 \\ 
  Namibia & CAPRIVI & 90-94 & 74.33 & 106.00 & 51.58 & HT-Direct \\ 
  Namibia & CAPRIVI & 90-94 & 83.48 & 64.89 & 106.54 & RW2 \\ 
  Namibia & CAPRIVI & 95-99 & 64.11 & 97.96 & 41.42 & HT-Direct \\ 
  Namibia & CAPRIVI & 95-99 & 90.52 & 71.82 & 113.18 & RW2 \\ 
  Namibia & CAPRIVI & 00-04 & 113.30 & 161.55 & 78.11 & HT-Direct \\ 
  Namibia & CAPRIVI & 00-04 & 94.29 & 76.36 & 116.09 & RW2 \\ 
  Namibia & CAPRIVI & 05-09 & 79.73 & 109.06 & 57.78 & HT-Direct \\ 
  Namibia & CAPRIVI & 05-09 & 83.44 & 66.28 & 104.49 & RW2 \\ 
  Namibia & CAPRIVI & 10-14 & 76.48 & 133.14 & 42.74 & HT-Direct \\ 
  Namibia & CAPRIVI & 10-14 & 71.54 & 49.75 & 101.78 & RW2 \\ 
  Namibia & CAPRIVI & 15-19 & 61.32 & 21.45 & 164.58 & RW2 \\ 
  Namibia & ERONGO & 80-84 & 42.21 & 81.94 & 21.29 & HT-Direct \\ 
  Namibia & ERONGO & 80-84 & 57.82 & 36.62 & 90.84 & RW2 \\ 
  Namibia & ERONGO & 85-89 & 64.51 & 98.04 & 41.91 & HT-Direct \\ 
  Namibia & ERONGO & 85-89 & 51.50 & 36.83 & 71.78 & RW2 \\ 
  Namibia & ERONGO & 90-94 & 33.31 & 56.63 & 19.39 & HT-Direct \\ 
  Namibia & ERONGO & 90-94 & 48.88 & 37.35 & 63.73 & RW2 \\ 
  Namibia & ERONGO & 95-99 & 34.55 & 57.34 & 20.62 & HT-Direct \\ 
  Namibia & ERONGO & 95-99 & 55.24 & 43.54 & 69.44 & RW2 \\ 
  Namibia & ERONGO & 00-04 & 56.20 & 85.17 & 36.69 & HT-Direct \\ 
  Namibia & ERONGO & 00-04 & 59.64 & 47.70 & 74.11 & RW2 \\ 
  Namibia & ERONGO & 05-09 & 58.43 & 88.99 & 37.93 & HT-Direct \\ 
  Namibia & ERONGO & 05-09 & 55.05 & 42.61 & 70.64 & RW2 \\ 
  Namibia & ERONGO & 10-14 & 54.86 & 90.32 & 32.82 & HT-Direct \\ 
  Namibia & ERONGO & 10-14 & 49.62 & 35.01 & 69.77 & RW2 \\ 
  Namibia & ERONGO & 15-19 & 45.20 & 16.77 & 116.35 & RW2 \\ 
  Namibia & HARDAP & 80-84 & 131.68 & 213.15 & 78.26 & HT-Direct \\ 
  Namibia & HARDAP & 80-84 & 126.32 & 84.53 & 184.16 & RW2 \\ 
  Namibia & HARDAP & 85-89 & 76.08 & 116.26 & 49.02 & HT-Direct \\ 
  Namibia & HARDAP & 85-89 & 92.56 & 68.38 & 123.74 & RW2 \\ 
  Namibia & HARDAP & 90-94 & 57.58 & 83.96 & 39.13 & HT-Direct \\ 
  Namibia & HARDAP & 90-94 & 71.81 & 56.36 & 91.00 & RW2 \\ 
  Namibia & HARDAP & 95-99 & 66.81 & 108.39 & 40.46 & HT-Direct \\ 
  Namibia & HARDAP & 95-99 & 65.56 & 51.92 & 82.14 & RW2 \\ 
  Namibia & HARDAP & 00-04 & 50.20 & 85.41 & 29.04 & HT-Direct \\ 
  Namibia & HARDAP & 00-04 & 56.86 & 44.44 & 72.39 & RW2 \\ 
  Namibia & HARDAP & 05-09 & 45.87 & 73.59 & 28.27 & HT-Direct \\ 
  Namibia & HARDAP & 05-09 & 41.87 & 31.09 & 56.34 & RW2 \\ 
  Namibia & HARDAP & 10-14 & 30.41 & 60.43 & 15.07 & HT-Direct \\ 
  Namibia & HARDAP & 10-14 & 29.91 & 19.75 & 45.17 & RW2 \\ 
  Namibia & HARDAP & 15-19 & 21.39 & 7.40 & 61.15 & RW2 \\ 
  Namibia & KARAS & 80-84 & 107.64 & 154.83 & 73.57 & HT-Direct \\ 
  Namibia & KARAS & 80-84 & 117.21 & 82.36 & 164.30 & RW2 \\ 
  Namibia & KARAS & 85-89 & 74.77 & 109.87 & 50.25 & HT-Direct \\ 
  Namibia & KARAS & 85-89 & 89.16 & 68.45 & 115.35 & RW2 \\ 
  Namibia & KARAS & 90-94 & 74.91 & 103.52 & 53.74 & HT-Direct \\ 
  Namibia & KARAS & 90-94 & 72.02 & 57.02 & 90.14 & RW2 \\ 
  Namibia & KARAS & 95-99 & 41.06 & 64.96 & 25.72 & HT-Direct \\ 
  Namibia & KARAS & 95-99 & 69.05 & 53.77 & 87.08 & RW2 \\ 
  Namibia & KARAS & 00-04 & 57.75 & 87.00 & 37.93 & HT-Direct \\ 
  Namibia & KARAS & 00-04 & 64.22 & 50.10 & 81.67 & RW2 \\ 
  Namibia & KARAS & 05-09 & 59.32 & 90.66 & 38.36 & HT-Direct \\ 
  Namibia & KARAS & 05-09 & 51.91 & 38.87 & 68.92 & RW2 \\ 
  Namibia & KARAS & 10-14 & 46.36 & 93.53 & 22.39 & HT-Direct \\ 
  Namibia & KARAS & 10-14 & 40.94 & 26.46 & 63.81 & RW2 \\ 
  Namibia & KARAS & 15-19 & 32.32 & 10.70 & 97.43 & RW2 \\ 
  Namibia & KAVANGO & 80-84 & 111.48 & 175.10 & 69.04 & HT-Direct \\ 
  Namibia & KAVANGO & 80-84 & 119.90 & 85.77 & 166.03 & RW2 \\ 
  Namibia & KAVANGO & 85-89 & 103.51 & 138.03 & 76.85 & HT-Direct \\ 
  Namibia & KAVANGO & 85-89 & 100.73 & 79.95 & 126.45 & RW2 \\ 
  Namibia & KAVANGO & 90-94 & 77.45 & 103.94 & 57.27 & HT-Direct \\ 
  Namibia & KAVANGO & 90-94 & 90.03 & 75.37 & 107.24 & RW2 \\ 
  Namibia & KAVANGO & 95-99 & 83.17 & 99.83 & 69.08 & HT-Direct \\ 
  Namibia & KAVANGO & 95-99 & 95.72 & 82.46 & 110.67 & RW2 \\ 
  Namibia & KAVANGO & 00-04 & 68.11 & 90.04 & 51.21 & HT-Direct \\ 
  Namibia & KAVANGO & 00-04 & 97.85 & 84.06 & 113.39 & RW2 \\ 
  Namibia & KAVANGO & 05-09 & 106.12 & 133.89 & 83.56 & HT-Direct \\ 
  Namibia & KAVANGO & 05-09 & 85.95 & 72.12 & 102.20 & RW2 \\ 
  Namibia & KAVANGO & 10-14 & 77.96 & 122.95 & 48.53 & HT-Direct \\ 
  Namibia & KAVANGO & 10-14 & 73.23 & 55.45 & 96.60 & RW2 \\ 
  Namibia & KAVANGO & 15-19 & 62.41 & 23.98 & 155.00 & RW2 \\ 
  Namibia & KHOMAS & 80-84 & 79.70 & 145.33 & 42.25 & HT-Direct \\ 
  Namibia & KHOMAS & 80-84 & 78.53 & 49.17 & 122.95 & RW2 \\ 
  Namibia & KHOMAS & 85-89 & 43.35 & 76.73 & 24.12 & HT-Direct \\ 
  Namibia & KHOMAS & 85-89 & 64.27 & 45.21 & 90.31 & RW2 \\ 
  Namibia & KHOMAS & 90-94 & 52.37 & 77.59 & 35.04 & HT-Direct \\ 
  Namibia & KHOMAS & 90-94 & 56.01 & 42.89 & 72.80 & RW2 \\ 
  Namibia & KHOMAS & 95-99 & 47.14 & 82.78 & 26.40 & HT-Direct \\ 
  Namibia & KHOMAS & 95-99 & 57.54 & 45.80 & 72.27 & RW2 \\ 
  Namibia & KHOMAS & 00-04 & 51.76 & 79.01 & 33.56 & HT-Direct \\ 
  Namibia & KHOMAS & 00-04 & 55.74 & 44.89 & 69.05 & RW2 \\ 
  Namibia & KHOMAS & 05-09 & 47.98 & 66.85 & 34.24 & HT-Direct \\ 
  Namibia & KHOMAS & 05-09 & 45.67 & 35.54 & 58.42 & RW2 \\ 
  Namibia & KHOMAS & 10-14 & 33.69 & 71.75 & 15.49 & HT-Direct \\ 
  Namibia & KHOMAS & 10-14 & 36.34 & 24.80 & 52.90 & RW2 \\ 
  Namibia & KHOMAS & 15-19 & 29.06 & 10.19 & 80.25 & RW2 \\ 
  Namibia & KUNENE & 80-84 & 83.05 & 145.22 & 46.06 & HT-Direct \\ 
  Namibia & KUNENE & 80-84 & 114.32 & 77.19 & 167.01 & RW2 \\ 
  Namibia & KUNENE & 85-89 & 96.75 & 144.44 & 63.63 & HT-Direct \\ 
  Namibia & KUNENE & 85-89 & 89.82 & 67.09 & 119.37 & RW2 \\ 
  Namibia & KUNENE & 90-94 & 85.35 & 119.76 & 60.15 & HT-Direct \\ 
  Namibia & KUNENE & 90-94 & 74.40 & 59.27 & 92.91 & RW2 \\ 
  Namibia & KUNENE & 95-99 & 53.44 & 78.09 & 36.26 & HT-Direct \\ 
  Namibia & KUNENE & 95-99 & 72.64 & 59.08 & 88.36 & RW2 \\ 
  Namibia & KUNENE & 00-04 & 36.18 & 58.36 & 22.24 & HT-Direct \\ 
  Namibia & KUNENE & 00-04 & 67.97 & 55.12 & 83.12 & RW2 \\ 
  Namibia & KUNENE & 05-09 & 50.84 & 74.05 & 34.64 & HT-Direct \\ 
  Namibia & KUNENE & 05-09 & 54.53 & 42.98 & 69.02 & RW2 \\ 
  Namibia & KUNENE & 10-14 & 68.24 & 109.19 & 41.92 & HT-Direct \\ 
  Namibia & KUNENE & 10-14 & 42.94 & 30.92 & 59.60 & RW2 \\ 
  Namibia & KUNENE & 15-19 & 33.83 & 12.59 & 89.20 & RW2 \\ 
  Namibia & OHANGWENA & 80-84 & 149.73 & 211.99 & 103.37 & HT-Direct \\ 
  Namibia & OHANGWENA & 80-84 & 135.95 & 98.94 & 184.19 & RW2 \\ 
  Namibia & OHANGWENA & 85-89 & 74.38 & 107.77 & 50.74 & HT-Direct \\ 
  Namibia & OHANGWENA & 85-89 & 111.31 & 87.34 & 141.25 & RW2 \\ 
  Namibia & OHANGWENA & 90-94 & 87.90 & 117.98 & 64.93 & HT-Direct \\ 
  Namibia & OHANGWENA & 90-94 & 97.01 & 80.03 & 116.95 & RW2 \\ 
  Namibia & OHANGWENA & 95-99 & 71.00 & 96.98 & 51.58 & HT-Direct \\ 
  Namibia & OHANGWENA & 95-99 & 100.04 & 84.37 & 118.38 & RW2 \\ 
  Namibia & OHANGWENA & 00-04 & 123.21 & 155.85 & 96.63 & HT-Direct \\ 
  Namibia & OHANGWENA & 00-04 & 97.87 & 83.23 & 114.97 & RW2 \\ 
  Namibia & OHANGWENA & 05-09 & 70.09 & 94.42 & 51.67 & HT-Direct \\ 
  Namibia & OHANGWENA & 05-09 & 80.39 & 65.76 & 98.05 & RW2 \\ 
  Namibia & OHANGWENA & 10-14 & 71.68 & 124.08 & 40.39 & HT-Direct \\ 
  Namibia & OHANGWENA & 10-14 & 63.86 & 46.35 & 86.84 & RW2 \\ 
  Namibia & OHANGWENA & 15-19 & 50.62 & 18.71 & 128.87 & RW2 \\ 
  Namibia & OMAHEKE & 80-84 & 61.73 & 118.97 & 31.06 & HT-Direct \\ 
  Namibia & OMAHEKE & 80-84 & 86.79 & 54.47 & 131.47 & RW2 \\ 
  Namibia & OMAHEKE & 85-89 & 44.68 & 69.65 & 28.39 & HT-Direct \\ 
  Namibia & OMAHEKE & 85-89 & 76.17 & 56.52 & 101.42 & RW2 \\ 
  Namibia & OMAHEKE & 90-94 & 70.93 & 93.48 & 53.50 & HT-Direct \\ 
  Namibia & OMAHEKE & 90-94 & 70.96 & 58.03 & 86.63 & RW2 \\ 
  Namibia & OMAHEKE & 95-99 & 80.01 & 98.54 & 64.72 & HT-Direct \\ 
  Namibia & OMAHEKE & 95-99 & 76.15 & 63.86 & 91.26 & RW2 \\ 
  Namibia & OMAHEKE & 00-04 & 57.08 & 99.30 & 32.17 & HT-Direct \\ 
  Namibia & OMAHEKE & 00-04 & 75.42 & 60.88 & 93.34 & RW2 \\ 
  Namibia & OMAHEKE & 05-09 & 57.20 & 88.46 & 36.54 & HT-Direct \\ 
  Namibia & OMAHEKE & 05-09 & 62.26 & 47.08 & 82.18 & RW2 \\ 
  Namibia & OMAHEKE & 10-14 & 46.25 & 90.12 & 23.20 & HT-Direct \\ 
  Namibia & OMAHEKE & 10-14 & 49.78 & 32.70 & 74.17 & RW2 \\ 
  Namibia & OMAHEKE & 15-19 & 39.77 & 13.67 & 108.86 & RW2 \\ 
  Namibia & OMUSATI & 80-84 & 102.85 & 187.78 & 53.79 & HT-Direct \\ 
  Namibia & OMUSATI & 80-84 & 127.27 & 83.51 & 189.21 & RW2 \\ 
  Namibia & OMUSATI & 85-89 & 101.30 & 150.06 & 67.13 & HT-Direct \\ 
  Namibia & OMUSATI & 85-89 & 97.76 & 72.19 & 131.35 & RW2 \\ 
  Namibia & OMUSATI & 90-94 & 57.06 & 83.48 & 38.65 & HT-Direct \\ 
  Namibia & OMUSATI & 90-94 & 79.38 & 63.11 & 99.16 & RW2 \\ 
  Namibia & OMUSATI & 95-99 & 68.22 & 91.13 & 50.75 & HT-Direct \\ 
  Namibia & OMUSATI & 95-99 & 76.39 & 62.78 & 92.30 & RW2 \\ 
  Namibia & OMUSATI & 00-04 & 69.06 & 95.77 & 49.39 & HT-Direct \\ 
  Namibia & OMUSATI & 00-04 & 69.86 & 57.67 & 84.39 & RW2 \\ 
  Namibia & OMUSATI & 05-09 & 57.76 & 80.55 & 41.12 & HT-Direct \\ 
  Namibia & OMUSATI & 05-09 & 53.77 & 42.27 & 68.11 & RW2 \\ 
  Namibia & OMUSATI & 10-14 & 32.32 & 67.73 & 15.12 & HT-Direct \\ 
  Namibia & OMUSATI & 10-14 & 39.86 & 27.37 & 57.80 & RW2 \\ 
  Namibia & OMUSATI & 15-19 & 29.57 & 10.35 & 82.04 & RW2 \\ 
  Namibia & OSHANA & 80-84 & 105.26 & 154.18 & 70.57 & HT-Direct \\ 
  Namibia & OSHANA & 80-84 & 118.10 & 83.91 & 163.85 & RW2 \\ 
  Namibia & OSHANA & 85-89 & 90.19 & 135.89 & 58.82 & HT-Direct \\ 
  Namibia & OSHANA & 85-89 & 95.87 & 74.02 & 123.76 & RW2 \\ 
  Namibia & OSHANA & 90-94 & 64.90 & 88.85 & 47.06 & HT-Direct \\ 
  Namibia & OSHANA & 90-94 & 82.76 & 67.61 & 100.76 & RW2 \\ 
  Namibia & OSHANA & 95-99 & 81.35 & 107.09 & 61.37 & HT-Direct \\ 
  Namibia & OSHANA & 95-99 & 84.22 & 70.33 & 100.64 & RW2 \\ 
  Namibia & OSHANA & 00-04 & 76.67 & 106.52 & 54.67 & HT-Direct \\ 
  Namibia & OSHANA & 00-04 & 81.17 & 66.94 & 98.28 & RW2 \\ 
  Namibia & OSHANA & 05-09 & 74.36 & 108.05 & 50.58 & HT-Direct \\ 
  Namibia & OSHANA & 05-09 & 65.81 & 51.29 & 84.57 & RW2 \\ 
  Namibia & OSHANA & 10-14 & 39.32 & 88.60 & 16.94 & HT-Direct \\ 
  Namibia & OSHANA & 10-14 & 51.68 & 35.03 & 75.04 & RW2 \\ 
  Namibia & OSHANA & 15-19 & 40.50 & 14.27 & 110.29 & RW2 \\ 
  Namibia & OSHIKOTO & 80-84 & 38.42 & 79.53 & 18.14 & HT-Direct \\ 
  Namibia & OSHIKOTO & 80-84 & 87.42 & 57.73 & 128.14 & RW2 \\ 
  Namibia & OSHIKOTO & 85-89 & 64.98 & 92.35 & 45.32 & HT-Direct \\ 
  Namibia & OSHIKOTO & 85-89 & 78.89 & 60.01 & 103.00 & RW2 \\ 
  Namibia & OSHIKOTO & 90-94 & 79.69 & 108.33 & 58.12 & HT-Direct \\ 
  Namibia & OSHIKOTO & 90-94 & 75.20 & 61.60 & 91.54 & RW2 \\ 
  Namibia & OSHIKOTO & 95-99 & 76.50 & 97.17 & 59.94 & HT-Direct \\ 
  Namibia & OSHIKOTO & 95-99 & 83.98 & 71.21 & 98.93 & RW2 \\ 
  Namibia & OSHIKOTO & 00-04 & 92.63 & 118.77 & 71.78 & HT-Direct \\ 
  Namibia & OSHIKOTO & 00-04 & 87.92 & 74.81 & 103.29 & RW2 \\ 
  Namibia & OSHIKOTO & 05-09 & 66.24 & 94.08 & 46.22 & HT-Direct \\ 
  Namibia & OSHIKOTO & 05-09 & 77.40 & 62.78 & 95.17 & RW2 \\ 
  Namibia & OSHIKOTO & 10-14 & 67.35 & 108.58 & 41.06 & HT-Direct \\ 
  Namibia & OSHIKOTO & 10-14 & 66.15 & 48.45 & 89.35 & RW2 \\ 
  Namibia & OSHIKOTO & 15-19 & 56.47 & 21.38 & 142.25 & RW2 \\ 
  Namibia & OTJOZONDJUPA & 80-84 & 65.78 & 121.48 & 34.62 & HT-Direct \\ 
  Namibia & OTJOZONDJUPA & 80-84 & 82.84 & 55.47 & 120.75 & RW2 \\ 
  Namibia & OTJOZONDJUPA & 85-89 & 62.43 & 89.12 & 43.35 & HT-Direct \\ 
  Namibia & OTJOZONDJUPA & 85-89 & 70.44 & 53.33 & 92.29 & RW2 \\ 
  Namibia & OTJOZONDJUPA & 90-94 & 55.50 & 76.66 & 39.93 & HT-Direct \\ 
  Namibia & OTJOZONDJUPA & 90-94 & 63.69 & 51.56 & 78.36 & RW2 \\ 
  Namibia & OTJOZONDJUPA & 95-99 & 45.17 & 68.64 & 29.48 & HT-Direct \\ 
  Namibia & OTJOZONDJUPA & 95-99 & 67.66 & 56.23 & 81.30 & RW2 \\ 
  Namibia & OTJOZONDJUPA & 00-04 & 81.00 & 106.34 & 61.28 & HT-Direct \\ 
  Namibia & OTJOZONDJUPA & 00-04 & 67.91 & 56.75 & 81.12 & RW2 \\ 
  Namibia & OTJOZONDJUPA & 05-09 & 52.05 & 76.85 & 34.96 & HT-Direct \\ 
  Namibia & OTJOZONDJUPA & 05-09 & 57.32 & 45.38 & 72.19 & RW2 \\ 
  Namibia & OTJOZONDJUPA & 10-14 & 41.29 & 73.38 & 22.89 & HT-Direct \\ 
  Namibia & OTJOZONDJUPA & 10-14 & 46.92 & 33.40 & 65.46 & RW2 \\ 
  Namibia & OTJOZONDJUPA & 15-19 & 38.65 & 14.27 & 100.95 & RW2 \\ 
  Niger & ALL & 80-84 & 313.59 & 309.88 & 317.33 & IHME \\ 
  Niger & ALL & 80-84 & 323.38 & 310.60 & 336.43 & RW2 \\ 
  Niger & ALL & 80-84 & 323.45 & 313.75 & 332.99 & UN \\ 
  Niger & ALL & 85-89 & 306.35 & 303.06 & 309.70 & IHME \\ 
  Niger & ALL & 85-89 & 335.58 & 324.87 & 346.43 & RW2 \\ 
  Niger & ALL & 85-89 & 335.45 & 325.61 & 344.98 & UN \\ 
  Niger & ALL & 90-94 & 285.11 & 281.84 & 288.73 & IHME \\ 
  Niger & ALL & 90-94 & 311.54 & 301.20 & 322.09 & RW2 \\ 
  Niger & ALL & 90-94 & 311.64 & 303.32 & 320.12 & UN \\ 
  Niger & ALL & 95-99 & 248.85 & 245.15 & 252.33 & IHME \\ 
  Niger & ALL & 95-99 & 255.79 & 246.55 & 265.17 & RW2 \\ 
  Niger & ALL & 95-99 & 255.87 & 248.96 & 263.81 & UN \\ 
  Niger & ALL & 00-04 & 203.87 & 200.76 & 207.10 & IHME \\ 
  Niger & ALL & 00-04 & 206.12 & 197.38 & 215.21 & RW2 \\ 
  Niger & ALL & 00-04 & 205.98 & 199.46 & 212.56 & UN \\ 
  Niger & ALL & 05-09 & 158.01 & 154.33 & 161.76 & IHME \\ 
  Niger & ALL & 05-09 & 151.01 & 142.43 & 160.00 & RW2 \\ 
  Niger & ALL & 05-09 & 151.08 & 144.77 & 157.85 & UN \\ 
  Niger & ALL & 10-14 & 127.02 & 121.93 & 131.92 & IHME \\ 
  Niger & ALL & 10-14 & 104.71 & 40.31 & 241.38 & RW2 \\ 
  Niger & ALL & 10-14 & 110.69 & 101.57 & 121.65 & UN \\ 
  Niger & DOSSO & 80-84 & 270.51 & 295.75 & 246.66 & HT-Direct \\ 
  Niger & DOSSO & 80-84 & 272.11 & 249.69 & 296.03 & RW2 \\ 
  Niger & DOSSO & 85-89 & 276.17 & 298.99 & 254.46 & HT-Direct \\ 
  Niger & DOSSO & 85-89 & 281.29 & 264.79 & 298.71 & RW2 \\ 
  Niger & DOSSO & 90-94 & 237.39 & 256.62 & 219.18 & HT-Direct \\ 
  Niger & DOSSO & 90-94 & 259.36 & 243.02 & 275.58 & RW2 \\ 
  Niger & DOSSO & 95-99 & 225.22 & 246.18 & 205.56 & HT-Direct \\ 
  Niger & DOSSO & 95-99 & 229.97 & 215.00 & 245.26 & RW2 \\ 
  Niger & DOSSO & 00-04 & 214.44 & 231.03 & 198.74 & HT-Direct \\ 
  Niger & DOSSO & 00-04 & 214.48 & 201.91 & 227.82 & RW2 \\ 
  Niger & DOSSO & 05-09 & 185.22 & 203.99 & 167.81 & HT-Direct \\ 
  Niger & DOSSO & 05-09 & 188.53 & 171.96 & 206.53 & RW2 \\ 
  Niger & DOSSO & 10-14 & 160.11 & 71.27 & 322.36 & RW2 \\ 
  Niger & DOSSO & 15-19 & 134.17 & 15.05 & 603.94 & RW2 \\ 
  Niger & MARADI & 80-84 & 369.76 & 393.72 & 346.42 & HT-Direct \\ 
  Niger & MARADI & 80-84 & 375.33 & 352.86 & 398.35 & RW2 \\ 
  Niger & MARADI & 85-89 & 375.22 & 395.04 & 355.80 & HT-Direct \\ 
  Niger & MARADI & 85-89 & 398.54 & 381.57 & 415.32 & RW2 \\ 
  Niger & MARADI & 90-94 & 376.13 & 395.11 & 357.51 & HT-Direct \\ 
  Niger & MARADI & 90-94 & 382.33 & 366.17 & 399.22 & RW2 \\ 
  Niger & MARADI & 95-99 & 305.75 & 324.28 & 287.82 & HT-Direct \\ 
  Niger & MARADI & 95-99 & 314.46 & 299.29 & 330.54 & RW2 \\ 
  Niger & MARADI & 00-04 & 236.53 & 254.91 & 219.08 & HT-Direct \\ 
  Niger & MARADI & 00-04 & 237.16 & 223.37 & 251.57 & RW2 \\ 
  Niger & MARADI & 05-09 & 144.75 & 161.41 & 129.55 & HT-Direct \\ 
  Niger & MARADI & 05-09 & 153.00 & 137.78 & 169.23 & RW2 \\ 
  Niger & MARADI & 10-14 & 90.32 & 37.86 & 197.77 & RW2 \\ 
  Niger & MARADI & 15-19 & 51.11 & 5.17 & 351.10 & RW2 \\ 
  Niger & NIAMEY & 80-84 & 152.87 & 174.93 & 133.14 & HT-Direct \\ 
  Niger & NIAMEY & 80-84 & 150.60 & 132.52 & 170.68 & RW2 \\ 
  Niger & NIAMEY & 85-89 & 150.39 & 168.75 & 133.71 & HT-Direct \\ 
  Niger & NIAMEY & 85-89 & 163.96 & 149.81 & 178.84 & RW2 \\ 
  Niger & NIAMEY & 90-94 & 151.82 & 171.61 & 133.94 & HT-Direct \\ 
  Niger & NIAMEY & 90-94 & 156.24 & 142.28 & 171.19 & RW2 \\ 
  Niger & NIAMEY & 95-99 & 136.59 & 156.34 & 118.98 & HT-Direct \\ 
  Niger & NIAMEY & 95-99 & 137.68 & 124.62 & 152.15 & RW2 \\ 
  Niger & NIAMEY & 00-04 & 123.33 & 144.00 & 105.27 & HT-Direct \\ 
  Niger & NIAMEY & 00-04 & 122.21 & 108.74 & 137.27 & RW2 \\ 
  Niger & NIAMEY & 05-09 & 88.35 & 116.13 & 66.71 & HT-Direct \\ 
  Niger & NIAMEY & 05-09 & 98.41 & 78.89 & 121.94 & RW2 \\ 
  Niger & NIAMEY & 10-14 & 76.00 & 29.03 & 181.98 & RW2 \\ 
  Niger & NIAMEY & 15-19 & 58.40 & 5.33 & 405.34 & RW2 \\ 
  Niger & TASHOUA/AGADEZ & 80-84 & 320.93 & 347.19 & 295.75 & HT-Direct \\ 
  Niger & TASHOUA/AGADEZ & 80-84 & 328.74 & 305.67 & 352.76 & RW2 \\ 
  Niger & TASHOUA/AGADEZ & 85-89 & 322.83 & 346.89 & 299.67 & HT-Direct \\ 
  Niger & TASHOUA/AGADEZ & 85-89 & 334.35 & 316.88 & 352.25 & RW2 \\ 
  Niger & TASHOUA/AGADEZ & 90-94 & 294.34 & 318.36 & 271.42 & HT-Direct \\ 
  Niger & TASHOUA/AGADEZ & 90-94 & 301.01 & 284.76 & 318.01 & RW2 \\ 
  Niger & TASHOUA/AGADEZ & 95-99 & 230.70 & 248.34 & 213.96 & HT-Direct \\ 
  Niger & TASHOUA/AGADEZ & 95-99 & 242.58 & 228.70 & 256.81 & RW2 \\ 
  Niger & TASHOUA/AGADEZ & 00-04 & 193.26 & 217.28 & 171.32 & HT-Direct \\ 
  Niger & TASHOUA/AGADEZ & 00-04 & 193.51 & 179.50 & 208.45 & RW2 \\ 
  Niger & TASHOUA/AGADEZ & 05-09 & 136.33 & 155.42 & 119.25 & HT-Direct \\ 
  Niger & TASHOUA/AGADEZ & 05-09 & 139.86 & 124.65 & 156.53 & RW2 \\ 
  Niger & TASHOUA/AGADEZ & 10-14 & 95.44 & 40.48 & 208.86 & RW2 \\ 
  Niger & TASHOUA/AGADEZ & 15-19 & 64.34 & 6.69 & 403.03 & RW2 \\ 
  Niger & TILLABERI & 80-84 & 247.86 & 275.21 & 222.39 & HT-Direct \\ 
  Niger & TILLABERI & 80-84 & 276.19 & 251.62 & 301.68 & RW2 \\ 
  Niger & TILLABERI & 85-89 & 313.85 & 336.75 & 291.82 & HT-Direct \\ 
  Niger & TILLABERI & 85-89 & 292.60 & 275.76 & 310.81 & RW2 \\ 
  Niger & TILLABERI & 90-94 & 235.83 & 256.34 & 216.48 & HT-Direct \\ 
  Niger & TILLABERI & 90-94 & 263.56 & 247.50 & 279.90 & RW2 \\ 
  Niger & TILLABERI & 95-99 & 210.20 & 234.65 & 187.67 & HT-Direct \\ 
  Niger & TILLABERI & 95-99 & 219.95 & 204.28 & 235.88 & RW2 \\ 
  Niger & TILLABERI & 00-04 & 187.99 & 204.92 & 172.16 & HT-Direct \\ 
  Niger & TILLABERI & 00-04 & 189.53 & 176.87 & 202.94 & RW2 \\ 
  Niger & TILLABERI & 05-09 & 152.63 & 174.92 & 132.73 & HT-Direct \\ 
  Niger & TILLABERI & 05-09 & 152.42 & 135.48 & 171.32 & RW2 \\ 
  Niger & TILLABERI & 10-14 & 117.59 & 50.01 & 251.08 & RW2 \\ 
  Niger & TILLABERI & 15-19 & 89.90 & 9.66 & 492.96 & RW2 \\ 
  Niger & ZINDA/DIFFA & 80-84 & 360.69 & 392.59 & 329.98 & HT-Direct \\ 
  Niger & ZINDA/DIFFA & 80-84 & 368.98 & 340.90 & 398.49 & RW2 \\ 
  Niger & ZINDA/DIFFA & 85-89 & 367.41 & 392.18 & 343.33 & HT-Direct \\ 
  Niger & ZINDA/DIFFA & 85-89 & 376.26 & 357.19 & 395.77 & RW2 \\ 
  Niger & ZINDA/DIFFA & 90-94 & 329.65 & 352.07 & 307.99 & HT-Direct \\ 
  Niger & ZINDA/DIFFA & 90-94 & 344.43 & 326.77 & 362.58 & RW2 \\ 
  Niger & ZINDA/DIFFA & 95-99 & 263.60 & 284.43 & 243.78 & HT-Direct \\ 
  Niger & ZINDA/DIFFA & 95-99 & 279.07 & 262.87 & 295.55 & RW2 \\ 
  Niger & ZINDA/DIFFA & 00-04 & 223.15 & 245.37 & 202.41 & HT-Direct \\ 
  Niger & ZINDA/DIFFA & 00-04 & 219.19 & 204.06 & 235.18 & RW2 \\ 
  Niger & ZINDA/DIFFA & 05-09 & 148.38 & 168.86 & 130.00 & HT-Direct \\ 
  Niger & ZINDA/DIFFA & 05-09 & 152.73 & 135.91 & 171.28 & RW2 \\ 
  Niger & ZINDA/DIFFA & 10-14 & 99.14 & 41.55 & 217.74 & RW2 \\ 
  Niger & ZINDA/DIFFA & 15-19 & 63.22 & 6.40 & 400.32 & RW2 \\ 
  Nigeria & ALL & 80-84 & 218.11 & 214.92 & 221.28 & IHME \\ 
  Nigeria & ALL & 80-84 & 210.64 & 200.57 & 221.08 & RW2 \\ 
  Nigeria & ALL & 80-84 & 210.74 & 203.42 & 218.05 & UN \\ 
  Nigeria & ALL & 85-89 & 210.74 & 207.86 & 213.88 & IHME \\ 
  Nigeria & ALL & 85-89 & 211.64 & 203.72 & 219.76 & RW2 \\ 
  Nigeria & ALL & 85-89 & 211.46 & 205.53 & 217.77 & UN \\ 
  Nigeria & ALL & 90-94 & 204.18 & 201.54 & 206.97 & IHME \\ 
  Nigeria & ALL & 90-94 & 211.08 & 203.29 & 219.09 & RW2 \\ 
  Nigeria & ALL & 90-94 & 211.30 & 205.57 & 217.09 & UN \\ 
  Nigeria & ALL & 95-99 & 192.15 & 189.68 & 194.71 & IHME \\ 
  Nigeria & ALL & 95-99 & 200.57 & 194.44 & 206.80 & RW2 \\ 
  Nigeria & ALL & 95-99 & 200.45 & 194.94 & 205.96 & UN \\ 
  Nigeria & ALL & 00-04 & 175.49 & 172.92 & 178.00 & IHME \\ 
  Nigeria & ALL & 00-04 & 175.20 & 169.96 & 180.59 & RW2 \\ 
  Nigeria & ALL & 00-04 & 175.32 & 170.33 & 180.06 & UN \\ 
  Nigeria & ALL & 05-09 & 149.69 & 146.71 & 152.37 & IHME \\ 
  Nigeria & ALL & 05-09 & 146.59 & 141.58 & 151.74 & RW2 \\ 
  Nigeria & ALL & 05-09 & 146.49 & 141.68 & 151.76 & UN \\ 
  Nigeria & ALL & 10-14 & 120.69 & 116.52 & 124.95 & IHME \\ 
  Nigeria & ALL & 10-14 & 120.69 & 113.21 & 128.53 & RW2 \\ 
  Nigeria & ALL & 10-14 & 120.81 & 112.42 & 130.51 & UN \\ 
  Nigeria & NORTH CENTRAL & 80-84 & 169.54 & 194.77 & 146.98 & HT-Direct \\ 
  Nigeria & NORTH CENTRAL & 80-84 & 166.20 & 148.97 & 185.36 & RW2 \\ 
  Nigeria & NORTH CENTRAL & 85-89 & 159.27 & 175.08 & 144.64 & HT-Direct \\ 
  Nigeria & NORTH CENTRAL & 85-89 & 159.12 & 148.43 & 170.30 & RW2 \\ 
  Nigeria & NORTH CENTRAL & 90-94 & 149.02 & 164.20 & 135.02 & HT-Direct \\ 
  Nigeria & NORTH CENTRAL & 90-94 & 149.42 & 140.36 & 158.93 & RW2 \\ 
  Nigeria & NORTH CENTRAL & 95-99 & 145.76 & 157.89 & 134.41 & HT-Direct \\ 
  Nigeria & NORTH CENTRAL & 95-99 & 142.03 & 134.28 & 150.02 & RW2 \\ 
  Nigeria & NORTH CENTRAL & 00-04 & 123.67 & 134.21 & 113.85 & HT-Direct \\ 
  Nigeria & NORTH CENTRAL & 00-04 & 127.37 & 120.50 & 134.57 & RW2 \\ 
  Nigeria & NORTH CENTRAL & 05-09 & 121.60 & 133.53 & 110.60 & HT-Direct \\ 
  Nigeria & NORTH CENTRAL & 05-09 & 109.80 & 102.75 & 117.41 & RW2 \\ 
  Nigeria & NORTH CENTRAL & 10-14 & 82.33 & 94.25 & 71.80 & HT-Direct \\ 
  Nigeria & NORTH CENTRAL & 10-14 & 89.29 & 79.92 & 99.71 & RW2 \\ 
  Nigeria & NORTH CENTRAL & 15-19 & 71.04 & 33.09 & 144.88 & RW2 \\ 
  Nigeria & NORTH EAST & 80-84 & 253.01 & 274.84 & 232.36 & HT-Direct \\ 
  Nigeria & NORTH EAST & 80-84 & 254.96 & 236.08 & 274.61 & RW2 \\ 
  Nigeria & NORTH EAST & 85-89 & 270.05 & 289.94 & 251.04 & HT-Direct \\ 
  Nigeria & NORTH EAST & 85-89 & 265.23 & 251.31 & 279.63 & RW2 \\ 
  Nigeria & NORTH EAST & 90-94 & 255.21 & 272.22 & 238.92 & HT-Direct \\ 
  Nigeria & NORTH EAST & 90-94 & 263.60 & 251.02 & 276.45 & RW2 \\ 
  Nigeria & NORTH EAST & 95-99 & 261.63 & 274.25 & 249.39 & HT-Direct \\ 
  Nigeria & NORTH EAST & 95-99 & 253.81 & 244.14 & 263.88 & RW2 \\ 
  Nigeria & NORTH EAST & 00-04 & 227.90 & 239.77 & 216.44 & HT-Direct \\ 
  Nigeria & NORTH EAST & 00-04 & 218.86 & 210.13 & 228.06 & RW2 \\ 
  Nigeria & NORTH EAST & 05-09 & 172.43 & 184.84 & 160.69 & HT-Direct \\ 
  Nigeria & NORTH EAST & 05-09 & 173.99 & 164.55 & 183.73 & RW2 \\ 
  Nigeria & NORTH EAST & 10-14 & 134.37 & 154.03 & 116.87 & HT-Direct \\ 
  Nigeria & NORTH EAST & 10-14 & 133.35 & 118.82 & 149.33 & RW2 \\ 
  Nigeria & NORTH EAST & 15-19 & 100.16 & 47.14 & 199.66 & RW2 \\ 
  Nigeria & NORTH WEST & 80-84 & 279.66 & 300.69 & 259.56 & HT-Direct \\ 
  Nigeria & NORTH WEST & 80-84 & 285.94 & 266.87 & 305.61 & RW2 \\ 
  Nigeria & NORTH WEST & 85-89 & 302.35 & 319.22 & 285.98 & HT-Direct \\ 
  Nigeria & NORTH WEST & 85-89 & 299.24 & 286.24 & 312.77 & RW2 \\ 
  Nigeria & NORTH WEST & 90-94 & 301.66 & 318.89 & 284.97 & HT-Direct \\ 
  Nigeria & NORTH WEST & 90-94 & 293.41 & 280.54 & 307.03 & RW2 \\ 
  Nigeria & NORTH WEST & 95-99 & 273.90 & 287.43 & 260.78 & HT-Direct \\ 
  Nigeria & NORTH WEST & 95-99 & 270.01 & 259.80 & 280.56 & RW2 \\ 
  Nigeria & NORTH WEST & 00-04 & 229.44 & 241.20 & 218.08 & HT-Direct \\ 
  Nigeria & NORTH WEST & 00-04 & 227.94 & 218.73 & 237.25 & RW2 \\ 
  Nigeria & NORTH WEST & 05-09 & 193.05 & 204.21 & 182.36 & HT-Direct \\ 
  Nigeria & NORTH WEST & 05-09 & 186.68 & 178.24 & 195.41 & RW2 \\ 
  Nigeria & NORTH WEST & 10-14 & 145.66 & 159.50 & 132.83 & HT-Direct \\ 
  Nigeria & NORTH WEST & 10-14 & 148.72 & 136.64 & 161.77 & RW2 \\ 
  Nigeria & NORTH WEST & 15-19 & 116.12 & 55.93 & 225.60 & RW2 \\ 
  Nigeria & SOUTH EAST & 80-84 & 132.17 & 153.81 & 113.16 & HT-Direct \\ 
  Nigeria & SOUTH EAST & 80-84 & 133.83 & 117.67 & 152.11 & RW2 \\ 
  Nigeria & SOUTH EAST & 85-89 & 141.31 & 158.64 & 125.59 & HT-Direct \\ 
  Nigeria & SOUTH EAST & 85-89 & 137.33 & 126.43 & 148.92 & RW2 \\ 
  Nigeria & SOUTH EAST & 90-94 & 138.12 & 152.75 & 124.69 & HT-Direct \\ 
  Nigeria & SOUTH EAST & 90-94 & 139.52 & 129.77 & 149.82 & RW2 \\ 
  Nigeria & SOUTH EAST & 95-99 & 144.42 & 159.61 & 130.45 & HT-Direct \\ 
  Nigeria & SOUTH EAST & 95-99 & 145.87 & 136.26 & 155.87 & RW2 \\ 
  Nigeria & SOUTH EAST & 00-04 & 152.62 & 166.41 & 139.78 & HT-Direct \\ 
  Nigeria & SOUTH EAST & 00-04 & 144.92 & 136.09 & 154.42 & RW2 \\ 
  Nigeria & SOUTH EAST & 05-09 & 136.46 & 150.87 & 123.22 & HT-Direct \\ 
  Nigeria & SOUTH EAST & 05-09 & 134.75 & 124.73 & 145.38 & RW2 \\ 
  Nigeria & SOUTH EAST & 10-14 & 115.56 & 142.17 & 93.39 & HT-Direct \\ 
  Nigeria & SOUTH EAST & 10-14 & 117.76 & 100.44 & 137.33 & RW2 \\ 
  Nigeria & SOUTH EAST & 15-19 & 100.99 & 46.03 & 205.26 & RW2 \\ 
  Nigeria & SOUTH SOUTH & 80-84 & 147.74 & 169.07 & 128.69 & HT-Direct \\ 
  Nigeria & SOUTH SOUTH & 80-84 & 141.78 & 126.17 & 159.21 & RW2 \\ 
  Nigeria & SOUTH SOUTH & 85-89 & 126.75 & 143.78 & 111.48 & HT-Direct \\ 
  Nigeria & SOUTH SOUTH & 85-89 & 136.24 & 125.33 & 147.79 & RW2 \\ 
  Nigeria & SOUTH SOUTH & 90-94 & 133.12 & 147.38 & 120.05 & HT-Direct \\ 
  Nigeria & SOUTH SOUTH & 90-94 & 132.17 & 122.92 & 141.68 & RW2 \\ 
  Nigeria & SOUTH SOUTH & 95-99 & 134.25 & 147.92 & 121.67 & HT-Direct \\ 
  Nigeria & SOUTH SOUTH & 95-99 & 132.40 & 124.07 & 141.31 & RW2 \\ 
  Nigeria & SOUTH SOUTH & 00-04 & 131.85 & 143.64 & 120.89 & HT-Direct \\ 
  Nigeria & SOUTH SOUTH & 00-04 & 124.34 & 116.86 & 132.52 & RW2 \\ 
  Nigeria & SOUTH SOUTH & 05-09 & 109.43 & 121.73 & 98.23 & HT-Direct \\ 
  Nigeria & SOUTH SOUTH & 05-09 & 107.62 & 99.78 & 116.00 & RW2 \\ 
  Nigeria & SOUTH SOUTH & 10-14 & 80.06 & 93.99 & 68.04 & HT-Direct \\ 
  Nigeria & SOUTH SOUTH & 10-14 & 86.29 & 75.12 & 98.50 & RW2 \\ 
  Nigeria & SOUTH SOUTH & 15-19 & 67.36 & 30.93 & 139.27 & RW2 \\ 
  Nigeria & SOUTH WEST & 80-84 & 152.94 & 180.09 & 129.24 & HT-Direct \\ 
  Nigeria & SOUTH WEST & 80-84 & 151.87 & 132.13 & 174.91 & RW2 \\ 
  Nigeria & SOUTH WEST & 85-89 & 139.67 & 158.09 & 123.09 & HT-Direct \\ 
  Nigeria & SOUTH WEST & 85-89 & 131.22 & 120.17 & 143.29 & RW2 \\ 
  Nigeria & SOUTH WEST & 90-94 & 105.02 & 118.55 & 92.87 & HT-Direct \\ 
  Nigeria & SOUTH WEST & 90-94 & 113.96 & 104.97 & 123.33 & RW2 \\ 
  Nigeria & SOUTH WEST & 95-99 & 109.77 & 121.01 & 99.46 & HT-Direct \\ 
  Nigeria & SOUTH WEST & 95-99 & 106.23 & 98.81 & 113.89 & RW2 \\ 
  Nigeria & SOUTH WEST & 00-04 & 97.15 & 107.19 & 87.97 & HT-Direct \\ 
  Nigeria & SOUTH WEST & 00-04 & 97.26 & 90.73 & 104.17 & RW2 \\ 
  Nigeria & SOUTH WEST & 05-09 & 91.64 & 102.42 & 81.90 & HT-Direct \\ 
  Nigeria & SOUTH WEST & 05-09 & 86.97 & 79.80 & 94.78 & RW2 \\ 
  Nigeria & SOUTH WEST & 10-14 & 73.40 & 93.53 & 57.33 & HT-Direct \\ 
  Nigeria & SOUTH WEST & 10-14 & 73.92 & 62.15 & 88.19 & RW2 \\ 
  Nigeria & SOUTH WEST & 15-19 & 61.79 & 27.42 & 132.06 & RW2 \\ 
  Rwanda & ALL & 80-84 & 183.05 & 178.72 & 187.12 & IHME \\ 
  Rwanda & ALL & 80-84 & 188.89 & 175.97 & 202.53 & RW2 \\ 
  Rwanda & ALL & 80-84 & 188.63 & 182.96 & 193.81 & UN \\ 
  Rwanda & ALL & 85-89 & 156.43 & 152.45 & 159.97 & IHME \\ 
  Rwanda & ALL & 85-89 & 152.46 & 144.80 & 160.41 & RW2 \\ 
  Rwanda & ALL & 85-89 & 153.07 & 148.61 & 157.43 & UN \\ 
  Rwanda & ALL & 90-94 & 187.82 & 163.45 & 216.87 & IHME \\ 
  Rwanda & ALL & 90-94 & 183.25 & 177.74 & 188.89 & RW2 \\ 
  Rwanda & ALL & 90-94 & 182.69 & 176.51 & 189.45 & UN \\ 
  Rwanda & ALL & 95-99 & 176.38 & 171.82 & 180.55 & IHME \\ 
  Rwanda & ALL & 95-99 & 224.40 & 216.90 & 232.04 & RW2 \\ 
  Rwanda & ALL & 95-99 & 225.09 & 212.46 & 236.58 & UN \\ 
  Rwanda & ALL & 00-04 & 137.76 & 133.65 & 141.35 & IHME \\ 
  Rwanda & ALL & 00-04 & 154.22 & 148.40 & 160.24 & RW2 \\ 
  Rwanda & ALL & 00-04 & 153.90 & 149.44 & 158.95 & UN \\ 
  Rwanda & ALL & 05-09 & 86.74 & 83.70 & 89.98 & IHME \\ 
  Rwanda & ALL & 05-09 & 88.66 & 83.90 & 93.66 & RW2 \\ 
  Rwanda & ALL & 05-09 & 88.79 & 85.42 & 92.16 & UN \\ 
  Rwanda & ALL & 10-14 & 71.42 & 66.66 & 76.65 & IHME \\ 
  Rwanda & ALL & 10-14 & 53.39 & 47.04 & 60.49 & RW2 \\ 
  Rwanda & ALL & 10-14 & 53.34 & 49.12 & 57.82 & UN \\ 
  Rwanda & EAST & 80-84 & 181.85 & 210.81 & 156.09 & HT-Direct \\ 
  Rwanda & EAST & 80-84 & 183.93 & 160.50 & 210.53 & RW2 \\ 
  Rwanda & EAST & 85-89 & 163.34 & 179.61 & 148.28 & HT-Direct \\ 
  Rwanda & EAST & 85-89 & 160.32 & 148.41 & 173.05 & RW2 \\ 
  Rwanda & EAST & 90-94 & 236.73 & 252.10 & 222.02 & HT-Direct \\ 
  Rwanda & EAST & 90-94 & 205.65 & 195.50 & 216.05 & RW2 \\ 
  Rwanda & EAST & 95-99 & 249.38 & 264.70 & 234.66 & HT-Direct \\ 
  Rwanda & EAST & 95-99 & 261.20 & 247.79 & 274.60 & RW2 \\ 
  Rwanda & EAST & 00-04 & 202.32 & 215.71 & 189.57 & HT-Direct \\ 
  Rwanda & EAST & 00-04 & 198.54 & 187.33 & 210.89 & RW2 \\ 
  Rwanda & EAST & 05-09 & 108.27 & 118.56 & 98.77 & HT-Direct \\ 
  Rwanda & EAST & 05-09 & 117.21 & 107.88 & 127.10 & RW2 \\ 
  Rwanda & EAST & 10-14 & 60.79 & 74.03 & 49.79 & HT-Direct \\ 
  Rwanda & EAST & 10-14 & 67.50 & 55.53 & 81.22 & RW2 \\ 
  Rwanda & EAST & 15-19 & 38.44 & 12.23 & 112.22 & RW2 \\ 
  Rwanda & KIGALI & 80-84 & 116.56 & 151.21 & 89.01 & HT-Direct \\ 
  Rwanda & KIGALI & 80-84 & 127.18 & 100.77 & 157.85 & RW2 \\ 
  Rwanda & KIGALI & 85-89 & 109.49 & 133.06 & 89.67 & HT-Direct \\ 
  Rwanda & KIGALI & 85-89 & 111.23 & 97.52 & 126.95 & RW2 \\ 
  Rwanda & KIGALI & 90-94 & 174.20 & 198.31 & 152.46 & HT-Direct \\ 
  Rwanda & KIGALI & 90-94 & 137.94 & 125.59 & 152.13 & RW2 \\ 
  Rwanda & KIGALI & 95-99 & 135.05 & 152.81 & 119.07 & HT-Direct \\ 
  Rwanda & KIGALI & 95-99 & 156.54 & 141.39 & 171.67 & RW2 \\ 
  Rwanda & KIGALI & 00-04 & 108.13 & 127.26 & 91.57 & HT-Direct \\ 
  Rwanda & KIGALI & 00-04 & 104.51 & 93.09 & 116.87 & RW2 \\ 
  Rwanda & KIGALI & 05-09 & 56.50 & 68.32 & 46.62 & HT-Direct \\ 
  Rwanda & KIGALI & 05-09 & 59.23 & 50.92 & 68.98 & RW2 \\ 
  Rwanda & KIGALI & 10-14 & 31.42 & 44.99 & 21.85 & HT-Direct \\ 
  Rwanda & KIGALI & 10-14 & 34.75 & 25.95 & 46.74 & RW2 \\ 
  Rwanda & KIGALI & 15-19 & 20.52 & 6.15 & 66.80 & RW2 \\ 
  Rwanda & NORTH & 80-84 & 211.41 & 247.33 & 179.46 & HT-Direct \\ 
  Rwanda & NORTH & 80-84 & 202.24 & 175.19 & 234.28 & RW2 \\ 
  Rwanda & NORTH & 85-89 & 148.97 & 166.40 & 133.07 & HT-Direct \\ 
  Rwanda & NORTH & 85-89 & 152.04 & 139.51 & 165.18 & RW2 \\ 
  Rwanda & NORTH & 90-94 & 203.16 & 219.86 & 187.42 & HT-Direct \\ 
  Rwanda & NORTH & 90-94 & 179.05 & 168.64 & 189.69 & RW2 \\ 
  Rwanda & NORTH & 95-99 & 219.69 & 236.44 & 203.81 & HT-Direct \\ 
  Rwanda & NORTH & 95-99 & 213.14 & 200.60 & 226.91 & RW2 \\ 
  Rwanda & NORTH & 00-04 & 127.54 & 140.62 & 115.52 & HT-Direct \\ 
  Rwanda & NORTH & 00-04 & 140.73 & 130.36 & 151.61 & RW2 \\ 
  Rwanda & NORTH & 05-09 & 81.73 & 92.47 & 72.13 & HT-Direct \\ 
  Rwanda & NORTH & 05-09 & 79.17 & 71.18 & 88.09 & RW2 \\ 
  Rwanda & NORTH & 10-14 & 35.60 & 47.97 & 26.33 & HT-Direct \\ 
  Rwanda & NORTH & 10-14 & 45.08 & 35.44 & 56.94 & RW2 \\ 
  Rwanda & NORTH & 15-19 & 25.17 & 7.80 & 77.46 & RW2 \\ 
  Rwanda & SOUTH & 80-84 & 162.21 & 181.70 & 144.45 & HT-Direct \\ 
  Rwanda & SOUTH & 80-84 & 173.41 & 154.23 & 194.05 & RW2 \\ 
  Rwanda & SOUTH & 85-89 & 162.75 & 179.84 & 146.98 & HT-Direct \\ 
  Rwanda & SOUTH & 85-89 & 158.42 & 146.74 & 171.15 & RW2 \\ 
  Rwanda & SOUTH & 90-94 & 236.13 & 251.10 & 221.78 & HT-Direct \\ 
  Rwanda & SOUTH & 90-94 & 198.28 & 188.36 & 208.91 & RW2 \\ 
  Rwanda & SOUTH & 95-99 & 209.25 & 222.49 & 196.60 & HT-Direct \\ 
  Rwanda & SOUTH & 95-99 & 225.35 & 213.52 & 237.10 & RW2 \\ 
  Rwanda & SOUTH & 00-04 & 152.05 & 163.39 & 141.36 & HT-Direct \\ 
  Rwanda & SOUTH & 00-04 & 149.14 & 140.13 & 158.50 & RW2 \\ 
  Rwanda & SOUTH & 05-09 & 72.34 & 80.77 & 64.72 & HT-Direct \\ 
  Rwanda & SOUTH & 05-09 & 82.45 & 74.99 & 90.60 & RW2 \\ 
  Rwanda & SOUTH & 10-14 & 52.69 & 66.76 & 41.46 & HT-Direct \\ 
  Rwanda & SOUTH & 10-14 & 49.37 & 40.20 & 61.16 & RW2 \\ 
  Rwanda & SOUTH & 15-19 & 30.22 & 9.55 & 91.48 & RW2 \\ 
  Rwanda & WEST & 80-84 & 201.90 & 227.39 & 178.60 & HT-Direct \\ 
  Rwanda & WEST & 80-84 & 201.09 & 178.07 & 226.61 & RW2 \\ 
  Rwanda & WEST & 85-89 & 144.45 & 160.95 & 129.38 & HT-Direct \\ 
  Rwanda & WEST & 85-89 & 143.21 & 131.15 & 155.76 & RW2 \\ 
  Rwanda & WEST & 90-94 & 186.78 & 201.47 & 172.94 & HT-Direct \\ 
  Rwanda & WEST & 90-94 & 166.39 & 156.23 & 176.59 & RW2 \\ 
  Rwanda & WEST & 95-99 & 211.09 & 225.58 & 197.29 & HT-Direct \\ 
  Rwanda & WEST & 95-99 & 207.40 & 195.23 & 221.24 & RW2 \\ 
  Rwanda & WEST & 00-04 & 132.96 & 144.04 & 122.62 & HT-Direct \\ 
  Rwanda & WEST & 00-04 & 139.67 & 130.25 & 149.46 & RW2 \\ 
  Rwanda & WEST & 05-09 & 79.71 & 88.47 & 71.74 & HT-Direct \\ 
  Rwanda & WEST & 05-09 & 83.18 & 75.59 & 91.46 & RW2 \\ 
  Rwanda & WEST & 10-14 & 49.89 & 64.21 & 38.63 & HT-Direct \\ 
  Rwanda & WEST & 10-14 & 53.68 & 42.91 & 67.61 & RW2 \\ 
  Rwanda & WEST & 15-19 & 34.80 & 10.79 & 106.65 & RW2 \\ 
  Senegal & ALL & 80-84 & 182.52 & 180.58 & 184.68 & IHME \\ 
  Senegal & ALL & 80-84 & 192.12 & 183.26 & 201.29 & RW2 \\ 
  Senegal & ALL & 80-84 & 192.10 & 187.98 & 196.67 & UN \\ 
  Senegal & ALL & 85-89 & 155.77 & 154.15 & 157.31 & IHME \\ 
  Senegal & ALL & 85-89 & 157.11 & 150.44 & 163.99 & RW2 \\ 
  Senegal & ALL & 85-89 & 157.15 & 153.53 & 160.55 & UN \\ 
  Senegal & ALL & 90-94 & 139.81 & 138.06 & 141.32 & IHME \\ 
  Senegal & ALL & 90-94 & 139.54 & 134.03 & 145.23 & RW2 \\ 
  Senegal & ALL & 90-94 & 139.52 & 136.45 & 142.90 & UN \\ 
  Senegal & ALL & 95-99 & 128.38 & 126.78 & 129.94 & IHME \\ 
  Senegal & ALL & 95-99 & 142.37 & 136.89 & 148.01 & RW2 \\ 
  Senegal & ALL & 95-99 & 142.32 & 138.72 & 145.75 & UN \\ 
  Senegal & ALL & 00-04 & 107.37 & 106.02 & 108.76 & IHME \\ 
  Senegal & ALL & 00-04 & 119.97 & 115.18 & 124.97 & RW2 \\ 
  Senegal & ALL & 00-04 & 120.05 & 116.36 & 123.45 & UN \\ 
  Senegal & ALL & 05-09 & 81.94 & 80.70 & 83.26 & IHME \\ 
  Senegal & ALL & 05-09 & 82.31 & 78.33 & 86.47 & RW2 \\ 
  Senegal & ALL & 05-09 & 82.31 & 79.05 & 85.76 & UN \\ 
  Senegal & ALL & 10-14 & 61.65 & 59.91 & 63.20 & IHME \\ 
  Senegal & ALL & 10-14 & 56.69 & 52.88 & 60.73 & RW2 \\ 
  Senegal & ALL & 10-14 & 56.68 & 51.79 & 62.09 & UN \\ 
  Senegal & DAKAR & 80-84 & 130.04 & 150.57 & 111.95 & HT-Direct \\ 
  Senegal & DAKAR & 80-84 & 132.79 & 116.22 & 151.53 & RW2 \\ 
  Senegal & DAKAR & 85-89 & 106.71 & 122.05 & 93.09 & HT-Direct \\ 
  Senegal & DAKAR & 85-89 & 109.60 & 99.28 & 120.97 & RW2 \\ 
  Senegal & DAKAR & 90-94 & 94.71 & 108.77 & 82.31 & HT-Direct \\ 
  Senegal & DAKAR & 90-94 & 94.47 & 85.42 & 104.55 & RW2 \\ 
  Senegal & DAKAR & 95-99 & 92.34 & 109.25 & 77.82 & HT-Direct \\ 
  Senegal & DAKAR & 95-99 & 96.15 & 86.12 & 106.86 & RW2 \\ 
  Senegal & DAKAR & 00-04 & 68.79 & 80.44 & 58.72 & HT-Direct \\ 
  Senegal & DAKAR & 00-04 & 82.87 & 73.82 & 92.55 & RW2 \\ 
  Senegal & DAKAR & 05-09 & 60.75 & 71.50 & 51.52 & HT-Direct \\ 
  Senegal & DAKAR & 05-09 & 57.68 & 50.70 & 65.67 & RW2 \\ 
  Senegal & DAKAR & 10-14 & 38.15 & 50.98 & 28.46 & HT-Direct \\ 
  Senegal & DAKAR & 10-14 & 41.98 & 33.50 & 52.56 & RW2 \\ 
  Senegal & DAKAR & 15-19 & 31.52 & 12.15 & 78.43 & RW2 \\ 
  Senegal & DIOURBEL & 80-84 & 236.15 & 262.57 & 211.62 & HT-Direct \\ 
  Senegal & DIOURBEL & 80-84 & 227.48 & 206.68 & 250.16 & RW2 \\ 
  Senegal & DIOURBEL & 85-89 & 171.17 & 188.97 & 154.73 & HT-Direct \\ 
  Senegal & DIOURBEL & 85-89 & 185.31 & 172.20 & 198.81 & RW2 \\ 
  Senegal & DIOURBEL & 90-94 & 152.41 & 169.02 & 137.16 & HT-Direct \\ 
  Senegal & DIOURBEL & 90-94 & 161.28 & 150.10 & 172.68 & RW2 \\ 
  Senegal & DIOURBEL & 95-99 & 168.65 & 184.77 & 153.66 & HT-Direct \\ 
  Senegal & DIOURBEL & 95-99 & 168.94 & 158.04 & 180.43 & RW2 \\ 
  Senegal & DIOURBEL & 00-04 & 142.76 & 158.26 & 128.54 & HT-Direct \\ 
  Senegal & DIOURBEL & 00-04 & 148.89 & 138.38 & 160.03 & RW2 \\ 
  Senegal & DIOURBEL & 05-09 & 92.69 & 103.97 & 82.52 & HT-Direct \\ 
  Senegal & DIOURBEL & 05-09 & 103.31 & 94.70 & 112.68 & RW2 \\ 
  Senegal & DIOURBEL & 10-14 & 78.40 & 92.43 & 66.34 & HT-Direct \\ 
  Senegal & DIOURBEL & 10-14 & 76.73 & 67.05 & 87.85 & RW2 \\ 
  Senegal & DIOURBEL & 15-19 & 59.26 & 25.40 & 133.82 & RW2 \\ 
  Senegal & FATICK & 80-84 & 209.94 & 236.79 & 185.40 & HT-Direct \\ 
  Senegal & FATICK & 80-84 & 210.72 & 189.52 & 233.11 & RW2 \\ 
  Senegal & FATICK & 85-89 & 160.18 & 181.64 & 140.82 & HT-Direct \\ 
  Senegal & FATICK & 85-89 & 173.90 & 160.05 & 187.99 & RW2 \\ 
  Senegal & FATICK & 90-94 & 148.13 & 164.58 & 133.07 & HT-Direct \\ 
  Senegal & FATICK & 90-94 & 150.71 & 140.27 & 161.86 & RW2 \\ 
  Senegal & FATICK & 95-99 & 153.21 & 170.33 & 137.53 & HT-Direct \\ 
  Senegal & FATICK & 95-99 & 153.33 & 142.79 & 165.04 & RW2 \\ 
  Senegal & FATICK & 00-04 & 123.84 & 138.37 & 110.64 & HT-Direct \\ 
  Senegal & FATICK & 00-04 & 128.18 & 118.54 & 138.72 & RW2 \\ 
  Senegal & FATICK & 05-09 & 69.35 & 81.04 & 59.24 & HT-Direct \\ 
  Senegal & FATICK & 05-09 & 82.25 & 74.01 & 91.21 & RW2 \\ 
  Senegal & FATICK & 10-14 & 56.96 & 70.94 & 45.59 & HT-Direct \\ 
  Senegal & FATICK & 10-14 & 55.97 & 47.19 & 65.81 & RW2 \\ 
  Senegal & FATICK & 15-19 & 39.43 & 16.46 & 92.27 & RW2 \\ 
  Senegal & KAOLACK & 80-84 & 217.72 & 244.39 & 193.22 & HT-Direct \\ 
  Senegal & KAOLACK & 80-84 & 209.29 & 189.28 & 231.19 & RW2 \\ 
  Senegal & KAOLACK & 85-89 & 159.66 & 175.79 & 144.75 & HT-Direct \\ 
  Senegal & KAOLACK & 85-89 & 175.68 & 163.48 & 188.36 & RW2 \\ 
  Senegal & KAOLACK & 90-94 & 149.94 & 163.79 & 137.07 & HT-Direct \\ 
  Senegal & KAOLACK & 90-94 & 155.12 & 145.50 & 165.05 & RW2 \\ 
  Senegal & KAOLACK & 95-99 & 157.77 & 170.41 & 145.90 & HT-Direct \\ 
  Senegal & KAOLACK & 95-99 & 161.49 & 152.48 & 171.16 & RW2 \\ 
  Senegal & KAOLACK & 00-04 & 131.01 & 141.06 & 121.58 & HT-Direct \\ 
  Senegal & KAOLACK & 00-04 & 138.30 & 130.37 & 147.04 & RW2 \\ 
  Senegal & KAOLACK & 05-09 & 85.24 & 91.98 & 78.95 & HT-Direct \\ 
  Senegal & KAOLACK & 05-09 & 89.09 & 83.39 & 95.10 & RW2 \\ 
  Senegal & KAOLACK & 10-14 & 53.93 & 61.78 & 47.03 & HT-Direct \\ 
  Senegal & KAOLACK & 10-14 & 58.83 & 52.06 & 65.95 & RW2 \\ 
  Senegal & KAOLACK & 15-19 & 40.08 & 17.01 & 91.12 & RW2 \\ 
  Senegal & KOLDA & 80-84 & 247.03 & 275.21 & 220.85 & HT-Direct \\ 
  Senegal & KOLDA & 80-84 & 252.15 & 228.88 & 276.71 & RW2 \\ 
  Senegal & KOLDA & 85-89 & 224.04 & 242.06 & 206.99 & HT-Direct \\ 
  Senegal & KOLDA & 85-89 & 225.14 & 211.10 & 239.75 & RW2 \\ 
  Senegal & KOLDA & 90-94 & 188.44 & 205.14 & 172.80 & HT-Direct \\ 
  Senegal & KOLDA & 90-94 & 203.05 & 190.66 & 215.75 & RW2 \\ 
  Senegal & KOLDA & 95-99 & 209.79 & 224.41 & 195.87 & HT-Direct \\ 
  Senegal & KOLDA & 95-99 & 214.57 & 203.17 & 226.35 & RW2 \\ 
  Senegal & KOLDA & 00-04 & 180.67 & 193.60 & 168.42 & HT-Direct \\ 
  Senegal & KOLDA & 00-04 & 189.80 & 179.05 & 201.22 & RW2 \\ 
  Senegal & KOLDA & 05-09 & 115.91 & 127.45 & 105.28 & HT-Direct \\ 
  Senegal & KOLDA & 05-09 & 128.97 & 119.48 & 139.09 & RW2 \\ 
  Senegal & KOLDA & 10-14 & 95.21 & 108.59 & 83.32 & HT-Direct \\ 
  Senegal & KOLDA & 10-14 & 93.85 & 83.52 & 105.36 & RW2 \\ 
  Senegal & KOLDA & 15-19 & 71.32 & 30.48 & 160.04 & RW2 \\ 
  Senegal & LOUGA & 80-84 & 199.32 & 227.80 & 173.59 & HT-Direct \\ 
  Senegal & LOUGA & 80-84 & 194.62 & 173.50 & 218.22 & RW2 \\ 
  Senegal & LOUGA & 85-89 & 152.39 & 174.65 & 132.51 & HT-Direct \\ 
  Senegal & LOUGA & 85-89 & 153.63 & 140.91 & 167.35 & RW2 \\ 
  Senegal & LOUGA & 90-94 & 119.16 & 135.43 & 104.60 & HT-Direct \\ 
  Senegal & LOUGA & 90-94 & 124.77 & 114.96 & 134.91 & RW2 \\ 
  Senegal & LOUGA & 95-99 & 118.54 & 135.93 & 103.10 & HT-Direct \\ 
  Senegal & LOUGA & 95-99 & 121.45 & 110.89 & 132.04 & RW2 \\ 
  Senegal & LOUGA & 00-04 & 82.62 & 95.00 & 71.72 & HT-Direct \\ 
  Senegal & LOUGA & 00-04 & 101.64 & 92.24 & 111.38 & RW2 \\ 
  Senegal & LOUGA & 05-09 & 59.03 & 71.27 & 48.78 & HT-Direct \\ 
  Senegal & LOUGA & 05-09 & 69.30 & 62.21 & 77.05 & RW2 \\ 
  Senegal & LOUGA & 10-14 & 61.26 & 73.01 & 51.30 & HT-Direct \\ 
  Senegal & LOUGA & 10-14 & 52.40 & 45.00 & 61.35 & RW2 \\ 
  Senegal & LOUGA & 15-19 & 41.67 & 17.70 & 96.51 & RW2 \\ 
  Senegal & MATAM & 80-84 & 201.37 & 233.88 & 172.36 & HT-Direct \\ 
  Senegal & MATAM & 80-84 & 228.04 & 200.38 & 255.59 & RW2 \\ 
  Senegal & MATAM & 85-89 & 194.80 & 212.37 & 178.35 & HT-Direct \\ 
  Senegal & MATAM & 85-89 & 192.64 & 179.12 & 207.27 & RW2 \\ 
  Senegal & MATAM & 90-94 & 159.03 & 178.99 & 140.91 & HT-Direct \\ 
  Senegal & MATAM & 90-94 & 157.47 & 145.81 & 170.81 & RW2 \\ 
  Senegal & MATAM & 95-99 & 136.28 & 155.16 & 119.38 & HT-Direct \\ 
  Senegal & MATAM & 95-99 & 145.67 & 134.22 & 157.82 & RW2 \\ 
  Senegal & MATAM & 00-04 & 107.19 & 121.34 & 94.51 & HT-Direct \\ 
  Senegal & MATAM & 00-04 & 114.60 & 104.39 & 125.10 & RW2 \\ 
  Senegal & MATAM & 05-09 & 60.24 & 71.25 & 50.84 & HT-Direct \\ 
  Senegal & MATAM & 05-09 & 72.73 & 64.90 & 81.31 & RW2 \\ 
  Senegal & MATAM & 10-14 & 56.86 & 71.00 & 45.39 & HT-Direct \\ 
  Senegal & MATAM & 10-14 & 51.48 & 43.30 & 61.51 & RW2 \\ 
  Senegal & MATAM & 15-19 & 38.45 & 16.01 & 90.88 & RW2 \\ 
  Senegal & SAINT-LOUIS & 80-84 & 192.57 & 221.99 & 166.22 & HT-Direct \\ 
  Senegal & SAINT-LOUIS & 80-84 & 205.73 & 181.84 & 231.75 & RW2 \\ 
  Senegal & SAINT-LOUIS & 85-89 & 176.11 & 199.47 & 154.95 & HT-Direct \\ 
  Senegal & SAINT-LOUIS & 85-89 & 162.46 & 148.07 & 178.50 & RW2 \\ 
  Senegal & SAINT-LOUIS & 90-94 & 118.09 & 135.36 & 102.75 & HT-Direct \\ 
  Senegal & SAINT-LOUIS & 90-94 & 124.61 & 113.82 & 136.38 & RW2 \\ 
  Senegal & SAINT-LOUIS & 95-99 & 98.69 & 114.09 & 85.18 & HT-Direct \\ 
  Senegal & SAINT-LOUIS & 95-99 & 112.73 & 101.87 & 124.00 & RW2 \\ 
  Senegal & SAINT-LOUIS & 00-04 & 86.69 & 100.09 & 74.93 & HT-Direct \\ 
  Senegal & SAINT-LOUIS & 00-04 & 90.53 & 81.23 & 100.36 & RW2 \\ 
  Senegal & SAINT-LOUIS & 05-09 & 57.03 & 69.04 & 47.00 & HT-Direct \\ 
  Senegal & SAINT-LOUIS & 05-09 & 60.16 & 52.78 & 68.56 & RW2 \\ 
  Senegal & SAINT-LOUIS & 10-14 & 46.09 & 60.77 & 34.83 & HT-Direct \\ 
  Senegal & SAINT-LOUIS & 10-14 & 44.86 & 36.16 & 56.25 & RW2 \\ 
  Senegal & SAINT-LOUIS & 15-19 & 35.51 & 14.29 & 86.98 & RW2 \\ 
  Senegal & TAMBACOUNDA & 80-84 & 212.46 & 249.52 & 179.59 & HT-Direct \\ 
  Senegal & TAMBACOUNDA & 80-84 & 223.42 & 195.51 & 252.55 & RW2 \\ 
  Senegal & TAMBACOUNDA & 85-89 & 198.62 & 225.39 & 174.32 & HT-Direct \\ 
  Senegal & TAMBACOUNDA & 85-89 & 206.24 & 189.77 & 223.70 & RW2 \\ 
  Senegal & TAMBACOUNDA & 90-94 & 189.97 & 208.21 & 172.98 & HT-Direct \\ 
  Senegal & TAMBACOUNDA & 90-94 & 191.84 & 179.44 & 205.17 & RW2 \\ 
  Senegal & TAMBACOUNDA & 95-99 & 201.03 & 221.68 & 181.86 & HT-Direct \\ 
  Senegal & TAMBACOUNDA & 95-99 & 203.63 & 191.27 & 216.85 & RW2 \\ 
  Senegal & TAMBACOUNDA & 00-04 & 167.25 & 181.35 & 154.05 & HT-Direct \\ 
  Senegal & TAMBACOUNDA & 00-04 & 180.39 & 169.50 & 191.96 & RW2 \\ 
  Senegal & TAMBACOUNDA & 05-09 & 116.59 & 129.97 & 104.43 & HT-Direct \\ 
  Senegal & TAMBACOUNDA & 05-09 & 125.08 & 115.59 & 135.22 & RW2 \\ 
  Senegal & TAMBACOUNDA & 10-14 & 90.71 & 104.72 & 78.42 & HT-Direct \\ 
  Senegal & TAMBACOUNDA & 10-14 & 92.31 & 81.82 & 103.80 & RW2 \\ 
  Senegal & TAMBACOUNDA & 15-19 & 70.67 & 30.65 & 155.70 & RW2 \\ 
  Senegal & THIES & 80-84 & 165.76 & 192.82 & 141.84 & HT-Direct \\ 
  Senegal & THIES & 80-84 & 159.01 & 139.86 & 180.45 & RW2 \\ 
  Senegal & THIES & 85-89 & 107.64 & 126.01 & 91.66 & HT-Direct \\ 
  Senegal & THIES & 85-89 & 125.57 & 113.88 & 138.21 & RW2 \\ 
  Senegal & THIES & 90-94 & 102.53 & 116.56 & 90.02 & HT-Direct \\ 
  Senegal & THIES & 90-94 & 105.30 & 96.66 & 114.54 & RW2 \\ 
  Senegal & THIES & 95-99 & 108.17 & 123.43 & 94.59 & HT-Direct \\ 
  Senegal & THIES & 95-99 & 105.58 & 97.00 & 115.03 & RW2 \\ 
  Senegal & THIES & 00-04 & 76.93 & 88.51 & 66.75 & HT-Direct \\ 
  Senegal & THIES & 00-04 & 87.46 & 79.57 & 96.14 & RW2 \\ 
  Senegal & THIES & 05-09 & 60.40 & 72.47 & 50.24 & HT-Direct \\ 
  Senegal & THIES & 05-09 & 56.43 & 49.84 & 63.91 & RW2 \\ 
  Senegal & THIES & 10-14 & 31.22 & 41.39 & 23.49 & HT-Direct \\ 
  Senegal & THIES & 10-14 & 38.47 & 31.43 & 46.69 & RW2 \\ 
  Senegal & THIES & 15-19 & 27.17 & 11.06 & 64.87 & RW2 \\ 
  Senegal & ZUGUINCHOR & 80-84 & 215.24 & 258.63 & 177.38 & HT-Direct \\ 
  Senegal & ZUGUINCHOR & 80-84 & 210.13 & 178.39 & 246.09 & RW2 \\ 
  Senegal & ZUGUINCHOR & 85-89 & 155.69 & 183.78 & 131.21 & HT-Direct \\ 
  Senegal & ZUGUINCHOR & 85-89 & 173.96 & 155.40 & 194.10 & RW2 \\ 
  Senegal & ZUGUINCHOR & 90-94 & 146.86 & 167.08 & 128.72 & HT-Direct \\ 
  Senegal & ZUGUINCHOR & 90-94 & 147.17 & 133.75 & 161.66 & RW2 \\ 
  Senegal & ZUGUINCHOR & 95-99 & 142.88 & 160.92 & 126.55 & HT-Direct \\ 
  Senegal & ZUGUINCHOR & 95-99 & 145.68 & 133.27 & 159.23 & RW2 \\ 
  Senegal & ZUGUINCHOR & 00-04 & 108.17 & 124.43 & 93.80 & HT-Direct \\ 
  Senegal & ZUGUINCHOR & 00-04 & 117.46 & 105.73 & 130.36 & RW2 \\ 
  Senegal & ZUGUINCHOR & 05-09 & 64.92 & 80.18 & 52.40 & HT-Direct \\ 
  Senegal & ZUGUINCHOR & 05-09 & 71.20 & 61.30 & 82.60 & RW2 \\ 
  Senegal & ZUGUINCHOR & 10-14 & 46.67 & 65.19 & 33.22 & HT-Direct \\ 
  Senegal & ZUGUINCHOR & 10-14 & 46.79 & 36.15 & 60.32 & RW2 \\ 
  Senegal & ZUGUINCHOR & 15-19 & 32.37 & 12.32 & 83.21 & RW2 \\ 
  Sierra Leone & ALL & 80-84 & 270.31 & 265.06 & 276.10 & IHME \\ 
  Sierra Leone & ALL & 80-84 & 277.79 & 231.92 & 328.82 & RW2 \\ 
  Sierra Leone & ALL & 80-84 & 281.41 & 268.54 & 294.89 & UN \\ 
  Sierra Leone & ALL & 85-89 & 256.50 & 252.65 & 260.16 & IHME \\ 
  Sierra Leone & ALL & 85-89 & 271.75 & 244.82 & 300.40 & RW2 \\ 
  Sierra Leone & ALL & 85-89 & 268.02 & 258.06 & 278.78 & UN \\ 
  Sierra Leone & ALL & 90-94 & 238.75 & 235.46 & 241.96 & IHME \\ 
  Sierra Leone & ALL & 90-94 & 261.95 & 242.13 & 282.77 & RW2 \\ 
  Sierra Leone & ALL & 90-94 & 262.55 & 253.37 & 271.50 & UN \\ 
  Sierra Leone & ALL & 95-99 & 221.12 & 218.35 & 224.35 & IHME \\ 
  Sierra Leone & ALL & 95-99 & 249.36 & 231.48 & 268.06 & RW2 \\ 
  Sierra Leone & ALL & 95-99 & 250.10 & 242.01 & 258.61 & UN \\ 
  Sierra Leone & ALL & 00-04 & 198.59 & 196.17 & 201.45 & IHME \\ 
  Sierra Leone & ALL & 00-04 & 223.42 & 210.05 & 237.42 & RW2 \\ 
  Sierra Leone & ALL & 00-04 & 223.39 & 216.57 & 230.81 & UN \\ 
  Sierra Leone & ALL & 05-09 & 168.80 & 165.82 & 171.71 & IHME \\ 
  Sierra Leone & ALL & 05-09 & 187.88 & 178.30 & 197.86 & RW2 \\ 
  Sierra Leone & ALL & 05-09 & 187.57 & 180.45 & 194.59 & UN \\ 
  Sierra Leone & ALL & 10-14 & 138.76 & 135.08 & 142.62 & IHME \\ 
  Sierra Leone & ALL & 10-14 & 142.03 & 130.56 & 154.24 & RW2 \\ 
  Sierra Leone & ALL & 10-14 & 142.26 & 134.04 & 150.47 & UN \\ 
  Sierra Leone & EASTERN & 80-84 & 328.65 & 414.01 & 253.28 & HT-Direct \\ 
  Sierra Leone & EASTERN & 80-84 & 322.49 & 259.38 & 391.71 & RW2 \\ 
  Sierra Leone & EASTERN & 85-89 & 337.21 & 415.51 & 266.94 & HT-Direct \\ 
  Sierra Leone & EASTERN & 85-89 & 324.56 & 282.23 & 370.03 & RW2 \\ 
  Sierra Leone & EASTERN & 90-94 & 303.14 & 338.63 & 269.86 & HT-Direct \\ 
  Sierra Leone & EASTERN & 90-94 & 319.95 & 290.98 & 350.38 & RW2 \\ 
  Sierra Leone & EASTERN & 95-99 & 285.39 & 318.36 & 254.56 & HT-Direct \\ 
  Sierra Leone & EASTERN & 95-99 & 303.36 & 277.27 & 331.01 & RW2 \\ 
  Sierra Leone & EASTERN & 00-04 & 253.79 & 281.40 & 228.03 & HT-Direct \\ 
  Sierra Leone & EASTERN & 00-04 & 265.75 & 244.82 & 288.18 & RW2 \\ 
  Sierra Leone & EASTERN & 05-09 & 222.90 & 247.65 & 199.97 & HT-Direct \\ 
  Sierra Leone & EASTERN & 05-09 & 214.07 & 196.72 & 232.58 & RW2 \\ 
  Sierra Leone & EASTERN & 10-14 & 139.96 & 161.80 & 120.64 & HT-Direct \\ 
  Sierra Leone & EASTERN & 10-14 & 154.52 & 133.82 & 177.30 & RW2 \\ 
  Sierra Leone & EASTERN & 15-19 & 105.86 & 46.08 & 223.52 & RW2 \\ 
  Sierra Leone & NORTHERN & 80-84 & 172.52 & 239.84 & 121.08 & HT-Direct \\ 
  Sierra Leone & NORTHERN & 80-84 & 205.16 & 156.77 & 261.83 & RW2 \\ 
  Sierra Leone & NORTHERN & 85-89 & 251.41 & 307.73 & 202.38 & HT-Direct \\ 
  Sierra Leone & NORTHERN & 85-89 & 223.12 & 191.99 & 257.35 & RW2 \\ 
  Sierra Leone & NORTHERN & 90-94 & 221.01 & 251.41 & 193.34 & HT-Direct \\ 
  Sierra Leone & NORTHERN & 90-94 & 230.58 & 207.44 & 255.92 & RW2 \\ 
  Sierra Leone & NORTHERN & 95-99 & 214.10 & 240.19 & 190.13 & HT-Direct \\ 
  Sierra Leone & NORTHERN & 95-99 & 225.61 & 205.22 & 247.67 & RW2 \\ 
  Sierra Leone & NORTHERN & 00-04 & 190.87 & 213.22 & 170.35 & HT-Direct \\ 
  Sierra Leone & NORTHERN & 00-04 & 205.66 & 189.03 & 223.37 & RW2 \\ 
  Sierra Leone & NORTHERN & 05-09 & 183.47 & 199.37 & 168.57 & HT-Direct \\ 
  Sierra Leone & NORTHERN & 05-09 & 176.48 & 164.40 & 189.29 & RW2 \\ 
  Sierra Leone & NORTHERN & 10-14 & 124.95 & 141.90 & 109.77 & HT-Direct \\ 
  Sierra Leone & NORTHERN & 10-14 & 138.08 & 121.87 & 155.92 & RW2 \\ 
  Sierra Leone & NORTHERN & 15-19 & 103.85 & 46.29 & 216.34 & RW2 \\ 
  Sierra Leone & SOUTHERN & 80-84 & 363.86 & 490.77 & 253.43 & HT-Direct \\ 
  Sierra Leone & SOUTHERN & 80-84 & 328.15 & 253.86 & 411.73 & RW2 \\ 
  Sierra Leone & SOUTHERN & 85-89 & 311.62 & 379.99 & 250.58 & HT-Direct \\ 
  Sierra Leone & SOUTHERN & 85-89 & 317.25 & 274.63 & 362.03 & RW2 \\ 
  Sierra Leone & SOUTHERN & 90-94 & 291.24 & 328.34 & 256.73 & HT-Direct \\ 
  Sierra Leone & SOUTHERN & 90-94 & 302.78 & 274.29 & 332.66 & RW2 \\ 
  Sierra Leone & SOUTHERN & 95-99 & 251.08 & 278.09 & 225.87 & HT-Direct \\ 
  Sierra Leone & SOUTHERN & 95-99 & 281.74 & 258.59 & 306.34 & RW2 \\ 
  Sierra Leone & SOUTHERN & 00-04 & 258.84 & 286.52 & 232.96 & HT-Direct \\ 
  Sierra Leone & SOUTHERN & 00-04 & 243.86 & 224.97 & 264.87 & RW2 \\ 
  Sierra Leone & SOUTHERN & 05-09 & 186.96 & 203.68 & 171.32 & HT-Direct \\ 
  Sierra Leone & SOUTHERN & 05-09 & 189.42 & 176.28 & 203.42 & RW2 \\ 
  Sierra Leone & SOUTHERN & 10-14 & 125.42 & 144.05 & 108.90 & HT-Direct \\ 
  Sierra Leone & SOUTHERN & 10-14 & 133.93 & 116.97 & 152.52 & RW2 \\ 
  Sierra Leone & SOUTHERN & 15-19 & 90.86 & 39.57 & 193.15 & RW2 \\ 
  Sierra Leone & WESTERN & 80-84 & 318.37 & 506.39 & 175.35 & HT-Direct \\ 
  Sierra Leone & WESTERN & 80-84 & 244.26 & 158.87 & 370.67 & RW2 \\ 
  Sierra Leone & WESTERN & 85-89 & 250.80 & 346.08 & 174.75 & HT-Direct \\ 
  Sierra Leone & WESTERN & 85-89 & 217.07 & 168.09 & 277.45 & RW2 \\ 
  Sierra Leone & WESTERN & 90-94 & 172.34 & 251.84 & 114.11 & HT-Direct \\ 
  Sierra Leone & WESTERN & 90-94 & 191.67 & 152.50 & 234.99 & RW2 \\ 
  Sierra Leone & WESTERN & 95-99 & 123.36 & 173.41 & 86.26 & HT-Direct \\ 
  Sierra Leone & WESTERN & 95-99 & 175.31 & 138.35 & 213.90 & RW2 \\ 
  Sierra Leone & WESTERN & 00-04 & 152.96 & 193.32 & 119.78 & HT-Direct \\ 
  Sierra Leone & WESTERN & 00-04 & 165.93 & 138.02 & 195.69 & RW2 \\ 
  Sierra Leone & WESTERN & 05-09 & 165.60 & 201.26 & 135.18 & HT-Direct \\ 
  Sierra Leone & WESTERN & 05-09 & 158.17 & 136.82 & 182.13 & RW2 \\ 
  Sierra Leone & WESTERN & 10-14 & 142.23 & 180.87 & 110.72 & HT-Direct \\ 
  Sierra Leone & WESTERN & 10-14 & 144.12 & 113.92 & 183.57 & RW2 \\ 
  Sierra Leone & WESTERN & 15-19 & 128.75 & 53.45 & 284.57 & RW2 \\ 
  Tanzania & ALL & 80-84 & 171.36 & 169.32 & 173.54 & IHME \\ 
  Tanzania & ALL & 80-84 & 179.37 & 167.40 & 192.00 & RW2 \\ 
  Tanzania & ALL & 80-84 & 179.27 & 174.32 & 184.40 & UN \\ 
  Tanzania & ALL & 85-89 & 161.41 & 159.64 & 163.27 & IHME \\ 
  Tanzania & ALL & 85-89 & 171.57 & 162.25 & 181.27 & RW2 \\ 
  Tanzania & ALL & 85-89 & 171.80 & 167.30 & 176.39 & UN \\ 
  Tanzania & ALL & 90-94 & 150.46 & 148.53 & 152.19 & IHME \\ 
  Tanzania & ALL & 90-94 & 162.73 & 155.68 & 170.03 & RW2 \\ 
  Tanzania & ALL & 90-94 & 162.57 & 158.29 & 166.92 & UN \\ 
  Tanzania & ALL & 95-99 & 134.95 & 133.30 & 136.87 & IHME \\ 
  Tanzania & ALL & 95-99 & 149.01 & 142.48 & 155.75 & RW2 \\ 
  Tanzania & ALL & 95-99 & 149.00 & 144.89 & 152.97 & UN \\ 
  Tanzania & ALL & 00-04 & 110.67 & 108.67 & 112.55 & IHME \\ 
  Tanzania & ALL & 00-04 & 114.16 & 108.43 & 120.18 & RW2 \\ 
  Tanzania & ALL & 00-04 & 114.33 & 110.51 & 118.20 & UN \\ 
  Tanzania & ALL & 05-09 & 87.55 & 85.11 & 89.87 & IHME \\ 
  Tanzania & ALL & 05-09 & 79.67 & 74.92 & 84.69 & RW2 \\ 
  Tanzania & ALL & 05-09 & 79.59 & 75.50 & 84.15 & UN \\ 
  Tanzania & ALL & 10-14 & 69.44 & 66.37 & 72.79 & IHME \\ 
  Tanzania & ALL & 10-14 & 56.63 & 52.16 & 61.43 & RW2 \\ 
  Tanzania & ALL & 10-14 & 56.64 & 50.70 & 63.71 & UN \\ 
  Tanzania & ARUSHA & 80-84 & 93.90 & 124.83 & 70.02 & HT-Direct \\ 
  Tanzania & ARUSHA & 80-84 & 109.75 & 87.20 & 137.40 & RW2 \\ 
  Tanzania & ARUSHA & 85-89 & 103.58 & 136.04 & 78.17 & HT-Direct \\ 
  Tanzania & ARUSHA & 85-89 & 104.30 & 88.79 & 122.15 & RW2 \\ 
  Tanzania & ARUSHA & 90-94 & 97.58 & 121.12 & 78.21 & HT-Direct \\ 
  Tanzania & ARUSHA & 90-94 & 97.11 & 85.27 & 110.45 & RW2 \\ 
  Tanzania & ARUSHA & 95-99 & 86.27 & 106.67 & 69.47 & HT-Direct \\ 
  Tanzania & ARUSHA & 95-99 & 87.40 & 77.15 & 98.84 & RW2 \\ 
  Tanzania & ARUSHA & 00-04 & 56.77 & 71.58 & 44.87 & HT-Direct \\ 
  Tanzania & ARUSHA & 00-04 & 65.75 & 57.52 & 75.08 & RW2 \\ 
  Tanzania & ARUSHA & 05-09 & 57.41 & 75.99 & 43.16 & HT-Direct \\ 
  Tanzania & ARUSHA & 05-09 & 43.77 & 37.06 & 51.75 & RW2 \\ 
  Tanzania & ARUSHA & 10-14 & 27.76 & 43.03 & 17.81 & HT-Direct \\ 
  Tanzania & ARUSHA & 10-14 & 29.57 & 22.82 & 38.12 & RW2 \\ 
  Tanzania & ARUSHA & 15-19 & 20.21 & 8.29 & 48.48 & RW2 \\ 
  Tanzania & DAR ES SALAAM & 80-84 & 183.66 & 223.51 & 149.54 & HT-Direct \\ 
  Tanzania & DAR ES SALAAM & 80-84 & 199.32 & 164.11 & 240.00 & RW2 \\ 
  Tanzania & DAR ES SALAAM & 85-89 & 194.14 & 235.18 & 158.77 & HT-Direct \\ 
  Tanzania & DAR ES SALAAM & 85-89 & 175.21 & 151.66 & 202.78 & RW2 \\ 
  Tanzania & DAR ES SALAAM & 90-94 & 127.47 & 155.26 & 104.04 & HT-Direct \\ 
  Tanzania & DAR ES SALAAM & 90-94 & 141.76 & 122.63 & 162.60 & RW2 \\ 
  Tanzania & DAR ES SALAAM & 95-99 & 117.17 & 144.47 & 94.46 & HT-Direct \\ 
  Tanzania & DAR ES SALAAM & 95-99 & 123.94 & 106.22 & 143.51 & RW2 \\ 
  Tanzania & DAR ES SALAAM & 00-04 & 107.46 & 141.23 & 81.01 & HT-Direct \\ 
  Tanzania & DAR ES SALAAM & 00-04 & 105.29 & 88.73 & 124.39 & RW2 \\ 
  Tanzania & DAR ES SALAAM & 05-09 & 102.43 & 135.42 & 76.77 & HT-Direct \\ 
  Tanzania & DAR ES SALAAM & 05-09 & 86.90 & 73.42 & 102.98 & RW2 \\ 
  Tanzania & DAR ES SALAAM & 10-14 & 91.88 & 114.80 & 73.17 & HT-Direct \\ 
  Tanzania & DAR ES SALAAM & 10-14 & 74.88 & 62.54 & 89.42 & RW2 \\ 
  Tanzania & DAR ES SALAAM & 15-19 & 65.16 & 26.87 & 152.11 & RW2 \\ 
  Tanzania & DODOMA & 80-84 & 183.08 & 233.02 & 141.86 & HT-Direct \\ 
  Tanzania & DODOMA & 80-84 & 230.37 & 189.48 & 276.22 & RW2 \\ 
  Tanzania & DODOMA & 85-89 & 240.05 & 280.97 & 203.41 & HT-Direct \\ 
  Tanzania & DODOMA & 85-89 & 225.18 & 199.16 & 254.00 & RW2 \\ 
  Tanzania & DODOMA & 90-94 & 225.95 & 273.26 & 184.74 & HT-Direct \\ 
  Tanzania & DODOMA & 90-94 & 211.45 & 189.07 & 236.33 & RW2 \\ 
  Tanzania & DODOMA & 95-99 & 187.08 & 221.43 & 156.99 & HT-Direct \\ 
  Tanzania & DODOMA & 95-99 & 185.60 & 165.77 & 207.35 & RW2 \\ 
  Tanzania & DODOMA & 00-04 & 119.29 & 147.69 & 95.74 & HT-Direct \\ 
  Tanzania & DODOMA & 00-04 & 134.97 & 118.29 & 153.29 & RW2 \\ 
  Tanzania & DODOMA & 05-09 & 91.26 & 123.75 & 66.64 & HT-Direct \\ 
  Tanzania & DODOMA & 05-09 & 89.20 & 75.50 & 105.14 & RW2 \\ 
  Tanzania & DODOMA & 10-14 & 92.59 & 136.52 & 61.78 & HT-Direct \\ 
  Tanzania & DODOMA & 10-14 & 61.17 & 47.16 & 79.02 & RW2 \\ 
  Tanzania & DODOMA & 15-19 & 42.32 & 17.31 & 99.64 & RW2 \\ 
  Tanzania & IRINGA & 80-84 & 226.04 & 267.96 & 188.99 & HT-Direct \\ 
  Tanzania & IRINGA & 80-84 & 218.04 & 184.75 & 256.35 & RW2 \\ 
  Tanzania & IRINGA & 85-89 & 156.36 & 201.94 & 119.52 & HT-Direct \\ 
  Tanzania & IRINGA & 85-89 & 191.20 & 168.28 & 216.02 & RW2 \\ 
  Tanzania & IRINGA & 90-94 & 174.93 & 204.04 & 149.20 & HT-Direct \\ 
  Tanzania & IRINGA & 90-94 & 170.88 & 152.72 & 190.14 & RW2 \\ 
  Tanzania & IRINGA & 95-99 & 126.82 & 154.57 & 103.44 & HT-Direct \\ 
  Tanzania & IRINGA & 95-99 & 149.78 & 133.59 & 166.93 & RW2 \\ 
  Tanzania & IRINGA & 00-04 & 108.16 & 132.12 & 88.10 & HT-Direct \\ 
  Tanzania & IRINGA & 00-04 & 111.44 & 99.07 & 125.22 & RW2 \\ 
  Tanzania & IRINGA & 05-09 & 98.20 & 119.20 & 80.56 & HT-Direct \\ 
  Tanzania & IRINGA & 05-09 & 74.43 & 64.96 & 85.15 & RW2 \\ 
  Tanzania & IRINGA & 10-14 & 48.49 & 71.36 & 32.69 & HT-Direct \\ 
  Tanzania & IRINGA & 10-14 & 49.85 & 39.76 & 62.51 & RW2 \\ 
  Tanzania & IRINGA & 15-19 & 33.37 & 13.80 & 78.08 & RW2 \\ 
  Tanzania & KAGERA & 80-84 & 203.38 & 253.49 & 161.04 & HT-Direct \\ 
  Tanzania & KAGERA & 80-84 & 207.34 & 170.86 & 249.16 & RW2 \\ 
  Tanzania & KAGERA & 85-89 & 181.10 & 217.29 & 149.78 & HT-Direct \\ 
  Tanzania & KAGERA & 85-89 & 194.37 & 171.28 & 219.78 & RW2 \\ 
  Tanzania & KAGERA & 90-94 & 173.29 & 204.11 & 146.28 & HT-Direct \\ 
  Tanzania & KAGERA & 90-94 & 182.71 & 164.59 & 202.50 & RW2 \\ 
  Tanzania & KAGERA & 95-99 & 178.56 & 205.23 & 154.68 & HT-Direct \\ 
  Tanzania & KAGERA & 95-99 & 168.98 & 152.69 & 186.73 & RW2 \\ 
  Tanzania & KAGERA & 00-04 & 126.88 & 156.97 & 101.87 & HT-Direct \\ 
  Tanzania & KAGERA & 00-04 & 131.47 & 116.34 & 148.75 & RW2 \\ 
  Tanzania & KAGERA & 05-09 & 95.49 & 125.77 & 71.90 & HT-Direct \\ 
  Tanzania & KAGERA & 05-09 & 88.45 & 74.99 & 104.14 & RW2 \\ 
  Tanzania & KAGERA & 10-14 & 73.16 & 115.66 & 45.47 & HT-Direct \\ 
  Tanzania & KAGERA & 10-14 & 59.42 & 44.95 & 77.82 & RW2 \\ 
  Tanzania & KAGERA & 15-19 & 40.08 & 15.99 & 97.72 & RW2 \\ 
  Tanzania & KIGOMA & 80-84 & 175.48 & 238.27 & 126.48 & HT-Direct \\ 
  Tanzania & KIGOMA & 80-84 & 207.22 & 164.67 & 256.53 & RW2 \\ 
  Tanzania & KIGOMA & 85-89 & 187.69 & 224.26 & 155.88 & HT-Direct \\ 
  Tanzania & KIGOMA & 85-89 & 192.80 & 168.20 & 220.18 & RW2 \\ 
  Tanzania & KIGOMA & 90-94 & 187.56 & 221.91 & 157.45 & HT-Direct \\ 
  Tanzania & KIGOMA & 90-94 & 178.54 & 159.24 & 199.79 & RW2 \\ 
  Tanzania & KIGOMA & 95-99 & 137.33 & 171.51 & 109.07 & HT-Direct \\ 
  Tanzania & KIGOMA & 95-99 & 158.12 & 140.28 & 178.21 & RW2 \\ 
  Tanzania & KIGOMA & 00-04 & 123.11 & 149.15 & 101.08 & HT-Direct \\ 
  Tanzania & KIGOMA & 00-04 & 117.09 & 102.64 & 133.35 & RW2 \\ 
  Tanzania & KIGOMA & 05-09 & 91.19 & 122.40 & 67.33 & HT-Direct \\ 
  Tanzania & KIGOMA & 05-09 & 76.34 & 63.84 & 91.04 & RW2 \\ 
  Tanzania & KIGOMA & 10-14 & 47.76 & 82.77 & 27.12 & HT-Direct \\ 
  Tanzania & KIGOMA & 10-14 & 50.64 & 37.29 & 68.00 & RW2 \\ 
  Tanzania & KIGOMA & 15-19 & 33.95 & 13.24 & 83.70 & RW2 \\ 
  Tanzania & KILIMANJARO & 80-84 & 119.76 & 155.32 & 91.45 & HT-Direct \\ 
  Tanzania & KILIMANJARO & 80-84 & 107.24 & 83.79 & 137.03 & RW2 \\ 
  Tanzania & KILIMANJARO & 85-89 & 75.89 & 97.84 & 58.54 & HT-Direct \\ 
  Tanzania & KILIMANJARO & 85-89 & 87.98 & 74.19 & 104.11 & RW2 \\ 
  Tanzania & KILIMANJARO & 90-94 & 71.65 & 91.95 & 55.56 & HT-Direct \\ 
  Tanzania & KILIMANJARO & 90-94 & 74.21 & 62.13 & 87.58 & RW2 \\ 
  Tanzania & KILIMANJARO & 95-99 & 48.75 & 68.41 & 34.52 & HT-Direct \\ 
  Tanzania & KILIMANJARO & 95-99 & 66.94 & 54.85 & 80.49 & RW2 \\ 
  Tanzania & KILIMANJARO & 00-04 & 71.32 & 98.50 & 51.22 & HT-Direct \\ 
  Tanzania & KILIMANJARO & 00-04 & 56.03 & 45.21 & 69.27 & RW2 \\ 
  Tanzania & KILIMANJARO & 05-09 & 41.55 & 77.52 & 21.88 & HT-Direct \\ 
  Tanzania & KILIMANJARO & 05-09 & 43.41 & 32.06 & 58.84 & RW2 \\ 
  Tanzania & KILIMANJARO & 10-14 & 49.22 & 108.66 & 21.51 & HT-Direct \\ 
  Tanzania & KILIMANJARO & 10-14 & 34.77 & 21.72 & 56.64 & RW2 \\ 
  Tanzania & KILIMANJARO & 15-19 & 28.48 & 9.56 & 84.70 & RW2 \\ 
  Tanzania & LINDI & 80-84 & 228.32 & 270.63 & 190.89 & HT-Direct \\ 
  Tanzania & LINDI & 80-84 & 247.58 & 210.16 & 288.06 & RW2 \\ 
  Tanzania & LINDI & 85-89 & 241.50 & 289.26 & 199.42 & HT-Direct \\ 
  Tanzania & LINDI & 85-89 & 255.11 & 227.49 & 284.76 & RW2 \\ 
  Tanzania & LINDI & 90-94 & 257.03 & 299.28 & 218.88 & HT-Direct \\ 
  Tanzania & LINDI & 90-94 & 251.15 & 227.45 & 276.77 & RW2 \\ 
  Tanzania & LINDI & 95-99 & 228.34 & 260.03 & 199.47 & HT-Direct \\ 
  Tanzania & LINDI & 95-99 & 223.88 & 202.77 & 247.08 & RW2 \\ 
  Tanzania & LINDI & 00-04 & 151.11 & 191.98 & 117.67 & HT-Direct \\ 
  Tanzania & LINDI & 00-04 & 154.45 & 135.49 & 175.54 & RW2 \\ 
  Tanzania & LINDI & 05-09 & 100.97 & 135.81 & 74.30 & HT-Direct \\ 
  Tanzania & LINDI & 05-09 & 89.85 & 74.16 & 108.20 & RW2 \\ 
  Tanzania & LINDI & 10-14 & 55.71 & 101.48 & 29.89 & HT-Direct \\ 
  Tanzania & LINDI & 10-14 & 52.18 & 37.29 & 71.82 & RW2 \\ 
  Tanzania & LINDI & 15-19 & 30.21 & 11.38 & 77.02 & RW2 \\ 
  Tanzania & MARA & 80-84 & 194.42 & 244.65 & 152.42 & HT-Direct \\ 
  Tanzania & MARA & 80-84 & 207.16 & 170.67 & 249.27 & RW2 \\ 
  Tanzania & MARA & 85-89 & 211.38 & 250.79 & 176.71 & HT-Direct \\ 
  Tanzania & MARA & 85-89 & 201.00 & 178.51 & 225.60 & RW2 \\ 
  Tanzania & MARA & 90-94 & 180.99 & 204.33 & 159.78 & HT-Direct \\ 
  Tanzania & MARA & 90-94 & 194.69 & 177.89 & 212.40 & RW2 \\ 
  Tanzania & MARA & 95-99 & 187.63 & 216.05 & 162.17 & HT-Direct \\ 
  Tanzania & MARA & 95-99 & 188.54 & 172.15 & 206.19 & RW2 \\ 
  Tanzania & MARA & 00-04 & 168.89 & 195.15 & 145.52 & HT-Direct \\ 
  Tanzania & MARA & 00-04 & 155.57 & 140.53 & 172.42 & RW2 \\ 
  Tanzania & MARA & 05-09 & 121.70 & 149.45 & 98.51 & HT-Direct \\ 
  Tanzania & MARA & 05-09 & 108.38 & 95.27 & 123.19 & RW2 \\ 
  Tanzania & MARA & 10-14 & 80.97 & 112.97 & 57.45 & HT-Direct \\ 
  Tanzania & MARA & 10-14 & 73.00 & 58.34 & 90.47 & RW2 \\ 
  Tanzania & MARA & 15-19 & 48.64 & 20.09 & 112.28 & RW2 \\ 
  Tanzania & MBEYA & 80-84 & 145.34 & 184.21 & 113.54 & HT-Direct \\ 
  Tanzania & MBEYA & 80-84 & 151.04 & 122.02 & 185.29 & RW2 \\ 
  Tanzania & MBEYA & 85-89 & 127.93 & 163.27 & 99.32 & HT-Direct \\ 
  Tanzania & MBEYA & 85-89 & 149.79 & 129.11 & 173.03 & RW2 \\ 
  Tanzania & MBEYA & 90-94 & 168.16 & 204.57 & 137.12 & HT-Direct \\ 
  Tanzania & MBEYA & 90-94 & 150.80 & 132.79 & 170.97 & RW2 \\ 
  Tanzania & MBEYA & 95-99 & 140.71 & 173.24 & 113.46 & HT-Direct \\ 
  Tanzania & MBEYA & 95-99 & 144.93 & 127.84 & 163.91 & RW2 \\ 
  Tanzania & MBEYA & 00-04 & 114.27 & 141.01 & 92.06 & HT-Direct \\ 
  Tanzania & MBEYA & 00-04 & 116.52 & 101.71 & 133.06 & RW2 \\ 
  Tanzania & MBEYA & 05-09 & 87.43 & 121.73 & 62.11 & HT-Direct \\ 
  Tanzania & MBEYA & 05-09 & 86.27 & 72.78 & 102.09 & RW2 \\ 
  Tanzania & MBEYA & 10-14 & 90.23 & 132.68 & 60.41 & HT-Direct \\ 
  Tanzania & MBEYA & 10-14 & 67.71 & 52.26 & 87.27 & RW2 \\ 
  Tanzania & MBEYA & 15-19 & 54.13 & 22.22 & 126.50 & RW2 \\ 
  Tanzania & MOROGORO & 80-84 & 231.80 & 283.50 & 187.07 & HT-Direct \\ 
  Tanzania & MOROGORO & 80-84 & 219.87 & 184.45 & 259.66 & RW2 \\ 
  Tanzania & MOROGORO & 85-89 & 191.27 & 226.30 & 160.53 & HT-Direct \\ 
  Tanzania & MOROGORO & 85-89 & 210.86 & 187.75 & 235.96 & RW2 \\ 
  Tanzania & MOROGORO & 90-94 & 167.27 & 197.22 & 141.08 & HT-Direct \\ 
  Tanzania & MOROGORO & 90-94 & 198.78 & 180.36 & 218.52 & RW2 \\ 
  Tanzania & MOROGORO & 95-99 & 212.36 & 242.61 & 184.96 & HT-Direct \\ 
  Tanzania & MOROGORO & 95-99 & 179.86 & 162.56 & 198.65 & RW2 \\ 
  Tanzania & MOROGORO & 00-04 & 110.90 & 151.60 & 80.09 & HT-Direct \\ 
  Tanzania & MOROGORO & 00-04 & 130.67 & 114.05 & 149.55 & RW2 \\ 
  Tanzania & MOROGORO & 05-09 & 87.42 & 122.40 & 61.74 & HT-Direct \\ 
  Tanzania & MOROGORO & 05-09 & 82.21 & 67.96 & 99.19 & RW2 \\ 
  Tanzania & MOROGORO & 10-14 & 62.52 & 107.36 & 35.67 & HT-Direct \\ 
  Tanzania & MOROGORO & 10-14 & 52.10 & 38.57 & 69.65 & RW2 \\ 
  Tanzania & MOROGORO & 15-19 & 33.01 & 13.21 & 80.37 & RW2 \\ 
  Tanzania & MTWARA & 80-84 & 193.03 & 238.22 & 154.67 & HT-Direct \\ 
  Tanzania & MTWARA & 80-84 & 200.55 & 164.21 & 242.11 & RW2 \\ 
  Tanzania & MTWARA & 85-89 & 182.64 & 220.22 & 150.23 & HT-Direct \\ 
  Tanzania & MTWARA & 85-89 & 213.97 & 188.17 & 241.87 & RW2 \\ 
  Tanzania & MTWARA & 90-94 & 237.53 & 272.91 & 205.44 & HT-Direct \\ 
  Tanzania & MTWARA & 90-94 & 220.25 & 197.63 & 245.20 & RW2 \\ 
  Tanzania & MTWARA & 95-99 & 203.11 & 242.02 & 169.05 & HT-Direct \\ 
  Tanzania & MTWARA & 95-99 & 196.27 & 174.30 & 220.53 & RW2 \\ 
  Tanzania & MTWARA & 00-04 & 114.76 & 143.66 & 91.05 & HT-Direct \\ 
  Tanzania & MTWARA & 00-04 & 130.87 & 112.00 & 151.93 & RW2 \\ 
  Tanzania & MTWARA & 05-09 & 78.09 & 122.69 & 48.80 & HT-Direct \\ 
  Tanzania & MTWARA & 05-09 & 75.74 & 60.47 & 94.12 & RW2 \\ 
  Tanzania & MTWARA & 10-14 & 64.52 & 109.40 & 37.27 & HT-Direct \\ 
  Tanzania & MTWARA & 10-14 & 45.62 & 31.88 & 65.01 & RW2 \\ 
  Tanzania & MTWARA & 15-19 & 27.83 & 10.09 & 74.49 & RW2 \\ 
  Tanzania & MWANZA & 80-84 & 153.58 & 203.55 & 114.11 & HT-Direct \\ 
  Tanzania & MWANZA & 80-84 & 179.07 & 143.07 & 221.77 & RW2 \\ 
  Tanzania & MWANZA & 85-89 & 170.33 & 202.90 & 142.06 & HT-Direct \\ 
  Tanzania & MWANZA & 85-89 & 173.00 & 151.58 & 196.60 & RW2 \\ 
  Tanzania & MWANZA & 90-94 & 176.48 & 205.67 & 150.65 & HT-Direct \\ 
  Tanzania & MWANZA & 90-94 & 163.78 & 147.53 & 181.55 & RW2 \\ 
  Tanzania & MWANZA & 95-99 & 135.66 & 156.48 & 117.21 & HT-Direct \\ 
  Tanzania & MWANZA & 95-99 & 151.20 & 136.46 & 166.86 & RW2 \\ 
  Tanzania & MWANZA & 00-04 & 130.47 & 156.30 & 108.36 & HT-Direct \\ 
  Tanzania & MWANZA & 00-04 & 121.07 & 107.96 & 135.72 & RW2 \\ 
  Tanzania & MWANZA & 05-09 & 91.34 & 113.06 & 73.45 & HT-Direct \\ 
  Tanzania & MWANZA & 05-09 & 84.88 & 73.80 & 97.55 & RW2 \\ 
  Tanzania & MWANZA & 10-14 & 73.88 & 106.42 & 50.73 & HT-Direct \\ 
  Tanzania & MWANZA & 10-14 & 59.60 & 46.94 & 75.37 & RW2 \\ 
  Tanzania & MWANZA & 15-19 & 42.03 & 16.97 & 100.13 & RW2 \\ 
  Tanzania & PWANI & 80-84 & 181.74 & 238.36 & 136.17 & HT-Direct \\ 
  Tanzania & PWANI & 80-84 & 236.01 & 191.82 & 285.58 & RW2 \\ 
  Tanzania & PWANI & 85-89 & 261.14 & 303.19 & 223.06 & HT-Direct \\ 
  Tanzania & PWANI & 85-89 & 220.93 & 194.78 & 249.53 & RW2 \\ 
  Tanzania & PWANI & 90-94 & 157.87 & 189.81 & 130.44 & HT-Direct \\ 
  Tanzania & PWANI & 90-94 & 192.11 & 171.70 & 214.02 & RW2 \\ 
  Tanzania & PWANI & 95-99 & 174.53 & 209.38 & 144.41 & HT-Direct \\ 
  Tanzania & PWANI & 95-99 & 163.94 & 145.59 & 183.63 & RW2 \\ 
  Tanzania & PWANI & 00-04 & 113.70 & 139.14 & 92.42 & HT-Direct \\ 
  Tanzania & PWANI & 00-04 & 120.28 & 104.90 & 137.49 & RW2 \\ 
  Tanzania & PWANI & 05-09 & 81.29 & 121.76 & 53.46 & HT-Direct \\ 
  Tanzania & PWANI & 05-09 & 80.82 & 66.52 & 98.05 & RW2 \\ 
  Tanzania & PWANI & 10-14 & 85.21 & 137.96 & 51.43 & HT-Direct \\ 
  Tanzania & PWANI & 10-14 & 56.50 & 41.61 & 76.85 & RW2 \\ 
  Tanzania & PWANI & 15-19 & 40.17 & 15.78 & 99.52 & RW2 \\ 
  Tanzania & RUKWA & 80-84 & 232.14 & 290.25 & 182.67 & HT-Direct \\ 
  Tanzania & RUKWA & 80-84 & 235.44 & 193.79 & 284.17 & RW2 \\ 
  Tanzania & RUKWA & 85-89 & 195.34 & 229.90 & 164.87 & HT-Direct \\ 
  Tanzania & RUKWA & 85-89 & 204.70 & 182.09 & 229.68 & RW2 \\ 
  Tanzania & RUKWA & 90-94 & 179.72 & 203.26 & 158.36 & HT-Direct \\ 
  Tanzania & RUKWA & 90-94 & 180.83 & 164.38 & 198.56 & RW2 \\ 
  Tanzania & RUKWA & 95-99 & 169.25 & 205.57 & 138.24 & HT-Direct \\ 
  Tanzania & RUKWA & 95-99 & 155.79 & 139.59 & 173.36 & RW2 \\ 
  Tanzania & RUKWA & 00-04 & 106.35 & 124.23 & 90.77 & HT-Direct \\ 
  Tanzania & RUKWA & 00-04 & 115.63 & 102.85 & 129.19 & RW2 \\ 
  Tanzania & RUKWA & 05-09 & 86.17 & 106.89 & 69.15 & HT-Direct \\ 
  Tanzania & RUKWA & 05-09 & 83.24 & 73.05 & 94.61 & RW2 \\ 
  Tanzania & RUKWA & 10-14 & 93.70 & 122.01 & 71.42 & HT-Direct \\ 
  Tanzania & RUKWA & 10-14 & 66.63 & 54.73 & 81.31 & RW2 \\ 
  Tanzania & RUKWA & 15-19 & 55.23 & 23.51 & 127.35 & RW2 \\ 
  Tanzania & RUVUMA & 80-84 & 159.66 & 209.12 & 120.13 & HT-Direct \\ 
  Tanzania & RUVUMA & 80-84 & 147.41 & 116.94 & 184.18 & RW2 \\ 
  Tanzania & RUVUMA & 85-89 & 118.21 & 148.04 & 93.73 & HT-Direct \\ 
  Tanzania & RUVUMA & 85-89 & 157.00 & 134.96 & 181.88 & RW2 \\ 
  Tanzania & RUVUMA & 90-94 & 183.01 & 224.41 & 147.80 & HT-Direct \\ 
  Tanzania & RUVUMA & 90-94 & 166.26 & 146.64 & 188.20 & RW2 \\ 
  Tanzania & RUVUMA & 95-99 & 162.84 & 196.78 & 133.77 & HT-Direct \\ 
  Tanzania & RUVUMA & 95-99 & 160.62 & 142.16 & 181.54 & RW2 \\ 
  Tanzania & RUVUMA & 00-04 & 129.48 & 159.50 & 104.41 & HT-Direct \\ 
  Tanzania & RUVUMA & 00-04 & 119.77 & 104.20 & 137.16 & RW2 \\ 
  Tanzania & RUVUMA & 05-09 & 66.21 & 92.09 & 47.22 & HT-Direct \\ 
  Tanzania & RUVUMA & 05-09 & 76.35 & 62.98 & 92.14 & RW2 \\ 
  Tanzania & RUVUMA & 10-14 & 68.63 & 121.95 & 37.62 & HT-Direct \\ 
  Tanzania & RUVUMA & 10-14 & 49.89 & 35.80 & 68.20 & RW2 \\ 
  Tanzania & RUVUMA & 15-19 & 32.71 & 12.50 & 83.02 & RW2 \\ 
  Tanzania & SHINYANGA & 80-84 & 190.72 & 231.11 & 155.96 & HT-Direct \\ 
  Tanzania & SHINYANGA & 80-84 & 188.39 & 159.50 & 221.84 & RW2 \\ 
  Tanzania & SHINYANGA & 85-89 & 162.95 & 193.74 & 136.22 & HT-Direct \\ 
  Tanzania & SHINYANGA & 85-89 & 175.55 & 156.94 & 196.02 & RW2 \\ 
  Tanzania & SHINYANGA & 90-94 & 168.27 & 193.54 & 145.71 & HT-Direct \\ 
  Tanzania & SHINYANGA & 90-94 & 165.21 & 150.62 & 180.61 & RW2 \\ 
  Tanzania & SHINYANGA & 95-99 & 147.13 & 167.96 & 128.48 & HT-Direct \\ 
  Tanzania & SHINYANGA & 95-99 & 154.02 & 140.87 & 167.85 & RW2 \\ 
  Tanzania & SHINYANGA & 00-04 & 126.54 & 146.34 & 109.07 & HT-Direct \\ 
  Tanzania & SHINYANGA & 00-04 & 124.18 & 113.03 & 136.30 & RW2 \\ 
  Tanzania & SHINYANGA & 05-09 & 93.72 & 113.56 & 77.04 & HT-Direct \\ 
  Tanzania & SHINYANGA & 05-09 & 90.11 & 80.75 & 100.54 & RW2 \\ 
  Tanzania & SHINYANGA & 10-14 & 88.66 & 108.92 & 71.86 & HT-Direct \\ 
  Tanzania & SHINYANGA & 10-14 & 67.39 & 57.98 & 78.31 & RW2 \\ 
  Tanzania & SHINYANGA & 15-19 & 51.09 & 22.59 & 111.27 & RW2 \\ 
  Tanzania & SINGIDA & 80-84 & 173.27 & 228.70 & 129.03 & HT-Direct \\ 
  Tanzania & SINGIDA & 80-84 & 154.18 & 121.77 & 193.58 & RW2 \\ 
  Tanzania & SINGIDA & 85-89 & 92.39 & 123.10 & 68.74 & HT-Direct \\ 
  Tanzania & SINGIDA & 85-89 & 140.68 & 119.82 & 164.12 & RW2 \\ 
  Tanzania & SINGIDA & 90-94 & 133.16 & 159.69 & 110.46 & HT-Direct \\ 
  Tanzania & SINGIDA & 90-94 & 130.39 & 115.25 & 147.08 & RW2 \\ 
  Tanzania & SINGIDA & 95-99 & 121.56 & 148.91 & 98.64 & HT-Direct \\ 
  Tanzania & SINGIDA & 95-99 & 116.24 & 102.55 & 131.66 & RW2 \\ 
  Tanzania & SINGIDA & 00-04 & 90.11 & 122.18 & 65.83 & HT-Direct \\ 
  Tanzania & SINGIDA & 00-04 & 84.75 & 72.67 & 98.58 & RW2 \\ 
  Tanzania & SINGIDA & 05-09 & 58.54 & 82.86 & 41.04 & HT-Direct \\ 
  Tanzania & SINGIDA & 05-09 & 54.32 & 44.28 & 66.53 & RW2 \\ 
  Tanzania & SINGIDA & 10-14 & 33.32 & 60.69 & 18.06 & HT-Direct \\ 
  Tanzania & SINGIDA & 10-14 & 35.52 & 25.82 & 48.42 & RW2 \\ 
  Tanzania & SINGIDA & 15-19 & 23.51 & 9.22 & 59.03 & RW2 \\ 
  Tanzania & TABORA & 80-84 & 177.31 & 229.12 & 135.16 & HT-Direct \\ 
  Tanzania & TABORA & 80-84 & 174.31 & 139.19 & 215.18 & RW2 \\ 
  Tanzania & TABORA & 85-89 & 129.93 & 168.42 & 99.19 & HT-Direct \\ 
  Tanzania & TABORA & 85-89 & 165.19 & 142.16 & 190.87 & RW2 \\ 
  Tanzania & TABORA & 90-94 & 156.58 & 188.39 & 129.29 & HT-Direct \\ 
  Tanzania & TABORA & 90-94 & 159.56 & 142.77 & 177.80 & RW2 \\ 
  Tanzania & TABORA & 95-99 & 157.20 & 178.14 & 138.32 & HT-Direct \\ 
  Tanzania & TABORA & 95-99 & 147.37 & 133.92 & 162.04 & RW2 \\ 
  Tanzania & TABORA & 00-04 & 106.21 & 126.98 & 88.50 & HT-Direct \\ 
  Tanzania & TABORA & 00-04 & 111.05 & 99.69 & 123.48 & RW2 \\ 
  Tanzania & TABORA & 05-09 & 78.88 & 98.17 & 63.11 & HT-Direct \\ 
  Tanzania & TABORA & 05-09 & 74.86 & 65.31 & 85.71 & RW2 \\ 
  Tanzania & TABORA & 10-14 & 65.78 & 93.23 & 46.00 & HT-Direct \\ 
  Tanzania & TABORA & 10-14 & 52.80 & 42.28 & 65.61 & RW2 \\ 
  Tanzania & TABORA & 15-19 & 37.82 & 15.76 & 87.85 & RW2 \\ 
  Tanzania & TANGA & 80-84 & 140.51 & 178.03 & 109.84 & HT-Direct \\ 
  Tanzania & TANGA & 80-84 & 180.18 & 146.70 & 218.11 & RW2 \\ 
  Tanzania & TANGA & 85-89 & 197.98 & 237.71 & 163.47 & HT-Direct \\ 
  Tanzania & TANGA & 85-89 & 181.83 & 160.82 & 205.39 & RW2 \\ 
  Tanzania & TANGA & 90-94 & 179.36 & 204.06 & 157.06 & HT-Direct \\ 
  Tanzania & TANGA & 90-94 & 171.19 & 154.93 & 188.92 & RW2 \\ 
  Tanzania & TANGA & 95-99 & 131.61 & 159.45 & 108.01 & HT-Direct \\ 
  Tanzania & TANGA & 95-99 & 151.25 & 135.28 & 168.88 & RW2 \\ 
  Tanzania & TANGA & 00-04 & 124.57 & 153.27 & 100.60 & HT-Direct \\ 
  Tanzania & TANGA & 00-04 & 111.51 & 97.45 & 127.49 & RW2 \\ 
  Tanzania & TANGA & 05-09 & 71.88 & 99.55 & 51.46 & HT-Direct \\ 
  Tanzania & TANGA & 05-09 & 71.48 & 59.27 & 86.28 & RW2 \\ 
  Tanzania & TANGA & 10-14 & 55.12 & 96.47 & 30.88 & HT-Direct \\ 
  Tanzania & TANGA & 10-14 & 46.23 & 33.39 & 62.86 & RW2 \\ 
  Tanzania & TANGA & 15-19 & 29.98 & 11.47 & 75.66 & RW2 \\ 
  Togo & ALL & 80-84 & 162.29 & 159.60 & 164.74 & IHME \\ 
  Togo & ALL & 80-84 & 169.92 & 156.97 & 183.70 & RW2 \\ 
  Togo & ALL & 80-84 & 169.65 & 163.43 & 175.54 & UN \\ 
  Togo & ALL & 85-89 & 150.47 & 148.21 & 152.78 & IHME \\ 
  Togo & ALL & 85-89 & 152.75 & 142.29 & 163.77 & RW2 \\ 
  Togo & ALL & 85-89 & 153.33 & 148.78 & 158.66 & UN \\ 
  Togo & ALL & 90-94 & 140.73 & 138.55 & 143.15 & IHME \\ 
  Togo & ALL & 90-94 & 142.54 & 134.78 & 150.69 & RW2 \\ 
  Togo & ALL & 90-94 & 142.10 & 137.90 & 146.75 & UN \\ 
  Togo & ALL & 95-99 & 130.01 & 127.73 & 132.37 & IHME \\ 
  Togo & ALL & 95-99 & 129.65 & 121.30 & 138.41 & RW2 \\ 
  Togo & ALL & 95-99 & 129.92 & 125.92 & 134.00 & UN \\ 
  Togo & ALL & 00-04 & 117.78 & 115.57 & 120.26 & IHME \\ 
  Togo & ALL & 00-04 & 114.36 & 104.74 & 124.80 & RW2 \\ 
  Togo & ALL & 00-04 & 114.23 & 110.54 & 118.01 & UN \\ 
  Togo & ALL & 05-09 & 104.17 & 101.75 & 106.69 & IHME \\ 
  Togo & ALL & 05-09 & 99.05 & 90.33 & 108.52 & RW2 \\ 
  Togo & ALL & 05-09 & 99.20 & 95.31 & 103.13 & UN \\ 
  Togo & ALL & 10-14 & 88.39 & 85.23 & 91.33 & IHME \\ 
  Togo & ALL & 10-14 & 85.70 & 76.61 & 95.67 & RW2 \\ 
  Togo & ALL & 10-14 & 85.64 & 80.54 & 90.83 & UN \\ 
  Togo & CENTRALE & 80-84 & 193.64 & 235.13 & 157.97 & HT-Direct \\ 
  Togo & CENTRALE & 80-84 & 180.86 & 156.40 & 209.14 & RW2 \\ 
  Togo & CENTRALE & 85-89 & 165.58 & 193.29 & 141.14 & HT-Direct \\ 
  Togo & CENTRALE & 85-89 & 156.78 & 140.99 & 173.81 & RW2 \\ 
  Togo & CENTRALE & 90-94 & 150.06 & 172.90 & 129.76 & HT-Direct \\ 
  Togo & CENTRALE & 90-94 & 147.04 & 134.06 & 160.82 & RW2 \\ 
  Togo & CENTRALE & 95-99 & 127.34 & 152.62 & 105.74 & HT-Direct \\ 
  Togo & CENTRALE & 95-99 & 134.78 & 121.12 & 148.94 & RW2 \\ 
  Togo & CENTRALE & 00-04 & 147.89 & 186.12 & 116.39 & HT-Direct \\ 
  Togo & CENTRALE & 00-04 & 127.54 & 112.86 & 143.72 & RW2 \\ 
  Togo & CENTRALE & 05-09 & 104.04 & 126.91 & 84.89 & HT-Direct \\ 
  Togo & CENTRALE & 05-09 & 115.96 & 101.64 & 132.25 & RW2 \\ 
  Togo & CENTRALE & 10-14 & 114.98 & 142.30 & 92.34 & HT-Direct \\ 
  Togo & CENTRALE & 10-14 & 104.12 & 87.65 & 123.71 & RW2 \\ 
  Togo & CENTRALE & 15-19 & 93.28 & 40.99 & 198.68 & RW2 \\ 
  Togo & GRANDE AGGLOMÉRATION DE LOMÉ & 80-84 & 131.45 & 181.08 & 93.87 & HT-Direct \\ 
  Togo & GRANDE AGGLOMÉRATION DE LOMÉ & 80-84 & 122.97 & 97.23 & 153.14 & RW2 \\ 
  Togo & GRANDE AGGLOMÉRATION DE LOMÉ & 85-89 & 100.95 & 129.16 & 78.35 & HT-Direct \\ 
  Togo & GRANDE AGGLOMÉRATION DE LOMÉ & 85-89 & 107.76 & 92.25 & 125.57 & RW2 \\ 
  Togo & GRANDE AGGLOMÉRATION DE LOMÉ & 90-94 & 105.17 & 129.66 & 84.85 & HT-Direct \\ 
  Togo & GRANDE AGGLOMÉRATION DE LOMÉ & 90-94 & 102.03 & 89.75 & 115.91 & RW2 \\ 
  Togo & GRANDE AGGLOMÉRATION DE LOMÉ & 95-99 & 105.65 & 131.83 & 84.16 & HT-Direct \\ 
  Togo & GRANDE AGGLOMÉRATION DE LOMÉ & 95-99 & 91.29 & 80.18 & 104.41 & RW2 \\ 
  Togo & GRANDE AGGLOMÉRATION DE LOMÉ & 00-04 & 80.15 & 103.13 & 61.93 & HT-Direct \\ 
  Togo & GRANDE AGGLOMÉRATION DE LOMÉ & 00-04 & 80.29 & 69.44 & 92.84 & RW2 \\ 
  Togo & GRANDE AGGLOMÉRATION DE LOMÉ & 05-09 & 61.64 & 79.11 & 47.82 & HT-Direct \\ 
  Togo & GRANDE AGGLOMÉRATION DE LOMÉ & 05-09 & 67.19 & 56.91 & 79.00 & RW2 \\ 
  Togo & GRANDE AGGLOMÉRATION DE LOMÉ & 10-14 & 56.67 & 75.03 & 42.59 & HT-Direct \\ 
  Togo & GRANDE AGGLOMÉRATION DE LOMÉ & 10-14 & 56.31 & 45.04 & 70.22 & RW2 \\ 
  Togo & GRANDE AGGLOMÉRATION DE LOMÉ & 15-19 & 47.35 & 19.71 & 109.85 & RW2 \\ 
  Togo & KARA & 80-84 & 195.98 & 238.72 & 159.29 & HT-Direct \\ 
  Togo & KARA & 80-84 & 197.38 & 171.50 & 226.44 & RW2 \\ 
  Togo & KARA & 85-89 & 182.88 & 210.30 & 158.32 & HT-Direct \\ 
  Togo & KARA & 85-89 & 174.61 & 158.34 & 192.28 & RW2 \\ 
  Togo & KARA & 90-94 & 182.71 & 206.33 & 161.24 & HT-Direct \\ 
  Togo & KARA & 90-94 & 166.46 & 153.41 & 180.33 & RW2 \\ 
  Togo & KARA & 95-99 & 137.33 & 159.88 & 117.51 & HT-Direct \\ 
  Togo & KARA & 95-99 & 152.98 & 139.37 & 167.54 & RW2 \\ 
  Togo & KARA & 00-04 & 158.44 & 196.44 & 126.63 & HT-Direct \\ 
  Togo & KARA & 00-04 & 143.79 & 128.20 & 161.20 & RW2 \\ 
  Togo & KARA & 05-09 & 141.04 & 168.28 & 117.59 & HT-Direct \\ 
  Togo & KARA & 05-09 & 127.65 & 112.77 & 144.34 & RW2 \\ 
  Togo & KARA & 10-14 & 102.27 & 124.39 & 83.70 & HT-Direct \\ 
  Togo & KARA & 10-14 & 109.12 & 92.99 & 127.45 & RW2 \\ 
  Togo & KARA & 15-19 & 92.15 & 40.78 & 195.64 & RW2 \\ 
  Togo & MARITIME (SANS AGGLOMÉRATION DE LOMÉ) & 80-84 & 159.26 & 195.63 & 128.58 & HT-Direct \\ 
  Togo & MARITIME (SANS AGGLOMÉRATION DE LOMÉ) & 80-84 & 161.95 & 137.42 & 189.61 & RW2 \\ 
  Togo & MARITIME (SANS AGGLOMÉRATION DE LOMÉ) & 85-89 & 154.28 & 192.27 & 122.66 & HT-Direct \\ 
  Togo & MARITIME (SANS AGGLOMÉRATION DE LOMÉ) & 85-89 & 142.21 & 125.59 & 160.50 & RW2 \\ 
  Togo & MARITIME (SANS AGGLOMÉRATION DE LOMÉ) & 90-94 & 139.09 & 160.78 & 119.90 & HT-Direct \\ 
  Togo & MARITIME (SANS AGGLOMÉRATION DE LOMÉ) & 90-94 & 134.08 & 121.53 & 148.01 & RW2 \\ 
  Togo & MARITIME (SANS AGGLOMÉRATION DE LOMÉ) & 95-99 & 142.11 & 169.17 & 118.75 & HT-Direct \\ 
  Togo & MARITIME (SANS AGGLOMÉRATION DE LOMÉ) & 95-99 & 120.29 & 107.53 & 134.43 & RW2 \\ 
  Togo & MARITIME (SANS AGGLOMÉRATION DE LOMÉ) & 00-04 & 76.84 & 106.96 & 54.69 & HT-Direct \\ 
  Togo & MARITIME (SANS AGGLOMÉRATION DE LOMÉ) & 00-04 & 107.04 & 92.06 & 124.03 & RW2 \\ 
  Togo & MARITIME (SANS AGGLOMÉRATION DE LOMÉ) & 05-09 & 79.65 & 112.58 & 55.74 & HT-Direct \\ 
  Togo & MARITIME (SANS AGGLOMÉRATION DE LOMÉ) & 05-09 & 91.54 & 75.55 & 110.08 & RW2 \\ 
  Togo & MARITIME (SANS AGGLOMÉRATION DE LOMÉ) & 10-14 & 88.47 & 123.68 & 62.56 & HT-Direct \\ 
  Togo & MARITIME (SANS AGGLOMÉRATION DE LOMÉ) & 10-14 & 78.37 & 60.89 & 100.40 & RW2 \\ 
  Togo & MARITIME (SANS AGGLOMÉRATION DE LOMÉ) & 15-19 & 67.60 & 28.23 & 153.30 & RW2 \\ 
  Togo & PLATEAUX & 80-84 & 180.33 & 210.53 & 153.63 & HT-Direct \\ 
  Togo & PLATEAUX & 80-84 & 169.28 & 149.72 & 191.19 & RW2 \\ 
  Togo & PLATEAUX & 85-89 & 146.69 & 166.04 & 129.24 & HT-Direct \\ 
  Togo & PLATEAUX & 85-89 & 146.56 & 133.74 & 160.48 & RW2 \\ 
  Togo & PLATEAUX & 90-94 & 146.52 & 169.98 & 125.81 & HT-Direct \\ 
  Togo & PLATEAUX & 90-94 & 137.39 & 125.64 & 149.74 & RW2 \\ 
  Togo & PLATEAUX & 95-99 & 122.96 & 143.85 & 104.74 & HT-Direct \\ 
  Togo & PLATEAUX & 95-99 & 124.53 & 112.85 & 137.00 & RW2 \\ 
  Togo & PLATEAUX & 00-04 & 116.97 & 140.22 & 97.13 & HT-Direct \\ 
  Togo & PLATEAUX & 00-04 & 115.17 & 102.74 & 128.89 & RW2 \\ 
  Togo & PLATEAUX & 05-09 & 101.42 & 123.60 & 82.84 & HT-Direct \\ 
  Togo & PLATEAUX & 05-09 & 102.82 & 89.98 & 117.20 & RW2 \\ 
  Togo & PLATEAUX & 10-14 & 95.34 & 121.10 & 74.59 & HT-Direct \\ 
  Togo & PLATEAUX & 10-14 & 91.39 & 76.11 & 109.92 & RW2 \\ 
  Togo & PLATEAUX & 15-19 & 81.55 & 35.70 & 176.49 & RW2 \\ 
  Togo & SAVANES & 80-84 & 223.03 & 250.47 & 197.79 & HT-Direct \\ 
  Togo & SAVANES & 80-84 & 210.24 & 189.68 & 232.44 & RW2 \\ 
  Togo & SAVANES & 85-89 & 171.90 & 194.50 & 151.44 & HT-Direct \\ 
  Togo & SAVANES & 85-89 & 182.51 & 167.64 & 198.35 & RW2 \\ 
  Togo & SAVANES & 90-94 & 183.47 & 199.91 & 168.10 & HT-Direct \\ 
  Togo & SAVANES & 90-94 & 171.59 & 160.42 & 183.46 & RW2 \\ 
  Togo & SAVANES & 95-99 & 165.85 & 187.96 & 145.88 & HT-Direct \\ 
  Togo & SAVANES & 95-99 & 154.15 & 141.46 & 167.88 & RW2 \\ 
  Togo & SAVANES & 00-04 & 135.77 & 159.50 & 115.08 & HT-Direct \\ 
  Togo & SAVANES & 00-04 & 138.35 & 124.24 & 154.18 & RW2 \\ 
  Togo & SAVANES & 05-09 & 117.79 & 140.33 & 98.46 & HT-Direct \\ 
  Togo & SAVANES & 05-09 & 115.39 & 101.63 & 130.70 & RW2 \\ 
  Togo & SAVANES & 10-14 & 84.06 & 108.25 & 64.88 & HT-Direct \\ 
  Togo & SAVANES & 10-14 & 91.60 & 74.68 & 110.83 & RW2 \\ 
  Togo & SAVANES & 15-19 & 71.78 & 30.07 & 158.96 & RW2 \\ 
  Uganda & ALL & 80-84 & 192.81 & 189.88 & 195.87 & IHME \\ 
  Uganda & ALL & 80-84 & 209.25 & 199.14 & 219.74 & RW2 \\ 
  Uganda & ALL & 80-84 & 209.20 & 203.14 & 215.58 & UN \\ 
  Uganda & ALL & 85-89 & 177.00 & 174.76 & 179.15 & IHME \\ 
  Uganda & ALL & 85-89 & 192.10 & 182.98 & 201.52 & RW2 \\ 
  Uganda & ALL & 85-89 & 192.14 & 187.02 & 197.24 & UN \\ 
  Uganda & ALL & 90-94 & 162.28 & 160.44 & 164.15 & IHME \\ 
  Uganda & ALL & 90-94 & 179.53 & 171.17 & 188.21 & RW2 \\ 
  Uganda & ALL & 90-94 & 179.55 & 174.34 & 184.65 & UN \\ 
  Uganda & ALL & 95-99 & 143.81 & 141.82 & 145.60 & IHME \\ 
  Uganda & ALL & 95-99 & 163.10 & 155.34 & 171.11 & RW2 \\ 
  Uganda & ALL & 95-99 & 163.03 & 158.73 & 167.42 & UN \\ 
  Uganda & ALL & 00-04 & 121.66 & 119.87 & 123.38 & IHME \\ 
  Uganda & ALL & 00-04 & 132.13 & 125.76 & 138.81 & RW2 \\ 
  Uganda & ALL & 00-04 & 132.14 & 128.32 & 136.05 & UN \\ 
  Uganda & ALL & 05-09 & 99.98 & 97.90 & 102.17 & IHME \\ 
  Uganda & ALL & 05-09 & 92.64 & 86.77 & 98.85 & RW2 \\ 
  Uganda & ALL & 05-09 & 92.82 & 89.49 & 96.21 & UN \\ 
  Uganda & ALL & 10-14 & 80.52 & 77.72 & 83.31 & IHME \\ 
  Uganda & ALL & 10-14 & 66.06 & 56.63 & 76.87 & RW2 \\ 
  Uganda & ALL & 10-14 & 65.56 & 59.96 & 71.10 & UN \\ 
  Uganda & CENTRAL & 80-84 & 199.34 & 216.78 & 182.98 & HT-Direct \\ 
  Uganda & CENTRAL & 80-84 & 205.05 & 188.71 & 222.59 & RW2 \\ 
  Uganda & CENTRAL & 85-89 & 160.57 & 174.67 & 147.41 & HT-Direct \\ 
  Uganda & CENTRAL & 85-89 & 179.52 & 167.05 & 192.61 & RW2 \\ 
  Uganda & CENTRAL & 90-94 & 141.10 & 153.63 & 129.44 & HT-Direct \\ 
  Uganda & CENTRAL & 90-94 & 159.40 & 147.61 & 171.34 & RW2 \\ 
  Uganda & CENTRAL & 95-99 & 130.62 & 146.02 & 116.62 & HT-Direct \\ 
  Uganda & CENTRAL & 95-99 & 143.28 & 132.09 & 154.87 & RW2 \\ 
  Uganda & CENTRAL & 00-04 & 121.10 & 134.73 & 108.68 & HT-Direct \\ 
  Uganda & CENTRAL & 00-04 & 116.86 & 107.73 & 127.25 & RW2 \\ 
  Uganda & CENTRAL & 05-09 & 99.55 & 116.96 & 84.49 & HT-Direct \\ 
  Uganda & CENTRAL & 05-09 & 80.79 & 72.01 & 90.69 & RW2 \\ 
  Uganda & CENTRAL & 10-14 & 49.81 & 76.54 & 32.10 & HT-Direct \\ 
  Uganda & CENTRAL & 10-14 & 54.13 & 41.62 & 68.35 & RW2 \\ 
  Uganda & CENTRAL & 15-19 & 36.07 & 13.96 & 87.93 & RW2 \\ 
  Uganda & EASTERN & 80-84 & 206.16 & 224.77 & 188.72 & HT-Direct \\ 
  Uganda & EASTERN & 80-84 & 219.33 & 201.13 & 238.52 & RW2 \\ 
  Uganda & EASTERN & 85-89 & 183.74 & 198.11 & 170.20 & HT-Direct \\ 
  Uganda & EASTERN & 85-89 & 201.02 & 187.34 & 216.11 & RW2 \\ 
  Uganda & EASTERN & 90-94 & 164.58 & 178.53 & 151.51 & HT-Direct \\ 
  Uganda & EASTERN & 90-94 & 175.65 & 163.41 & 189.44 & RW2 \\ 
  Uganda & EASTERN & 95-99 & 136.88 & 148.83 & 125.75 & HT-Direct \\ 
  Uganda & EASTERN & 95-99 & 145.83 & 135.58 & 156.33 & RW2 \\ 
  Uganda & EASTERN & 00-04 & 108.13 & 119.40 & 97.81 & HT-Direct \\ 
  Uganda & EASTERN & 00-04 & 112.33 & 103.32 & 121.50 & RW2 \\ 
  Uganda & EASTERN & 05-09 & 92.10 & 104.86 & 80.75 & HT-Direct \\ 
  Uganda & EASTERN & 05-09 & 81.19 & 73.51 & 89.75 & RW2 \\ 
  Uganda & EASTERN & 10-14 & 92.51 & 133.51 & 63.18 & HT-Direct \\ 
  Uganda & EASTERN & 10-14 & 61.75 & 48.99 & 79.03 & RW2 \\ 
  Uganda & EASTERN & 15-19 & 47.94 & 19.37 & 117.89 & RW2 \\ 
  Uganda & NORTHERN & 80-84 & 240.53 & 267.97 & 215.08 & HT-Direct \\ 
  Uganda & NORTHERN & 80-84 & 249.69 & 225.48 & 275.59 & RW2 \\ 
  Uganda & NORTHERN & 85-89 & 207.01 & 226.98 & 188.36 & HT-Direct \\ 
  Uganda & NORTHERN & 85-89 & 229.40 & 212.64 & 246.74 & RW2 \\ 
  Uganda & NORTHERN & 90-94 & 187.23 & 202.19 & 173.14 & HT-Direct \\ 
  Uganda & NORTHERN & 90-94 & 210.86 & 197.25 & 225.08 & RW2 \\ 
  Uganda & NORTHERN & 95-99 & 183.51 & 198.25 & 169.62 & HT-Direct \\ 
  Uganda & NORTHERN & 95-99 & 191.99 & 180.23 & 204.40 & RW2 \\ 
  Uganda & NORTHERN & 00-04 & 160.86 & 174.74 & 147.88 & HT-Direct \\ 
  Uganda & NORTHERN & 00-04 & 155.41 & 145.46 & 166.10 & RW2 \\ 
  Uganda & NORTHERN & 05-09 & 113.60 & 130.90 & 98.32 & HT-Direct \\ 
  Uganda & NORTHERN & 05-09 & 108.29 & 97.62 & 120.03 & RW2 \\ 
  Uganda & NORTHERN & 10-14 & 119.45 & 185.76 & 74.64 & HT-Direct \\ 
  Uganda & NORTHERN & 10-14 & 77.15 & 60.67 & 97.50 & RW2 \\ 
  Uganda & NORTHERN & 15-19 & 55.68 & 22.33 & 132.93 & RW2 \\ 
  Uganda & WESTERN & 80-84 & 180.34 & 196.02 & 165.64 & HT-Direct \\ 
  Uganda & WESTERN & 80-84 & 186.09 & 171.22 & 201.88 & RW2 \\ 
  Uganda & WESTERN & 85-89 & 157.84 & 171.68 & 144.92 & HT-Direct \\ 
  Uganda & WESTERN & 85-89 & 180.95 & 167.60 & 194.57 & RW2 \\ 
  Uganda & WESTERN & 90-94 & 160.98 & 175.76 & 147.22 & HT-Direct \\ 
  Uganda & WESTERN & 90-94 & 180.79 & 167.74 & 194.52 & RW2 \\ 
  Uganda & WESTERN & 95-99 & 178.83 & 195.93 & 162.91 & HT-Direct \\ 
  Uganda & WESTERN & 95-99 & 177.12 & 164.32 & 191.62 & RW2 \\ 
  Uganda & WESTERN & 00-04 & 147.38 & 162.37 & 133.56 & HT-Direct \\ 
  Uganda & WESTERN & 00-04 & 146.32 & 135.77 & 157.89 & RW2 \\ 
  Uganda & WESTERN & 05-09 & 106.91 & 122.47 & 93.13 & HT-Direct \\ 
  Uganda & WESTERN & 05-09 & 101.81 & 91.88 & 112.39 & RW2 \\ 
  Uganda & WESTERN & 10-14 & 103.80 & 135.27 & 78.98 & HT-Direct \\ 
  Uganda & WESTERN & 10-14 & 72.82 & 60.48 & 87.38 & RW2 \\ 
  Uganda & WESTERN & 15-19 & 52.94 & 22.33 & 121.53 & RW2 \\ 
  Zambia & ALL & 80-84 & 160.63 & 156.43 & 164.93 & IHME \\ 
  Zambia & ALL & 80-84 & 162.11 & 154.29 & 170.24 & RW2 \\ 
  Zambia & ALL & 80-84 & 162.09 & 157.42 & 167.22 & UN \\ 
  Zambia & ALL & 85-89 & 170.45 & 166.39 & 174.58 & IHME \\ 
  Zambia & ALL & 85-89 & 183.49 & 176.41 & 190.76 & RW2 \\ 
  Zambia & ALL & 85-89 & 183.57 & 178.67 & 188.63 & UN \\ 
  Zambia & ALL & 90-94 & 168.58 & 164.39 & 172.06 & IHME \\ 
  Zambia & ALL & 90-94 & 188.48 & 181.48 & 195.69 & RW2 \\ 
  Zambia & ALL & 90-94 & 188.27 & 183.24 & 193.10 & UN \\ 
  Zambia & ALL & 95-99 & 154.77 & 150.93 & 158.70 & IHME \\ 
  Zambia & ALL & 95-99 & 174.79 & 166.50 & 183.32 & RW2 \\ 
  Zambia & ALL & 95-99 & 175.17 & 170.41 & 180.08 & UN \\ 
  Zambia & ALL & 00-04 & 130.72 & 126.83 & 134.32 & IHME \\ 
  Zambia & ALL & 00-04 & 142.23 & 134.50 & 150.41 & RW2 \\ 
  Zambia & ALL & 00-04 & 141.89 & 137.34 & 147.01 & UN \\ 
  Zambia & ALL & 05-09 & 98.98 & 95.39 & 102.72 & IHME \\ 
  Zambia & ALL & 05-09 & 98.44 & 90.12 & 107.34 & RW2 \\ 
  Zambia & ALL & 05-09 & 98.69 & 95.12 & 102.47 & UN \\ 
  Zambia & ALL & 10-14 & 74.07 & 70.17 & 78.06 & IHME \\ 
  Zambia & ALL & 10-14 & 74.47 & 67.75 & 81.74 & RW2 \\ 
  Zambia & ALL & 10-14 & 74.30 & 70.13 & 78.89 & UN \\ 
  Zambia & CENTRAL & 80-84 & 139.03 & 166.84 & 115.21 & HT-Direct \\ 
  Zambia & CENTRAL & 80-84 & 141.44 & 123.97 & 160.98 & RW2 \\ 
  Zambia & CENTRAL & 85-89 & 159.22 & 180.17 & 140.30 & HT-Direct \\ 
  Zambia & CENTRAL & 85-89 & 162.15 & 148.34 & 176.89 & RW2 \\ 
  Zambia & CENTRAL & 90-94 & 155.41 & 177.30 & 135.77 & HT-Direct \\ 
  Zambia & CENTRAL & 90-94 & 170.36 & 157.73 & 183.94 & RW2 \\ 
  Zambia & CENTRAL & 95-99 & 154.71 & 175.45 & 136.01 & HT-Direct \\ 
  Zambia & CENTRAL & 95-99 & 160.88 & 147.78 & 174.88 & RW2 \\ 
  Zambia & CENTRAL & 00-04 & 115.39 & 134.57 & 98.63 & HT-Direct \\ 
  Zambia & CENTRAL & 00-04 & 131.47 & 118.55 & 145.74 & RW2 \\ 
  Zambia & CENTRAL & 05-09 & 89.09 & 113.77 & 69.35 & HT-Direct \\ 
  Zambia & CENTRAL & 05-09 & 94.23 & 80.87 & 109.38 & RW2 \\ 
  Zambia & CENTRAL & 10-14 & 62.44 & 82.39 & 47.07 & HT-Direct \\ 
  Zambia & CENTRAL & 10-14 & 71.10 & 58.09 & 86.68 & RW2 \\ 
  Zambia & CENTRAL & 15-19 & 55.20 & 23.65 & 124.47 & RW2 \\ 
  Zambia & COPPERBELT & 80-84 & 115.91 & 135.07 & 99.16 & HT-Direct \\ 
  Zambia & COPPERBELT & 80-84 & 126.88 & 111.37 & 143.62 & RW2 \\ 
  Zambia & COPPERBELT & 85-89 & 153.23 & 170.44 & 137.47 & HT-Direct \\ 
  Zambia & COPPERBELT & 85-89 & 152.56 & 140.97 & 165.04 & RW2 \\ 
  Zambia & COPPERBELT & 90-94 & 163.17 & 180.78 & 146.97 & HT-Direct \\ 
  Zambia & COPPERBELT & 90-94 & 163.91 & 151.80 & 177.13 & RW2 \\ 
  Zambia & COPPERBELT & 95-99 & 135.13 & 155.96 & 116.69 & HT-Direct \\ 
  Zambia & COPPERBELT & 95-99 & 153.78 & 139.90 & 169.01 & RW2 \\ 
  Zambia & COPPERBELT & 00-04 & 113.06 & 134.66 & 94.54 & HT-Direct \\ 
  Zambia & COPPERBELT & 00-04 & 123.56 & 109.89 & 138.65 & RW2 \\ 
  Zambia & COPPERBELT & 05-09 & 61.52 & 78.03 & 48.33 & HT-Direct \\ 
  Zambia & COPPERBELT & 05-09 & 86.87 & 73.52 & 102.11 & RW2 \\ 
  Zambia & COPPERBELT & 10-14 & 68.20 & 92.24 & 50.09 & HT-Direct \\ 
  Zambia & COPPERBELT & 10-14 & 65.05 & 51.13 & 82.02 & RW2 \\ 
  Zambia & COPPERBELT & 15-19 & 50.17 & 20.84 & 117.01 & RW2 \\ 
  Zambia & EASTERN & 80-84 & 225.13 & 250.32 & 201.80 & HT-Direct \\ 
  Zambia & EASTERN & 80-84 & 224.73 & 204.93 & 246.95 & RW2 \\ 
  Zambia & EASTERN & 85-89 & 237.44 & 256.59 & 219.29 & HT-Direct \\ 
  Zambia & EASTERN & 85-89 & 237.56 & 224.00 & 251.88 & RW2 \\ 
  Zambia & EASTERN & 90-94 & 223.14 & 239.77 & 207.36 & HT-Direct \\ 
  Zambia & EASTERN & 90-94 & 233.48 & 220.22 & 246.87 & RW2 \\ 
  Zambia & EASTERN & 95-99 & 174.24 & 192.45 & 157.42 & HT-Direct \\ 
  Zambia & EASTERN & 95-99 & 212.28 & 196.76 & 227.70 & RW2 \\ 
  Zambia & EASTERN & 00-04 & 155.98 & 175.99 & 137.86 & HT-Direct \\ 
  Zambia & EASTERN & 00-04 & 174.63 & 160.20 & 189.88 & RW2 \\ 
  Zambia & EASTERN & 05-09 & 122.48 & 140.91 & 106.16 & HT-Direct \\ 
  Zambia & EASTERN & 05-09 & 129.50 & 115.13 & 145.49 & RW2 \\ 
  Zambia & EASTERN & 10-14 & 102.80 & 125.31 & 83.94 & HT-Direct \\ 
  Zambia & EASTERN & 10-14 & 101.32 & 85.70 & 120.46 & RW2 \\ 
  Zambia & EASTERN & 15-19 & 81.46 & 35.55 & 177.18 & RW2 \\ 
  Zambia & LUAPULA & 80-84 & 209.28 & 232.85 & 187.51 & HT-Direct \\ 
  Zambia & LUAPULA & 80-84 & 217.36 & 196.60 & 239.18 & RW2 \\ 
  Zambia & LUAPULA & 85-89 & 245.51 & 270.35 & 222.27 & HT-Direct \\ 
  Zambia & LUAPULA & 85-89 & 244.74 & 227.99 & 262.60 & RW2 \\ 
  Zambia & LUAPULA & 90-94 & 238.09 & 261.43 & 216.22 & HT-Direct \\ 
  Zambia & LUAPULA & 90-94 & 248.00 & 230.90 & 266.23 & RW2 \\ 
  Zambia & LUAPULA & 95-99 & 208.33 & 232.79 & 185.81 & HT-Direct \\ 
  Zambia & LUAPULA & 95-99 & 222.94 & 204.56 & 242.41 & RW2 \\ 
  Zambia & LUAPULA & 00-04 & 136.74 & 160.11 & 116.31 & HT-Direct \\ 
  Zambia & LUAPULA & 00-04 & 170.91 & 152.08 & 190.78 & RW2 \\ 
  Zambia & LUAPULA & 05-09 & 103.77 & 132.77 & 80.51 & HT-Direct \\ 
  Zambia & LUAPULA & 05-09 & 115.40 & 97.04 & 136.61 & RW2 \\ 
  Zambia & LUAPULA & 10-14 & 85.84 & 119.46 & 61.03 & HT-Direct \\ 
  Zambia & LUAPULA & 10-14 & 81.73 & 62.58 & 106.53 & RW2 \\ 
  Zambia & LUAPULA & 15-19 & 59.06 & 23.86 & 141.61 & RW2 \\ 
  Zambia & LUSAKA & 80-84 & 120.34 & 139.56 & 103.45 & HT-Direct \\ 
  Zambia & LUSAKA & 80-84 & 122.99 & 108.67 & 138.45 & RW2 \\ 
  Zambia & LUSAKA & 85-89 & 133.64 & 150.01 & 118.82 & HT-Direct \\ 
  Zambia & LUSAKA & 85-89 & 145.29 & 133.75 & 157.32 & RW2 \\ 
  Zambia & LUSAKA & 90-94 & 170.62 & 191.12 & 151.91 & HT-Direct \\ 
  Zambia & LUSAKA & 90-94 & 156.96 & 145.08 & 169.90 & RW2 \\ 
  Zambia & LUSAKA & 95-99 & 111.96 & 131.64 & 94.91 & HT-Direct \\ 
  Zambia & LUSAKA & 95-99 & 150.31 & 136.84 & 165.11 & RW2 \\ 
  Zambia & LUSAKA & 00-04 & 128.59 & 149.42 & 110.29 & HT-Direct \\ 
  Zambia & LUSAKA & 00-04 & 125.63 & 112.60 & 140.08 & RW2 \\ 
  Zambia & LUSAKA & 05-09 & 60.99 & 80.81 & 45.79 & HT-Direct \\ 
  Zambia & LUSAKA & 05-09 & 92.13 & 78.62 & 107.55 & RW2 \\ 
  Zambia & LUSAKA & 10-14 & 67.92 & 87.77 & 52.31 & HT-Direct \\ 
  Zambia & LUSAKA & 10-14 & 71.54 & 57.96 & 87.67 & RW2 \\ 
  Zambia & LUSAKA & 15-19 & 56.98 & 24.38 & 129.54 & RW2 \\ 
  Zambia & NORTH-WESTERN & 80-84 & 151.29 & 171.09 & 133.40 & HT-Direct \\ 
  Zambia & NORTH-WESTERN & 80-84 & 150.24 & 134.84 & 167.14 & RW2 \\ 
  Zambia & NORTH-WESTERN & 85-89 & 155.72 & 176.06 & 137.34 & HT-Direct \\ 
  Zambia & NORTH-WESTERN & 85-89 & 162.18 & 149.90 & 175.08 & RW2 \\ 
  Zambia & NORTH-WESTERN & 90-94 & 156.39 & 171.39 & 142.47 & HT-Direct \\ 
  Zambia & NORTH-WESTERN & 90-94 & 161.43 & 150.37 & 172.97 & RW2 \\ 
  Zambia & NORTH-WESTERN & 95-99 & 120.63 & 137.15 & 105.86 & HT-Direct \\ 
  Zambia & NORTH-WESTERN & 95-99 & 144.85 & 132.62 & 157.72 & RW2 \\ 
  Zambia & NORTH-WESTERN & 00-04 & 109.08 & 128.30 & 92.43 & HT-Direct \\ 
  Zambia & NORTH-WESTERN & 00-04 & 113.42 & 101.48 & 126.59 & RW2 \\ 
  Zambia & NORTH-WESTERN & 05-09 & 63.32 & 86.99 & 45.76 & HT-Direct \\ 
  Zambia & NORTH-WESTERN & 05-09 & 77.99 & 66.34 & 91.71 & RW2 \\ 
  Zambia & NORTH-WESTERN & 10-14 & 56.94 & 72.97 & 44.26 & HT-Direct \\ 
  Zambia & NORTH-WESTERN & 10-14 & 56.98 & 46.30 & 70.27 & RW2 \\ 
  Zambia & NORTH-WESTERN & 15-19 & 42.85 & 18.19 & 99.74 & RW2 \\ 
  Zambia & NORTHERN & 80-84 & 187.68 & 210.16 & 167.10 & HT-Direct \\ 
  Zambia & NORTHERN & 80-84 & 190.68 & 172.76 & 209.89 & RW2 \\ 
  Zambia & NORTHERN & 85-89 & 210.22 & 231.13 & 190.73 & HT-Direct \\ 
  Zambia & NORTHERN & 85-89 & 215.98 & 201.95 & 230.74 & RW2 \\ 
  Zambia & NORTHERN & 90-94 & 204.84 & 223.59 & 187.29 & HT-Direct \\ 
  Zambia & NORTHERN & 90-94 & 224.00 & 210.66 & 238.15 & RW2 \\ 
  Zambia & NORTHERN & 95-99 & 202.68 & 221.12 & 185.41 & HT-Direct \\ 
  Zambia & NORTHERN & 95-99 & 209.18 & 195.22 & 224.38 & RW2 \\ 
  Zambia & NORTHERN & 00-04 & 154.61 & 173.16 & 137.71 & HT-Direct \\ 
  Zambia & NORTHERN & 00-04 & 166.47 & 153.05 & 181.21 & RW2 \\ 
  Zambia & NORTHERN & 05-09 & 97.49 & 113.13 & 83.80 & HT-Direct \\ 
  Zambia & NORTHERN & 05-09 & 113.94 & 100.90 & 128.16 & RW2 \\ 
  Zambia & NORTHERN & 10-14 & 68.49 & 85.18 & 54.88 & HT-Direct \\ 
  Zambia & NORTHERN & 10-14 & 80.42 & 66.77 & 95.55 & RW2 \\ 
  Zambia & NORTHERN & 15-19 & 57.84 & 24.84 & 129.04 & RW2 \\ 
  Zambia & SOUTHERN & 80-84 & 133.64 & 152.85 & 116.51 & HT-Direct \\ 
  Zambia & SOUTHERN & 80-84 & 132.61 & 118.27 & 148.39 & RW2 \\ 
  Zambia & SOUTHERN & 85-89 & 140.17 & 156.59 & 125.21 & HT-Direct \\ 
  Zambia & SOUTHERN & 85-89 & 147.88 & 136.65 & 159.79 & RW2 \\ 
  Zambia & SOUTHERN & 90-94 & 151.73 & 170.32 & 134.84 & HT-Direct \\ 
  Zambia & SOUTHERN & 90-94 & 152.38 & 140.89 & 164.43 & RW2 \\ 
  Zambia & SOUTHERN & 95-99 & 122.63 & 141.13 & 106.25 & HT-Direct \\ 
  Zambia & SOUTHERN & 95-99 & 141.17 & 128.88 & 154.25 & RW2 \\ 
  Zambia & SOUTHERN & 00-04 & 101.32 & 116.45 & 87.96 & HT-Direct \\ 
  Zambia & SOUTHERN & 00-04 & 114.03 & 102.84 & 126.39 & RW2 \\ 
  Zambia & SOUTHERN & 05-09 & 65.83 & 83.22 & 51.88 & HT-Direct \\ 
  Zambia & SOUTHERN & 05-09 & 81.51 & 70.28 & 94.23 & RW2 \\ 
  Zambia & SOUTHERN & 10-14 & 64.05 & 80.96 & 50.48 & HT-Direct \\ 
  Zambia & SOUTHERN & 10-14 & 62.11 & 51.18 & 75.36 & RW2 \\ 
  Zambia & SOUTHERN & 15-19 & 48.71 & 20.96 & 111.45 & RW2 \\ 
  Zambia & WESTERN & 80-84 & 209.67 & 234.78 & 186.59 & HT-Direct \\ 
  Zambia & WESTERN & 80-84 & 208.63 & 188.23 & 230.49 & RW2 \\ 
  Zambia & WESTERN & 85-89 & 207.78 & 235.99 & 182.14 & HT-Direct \\ 
  Zambia & WESTERN & 85-89 & 217.38 & 200.70 & 234.89 & RW2 \\ 
  Zambia & WESTERN & 90-94 & 187.88 & 210.81 & 166.92 & HT-Direct \\ 
  Zambia & WESTERN & 90-94 & 211.43 & 196.01 & 227.72 & RW2 \\ 
  Zambia & WESTERN & 95-99 & 190.19 & 213.64 & 168.77 & HT-Direct \\ 
  Zambia & WESTERN & 95-99 & 186.99 & 171.42 & 204.08 & RW2 \\ 
  Zambia & WESTERN & 00-04 & 128.18 & 152.08 & 107.57 & HT-Direct \\ 
  Zambia & WESTERN & 00-04 & 141.85 & 126.27 & 159.09 & RW2 \\ 
  Zambia & WESTERN & 05-09 & 76.02 & 104.75 & 54.69 & HT-Direct \\ 
  Zambia & WESTERN & 05-09 & 92.84 & 77.56 & 110.72 & RW2 \\ 
  Zambia & WESTERN & 10-14 & 55.93 & 79.49 & 39.05 & HT-Direct \\ 
  Zambia & WESTERN & 10-14 & 63.51 & 48.98 & 81.41 & RW2 \\ 
  Zambia & WESTERN & 15-19 & 44.33 & 18.33 & 104.42 & RW2 \\ 
  Zimbabwe & ALL & 80-84 & 83.19 & 81.66 & 84.71 & IHME \\ 
  Zimbabwe & ALL & 80-84 & 94.31 & 86.38 & 102.91 & RW2 \\ 
  Zimbabwe & ALL & 80-84 & 94.36 & 91.13 & 97.64 & UN \\ 
  Zimbabwe & ALL & 85-89 & 69.14 & 68.10 & 70.27 & IHME \\ 
  Zimbabwe & ALL & 85-89 & 75.82 & 69.42 & 82.69 & RW2 \\ 
  Zimbabwe & ALL & 85-89 & 75.76 & 73.21 & 78.66 & UN \\ 
  Zimbabwe & ALL & 90-94 & 66.99 & 65.87 & 68.08 & IHME \\ 
  Zimbabwe & ALL & 90-94 & 83.27 & 75.84 & 91.36 & RW2 \\ 
  Zimbabwe & ALL & 90-94 & 83.14 & 80.33 & 86.01 & UN \\ 
  Zimbabwe & ALL & 95-99 & 74.12 & 72.87 & 75.48 & IHME \\ 
  Zimbabwe & ALL & 95-99 & 100.70 & 89.87 & 112.71 & RW2 \\ 
  Zimbabwe & ALL & 95-99 & 101.16 & 97.31 & 105.10 & UN \\ 
  Zimbabwe & ALL & 00-04 & 81.03 & 79.51 & 82.40 & IHME \\ 
  Zimbabwe & ALL & 00-04 & 104.27 & 92.34 & 117.61 & RW2 \\ 
  Zimbabwe & ALL & 00-04 & 104.55 & 100.33 & 109.08 & UN \\ 
  Zimbabwe & ALL & 05-09 & 81.90 & 79.93 & 83.63 & IHME \\ 
  Zimbabwe & ALL & 05-09 & 97.79 & 89.06 & 107.28 & RW2 \\ 
  Zimbabwe & ALL & 05-09 & 97.60 & 93.32 & 102.22 & UN \\ 
  Zimbabwe & ALL & 10-14 & 68.31 & 65.57 & 71.12 & IHME \\ 
  Zimbabwe & ALL & 10-14 & 89.65 & 30.17 & 235.98 & RW2 \\ 
  Zimbabwe & ALL & 10-14 & 81.03 & 73.89 & 88.62 & UN \\ 
  Zimbabwe & BULAWAYO & 80-84 & 56.78 & 80.95 & 39.52 & HT-Direct \\ 
  Zimbabwe & BULAWAYO & 80-84 & 57.19 & 42.20 & 77.19 & RW2 \\ 
  Zimbabwe & BULAWAYO & 85-89 & 37.93 & 56.14 & 25.46 & HT-Direct \\ 
  Zimbabwe & BULAWAYO & 85-89 & 46.86 & 36.77 & 59.54 & RW2 \\ 
  Zimbabwe & BULAWAYO & 90-94 & 41.02 & 60.73 & 27.52 & HT-Direct \\ 
  Zimbabwe & BULAWAYO & 90-94 & 55.60 & 44.28 & 69.69 & RW2 \\ 
  Zimbabwe & BULAWAYO & 95-99 & 77.88 & 105.46 & 57.06 & HT-Direct \\ 
  Zimbabwe & BULAWAYO & 95-99 & 77.64 & 61.27 & 98.21 & RW2 \\ 
  Zimbabwe & BULAWAYO & 00-04 & 73.18 & 105.05 & 50.44 & HT-Direct \\ 
  Zimbabwe & BULAWAYO & 00-04 & 76.29 & 58.43 & 99.21 & RW2 \\ 
  Zimbabwe & BULAWAYO & 05-09 & 48.23 & 64.27 & 36.04 & HT-Direct \\ 
  Zimbabwe & BULAWAYO & 05-09 & 64.13 & 47.44 & 86.20 & RW2 \\ 
  Zimbabwe & BULAWAYO & 10-14 & 53.07 & 16.78 & 160.27 & RW2 \\ 
  Zimbabwe & BULAWAYO & 15-19 & 43.35 & 2.57 & 463.19 & RW2 \\ 
  Zimbabwe & HARARE & 80-84 & 61.37 & 87.32 & 42.76 & HT-Direct \\ 
  Zimbabwe & HARARE & 80-84 & 68.29 & 51.50 & 89.95 & RW2 \\ 
  Zimbabwe & HARARE & 85-89 & 55.35 & 75.34 & 40.43 & HT-Direct \\ 
  Zimbabwe & HARARE & 85-89 & 57.53 & 46.29 & 71.19 & RW2 \\ 
  Zimbabwe & HARARE & 90-94 & 55.23 & 76.41 & 39.66 & HT-Direct \\ 
  Zimbabwe & HARARE & 90-94 & 68.99 & 56.38 & 84.34 & RW2 \\ 
  Zimbabwe & HARARE & 95-99 & 80.53 & 113.45 & 56.56 & HT-Direct \\ 
  Zimbabwe & HARARE & 95-99 & 96.65 & 77.48 & 120.57 & RW2 \\ 
  Zimbabwe & HARARE & 00-04 & 77.85 & 108.96 & 55.08 & HT-Direct \\ 
  Zimbabwe & HARARE & 00-04 & 96.16 & 74.29 & 123.64 & RW2 \\ 
  Zimbabwe & HARARE & 05-09 & 63.85 & 85.08 & 47.64 & HT-Direct \\ 
  Zimbabwe & HARARE & 05-09 & 82.78 & 61.47 & 110.32 & RW2 \\ 
  Zimbabwe & HARARE & 10-14 & 70.20 & 22.76 & 202.06 & RW2 \\ 
  Zimbabwe & HARARE & 15-19 & 59.55 & 3.63 & 541.77 & RW2 \\ 
  Zimbabwe & MANICALAND & 80-84 & 115.70 & 141.13 & 94.34 & HT-Direct \\ 
  Zimbabwe & MANICALAND & 80-84 & 111.94 & 93.48 & 133.71 & RW2 \\ 
  Zimbabwe & MANICALAND & 85-89 & 75.61 & 95.13 & 59.83 & HT-Direct \\ 
  Zimbabwe & MANICALAND & 85-89 & 93.95 & 80.70 & 109.12 & RW2 \\ 
  Zimbabwe & MANICALAND & 90-94 & 96.11 & 115.87 & 79.41 & HT-Direct \\ 
  Zimbabwe & MANICALAND & 90-94 & 112.84 & 96.86 & 130.65 & RW2 \\ 
  Zimbabwe & MANICALAND & 95-99 & 122.78 & 156.90 & 95.24 & HT-Direct \\ 
  Zimbabwe & MANICALAND & 95-99 & 157.80 & 131.55 & 188.56 & RW2 \\ 
  Zimbabwe & MANICALAND & 00-04 & 125.63 & 157.98 & 99.12 & HT-Direct \\ 
  Zimbabwe & MANICALAND & 00-04 & 160.39 & 130.72 & 195.71 & RW2 \\ 
  Zimbabwe & MANICALAND & 05-09 & 122.91 & 153.24 & 97.89 & HT-Direct \\ 
  Zimbabwe & MANICALAND & 05-09 & 142.10 & 112.54 & 177.86 & RW2 \\ 
  Zimbabwe & MANICALAND & 10-14 & 123.37 & 41.90 & 319.44 & RW2 \\ 
  Zimbabwe & MANICALAND & 15-19 & 106.67 & 6.88 & 688.63 & RW2 \\ 
  Zimbabwe & MASHONALAND CENTRAL & 80-84 & 143.98 & 167.49 & 123.28 & HT-Direct \\ 
  Zimbabwe & MASHONALAND CENTRAL & 80-84 & 144.76 & 125.40 & 166.79 & RW2 \\ 
  Zimbabwe & MASHONALAND CENTRAL & 85-89 & 96.14 & 118.66 & 77.52 & HT-Direct \\ 
  Zimbabwe & MASHONALAND CENTRAL & 85-89 & 110.65 & 96.27 & 126.98 & RW2 \\ 
  Zimbabwe & MASHONALAND CENTRAL & 90-94 & 96.26 & 120.80 & 76.28 & HT-Direct \\ 
  Zimbabwe & MASHONALAND CENTRAL & 90-94 & 118.62 & 101.74 & 137.69 & RW2 \\ 
  Zimbabwe & MASHONALAND CENTRAL & 95-99 & 111.69 & 140.45 & 88.21 & HT-Direct \\ 
  Zimbabwe & MASHONALAND CENTRAL & 95-99 & 148.64 & 124.45 & 176.39 & RW2 \\ 
  Zimbabwe & MASHONALAND CENTRAL & 00-04 & 104.70 & 133.12 & 81.77 & HT-Direct \\ 
  Zimbabwe & MASHONALAND CENTRAL & 00-04 & 135.94 & 111.26 & 165.03 & RW2 \\ 
  Zimbabwe & MASHONALAND CENTRAL & 05-09 & 94.18 & 114.24 & 77.34 & HT-Direct \\ 
  Zimbabwe & MASHONALAND CENTRAL & 05-09 & 108.00 & 87.77 & 132.19 & RW2 \\ 
  Zimbabwe & MASHONALAND CENTRAL & 10-14 & 83.94 & 28.27 & 231.48 & RW2 \\ 
  Zimbabwe & MASHONALAND CENTRAL & 15-19 & 65.22 & 4.00 & 558.63 & RW2 \\ 
  Zimbabwe & MASHONALAND EAST & 80-84 & 72.89 & 99.69 & 52.87 & HT-Direct \\ 
  Zimbabwe & MASHONALAND EAST & 80-84 & 86.79 & 68.40 & 109.15 & RW2 \\ 
  Zimbabwe & MASHONALAND EAST & 85-89 & 75.65 & 95.97 & 59.35 & HT-Direct \\ 
  Zimbabwe & MASHONALAND EAST & 85-89 & 74.52 & 62.27 & 88.99 & RW2 \\ 
  Zimbabwe & MASHONALAND EAST & 90-94 & 72.74 & 97.42 & 53.94 & HT-Direct \\ 
  Zimbabwe & MASHONALAND EAST & 90-94 & 90.61 & 76.77 & 106.79 & RW2 \\ 
  Zimbabwe & MASHONALAND EAST & 95-99 & 114.25 & 140.56 & 92.34 & HT-Direct \\ 
  Zimbabwe & MASHONALAND EAST & 95-99 & 127.98 & 107.46 & 152.02 & RW2 \\ 
  Zimbabwe & MASHONALAND EAST & 00-04 & 68.65 & 95.90 & 48.73 & HT-Direct \\ 
  Zimbabwe & MASHONALAND EAST & 00-04 & 129.48 & 105.37 & 157.67 & RW2 \\ 
  Zimbabwe & MASHONALAND EAST & 05-09 & 96.47 & 116.86 & 79.32 & HT-Direct \\ 
  Zimbabwe & MASHONALAND EAST & 05-09 & 113.82 & 92.45 & 139.48 & RW2 \\ 
  Zimbabwe & MASHONALAND EAST & 10-14 & 98.29 & 33.97 & 263.36 & RW2 \\ 
  Zimbabwe & MASHONALAND EAST & 15-19 & 84.35 & 5.55 & 627.45 & RW2 \\ 
  Zimbabwe & MASHONALAND WEST & 80-84 & 90.50 & 116.39 & 69.92 & HT-Direct \\ 
  Zimbabwe & MASHONALAND WEST & 80-84 & 109.32 & 90.15 & 131.52 & RW2 \\ 
  Zimbabwe & MASHONALAND WEST & 85-89 & 98.02 & 115.83 & 82.70 & HT-Direct \\ 
  Zimbabwe & MASHONALAND WEST & 85-89 & 92.03 & 79.95 & 105.35 & RW2 \\ 
  Zimbabwe & MASHONALAND WEST & 90-94 & 90.27 & 109.62 & 74.05 & HT-Direct \\ 
  Zimbabwe & MASHONALAND WEST & 90-94 & 108.46 & 94.43 & 124.26 & RW2 \\ 
  Zimbabwe & MASHONALAND WEST & 95-99 & 90.24 & 114.00 & 71.03 & HT-Direct \\ 
  Zimbabwe & MASHONALAND WEST & 95-99 & 148.28 & 125.64 & 174.55 & RW2 \\ 
  Zimbabwe & MASHONALAND WEST & 00-04 & 93.71 & 113.10 & 77.35 & HT-Direct \\ 
  Zimbabwe & MASHONALAND WEST & 00-04 & 147.78 & 122.51 & 177.38 & RW2 \\ 
  Zimbabwe & MASHONALAND WEST & 05-09 & 124.64 & 154.78 & 99.68 & HT-Direct \\ 
  Zimbabwe & MASHONALAND WEST & 05-09 & 129.14 & 103.69 & 159.74 & RW2 \\ 
  Zimbabwe & MASHONALAND WEST & 10-14 & 110.83 & 37.94 & 289.14 & RW2 \\ 
  Zimbabwe & MASHONALAND WEST & 15-19 & 94.23 & 6.08 & 661.02 & RW2 \\ 
  Zimbabwe & MASVINGO & 80-84 & 95.18 & 119.74 & 75.23 & HT-Direct \\ 
  Zimbabwe & MASVINGO & 80-84 & 97.23 & 80.40 & 116.95 & RW2 \\ 
  Zimbabwe & MASVINGO & 85-89 & 67.72 & 82.28 & 55.57 & HT-Direct \\ 
  Zimbabwe & MASVINGO & 85-89 & 75.27 & 64.95 & 87.13 & RW2 \\ 
  Zimbabwe & MASVINGO & 90-94 & 69.50 & 85.02 & 56.64 & HT-Direct \\ 
  Zimbabwe & MASVINGO & 90-94 & 83.29 & 71.93 & 96.37 & RW2 \\ 
  Zimbabwe & MASVINGO & 95-99 & 80.97 & 107.94 & 60.29 & HT-Direct \\ 
  Zimbabwe & MASVINGO & 95-99 & 108.23 & 90.50 & 129.47 & RW2 \\ 
  Zimbabwe & MASVINGO & 00-04 & 87.49 & 116.94 & 64.91 & HT-Direct \\ 
  Zimbabwe & MASVINGO & 00-04 & 100.73 & 81.96 & 123.09 & RW2 \\ 
  Zimbabwe & MASVINGO & 05-09 & 67.26 & 80.95 & 55.74 & HT-Direct \\ 
  Zimbabwe & MASVINGO & 05-09 & 80.86 & 66.06 & 98.65 & RW2 \\ 
  Zimbabwe & MASVINGO & 10-14 & 63.58 & 21.33 & 180.15 & RW2 \\ 
  Zimbabwe & MASVINGO & 15-19 & 49.57 & 3.12 & 486.81 & RW2 \\ 
  Zimbabwe & MATABELELAND NORTH & 80-84 & 100.95 & 129.63 & 78.04 & HT-Direct \\ 
  Zimbabwe & MATABELELAND NORTH & 80-84 & 95.93 & 77.69 & 118.28 & RW2 \\ 
  Zimbabwe & MATABELELAND NORTH & 85-89 & 65.79 & 82.38 & 52.35 & HT-Direct \\ 
  Zimbabwe & MATABELELAND NORTH & 85-89 & 74.03 & 62.50 & 87.59 & RW2 \\ 
  Zimbabwe & MATABELELAND NORTH & 90-94 & 57.62 & 80.42 & 40.99 & HT-Direct \\ 
  Zimbabwe & MATABELELAND NORTH & 90-94 & 81.95 & 68.44 & 97.49 & RW2 \\ 
  Zimbabwe & MATABELELAND NORTH & 95-99 & 69.66 & 95.20 & 50.59 & HT-Direct \\ 
  Zimbabwe & MATABELELAND NORTH & 95-99 & 107.35 & 87.37 & 130.98 & RW2 \\ 
  Zimbabwe & MATABELELAND NORTH & 00-04 & 95.51 & 134.91 & 66.74 & HT-Direct \\ 
  Zimbabwe & MATABELELAND NORTH & 00-04 & 101.50 & 80.18 & 127.79 & RW2 \\ 
  Zimbabwe & MATABELELAND NORTH & 05-09 & 73.65 & 93.36 & 57.84 & HT-Direct \\ 
  Zimbabwe & MATABELELAND NORTH & 05-09 & 83.37 & 64.54 & 106.81 & RW2 \\ 
  Zimbabwe & MATABELELAND NORTH & 10-14 & 67.05 & 21.78 & 190.59 & RW2 \\ 
  Zimbabwe & MATABELELAND NORTH & 15-19 & 54.33 & 3.39 & 504.17 & RW2 \\ 
  Zimbabwe & MATABELELAND SOUTH & 80-84 & 65.10 & 81.82 & 51.60 & HT-Direct \\ 
  Zimbabwe & MATABELELAND SOUTH & 80-84 & 68.33 & 55.45 & 83.79 & RW2 \\ 
  Zimbabwe & MATABELELAND SOUTH & 85-89 & 49.28 & 67.95 & 35.55 & HT-Direct \\ 
  Zimbabwe & MATABELELAND SOUTH & 85-89 & 55.58 & 46.28 & 66.47 & RW2 \\ 
  Zimbabwe & MATABELELAND SOUTH & 90-94 & 60.62 & 79.08 & 46.26 & HT-Direct \\ 
  Zimbabwe & MATABELELAND SOUTH & 90-94 & 65.09 & 54.37 & 77.75 & RW2 \\ 
  Zimbabwe & MATABELELAND SOUTH & 95-99 & 70.33 & 96.97 & 50.61 & HT-Direct \\ 
  Zimbabwe & MATABELELAND SOUTH & 95-99 & 89.49 & 72.42 & 110.10 & RW2 \\ 
  Zimbabwe & MATABELELAND SOUTH & 00-04 & 39.95 & 57.61 & 27.54 & HT-Direct \\ 
  Zimbabwe & MATABELELAND SOUTH & 00-04 & 87.56 & 68.27 & 111.61 & RW2 \\ 
  Zimbabwe & MATABELELAND SOUTH & 05-09 & 66.41 & 87.27 & 50.27 & HT-Direct \\ 
  Zimbabwe & MATABELELAND SOUTH & 05-09 & 74.18 & 56.12 & 97.70 & RW2 \\ 
  Zimbabwe & MATABELELAND SOUTH & 10-14 & 61.83 & 20.22 & 179.94 & RW2 \\ 
  Zimbabwe & MATABELELAND SOUTH & 15-19 & 51.27 & 3.12 & 508.54 & RW2 \\ 
  Zimbabwe & MIDLANDS & 80-84 & 98.47 & 120.91 & 79.82 & HT-Direct \\ 
  Zimbabwe & MIDLANDS & 80-84 & 99.51 & 82.92 & 118.72 & RW2 \\ 
  Zimbabwe & MIDLANDS & 85-89 & 66.85 & 84.91 & 52.41 & HT-Direct \\ 
  Zimbabwe & MIDLANDS & 85-89 & 79.19 & 67.94 & 92.05 & RW2 \\ 
  Zimbabwe & MIDLANDS & 90-94 & 82.64 & 100.54 & 67.69 & HT-Direct \\ 
  Zimbabwe & MIDLANDS & 90-94 & 89.63 & 77.53 & 103.26 & RW2 \\ 
  Zimbabwe & MIDLANDS & 95-99 & 80.85 & 100.77 & 64.59 & HT-Direct \\ 
  Zimbabwe & MIDLANDS & 95-99 & 118.64 & 100.19 & 140.11 & RW2 \\ 
  Zimbabwe & MIDLANDS & 00-04 & 75.77 & 95.00 & 60.18 & HT-Direct \\ 
  Zimbabwe & MIDLANDS & 00-04 & 113.26 & 93.30 & 137.10 & RW2 \\ 
  Zimbabwe & MIDLANDS & 05-09 & 83.25 & 100.92 & 68.43 & HT-Direct \\ 
  Zimbabwe & MIDLANDS & 05-09 & 94.13 & 76.82 & 114.67 & RW2 \\ 
  Zimbabwe & MIDLANDS & 10-14 & 76.81 & 25.89 & 211.80 & RW2 \\ 
  Zimbabwe & MIDLANDS & 15-19 & 62.08 & 3.97 & 549.10 & RW2 \\ 
  \hline
\caption{Complete results.} 
\label{fulltable}
\end{longtable}

}

\clearpage
\section{Full table of the complete results: yearly estimates}
{\scriptsize
% latex table generated in R 3.4.3 by xtable 1.8-2 package
% Mon Apr  2 21:00:32 2018
\begin{longtable}{lllrrrl}
  \hline
Country & Region & Year & Median & Lower & Upper & Method \\ 
  \hline 
\endhead 
\hline 
{\footnotesize Continued on next page} 
\endfoot 
\endlastfoot 
Angola & ALL & 1980 & 192.05 & 166.46 & 218.22 & IHME \\ 
  Angola & ALL & 1980 & 267.19 & 161.54 & 414.02 & RW2 \\ 
  Angola & ALL & 1980 & 234.10 & 187.40 & 290.20 & UN \\ 
  Angola & ALL & 1981 & 190.62 & 166.49 & 216.75 & IHME \\ 
  Angola & ALL & 1981 & 258.88 & 173.81 & 369.41 & RW2 \\ 
  Angola & ALL & 1981 & 232.80 & 190.00 & 282.90 & UN \\ 
  Angola & ALL & 1982 & 190.06 & 166.87 & 215.42 & IHME \\ 
  Angola & ALL & 1982 & 250.84 & 179.92 & 340.18 & RW2 \\ 
  Angola & ALL & 1982 & 231.50 & 192.70 & 276.80 & UN \\ 
  Angola & ALL & 1983 & 188.67 & 166.89 & 214.52 & IHME \\ 
  Angola & ALL & 1983 & 242.50 & 178.35 & 323.38 & RW2 \\ 
  Angola & ALL & 1983 & 230.20 & 194.30 & 271.50 & UN \\ 
  Angola & ALL & 1984 & 186.39 & 165.28 & 208.88 & IHME \\ 
  Angola & ALL & 1984 & 235.80 & 173.17 & 313.87 & RW2 \\ 
  Angola & ALL & 1984 & 229.10 & 196.00 & 267.40 & UN \\ 
  Angola & ALL & 1985 & 184.66 & 163.76 & 206.17 & IHME \\ 
  Angola & ALL & 1985 & 228.29 & 169.87 & 296.79 & RW2 \\ 
  Angola & ALL & 1985 & 228.30 & 197.40 & 264.00 & UN \\ 
  Angola & ALL & 1986 & 185.06 & 167.54 & 205.99 & IHME \\ 
  Angola & ALL & 1986 & 223.80 & 168.72 & 287.44 & RW2 \\ 
  Angola & ALL & 1986 & 227.50 & 198.40 & 260.90 & UN \\ 
  Angola & ALL & 1987 & 187.09 & 169.97 & 207.37 & IHME \\ 
  Angola & ALL & 1987 & 221.17 & 168.73 & 282.04 & RW2 \\ 
  Angola & ALL & 1987 & 226.90 & 199.80 & 258.60 & UN \\ 
  Angola & ALL & 1988 & 189.57 & 171.66 & 208.66 & IHME \\ 
  Angola & ALL & 1988 & 220.03 & 166.98 & 280.41 & RW2 \\ 
  Angola & ALL & 1988 & 226.50 & 200.60 & 257.30 & UN \\ 
  Angola & ALL & 1989 & 190.63 & 172.40 & 210.19 & IHME \\ 
  Angola & ALL & 1989 & 220.42 & 166.82 & 281.28 & RW2 \\ 
  Angola & ALL & 1989 & 226.20 & 201.10 & 256.10 & UN \\ 
  Angola & ALL & 1990 & 190.46 & 172.22 & 210.74 & IHME \\ 
  Angola & ALL & 1990 & 222.95 & 172.25 & 285.58 & RW2 \\ 
  Angola & ALL & 1990 & 226.00 & 201.60 & 255.30 & UN \\ 
  Angola & ALL & 1991 & 187.49 & 169.26 & 207.69 & IHME \\ 
  Angola & ALL & 1991 & 224.71 & 175.71 & 283.55 & RW2 \\ 
  Angola & ALL & 1991 & 225.90 & 201.90 & 254.40 & UN \\ 
  Angola & ALL & 1992 & 183.58 & 165.23 & 202.16 & IHME \\ 
  Angola & ALL & 1992 & 226.33 & 177.82 & 284.13 & RW2 \\ 
  Angola & ALL & 1992 & 226.00 & 202.30 & 253.60 & UN \\ 
  Angola & ALL & 1993 & 180.97 & 163.00 & 198.26 & IHME \\ 
  Angola & ALL & 1993 & 227.46 & 178.37 & 286.60 & RW2 \\ 
  Angola & ALL & 1993 & 225.80 & 202.10 & 253.10 & UN \\ 
  Angola & ALL & 1994 & 173.34 & 157.40 & 190.41 & IHME \\ 
  Angola & ALL & 1994 & 227.72 & 177.26 & 290.12 & RW2 \\ 
  Angola & ALL & 1994 & 225.50 & 201.90 & 252.80 & UN \\ 
  Angola & ALL & 1995 & 166.87 & 151.71 & 184.23 & IHME \\ 
  Angola & ALL & 1995 & 227.10 & 177.26 & 285.88 & RW2 \\ 
  Angola & ALL & 1995 & 224.80 & 201.10 & 252.50 & UN \\ 
  Angola & ALL & 1996 & 162.48 & 147.40 & 179.06 & IHME \\ 
  Angola & ALL & 1996 & 226.08 & 178.44 & 282.72 & RW2 \\ 
  Angola & ALL & 1996 & 224.00 & 199.90 & 251.70 & UN \\ 
  Angola & ALL & 1997 & 160.06 & 145.68 & 176.16 & IHME \\ 
  Angola & ALL & 1997 & 224.62 & 178.48 & 278.69 & RW2 \\ 
  Angola & ALL & 1997 & 222.60 & 198.20 & 250.40 & UN \\ 
  Angola & ALL & 1998 & 159.19 & 143.73 & 174.19 & IHME \\ 
  Angola & ALL & 1998 & 222.95 & 175.85 & 279.68 & RW2 \\ 
  Angola & ALL & 1998 & 220.80 & 195.90 & 249.20 & UN \\ 
  Angola & ALL & 1999 & 158.04 & 143.43 & 174.54 & IHME \\ 
  Angola & ALL & 1999 & 220.85 & 171.50 & 278.76 & RW2 \\ 
  Angola & ALL & 1999 & 218.90 & 193.30 & 247.90 & UN \\ 
  Angola & ALL & 2000 & 155.77 & 141.10 & 172.43 & IHME \\ 
  Angola & ALL & 2000 & 218.43 & 168.38 & 277.79 & RW2 \\ 
  Angola & ALL & 2000 & 216.70 & 190.40 & 246.60 & UN \\ 
  Angola & ALL & 2001 & 152.29 & 137.14 & 169.36 & IHME \\ 
  Angola & ALL & 2001 & 215.76 & 167.13 & 273.55 & RW2 \\ 
  Angola & ALL & 2001 & 214.10 & 186.90 & 245.40 & UN \\ 
  Angola & ALL & 2002 & 145.79 & 131.78 & 161.52 & IHME \\ 
  Angola & ALL & 2002 & 212.74 & 165.51 & 269.31 & RW2 \\ 
  Angola & ALL & 2002 & 211.70 & 182.50 & 244.70 & UN \\ 
  Angola & ALL & 2003 & 138.24 & 124.62 & 153.25 & IHME \\ 
  Angola & ALL & 2003 & 209.68 & 160.49 & 269.22 & RW2 \\ 
  Angola & ALL & 2003 & 209.20 & 178.30 & 244.80 & UN \\ 
  Angola & ALL & 2004 & 130.47 & 117.15 & 144.56 & IHME \\ 
  Angola & ALL & 2004 & 205.83 & 150.29 & 275.03 & RW2 \\ 
  Angola & ALL & 2004 & 206.70 & 172.70 & 245.10 & UN \\ 
  Angola & ALL & 2005 & 123.26 & 109.99 & 137.53 & IHME \\ 
  Angola & ALL & 2005 & 202.32 & 136.46 & 289.66 & RW2 \\ 
  Angola & ALL & 2005 & 203.90 & 166.40 & 245.40 & UN \\ 
  Angola & ALL & 2006 & 117.63 & 104.59 & 131.06 & IHME \\ 
  Angola & ALL & 2006 & 198.22 & 127.65 & 294.62 & RW2 \\ 
  Angola & ALL & 2006 & 200.50 & 159.90 & 245.20 & UN \\ 
  Angola & ALL & 2007 & 113.34 & 100.96 & 127.82 & IHME \\ 
  Angola & ALL & 2007 & 194.18 & 122.07 & 294.94 & RW2 \\ 
  Angola & ALL & 2007 & 196.40 & 153.20 & 245.60 & UN \\ 
  Angola & ALL & 2008 & 109.20 & 97.08 & 123.16 & IHME \\ 
  Angola & ALL & 2008 & 190.60 & 119.57 & 290.51 & RW2 \\ 
  Angola & ALL & 2008 & 192.00 & 145.70 & 245.80 & UN \\ 
  Angola & ALL & 2009 & 105.19 & 92.21 & 119.21 & IHME \\ 
  Angola & ALL & 2009 & 186.12 & 119.92 & 279.04 & RW2 \\ 
  Angola & ALL & 2009 & 187.30 & 137.80 & 246.80 & UN \\ 
  Angola & ALL & 2010 & 100.82 & 86.82 & 117.10 & IHME \\ 
  Angola & ALL & 2010 & 181.94 & 125.91 & 257.36 & RW2 \\ 
  Angola & ALL & 2010 & 182.50 & 130.10 & 248.30 & UN \\ 
  Angola & ALL & 2011 & 96.34 & 79.62 & 113.74 & IHME \\ 
  Angola & ALL & 2011 & 177.99 & 131.12 & 237.04 & RW2 \\ 
  Angola & ALL & 2011 & 177.30 & 122.10 & 249.50 & UN \\ 
  Angola & ALL & 2012 & 91.75 & 72.10 & 111.10 & IHME \\ 
  Angola & ALL & 2012 & 173.89 & 133.54 & 223.05 & RW2 \\ 
  Angola & ALL & 2012 & 172.20 & 114.90 & 250.80 & UN \\ 
  Angola & ALL & 2013 & 87.22 & 66.86 & 108.42 & IHME \\ 
  Angola & ALL & 2013 & 170.00 & 124.33 & 227.55 & RW2 \\ 
  Angola & ALL & 2013 & 167.10 & 107.70 & 252.20 & UN \\ 
  Angola & ALL & 2014 & 82.88 & 64.29 & 105.43 & IHME \\ 
  Angola & ALL & 2014 & 165.97 & 104.06 & 253.64 & RW2 \\ 
  Angola & ALL & 2014 & 162.20 & 101.10 & 253.60 & UN \\ 
  Angola & ALL & 2015 & 78.70 & 60.29 & 100.90 & IHME \\ 
  Angola & ALL & 2015 & 161.83 & 80.27 & 299.23 & RW2 \\ 
  Angola & ALL & 2015 & 156.90 & 94.80 & 253.70 & UN \\ 
  Angola & ALL & 2016 & 158.45 & 61.59 & 352.80 & RW2 \\ 
  Angola & ALL & 2017 & 154.25 & 45.15 & 415.77 & RW2 \\ 
  Angola & ALL & 2018 & 150.51 & 32.82 & 490.34 & RW2 \\ 
  Angola & ALL & 2019 & 146.61 & 21.99 & 561.79 & RW2 \\ 
  Angola & ALL & 80-84 & 406.97 & 547.82 & 279.93 & HT-Direct \\ 
  Angola & ALL & 85-89 & 235.78 & 283.55 & 193.88 & HT-Direct \\ 
  Angola & ALL & 90-94 & 223.20 & 253.70 & 195.41 & HT-Direct \\ 
  Angola & ALL & 95-99 & 210.03 & 231.55 & 190.01 & HT-Direct \\ 
  Angola & ALL & 00-04 & 162.11 & 178.41 & 147.03 & HT-Direct \\ 
  Angola & ALL & 05-09 & 105.82 & 117.60 & 95.09 & HT-Direct \\ 
  Angola & ALL & 10-14 & 67.86 & 75.65 & 60.83 & HT-Direct \\ 
  Angola & ALL & 15-19 & 154.26 & 46.12 & 408.89 & RW2 \\ 
  Angola & BENGO & 1980 & 117.65 & 24.86 & 414.58 & RW2 \\ 
  Angola & BENGO & 1981 & 113.45 & 26.83 & 376.56 & RW2 \\ 
  Angola & BENGO & 1982 & 108.93 & 28.17 & 346.39 & RW2 \\ 
  Angola & BENGO & 1983 & 105.14 & 29.17 & 311.47 & RW2 \\ 
  Angola & BENGO & 1984 & 101.61 & 30.71 & 288.88 & RW2 \\ 
  Angola & BENGO & 1985 & 98.63 & 31.89 & 265.29 & RW2 \\ 
  Angola & BENGO & 1986 & 96.75 & 33.56 & 247.32 & RW2 \\ 
  Angola & BENGO & 1987 & 96.57 & 36.02 & 233.35 & RW2 \\ 
  Angola & BENGO & 1988 & 97.74 & 38.53 & 225.09 & RW2 \\ 
  Angola & BENGO & 1989 & 99.66 & 41.91 & 218.55 & RW2 \\ 
  Angola & BENGO & 1990 & 103.59 & 46.22 & 215.33 & RW2 \\ 
  Angola & BENGO & 1991 & 105.82 & 49.88 & 209.25 & RW2 \\ 
  Angola & BENGO & 1992 & 107.82 & 53.83 & 204.15 & RW2 \\ 
  Angola & BENGO & 1993 & 109.05 & 56.69 & 198.08 & RW2 \\ 
  Angola & BENGO & 1994 & 109.47 & 59.17 & 193.26 & RW2 \\ 
  Angola & BENGO & 1995 & 108.50 & 60.33 & 185.24 & RW2 \\ 
  Angola & BENGO & 1996 & 108.75 & 63.02 & 180.31 & RW2 \\ 
  Angola & BENGO & 1997 & 109.66 & 65.22 & 177.03 & RW2 \\ 
  Angola & BENGO & 1998 & 111.22 & 67.41 & 176.78 & RW2 \\ 
  Angola & BENGO & 1999 & 113.70 & 69.35 & 180.07 & RW2 \\ 
  Angola & BENGO & 2000 & 117.07 & 71.99 & 185.02 & RW2 \\ 
  Angola & BENGO & 2001 & 119.74 & 74.05 & 188.35 & RW2 \\ 
  Angola & BENGO & 2002 & 121.86 & 74.75 & 192.11 & RW2 \\ 
  Angola & BENGO & 2003 & 123.45 & 73.92 & 198.22 & RW2 \\ 
  Angola & BENGO & 2004 & 123.77 & 71.65 & 207.08 & RW2 \\ 
  Angola & BENGO & 2005 & 123.09 & 66.24 & 216.73 & RW2 \\ 
  Angola & BENGO & 2006 & 121.95 & 62.10 & 223.87 & RW2 \\ 
  Angola & BENGO & 2007 & 120.76 & 58.80 & 229.46 & RW2 \\ 
  Angola & BENGO & 2008 & 118.81 & 56.14 & 232.94 & RW2 \\ 
  Angola & BENGO & 2009 & 116.59 & 53.38 & 234.12 & RW2 \\ 
  Angola & BENGO & 2010 & 114.67 & 52.09 & 233.14 & RW2 \\ 
  Angola & BENGO & 2011 & 112.48 & 50.00 & 233.39 & RW2 \\ 
  Angola & BENGO & 2012 & 109.92 & 47.05 & 236.63 & RW2 \\ 
  Angola & BENGO & 2013 & 107.87 & 42.79 & 245.12 & RW2 \\ 
  Angola & BENGO & 2014 & 105.98 & 37.30 & 262.87 & RW2 \\ 
  Angola & BENGO & 2015 & 103.01 & 31.03 & 293.42 & RW2 \\ 
  Angola & BENGO & 2016 & 101.27 & 24.91 & 331.19 & RW2 \\ 
  Angola & BENGO & 2017 & 99.53 & 19.68 & 379.67 & RW2 \\ 
  Angola & BENGO & 2018 & 96.58 & 14.34 & 436.41 & RW2 \\ 
  Angola & BENGO & 2019 & 94.92 & 10.71 & 510.07 & RW2 \\ 
  Angola & BENGUELA & 1980 & 393.18 & 260.51 & 542.13 & RW2 \\ 
  Angola & BENGUELA & 1981 & 387.58 & 277.18 & 506.85 & RW2 \\ 
  Angola & BENGUELA & 1982 & 381.10 & 285.08 & 485.87 & RW2 \\ 
  Angola & BENGUELA & 1983 & 375.53 & 284.66 & 474.54 & RW2 \\ 
  Angola & BENGUELA & 1984 & 370.74 & 282.33 & 469.15 & RW2 \\ 
  Angola & BENGUELA & 1985 & 366.96 & 279.96 & 457.98 & RW2 \\ 
  Angola & BENGUELA & 1986 & 366.13 & 283.66 & 453.38 & RW2 \\ 
  Angola & BENGUELA & 1987 & 368.73 & 289.42 & 451.78 & RW2 \\ 
  Angola & BENGUELA & 1988 & 374.17 & 296.00 & 455.72 & RW2 \\ 
  Angola & BENGUELA & 1989 & 381.27 & 303.53 & 462.02 & RW2 \\ 
  Angola & BENGUELA & 1990 & 391.28 & 319.01 & 470.46 & RW2 \\ 
  Angola & BENGUELA & 1991 & 398.03 & 328.86 & 473.85 & RW2 \\ 
  Angola & BENGUELA & 1992 & 402.24 & 334.86 & 474.48 & RW2 \\ 
  Angola & BENGUELA & 1993 & 403.65 & 336.49 & 477.83 & RW2 \\ 
  Angola & BENGUELA & 1994 & 401.86 & 333.26 & 478.89 & RW2 \\ 
  Angola & BENGUELA & 1995 & 396.90 & 327.15 & 470.28 & RW2 \\ 
  Angola & BENGUELA & 1996 & 393.30 & 326.38 & 463.76 & RW2 \\ 
  Angola & BENGUELA & 1997 & 390.71 & 325.85 & 460.66 & RW2 \\ 
  Angola & BENGUELA & 1998 & 389.44 & 322.87 & 462.35 & RW2 \\ 
  Angola & BENGUELA & 1999 & 389.23 & 318.57 & 465.83 & RW2 \\ 
  Angola & BENGUELA & 2000 & 391.45 & 318.62 & 471.98 & RW2 \\ 
  Angola & BENGUELA & 2001 & 391.47 & 319.78 & 471.56 & RW2 \\ 
  Angola & BENGUELA & 2002 & 390.37 & 317.36 & 470.68 & RW2 \\ 
  Angola & BENGUELA & 2003 & 387.45 & 309.95 & 472.57 & RW2 \\ 
  Angola & BENGUELA & 2004 & 382.05 & 295.74 & 478.05 & RW2 \\ 
  Angola & BENGUELA & 2005 & 374.56 & 274.37 & 486.47 & RW2 \\ 
  Angola & BENGUELA & 2006 & 366.73 & 258.36 & 488.41 & RW2 \\ 
  Angola & BENGUELA & 2007 & 356.81 & 245.10 & 484.93 & RW2 \\ 
  Angola & BENGUELA & 2008 & 347.59 & 236.68 & 475.32 & RW2 \\ 
  Angola & BENGUELA & 2009 & 337.53 & 232.73 & 459.52 & RW2 \\ 
  Angola & BENGUELA & 2010 & 326.83 & 233.77 & 436.14 & RW2 \\ 
  Angola & BENGUELA & 2011 & 316.68 & 233.62 & 412.06 & RW2 \\ 
  Angola & BENGUELA & 2012 & 306.31 & 229.34 & 394.07 & RW2 \\ 
  Angola & BENGUELA & 2013 & 295.78 & 212.29 & 394.02 & RW2 \\ 
  Angola & BENGUELA & 2014 & 285.97 & 183.15 & 416.63 & RW2 \\ 
  Angola & BENGUELA & 2015 & 275.54 & 146.56 & 457.45 & RW2 \\ 
  Angola & BENGUELA & 2016 & 265.59 & 115.08 & 503.97 & RW2 \\ 
  Angola & BENGUELA & 2017 & 255.94 & 86.15 & 563.33 & RW2 \\ 
  Angola & BENGUELA & 2018 & 246.45 & 62.92 & 618.42 & RW2 \\ 
  Angola & BENGUELA & 2019 & 237.81 & 44.29 & 675.96 & RW2 \\ 
  Angola & BIÉ & 1980 & 225.17 & 129.27 & 375.26 & RW2 \\ 
  Angola & BIÉ & 1981 & 216.26 & 135.33 & 335.51 & RW2 \\ 
  Angola & BIÉ & 1982 & 206.94 & 134.90 & 310.02 & RW2 \\ 
  Angola & BIÉ & 1983 & 198.51 & 131.24 & 293.14 & RW2 \\ 
  Angola & BIÉ & 1984 & 191.08 & 126.95 & 279.75 & RW2 \\ 
  Angola & BIÉ & 1985 & 184.03 & 123.50 & 264.91 & RW2 \\ 
  Angola & BIÉ & 1986 & 179.91 & 122.40 & 254.26 & RW2 \\ 
  Angola & BIÉ & 1987 & 177.77 & 123.15 & 248.64 & RW2 \\ 
  Angola & BIÉ & 1988 & 177.62 & 124.29 & 244.90 & RW2 \\ 
  Angola & BIÉ & 1989 & 179.04 & 127.23 & 244.06 & RW2 \\ 
  Angola & BIÉ & 1990 & 183.24 & 133.14 & 247.49 & RW2 \\ 
  Angola & BIÉ & 1991 & 185.32 & 137.27 & 245.77 & RW2 \\ 
  Angola & BIÉ & 1992 & 186.25 & 139.73 & 244.48 & RW2 \\ 
  Angola & BIÉ & 1993 & 185.78 & 140.72 & 242.56 & RW2 \\ 
  Angola & BIÉ & 1994 & 184.14 & 140.34 & 240.16 & RW2 \\ 
  Angola & BIÉ & 1995 & 180.76 & 137.30 & 231.44 & RW2 \\ 
  Angola & BIÉ & 1996 & 178.90 & 138.96 & 226.32 & RW2 \\ 
  Angola & BIÉ & 1997 & 178.28 & 139.42 & 223.16 & RW2 \\ 
  Angola & BIÉ & 1998 & 178.64 & 139.04 & 224.51 & RW2 \\ 
  Angola & BIÉ & 1999 & 180.52 & 139.05 & 229.67 & RW2 \\ 
  Angola & BIÉ & 2000 & 184.17 & 140.58 & 237.15 & RW2 \\ 
  Angola & BIÉ & 2001 & 187.07 & 143.70 & 240.35 & RW2 \\ 
  Angola & BIÉ & 2002 & 189.40 & 145.53 & 243.15 & RW2 \\ 
  Angola & BIÉ & 2003 & 190.92 & 144.16 & 248.95 & RW2 \\ 
  Angola & BIÉ & 2004 & 191.43 & 138.79 & 258.34 & RW2 \\ 
  Angola & BIÉ & 2005 & 190.34 & 129.72 & 270.33 & RW2 \\ 
  Angola & BIÉ & 2006 & 189.33 & 123.84 & 278.01 & RW2 \\ 
  Angola & BIÉ & 2007 & 187.67 & 119.50 & 282.45 & RW2 \\ 
  Angola & BIÉ & 2008 & 185.73 & 117.54 & 280.43 & RW2 \\ 
  Angola & BIÉ & 2009 & 183.37 & 117.05 & 275.36 & RW2 \\ 
  Angola & BIÉ & 2010 & 181.22 & 120.14 & 263.14 & RW2 \\ 
  Angola & BIÉ & 2011 & 178.85 & 123.28 & 252.25 & RW2 \\ 
  Angola & BIÉ & 2012 & 176.23 & 123.02 & 247.32 & RW2 \\ 
  Angola & BIÉ & 2013 & 173.59 & 116.27 & 253.49 & RW2 \\ 
  Angola & BIÉ & 2014 & 171.01 & 102.20 & 275.23 & RW2 \\ 
  Angola & BIÉ & 2015 & 168.68 & 82.57 & 315.63 & RW2 \\ 
  Angola & BIÉ & 2016 & 166.65 & 66.12 & 363.76 & RW2 \\ 
  Angola & BIÉ & 2017 & 163.59 & 50.55 & 425.82 & RW2 \\ 
  Angola & BIÉ & 2018 & 161.45 & 37.90 & 490.22 & RW2 \\ 
  Angola & BIÉ & 2019 & 159.31 & 27.75 & 568.46 & RW2 \\ 
  Angola & CABINDA & 1980 & 107.50 & 28.37 & 351.19 & RW2 \\ 
  Angola & CABINDA & 1981 & 103.08 & 30.65 & 312.78 & RW2 \\ 
  Angola & CABINDA & 1982 & 99.12 & 31.62 & 284.14 & RW2 \\ 
  Angola & CABINDA & 1983 & 95.03 & 32.71 & 258.42 & RW2 \\ 
  Angola & CABINDA & 1984 & 91.94 & 33.35 & 235.58 & RW2 \\ 
  Angola & CABINDA & 1985 & 88.51 & 34.37 & 214.93 & RW2 \\ 
  Angola & CABINDA & 1986 & 87.29 & 36.06 & 200.09 & RW2 \\ 
  Angola & CABINDA & 1987 & 86.67 & 37.78 & 187.16 & RW2 \\ 
  Angola & CABINDA & 1988 & 87.26 & 40.17 & 178.66 & RW2 \\ 
  Angola & CABINDA & 1989 & 88.29 & 42.75 & 172.45 & RW2 \\ 
  Angola & CABINDA & 1990 & 91.07 & 46.68 & 170.26 & RW2 \\ 
  Angola & CABINDA & 1991 & 92.53 & 49.93 & 165.76 & RW2 \\ 
  Angola & CABINDA & 1992 & 93.37 & 52.36 & 159.95 & RW2 \\ 
  Angola & CABINDA & 1993 & 93.26 & 53.92 & 156.61 & RW2 \\ 
  Angola & CABINDA & 1994 & 92.54 & 54.90 & 152.47 & RW2 \\ 
  Angola & CABINDA & 1995 & 90.41 & 54.66 & 144.99 & RW2 \\ 
  Angola & CABINDA & 1996 & 89.60 & 55.80 & 140.24 & RW2 \\ 
  Angola & CABINDA & 1997 & 89.23 & 56.29 & 136.80 & RW2 \\ 
  Angola & CABINDA & 1998 & 89.44 & 56.59 & 136.85 & RW2 \\ 
  Angola & CABINDA & 1999 & 90.35 & 57.33 & 137.99 & RW2 \\ 
  Angola & CABINDA & 2000 & 92.32 & 58.42 & 142.17 & RW2 \\ 
  Angola & CABINDA & 2001 & 93.81 & 59.57 & 143.37 & RW2 \\ 
  Angola & CABINDA & 2002 & 94.92 & 60.25 & 145.95 & RW2 \\ 
  Angola & CABINDA & 2003 & 95.58 & 59.55 & 148.59 & RW2 \\ 
  Angola & CABINDA & 2004 & 95.84 & 57.47 & 154.50 & RW2 \\ 
  Angola & CABINDA & 2005 & 95.12 & 53.83 & 160.73 & RW2 \\ 
  Angola & CABINDA & 2006 & 94.30 & 51.12 & 166.23 & RW2 \\ 
  Angola & CABINDA & 2007 & 92.97 & 48.76 & 168.95 & RW2 \\ 
  Angola & CABINDA & 2008 & 92.04 & 47.08 & 170.90 & RW2 \\ 
  Angola & CABINDA & 2009 & 90.49 & 46.13 & 168.65 & RW2 \\ 
  Angola & CABINDA & 2010 & 89.16 & 45.84 & 165.53 & RW2 \\ 
  Angola & CABINDA & 2011 & 87.80 & 45.07 & 163.98 & RW2 \\ 
  Angola & CABINDA & 2012 & 86.29 & 43.64 & 163.65 & RW2 \\ 
  Angola & CABINDA & 2013 & 85.06 & 40.59 & 170.48 & RW2 \\ 
  Angola & CABINDA & 2014 & 83.79 & 35.70 & 185.56 & RW2 \\ 
  Angola & CABINDA & 2015 & 82.54 & 29.34 & 209.66 & RW2 \\ 
  Angola & CABINDA & 2016 & 80.85 & 23.47 & 245.29 & RW2 \\ 
  Angola & CABINDA & 2017 & 79.52 & 18.30 & 287.91 & RW2 \\ 
  Angola & CABINDA & 2018 & 77.93 & 13.78 & 345.11 & RW2 \\ 
  Angola & CABINDA & 2019 & 77.31 & 10.05 & 406.32 & RW2 \\ 
  Angola & CUANDO CUBANGO & 1980 & 120.89 & 28.77 & 397.30 & RW2 \\ 
  Angola & CUANDO CUBANGO & 1981 & 119.20 & 31.58 & 363.43 & RW2 \\ 
  Angola & CUANDO CUBANGO & 1982 & 116.72 & 33.42 & 338.95 & RW2 \\ 
  Angola & CUANDO CUBANGO & 1983 & 115.08 & 35.04 & 319.43 & RW2 \\ 
  Angola & CUANDO CUBANGO & 1984 & 112.96 & 36.94 & 299.00 & RW2 \\ 
  Angola & CUANDO CUBANGO & 1985 & 111.32 & 38.88 & 280.22 & RW2 \\ 
  Angola & CUANDO CUBANGO & 1986 & 112.23 & 41.49 & 268.75 & RW2 \\ 
  Angola & CUANDO CUBANGO & 1987 & 113.94 & 45.21 & 258.08 & RW2 \\ 
  Angola & CUANDO CUBANGO & 1988 & 116.50 & 49.03 & 251.08 & RW2 \\ 
  Angola & CUANDO CUBANGO & 1989 & 120.87 & 53.87 & 248.13 & RW2 \\ 
  Angola & CUANDO CUBANGO & 1990 & 127.07 & 60.59 & 248.47 & RW2 \\ 
  Angola & CUANDO CUBANGO & 1991 & 131.78 & 66.00 & 245.98 & RW2 \\ 
  Angola & CUANDO CUBANGO & 1992 & 135.65 & 71.15 & 240.01 & RW2 \\ 
  Angola & CUANDO CUBANGO & 1993 & 137.95 & 75.72 & 237.47 & RW2 \\ 
  Angola & CUANDO CUBANGO & 1994 & 139.68 & 79.78 & 233.15 & RW2 \\ 
  Angola & CUANDO CUBANGO & 1995 & 139.09 & 81.29 & 225.27 & RW2 \\ 
  Angola & CUANDO CUBANGO & 1996 & 140.36 & 85.33 & 220.99 & RW2 \\ 
  Angola & CUANDO CUBANGO & 1997 & 141.85 & 88.49 & 217.39 & RW2 \\ 
  Angola & CUANDO CUBANGO & 1998 & 144.67 & 92.26 & 218.91 & RW2 \\ 
  Angola & CUANDO CUBANGO & 1999 & 148.05 & 95.30 & 221.89 & RW2 \\ 
  Angola & CUANDO CUBANGO & 2000 & 153.34 & 100.01 & 228.33 & RW2 \\ 
  Angola & CUANDO CUBANGO & 2001 & 157.41 & 103.73 & 231.98 & RW2 \\ 
  Angola & CUANDO CUBANGO & 2002 & 161.15 & 106.89 & 234.88 & RW2 \\ 
  Angola & CUANDO CUBANGO & 2003 & 164.17 & 107.67 & 242.32 & RW2 \\ 
  Angola & CUANDO CUBANGO & 2004 & 165.92 & 106.45 & 250.38 & RW2 \\ 
  Angola & CUANDO CUBANGO & 2005 & 166.57 & 101.47 & 261.63 & RW2 \\ 
  Angola & CUANDO CUBANGO & 2006 & 166.81 & 97.47 & 269.71 & RW2 \\ 
  Angola & CUANDO CUBANGO & 2007 & 166.54 & 95.12 & 275.12 & RW2 \\ 
  Angola & CUANDO CUBANGO & 2008 & 166.04 & 93.13 & 276.13 & RW2 \\ 
  Angola & CUANDO CUBANGO & 2009 & 165.02 & 92.97 & 274.21 & RW2 \\ 
  Angola & CUANDO CUBANGO & 2010 & 164.23 & 93.96 & 270.54 & RW2 \\ 
  Angola & CUANDO CUBANGO & 2011 & 162.76 & 94.00 & 265.77 & RW2 \\ 
  Angola & CUANDO CUBANGO & 2012 & 161.19 & 92.80 & 265.31 & RW2 \\ 
  Angola & CUANDO CUBANGO & 2013 & 160.00 & 88.00 & 274.55 & RW2 \\ 
  Angola & CUANDO CUBANGO & 2014 & 159.12 & 77.96 & 298.36 & RW2 \\ 
  Angola & CUANDO CUBANGO & 2015 & 157.65 & 65.15 & 335.15 & RW2 \\ 
  Angola & CUANDO CUBANGO & 2016 & 156.29 & 52.41 & 384.14 & RW2 \\ 
  Angola & CUANDO CUBANGO & 2017 & 154.74 & 41.13 & 440.17 & RW2 \\ 
  Angola & CUANDO CUBANGO & 2018 & 153.86 & 31.27 & 508.11 & RW2 \\ 
  Angola & CUANDO CUBANGO & 2019 & 153.11 & 22.97 & 578.46 & RW2 \\ 
  Angola & CUANZA NORTE & 1980 & 202.27 & 68.63 & 457.30 & RW2 \\ 
  Angola & CUANZA NORTE & 1981 & 197.15 & 73.36 & 423.99 & RW2 \\ 
  Angola & CUANZA NORTE & 1982 & 191.53 & 76.78 & 392.50 & RW2 \\ 
  Angola & CUANZA NORTE & 1983 & 186.64 & 79.90 & 372.25 & RW2 \\ 
  Angola & CUANZA NORTE & 1984 & 182.64 & 82.29 & 353.24 & RW2 \\ 
  Angola & CUANZA NORTE & 1985 & 177.77 & 84.16 & 330.48 & RW2 \\ 
  Angola & CUANZA NORTE & 1986 & 176.63 & 88.64 & 318.16 & RW2 \\ 
  Angola & CUANZA NORTE & 1987 & 177.16 & 93.59 & 308.29 & RW2 \\ 
  Angola & CUANZA NORTE & 1988 & 179.94 & 99.07 & 301.12 & RW2 \\ 
  Angola & CUANZA NORTE & 1989 & 184.38 & 106.54 & 299.12 & RW2 \\ 
  Angola & CUANZA NORTE & 1990 & 191.06 & 116.08 & 298.98 & RW2 \\ 
  Angola & CUANZA NORTE & 1991 & 196.20 & 123.93 & 296.54 & RW2 \\ 
  Angola & CUANZA NORTE & 1992 & 199.75 & 130.64 & 294.30 & RW2 \\ 
  Angola & CUANZA NORTE & 1993 & 202.20 & 134.79 & 292.89 & RW2 \\ 
  Angola & CUANZA NORTE & 1994 & 203.10 & 137.63 & 289.78 & RW2 \\ 
  Angola & CUANZA NORTE & 1995 & 201.73 & 138.19 & 283.07 & RW2 \\ 
  Angola & CUANZA NORTE & 1996 & 201.85 & 141.22 & 280.22 & RW2 \\ 
  Angola & CUANZA NORTE & 1997 & 202.82 & 142.84 & 278.40 & RW2 \\ 
  Angola & CUANZA NORTE & 1998 & 204.87 & 145.86 & 281.29 & RW2 \\ 
  Angola & CUANZA NORTE & 1999 & 208.00 & 147.09 & 286.49 & RW2 \\ 
  Angola & CUANZA NORTE & 2000 & 213.07 & 151.18 & 295.78 & RW2 \\ 
  Angola & CUANZA NORTE & 2001 & 216.65 & 154.03 & 300.14 & RW2 \\ 
  Angola & CUANZA NORTE & 2002 & 219.29 & 155.85 & 302.91 & RW2 \\ 
  Angola & CUANZA NORTE & 2003 & 220.44 & 153.55 & 309.38 & RW2 \\ 
  Angola & CUANZA NORTE & 2004 & 220.71 & 148.18 & 316.66 & RW2 \\ 
  Angola & CUANZA NORTE & 2005 & 218.69 & 138.07 & 327.84 & RW2 \\ 
  Angola & CUANZA NORTE & 2006 & 216.44 & 131.39 & 333.88 & RW2 \\ 
  Angola & CUANZA NORTE & 2007 & 213.41 & 125.92 & 338.05 & RW2 \\ 
  Angola & CUANZA NORTE & 2008 & 209.57 & 121.39 & 337.16 & RW2 \\ 
  Angola & CUANZA NORTE & 2009 & 205.88 & 119.13 & 331.43 & RW2 \\ 
  Angola & CUANZA NORTE & 2010 & 201.19 & 117.64 & 321.52 & RW2 \\ 
  Angola & CUANZA NORTE & 2011 & 196.90 & 116.49 & 312.22 & RW2 \\ 
  Angola & CUANZA NORTE & 2012 & 192.59 & 112.56 & 310.06 & RW2 \\ 
  Angola & CUANZA NORTE & 2013 & 187.86 & 104.08 & 315.78 & RW2 \\ 
  Angola & CUANZA NORTE & 2014 & 183.56 & 91.48 & 335.36 & RW2 \\ 
  Angola & CUANZA NORTE & 2015 & 179.50 & 75.59 & 369.06 & RW2 \\ 
  Angola & CUANZA NORTE & 2016 & 175.60 & 60.53 & 417.14 & RW2 \\ 
  Angola & CUANZA NORTE & 2017 & 170.70 & 46.33 & 470.95 & RW2 \\ 
  Angola & CUANZA NORTE & 2018 & 167.00 & 34.42 & 535.18 & RW2 \\ 
  Angola & CUANZA NORTE & 2019 & 163.42 & 25.01 & 601.65 & RW2 \\ 
  Angola & CUANZA SUL & 1980 & 231.39 & 120.67 & 403.55 & RW2 \\ 
  Angola & CUANZA SUL & 1981 & 226.85 & 129.34 & 371.02 & RW2 \\ 
  Angola & CUANZA SUL & 1982 & 223.06 & 135.27 & 347.06 & RW2 \\ 
  Angola & CUANZA SUL & 1983 & 218.99 & 137.17 & 331.57 & RW2 \\ 
  Angola & CUANZA SUL & 1984 & 215.52 & 139.51 & 318.89 & RW2 \\ 
  Angola & CUANZA SUL & 1985 & 212.46 & 141.85 & 305.57 & RW2 \\ 
  Angola & CUANZA SUL & 1986 & 212.52 & 145.10 & 298.74 & RW2 \\ 
  Angola & CUANZA SUL & 1987 & 214.90 & 151.35 & 294.82 & RW2 \\ 
  Angola & CUANZA SUL & 1988 & 219.40 & 156.97 & 295.97 & RW2 \\ 
  Angola & CUANZA SUL & 1989 & 225.52 & 163.08 & 300.84 & RW2 \\ 
  Angola & CUANZA SUL & 1990 & 234.70 & 174.50 & 310.37 & RW2 \\ 
  Angola & CUANZA SUL & 1991 & 241.23 & 182.18 & 313.06 & RW2 \\ 
  Angola & CUANZA SUL & 1992 & 246.37 & 189.19 & 315.60 & RW2 \\ 
  Angola & CUANZA SUL & 1993 & 249.76 & 192.43 & 319.88 & RW2 \\ 
  Angola & CUANZA SUL & 1994 & 251.21 & 193.58 & 321.78 & RW2 \\ 
  Angola & CUANZA SUL & 1995 & 249.90 & 192.93 & 316.02 & RW2 \\ 
  Angola & CUANZA SUL & 1996 & 250.48 & 196.09 & 313.78 & RW2 \\ 
  Angola & CUANZA SUL & 1997 & 252.23 & 198.35 & 314.28 & RW2 \\ 
  Angola & CUANZA SUL & 1998 & 255.56 & 199.43 & 320.32 & RW2 \\ 
  Angola & CUANZA SUL & 1999 & 259.72 & 200.62 & 329.24 & RW2 \\ 
  Angola & CUANZA SUL & 2000 & 266.36 & 204.26 & 340.89 & RW2 \\ 
  Angola & CUANZA SUL & 2001 & 271.72 & 208.61 & 346.79 & RW2 \\ 
  Angola & CUANZA SUL & 2002 & 276.25 & 210.87 & 353.44 & RW2 \\ 
  Angola & CUANZA SUL & 2003 & 279.41 & 209.10 & 362.25 & RW2 \\ 
  Angola & CUANZA SUL & 2004 & 280.92 & 203.35 & 373.91 & RW2 \\ 
  Angola & CUANZA SUL & 2005 & 280.23 & 190.66 & 389.41 & RW2 \\ 
  Angola & CUANZA SUL & 2006 & 279.79 & 183.87 & 401.49 & RW2 \\ 
  Angola & CUANZA SUL & 2007 & 277.71 & 177.66 & 405.02 & RW2 \\ 
  Angola & CUANZA SUL & 2008 & 275.69 & 174.61 & 405.04 & RW2 \\ 
  Angola & CUANZA SUL & 2009 & 273.41 & 174.38 & 399.45 & RW2 \\ 
  Angola & CUANZA SUL & 2010 & 270.97 & 178.38 & 386.47 & RW2 \\ 
  Angola & CUANZA SUL & 2011 & 267.55 & 181.04 & 375.60 & RW2 \\ 
  Angola & CUANZA SUL & 2012 & 264.86 & 180.85 & 371.32 & RW2 \\ 
  Angola & CUANZA SUL & 2013 & 261.55 & 170.48 & 377.47 & RW2 \\ 
  Angola & CUANZA SUL & 2014 & 258.19 & 152.54 & 403.48 & RW2 \\ 
  Angola & CUANZA SUL & 2015 & 255.22 & 125.85 & 446.57 & RW2 \\ 
  Angola & CUANZA SUL & 2016 & 252.20 & 102.14 & 500.15 & RW2 \\ 
  Angola & CUANZA SUL & 2017 & 248.89 & 79.32 & 561.66 & RW2 \\ 
  Angola & CUANZA SUL & 2018 & 245.81 & 59.38 & 622.16 & RW2 \\ 
  Angola & CUANZA SUL & 2019 & 242.63 & 44.00 & 687.84 & RW2 \\ 
  Angola & CUNENE & 1980 & 115.71 & 36.65 & 293.92 & RW2 \\ 
  Angola & CUNENE & 1981 & 115.15 & 41.49 & 268.84 & RW2 \\ 
  Angola & CUNENE & 1982 & 114.61 & 44.84 & 248.19 & RW2 \\ 
  Angola & CUNENE & 1983 & 113.93 & 48.23 & 235.04 & RW2 \\ 
  Angola & CUNENE & 1984 & 113.90 & 51.88 & 224.89 & RW2 \\ 
  Angola & CUNENE & 1985 & 113.54 & 54.92 & 212.56 & RW2 \\ 
  Angola & CUNENE & 1986 & 115.35 & 59.85 & 206.09 & RW2 \\ 
  Angola & CUNENE & 1987 & 118.66 & 65.60 & 201.70 & RW2 \\ 
  Angola & CUNENE & 1988 & 123.25 & 71.93 & 199.17 & RW2 \\ 
  Angola & CUNENE & 1989 & 129.04 & 79.55 & 200.43 & RW2 \\ 
  Angola & CUNENE & 1990 & 137.04 & 89.44 & 203.36 & RW2 \\ 
  Angola & CUNENE & 1991 & 143.65 & 98.37 & 205.22 & RW2 \\ 
  Angola & CUNENE & 1992 & 148.84 & 105.75 & 205.06 & RW2 \\ 
  Angola & CUNENE & 1993 & 153.00 & 111.30 & 207.64 & RW2 \\ 
  Angola & CUNENE & 1994 & 156.02 & 114.88 & 209.60 & RW2 \\ 
  Angola & CUNENE & 1995 & 156.59 & 115.72 & 206.93 & RW2 \\ 
  Angola & CUNENE & 1996 & 158.84 & 119.65 & 206.38 & RW2 \\ 
  Angola & CUNENE & 1997 & 161.24 & 122.44 & 208.92 & RW2 \\ 
  Angola & CUNENE & 1998 & 164.65 & 124.22 & 214.51 & RW2 \\ 
  Angola & CUNENE & 1999 & 169.37 & 125.85 & 223.94 & RW2 \\ 
  Angola & CUNENE & 2000 & 175.40 & 129.29 & 235.24 & RW2 \\ 
  Angola & CUNENE & 2001 & 180.26 & 132.88 & 242.16 & RW2 \\ 
  Angola & CUNENE & 2002 & 184.64 & 133.81 & 250.10 & RW2 \\ 
  Angola & CUNENE & 2003 & 187.75 & 132.71 & 260.41 & RW2 \\ 
  Angola & CUNENE & 2004 & 189.80 & 128.46 & 271.81 & RW2 \\ 
  Angola & CUNENE & 2005 & 189.65 & 120.42 & 285.67 & RW2 \\ 
  Angola & CUNENE & 2006 & 189.83 & 115.25 & 296.75 & RW2 \\ 
  Angola & CUNENE & 2007 & 188.49 & 110.52 & 300.54 & RW2 \\ 
  Angola & CUNENE & 2008 & 186.75 & 107.61 & 304.09 & RW2 \\ 
  Angola & CUNENE & 2009 & 185.42 & 106.44 & 302.07 & RW2 \\ 
  Angola & CUNENE & 2010 & 183.39 & 107.29 & 295.39 & RW2 \\ 
  Angola & CUNENE & 2011 & 181.38 & 106.95 & 289.63 & RW2 \\ 
  Angola & CUNENE & 2012 & 178.89 & 104.30 & 289.60 & RW2 \\ 
  Angola & CUNENE & 2013 & 176.72 & 97.78 & 298.22 & RW2 \\ 
  Angola & CUNENE & 2014 & 174.51 & 86.28 & 319.32 & RW2 \\ 
  Angola & CUNENE & 2015 & 172.22 & 71.71 & 359.63 & RW2 \\ 
  Angola & CUNENE & 2016 & 169.54 & 57.93 & 407.44 & RW2 \\ 
  Angola & CUNENE & 2017 & 167.83 & 44.43 & 465.45 & RW2 \\ 
  Angola & CUNENE & 2018 & 165.07 & 33.30 & 528.74 & RW2 \\ 
  Angola & CUNENE & 2019 & 162.65 & 24.71 & 600.64 & RW2 \\ 
  Angola & HUAMBO & 1980 & 318.62 & 198.99 & 468.20 & RW2 \\ 
  Angola & HUAMBO & 1981 & 309.85 & 208.59 & 430.39 & RW2 \\ 
  Angola & HUAMBO & 1982 & 301.04 & 213.03 & 406.46 & RW2 \\ 
  Angola & HUAMBO & 1983 & 292.74 & 210.22 & 391.16 & RW2 \\ 
  Angola & HUAMBO & 1984 & 285.20 & 205.47 & 381.36 & RW2 \\ 
  Angola & HUAMBO & 1985 & 278.29 & 201.71 & 367.19 & RW2 \\ 
  Angola & HUAMBO & 1986 & 275.09 & 202.56 & 358.39 & RW2 \\ 
  Angola & HUAMBO & 1987 & 274.41 & 204.15 & 354.12 & RW2 \\ 
  Angola & HUAMBO & 1988 & 275.98 & 207.10 & 353.77 & RW2 \\ 
  Angola & HUAMBO & 1989 & 279.17 & 210.59 & 357.02 & RW2 \\ 
  Angola & HUAMBO & 1990 & 285.20 & 220.17 & 363.03 & RW2 \\ 
  Angola & HUAMBO & 1991 & 288.18 & 225.38 & 362.58 & RW2 \\ 
  Angola & HUAMBO & 1992 & 289.60 & 227.45 & 362.10 & RW2 \\ 
  Angola & HUAMBO & 1993 & 287.97 & 226.10 & 362.31 & RW2 \\ 
  Angola & HUAMBO & 1994 & 284.74 & 223.06 & 359.84 & RW2 \\ 
  Angola & HUAMBO & 1995 & 277.86 & 216.46 & 349.66 & RW2 \\ 
  Angola & HUAMBO & 1996 & 273.35 & 214.03 & 342.12 & RW2 \\ 
  Angola & HUAMBO & 1997 & 269.56 & 211.76 & 336.60 & RW2 \\ 
  Angola & HUAMBO & 1998 & 266.99 & 207.78 & 336.43 & RW2 \\ 
  Angola & HUAMBO & 1999 & 265.80 & 204.63 & 337.79 & RW2 \\ 
  Angola & HUAMBO & 2000 & 266.87 & 203.68 & 343.07 & RW2 \\ 
  Angola & HUAMBO & 2001 & 266.13 & 204.31 & 341.52 & RW2 \\ 
  Angola & HUAMBO & 2002 & 264.76 & 202.38 & 339.68 & RW2 \\ 
  Angola & HUAMBO & 2003 & 261.66 & 196.35 & 340.70 & RW2 \\ 
  Angola & HUAMBO & 2004 & 257.31 & 185.95 & 345.82 & RW2 \\ 
  Angola & HUAMBO & 2005 & 250.94 & 170.75 & 352.30 & RW2 \\ 
  Angola & HUAMBO & 2006 & 244.24 & 160.55 & 354.20 & RW2 \\ 
  Angola & HUAMBO & 2007 & 236.67 & 151.17 & 348.68 & RW2 \\ 
  Angola & HUAMBO & 2008 & 229.03 & 145.09 & 341.69 & RW2 \\ 
  Angola & HUAMBO & 2009 & 221.10 & 141.71 & 327.22 & RW2 \\ 
  Angola & HUAMBO & 2010 & 213.28 & 141.65 & 308.65 & RW2 \\ 
  Angola & HUAMBO & 2011 & 205.12 & 140.81 & 288.68 & RW2 \\ 
  Angola & HUAMBO & 2012 & 197.46 & 136.82 & 275.85 & RW2 \\ 
  Angola & HUAMBO & 2013 & 189.88 & 125.54 & 276.45 & RW2 \\ 
  Angola & HUAMBO & 2014 & 182.52 & 107.43 & 292.89 & RW2 \\ 
  Angola & HUAMBO & 2015 & 174.88 & 84.63 & 325.55 & RW2 \\ 
  Angola & HUAMBO & 2016 & 167.89 & 65.63 & 367.17 & RW2 \\ 
  Angola & HUAMBO & 2017 & 161.37 & 49.27 & 418.30 & RW2 \\ 
  Angola & HUAMBO & 2018 & 154.88 & 36.13 & 479.28 & RW2 \\ 
  Angola & HUAMBO & 2019 & 148.37 & 24.68 & 547.60 & RW2 \\ 
  Angola & HUÍLA & 1980 & 257.55 & 135.88 & 440.12 & RW2 \\ 
  Angola & HUÍLA & 1981 & 252.10 & 145.29 & 403.65 & RW2 \\ 
  Angola & HUÍLA & 1982 & 246.01 & 150.87 & 378.45 & RW2 \\ 
  Angola & HUÍLA & 1983 & 240.57 & 152.59 & 359.53 & RW2 \\ 
  Angola & HUÍLA & 1984 & 236.03 & 153.19 & 346.63 & RW2 \\ 
  Angola & HUÍLA & 1985 & 231.55 & 154.55 & 330.20 & RW2 \\ 
  Angola & HUÍLA & 1986 & 230.22 & 159.84 & 320.41 & RW2 \\ 
  Angola & HUÍLA & 1987 & 231.23 & 164.31 & 314.55 & RW2 \\ 
  Angola & HUÍLA & 1988 & 234.88 & 169.01 & 314.39 & RW2 \\ 
  Angola & HUÍLA & 1989 & 240.14 & 176.02 & 316.68 & RW2 \\ 
  Angola & HUÍLA & 1990 & 248.09 & 187.05 & 323.05 & RW2 \\ 
  Angola & HUÍLA & 1991 & 253.25 & 194.26 & 324.23 & RW2 \\ 
  Angola & HUÍLA & 1992 & 256.76 & 199.37 & 325.69 & RW2 \\ 
  Angola & HUÍLA & 1993 & 258.66 & 201.14 & 327.36 & RW2 \\ 
  Angola & HUÍLA & 1994 & 258.65 & 201.21 & 328.74 & RW2 \\ 
  Angola & HUÍLA & 1995 & 255.15 & 197.36 & 320.98 & RW2 \\ 
  Angola & HUÍLA & 1996 & 254.11 & 199.65 & 315.99 & RW2 \\ 
  Angola & HUÍLA & 1997 & 253.47 & 200.20 & 314.98 & RW2 \\ 
  Angola & HUÍLA & 1998 & 254.69 & 200.16 & 317.11 & RW2 \\ 
  Angola & HUÍLA & 1999 & 257.15 & 199.94 & 322.63 & RW2 \\ 
  Angola & HUÍLA & 2000 & 261.39 & 203.24 & 330.81 & RW2 \\ 
  Angola & HUÍLA & 2001 & 264.59 & 207.04 & 332.85 & RW2 \\ 
  Angola & HUÍLA & 2002 & 266.87 & 208.33 & 334.91 & RW2 \\ 
  Angola & HUÍLA & 2003 & 267.76 & 205.43 & 340.31 & RW2 \\ 
  Angola & HUÍLA & 2004 & 267.53 & 198.97 & 349.55 & RW2 \\ 
  Angola & HUÍLA & 2005 & 264.93 & 184.47 & 364.47 & RW2 \\ 
  Angola & HUÍLA & 2006 & 262.23 & 176.27 & 369.88 & RW2 \\ 
  Angola & HUÍLA & 2007 & 258.57 & 169.74 & 371.91 & RW2 \\ 
  Angola & HUÍLA & 2008 & 254.63 & 164.81 & 369.40 & RW2 \\ 
  Angola & HUÍLA & 2009 & 250.32 & 164.54 & 360.17 & RW2 \\ 
  Angola & HUÍLA & 2010 & 245.62 & 167.65 & 345.15 & RW2 \\ 
  Angola & HUÍLA & 2011 & 241.23 & 169.85 & 329.98 & RW2 \\ 
  Angola & HUÍLA & 2012 & 236.60 & 169.03 & 320.54 & RW2 \\ 
  Angola & HUÍLA & 2013 & 231.84 & 158.61 & 325.50 & RW2 \\ 
  Angola & HUÍLA & 2014 & 227.05 & 138.50 & 349.35 & RW2 \\ 
  Angola & HUÍLA & 2015 & 222.44 & 113.04 & 393.06 & RW2 \\ 
  Angola & HUÍLA & 2016 & 218.58 & 88.19 & 443.06 & RW2 \\ 
  Angola & HUÍLA & 2017 & 213.07 & 68.17 & 502.79 & RW2 \\ 
  Angola & HUÍLA & 2018 & 208.55 & 50.84 & 565.88 & RW2 \\ 
  Angola & HUÍLA & 2019 & 205.15 & 36.35 & 633.60 & RW2 \\ 
  Angola & LUANDA & 1980 & 208.38 & 107.19 & 367.85 & RW2 \\ 
  Angola & LUANDA & 1981 & 200.13 & 112.65 & 330.10 & RW2 \\ 
  Angola & LUANDA & 1982 & 190.68 & 113.66 & 302.51 & RW2 \\ 
  Angola & LUANDA & 1983 & 182.91 & 111.75 & 283.84 & RW2 \\ 
  Angola & LUANDA & 1984 & 175.69 & 109.70 & 269.87 & RW2 \\ 
  Angola & LUANDA & 1985 & 168.94 & 107.71 & 253.80 & RW2 \\ 
  Angola & LUANDA & 1986 & 164.90 & 108.16 & 242.05 & RW2 \\ 
  Angola & LUANDA & 1987 & 163.10 & 109.04 & 234.93 & RW2 \\ 
  Angola & LUANDA & 1988 & 162.71 & 110.61 & 230.28 & RW2 \\ 
  Angola & LUANDA & 1989 & 164.29 & 113.23 & 229.48 & RW2 \\ 
  Angola & LUANDA & 1990 & 167.86 & 118.90 & 231.29 & RW2 \\ 
  Angola & LUANDA & 1991 & 169.45 & 123.18 & 228.35 & RW2 \\ 
  Angola & LUANDA & 1992 & 170.05 & 126.24 & 226.18 & RW2 \\ 
  Angola & LUANDA & 1993 & 169.48 & 126.53 & 224.87 & RW2 \\ 
  Angola & LUANDA & 1994 & 167.83 & 125.73 & 222.87 & RW2 \\ 
  Angola & LUANDA & 1995 & 163.81 & 122.83 & 213.60 & RW2 \\ 
  Angola & LUANDA & 1996 & 161.27 & 122.52 & 208.33 & RW2 \\ 
  Angola & LUANDA & 1997 & 159.32 & 122.19 & 205.21 & RW2 \\ 
  Angola & LUANDA & 1998 & 158.47 & 120.61 & 205.11 & RW2 \\ 
  Angola & LUANDA & 1999 & 158.09 & 118.66 & 207.46 & RW2 \\ 
  Angola & LUANDA & 2000 & 159.26 & 119.10 & 212.02 & RW2 \\ 
  Angola & LUANDA & 2001 & 159.23 & 118.69 & 212.00 & RW2 \\ 
  Angola & LUANDA & 2002 & 158.28 & 117.39 & 211.18 & RW2 \\ 
  Angola & LUANDA & 2003 & 156.78 & 113.52 & 214.57 & RW2 \\ 
  Angola & LUANDA & 2004 & 153.79 & 106.50 & 218.10 & RW2 \\ 
  Angola & LUANDA & 2005 & 149.49 & 96.83 & 223.81 & RW2 \\ 
  Angola & LUANDA & 2006 & 144.96 & 89.45 & 226.45 & RW2 \\ 
  Angola & LUANDA & 2007 & 140.10 & 83.19 & 226.39 & RW2 \\ 
  Angola & LUANDA & 2008 & 134.95 & 79.13 & 220.84 & RW2 \\ 
  Angola & LUANDA & 2009 & 129.81 & 75.58 & 213.32 & RW2 \\ 
  Angola & LUANDA & 2010 & 124.42 & 73.95 & 200.89 & RW2 \\ 
  Angola & LUANDA & 2011 & 119.42 & 71.62 & 191.27 & RW2 \\ 
  Angola & LUANDA & 2012 & 114.55 & 67.59 & 185.63 & RW2 \\ 
  Angola & LUANDA & 2013 & 109.41 & 60.87 & 187.14 & RW2 \\ 
  Angola & LUANDA & 2014 & 104.65 & 51.75 & 198.87 & RW2 \\ 
  Angola & LUANDA & 2015 & 99.99 & 40.47 & 222.30 & RW2 \\ 
  Angola & LUANDA & 2016 & 95.56 & 31.57 & 252.67 & RW2 \\ 
  Angola & LUANDA & 2017 & 91.08 & 23.29 & 291.45 & RW2 \\ 
  Angola & LUANDA & 2018 & 87.61 & 16.70 & 345.13 & RW2 \\ 
  Angola & LUANDA & 2019 & 83.23 & 11.93 & 410.49 & RW2 \\ 
  Angola & LUNDA NORTE & 1980 & 207.09 & 71.89 & 473.43 & RW2 \\ 
  Angola & LUNDA NORTE & 1981 & 200.22 & 77.95 & 427.42 & RW2 \\ 
  Angola & LUNDA NORTE & 1982 & 193.53 & 82.54 & 392.62 & RW2 \\ 
  Angola & LUNDA NORTE & 1983 & 186.32 & 85.66 & 364.04 & RW2 \\ 
  Angola & LUNDA NORTE & 1984 & 181.31 & 88.24 & 340.22 & RW2 \\ 
  Angola & LUNDA NORTE & 1985 & 174.85 & 90.06 & 312.69 & RW2 \\ 
  Angola & LUNDA NORTE & 1986 & 172.55 & 93.83 & 294.95 & RW2 \\ 
  Angola & LUNDA NORTE & 1987 & 171.99 & 98.26 & 283.49 & RW2 \\ 
  Angola & LUNDA NORTE & 1988 & 173.13 & 103.51 & 274.68 & RW2 \\ 
  Angola & LUNDA NORTE & 1989 & 175.83 & 107.86 & 271.18 & RW2 \\ 
  Angola & LUNDA NORTE & 1990 & 181.17 & 115.99 & 273.02 & RW2 \\ 
  Angola & LUNDA NORTE & 1991 & 183.49 & 121.06 & 269.46 & RW2 \\ 
  Angola & LUNDA NORTE & 1992 & 184.97 & 125.28 & 265.69 & RW2 \\ 
  Angola & LUNDA NORTE & 1993 & 184.85 & 127.25 & 262.11 & RW2 \\ 
  Angola & LUNDA NORTE & 1994 & 183.42 & 127.14 & 258.72 & RW2 \\ 
  Angola & LUNDA NORTE & 1995 & 179.40 & 125.19 & 250.52 & RW2 \\ 
  Angola & LUNDA NORTE & 1996 & 177.08 & 124.95 & 244.11 & RW2 \\ 
  Angola & LUNDA NORTE & 1997 & 175.73 & 125.17 & 241.63 & RW2 \\ 
  Angola & LUNDA NORTE & 1998 & 175.79 & 123.41 & 242.27 & RW2 \\ 
  Angola & LUNDA NORTE & 1999 & 176.28 & 122.22 & 244.59 & RW2 \\ 
  Angola & LUNDA NORTE & 2000 & 178.62 & 123.67 & 252.29 & RW2 \\ 
  Angola & LUNDA NORTE & 2001 & 180.71 & 124.08 & 254.28 & RW2 \\ 
  Angola & LUNDA NORTE & 2002 & 181.63 & 123.67 & 257.13 & RW2 \\ 
  Angola & LUNDA NORTE & 2003 & 181.76 & 121.60 & 261.49 & RW2 \\ 
  Angola & LUNDA NORTE & 2004 & 180.83 & 116.89 & 269.11 & RW2 \\ 
  Angola & LUNDA NORTE & 2005 & 178.96 & 109.02 & 277.20 & RW2 \\ 
  Angola & LUNDA NORTE & 2006 & 176.30 & 103.14 & 284.04 & RW2 \\ 
  Angola & LUNDA NORTE & 2007 & 174.10 & 99.01 & 287.33 & RW2 \\ 
  Angola & LUNDA NORTE & 2008 & 170.98 & 95.70 & 285.36 & RW2 \\ 
  Angola & LUNDA NORTE & 2009 & 167.97 & 93.99 & 280.53 & RW2 \\ 
  Angola & LUNDA NORTE & 2010 & 165.08 & 94.60 & 272.52 & RW2 \\ 
  Angola & LUNDA NORTE & 2011 & 161.99 & 93.91 & 265.49 & RW2 \\ 
  Angola & LUNDA NORTE & 2012 & 159.05 & 90.54 & 263.14 & RW2 \\ 
  Angola & LUNDA NORTE & 2013 & 155.67 & 83.97 & 270.55 & RW2 \\ 
  Angola & LUNDA NORTE & 2014 & 152.52 & 73.78 & 290.48 & RW2 \\ 
  Angola & LUNDA NORTE & 2015 & 149.50 & 60.94 & 325.93 & RW2 \\ 
  Angola & LUNDA NORTE & 2016 & 145.99 & 47.89 & 368.83 & RW2 \\ 
  Angola & LUNDA NORTE & 2017 & 143.23 & 37.15 & 423.92 & RW2 \\ 
  Angola & LUNDA NORTE & 2018 & 140.69 & 26.91 & 489.65 & RW2 \\ 
  Angola & LUNDA NORTE & 2019 & 136.91 & 19.64 & 563.14 & RW2 \\ 
  Angola & LUNDA SUL & 1980 & 113.38 & 47.29 & 247.28 & RW2 \\ 
  Angola & LUNDA SUL & 1981 & 110.76 & 50.55 & 225.04 & RW2 \\ 
  Angola & LUNDA SUL & 1982 & 108.36 & 52.52 & 210.33 & RW2 \\ 
  Angola & LUNDA SUL & 1983 & 106.43 & 53.43 & 199.52 & RW2 \\ 
  Angola & LUNDA SUL & 1984 & 104.64 & 54.54 & 191.50 & RW2 \\ 
  Angola & LUNDA SUL & 1985 & 103.14 & 55.31 & 183.71 & RW2 \\ 
  Angola & LUNDA SUL & 1986 & 103.30 & 56.88 & 178.30 & RW2 \\ 
  Angola & LUNDA SUL & 1987 & 104.93 & 59.89 & 175.88 & RW2 \\ 
  Angola & LUNDA SUL & 1988 & 107.48 & 63.18 & 176.14 & RW2 \\ 
  Angola & LUNDA SUL & 1989 & 111.58 & 67.01 & 178.22 & RW2 \\ 
  Angola & LUNDA SUL & 1990 & 117.02 & 73.05 & 182.04 & RW2 \\ 
  Angola & LUNDA SUL & 1991 & 120.83 & 77.70 & 182.82 & RW2 \\ 
  Angola & LUNDA SUL & 1992 & 123.79 & 81.45 & 184.29 & RW2 \\ 
  Angola & LUNDA SUL & 1993 & 126.08 & 84.06 & 184.33 & RW2 \\ 
  Angola & LUNDA SUL & 1994 & 126.94 & 85.76 & 185.48 & RW2 \\ 
  Angola & LUNDA SUL & 1995 & 125.74 & 85.01 & 181.83 & RW2 \\ 
  Angola & LUNDA SUL & 1996 & 126.22 & 86.73 & 179.76 & RW2 \\ 
  Angola & LUNDA SUL & 1997 & 126.95 & 87.71 & 180.20 & RW2 \\ 
  Angola & LUNDA SUL & 1998 & 128.53 & 89.07 & 181.72 & RW2 \\ 
  Angola & LUNDA SUL & 1999 & 130.99 & 90.19 & 186.37 & RW2 \\ 
  Angola & LUNDA SUL & 2000 & 134.81 & 92.66 & 193.06 & RW2 \\ 
  Angola & LUNDA SUL & 2001 & 137.76 & 95.31 & 195.81 & RW2 \\ 
  Angola & LUNDA SUL & 2002 & 140.29 & 96.92 & 199.87 & RW2 \\ 
  Angola & LUNDA SUL & 2003 & 142.45 & 96.68 & 204.96 & RW2 \\ 
  Angola & LUNDA SUL & 2004 & 143.50 & 94.39 & 212.70 & RW2 \\ 
  Angola & LUNDA SUL & 2005 & 142.89 & 88.64 & 221.50 & RW2 \\ 
  Angola & LUNDA SUL & 2006 & 142.71 & 84.87 & 229.81 & RW2 \\ 
  Angola & LUNDA SUL & 2007 & 142.20 & 82.03 & 233.89 & RW2 \\ 
  Angola & LUNDA SUL & 2008 & 141.02 & 80.59 & 235.36 & RW2 \\ 
  Angola & LUNDA SUL & 2009 & 140.16 & 80.25 & 233.99 & RW2 \\ 
  Angola & LUNDA SUL & 2010 & 138.89 & 81.06 & 226.15 & RW2 \\ 
  Angola & LUNDA SUL & 2011 & 137.46 & 81.57 & 222.11 & RW2 \\ 
  Angola & LUNDA SUL & 2012 & 136.26 & 80.31 & 221.94 & RW2 \\ 
  Angola & LUNDA SUL & 2013 & 134.65 & 75.22 & 228.36 & RW2 \\ 
  Angola & LUNDA SUL & 2014 & 133.26 & 66.61 & 247.64 & RW2 \\ 
  Angola & LUNDA SUL & 2015 & 132.18 & 55.17 & 282.92 & RW2 \\ 
  Angola & LUNDA SUL & 2016 & 130.96 & 44.64 & 328.31 & RW2 \\ 
  Angola & LUNDA SUL & 2017 & 128.90 & 34.36 & 384.36 & RW2 \\ 
  Angola & LUNDA SUL & 2018 & 127.91 & 26.25 & 448.53 & RW2 \\ 
  Angola & LUNDA SUL & 2019 & 126.55 & 19.48 & 520.23 & RW2 \\ 
  Angola & MALANJE & 1980 & 293.22 & 125.86 & 555.11 & RW2 \\ 
  Angola & MALANJE & 1981 & 278.35 & 128.38 & 510.52 & RW2 \\ 
  Angola & MALANJE & 1982 & 263.70 & 128.28 & 475.68 & RW2 \\ 
  Angola & MALANJE & 1983 & 250.60 & 126.89 & 442.47 & RW2 \\ 
  Angola & MALANJE & 1984 & 238.88 & 123.55 & 414.95 & RW2 \\ 
  Angola & MALANJE & 1985 & 227.86 & 121.55 & 385.93 & RW2 \\ 
  Angola & MALANJE & 1986 & 220.29 & 121.83 & 363.56 & RW2 \\ 
  Angola & MALANJE & 1987 & 214.98 & 123.40 & 347.56 & RW2 \\ 
  Angola & MALANJE & 1988 & 211.89 & 125.78 & 332.95 & RW2 \\ 
  Angola & MALANJE & 1989 & 210.85 & 128.18 & 325.64 & RW2 \\ 
  Angola & MALANJE & 1990 & 211.90 & 134.06 & 319.96 & RW2 \\ 
  Angola & MALANJE & 1991 & 211.50 & 138.22 & 310.58 & RW2 \\ 
  Angola & MALANJE & 1992 & 209.55 & 139.84 & 301.89 & RW2 \\ 
  Angola & MALANJE & 1993 & 205.93 & 139.95 & 293.51 & RW2 \\ 
  Angola & MALANJE & 1994 & 200.86 & 138.11 & 283.48 & RW2 \\ 
  Angola & MALANJE & 1995 & 193.89 & 134.20 & 269.31 & RW2 \\ 
  Angola & MALANJE & 1996 & 188.36 & 133.18 & 259.00 & RW2 \\ 
  Angola & MALANJE & 1997 & 184.42 & 131.96 & 249.99 & RW2 \\ 
  Angola & MALANJE & 1998 & 181.62 & 130.76 & 246.06 & RW2 \\ 
  Angola & MALANJE & 1999 & 180.01 & 129.03 & 244.36 & RW2 \\ 
  Angola & MALANJE & 2000 & 180.27 & 129.76 & 244.97 & RW2 \\ 
  Angola & MALANJE & 2001 & 179.25 & 130.10 & 241.88 & RW2 \\ 
  Angola & MALANJE & 2002 & 177.82 & 129.60 & 239.53 & RW2 \\ 
  Angola & MALANJE & 2003 & 175.50 & 125.95 & 239.48 & RW2 \\ 
  Angola & MALANJE & 2004 & 172.34 & 119.33 & 243.67 & RW2 \\ 
  Angola & MALANJE & 2005 & 167.90 & 109.17 & 248.03 & RW2 \\ 
  Angola & MALANJE & 2006 & 163.21 & 101.64 & 250.07 & RW2 \\ 
  Angola & MALANJE & 2007 & 158.34 & 95.21 & 249.68 & RW2 \\ 
  Angola & MALANJE & 2008 & 152.92 & 91.11 & 242.81 & RW2 \\ 
  Angola & MALANJE & 2009 & 147.83 & 88.96 & 234.38 & RW2 \\ 
  Angola & MALANJE & 2010 & 142.56 & 88.25 & 221.79 & RW2 \\ 
  Angola & MALANJE & 2011 & 137.41 & 87.25 & 209.15 & RW2 \\ 
  Angola & MALANJE & 2012 & 132.57 & 84.09 & 202.19 & RW2 \\ 
  Angola & MALANJE & 2013 & 127.48 & 77.35 & 204.65 & RW2 \\ 
  Angola & MALANJE & 2014 & 122.98 & 66.40 & 216.52 & RW2 \\ 
  Angola & MALANJE & 2015 & 118.04 & 53.20 & 244.90 & RW2 \\ 
  Angola & MALANJE & 2016 & 113.85 & 41.15 & 279.42 & RW2 \\ 
  Angola & MALANJE & 2017 & 109.09 & 30.54 & 323.28 & RW2 \\ 
  Angola & MALANJE & 2018 & 105.34 & 22.29 & 378.66 & RW2 \\ 
  Angola & MALANJE & 2019 & 101.22 & 15.50 & 442.90 & RW2 \\ 
  Angola & MOXICO & 1980 & 488.52 & 20.71 & 973.37 & RW2 \\ 
  Angola & MOXICO & 1981 & 451.90 & 20.53 & 964.30 & RW2 \\ 
  Angola & MOXICO & 1982 & 423.52 & 21.88 & 952.06 & RW2 \\ 
  Angola & MOXICO & 1983 & 388.32 & 21.99 & 936.07 & RW2 \\ 
  Angola & MOXICO & 1984 & 356.02 & 23.09 & 917.95 & RW2 \\ 
  Angola & MOXICO & 1985 & 328.45 & 22.64 & 894.16 & RW2 \\ 
  Angola & MOXICO & 1986 & 306.10 & 24.48 & 867.04 & RW2 \\ 
  Angola & MOXICO & 1987 & 285.68 & 25.55 & 834.17 & RW2 \\ 
  Angola & MOXICO & 1988 & 266.11 & 27.60 & 799.30 & RW2 \\ 
  Angola & MOXICO & 1989 & 252.99 & 29.40 & 764.87 & RW2 \\ 
  Angola & MOXICO & 1990 & 242.58 & 31.85 & 727.45 & RW2 \\ 
  Angola & MOXICO & 1991 & 229.93 & 34.01 & 683.90 & RW2 \\ 
  Angola & MOXICO & 1992 & 216.28 & 36.09 & 634.40 & RW2 \\ 
  Angola & MOXICO & 1993 & 202.11 & 37.32 & 585.47 & RW2 \\ 
  Angola & MOXICO & 1994 & 186.22 & 39.00 & 528.86 & RW2 \\ 
  Angola & MOXICO & 1995 & 168.91 & 39.17 & 468.67 & RW2 \\ 
  Angola & MOXICO & 1996 & 155.52 & 40.04 & 416.01 & RW2 \\ 
  Angola & MOXICO & 1997 & 143.58 & 40.76 & 364.14 & RW2 \\ 
  Angola & MOXICO & 1998 & 132.73 & 41.25 & 324.96 & RW2 \\ 
  Angola & MOXICO & 1999 & 123.49 & 41.77 & 289.59 & RW2 \\ 
  Angola & MOXICO & 2000 & 116.53 & 42.64 & 261.80 & RW2 \\ 
  Angola & MOXICO & 2001 & 109.41 & 42.45 & 238.25 & RW2 \\ 
  Angola & MOXICO & 2002 & 101.44 & 41.36 & 215.64 & RW2 \\ 
  Angola & MOXICO & 2003 & 94.13 & 38.83 & 201.04 & RW2 \\ 
  Angola & MOXICO & 2004 & 86.70 & 35.44 & 190.20 & RW2 \\ 
  Angola & MOXICO & 2005 & 78.74 & 30.30 & 184.51 & RW2 \\ 
  Angola & MOXICO & 2006 & 72.17 & 26.29 & 180.09 & RW2 \\ 
  Angola & MOXICO & 2007 & 65.18 & 22.24 & 176.55 & RW2 \\ 
  Angola & MOXICO & 2008 & 59.18 & 18.52 & 171.86 & RW2 \\ 
  Angola & MOXICO & 2009 & 53.36 & 15.58 & 168.98 & RW2 \\ 
  Angola & MOXICO & 2010 & 48.20 & 12.91 & 165.35 & RW2 \\ 
  Angola & MOXICO & 2011 & 43.59 & 10.49 & 164.12 & RW2 \\ 
  Angola & MOXICO & 2012 & 39.37 & 8.39 & 165.34 & RW2 \\ 
  Angola & MOXICO & 2013 & 35.31 & 6.64 & 169.03 & RW2 \\ 
  Angola & MOXICO & 2014 & 32.05 & 5.10 & 175.80 & RW2 \\ 
  Angola & MOXICO & 2015 & 28.54 & 3.70 & 187.61 & RW2 \\ 
  Angola & MOXICO & 2016 & 25.64 & 2.69 & 206.47 & RW2 \\ 
  Angola & MOXICO & 2017 & 23.13 & 1.87 & 231.22 & RW2 \\ 
  Angola & MOXICO & 2018 & 20.77 & 1.29 & 260.34 & RW2 \\ 
  Angola & MOXICO & 2019 & 18.72 & 0.86 & 300.06 & RW2 \\ 
  Angola & NAMIBE & 1980 & 193.05 & 95.58 & 347.43 & RW2 \\ 
  Angola & NAMIBE & 1981 & 190.14 & 102.65 & 318.65 & RW2 \\ 
  Angola & NAMIBE & 1982 & 187.61 & 108.28 & 300.18 & RW2 \\ 
  Angola & NAMIBE & 1983 & 185.23 & 110.84 & 291.18 & RW2 \\ 
  Angola & NAMIBE & 1984 & 183.31 & 112.53 & 283.47 & RW2 \\ 
  Angola & NAMIBE & 1985 & 182.05 & 114.38 & 274.02 & RW2 \\ 
  Angola & NAMIBE & 1986 & 182.89 & 118.57 & 269.71 & RW2 \\ 
  Angola & NAMIBE & 1987 & 186.21 & 123.82 & 268.31 & RW2 \\ 
  Angola & NAMIBE & 1988 & 191.25 & 129.72 & 270.79 & RW2 \\ 
  Angola & NAMIBE & 1989 & 197.62 & 136.27 & 275.20 & RW2 \\ 
  Angola & NAMIBE & 1990 & 206.81 & 147.27 & 284.05 & RW2 \\ 
  Angola & NAMIBE & 1991 & 213.73 & 155.49 & 287.21 & RW2 \\ 
  Angola & NAMIBE & 1992 & 218.95 & 161.32 & 290.33 & RW2 \\ 
  Angola & NAMIBE & 1993 & 222.04 & 164.88 & 293.50 & RW2 \\ 
  Angola & NAMIBE & 1994 & 223.30 & 166.24 & 295.36 & RW2 \\ 
  Angola & NAMIBE & 1995 & 221.48 & 165.20 & 291.21 & RW2 \\ 
  Angola & NAMIBE & 1996 & 221.71 & 166.99 & 289.02 & RW2 \\ 
  Angola & NAMIBE & 1997 & 221.96 & 167.93 & 288.52 & RW2 \\ 
  Angola & NAMIBE & 1998 & 223.77 & 168.21 & 293.04 & RW2 \\ 
  Angola & NAMIBE & 1999 & 226.39 & 167.34 & 298.29 & RW2 \\ 
  Angola & NAMIBE & 2000 & 230.59 & 170.59 & 306.81 & RW2 \\ 
  Angola & NAMIBE & 2001 & 233.56 & 172.56 & 310.90 & RW2 \\ 
  Angola & NAMIBE & 2002 & 235.15 & 172.91 & 313.69 & RW2 \\ 
  Angola & NAMIBE & 2003 & 235.51 & 169.79 & 318.66 & RW2 \\ 
  Angola & NAMIBE & 2004 & 234.59 & 163.87 & 325.78 & RW2 \\ 
  Angola & NAMIBE & 2005 & 231.34 & 151.56 & 335.42 & RW2 \\ 
  Angola & NAMIBE & 2006 & 227.83 & 143.12 & 341.32 & RW2 \\ 
  Angola & NAMIBE & 2007 & 223.54 & 137.37 & 342.02 & RW2 \\ 
  Angola & NAMIBE & 2008 & 218.73 & 132.97 & 338.80 & RW2 \\ 
  Angola & NAMIBE & 2009 & 213.48 & 130.32 & 328.57 & RW2 \\ 
  Angola & NAMIBE & 2010 & 208.31 & 131.05 & 313.86 & RW2 \\ 
  Angola & NAMIBE & 2011 & 202.70 & 129.88 & 301.00 & RW2 \\ 
  Angola & NAMIBE & 2012 & 197.15 & 126.44 & 293.44 & RW2 \\ 
  Angola & NAMIBE & 2013 & 191.75 & 117.05 & 298.26 & RW2 \\ 
  Angola & NAMIBE & 2014 & 186.24 & 101.35 & 316.00 & RW2 \\ 
  Angola & NAMIBE & 2015 & 181.60 & 82.66 & 354.88 & RW2 \\ 
  Angola & NAMIBE & 2016 & 176.54 & 64.13 & 397.99 & RW2 \\ 
  Angola & NAMIBE & 2017 & 170.93 & 49.19 & 452.11 & RW2 \\ 
  Angola & NAMIBE & 2018 & 166.00 & 35.69 & 520.79 & RW2 \\ 
  Angola & NAMIBE & 2019 & 161.68 & 25.63 & 580.42 & RW2 \\ 
  Angola & UÍGE & 1980 & 440.20 & 275.17 & 632.91 & RW2 \\ 
  Angola & UÍGE & 1981 & 413.77 & 275.62 & 578.65 & RW2 \\ 
  Angola & UÍGE & 1982 & 389.45 & 268.15 & 533.96 & RW2 \\ 
  Angola & UÍGE & 1983 & 365.42 & 254.93 & 496.99 & RW2 \\ 
  Angola & UÍGE & 1984 & 343.06 & 240.61 & 466.86 & RW2 \\ 
  Angola & UÍGE & 1985 & 321.70 & 225.79 & 432.20 & RW2 \\ 
  Angola & UÍGE & 1986 & 304.32 & 217.25 & 406.70 & RW2 \\ 
  Angola & UÍGE & 1987 & 291.18 & 209.18 & 386.21 & RW2 \\ 
  Angola & UÍGE & 1988 & 280.55 & 202.91 & 371.88 & RW2 \\ 
  Angola & UÍGE & 1989 & 272.97 & 197.48 & 361.76 & RW2 \\ 
  Angola & UÍGE & 1990 & 268.72 & 196.03 & 355.47 & RW2 \\ 
  Angola & UÍGE & 1991 & 261.98 & 192.44 & 344.78 & RW2 \\ 
  Angola & UÍGE & 1992 & 254.78 & 187.34 & 333.95 & RW2 \\ 
  Angola & UÍGE & 1993 & 246.38 & 180.29 & 325.58 & RW2 \\ 
  Angola & UÍGE & 1994 & 236.47 & 171.60 & 313.95 & RW2 \\ 
  Angola & UÍGE & 1995 & 225.25 & 161.95 & 297.75 & RW2 \\ 
  Angola & UÍGE & 1996 & 216.28 & 157.06 & 284.76 & RW2 \\ 
  Angola & UÍGE & 1997 & 209.09 & 152.88 & 274.20 & RW2 \\ 
  Angola & UÍGE & 1998 & 203.64 & 148.06 & 268.74 & RW2 \\ 
  Angola & UÍGE & 1999 & 199.46 & 144.04 & 265.06 & RW2 \\ 
  Angola & UÍGE & 2000 & 197.72 & 143.07 & 265.47 & RW2 \\ 
  Angola & UÍGE & 2001 & 194.47 & 141.23 & 260.00 & RW2 \\ 
  Angola & UÍGE & 2002 & 190.90 & 138.94 & 255.38 & RW2 \\ 
  Angola & UÍGE & 2003 & 186.69 & 134.29 & 253.43 & RW2 \\ 
  Angola & UÍGE & 2004 & 181.15 & 125.46 & 254.57 & RW2 \\ 
  Angola & UÍGE & 2005 & 174.41 & 113.83 & 257.41 & RW2 \\ 
  Angola & UÍGE & 2006 & 167.86 & 104.74 & 255.91 & RW2 \\ 
  Angola & UÍGE & 2007 & 160.57 & 97.93 & 251.69 & RW2 \\ 
  Angola & UÍGE & 2008 & 153.88 & 92.78 & 243.86 & RW2 \\ 
  Angola & UÍGE & 2009 & 146.93 & 89.10 & 231.79 & RW2 \\ 
  Angola & UÍGE & 2010 & 139.86 & 88.16 & 214.93 & RW2 \\ 
  Angola & UÍGE & 2011 & 132.95 & 85.96 & 200.54 & RW2 \\ 
  Angola & UÍGE & 2012 & 126.53 & 82.09 & 190.05 & RW2 \\ 
  Angola & UÍGE & 2013 & 120.33 & 74.41 & 189.80 & RW2 \\ 
  Angola & UÍGE & 2014 & 114.24 & 62.83 & 202.12 & RW2 \\ 
  Angola & UÍGE & 2015 & 108.69 & 48.95 & 225.83 & RW2 \\ 
  Angola & UÍGE & 2016 & 103.00 & 37.35 & 257.21 & RW2 \\ 
  Angola & UÍGE & 2017 & 98.03 & 27.18 & 298.12 & RW2 \\ 
  Angola & UÍGE & 2018 & 92.73 & 19.55 & 349.21 & RW2 \\ 
  Angola & UÍGE & 2019 & 88.06 & 13.48 & 410.62 & RW2 \\ 
  Angola & ZAIRE & 1980 & 77.01 & 22.86 & 249.93 & RW2 \\ 
  Angola & ZAIRE & 1981 & 74.93 & 25.11 & 218.97 & RW2 \\ 
  Angola & ZAIRE & 1982 & 73.06 & 27.01 & 193.88 & RW2 \\ 
  Angola & ZAIRE & 1983 & 71.33 & 28.32 & 177.14 & RW2 \\ 
  Angola & ZAIRE & 1984 & 70.27 & 29.85 & 162.21 & RW2 \\ 
  Angola & ZAIRE & 1985 & 68.75 & 31.49 & 147.03 & RW2 \\ 
  Angola & ZAIRE & 1986 & 68.74 & 33.29 & 138.54 & RW2 \\ 
  Angola & ZAIRE & 1987 & 69.54 & 35.69 & 131.98 & RW2 \\ 
  Angola & ZAIRE & 1988 & 71.28 & 38.37 & 128.80 & RW2 \\ 
  Angola & ZAIRE & 1989 & 74.00 & 41.39 & 127.59 & RW2 \\ 
  Angola & ZAIRE & 1990 & 77.70 & 45.57 & 130.02 & RW2 \\ 
  Angola & ZAIRE & 1991 & 81.04 & 49.60 & 129.87 & RW2 \\ 
  Angola & ZAIRE & 1992 & 83.77 & 52.80 & 130.58 & RW2 \\ 
  Angola & ZAIRE & 1993 & 85.84 & 55.30 & 131.03 & RW2 \\ 
  Angola & ZAIRE & 1994 & 87.73 & 57.63 & 132.64 & RW2 \\ 
  Angola & ZAIRE & 1995 & 88.61 & 58.81 & 130.76 & RW2 \\ 
  Angola & ZAIRE & 1996 & 90.41 & 61.36 & 130.66 & RW2 \\ 
  Angola & ZAIRE & 1997 & 92.78 & 63.74 & 133.30 & RW2 \\ 
  Angola & ZAIRE & 1998 & 96.02 & 65.71 & 138.02 & RW2 \\ 
  Angola & ZAIRE & 1999 & 100.14 & 67.92 & 144.97 & RW2 \\ 
  Angola & ZAIRE & 2000 & 105.12 & 71.26 & 154.37 & RW2 \\ 
  Angola & ZAIRE & 2001 & 109.76 & 74.01 & 161.59 & RW2 \\ 
  Angola & ZAIRE & 2002 & 114.30 & 76.78 & 167.84 & RW2 \\ 
  Angola & ZAIRE & 2003 & 117.73 & 77.70 & 176.44 & RW2 \\ 
  Angola & ZAIRE & 2004 & 120.90 & 77.10 & 185.95 & RW2 \\ 
  Angola & ZAIRE & 2005 & 122.54 & 73.52 & 198.10 & RW2 \\ 
  Angola & ZAIRE & 2006 & 124.23 & 71.49 & 205.65 & RW2 \\ 
  Angola & ZAIRE & 2007 & 125.31 & 69.82 & 212.85 & RW2 \\ 
  Angola & ZAIRE & 2008 & 126.09 & 69.70 & 217.34 & RW2 \\ 
  Angola & ZAIRE & 2009 & 126.29 & 70.10 & 215.61 & RW2 \\ 
  Angola & ZAIRE & 2010 & 126.73 & 71.71 & 212.55 & RW2 \\ 
  Angola & ZAIRE & 2011 & 126.81 & 73.07 & 209.80 & RW2 \\ 
  Angola & ZAIRE & 2012 & 126.84 & 72.52 & 212.04 & RW2 \\ 
  Angola & ZAIRE & 2013 & 126.87 & 68.66 & 222.84 & RW2 \\ 
  Angola & ZAIRE & 2014 & 126.93 & 61.52 & 244.59 & RW2 \\ 
  Angola & ZAIRE & 2015 & 126.42 & 51.27 & 281.03 & RW2 \\ 
  Angola & ZAIRE & 2016 & 126.81 & 41.32 & 327.70 & RW2 \\ 
  Angola & ZAIRE & 2017 & 126.52 & 31.92 & 383.88 & RW2 \\ 
  Angola & ZAIRE & 2018 & 127.50 & 24.59 & 456.30 & RW2 \\ 
  Angola & ZAIRE & 2019 & 127.39 & 17.97 & 533.64 & RW2 \\ 
  Benin & ALL & 1980 & 223.93 & 217.05 & 230.95 & IHME \\ 
  Benin & ALL & 1980 & 216.37 & 162.37 & 281.94 & RW2 \\ 
  Benin & ALL & 1980 & 217.40 & 203.60 & 232.00 & UN \\ 
  Benin & ALL & 1981 & 218.77 & 212.00 & 225.59 & IHME \\ 
  Benin & ALL & 1981 & 213.27 & 174.73 & 257.40 & RW2 \\ 
  Benin & ALL & 1981 & 213.70 & 200.60 & 227.80 & UN \\ 
  Benin & ALL & 1982 & 213.78 & 207.45 & 220.20 & IHME \\ 
  Benin & ALL & 1982 & 210.26 & 175.10 & 250.22 & RW2 \\ 
  Benin & ALL & 1982 & 210.20 & 197.60 & 224.00 & UN \\ 
  Benin & ALL & 1983 & 209.44 & 203.18 & 215.64 & IHME \\ 
  Benin & ALL & 1983 & 206.95 & 168.16 & 251.47 & RW2 \\ 
  Benin & ALL & 1983 & 206.80 & 194.40 & 220.40 & UN \\ 
  Benin & ALL & 1984 & 205.04 & 199.05 & 211.06 & IHME \\ 
  Benin & ALL & 1984 & 203.90 & 161.62 & 252.56 & RW2 \\ 
  Benin & ALL & 1984 & 203.30 & 191.00 & 216.60 & UN \\ 
  Benin & ALL & 1985 & 200.55 & 194.56 & 206.46 & IHME \\ 
  Benin & ALL & 1985 & 200.44 & 162.65 & 244.66 & RW2 \\ 
  Benin & ALL & 1985 & 199.70 & 187.50 & 213.00 & UN \\ 
  Benin & ALL & 1986 & 196.11 & 190.30 & 201.82 & IHME \\ 
  Benin & ALL & 1986 & 196.70 & 162.27 & 236.16 & RW2 \\ 
  Benin & ALL & 1986 & 196.10 & 183.80 & 209.30 & UN \\ 
  Benin & ALL & 1987 & 192.01 & 186.55 & 197.23 & IHME \\ 
  Benin & ALL & 1987 & 192.68 & 160.61 & 229.98 & RW2 \\ 
  Benin & ALL & 1987 & 192.20 & 180.20 & 205.30 & UN \\ 
  Benin & ALL & 1988 & 187.91 & 182.84 & 193.22 & IHME \\ 
  Benin & ALL & 1988 & 188.22 & 155.09 & 226.67 & RW2 \\ 
  Benin & ALL & 1988 & 188.10 & 176.40 & 201.10 & UN \\ 
  Benin & ALL & 1989 & 184.16 & 178.89 & 189.64 & IHME \\ 
  Benin & ALL & 1989 & 183.60 & 149.37 & 224.22 & RW2 \\ 
  Benin & ALL & 1989 & 183.90 & 172.60 & 196.30 & UN \\ 
  Benin & ALL & 1990 & 180.25 & 175.05 & 185.80 & IHME \\ 
  Benin & ALL & 1990 & 178.71 & 145.13 & 217.92 & RW2 \\ 
  Benin & ALL & 1990 & 179.50 & 168.40 & 191.40 & UN \\ 
  Benin & ALL & 1991 & 176.26 & 171.23 & 181.69 & IHME \\ 
  Benin & ALL & 1991 & 174.25 & 143.40 & 209.41 & RW2 \\ 
  Benin & ALL & 1991 & 175.00 & 164.30 & 186.30 & UN \\ 
  Benin & ALL & 1992 & 172.37 & 167.57 & 177.99 & IHME \\ 
  Benin & ALL & 1992 & 170.07 & 140.75 & 203.65 & RW2 \\ 
  Benin & ALL & 1992 & 170.40 & 159.90 & 181.40 & UN \\ 
  Benin & ALL & 1993 & 168.45 & 163.67 & 173.86 & IHME \\ 
  Benin & ALL & 1993 & 166.28 & 136.76 & 200.73 & RW2 \\ 
  Benin & ALL & 1993 & 166.20 & 155.60 & 177.00 & UN \\ 
  Benin & ALL & 1994 & 164.77 & 159.82 & 170.12 & IHME \\ 
  Benin & ALL & 1994 & 162.75 & 131.84 & 199.86 & RW2 \\ 
  Benin & ALL & 1994 & 162.40 & 152.10 & 173.10 & UN \\ 
  Benin & ALL & 1995 & 160.98 & 156.09 & 166.22 & IHME \\ 
  Benin & ALL & 1995 & 159.76 & 129.68 & 195.71 & RW2 \\ 
  Benin & ALL & 1995 & 159.10 & 149.00 & 169.70 & UN \\ 
  Benin & ALL & 1996 & 157.01 & 152.05 & 162.11 & IHME \\ 
  Benin & ALL & 1996 & 156.63 & 128.66 & 190.23 & RW2 \\ 
  Benin & ALL & 1996 & 156.20 & 146.30 & 166.70 & UN \\ 
  Benin & ALL & 1997 & 152.93 & 147.92 & 157.98 & IHME \\ 
  Benin & ALL & 1997 & 153.58 & 126.95 & 184.71 & RW2 \\ 
  Benin & ALL & 1997 & 153.60 & 143.80 & 164.10 & UN \\ 
  Benin & ALL & 1998 & 148.78 & 143.88 & 153.72 & IHME \\ 
  Benin & ALL & 1998 & 150.56 & 123.38 & 183.11 & RW2 \\ 
  Benin & ALL & 1998 & 150.90 & 141.10 & 161.80 & UN \\ 
  Benin & ALL & 1999 & 144.45 & 139.48 & 149.58 & IHME \\ 
  Benin & ALL & 1999 & 147.40 & 118.79 & 180.73 & RW2 \\ 
  Benin & ALL & 1999 & 147.90 & 138.00 & 159.30 & UN \\ 
  Benin & ALL & 2000 & 140.00 & 135.08 & 145.22 & IHME \\ 
  Benin & ALL & 2000 & 144.12 & 116.03 & 177.43 & RW2 \\ 
  Benin & ALL & 2000 & 144.70 & 134.60 & 155.70 & UN \\ 
  Benin & ALL & 2001 & 135.42 & 130.46 & 140.41 & IHME \\ 
  Benin & ALL & 2001 & 140.71 & 114.48 & 171.58 & RW2 \\ 
  Benin & ALL & 2001 & 141.10 & 131.10 & 151.70 & UN \\ 
  Benin & ALL & 2002 & 130.72 & 125.83 & 135.62 & IHME \\ 
  Benin & ALL & 2002 & 137.12 & 112.63 & 166.01 & RW2 \\ 
  Benin & ALL & 2002 & 137.10 & 127.40 & 147.60 & UN \\ 
  Benin & ALL & 2003 & 126.07 & 121.13 & 130.81 & IHME \\ 
  Benin & ALL & 2003 & 133.50 & 108.96 & 162.72 & RW2 \\ 
  Benin & ALL & 2003 & 133.10 & 123.30 & 143.70 & UN \\ 
  Benin & ALL & 2004 & 121.39 & 116.45 & 126.45 & IHME \\ 
  Benin & ALL & 2004 & 129.57 & 103.02 & 161.51 & RW2 \\ 
  Benin & ALL & 2004 & 129.20 & 118.70 & 140.10 & UN \\ 
  Benin & ALL & 2005 & 116.84 & 111.73 & 122.01 & IHME \\ 
  Benin & ALL & 2005 & 125.75 & 97.47 & 160.53 & RW2 \\ 
  Benin & ALL & 2005 & 125.50 & 114.00 & 136.90 & UN \\ 
  Benin & ALL & 2006 & 112.41 & 106.94 & 117.66 & IHME \\ 
  Benin & ALL & 2006 & 121.85 & 95.38 & 154.06 & RW2 \\ 
  Benin & ALL & 2006 & 122.10 & 109.50 & 134.20 & UN \\ 
  Benin & ALL & 2007 & 108.01 & 102.17 & 113.64 & IHME \\ 
  Benin & ALL & 2007 & 118.06 & 93.65 & 147.70 & RW2 \\ 
  Benin & ALL & 2007 & 119.00 & 105.60 & 131.90 & UN \\ 
  Benin & ALL & 2008 & 103.73 & 97.81 & 109.78 & IHME \\ 
  Benin & ALL & 2008 & 114.52 & 88.12 & 147.62 & RW2 \\ 
  Benin & ALL & 2008 & 116.30 & 102.00 & 129.70 & UN \\ 
  Benin & ALL & 2009 & 99.63 & 93.34 & 105.90 & IHME \\ 
  Benin & ALL & 2009 & 110.75 & 77.03 & 157.33 & RW2 \\ 
  Benin & ALL & 2009 & 113.90 & 98.70 & 128.20 & UN \\ 
  Benin & ALL & 2010 & 95.78 & 89.20 & 102.94 & IHME \\ 
  Benin & ALL & 2010 & 107.15 & 62.14 & 180.23 & RW2 \\ 
  Benin & ALL & 2010 & 111.60 & 95.40 & 127.10 & UN \\ 
  Benin & ALL & 2011 & 92.12 & 85.19 & 99.97 & IHME \\ 
  Benin & ALL & 2011 & 103.83 & 48.94 & 206.73 & RW2 \\ 
  Benin & ALL & 2011 & 109.30 & 92.10 & 126.60 & UN \\ 
  Benin & ALL & 2012 & 88.65 & 81.20 & 97.16 & IHME \\ 
  Benin & ALL & 2012 & 100.38 & 37.26 & 242.48 & RW2 \\ 
  Benin & ALL & 2012 & 107.00 & 88.70 & 126.60 & UN \\ 
  Benin & ALL & 2013 & 85.00 & 77.16 & 94.39 & IHME \\ 
  Benin & ALL & 2013 & 97.43 & 28.01 & 285.54 & RW2 \\ 
  Benin & ALL & 2013 & 104.80 & 85.40 & 126.50 & UN \\ 
  Benin & ALL & 2014 & 81.41 & 73.17 & 91.20 & IHME \\ 
  Benin & ALL & 2014 & 94.18 & 20.48 & 338.63 & RW2 \\ 
  Benin & ALL & 2014 & 102.10 & 81.60 & 126.70 & UN \\ 
  Benin & ALL & 2015 & 77.88 & 69.51 & 88.30 & IHME \\ 
  Benin & ALL & 2015 & 90.77 & 14.90 & 396.73 & RW2 \\ 
  Benin & ALL & 2015 & 99.50 & 78.30 & 126.90 & UN \\ 
  Benin & ALL & 2016 & 88.43 & 10.69 & 470.34 & RW2 \\ 
  Benin & ALL & 2017 & 85.05 & 7.31 & 545.59 & RW2 \\ 
  Benin & ALL & 2018 & 82.24 & 5.09 & 627.02 & RW2 \\ 
  Benin & ALL & 2019 & 79.29 & 3.17 & 692.07 & RW2 \\ 
  Benin & ALL & 80-84 & 219.81 & 230.57 & 209.41 & HT-Direct \\ 
  Benin & ALL & 85-89 & 195.59 & 204.27 & 187.20 & HT-Direct \\ 
  Benin & ALL & 90-94 & 169.56 & 176.23 & 163.09 & HT-Direct \\ 
  Benin & ALL & 95-99 & 157.73 & 164.11 & 151.55 & HT-Direct \\ 
  Benin & ALL & 00-04 & 137.32 & 144.44 & 130.49 & HT-Direct \\ 
  Benin & ALL & 05-09 & 104.32 & 116.61 & 93.18 & HT-Direct \\ 
  Benin & ALL & 15-19 & 85.05 & 7.37 & 540.55 & RW2 \\ 
  Benin & ATACORA & 1980 & 281.56 & 222.81 & 349.24 & RW2 \\ 
  Benin & ATACORA & 1981 & 275.58 & 234.62 & 321.39 & RW2 \\ 
  Benin & ATACORA & 1982 & 269.41 & 232.77 & 309.38 & RW2 \\ 
  Benin & ATACORA & 1983 & 263.34 & 223.39 & 307.40 & RW2 \\ 
  Benin & ATACORA & 1984 & 257.25 & 214.53 & 304.30 & RW2 \\ 
  Benin & ATACORA & 1985 & 251.44 & 213.38 & 293.87 & RW2 \\ 
  Benin & ATACORA & 1986 & 245.43 & 211.59 & 282.11 & RW2 \\ 
  Benin & ATACORA & 1987 & 239.36 & 207.92 & 273.77 & RW2 \\ 
  Benin & ATACORA & 1988 & 233.34 & 200.81 & 269.33 & RW2 \\ 
  Benin & ATACORA & 1989 & 227.40 & 193.40 & 265.52 & RW2 \\ 
  Benin & ATACORA & 1990 & 221.44 & 188.09 & 258.40 & RW2 \\ 
  Benin & ATACORA & 1991 & 216.18 & 186.08 & 249.51 & RW2 \\ 
  Benin & ATACORA & 1992 & 211.19 & 182.55 & 242.67 & RW2 \\ 
  Benin & ATACORA & 1993 & 206.55 & 177.46 & 238.95 & RW2 \\ 
  Benin & ATACORA & 1994 & 202.57 & 171.49 & 237.10 & RW2 \\ 
  Benin & ATACORA & 1995 & 198.82 & 168.46 & 233.17 & RW2 \\ 
  Benin & ATACORA & 1996 & 195.36 & 167.50 & 226.48 & RW2 \\ 
  Benin & ATACORA & 1997 & 191.95 & 165.73 & 221.58 & RW2 \\ 
  Benin & ATACORA & 1998 & 188.60 & 161.94 & 218.83 & RW2 \\ 
  Benin & ATACORA & 1999 & 185.21 & 155.66 & 217.72 & RW2 \\ 
  Benin & ATACORA & 2000 & 181.67 & 152.19 & 215.01 & RW2 \\ 
  Benin & ATACORA & 2001 & 177.95 & 150.50 & 208.73 & RW2 \\ 
  Benin & ATACORA & 2002 & 173.92 & 147.98 & 203.27 & RW2 \\ 
  Benin & ATACORA & 2003 & 169.55 & 142.71 & 200.70 & RW2 \\ 
  Benin & ATACORA & 2004 & 165.08 & 135.43 & 199.67 & RW2 \\ 
  Benin & ATACORA & 2005 & 160.27 & 128.44 & 198.21 & RW2 \\ 
  Benin & ATACORA & 2006 & 155.62 & 124.77 & 192.59 & RW2 \\ 
  Benin & ATACORA & 2007 & 150.85 & 120.04 & 187.50 & RW2 \\ 
  Benin & ATACORA & 2008 & 146.30 & 112.14 & 188.47 & RW2 \\ 
  Benin & ATACORA & 2009 & 141.56 & 99.47 & 197.48 & RW2 \\ 
  Benin & ATACORA & 2010 & 137.27 & 83.10 & 219.01 & RW2 \\ 
  Benin & ATACORA & 2011 & 132.75 & 67.65 & 244.80 & RW2 \\ 
  Benin & ATACORA & 2012 & 128.54 & 53.78 & 277.01 & RW2 \\ 
  Benin & ATACORA & 2013 & 124.64 & 41.78 & 320.49 & RW2 \\ 
  Benin & ATACORA & 2014 & 120.87 & 31.55 & 365.12 & RW2 \\ 
  Benin & ATACORA & 2015 & 117.02 & 23.49 & 418.85 & RW2 \\ 
  Benin & ATACORA & 2016 & 112.84 & 17.12 & 477.36 & RW2 \\ 
  Benin & ATACORA & 2017 & 109.60 & 12.40 & 544.09 & RW2 \\ 
  Benin & ATACORA & 2018 & 105.45 & 8.67 & 613.24 & RW2 \\ 
  Benin & ATACORA & 2019 & 102.19 & 6.03 & 684.48 & RW2 \\ 
  Benin & ATLANTIQUE & 1980 & 178.33 & 136.08 & 230.88 & RW2 \\ 
  Benin & ATLANTIQUE & 1981 & 174.97 & 144.88 & 209.92 & RW2 \\ 
  Benin & ATLANTIQUE & 1982 & 171.69 & 144.53 & 202.13 & RW2 \\ 
  Benin & ATLANTIQUE & 1983 & 168.22 & 139.24 & 202.00 & RW2 \\ 
  Benin & ATLANTIQUE & 1984 & 164.90 & 134.63 & 200.06 & RW2 \\ 
  Benin & ATLANTIQUE & 1985 & 161.51 & 133.60 & 193.11 & RW2 \\ 
  Benin & ATLANTIQUE & 1986 & 157.74 & 133.32 & 185.94 & RW2 \\ 
  Benin & ATLANTIQUE & 1987 & 153.92 & 131.37 & 179.56 & RW2 \\ 
  Benin & ATLANTIQUE & 1988 & 149.89 & 126.86 & 176.33 & RW2 \\ 
  Benin & ATLANTIQUE & 1989 & 145.70 & 121.69 & 174.04 & RW2 \\ 
  Benin & ATLANTIQUE & 1990 & 141.49 & 118.06 & 168.60 & RW2 \\ 
  Benin & ATLANTIQUE & 1991 & 137.72 & 116.45 & 161.46 & RW2 \\ 
  Benin & ATLANTIQUE & 1992 & 134.05 & 114.27 & 156.41 & RW2 \\ 
  Benin & ATLANTIQUE & 1993 & 130.82 & 111.02 & 154.01 & RW2 \\ 
  Benin & ATLANTIQUE & 1994 & 128.02 & 106.86 & 152.88 & RW2 \\ 
  Benin & ATLANTIQUE & 1995 & 125.45 & 104.65 & 150.07 & RW2 \\ 
  Benin & ATLANTIQUE & 1996 & 123.15 & 104.24 & 144.74 & RW2 \\ 
  Benin & ATLANTIQUE & 1997 & 121.07 & 103.23 & 141.89 & RW2 \\ 
  Benin & ATLANTIQUE & 1998 & 119.20 & 100.75 & 140.57 & RW2 \\ 
  Benin & ATLANTIQUE & 1999 & 117.24 & 97.44 & 140.12 & RW2 \\ 
  Benin & ATLANTIQUE & 2000 & 115.38 & 95.69 & 138.88 & RW2 \\ 
  Benin & ATLANTIQUE & 2001 & 113.40 & 94.96 & 135.01 & RW2 \\ 
  Benin & ATLANTIQUE & 2002 & 111.19 & 93.70 & 131.71 & RW2 \\ 
  Benin & ATLANTIQUE & 2003 & 108.80 & 90.44 & 130.41 & RW2 \\ 
  Benin & ATLANTIQUE & 2004 & 106.29 & 85.99 & 130.27 & RW2 \\ 
  Benin & ATLANTIQUE & 2005 & 103.53 & 81.62 & 130.25 & RW2 \\ 
  Benin & ATLANTIQUE & 2006 & 100.74 & 79.68 & 127.07 & RW2 \\ 
  Benin & ATLANTIQUE & 2007 & 98.11 & 77.35 & 124.29 & RW2 \\ 
  Benin & ATLANTIQUE & 2008 & 95.50 & 72.41 & 125.40 & RW2 \\ 
  Benin & ATLANTIQUE & 2009 & 92.77 & 64.29 & 132.31 & RW2 \\ 
  Benin & ATLANTIQUE & 2010 & 90.34 & 53.42 & 148.82 & RW2 \\ 
  Benin & ATLANTIQUE & 2011 & 87.85 & 43.80 & 168.91 & RW2 \\ 
  Benin & ATLANTIQUE & 2012 & 85.41 & 34.73 & 195.04 & RW2 \\ 
  Benin & ATLANTIQUE & 2013 & 83.07 & 27.01 & 227.68 & RW2 \\ 
  Benin & ATLANTIQUE & 2014 & 80.48 & 20.12 & 266.72 & RW2 \\ 
  Benin & ATLANTIQUE & 2015 & 78.58 & 15.19 & 317.32 & RW2 \\ 
  Benin & ATLANTIQUE & 2016 & 76.16 & 10.99 & 372.91 & RW2 \\ 
  Benin & ATLANTIQUE & 2017 & 74.38 & 8.01 & 439.20 & RW2 \\ 
  Benin & ATLANTIQUE & 2018 & 71.87 & 5.73 & 506.42 & RW2 \\ 
  Benin & ATLANTIQUE & 2019 & 70.30 & 4.10 & 580.09 & RW2 \\ 
  Benin & BORGOU & 1980 & 211.49 & 162.17 & 270.50 & RW2 \\ 
  Benin & BORGOU & 1981 & 209.36 & 174.48 & 250.00 & RW2 \\ 
  Benin & BORGOU & 1982 & 207.22 & 175.34 & 243.09 & RW2 \\ 
  Benin & BORGOU & 1983 & 204.97 & 170.31 & 244.35 & RW2 \\ 
  Benin & BORGOU & 1984 & 202.70 & 166.53 & 244.01 & RW2 \\ 
  Benin & BORGOU & 1985 & 200.48 & 167.63 & 238.36 & RW2 \\ 
  Benin & BORGOU & 1986 & 197.98 & 168.32 & 231.25 & RW2 \\ 
  Benin & BORGOU & 1987 & 195.21 & 167.50 & 226.16 & RW2 \\ 
  Benin & BORGOU & 1988 & 192.20 & 163.72 & 224.61 & RW2 \\ 
  Benin & BORGOU & 1989 & 189.18 & 159.41 & 223.77 & RW2 \\ 
  Benin & BORGOU & 1990 & 185.67 & 156.47 & 219.22 & RW2 \\ 
  Benin & BORGOU & 1991 & 182.66 & 155.47 & 212.74 & RW2 \\ 
  Benin & BORGOU & 1992 & 179.63 & 154.10 & 208.30 & RW2 \\ 
  Benin & BORGOU & 1993 & 176.65 & 150.60 & 206.57 & RW2 \\ 
  Benin & BORGOU & 1994 & 173.93 & 146.29 & 206.11 & RW2 \\ 
  Benin & BORGOU & 1995 & 171.40 & 144.38 & 202.75 & RW2 \\ 
  Benin & BORGOU & 1996 & 168.66 & 143.50 & 197.27 & RW2 \\ 
  Benin & BORGOU & 1997 & 166.10 & 142.60 & 193.14 & RW2 \\ 
  Benin & BORGOU & 1998 & 163.38 & 139.01 & 191.56 & RW2 \\ 
  Benin & BORGOU & 1999 & 160.58 & 134.02 & 190.73 & RW2 \\ 
  Benin & BORGOU & 2000 & 157.58 & 130.77 & 188.74 & RW2 \\ 
  Benin & BORGOU & 2001 & 154.28 & 129.29 & 183.30 & RW2 \\ 
  Benin & BORGOU & 2002 & 150.77 & 126.53 & 178.26 & RW2 \\ 
  Benin & BORGOU & 2003 & 146.88 & 121.52 & 176.38 & RW2 \\ 
  Benin & BORGOU & 2004 & 142.94 & 115.26 & 176.05 & RW2 \\ 
  Benin & BORGOU & 2005 & 138.84 & 108.84 & 174.76 & RW2 \\ 
  Benin & BORGOU & 2006 & 134.65 & 105.01 & 169.91 & RW2 \\ 
  Benin & BORGOU & 2007 & 130.45 & 100.82 & 165.70 & RW2 \\ 
  Benin & BORGOU & 2008 & 126.30 & 93.69 & 166.15 & RW2 \\ 
  Benin & BORGOU & 2009 & 122.37 & 83.34 & 174.20 & RW2 \\ 
  Benin & BORGOU & 2010 & 118.45 & 69.48 & 192.87 & RW2 \\ 
  Benin & BORGOU & 2011 & 114.56 & 56.53 & 216.45 & RW2 \\ 
  Benin & BORGOU & 2012 & 111.11 & 44.70 & 247.09 & RW2 \\ 
  Benin & BORGOU & 2013 & 107.54 & 35.42 & 284.68 & RW2 \\ 
  Benin & BORGOU & 2014 & 103.84 & 26.64 & 328.24 & RW2 \\ 
  Benin & BORGOU & 2015 & 100.47 & 19.65 & 384.41 & RW2 \\ 
  Benin & BORGOU & 2016 & 97.00 & 14.03 & 441.38 & RW2 \\ 
  Benin & BORGOU & 2017 & 94.10 & 10.22 & 506.99 & RW2 \\ 
  Benin & BORGOU & 2018 & 90.41 & 7.17 & 572.40 & RW2 \\ 
  Benin & BORGOU & 2019 & 87.99 & 4.95 & 649.75 & RW2 \\ 
  Benin & MONO & 1980 & 208.27 & 159.34 & 267.93 & RW2 \\ 
  Benin & MONO & 1981 & 204.30 & 169.07 & 245.02 & RW2 \\ 
  Benin & MONO & 1982 & 200.42 & 168.87 & 235.91 & RW2 \\ 
  Benin & MONO & 1983 & 196.31 & 162.76 & 234.50 & RW2 \\ 
  Benin & MONO & 1984 & 192.25 & 157.37 & 232.95 & RW2 \\ 
  Benin & MONO & 1985 & 188.22 & 156.63 & 224.35 & RW2 \\ 
  Benin & MONO & 1986 & 183.79 & 156.15 & 215.21 & RW2 \\ 
  Benin & MONO & 1987 & 179.10 & 153.20 & 208.22 & RW2 \\ 
  Benin & MONO & 1988 & 174.23 & 148.04 & 203.99 & RW2 \\ 
  Benin & MONO & 1989 & 169.16 & 141.79 & 200.58 & RW2 \\ 
  Benin & MONO & 1990 & 163.93 & 137.26 & 193.70 & RW2 \\ 
  Benin & MONO & 1991 & 158.96 & 134.85 & 186.14 & RW2 \\ 
  Benin & MONO & 1992 & 154.13 & 131.99 & 179.82 & RW2 \\ 
  Benin & MONO & 1993 & 149.70 & 127.35 & 175.91 & RW2 \\ 
  Benin & MONO & 1994 & 145.40 & 121.27 & 172.75 & RW2 \\ 
  Benin & MONO & 1995 & 141.61 & 118.31 & 168.60 & RW2 \\ 
  Benin & MONO & 1996 & 137.95 & 116.74 & 162.74 & RW2 \\ 
  Benin & MONO & 1997 & 134.33 & 114.50 & 157.20 & RW2 \\ 
  Benin & MONO & 1998 & 131.01 & 110.46 & 154.49 & RW2 \\ 
  Benin & MONO & 1999 & 127.53 & 105.69 & 152.39 & RW2 \\ 
  Benin & MONO & 2000 & 124.15 & 102.75 & 149.17 & RW2 \\ 
  Benin & MONO & 2001 & 120.56 & 100.72 & 143.87 & RW2 \\ 
  Benin & MONO & 2002 & 117.01 & 98.12 & 138.93 & RW2 \\ 
  Benin & MONO & 2003 & 113.11 & 93.31 & 136.04 & RW2 \\ 
  Benin & MONO & 2004 & 109.29 & 87.75 & 135.08 & RW2 \\ 
  Benin & MONO & 2005 & 105.30 & 82.26 & 133.58 & RW2 \\ 
  Benin & MONO & 2006 & 101.31 & 78.67 & 129.12 & RW2 \\ 
  Benin & MONO & 2007 & 97.40 & 74.89 & 124.94 & RW2 \\ 
  Benin & MONO & 2008 & 93.81 & 69.25 & 124.92 & RW2 \\ 
  Benin & MONO & 2009 & 90.14 & 60.68 & 131.16 & RW2 \\ 
  Benin & MONO & 2010 & 86.73 & 50.38 & 145.48 & RW2 \\ 
  Benin & MONO & 2011 & 83.04 & 40.29 & 163.12 & RW2 \\ 
  Benin & MONO & 2012 & 79.78 & 31.69 & 185.76 & RW2 \\ 
  Benin & MONO & 2013 & 76.69 & 24.31 & 216.07 & RW2 \\ 
  Benin & MONO & 2014 & 73.81 & 18.19 & 252.40 & RW2 \\ 
  Benin & MONO & 2015 & 70.78 & 13.53 & 298.68 & RW2 \\ 
  Benin & MONO & 2016 & 68.05 & 9.58 & 346.73 & RW2 \\ 
  Benin & MONO & 2017 & 65.21 & 6.85 & 407.30 & RW2 \\ 
  Benin & MONO & 2018 & 62.61 & 4.90 & 477.75 & RW2 \\ 
  Benin & MONO & 2019 & 60.26 & 3.24 & 544.36 & RW2 \\ 
  Benin & OUEME & 1980 & 211.11 & 161.53 & 271.36 & RW2 \\ 
  Benin & OUEME & 1981 & 208.01 & 172.49 & 249.17 & RW2 \\ 
  Benin & OUEME & 1982 & 204.77 & 173.01 & 241.09 & RW2 \\ 
  Benin & OUEME & 1983 & 201.91 & 167.56 & 240.72 & RW2 \\ 
  Benin & OUEME & 1984 & 198.57 & 162.50 & 239.29 & RW2 \\ 
  Benin & OUEME & 1985 & 195.45 & 162.90 & 232.09 & RW2 \\ 
  Benin & OUEME & 1986 & 191.80 & 163.02 & 223.69 & RW2 \\ 
  Benin & OUEME & 1987 & 187.96 & 161.18 & 217.93 & RW2 \\ 
  Benin & OUEME & 1988 & 183.84 & 156.78 & 214.07 & RW2 \\ 
  Benin & OUEME & 1989 & 179.67 & 151.40 & 212.60 & RW2 \\ 
  Benin & OUEME & 1990 & 175.34 & 147.63 & 206.95 & RW2 \\ 
  Benin & OUEME & 1991 & 171.23 & 146.14 & 199.72 & RW2 \\ 
  Benin & OUEME & 1992 & 167.41 & 143.77 & 193.88 & RW2 \\ 
  Benin & OUEME & 1993 & 164.04 & 139.87 & 191.45 & RW2 \\ 
  Benin & OUEME & 1994 & 160.94 & 135.12 & 190.60 & RW2 \\ 
  Benin & OUEME & 1995 & 158.42 & 132.96 & 187.63 & RW2 \\ 
  Benin & OUEME & 1996 & 155.99 & 132.56 & 182.86 & RW2 \\ 
  Benin & OUEME & 1997 & 153.81 & 131.39 & 179.23 & RW2 \\ 
  Benin & OUEME & 1998 & 151.76 & 128.87 & 177.93 & RW2 \\ 
  Benin & OUEME & 1999 & 149.73 & 124.78 & 178.14 & RW2 \\ 
  Benin & OUEME & 2000 & 147.97 & 123.22 & 176.79 & RW2 \\ 
  Benin & OUEME & 2001 & 145.94 & 122.79 & 172.82 & RW2 \\ 
  Benin & OUEME & 2002 & 143.76 & 121.67 & 169.46 & RW2 \\ 
  Benin & OUEME & 2003 & 141.33 & 118.02 & 168.36 & RW2 \\ 
  Benin & OUEME & 2004 & 138.72 & 113.35 & 169.07 & RW2 \\ 
  Benin & OUEME & 2005 & 135.97 & 108.26 & 168.97 & RW2 \\ 
  Benin & OUEME & 2006 & 133.02 & 106.02 & 165.91 & RW2 \\ 
  Benin & OUEME & 2007 & 130.41 & 103.59 & 162.89 & RW2 \\ 
  Benin & OUEME & 2008 & 127.60 & 97.88 & 165.80 & RW2 \\ 
  Benin & OUEME & 2009 & 124.71 & 87.43 & 175.71 & RW2 \\ 
  Benin & OUEME & 2010 & 121.91 & 73.20 & 196.97 & RW2 \\ 
  Benin & OUEME & 2011 & 119.57 & 60.45 & 222.02 & RW2 \\ 
  Benin & OUEME & 2012 & 116.96 & 48.30 & 255.62 & RW2 \\ 
  Benin & OUEME & 2013 & 113.67 & 37.62 & 293.17 & RW2 \\ 
  Benin & OUEME & 2014 & 111.65 & 29.12 & 346.33 & RW2 \\ 
  Benin & OUEME & 2015 & 108.99 & 21.92 & 396.99 & RW2 \\ 
  Benin & OUEME & 2016 & 106.74 & 16.00 & 468.85 & RW2 \\ 
  Benin & OUEME & 2017 & 104.95 & 11.71 & 535.12 & RW2 \\ 
  Benin & OUEME & 2018 & 102.68 & 8.38 & 603.73 & RW2 \\ 
  Benin & OUEME & 2019 & 99.51 & 5.75 & 680.64 & RW2 \\ 
  Benin & ZOU & 1980 & 233.86 & 181.09 & 296.27 & RW2 \\ 
  Benin & ZOU & 1981 & 229.10 & 192.24 & 270.56 & RW2 \\ 
  Benin & ZOU & 1982 & 224.23 & 190.90 & 261.26 & RW2 \\ 
  Benin & ZOU & 1983 & 219.28 & 183.18 & 260.06 & RW2 \\ 
  Benin & ZOU & 1984 & 214.35 & 176.97 & 256.68 & RW2 \\ 
  Benin & ZOU & 1985 & 209.46 & 175.23 & 247.78 & RW2 \\ 
  Benin & ZOU & 1986 & 204.35 & 174.09 & 237.73 & RW2 \\ 
  Benin & ZOU & 1987 & 199.03 & 171.45 & 229.54 & RW2 \\ 
  Benin & ZOU & 1988 & 193.53 & 165.18 & 225.18 & RW2 \\ 
  Benin & ZOU & 1989 & 187.96 & 158.46 & 221.54 & RW2 \\ 
  Benin & ZOU & 1990 & 182.22 & 153.77 & 214.50 & RW2 \\ 
  Benin & ZOU & 1991 & 176.97 & 151.35 & 206.07 & RW2 \\ 
  Benin & ZOU & 1992 & 171.97 & 147.81 & 199.12 & RW2 \\ 
  Benin & ZOU & 1993 & 167.27 & 143.04 & 194.69 & RW2 \\ 
  Benin & ZOU & 1994 & 162.92 & 137.17 & 192.51 & RW2 \\ 
  Benin & ZOU & 1995 & 159.01 & 133.82 & 187.88 & RW2 \\ 
  Benin & ZOU & 1996 & 155.37 & 132.23 & 181.57 & RW2 \\ 
  Benin & ZOU & 1997 & 151.94 & 130.28 & 176.23 & RW2 \\ 
  Benin & ZOU & 1998 & 148.66 & 126.63 & 174.15 & RW2 \\ 
  Benin & ZOU & 1999 & 145.47 & 121.74 & 173.01 & RW2 \\ 
  Benin & ZOU & 2000 & 142.31 & 118.37 & 169.60 & RW2 \\ 
  Benin & ZOU & 2001 & 139.01 & 117.35 & 163.91 & RW2 \\ 
  Benin & ZOU & 2002 & 135.56 & 115.01 & 159.17 & RW2 \\ 
  Benin & ZOU & 2003 & 131.79 & 110.51 & 156.66 & RW2 \\ 
  Benin & ZOU & 2004 & 127.95 & 104.61 & 155.58 & RW2 \\ 
  Benin & ZOU & 2005 & 123.95 & 98.38 & 154.86 & RW2 \\ 
  Benin & ZOU & 2006 & 119.93 & 95.11 & 150.15 & RW2 \\ 
  Benin & ZOU & 2007 & 115.91 & 91.63 & 145.59 & RW2 \\ 
  Benin & ZOU & 2008 & 112.07 & 85.50 & 145.68 & RW2 \\ 
  Benin & ZOU & 2009 & 108.14 & 75.71 & 152.47 & RW2 \\ 
  Benin & ZOU & 2010 & 104.69 & 62.27 & 169.96 & RW2 \\ 
  Benin & ZOU & 2011 & 100.80 & 50.77 & 190.53 & RW2 \\ 
  Benin & ZOU & 2012 & 97.54 & 40.05 & 217.37 & RW2 \\ 
  Benin & ZOU & 2013 & 93.94 & 30.62 & 252.25 & RW2 \\ 
  Benin & ZOU & 2014 & 90.72 & 23.03 & 289.72 & RW2 \\ 
  Benin & ZOU & 2015 & 87.66 & 17.27 & 345.17 & RW2 \\ 
  Benin & ZOU & 2016 & 84.52 & 12.70 & 401.53 & RW2 \\ 
  Benin & ZOU & 2017 & 81.71 & 9.01 & 469.82 & RW2 \\ 
  Benin & ZOU & 2018 & 78.35 & 6.11 & 536.34 & RW2 \\ 
  Benin & ZOU & 2019 & 75.43 & 4.37 & 603.78 & RW2 \\ 
  Burkina Faso & ALL & 1980 & 254.55 & 247.90 & 261.61 & IHME \\ 
  Burkina Faso & ALL & 1980 & 243.08 & 183.37 & 313.56 & RW2 \\ 
  Burkina Faso & ALL & 1980 & 241.20 & 225.20 & 258.50 & UN \\ 
  Burkina Faso & ALL & 1981 & 248.24 & 241.57 & 254.92 & IHME \\ 
  Burkina Faso & ALL & 1981 & 237.43 & 195.11 & 284.41 & RW2 \\ 
  Burkina Faso & ALL & 1981 & 236.10 & 220.60 & 252.70 & UN \\ 
  Burkina Faso & ALL & 1982 & 242.20 & 235.74 & 248.59 & IHME \\ 
  Burkina Faso & ALL & 1982 & 231.88 & 193.90 & 274.87 & RW2 \\ 
  Burkina Faso & ALL & 1982 & 231.80 & 216.50 & 247.90 & UN \\ 
  Burkina Faso & ALL & 1983 & 236.96 & 230.65 & 243.04 & IHME \\ 
  Burkina Faso & ALL & 1983 & 226.29 & 184.82 & 274.51 & RW2 \\ 
  Burkina Faso & ALL & 1983 & 227.80 & 212.90 & 243.40 & UN \\ 
  Burkina Faso & ALL & 1984 & 232.18 & 226.11 & 238.02 & IHME \\ 
  Burkina Faso & ALL & 1984 & 221.45 & 176.60 & 274.27 & RW2 \\ 
  Burkina Faso & ALL & 1984 & 223.70 & 209.00 & 238.90 & UN \\ 
  Burkina Faso & ALL & 1985 & 227.01 & 221.24 & 232.86 & IHME \\ 
  Burkina Faso & ALL & 1985 & 216.27 & 175.78 & 263.05 & RW2 \\ 
  Burkina Faso & ALL & 1985 & 219.20 & 205.20 & 234.10 & UN \\ 
  Burkina Faso & ALL & 1986 & 221.98 & 216.28 & 227.82 & IHME \\ 
  Burkina Faso & ALL & 1986 & 212.36 & 175.69 & 254.28 & RW2 \\ 
  Burkina Faso & ALL & 1986 & 214.40 & 200.80 & 228.60 & UN \\ 
  Burkina Faso & ALL & 1987 & 217.78 & 212.35 & 223.53 & IHME \\ 
  Burkina Faso & ALL & 1987 & 209.28 & 174.96 & 249.08 & RW2 \\ 
  Burkina Faso & ALL & 1987 & 209.60 & 196.40 & 223.50 & UN \\ 
  Burkina Faso & ALL & 1988 & 213.85 & 208.59 & 219.25 & IHME \\ 
  Burkina Faso & ALL & 1988 & 206.82 & 170.74 & 247.96 & RW2 \\ 
  Burkina Faso & ALL & 1988 & 205.80 & 192.70 & 219.30 & UN \\ 
  Burkina Faso & ALL & 1989 & 210.51 & 205.31 & 215.92 & IHME \\ 
  Burkina Faso & ALL & 1989 & 205.04 & 166.96 & 248.77 & RW2 \\ 
  Burkina Faso & ALL & 1989 & 203.40 & 190.50 & 216.60 & UN \\ 
  Burkina Faso & ALL & 1990 & 207.73 & 202.61 & 213.20 & IHME \\ 
  Burkina Faso & ALL & 1990 & 204.10 & 166.81 & 248.00 & RW2 \\ 
  Burkina Faso & ALL & 1990 & 202.20 & 189.40 & 215.10 & UN \\ 
  Burkina Faso & ALL & 1991 & 204.88 & 199.72 & 210.36 & IHME \\ 
  Burkina Faso & ALL & 1991 & 203.00 & 168.02 & 242.62 & RW2 \\ 
  Burkina Faso & ALL & 1991 & 201.80 & 189.30 & 215.00 & UN \\ 
  Burkina Faso & ALL & 1992 & 202.45 & 197.27 & 207.88 & IHME \\ 
  Burkina Faso & ALL & 1992 & 201.88 & 167.97 & 240.03 & RW2 \\ 
  Burkina Faso & ALL & 1992 & 202.10 & 189.40 & 215.30 & UN \\ 
  Burkina Faso & ALL & 1993 & 200.11 & 194.99 & 205.50 & IHME \\ 
  Burkina Faso & ALL & 1993 & 200.66 & 166.05 & 240.38 & RW2 \\ 
  Burkina Faso & ALL & 1993 & 202.00 & 189.10 & 215.50 & UN \\ 
  Burkina Faso & ALL & 1994 & 198.03 & 192.96 & 203.64 & IHME \\ 
  Burkina Faso & ALL & 1994 & 199.17 & 162.53 & 242.67 & RW2 \\ 
  Burkina Faso & ALL & 1994 & 201.10 & 188.20 & 214.60 & UN \\ 
  Burkina Faso & ALL & 1995 & 195.71 & 190.49 & 201.18 & IHME \\ 
  Burkina Faso & ALL & 1995 & 197.38 & 161.15 & 239.54 & RW2 \\ 
  Burkina Faso & ALL & 1995 & 199.40 & 186.70 & 212.80 & UN \\ 
  Burkina Faso & ALL & 1996 & 192.73 & 187.45 & 198.05 & IHME \\ 
  Burkina Faso & ALL & 1996 & 195.64 & 161.80 & 235.57 & RW2 \\ 
  Burkina Faso & ALL & 1996 & 197.00 & 184.20 & 210.30 & UN \\ 
  Burkina Faso & ALL & 1997 & 189.38 & 184.26 & 194.96 & IHME \\ 
  Burkina Faso & ALL & 1997 & 193.86 & 161.43 & 231.27 & RW2 \\ 
  Burkina Faso & ALL & 1997 & 194.00 & 181.10 & 207.50 & UN \\ 
  Burkina Faso & ALL & 1998 & 185.65 & 180.29 & 191.19 & IHME \\ 
  Burkina Faso & ALL & 1998 & 191.99 & 158.47 & 231.32 & RW2 \\ 
  Burkina Faso & ALL & 1998 & 191.30 & 178.00 & 204.80 & UN \\ 
  Burkina Faso & ALL & 1999 & 181.60 & 176.12 & 187.31 & IHME \\ 
  Burkina Faso & ALL & 1999 & 189.67 & 153.88 & 230.05 & RW2 \\ 
  Burkina Faso & ALL & 1999 & 188.50 & 175.20 & 202.30 & UN \\ 
  Burkina Faso & ALL & 2000 & 177.59 & 171.86 & 183.32 & IHME \\ 
  Burkina Faso & ALL & 2000 & 187.18 & 152.28 & 228.43 & RW2 \\ 
  Burkina Faso & ALL & 2000 & 185.70 & 172.20 & 199.90 & UN \\ 
  Burkina Faso & ALL & 2001 & 173.31 & 167.63 & 179.09 & IHME \\ 
  Burkina Faso & ALL & 2001 & 183.24 & 150.46 & 221.36 & RW2 \\ 
  Burkina Faso & ALL & 2001 & 182.40 & 168.70 & 196.90 & UN \\ 
  Burkina Faso & ALL & 2002 & 168.85 & 163.03 & 174.32 & IHME \\ 
  Burkina Faso & ALL & 2002 & 178.10 & 147.41 & 213.57 & RW2 \\ 
  Burkina Faso & ALL & 2002 & 178.30 & 164.30 & 193.20 & UN \\ 
  Burkina Faso & ALL & 2003 & 164.11 & 158.27 & 169.71 & IHME \\ 
  Burkina Faso & ALL & 2003 & 171.90 & 141.34 & 207.60 & RW2 \\ 
  Burkina Faso & ALL & 2003 & 173.10 & 159.00 & 188.10 & UN \\ 
  Burkina Faso & ALL & 2004 & 159.32 & 153.10 & 165.32 & IHME \\ 
  Burkina Faso & ALL & 2004 & 164.44 & 131.92 & 203.14 & RW2 \\ 
  Burkina Faso & ALL & 2004 & 166.30 & 152.40 & 181.10 & UN \\ 
  Burkina Faso & ALL & 2005 & 154.28 & 147.87 & 160.50 & IHME \\ 
  Burkina Faso & ALL & 2005 & 156.10 & 121.87 & 196.28 & RW2 \\ 
  Burkina Faso & ALL & 2005 & 158.30 & 144.90 & 172.90 & UN \\ 
  Burkina Faso & ALL & 2006 & 149.25 & 142.44 & 155.85 & IHME \\ 
  Burkina Faso & ALL & 2006 & 147.82 & 117.37 & 183.33 & RW2 \\ 
  Burkina Faso & ALL & 2006 & 149.40 & 136.50 & 163.40 & UN \\ 
  Burkina Faso & ALL & 2007 & 144.32 & 137.05 & 151.37 & IHME \\ 
  Burkina Faso & ALL & 2007 & 139.63 & 113.88 & 169.77 & RW2 \\ 
  Burkina Faso & ALL & 2007 & 139.90 & 127.20 & 153.40 & UN \\ 
  Burkina Faso & ALL & 2008 & 139.55 & 131.99 & 147.18 & IHME \\ 
  Burkina Faso & ALL & 2008 & 131.79 & 105.98 & 163.49 & RW2 \\ 
  Burkina Faso & ALL & 2008 & 130.40 & 117.90 & 144.30 & UN \\ 
  Burkina Faso & ALL & 2009 & 134.99 & 127.18 & 142.96 & IHME \\ 
  Burkina Faso & ALL & 2009 & 124.07 & 90.29 & 169.30 & RW2 \\ 
  Burkina Faso & ALL & 2009 & 121.40 & 108.30 & 136.10 & UN \\ 
  Burkina Faso & ALL & 2010 & 130.57 & 122.28 & 139.57 & IHME \\ 
  Burkina Faso & ALL & 2010 & 116.91 & 70.45 & 190.46 & RW2 \\ 
  Burkina Faso & ALL & 2010 & 113.50 & 98.90 & 129.70 & UN \\ 
  Burkina Faso & ALL & 2011 & 126.42 & 117.84 & 136.33 & IHME \\ 
  Burkina Faso & ALL & 2011 & 110.19 & 53.65 & 213.17 & RW2 \\ 
  Burkina Faso & ALL & 2011 & 106.90 & 90.50 & 124.80 & UN \\ 
  Burkina Faso & ALL & 2012 & 122.35 & 113.11 & 132.86 & IHME \\ 
  Burkina Faso & ALL & 2012 & 103.61 & 39.43 & 244.32 & RW2 \\ 
  Burkina Faso & ALL & 2012 & 101.40 & 82.80 & 121.90 & UN \\ 
  Burkina Faso & ALL & 2013 & 118.34 & 108.63 & 129.85 & IHME \\ 
  Burkina Faso & ALL & 2013 & 97.75 & 28.60 & 281.67 & RW2 \\ 
  Burkina Faso & ALL & 2013 & 96.60 & 76.20 & 120.10 & UN \\ 
  Burkina Faso & ALL & 2014 & 114.25 & 104.03 & 126.19 & IHME \\ 
  Burkina Faso & ALL & 2014 & 91.83 & 20.17 & 327.98 & RW2 \\ 
  Burkina Faso & ALL & 2014 & 92.40 & 70.20 & 119.60 & UN \\ 
  Burkina Faso & ALL & 2015 & 110.38 & 100.10 & 123.03 & IHME \\ 
  Burkina Faso & ALL & 2015 & 86.01 & 14.14 & 378.47 & RW2 \\ 
  Burkina Faso & ALL & 2015 & 88.60 & 64.90 & 119.10 & UN \\ 
  Burkina Faso & ALL & 2016 & 81.38 & 9.78 & 444.22 & RW2 \\ 
  Burkina Faso & ALL & 2017 & 76.01 & 6.43 & 512.31 & RW2 \\ 
  Burkina Faso & ALL & 2018 & 71.36 & 4.32 & 588.53 & RW2 \\ 
  Burkina Faso & ALL & 2019 & 66.78 & 2.58 & 650.17 & RW2 \\ 
  Burkina Faso & ALL & 80-84 & 237.59 & 247.59 & 227.87 & HT-Direct \\ 
  Burkina Faso & ALL & 85-89 & 206.30 & 214.07 & 198.74 & HT-Direct \\ 
  Burkina Faso & ALL & 90-94 & 202.29 & 208.95 & 195.79 & HT-Direct \\ 
  Burkina Faso & ALL & 95-99 & 194.95 & 201.74 & 188.34 & HT-Direct \\ 
  Burkina Faso & ALL & 00-04 & 175.49 & 183.15 & 168.08 & HT-Direct \\ 
  Burkina Faso & ALL & 05-09 & 132.49 & 140.13 & 125.21 & HT-Direct \\ 
  Burkina Faso & ALL & 15-19 & 76.03 & 6.49 & 506.97 & RW2 \\ 
  Burkina Faso & CENTRAL/SOUTH & 1980 & 223.95 & 174.76 & 282.54 & RW2 \\ 
  Burkina Faso & CENTRAL/SOUTH & 1981 & 220.23 & 186.39 & 257.77 & RW2 \\ 
  Burkina Faso & CENTRAL/SOUTH & 1982 & 216.55 & 186.18 & 249.65 & RW2 \\ 
  Burkina Faso & CENTRAL/SOUTH & 1983 & 212.84 & 179.64 & 251.09 & RW2 \\ 
  Burkina Faso & CENTRAL/SOUTH & 1984 & 209.63 & 174.43 & 250.34 & RW2 \\ 
  Burkina Faso & CENTRAL/SOUTH & 1985 & 206.48 & 173.57 & 242.58 & RW2 \\ 
  Burkina Faso & CENTRAL/SOUTH & 1986 & 204.13 & 175.41 & 236.58 & RW2 \\ 
  Burkina Faso & CENTRAL/SOUTH & 1987 & 202.65 & 175.80 & 232.54 & RW2 \\ 
  Burkina Faso & CENTRAL/SOUTH & 1988 & 201.79 & 173.39 & 233.36 & RW2 \\ 
  Burkina Faso & CENTRAL/SOUTH & 1989 & 201.39 & 170.60 & 236.14 & RW2 \\ 
  Burkina Faso & CENTRAL/SOUTH & 1990 & 201.71 & 171.33 & 236.52 & RW2 \\ 
  Burkina Faso & CENTRAL/SOUTH & 1991 & 201.87 & 173.67 & 232.75 & RW2 \\ 
  Burkina Faso & CENTRAL/SOUTH & 1992 & 201.57 & 174.72 & 231.10 & RW2 \\ 
  Burkina Faso & CENTRAL/SOUTH & 1993 & 201.03 & 173.50 & 232.43 & RW2 \\ 
  Burkina Faso & CENTRAL/SOUTH & 1994 & 200.21 & 170.09 & 234.71 & RW2 \\ 
  Burkina Faso & CENTRAL/SOUTH & 1995 & 198.60 & 168.36 & 233.01 & RW2 \\ 
  Burkina Faso & CENTRAL/SOUTH & 1996 & 197.16 & 169.54 & 227.67 & RW2 \\ 
  Burkina Faso & CENTRAL/SOUTH & 1997 & 195.59 & 169.47 & 225.29 & RW2 \\ 
  Burkina Faso & CENTRAL/SOUTH & 1998 & 193.96 & 166.67 & 224.62 & RW2 \\ 
  Burkina Faso & CENTRAL/SOUTH & 1999 & 191.77 & 162.15 & 224.64 & RW2 \\ 
  Burkina Faso & CENTRAL/SOUTH & 2000 & 189.41 & 160.27 & 223.39 & RW2 \\ 
  Burkina Faso & CENTRAL/SOUTH & 2001 & 185.73 & 158.77 & 216.44 & RW2 \\ 
  Burkina Faso & CENTRAL/SOUTH & 2002 & 180.80 & 155.67 & 209.25 & RW2 \\ 
  Burkina Faso & CENTRAL/SOUTH & 2003 & 174.66 & 148.67 & 204.11 & RW2 \\ 
  Burkina Faso & CENTRAL/SOUTH & 2004 & 167.55 & 139.34 & 199.66 & RW2 \\ 
  Burkina Faso & CENTRAL/SOUTH & 2005 & 159.30 & 129.56 & 193.23 & RW2 \\ 
  Burkina Faso & CENTRAL/SOUTH & 2006 & 151.09 & 124.81 & 181.68 & RW2 \\ 
  Burkina Faso & CENTRAL/SOUTH & 2007 & 143.05 & 120.18 & 169.64 & RW2 \\ 
  Burkina Faso & CENTRAL/SOUTH & 2008 & 135.17 & 110.97 & 163.93 & RW2 \\ 
  Burkina Faso & CENTRAL/SOUTH & 2009 & 127.46 & 95.48 & 168.12 & RW2 \\ 
  Burkina Faso & CENTRAL/SOUTH & 2010 & 120.44 & 76.45 & 184.79 & RW2 \\ 
  Burkina Faso & CENTRAL/SOUTH & 2011 & 113.58 & 60.47 & 203.66 & RW2 \\ 
  Burkina Faso & CENTRAL/SOUTH & 2012 & 107.05 & 46.22 & 228.11 & RW2 \\ 
  Burkina Faso & CENTRAL/SOUTH & 2013 & 100.88 & 34.64 & 258.25 & RW2 \\ 
  Burkina Faso & CENTRAL/SOUTH & 2014 & 94.70 & 24.85 & 293.64 & RW2 \\ 
  Burkina Faso & CENTRAL/SOUTH & 2015 & 89.51 & 18.05 & 339.45 & RW2 \\ 
  Burkina Faso & CENTRAL/SOUTH & 2016 & 84.01 & 12.56 & 388.70 & RW2 \\ 
  Burkina Faso & CENTRAL/SOUTH & 2017 & 79.39 & 8.80 & 447.50 & RW2 \\ 
  Burkina Faso & CENTRAL/SOUTH & 2018 & 74.23 & 6.06 & 506.46 & RW2 \\ 
  Burkina Faso & CENTRAL/SOUTH & 2019 & 70.23 & 4.16 & 571.98 & RW2 \\ 
  Burkina Faso & EAST & 1980 & 273.38 & 214.87 & 339.74 & RW2 \\ 
  Burkina Faso & EAST & 1981 & 265.54 & 225.26 & 309.19 & RW2 \\ 
  Burkina Faso & EAST & 1982 & 257.73 & 221.78 & 297.28 & RW2 \\ 
  Burkina Faso & EAST & 1983 & 249.88 & 211.45 & 293.65 & RW2 \\ 
  Burkina Faso & EAST & 1984 & 242.89 & 202.75 & 288.64 & RW2 \\ 
  Burkina Faso & EAST & 1985 & 235.99 & 199.39 & 277.06 & RW2 \\ 
  Burkina Faso & EAST & 1986 & 230.27 & 197.96 & 266.20 & RW2 \\ 
  Burkina Faso & EAST & 1987 & 225.56 & 195.14 & 259.21 & RW2 \\ 
  Burkina Faso & EAST & 1988 & 221.59 & 190.16 & 255.80 & RW2 \\ 
  Burkina Faso & EAST & 1989 & 218.46 & 185.32 & 255.13 & RW2 \\ 
  Burkina Faso & EAST & 1990 & 216.03 & 183.66 & 252.09 & RW2 \\ 
  Burkina Faso & EAST & 1991 & 213.46 & 184.30 & 246.03 & RW2 \\ 
  Burkina Faso & EAST & 1992 & 210.92 & 183.17 & 241.47 & RW2 \\ 
  Burkina Faso & EAST & 1993 & 208.01 & 178.93 & 240.48 & RW2 \\ 
  Burkina Faso & EAST & 1994 & 204.95 & 174.03 & 240.03 & RW2 \\ 
  Burkina Faso & EAST & 1995 & 201.23 & 171.05 & 235.34 & RW2 \\ 
  Burkina Faso & EAST & 1996 & 197.87 & 170.02 & 228.73 & RW2 \\ 
  Burkina Faso & EAST & 1997 & 194.39 & 168.28 & 223.71 & RW2 \\ 
  Burkina Faso & EAST & 1998 & 190.86 & 163.67 & 221.26 & RW2 \\ 
  Burkina Faso & EAST & 1999 & 186.80 & 157.09 & 219.81 & RW2 \\ 
  Burkina Faso & EAST & 2000 & 182.90 & 153.85 & 215.86 & RW2 \\ 
  Burkina Faso & EAST & 2001 & 177.63 & 151.11 & 207.60 & RW2 \\ 
  Burkina Faso & EAST & 2002 & 171.50 & 146.85 & 199.46 & RW2 \\ 
  Burkina Faso & EAST & 2003 & 164.25 & 138.65 & 192.99 & RW2 \\ 
  Burkina Faso & EAST & 2004 & 156.29 & 129.05 & 188.32 & RW2 \\ 
  Burkina Faso & EAST & 2005 & 147.33 & 119.06 & 180.30 & RW2 \\ 
  Burkina Faso & EAST & 2006 & 138.70 & 113.05 & 168.32 & RW2 \\ 
  Burkina Faso & EAST & 2007 & 130.21 & 107.94 & 155.98 & RW2 \\ 
  Burkina Faso & EAST & 2008 & 121.91 & 99.03 & 149.85 & RW2 \\ 
  Burkina Faso & EAST & 2009 & 114.16 & 84.51 & 152.45 & RW2 \\ 
  Burkina Faso & EAST & 2010 & 106.86 & 66.73 & 166.59 & RW2 \\ 
  Burkina Faso & EAST & 2011 & 100.01 & 52.52 & 182.81 & RW2 \\ 
  Burkina Faso & EAST & 2012 & 93.51 & 39.87 & 202.38 & RW2 \\ 
  Burkina Faso & EAST & 2013 & 87.32 & 29.50 & 229.34 & RW2 \\ 
  Burkina Faso & EAST & 2014 & 81.50 & 21.35 & 262.14 & RW2 \\ 
  Burkina Faso & EAST & 2015 & 76.28 & 15.07 & 302.38 & RW2 \\ 
  Burkina Faso & EAST & 2016 & 71.11 & 10.55 & 347.90 & RW2 \\ 
  Burkina Faso & EAST & 2017 & 66.14 & 7.38 & 397.13 & RW2 \\ 
  Burkina Faso & EAST & 2018 & 61.78 & 5.06 & 457.30 & RW2 \\ 
  Burkina Faso & EAST & 2019 & 57.65 & 3.32 & 520.44 & RW2 \\ 
  Burkina Faso & NORTH & 1980 & 239.53 & 188.86 & 299.55 & RW2 \\ 
  Burkina Faso & NORTH & 1981 & 231.85 & 197.33 & 270.23 & RW2 \\ 
  Burkina Faso & NORTH & 1982 & 224.37 & 193.82 & 258.64 & RW2 \\ 
  Burkina Faso & NORTH & 1983 & 217.17 & 183.59 & 255.75 & RW2 \\ 
  Burkina Faso & NORTH & 1984 & 210.56 & 175.11 & 251.87 & RW2 \\ 
  Burkina Faso & NORTH & 1985 & 204.03 & 171.90 & 240.16 & RW2 \\ 
  Burkina Faso & NORTH & 1986 & 198.90 & 170.98 & 229.76 & RW2 \\ 
  Burkina Faso & NORTH & 1987 & 194.46 & 168.39 & 223.76 & RW2 \\ 
  Burkina Faso & NORTH & 1988 & 191.12 & 163.61 & 221.22 & RW2 \\ 
  Burkina Faso & NORTH & 1989 & 188.49 & 159.47 & 221.05 & RW2 \\ 
  Burkina Faso & NORTH & 1990 & 186.74 & 158.60 & 218.99 & RW2 \\ 
  Burkina Faso & NORTH & 1991 & 184.88 & 159.15 & 214.25 & RW2 \\ 
  Burkina Faso & NORTH & 1992 & 183.00 & 158.43 & 210.30 & RW2 \\ 
  Burkina Faso & NORTH & 1993 & 180.99 & 155.89 & 209.40 & RW2 \\ 
  Burkina Faso & NORTH & 1994 & 178.80 & 151.60 & 210.27 & RW2 \\ 
  Burkina Faso & NORTH & 1995 & 176.00 & 149.10 & 206.40 & RW2 \\ 
  Burkina Faso & NORTH & 1996 & 173.48 & 149.04 & 201.29 & RW2 \\ 
  Burkina Faso & NORTH & 1997 & 170.75 & 147.81 & 196.64 & RW2 \\ 
  Burkina Faso & NORTH & 1998 & 167.87 & 143.90 & 194.69 & RW2 \\ 
  Burkina Faso & NORTH & 1999 & 164.65 & 138.87 & 194.10 & RW2 \\ 
  Burkina Faso & NORTH & 2000 & 161.10 & 135.53 & 190.76 & RW2 \\ 
  Burkina Faso & NORTH & 2001 & 156.58 & 133.88 & 183.08 & RW2 \\ 
  Burkina Faso & NORTH & 2002 & 151.13 & 129.61 & 175.35 & RW2 \\ 
  Burkina Faso & NORTH & 2003 & 144.66 & 122.57 & 169.42 & RW2 \\ 
  Burkina Faso & NORTH & 2004 & 137.51 & 114.06 & 164.82 & RW2 \\ 
  Burkina Faso & NORTH & 2005 & 129.57 & 105.02 & 157.69 & RW2 \\ 
  Burkina Faso & NORTH & 2006 & 121.86 & 100.08 & 146.75 & RW2 \\ 
  Burkina Faso & NORTH & 2007 & 114.26 & 95.89 & 136.10 & RW2 \\ 
  Burkina Faso & NORTH & 2008 & 107.00 & 87.78 & 129.77 & RW2 \\ 
  Burkina Faso & NORTH & 2009 & 100.04 & 74.81 & 132.81 & RW2 \\ 
  Burkina Faso & NORTH & 2010 & 93.75 & 59.12 & 145.29 & RW2 \\ 
  Burkina Faso & NORTH & 2011 & 87.58 & 46.07 & 160.08 & RW2 \\ 
  Burkina Faso & NORTH & 2012 & 81.94 & 35.36 & 179.13 & RW2 \\ 
  Burkina Faso & NORTH & 2013 & 76.57 & 25.75 & 201.48 & RW2 \\ 
  Burkina Faso & NORTH & 2014 & 71.37 & 18.50 & 234.57 & RW2 \\ 
  Burkina Faso & NORTH & 2015 & 66.79 & 13.27 & 272.02 & RW2 \\ 
  Burkina Faso & NORTH & 2016 & 62.26 & 9.33 & 311.93 & RW2 \\ 
  Burkina Faso & NORTH & 2017 & 57.94 & 6.43 & 361.98 & RW2 \\ 
  Burkina Faso & NORTH & 2018 & 54.10 & 4.33 & 421.85 & RW2 \\ 
  Burkina Faso & NORTH & 2019 & 50.23 & 2.91 & 486.62 & RW2 \\ 
  Burkina Faso & WEST & 1980 & 238.16 & 185.68 & 299.68 & RW2 \\ 
  Burkina Faso & WEST & 1981 & 235.94 & 198.73 & 277.44 & RW2 \\ 
  Burkina Faso & WEST & 1982 & 233.42 & 200.09 & 270.26 & RW2 \\ 
  Burkina Faso & WEST & 1983 & 231.10 & 194.75 & 272.36 & RW2 \\ 
  Burkina Faso & WEST & 1984 & 229.01 & 190.02 & 273.52 & RW2 \\ 
  Burkina Faso & WEST & 1985 & 227.14 & 191.75 & 266.82 & RW2 \\ 
  Burkina Faso & WEST & 1986 & 226.20 & 194.61 & 260.79 & RW2 \\ 
  Burkina Faso & WEST & 1987 & 225.96 & 196.29 & 258.77 & RW2 \\ 
  Burkina Faso & WEST & 1988 & 226.50 & 194.97 & 261.27 & RW2 \\ 
  Burkina Faso & WEST & 1989 & 227.64 & 193.68 & 265.11 & RW2 \\ 
  Burkina Faso & WEST & 1990 & 229.52 & 195.96 & 267.30 & RW2 \\ 
  Burkina Faso & WEST & 1991 & 231.30 & 200.30 & 265.75 & RW2 \\ 
  Burkina Faso & WEST & 1992 & 232.79 & 202.57 & 265.76 & RW2 \\ 
  Burkina Faso & WEST & 1993 & 233.90 & 202.44 & 268.57 & RW2 \\ 
  Burkina Faso & WEST & 1994 & 234.93 & 200.54 & 272.78 & RW2 \\ 
  Burkina Faso & WEST & 1995 & 235.17 & 200.76 & 273.18 & RW2 \\ 
  Burkina Faso & WEST & 1996 & 235.70 & 203.79 & 270.65 & RW2 \\ 
  Burkina Faso & WEST & 1997 & 236.02 & 205.69 & 269.71 & RW2 \\ 
  Burkina Faso & WEST & 1998 & 236.16 & 204.87 & 270.88 & RW2 \\ 
  Burkina Faso & WEST & 1999 & 235.89 & 200.62 & 273.54 & RW2 \\ 
  Burkina Faso & WEST & 2000 & 235.24 & 200.14 & 274.33 & RW2 \\ 
  Burkina Faso & WEST & 2001 & 233.11 & 200.47 & 268.89 & RW2 \\ 
  Burkina Faso & WEST & 2002 & 229.49 & 198.84 & 263.17 & RW2 \\ 
  Burkina Faso & WEST & 2003 & 224.32 & 192.73 & 259.89 & RW2 \\ 
  Burkina Faso & WEST & 2004 & 217.94 & 183.26 & 257.25 & RW2 \\ 
  Burkina Faso & WEST & 2005 & 210.02 & 173.20 & 251.50 & RW2 \\ 
  Burkina Faso & WEST & 2006 & 202.13 & 168.94 & 239.65 & RW2 \\ 
  Burkina Faso & WEST & 2007 & 193.90 & 164.25 & 226.84 & RW2 \\ 
  Burkina Faso & WEST & 2008 & 185.84 & 154.32 & 222.20 & RW2 \\ 
  Burkina Faso & WEST & 2009 & 177.76 & 135.31 & 230.05 & RW2 \\ 
  Burkina Faso & WEST & 2010 & 170.37 & 111.15 & 253.43 & RW2 \\ 
  Burkina Faso & WEST & 2011 & 162.81 & 89.14 & 279.57 & RW2 \\ 
  Burkina Faso & WEST & 2012 & 155.70 & 69.78 & 311.81 & RW2 \\ 
  Burkina Faso & WEST & 2013 & 149.06 & 53.35 & 355.03 & RW2 \\ 
  Burkina Faso & WEST & 2014 & 142.70 & 39.63 & 398.29 & RW2 \\ 
  Burkina Faso & WEST & 2015 & 136.37 & 29.01 & 449.87 & RW2 \\ 
  Burkina Faso & WEST & 2016 & 129.81 & 20.77 & 505.33 & RW2 \\ 
  Burkina Faso & WEST & 2017 & 124.39 & 14.78 & 568.17 & RW2 \\ 
  Burkina Faso & WEST & 2018 & 118.10 & 10.15 & 632.76 & RW2 \\ 
  Burkina Faso & WEST & 2019 & 112.88 & 6.92 & 699.12 & RW2 \\ 
  Burundi & ALL & 1980 & 221.51 & 207.20 & 236.97 & IHME \\ 
  Burundi & ALL & 1980 & 219.20 & 141.59 & 327.50 & RW2 \\ 
  Burundi & ALL & 1980 & 221.80 & 200.80 & 244.60 & UN \\ 
  Burundi & ALL & 1981 & 212.72 & 198.89 & 227.79 & IHME \\ 
  Burundi & ALL & 1981 & 210.29 & 150.29 & 287.51 & RW2 \\ 
  Burundi & ALL & 1981 & 212.30 & 192.20 & 233.70 & UN \\ 
  Burundi & ALL & 1982 & 203.03 & 189.47 & 217.37 & IHME \\ 
  Burundi & ALL & 1982 & 201.67 & 152.14 & 263.92 & RW2 \\ 
  Burundi & ALL & 1982 & 202.30 & 182.60 & 222.70 & UN \\ 
  Burundi & ALL & 1983 & 193.26 & 181.47 & 206.81 & IHME \\ 
  Burundi & ALL & 1983 & 193.10 & 147.01 & 251.83 & RW2 \\ 
  Burundi & ALL & 1983 & 192.60 & 173.30 & 213.00 & UN \\ 
  Burundi & ALL & 1984 & 184.52 & 173.35 & 197.30 & IHME \\ 
  Burundi & ALL & 1984 & 186.03 & 140.00 & 244.46 & RW2 \\ 
  Burundi & ALL & 1984 & 184.20 & 164.80 & 204.60 & UN \\ 
  Burundi & ALL & 1985 & 178.26 & 167.38 & 190.07 & IHME \\ 
  Burundi & ALL & 1985 & 178.57 & 136.08 & 228.69 & RW2 \\ 
  Burundi & ALL & 1985 & 177.50 & 158.00 & 198.10 & UN \\ 
  Burundi & ALL & 1986 & 174.55 & 163.44 & 186.04 & IHME \\ 
  Burundi & ALL & 1986 & 173.94 & 134.71 & 219.59 & RW2 \\ 
  Burundi & ALL & 1986 & 172.90 & 154.00 & 193.40 & UN \\ 
  Burundi & ALL & 1987 & 172.97 & 161.55 & 184.31 & IHME \\ 
  Burundi & ALL & 1987 & 171.13 & 134.15 & 214.41 & RW2 \\ 
  Burundi & ALL & 1987 & 170.60 & 152.60 & 190.20 & UN \\ 
  Burundi & ALL & 1988 & 172.75 & 161.35 & 184.64 & IHME \\ 
  Burundi & ALL & 1988 & 169.86 & 131.93 & 213.15 & RW2 \\ 
  Burundi & ALL & 1988 & 170.10 & 153.00 & 188.70 & UN \\ 
  Burundi & ALL & 1989 & 173.03 & 161.52 & 185.51 & IHME \\ 
  Burundi & ALL & 1989 & 169.98 & 131.01 & 214.29 & RW2 \\ 
  Burundi & ALL & 1989 & 170.70 & 154.00 & 189.10 & UN \\ 
  Burundi & ALL & 1990 & 173.41 & 161.55 & 184.97 & IHME \\ 
  Burundi & ALL & 1990 & 172.03 & 135.41 & 218.01 & RW2 \\ 
  Burundi & ALL & 1990 & 171.80 & 155.40 & 190.30 & UN \\ 
  Burundi & ALL & 1991 & 174.16 & 162.26 & 184.60 & IHME \\ 
  Burundi & ALL & 1991 & 173.16 & 138.21 & 215.40 & RW2 \\ 
  Burundi & ALL & 1991 & 172.90 & 156.20 & 192.00 & UN \\ 
  Burundi & ALL & 1992 & 174.70 & 163.07 & 185.33 & IHME \\ 
  Burundi & ALL & 1992 & 173.89 & 139.47 & 214.90 & RW2 \\ 
  Burundi & ALL & 1992 & 173.50 & 156.60 & 193.60 & UN \\ 
  Burundi & ALL & 1993 & 175.20 & 163.59 & 187.01 & IHME \\ 
  Burundi & ALL & 1993 & 173.86 & 138.80 & 216.27 & RW2 \\ 
  Burundi & ALL & 1993 & 173.50 & 155.90 & 194.20 & UN \\ 
  Burundi & ALL & 1994 & 174.88 & 164.09 & 186.85 & IHME \\ 
  Burundi & ALL & 1994 & 172.77 & 136.37 & 218.34 & RW2 \\ 
  Burundi & ALL & 1994 & 172.50 & 154.20 & 193.70 & UN \\ 
  Burundi & ALL & 1995 & 174.67 & 163.04 & 186.67 & IHME \\ 
  Burundi & ALL & 1995 & 170.50 & 134.63 & 213.05 & RW2 \\ 
  Burundi & ALL & 1995 & 170.50 & 152.40 & 192.20 & UN \\ 
  Burundi & ALL & 1996 & 174.29 & 163.47 & 186.27 & IHME \\ 
  Burundi & ALL & 1996 & 167.91 & 134.36 & 208.14 & RW2 \\ 
  Burundi & ALL & 1996 & 167.70 & 149.60 & 189.40 & UN \\ 
  Burundi & ALL & 1997 & 172.49 & 161.65 & 184.79 & IHME \\ 
  Burundi & ALL & 1997 & 164.86 & 132.96 & 202.63 & RW2 \\ 
  Burundi & ALL & 1997 & 164.40 & 146.30 & 185.70 & UN \\ 
  Burundi & ALL & 1998 & 170.90 & 159.54 & 183.32 & IHME \\ 
  Burundi & ALL & 1998 & 161.54 & 129.20 & 200.99 & RW2 \\ 
  Burundi & ALL & 1998 & 160.50 & 142.80 & 181.40 & UN \\ 
  Burundi & ALL & 1999 & 168.72 & 156.97 & 181.43 & IHME \\ 
  Burundi & ALL & 1999 & 157.80 & 124.16 & 197.64 & RW2 \\ 
  Burundi & ALL & 1999 & 156.40 & 139.00 & 176.70 & UN \\ 
  Burundi & ALL & 2000 & 165.89 & 153.86 & 178.38 & IHME \\ 
  Burundi & ALL & 2000 & 153.84 & 121.35 & 193.01 & RW2 \\ 
  Burundi & ALL & 2000 & 152.20 & 134.90 & 172.00 & UN \\ 
  Burundi & ALL & 2001 & 159.91 & 148.29 & 172.13 & IHME \\ 
  Burundi & ALL & 2001 & 149.23 & 119.05 & 185.39 & RW2 \\ 
  Burundi & ALL & 2001 & 147.80 & 130.70 & 167.20 & UN \\ 
  Burundi & ALL & 2002 & 151.93 & 140.61 & 163.13 & IHME \\ 
  Burundi & ALL & 2002 & 144.10 & 116.01 & 177.75 & RW2 \\ 
  Burundi & ALL & 2002 & 143.20 & 126.30 & 162.10 & UN \\ 
  Burundi & ALL & 2003 & 141.58 & 130.71 & 151.96 & IHME \\ 
  Burundi & ALL & 2003 & 138.61 & 110.81 & 172.28 & RW2 \\ 
  Burundi & ALL & 2003 & 138.30 & 121.30 & 157.30 & UN \\ 
  Burundi & ALL & 2004 & 129.91 & 119.94 & 139.74 & IHME \\ 
  Burundi & ALL & 2004 & 132.53 & 103.11 & 168.71 & RW2 \\ 
  Burundi & ALL & 2004 & 133.00 & 115.60 & 152.30 & UN \\ 
  Burundi & ALL & 2005 & 118.19 & 109.03 & 127.49 & IHME \\ 
  Burundi & ALL & 2005 & 126.31 & 95.79 & 164.23 & RW2 \\ 
  Burundi & ALL & 2005 & 127.10 & 109.10 & 147.10 & UN \\ 
  Burundi & ALL & 2006 & 107.22 & 98.85 & 115.76 & IHME \\ 
  Burundi & ALL & 2006 & 120.10 & 92.59 & 153.78 & RW2 \\ 
  Burundi & ALL & 2006 & 120.90 & 102.00 & 142.10 & UN \\ 
  Burundi & ALL & 2007 & 97.85 & 89.49 & 105.94 & IHME \\ 
  Burundi & ALL & 2007 & 114.07 & 90.16 & 143.12 & RW2 \\ 
  Burundi & ALL & 2007 & 114.50 & 94.20 & 138.40 & UN \\ 
  Burundi & ALL & 2008 & 91.37 & 83.35 & 99.77 & IHME \\ 
  Burundi & ALL & 2008 & 108.37 & 83.84 & 139.26 & RW2 \\ 
  Burundi & ALL & 2008 & 108.60 & 86.80 & 135.80 & UN \\ 
  Burundi & ALL & 2009 & 87.46 & 78.74 & 96.72 & IHME \\ 
  Burundi & ALL & 2009 & 102.64 & 71.16 & 146.48 & RW2 \\ 
  Burundi & ALL & 2009 & 103.50 & 79.30 & 133.70 & UN \\ 
  Burundi & ALL & 2010 & 85.12 & 74.92 & 95.74 & IHME \\ 
  Burundi & ALL & 2010 & 97.29 & 55.00 & 168.19 & RW2 \\ 
  Burundi & ALL & 2010 & 98.80 & 72.80 & 132.30 & UN \\ 
  Burundi & ALL & 2011 & 83.92 & 71.22 & 97.84 & IHME \\ 
  Burundi & ALL & 2011 & 92.31 & 41.49 & 192.43 & RW2 \\ 
  Burundi & ALL & 2011 & 94.90 & 66.90 & 131.50 & UN \\ 
  Burundi & ALL & 2012 & 81.92 & 67.16 & 99.21 & IHME \\ 
  Burundi & ALL & 2012 & 87.37 & 30.15 & 225.85 & RW2 \\ 
  Burundi & ALL & 2012 & 91.20 & 61.90 & 130.50 & UN \\ 
  Burundi & ALL & 2013 & 80.12 & 63.79 & 101.44 & IHME \\ 
  Burundi & ALL & 2013 & 83.02 & 21.63 & 266.56 & RW2 \\ 
  Burundi & ALL & 2013 & 87.80 & 57.40 & 130.50 & UN \\ 
  Burundi & ALL & 2014 & 78.40 & 60.28 & 103.46 & IHME \\ 
  Burundi & ALL & 2014 & 78.53 & 15.06 & 317.64 & RW2 \\ 
  Burundi & ALL & 2014 & 84.60 & 53.20 & 130.70 & UN \\ 
  Burundi & ALL & 2015 & 76.81 & 57.35 & 104.34 & IHME \\ 
  Burundi & ALL & 2015 & 74.05 & 10.43 & 374.13 & RW2 \\ 
  Burundi & ALL & 2015 & 81.70 & 49.70 & 130.30 & UN \\ 
  Burundi & ALL & 2016 & 70.62 & 7.12 & 447.68 & RW2 \\ 
  Burundi & ALL & 2017 & 66.41 & 4.61 & 523.94 & RW2 \\ 
  Burundi & ALL & 2018 & 62.81 & 3.05 & 608.17 & RW2 \\ 
  Burundi & ALL & 2019 & 59.22 & 1.78 & 675.47 & RW2 \\ 
  Burundi & ALL & 80-84 & 220.17 & 273.54 & 174.70 & HT-Direct \\ 
  Burundi & ALL & 85-89 & 177.88 & 204.32 & 154.20 & HT-Direct \\ 
  Burundi & ALL & 90-94 & 194.77 & 215.33 & 175.74 & HT-Direct \\ 
  Burundi & ALL & 95-99 & 207.96 & 226.38 & 190.67 & HT-Direct \\ 
  Burundi & ALL & 00-04 & 181.35 & 195.11 & 168.35 & HT-Direct \\ 
  Burundi & ALL & 05-09 & 100.90 & 109.38 & 93.01 & HT-Direct \\ 
  Burundi & ALL & 15-19 & 66.42 & 4.65 & 518.18 & RW2 \\ 
  Burundi & BUJUMBURA & 1980 & 215.37 & 113.59 & 383.89 & RW2 \\ 
  Burundi & BUJUMBURA & 1981 & 200.31 & 116.02 & 332.05 & RW2 \\ 
  Burundi & BUJUMBURA & 1982 & 185.87 & 113.99 & 294.17 & RW2 \\ 
  Burundi & BUJUMBURA & 1983 & 171.69 & 109.42 & 263.95 & RW2 \\ 
  Burundi & BUJUMBURA & 1984 & 159.72 & 103.90 & 239.49 & RW2 \\ 
  Burundi & BUJUMBURA & 1985 & 148.12 & 98.36 & 217.71 & RW2 \\ 
  Burundi & BUJUMBURA & 1986 & 139.06 & 94.19 & 200.62 & RW2 \\ 
  Burundi & BUJUMBURA & 1987 & 131.99 & 90.46 & 187.80 & RW2 \\ 
  Burundi & BUJUMBURA & 1988 & 126.34 & 87.30 & 177.35 & RW2 \\ 
  Burundi & BUJUMBURA & 1989 & 122.25 & 85.08 & 170.32 & RW2 \\ 
  Burundi & BUJUMBURA & 1990 & 119.56 & 84.37 & 164.86 & RW2 \\ 
  Burundi & BUJUMBURA & 1991 & 116.91 & 84.04 & 158.82 & RW2 \\ 
  Burundi & BUJUMBURA & 1992 & 114.62 & 83.14 & 153.94 & RW2 \\ 
  Burundi & BUJUMBURA & 1993 & 112.07 & 81.33 & 150.78 & RW2 \\ 
  Burundi & BUJUMBURA & 1994 & 109.43 & 79.48 & 147.47 & RW2 \\ 
  Burundi & BUJUMBURA & 1995 & 106.08 & 77.90 & 141.83 & RW2 \\ 
  Burundi & BUJUMBURA & 1996 & 102.76 & 76.25 & 135.73 & RW2 \\ 
  Burundi & BUJUMBURA & 1997 & 99.02 & 74.18 & 130.17 & RW2 \\ 
  Burundi & BUJUMBURA & 1998 & 95.04 & 71.01 & 125.27 & RW2 \\ 
  Burundi & BUJUMBURA & 1999 & 90.60 & 67.00 & 120.52 & RW2 \\ 
  Burundi & BUJUMBURA & 2000 & 86.38 & 64.09 & 114.20 & RW2 \\ 
  Burundi & BUJUMBURA & 2001 & 81.98 & 61.55 & 107.49 & RW2 \\ 
  Burundi & BUJUMBURA & 2002 & 77.78 & 58.78 & 101.66 & RW2 \\ 
  Burundi & BUJUMBURA & 2003 & 73.57 & 54.94 & 97.13 & RW2 \\ 
  Burundi & BUJUMBURA & 2004 & 69.64 & 51.00 & 94.31 & RW2 \\ 
  Burundi & BUJUMBURA & 2005 & 65.74 & 47.16 & 91.00 & RW2 \\ 
  Burundi & BUJUMBURA & 2006 & 62.19 & 44.15 & 86.89 & RW2 \\ 
  Burundi & BUJUMBURA & 2007 & 58.82 & 41.13 & 83.89 & RW2 \\ 
  Burundi & BUJUMBURA & 2008 & 55.51 & 36.92 & 84.53 & RW2 \\ 
  Burundi & BUJUMBURA & 2009 & 52.48 & 31.25 & 89.19 & RW2 \\ 
  Burundi & BUJUMBURA & 2010 & 49.60 & 24.54 & 100.99 & RW2 \\ 
  Burundi & BUJUMBURA & 2011 & 46.95 & 19.13 & 116.13 & RW2 \\ 
  Burundi & BUJUMBURA & 2012 & 44.41 & 14.32 & 135.01 & RW2 \\ 
  Burundi & BUJUMBURA & 2013 & 41.98 & 10.42 & 161.79 & RW2 \\ 
  Burundi & BUJUMBURA & 2014 & 39.67 & 7.40 & 195.98 & RW2 \\ 
  Burundi & BUJUMBURA & 2015 & 37.66 & 5.12 & 239.95 & RW2 \\ 
  Burundi & BUJUMBURA & 2016 & 35.57 & 3.51 & 292.37 & RW2 \\ 
  Burundi & BUJUMBURA & 2017 & 33.52 & 2.40 & 351.83 & RW2 \\ 
  Burundi & BUJUMBURA & 2018 & 31.77 & 1.61 & 426.12 & RW2 \\ 
  Burundi & BUJUMBURA & 2019 & 30.08 & 1.02 & 505.83 & RW2 \\ 
  Burundi & CENTRE-EAST & 1980 & 227.34 & 136.15 & 373.80 & RW2 \\ 
  Burundi & CENTRE-EAST & 1981 & 214.78 & 140.90 & 327.13 & RW2 \\ 
  Burundi & CENTRE-EAST & 1982 & 203.06 & 140.83 & 292.33 & RW2 \\ 
  Burundi & CENTRE-EAST & 1983 & 191.68 & 137.25 & 268.44 & RW2 \\ 
  Burundi & CENTRE-EAST & 1984 & 181.80 & 132.52 & 247.93 & RW2 \\ 
  Burundi & CENTRE-EAST & 1985 & 172.33 & 128.00 & 227.35 & RW2 \\ 
  Burundi & CENTRE-EAST & 1986 & 165.12 & 126.34 & 213.29 & RW2 \\ 
  Burundi & CENTRE-EAST & 1987 & 160.05 & 124.85 & 202.38 & RW2 \\ 
  Burundi & CENTRE-EAST & 1988 & 156.68 & 122.84 & 196.57 & RW2 \\ 
  Burundi & CENTRE-EAST & 1989 & 154.71 & 121.37 & 193.96 & RW2 \\ 
  Burundi & CENTRE-EAST & 1990 & 154.55 & 123.20 & 192.56 & RW2 \\ 
  Burundi & CENTRE-EAST & 1991 & 154.45 & 125.02 & 188.64 & RW2 \\ 
  Burundi & CENTRE-EAST & 1992 & 154.14 & 125.96 & 186.89 & RW2 \\ 
  Burundi & CENTRE-EAST & 1993 & 153.72 & 125.56 & 187.32 & RW2 \\ 
  Burundi & CENTRE-EAST & 1994 & 153.03 & 123.67 & 188.17 & RW2 \\ 
  Burundi & CENTRE-EAST & 1995 & 151.51 & 122.79 & 185.41 & RW2 \\ 
  Burundi & CENTRE-EAST & 1996 & 150.02 & 123.56 & 180.20 & RW2 \\ 
  Burundi & CENTRE-EAST & 1997 & 148.30 & 123.29 & 177.62 & RW2 \\ 
  Burundi & CENTRE-EAST & 1998 & 146.49 & 121.02 & 176.37 & RW2 \\ 
  Burundi & CENTRE-EAST & 1999 & 144.21 & 117.51 & 175.56 & RW2 \\ 
  Burundi & CENTRE-EAST & 2000 & 141.63 & 115.52 & 172.96 & RW2 \\ 
  Burundi & CENTRE-EAST & 2001 & 138.52 & 114.33 & 167.23 & RW2 \\ 
  Burundi & CENTRE-EAST & 2002 & 134.70 & 112.04 & 161.73 & RW2 \\ 
  Burundi & CENTRE-EAST & 2003 & 130.26 & 106.92 & 158.11 & RW2 \\ 
  Burundi & CENTRE-EAST & 2004 & 125.34 & 100.27 & 155.24 & RW2 \\ 
  Burundi & CENTRE-EAST & 2005 & 119.92 & 93.88 & 151.67 & RW2 \\ 
  Burundi & CENTRE-EAST & 2006 & 114.40 & 90.66 & 143.75 & RW2 \\ 
  Burundi & CENTRE-EAST & 2007 & 109.06 & 87.07 & 136.10 & RW2 \\ 
  Burundi & CENTRE-EAST & 2008 & 103.83 & 79.85 & 133.95 & RW2 \\ 
  Burundi & CENTRE-EAST & 2009 & 98.62 & 68.43 & 139.73 & RW2 \\ 
  Burundi & CENTRE-EAST & 2010 & 93.87 & 54.34 & 156.73 & RW2 \\ 
  Burundi & CENTRE-EAST & 2011 & 89.19 & 42.55 & 177.16 & RW2 \\ 
  Burundi & CENTRE-EAST & 2012 & 84.72 & 32.11 & 203.91 & RW2 \\ 
  Burundi & CENTRE-EAST & 2013 & 80.48 & 23.73 & 237.42 & RW2 \\ 
  Burundi & CENTRE-EAST & 2014 & 76.12 & 16.72 & 277.51 & RW2 \\ 
  Burundi & CENTRE-EAST & 2015 & 72.59 & 11.96 & 329.84 & RW2 \\ 
  Burundi & CENTRE-EAST & 2016 & 68.66 & 8.16 & 387.03 & RW2 \\ 
  Burundi & CENTRE-EAST & 2017 & 65.48 & 5.62 & 455.36 & RW2 \\ 
  Burundi & CENTRE-EAST & 2018 & 61.71 & 3.79 & 523.98 & RW2 \\ 
  Burundi & CENTRE-EAST & 2019 & 58.94 & 2.55 & 598.93 & RW2 \\ 
  Burundi & NORTH & 1980 & 193.02 & 112.74 & 308.35 & RW2 \\ 
  Burundi & NORTH & 1981 & 188.91 & 121.40 & 278.60 & RW2 \\ 
  Burundi & NORTH & 1982 & 184.24 & 126.08 & 258.09 & RW2 \\ 
  Burundi & NORTH & 1983 & 180.01 & 127.32 & 246.66 & RW2 \\ 
  Burundi & NORTH & 1984 & 176.40 & 126.74 & 238.80 & RW2 \\ 
  Burundi & NORTH & 1985 & 173.26 & 127.80 & 228.45 & RW2 \\ 
  Burundi & NORTH & 1986 & 172.23 & 130.18 & 221.90 & RW2 \\ 
  Burundi & NORTH & 1987 & 172.76 & 133.28 & 219.34 & RW2 \\ 
  Burundi & NORTH & 1988 & 174.99 & 136.09 & 220.37 & RW2 \\ 
  Burundi & NORTH & 1989 & 178.54 & 139.85 & 223.31 & RW2 \\ 
  Burundi & NORTH & 1990 & 183.63 & 146.81 & 227.79 & RW2 \\ 
  Burundi & NORTH & 1991 & 188.25 & 153.48 & 229.55 & RW2 \\ 
  Burundi & NORTH & 1992 & 192.10 & 157.83 & 232.27 & RW2 \\ 
  Burundi & NORTH & 1993 & 194.77 & 159.78 & 236.11 & RW2 \\ 
  Burundi & NORTH & 1994 & 196.44 & 159.62 & 239.65 & RW2 \\ 
  Burundi & NORTH & 1995 & 196.24 & 159.99 & 238.60 & RW2 \\ 
  Burundi & NORTH & 1996 & 195.31 & 161.28 & 234.48 & RW2 \\ 
  Burundi & NORTH & 1997 & 193.28 & 160.97 & 230.81 & RW2 \\ 
  Burundi & NORTH & 1998 & 190.47 & 158.23 & 227.97 & RW2 \\ 
  Burundi & NORTH & 1999 & 186.97 & 152.64 & 225.72 & RW2 \\ 
  Burundi & NORTH & 2000 & 182.79 & 149.48 & 220.74 & RW2 \\ 
  Burundi & NORTH & 2001 & 178.20 & 147.71 & 212.58 & RW2 \\ 
  Burundi & NORTH & 2002 & 173.10 & 144.79 & 205.28 & RW2 \\ 
  Burundi & NORTH & 2003 & 167.57 & 138.92 & 201.05 & RW2 \\ 
  Burundi & NORTH & 2004 & 161.91 & 131.12 & 198.24 & RW2 \\ 
  Burundi & NORTH & 2005 & 155.94 & 124.03 & 194.28 & RW2 \\ 
  Burundi & NORTH & 2006 & 150.14 & 120.87 & 185.23 & RW2 \\ 
  Burundi & NORTH & 2007 & 144.27 & 116.93 & 176.35 & RW2 \\ 
  Burundi & NORTH & 2008 & 138.62 & 108.63 & 175.11 & RW2 \\ 
  Burundi & NORTH & 2009 & 132.84 & 94.03 & 184.34 & RW2 \\ 
  Burundi & NORTH & 2010 & 127.55 & 75.88 & 207.45 & RW2 \\ 
  Burundi & NORTH & 2011 & 122.08 & 59.60 & 234.75 & RW2 \\ 
  Burundi & NORTH & 2012 & 116.96 & 45.61 & 268.99 & RW2 \\ 
  Burundi & NORTH & 2013 & 112.21 & 34.04 & 315.55 & RW2 \\ 
  Burundi & NORTH & 2014 & 107.65 & 24.64 & 363.62 & RW2 \\ 
  Burundi & NORTH & 2015 & 103.08 & 17.56 & 421.77 & RW2 \\ 
  Burundi & NORTH & 2016 & 98.25 & 12.23 & 485.22 & RW2 \\ 
  Burundi & NORTH & 2017 & 94.40 & 8.46 & 557.43 & RW2 \\ 
  Burundi & NORTH & 2018 & 89.74 & 5.63 & 631.56 & RW2 \\ 
  Burundi & NORTH & 2019 & 86.00 & 3.72 & 706.68 & RW2 \\ 
  Burundi & SOUTH & 1980 & 242.02 & 151.66 & 367.27 & RW2 \\ 
  Burundi & SOUTH & 1981 & 230.14 & 158.04 & 325.49 & RW2 \\ 
  Burundi & SOUTH & 1982 & 218.82 & 158.21 & 296.39 & RW2 \\ 
  Burundi & SOUTH & 1983 & 207.65 & 153.22 & 277.24 & RW2 \\ 
  Burundi & SOUTH & 1984 & 197.52 & 146.74 & 263.07 & RW2 \\ 
  Burundi & SOUTH & 1985 & 187.64 & 141.14 & 244.81 & RW2 \\ 
  Burundi & SOUTH & 1986 & 179.61 & 137.82 & 231.27 & RW2 \\ 
  Burundi & SOUTH & 1987 & 172.86 & 133.90 & 220.84 & RW2 \\ 
  Burundi & SOUTH & 1988 & 167.45 & 130.03 & 213.19 & RW2 \\ 
  Burundi & SOUTH & 1989 & 162.96 & 126.20 & 207.38 & RW2 \\ 
  Burundi & SOUTH & 1990 & 159.84 & 125.25 & 202.29 & RW2 \\ 
  Burundi & SOUTH & 1991 & 156.29 & 124.03 & 195.63 & RW2 \\ 
  Burundi & SOUTH & 1992 & 152.61 & 122.33 & 189.68 & RW2 \\ 
  Burundi & SOUTH & 1993 & 148.94 & 119.22 & 185.14 & RW2 \\ 
  Burundi & SOUTH & 1994 & 144.75 & 114.64 & 180.43 & RW2 \\ 
  Burundi & SOUTH & 1995 & 140.48 & 112.28 & 173.66 & RW2 \\ 
  Burundi & SOUTH & 1996 & 136.31 & 110.81 & 166.36 & RW2 \\ 
  Burundi & SOUTH & 1997 & 132.01 & 108.54 & 159.45 & RW2 \\ 
  Burundi & SOUTH & 1998 & 128.02 & 104.53 & 155.37 & RW2 \\ 
  Burundi & SOUTH & 1999 & 123.83 & 99.87 & 151.72 & RW2 \\ 
  Burundi & SOUTH & 2000 & 119.72 & 96.98 & 146.58 & RW2 \\ 
  Burundi & SOUTH & 2001 & 115.43 & 94.96 & 139.79 & RW2 \\ 
  Burundi & SOUTH & 2002 & 111.15 & 92.20 & 133.47 & RW2 \\ 
  Burundi & SOUTH & 2003 & 106.51 & 87.04 & 129.39 & RW2 \\ 
  Burundi & SOUTH & 2004 & 101.92 & 81.14 & 127.30 & RW2 \\ 
  Burundi & SOUTH & 2005 & 97.18 & 75.48 & 124.32 & RW2 \\ 
  Burundi & SOUTH & 2006 & 92.46 & 71.98 & 118.16 & RW2 \\ 
  Burundi & SOUTH & 2007 & 87.86 & 68.39 & 112.46 & RW2 \\ 
  Burundi & SOUTH & 2008 & 83.59 & 62.47 & 111.71 & RW2 \\ 
  Burundi & SOUTH & 2009 & 79.39 & 53.30 & 117.84 & RW2 \\ 
  Burundi & SOUTH & 2010 & 75.52 & 42.68 & 132.36 & RW2 \\ 
  Burundi & SOUTH & 2011 & 71.45 & 32.85 & 150.25 & RW2 \\ 
  Burundi & SOUTH & 2012 & 67.82 & 24.85 & 173.40 & RW2 \\ 
  Burundi & SOUTH & 2013 & 64.43 & 18.30 & 204.80 & RW2 \\ 
  Burundi & SOUTH & 2014 & 61.29 & 13.12 & 242.96 & RW2 \\ 
  Burundi & SOUTH & 2015 & 58.06 & 9.35 & 292.27 & RW2 \\ 
  Burundi & SOUTH & 2016 & 55.16 & 6.31 & 344.01 & RW2 \\ 
  Burundi & SOUTH & 2017 & 52.22 & 4.31 & 409.81 & RW2 \\ 
  Burundi & SOUTH & 2018 & 49.54 & 2.94 & 486.64 & RW2 \\ 
  Burundi & SOUTH & 2019 & 47.12 & 1.84 & 559.00 & RW2 \\ 
  Burundi & WEST & 1980 & 202.19 & 114.68 & 321.66 & RW2 \\ 
  Burundi & WEST & 1981 & 198.59 & 125.18 & 293.35 & RW2 \\ 
  Burundi & WEST & 1982 & 194.99 & 130.52 & 274.59 & RW2 \\ 
  Burundi & WEST & 1983 & 191.46 & 132.90 & 264.16 & RW2 \\ 
  Burundi & WEST & 1984 & 188.61 & 134.71 & 255.85 & RW2 \\ 
  Burundi & WEST & 1985 & 185.96 & 136.83 & 245.73 & RW2 \\ 
  Burundi & WEST & 1986 & 185.30 & 140.11 & 239.09 & RW2 \\ 
  Burundi & WEST & 1987 & 185.95 & 143.85 & 235.58 & RW2 \\ 
  Burundi & WEST & 1988 & 187.89 & 146.77 & 236.22 & RW2 \\ 
  Burundi & WEST & 1989 & 191.12 & 150.08 & 239.17 & RW2 \\ 
  Burundi & WEST & 1990 & 195.27 & 156.03 & 243.10 & RW2 \\ 
  Burundi & WEST & 1991 & 198.89 & 160.79 & 243.73 & RW2 \\ 
  Burundi & WEST & 1992 & 201.54 & 164.38 & 245.57 & RW2 \\ 
  Burundi & WEST & 1993 & 202.81 & 165.11 & 248.28 & RW2 \\ 
  Burundi & WEST & 1994 & 202.84 & 163.95 & 250.17 & RW2 \\ 
  Burundi & WEST & 1995 & 201.18 & 163.35 & 246.20 & RW2 \\ 
  Burundi & WEST & 1996 & 198.23 & 162.72 & 239.88 & RW2 \\ 
  Burundi & WEST & 1997 & 194.48 & 161.35 & 233.80 & RW2 \\ 
  Burundi & WEST & 1998 & 189.66 & 156.55 & 229.16 & RW2 \\ 
  Burundi & WEST & 1999 & 184.07 & 149.78 & 224.08 & RW2 \\ 
  Burundi & WEST & 2000 & 177.60 & 144.45 & 216.55 & RW2 \\ 
  Burundi & WEST & 2001 & 170.37 & 140.57 & 205.44 & RW2 \\ 
  Burundi & WEST & 2002 & 162.59 & 134.73 & 194.65 & RW2 \\ 
  Burundi & WEST & 2003 & 154.17 & 126.06 & 187.42 & RW2 \\ 
  Burundi & WEST & 2004 & 145.61 & 115.99 & 181.83 & RW2 \\ 
  Burundi & WEST & 2005 & 136.95 & 106.04 & 174.89 & RW2 \\ 
  Burundi & WEST & 2006 & 128.36 & 99.03 & 164.14 & RW2 \\ 
  Burundi & WEST & 2007 & 120.07 & 91.76 & 154.29 & RW2 \\ 
  Burundi & WEST & 2008 & 112.15 & 81.62 & 149.74 & RW2 \\ 
  Burundi & WEST & 2009 & 104.77 & 68.93 & 153.12 & RW2 \\ 
  Burundi & WEST & 2010 & 97.71 & 54.20 & 166.87 & RW2 \\ 
  Burundi & WEST & 2011 & 91.00 & 41.47 & 184.62 & RW2 \\ 
  Burundi & WEST & 2012 & 84.97 & 30.75 & 208.57 & RW2 \\ 
  Burundi & WEST & 2013 & 79.13 & 22.86 & 238.69 & RW2 \\ 
  Burundi & WEST & 2014 & 73.47 & 16.05 & 274.28 & RW2 \\ 
  Burundi & WEST & 2015 & 68.34 & 11.04 & 322.47 & RW2 \\ 
  Burundi & WEST & 2016 & 63.40 & 7.33 & 372.15 & RW2 \\ 
  Burundi & WEST & 2017 & 59.12 & 4.97 & 432.22 & RW2 \\ 
  Burundi & WEST & 2018 & 54.52 & 3.24 & 494.37 & RW2 \\ 
  Burundi & WEST & 2019 & 51.01 & 2.08 & 573.57 & RW2 \\ 
  Cameroon & ADAM/NORD/EXT-NORD & 1980 & 266.62 & 206.52 & 336.03 & RW2 \\ 
  Cameroon & ADAM/NORD/EXT-NORD & 1981 & 254.14 & 211.49 & 301.11 & RW2 \\ 
  Cameroon & ADAM/NORD/EXT-NORD & 1982 & 241.91 & 205.53 & 282.58 & RW2 \\ 
  Cameroon & ADAM/NORD/EXT-NORD & 1983 & 230.06 & 193.73 & 272.37 & RW2 \\ 
  Cameroon & ADAM/NORD/EXT-NORD & 1984 & 219.68 & 183.01 & 262.55 & RW2 \\ 
  Cameroon & ADAM/NORD/EXT-NORD & 1985 & 209.77 & 176.22 & 247.32 & RW2 \\ 
  Cameroon & ADAM/NORD/EXT-NORD & 1986 & 202.36 & 172.75 & 235.36 & RW2 \\ 
  Cameroon & ADAM/NORD/EXT-NORD & 1987 & 196.98 & 169.33 & 227.87 & RW2 \\ 
  Cameroon & ADAM/NORD/EXT-NORD & 1988 & 193.47 & 165.39 & 224.40 & RW2 \\ 
  Cameroon & ADAM/NORD/EXT-NORD & 1989 & 191.88 & 162.56 & 224.58 & RW2 \\ 
  Cameroon & ADAM/NORD/EXT-NORD & 1990 & 192.12 & 163.55 & 224.69 & RW2 \\ 
  Cameroon & ADAM/NORD/EXT-NORD & 1991 & 193.18 & 166.88 & 223.14 & RW2 \\ 
  Cameroon & ADAM/NORD/EXT-NORD & 1992 & 195.12 & 169.41 & 223.69 & RW2 \\ 
  Cameroon & ADAM/NORD/EXT-NORD & 1993 & 197.35 & 169.80 & 227.82 & RW2 \\ 
  Cameroon & ADAM/NORD/EXT-NORD & 1994 & 199.72 & 169.85 & 232.78 & RW2 \\ 
  Cameroon & ADAM/NORD/EXT-NORD & 1995 & 201.78 & 172.47 & 235.29 & RW2 \\ 
  Cameroon & ADAM/NORD/EXT-NORD & 1996 & 203.17 & 175.21 & 234.16 & RW2 \\ 
  Cameroon & ADAM/NORD/EXT-NORD & 1997 & 203.53 & 176.61 & 233.43 & RW2 \\ 
  Cameroon & ADAM/NORD/EXT-NORD & 1998 & 202.76 & 174.71 & 233.82 & RW2 \\ 
  Cameroon & ADAM/NORD/EXT-NORD & 1999 & 200.43 & 170.19 & 233.87 & RW2 \\ 
  Cameroon & ADAM/NORD/EXT-NORD & 2000 & 197.11 & 167.61 & 229.30 & RW2 \\ 
  Cameroon & ADAM/NORD/EXT-NORD & 2001 & 192.29 & 165.48 & 221.64 & RW2 \\ 
  Cameroon & ADAM/NORD/EXT-NORD & 2002 & 186.44 & 161.64 & 214.13 & RW2 \\ 
  Cameroon & ADAM/NORD/EXT-NORD & 2003 & 179.59 & 154.32 & 207.83 & RW2 \\ 
  Cameroon & ADAM/NORD/EXT-NORD & 2004 & 172.38 & 146.24 & 202.99 & RW2 \\ 
  Cameroon & ADAM/NORD/EXT-NORD & 2005 & 164.53 & 138.55 & 193.60 & RW2 \\ 
  Cameroon & ADAM/NORD/EXT-NORD & 2006 & 157.79 & 134.14 & 184.04 & RW2 \\ 
  Cameroon & ADAM/NORD/EXT-NORD & 2007 & 151.83 & 130.13 & 176.12 & RW2 \\ 
  Cameroon & ADAM/NORD/EXT-NORD & 2008 & 146.69 & 124.67 & 172.14 & RW2 \\ 
  Cameroon & ADAM/NORD/EXT-NORD & 2009 & 142.54 & 118.44 & 170.26 & RW2 \\ 
  Cameroon & ADAM/NORD/EXT-NORD & 2010 & 139.42 & 113.60 & 170.54 & RW2 \\ 
  Cameroon & ADAM/NORD/EXT-NORD & 2011 & 136.49 & 112.09 & 166.15 & RW2 \\ 
  Cameroon & ADAM/NORD/EXT-NORD & 2012 & 133.84 & 110.23 & 161.94 & RW2 \\ 
  Cameroon & ADAM/NORD/EXT-NORD & 2013 & 131.32 & 104.68 & 163.74 & RW2 \\ 
  Cameroon & ADAM/NORD/EXT-NORD & 2014 & 128.82 & 94.14 & 174.26 & RW2 \\ 
  Cameroon & ADAM/NORD/EXT-NORD & 2015 & 126.36 & 79.16 & 195.95 & RW2 \\ 
  Cameroon & ADAM/NORD/EXT-NORD & 2016 & 123.87 & 65.90 & 221.71 & RW2 \\ 
  Cameroon & ADAM/NORD/EXT-NORD & 2017 & 121.31 & 53.72 & 253.70 & RW2 \\ 
  Cameroon & ADAM/NORD/EXT-NORD & 2018 & 119.00 & 42.80 & 294.50 & RW2 \\ 
  Cameroon & ADAM/NORD/EXT-NORD & 2019 & 116.71 & 33.02 & 342.47 & RW2 \\ 
  Cameroon & ALL & 1980 & 167.34 & 161.53 & 173.39 & IHME \\ 
  Cameroon & ALL & 1980 & 185.05 & 137.86 & 243.34 & RW2 \\ 
  Cameroon & ALL & 1980 & 181.60 & 167.20 & 197.50 & UN \\ 
  Cameroon & ALL & 1981 & 163.07 & 157.41 & 169.09 & IHME \\ 
  Cameroon & ALL & 1981 & 176.89 & 143.26 & 215.61 & RW2 \\ 
  Cameroon & ALL & 1981 & 176.30 & 162.90 & 191.30 & UN \\ 
  Cameroon & ALL & 1982 & 158.57 & 153.13 & 164.17 & IHME \\ 
  Cameroon & ALL & 1982 & 169.05 & 140.19 & 202.74 & RW2 \\ 
  Cameroon & ALL & 1982 & 170.10 & 157.60 & 184.20 & UN \\ 
  Cameroon & ALL & 1983 & 154.54 & 149.26 & 159.79 & IHME \\ 
  Cameroon & ALL & 1983 & 161.47 & 131.73 & 197.45 & RW2 \\ 
  Cameroon & ALL & 1983 & 163.40 & 151.70 & 176.20 & UN \\ 
  Cameroon & ALL & 1984 & 150.57 & 145.53 & 155.71 & IHME \\ 
  Cameroon & ALL & 1984 & 154.94 & 123.80 & 193.26 & RW2 \\ 
  Cameroon & ALL & 1984 & 156.60 & 145.60 & 168.40 & UN \\ 
  Cameroon & ALL & 1985 & 146.38 & 141.57 & 151.31 & IHME \\ 
  Cameroon & ALL & 1985 & 148.40 & 120.08 & 181.31 & RW2 \\ 
  Cameroon & ALL & 1985 & 150.20 & 139.90 & 161.20 & UN \\ 
  Cameroon & ALL & 1986 & 143.28 & 138.78 & 148.09 & IHME \\ 
  Cameroon & ALL & 1986 & 143.81 & 118.27 & 173.32 & RW2 \\ 
  Cameroon & ALL & 1986 & 144.70 & 135.00 & 155.20 & UN \\ 
  Cameroon & ALL & 1987 & 140.07 & 135.65 & 144.70 & IHME \\ 
  Cameroon & ALL & 1987 & 140.61 & 116.90 & 168.60 & RW2 \\ 
  Cameroon & ALL & 1987 & 140.70 & 131.10 & 150.80 & UN \\ 
  Cameroon & ALL & 1988 & 138.25 & 133.80 & 142.65 & IHME \\ 
  Cameroon & ALL & 1988 & 138.69 & 114.19 & 167.39 & RW2 \\ 
  Cameroon & ALL & 1988 & 138.10 & 128.60 & 148.00 & UN \\ 
  Cameroon & ALL & 1989 & 137.09 & 132.59 & 141.42 & IHME \\ 
  Cameroon & ALL & 1989 & 138.14 & 112.48 & 168.40 & RW2 \\ 
  Cameroon & ALL & 1989 & 137.20 & 127.70 & 147.00 & UN \\ 
  Cameroon & ALL & 1990 & 136.78 & 132.33 & 141.33 & IHME \\ 
  Cameroon & ALL & 1990 & 139.03 & 113.89 & 169.79 & RW2 \\ 
  Cameroon & ALL & 1990 & 138.00 & 128.40 & 147.70 & UN \\ 
  Cameroon & ALL & 1991 & 137.02 & 132.72 & 141.58 & IHME \\ 
  Cameroon & ALL & 1991 & 140.69 & 116.64 & 168.97 & RW2 \\ 
  Cameroon & ALL & 1991 & 140.00 & 130.30 & 150.10 & UN \\ 
  Cameroon & ALL & 1992 & 137.81 & 133.58 & 142.26 & IHME \\ 
  Cameroon & ALL & 1992 & 143.12 & 119.11 & 170.67 & RW2 \\ 
  Cameroon & ALL & 1992 & 142.80 & 132.70 & 153.20 & UN \\ 
  Cameroon & ALL & 1993 & 138.94 & 134.62 & 143.52 & IHME \\ 
  Cameroon & ALL & 1993 & 146.06 & 120.76 & 174.79 & RW2 \\ 
  Cameroon & ALL & 1993 & 146.00 & 135.60 & 156.60 & UN \\ 
  Cameroon & ALL & 1994 & 139.90 & 135.31 & 144.70 & IHME \\ 
  Cameroon & ALL & 1994 & 149.02 & 121.54 & 180.59 & RW2 \\ 
  Cameroon & ALL & 1994 & 149.00 & 138.10 & 160.30 & UN \\ 
  Cameroon & ALL & 1995 & 140.65 & 135.94 & 145.43 & IHME \\ 
  Cameroon & ALL & 1995 & 152.29 & 125.23 & 184.73 & RW2 \\ 
  Cameroon & ALL & 1995 & 151.80 & 140.80 & 163.60 & UN \\ 
  Cameroon & ALL & 1996 & 141.03 & 136.11 & 146.04 & IHME \\ 
  Cameroon & ALL & 1996 & 154.16 & 128.05 & 185.30 & RW2 \\ 
  Cameroon & ALL & 1996 & 154.00 & 142.60 & 166.40 & UN \\ 
  Cameroon & ALL & 1997 & 140.83 & 135.72 & 145.96 & IHME \\ 
  Cameroon & ALL & 1997 & 154.90 & 129.39 & 184.49 & RW2 \\ 
  Cameroon & ALL & 1997 & 155.30 & 143.20 & 168.20 & UN \\ 
  Cameroon & ALL & 1998 & 140.05 & 134.75 & 145.34 & IHME \\ 
  Cameroon & ALL & 1998 & 154.34 & 128.17 & 185.62 & RW2 \\ 
  Cameroon & ALL & 1998 & 155.30 & 142.90 & 168.90 & UN \\ 
  Cameroon & ALL & 1999 & 138.82 & 133.46 & 144.25 & IHME \\ 
  Cameroon & ALL & 1999 & 152.38 & 125.05 & 184.50 & RW2 \\ 
  Cameroon & ALL & 1999 & 153.80 & 140.90 & 167.70 & UN \\ 
  Cameroon & ALL & 2000 & 137.49 & 132.05 & 143.00 & IHME \\ 
  Cameroon & ALL & 2000 & 148.80 & 121.75 & 179.46 & RW2 \\ 
  Cameroon & ALL & 2000 & 150.40 & 137.50 & 164.20 & UN \\ 
  Cameroon & ALL & 2001 & 135.69 & 129.98 & 141.29 & IHME \\ 
  Cameroon & ALL & 2001 & 144.85 & 119.73 & 173.50 & RW2 \\ 
  Cameroon & ALL & 2001 & 145.70 & 132.90 & 159.00 & UN \\ 
  Cameroon & ALL & 2002 & 133.32 & 127.66 & 139.18 & IHME \\ 
  Cameroon & ALL & 2002 & 140.31 & 116.96 & 167.52 & RW2 \\ 
  Cameroon & ALL & 2002 & 140.20 & 127.30 & 154.00 & UN \\ 
  Cameroon & ALL & 2003 & 130.55 & 124.53 & 136.46 & IHME \\ 
  Cameroon & ALL & 2003 & 135.57 & 112.61 & 162.83 & RW2 \\ 
  Cameroon & ALL & 2003 & 134.70 & 121.00 & 149.60 & UN \\ 
  Cameroon & ALL & 2004 & 127.59 & 121.27 & 133.59 & IHME \\ 
  Cameroon & ALL & 2004 & 130.55 & 106.66 & 159.13 & RW2 \\ 
  Cameroon & ALL & 2004 & 129.60 & 114.20 & 146.80 & UN \\ 
  Cameroon & ALL & 2005 & 124.64 & 118.27 & 131.03 & IHME \\ 
  Cameroon & ALL & 2005 & 125.74 & 102.20 & 153.76 & RW2 \\ 
  Cameroon & ALL & 2005 & 125.10 & 107.30 & 144.90 & UN \\ 
  Cameroon & ALL & 2006 & 121.33 & 114.53 & 128.00 & IHME \\ 
  Cameroon & ALL & 2006 & 121.03 & 99.49 & 146.34 & RW2 \\ 
  Cameroon & ALL & 2006 & 120.70 & 100.20 & 143.90 & UN \\ 
  Cameroon & ALL & 2007 & 117.91 & 110.78 & 124.89 & IHME \\ 
  Cameroon & ALL & 2007 & 116.62 & 96.63 & 140.06 & RW2 \\ 
  Cameroon & ALL & 2007 & 116.60 & 93.00 & 144.70 & UN \\ 
  Cameroon & ALL & 2008 & 114.28 & 107.09 & 121.87 & IHME \\ 
  Cameroon & ALL & 2008 & 112.60 & 92.49 & 136.47 & RW2 \\ 
  Cameroon & ALL & 2008 & 112.80 & 85.40 & 146.10 & UN \\ 
  Cameroon & ALL & 2009 & 110.69 & 102.94 & 118.75 & IHME \\ 
  Cameroon & ALL & 2009 & 108.64 & 87.48 & 134.42 & RW2 \\ 
  Cameroon & ALL & 2009 & 109.00 & 78.60 & 147.40 & UN \\ 
  Cameroon & ALL & 2010 & 106.82 & 98.45 & 115.54 & IHME \\ 
  Cameroon & ALL & 2010 & 105.10 & 83.22 & 132.90 & RW2 \\ 
  Cameroon & ALL & 2010 & 104.80 & 72.10 & 148.30 & UN \\ 
  Cameroon & ALL & 2011 & 103.29 & 94.10 & 113.02 & IHME \\ 
  Cameroon & ALL & 2011 & 101.72 & 81.02 & 127.20 & RW2 \\ 
  Cameroon & ALL & 2011 & 101.00 & 66.60 & 149.10 & UN \\ 
  Cameroon & ALL & 2012 & 99.86 & 89.90 & 110.81 & IHME \\ 
  Cameroon & ALL & 2012 & 98.42 & 79.00 & 121.95 & RW2 \\ 
  Cameroon & ALL & 2012 & 97.40 & 61.90 & 149.20 & UN \\ 
  Cameroon & ALL & 2013 & 96.26 & 85.56 & 107.79 & IHME \\ 
  Cameroon & ALL & 2013 & 95.31 & 74.33 & 121.18 & RW2 \\ 
  Cameroon & ALL & 2013 & 93.90 & 57.30 & 148.60 & UN \\ 
  Cameroon & ALL & 2014 & 93.17 & 81.58 & 105.53 & IHME \\ 
  Cameroon & ALL & 2014 & 92.18 & 65.29 & 128.48 & RW2 \\ 
  Cameroon & ALL & 2014 & 90.60 & 53.50 & 148.20 & UN \\ 
  Cameroon & ALL & 2015 & 89.76 & 77.87 & 103.40 & IHME \\ 
  Cameroon & ALL & 2015 & 89.03 & 53.07 & 145.74 & RW2 \\ 
  Cameroon & ALL & 2015 & 87.90 & 50.30 & 147.90 & UN \\ 
  Cameroon & ALL & 2016 & 86.30 & 43.00 & 166.82 & RW2 \\ 
  Cameroon & ALL & 2017 & 83.26 & 33.54 & 193.94 & RW2 \\ 
  Cameroon & ALL & 2018 & 80.48 & 25.87 & 229.60 & RW2 \\ 
  Cameroon & ALL & 2019 & 77.67 & 18.78 & 268.82 & RW2 \\ 
  Cameroon & ALL & 80-84 & 159.28 & 171.85 & 147.47 & HT-Direct \\ 
  Cameroon & ALL & 85-89 & 134.49 & 142.92 & 126.49 & HT-Direct \\ 
  Cameroon & ALL & 90-94 & 144.50 & 152.09 & 137.23 & HT-Direct \\ 
  Cameroon & ALL & 95-99 & 150.04 & 157.95 & 142.46 & HT-Direct \\ 
  Cameroon & ALL & 00-04 & 142.83 & 150.15 & 135.80 & HT-Direct \\ 
  Cameroon & ALL & 05-09 & 127.48 & 135.69 & 119.70 & HT-Direct \\ 
  Cameroon & ALL & 10-14 & 104.36 & 118.38 & 91.84 & HT-Direct \\ 
  Cameroon & ALL & 15-19 & 83.25 & 34.06 & 190.98 & RW2 \\ 
  Cameroon & CENTRE/SUD/EST & 1980 & 175.36 & 130.52 & 231.36 & RW2 \\ 
  Cameroon & CENTRE/SUD/EST & 1981 & 167.08 & 134.01 & 205.95 & RW2 \\ 
  Cameroon & CENTRE/SUD/EST & 1982 & 158.85 & 130.83 & 191.35 & RW2 \\ 
  Cameroon & CENTRE/SUD/EST & 1983 & 151.22 & 123.76 & 184.03 & RW2 \\ 
  Cameroon & CENTRE/SUD/EST & 1984 & 144.36 & 116.90 & 177.58 & RW2 \\ 
  Cameroon & CENTRE/SUD/EST & 1985 & 138.10 & 113.46 & 166.61 & RW2 \\ 
  Cameroon & CENTRE/SUD/EST & 1986 & 133.65 & 111.87 & 158.34 & RW2 \\ 
  Cameroon & CENTRE/SUD/EST & 1987 & 130.62 & 110.53 & 153.66 & RW2 \\ 
  Cameroon & CENTRE/SUD/EST & 1988 & 129.09 & 108.53 & 152.68 & RW2 \\ 
  Cameroon & CENTRE/SUD/EST & 1989 & 128.90 & 107.36 & 153.62 & RW2 \\ 
  Cameroon & CENTRE/SUD/EST & 1990 & 130.14 & 109.03 & 155.17 & RW2 \\ 
  Cameroon & CENTRE/SUD/EST & 1991 & 132.06 & 112.17 & 155.24 & RW2 \\ 
  Cameroon & CENTRE/SUD/EST & 1992 & 134.46 & 114.70 & 156.90 & RW2 \\ 
  Cameroon & CENTRE/SUD/EST & 1993 & 137.09 & 116.31 & 160.61 & RW2 \\ 
  Cameroon & CENTRE/SUD/EST & 1994 & 139.89 & 117.12 & 165.39 & RW2 \\ 
  Cameroon & CENTRE/SUD/EST & 1995 & 142.42 & 119.79 & 168.76 & RW2 \\ 
  Cameroon & CENTRE/SUD/EST & 1996 & 144.35 & 122.82 & 168.90 & RW2 \\ 
  Cameroon & CENTRE/SUD/EST & 1997 & 145.38 & 124.58 & 169.24 & RW2 \\ 
  Cameroon & CENTRE/SUD/EST & 1998 & 145.41 & 124.13 & 169.87 & RW2 \\ 
  Cameroon & CENTRE/SUD/EST & 1999 & 144.26 & 120.98 & 170.32 & RW2 \\ 
  Cameroon & CENTRE/SUD/EST & 2000 & 141.83 & 118.91 & 167.74 & RW2 \\ 
  Cameroon & CENTRE/SUD/EST & 2001 & 138.19 & 117.12 & 161.85 & RW2 \\ 
  Cameroon & CENTRE/SUD/EST & 2002 & 133.35 & 113.87 & 155.63 & RW2 \\ 
  Cameroon & CENTRE/SUD/EST & 2003 & 127.56 & 108.34 & 150.35 & RW2 \\ 
  Cameroon & CENTRE/SUD/EST & 2004 & 121.32 & 101.50 & 144.98 & RW2 \\ 
  Cameroon & CENTRE/SUD/EST & 2005 & 114.53 & 95.12 & 136.91 & RW2 \\ 
  Cameroon & CENTRE/SUD/EST & 2006 & 108.49 & 91.20 & 128.55 & RW2 \\ 
  Cameroon & CENTRE/SUD/EST & 2007 & 102.95 & 86.64 & 121.64 & RW2 \\ 
  Cameroon & CENTRE/SUD/EST & 2008 & 98.18 & 81.44 & 117.73 & RW2 \\ 
  Cameroon & CENTRE/SUD/EST & 2009 & 93.96 & 75.98 & 115.37 & RW2 \\ 
  Cameroon & CENTRE/SUD/EST & 2010 & 90.67 & 71.77 & 114.41 & RW2 \\ 
  Cameroon & CENTRE/SUD/EST & 2011 & 87.45 & 68.76 & 110.65 & RW2 \\ 
  Cameroon & CENTRE/SUD/EST & 2012 & 84.55 & 65.79 & 107.42 & RW2 \\ 
  Cameroon & CENTRE/SUD/EST & 2013 & 81.82 & 61.13 & 108.00 & RW2 \\ 
  Cameroon & CENTRE/SUD/EST & 2014 & 79.14 & 54.09 & 113.37 & RW2 \\ 
  Cameroon & CENTRE/SUD/EST & 2015 & 76.43 & 44.98 & 126.12 & RW2 \\ 
  Cameroon & CENTRE/SUD/EST & 2016 & 73.74 & 36.82 & 141.86 & RW2 \\ 
  Cameroon & CENTRE/SUD/EST & 2017 & 71.35 & 29.44 & 163.42 & RW2 \\ 
  Cameroon & CENTRE/SUD/EST & 2018 & 68.73 & 22.84 & 191.13 & RW2 \\ 
  Cameroon & CENTRE/SUD/EST & 2019 & 66.44 & 17.42 & 226.90 & RW2 \\ 
  Cameroon & NORD-OUEST/SUD-OUEST & 1980 & 136.06 & 99.24 & 184.36 & RW2 \\ 
  Cameroon & NORD-OUEST/SUD-OUEST & 1981 & 128.95 & 101.55 & 162.25 & RW2 \\ 
  Cameroon & NORD-OUEST/SUD-OUEST & 1982 & 122.20 & 98.68 & 150.01 & RW2 \\ 
  Cameroon & NORD-OUEST/SUD-OUEST & 1983 & 115.79 & 93.07 & 143.87 & RW2 \\ 
  Cameroon & NORD-OUEST/SUD-OUEST & 1984 & 110.22 & 87.96 & 137.84 & RW2 \\ 
  Cameroon & NORD-OUEST/SUD-OUEST & 1985 & 105.05 & 84.52 & 128.83 & RW2 \\ 
  Cameroon & NORD-OUEST/SUD-OUEST & 1986 & 101.32 & 83.36 & 122.61 & RW2 \\ 
  Cameroon & NORD-OUEST/SUD-OUEST & 1987 & 98.90 & 82.34 & 118.25 & RW2 \\ 
  Cameroon & NORD-OUEST/SUD-OUEST & 1988 & 97.62 & 80.90 & 117.16 & RW2 \\ 
  Cameroon & NORD-OUEST/SUD-OUEST & 1989 & 97.41 & 79.98 & 118.05 & RW2 \\ 
  Cameroon & NORD-OUEST/SUD-OUEST & 1990 & 98.42 & 81.33 & 119.12 & RW2 \\ 
  Cameroon & NORD-OUEST/SUD-OUEST & 1991 & 100.02 & 83.74 & 118.85 & RW2 \\ 
  Cameroon & NORD-OUEST/SUD-OUEST & 1992 & 101.97 & 86.04 & 120.30 & RW2 \\ 
  Cameroon & NORD-OUEST/SUD-OUEST & 1993 & 104.27 & 87.68 & 123.71 & RW2 \\ 
  Cameroon & NORD-OUEST/SUD-OUEST & 1994 & 106.68 & 88.44 & 127.95 & RW2 \\ 
  Cameroon & NORD-OUEST/SUD-OUEST & 1995 & 108.89 & 90.52 & 130.93 & RW2 \\ 
  Cameroon & NORD-OUEST/SUD-OUEST & 1996 & 110.61 & 93.16 & 130.78 & RW2 \\ 
  Cameroon & NORD-OUEST/SUD-OUEST & 1997 & 111.69 & 94.70 & 131.70 & RW2 \\ 
  Cameroon & NORD-OUEST/SUD-OUEST & 1998 & 112.06 & 94.34 & 132.75 & RW2 \\ 
  Cameroon & NORD-OUEST/SUD-OUEST & 1999 & 111.36 & 92.54 & 133.29 & RW2 \\ 
  Cameroon & NORD-OUEST/SUD-OUEST & 2000 & 109.75 & 91.23 & 131.67 & RW2 \\ 
  Cameroon & NORD-OUEST/SUD-OUEST & 2001 & 107.17 & 89.93 & 127.29 & RW2 \\ 
  Cameroon & NORD-OUEST/SUD-OUEST & 2002 & 103.64 & 87.52 & 122.62 & RW2 \\ 
  Cameroon & NORD-OUEST/SUD-OUEST & 2003 & 99.41 & 83.23 & 118.67 & RW2 \\ 
  Cameroon & NORD-OUEST/SUD-OUEST & 2004 & 94.80 & 78.10 & 114.72 & RW2 \\ 
  Cameroon & NORD-OUEST/SUD-OUEST & 2005 & 89.75 & 73.24 & 109.04 & RW2 \\ 
  Cameroon & NORD-OUEST/SUD-OUEST & 2006 & 85.22 & 70.37 & 102.96 & RW2 \\ 
  Cameroon & NORD-OUEST/SUD-OUEST & 2007 & 81.24 & 67.25 & 98.05 & RW2 \\ 
  Cameroon & NORD-OUEST/SUD-OUEST & 2008 & 77.78 & 63.31 & 95.19 & RW2 \\ 
  Cameroon & NORD-OUEST/SUD-OUEST & 2009 & 74.72 & 59.30 & 93.53 & RW2 \\ 
  Cameroon & NORD-OUEST/SUD-OUEST & 2010 & 72.42 & 56.11 & 93.13 & RW2 \\ 
  Cameroon & NORD-OUEST/SUD-OUEST & 2011 & 70.16 & 54.10 & 90.51 & RW2 \\ 
  Cameroon & NORD-OUEST/SUD-OUEST & 2012 & 68.10 & 51.93 & 88.33 & RW2 \\ 
  Cameroon & NORD-OUEST/SUD-OUEST & 2013 & 66.15 & 48.53 & 88.75 & RW2 \\ 
  Cameroon & NORD-OUEST/SUD-OUEST & 2014 & 64.15 & 43.05 & 93.47 & RW2 \\ 
  Cameroon & NORD-OUEST/SUD-OUEST & 2015 & 62.30 & 36.10 & 104.72 & RW2 \\ 
  Cameroon & NORD-OUEST/SUD-OUEST & 2016 & 60.38 & 29.59 & 118.58 & RW2 \\ 
  Cameroon & NORD-OUEST/SUD-OUEST & 2017 & 58.70 & 23.77 & 137.70 & RW2 \\ 
  Cameroon & NORD-OUEST/SUD-OUEST & 2018 & 56.79 & 18.65 & 161.36 & RW2 \\ 
  Cameroon & NORD-OUEST/SUD-OUEST & 2019 & 55.27 & 14.43 & 191.98 & RW2 \\ 
  Cameroon & OUEST/LITTORAL & 1980 & 137.91 & 101.12 & 185.33 & RW2 \\ 
  Cameroon & OUEST/LITTORAL & 1981 & 131.29 & 104.49 & 164.46 & RW2 \\ 
  Cameroon & OUEST/LITTORAL & 1982 & 124.96 & 101.65 & 152.99 & RW2 \\ 
  Cameroon & OUEST/LITTORAL & 1983 & 119.03 & 95.86 & 147.40 & RW2 \\ 
  Cameroon & OUEST/LITTORAL & 1984 & 113.77 & 90.94 & 142.08 & RW2 \\ 
  Cameroon & OUEST/LITTORAL & 1985 & 108.95 & 88.09 & 133.83 & RW2 \\ 
  Cameroon & OUEST/LITTORAL & 1986 & 105.66 & 86.87 & 127.67 & RW2 \\ 
  Cameroon & OUEST/LITTORAL & 1987 & 103.53 & 86.03 & 123.97 & RW2 \\ 
  Cameroon & OUEST/LITTORAL & 1988 & 102.61 & 84.87 & 123.47 & RW2 \\ 
  Cameroon & OUEST/LITTORAL & 1989 & 102.94 & 84.48 & 124.79 & RW2 \\ 
  Cameroon & OUEST/LITTORAL & 1990 & 104.37 & 86.25 & 126.40 & RW2 \\ 
  Cameroon & OUEST/LITTORAL & 1991 & 106.55 & 88.96 & 126.86 & RW2 \\ 
  Cameroon & OUEST/LITTORAL & 1992 & 109.22 & 91.95 & 129.02 & RW2 \\ 
  Cameroon & OUEST/LITTORAL & 1993 & 112.17 & 94.05 & 133.11 & RW2 \\ 
  Cameroon & OUEST/LITTORAL & 1994 & 115.31 & 95.68 & 138.16 & RW2 \\ 
  Cameroon & OUEST/LITTORAL & 1995 & 118.44 & 98.91 & 141.56 & RW2 \\ 
  Cameroon & OUEST/LITTORAL & 1996 & 120.89 & 102.05 & 142.65 & RW2 \\ 
  Cameroon & OUEST/LITTORAL & 1997 & 122.83 & 104.68 & 143.91 & RW2 \\ 
  Cameroon & OUEST/LITTORAL & 1998 & 123.85 & 104.87 & 146.00 & RW2 \\ 
  Cameroon & OUEST/LITTORAL & 1999 & 123.90 & 103.33 & 147.53 & RW2 \\ 
  Cameroon & OUEST/LITTORAL & 2000 & 122.87 & 102.26 & 146.71 & RW2 \\ 
  Cameroon & OUEST/LITTORAL & 2001 & 120.75 & 101.72 & 142.83 & RW2 \\ 
  Cameroon & OUEST/LITTORAL & 2002 & 117.70 & 99.56 & 138.48 & RW2 \\ 
  Cameroon & OUEST/LITTORAL & 2003 & 113.81 & 95.51 & 135.54 & RW2 \\ 
  Cameroon & OUEST/LITTORAL & 2004 & 109.57 & 90.61 & 132.78 & RW2 \\ 
  Cameroon & OUEST/LITTORAL & 2005 & 104.86 & 85.72 & 127.32 & RW2 \\ 
  Cameroon & OUEST/LITTORAL & 2006 & 100.64 & 82.87 & 121.54 & RW2 \\ 
  Cameroon & OUEST/LITTORAL & 2007 & 96.86 & 79.83 & 117.09 & RW2 \\ 
  Cameroon & OUEST/LITTORAL & 2008 & 93.64 & 75.86 & 114.90 & RW2 \\ 
  Cameroon & OUEST/LITTORAL & 2009 & 91.06 & 72.01 & 114.29 & RW2 \\ 
  Cameroon & OUEST/LITTORAL & 2010 & 89.14 & 69.00 & 114.69 & RW2 \\ 
  Cameroon & OUEST/LITTORAL & 2011 & 87.26 & 67.09 & 112.85 & RW2 \\ 
  Cameroon & OUEST/LITTORAL & 2012 & 85.67 & 65.16 & 111.49 & RW2 \\ 
  Cameroon & OUEST/LITTORAL & 2013 & 84.11 & 61.96 & 113.19 & RW2 \\ 
  Cameroon & OUEST/LITTORAL & 2014 & 82.51 & 55.86 & 120.13 & RW2 \\ 
  Cameroon & OUEST/LITTORAL & 2015 & 80.91 & 47.27 & 135.48 & RW2 \\ 
  Cameroon & OUEST/LITTORAL & 2016 & 79.31 & 39.17 & 153.92 & RW2 \\ 
  Cameroon & OUEST/LITTORAL & 2017 & 77.90 & 31.93 & 178.90 & RW2 \\ 
  Cameroon & OUEST/LITTORAL & 2018 & 76.18 & 25.23 & 209.72 & RW2 \\ 
  Cameroon & OUEST/LITTORAL & 2019 & 74.96 & 19.53 & 251.78 & RW2 \\ 
  Cameroon & YAOUNDE/DOUALA & 1980 & 134.53 & 94.11 & 190.39 & RW2 \\ 
  Cameroon & YAOUNDE/DOUALA & 1981 & 126.58 & 95.11 & 167.73 & RW2 \\ 
  Cameroon & YAOUNDE/DOUALA & 1982 & 119.15 & 91.93 & 153.63 & RW2 \\ 
  Cameroon & YAOUNDE/DOUALA & 1983 & 112.09 & 86.52 & 144.64 & RW2 \\ 
  Cameroon & YAOUNDE/DOUALA & 1984 & 105.86 & 81.59 & 137.35 & RW2 \\ 
  Cameroon & YAOUNDE/DOUALA & 1985 & 100.14 & 78.28 & 126.86 & RW2 \\ 
  Cameroon & YAOUNDE/DOUALA & 1986 & 95.90 & 76.73 & 119.22 & RW2 \\ 
  Cameroon & YAOUNDE/DOUALA & 1987 & 92.85 & 75.14 & 114.12 & RW2 \\ 
  Cameroon & YAOUNDE/DOUALA & 1988 & 91.00 & 73.67 & 111.72 & RW2 \\ 
  Cameroon & YAOUNDE/DOUALA & 1989 & 90.19 & 72.53 & 111.19 & RW2 \\ 
  Cameroon & YAOUNDE/DOUALA & 1990 & 90.52 & 73.33 & 111.05 & RW2 \\ 
  Cameroon & YAOUNDE/DOUALA & 1991 & 91.31 & 75.00 & 110.65 & RW2 \\ 
  Cameroon & YAOUNDE/DOUALA & 1992 & 92.52 & 76.78 & 111.28 & RW2 \\ 
  Cameroon & YAOUNDE/DOUALA & 1993 & 94.09 & 77.80 & 113.43 & RW2 \\ 
  Cameroon & YAOUNDE/DOUALA & 1994 & 95.59 & 77.85 & 116.00 & RW2 \\ 
  Cameroon & YAOUNDE/DOUALA & 1995 & 97.22 & 79.68 & 118.06 & RW2 \\ 
  Cameroon & YAOUNDE/DOUALA & 1996 & 98.36 & 81.64 & 118.23 & RW2 \\ 
  Cameroon & YAOUNDE/DOUALA & 1997 & 98.87 & 82.72 & 117.79 & RW2 \\ 
  Cameroon & YAOUNDE/DOUALA & 1998 & 98.86 & 82.09 & 118.34 & RW2 \\ 
  Cameroon & YAOUNDE/DOUALA & 1999 & 97.89 & 80.33 & 118.29 & RW2 \\ 
  Cameroon & YAOUNDE/DOUALA & 2000 & 96.15 & 79.07 & 116.23 & RW2 \\ 
  Cameroon & YAOUNDE/DOUALA & 2001 & 93.51 & 77.72 & 112.21 & RW2 \\ 
  Cameroon & YAOUNDE/DOUALA & 2002 & 90.23 & 75.42 & 107.71 & RW2 \\ 
  Cameroon & YAOUNDE/DOUALA & 2003 & 86.19 & 71.41 & 103.74 & RW2 \\ 
  Cameroon & YAOUNDE/DOUALA & 2004 & 81.95 & 66.97 & 100.29 & RW2 \\ 
  Cameroon & YAOUNDE/DOUALA & 2005 & 77.33 & 62.71 & 94.81 & RW2 \\ 
  Cameroon & YAOUNDE/DOUALA & 2006 & 73.16 & 59.77 & 89.16 & RW2 \\ 
  Cameroon & YAOUNDE/DOUALA & 2007 & 69.39 & 56.73 & 84.45 & RW2 \\ 
  Cameroon & YAOUNDE/DOUALA & 2008 & 66.20 & 53.21 & 81.88 & RW2 \\ 
  Cameroon & YAOUNDE/DOUALA & 2009 & 63.37 & 49.45 & 80.75 & RW2 \\ 
  Cameroon & YAOUNDE/DOUALA & 2010 & 61.14 & 46.62 & 80.27 & RW2 \\ 
  Cameroon & YAOUNDE/DOUALA & 2011 & 58.88 & 44.21 & 78.10 & RW2 \\ 
  Cameroon & YAOUNDE/DOUALA & 2012 & 56.90 & 42.01 & 76.32 & RW2 \\ 
  Cameroon & YAOUNDE/DOUALA & 2013 & 55.03 & 38.84 & 76.88 & RW2 \\ 
  Cameroon & YAOUNDE/DOUALA & 2014 & 53.23 & 34.38 & 80.81 & RW2 \\ 
  Cameroon & YAOUNDE/DOUALA & 2015 & 51.36 & 28.74 & 90.09 & RW2 \\ 
  Cameroon & YAOUNDE/DOUALA & 2016 & 49.63 & 23.40 & 101.26 & RW2 \\ 
  Cameroon & YAOUNDE/DOUALA & 2017 & 47.89 & 18.69 & 117.17 & RW2 \\ 
  Cameroon & YAOUNDE/DOUALA & 2018 & 46.25 & 14.67 & 138.57 & RW2 \\ 
  Cameroon & YAOUNDE/DOUALA & 2019 & 44.72 & 10.99 & 164.01 & RW2 \\ 
  Chad & ALL & 1980 & 233.65 & 225.33 & 242.15 & IHME \\ 
  Chad & ALL & 1980 & 242.81 & 176.54 & 323.55 & RW2 \\ 
  Chad & ALL & 1980 & 240.10 & 219.40 & 263.20 & UN \\ 
  Chad & ALL & 1981 & 233.09 & 224.78 & 242.09 & IHME \\ 
  Chad & ALL & 1981 & 239.91 & 189.81 & 297.91 & RW2 \\ 
  Chad & ALL & 1981 & 238.10 & 218.00 & 260.10 & UN \\ 
  Chad & ALL & 1982 & 230.47 & 222.69 & 238.69 & IHME \\ 
  Chad & ALL & 1982 & 237.16 & 194.05 & 286.31 & RW2 \\ 
  Chad & ALL & 1982 & 236.00 & 216.50 & 257.30 & UN \\ 
  Chad & ALL & 1983 & 227.54 & 219.54 & 236.00 & IHME \\ 
  Chad & ALL & 1983 & 234.04 & 190.39 & 284.18 & RW2 \\ 
  Chad & ALL & 1983 & 233.50 & 214.50 & 254.20 & UN \\ 
  Chad & ALL & 1984 & 224.46 & 216.88 & 231.97 & IHME \\ 
  Chad & ALL & 1984 & 231.41 & 185.12 & 284.54 & RW2 \\ 
  Chad & ALL & 1984 & 230.90 & 212.70 & 250.80 & UN \\ 
  Chad & ALL & 1985 & 219.46 & 211.96 & 227.49 & IHME \\ 
  Chad & ALL & 1985 & 228.24 & 185.00 & 278.53 & RW2 \\ 
  Chad & ALL & 1985 & 228.20 & 210.50 & 247.50 & UN \\ 
  Chad & ALL & 1986 & 215.10 & 208.41 & 222.22 & IHME \\ 
  Chad & ALL & 1986 & 225.30 & 184.37 & 272.40 & RW2 \\ 
  Chad & ALL & 1986 & 225.50 & 208.40 & 244.30 & UN \\ 
  Chad & ALL & 1987 & 214.03 & 206.54 & 222.32 & IHME \\ 
  Chad & ALL & 1987 & 222.44 & 183.56 & 267.58 & RW2 \\ 
  Chad & ALL & 1987 & 222.80 & 206.10 & 241.00 & UN \\ 
  Chad & ALL & 1988 & 207.38 & 200.78 & 214.08 & IHME \\ 
  Chad & ALL & 1988 & 219.39 & 180.17 & 264.64 & RW2 \\ 
  Chad & ALL & 1988 & 220.10 & 203.80 & 237.80 & UN \\ 
  Chad & ALL & 1989 & 204.04 & 197.68 & 210.65 & IHME \\ 
  Chad & ALL & 1989 & 216.48 & 176.97 & 262.45 & RW2 \\ 
  Chad & ALL & 1989 & 217.30 & 201.50 & 234.60 & UN \\ 
  Chad & ALL & 1990 & 201.29 & 194.65 & 207.57 & IHME \\ 
  Chad & ALL & 1990 & 213.60 & 175.31 & 257.70 & RW2 \\ 
  Chad & ALL & 1990 & 214.60 & 199.00 & 231.50 & UN \\ 
  Chad & ALL & 1991 & 197.18 & 190.91 & 203.45 & IHME \\ 
  Chad & ALL & 1991 & 211.15 & 175.05 & 251.68 & RW2 \\ 
  Chad & ALL & 1991 & 212.00 & 196.60 & 228.60 & UN \\ 
  Chad & ALL & 1992 & 194.98 & 188.54 & 201.32 & IHME \\ 
  Chad & ALL & 1992 & 208.99 & 173.91 & 248.22 & RW2 \\ 
  Chad & ALL & 1992 & 209.40 & 194.40 & 225.80 & UN \\ 
  Chad & ALL & 1993 & 192.85 & 186.46 & 199.32 & IHME \\ 
  Chad & ALL & 1993 & 207.14 & 171.64 & 247.10 & RW2 \\ 
  Chad & ALL & 1993 & 207.00 & 192.30 & 223.10 & UN \\ 
  Chad & ALL & 1994 & 191.16 & 184.66 & 197.65 & IHME \\ 
  Chad & ALL & 1994 & 205.32 & 168.53 & 247.55 & RW2 \\ 
  Chad & ALL & 1994 & 204.80 & 190.10 & 220.50 & UN \\ 
  Chad & ALL & 1995 & 189.56 & 182.74 & 196.16 & IHME \\ 
  Chad & ALL & 1995 & 204.00 & 168.67 & 244.90 & RW2 \\ 
  Chad & ALL & 1995 & 202.70 & 188.10 & 218.30 & UN \\ 
  Chad & ALL & 1996 & 188.40 & 181.36 & 195.33 & IHME \\ 
  Chad & ALL & 1996 & 201.83 & 168.55 & 240.64 & RW2 \\ 
  Chad & ALL & 1996 & 200.60 & 186.10 & 216.20 & UN \\ 
  Chad & ALL & 1997 & 187.30 & 180.08 & 194.51 & IHME \\ 
  Chad & ALL & 1997 & 199.29 & 167.32 & 235.97 & RW2 \\ 
  Chad & ALL & 1997 & 198.30 & 183.70 & 213.90 & UN \\ 
  Chad & ALL & 1998 & 185.84 & 178.54 & 193.20 & IHME \\ 
  Chad & ALL & 1998 & 196.39 & 163.96 & 234.80 & RW2 \\ 
  Chad & ALL & 1998 & 195.80 & 181.10 & 211.60 & UN \\ 
  Chad & ALL & 1999 & 184.35 & 176.66 & 192.12 & IHME \\ 
  Chad & ALL & 1999 & 193.10 & 159.39 & 232.28 & RW2 \\ 
  Chad & ALL & 1999 & 193.00 & 178.10 & 209.00 & UN \\ 
  Chad & ALL & 2000 & 183.24 & 175.62 & 191.24 & IHME \\ 
  Chad & ALL & 2000 & 189.17 & 155.60 & 227.00 & RW2 \\ 
  Chad & ALL & 2000 & 190.20 & 175.00 & 206.30 & UN \\ 
  Chad & ALL & 2001 & 181.22 & 173.12 & 189.81 & IHME \\ 
  Chad & ALL & 2001 & 185.99 & 154.21 & 221.95 & RW2 \\ 
  Chad & ALL & 2001 & 187.30 & 171.90 & 203.90 & UN \\ 
  Chad & ALL & 2002 & 179.27 & 170.45 & 188.22 & IHME \\ 
  Chad & ALL & 2002 & 183.14 & 152.86 & 217.72 & RW2 \\ 
  Chad & ALL & 2002 & 184.60 & 168.60 & 201.50 & UN \\ 
  Chad & ALL & 2003 & 176.81 & 167.47 & 186.10 & IHME \\ 
  Chad & ALL & 2003 & 180.83 & 150.31 & 215.69 & RW2 \\ 
  Chad & ALL & 2003 & 181.90 & 165.20 & 199.50 & UN \\ 
  Chad & ALL & 2004 & 172.44 & 162.57 & 182.25 & IHME \\ 
  Chad & ALL & 2004 & 178.57 & 146.10 & 215.49 & RW2 \\ 
  Chad & ALL & 2004 & 179.40 & 161.30 & 198.30 & UN \\ 
  Chad & ALL & 2005 & 167.91 & 157.72 & 178.33 & IHME \\ 
  Chad & ALL & 2005 & 177.23 & 144.94 & 216.06 & RW2 \\ 
  Chad & ALL & 2005 & 177.00 & 157.90 & 197.50 & UN \\ 
  Chad & ALL & 2006 & 164.11 & 153.38 & 175.20 & IHME \\ 
  Chad & ALL & 2006 & 174.98 & 144.01 & 211.58 & RW2 \\ 
  Chad & ALL & 2006 & 174.60 & 154.00 & 196.40 & UN \\ 
  Chad & ALL & 2007 & 159.84 & 148.54 & 171.48 & IHME \\ 
  Chad & ALL & 2007 & 172.42 & 142.48 & 207.53 & RW2 \\ 
  Chad & ALL & 2007 & 171.50 & 149.40 & 195.50 & UN \\ 
  Chad & ALL & 2008 & 156.33 & 144.42 & 169.18 & IHME \\ 
  Chad & ALL & 2008 & 169.52 & 139.03 & 205.64 & RW2 \\ 
  Chad & ALL & 2008 & 168.00 & 143.80 & 195.40 & UN \\ 
  Chad & ALL & 2009 & 152.64 & 140.59 & 166.63 & IHME \\ 
  Chad & ALL & 2009 & 165.77 & 134.21 & 204.19 & RW2 \\ 
  Chad & ALL & 2009 & 164.00 & 137.30 & 196.00 & UN \\ 
  Chad & ALL & 2010 & 148.24 & 135.62 & 162.93 & IHME \\ 
  Chad & ALL & 2010 & 161.42 & 129.21 & 200.33 & RW2 \\ 
  Chad & ALL & 2010 & 160.10 & 129.80 & 196.70 & UN \\ 
  Chad & ALL & 2011 & 144.54 & 131.49 & 159.96 & IHME \\ 
  Chad & ALL & 2011 & 157.30 & 127.64 & 191.98 & RW2 \\ 
  Chad & ALL & 2011 & 156.00 & 122.00 & 198.20 & UN \\ 
  Chad & ALL & 2012 & 140.78 & 127.16 & 157.21 & IHME \\ 
  Chad & ALL & 2012 & 152.98 & 126.34 & 183.75 & RW2 \\ 
  Chad & ALL & 2012 & 151.60 & 114.30 & 199.90 & UN \\ 
  Chad & ALL & 2013 & 137.30 & 123.14 & 154.24 & IHME \\ 
  Chad & ALL & 2013 & 148.75 & 120.80 & 181.75 & RW2 \\ 
  Chad & ALL & 2013 & 147.10 & 107.60 & 201.30 & UN \\ 
  Chad & ALL & 2014 & 134.16 & 119.68 & 151.86 & IHME \\ 
  Chad & ALL & 2014 & 144.54 & 107.09 & 191.83 & RW2 \\ 
  Chad & ALL & 2014 & 142.90 & 100.70 & 202.30 & UN \\ 
  Chad & ALL & 2015 & 130.50 & 115.86 & 148.82 & IHME \\ 
  Chad & ALL & 2015 & 140.37 & 87.81 & 216.55 & RW2 \\ 
  Chad & ALL & 2015 & 138.70 & 94.00 & 202.80 & UN \\ 
  Chad & ALL & 2016 & 136.67 & 71.78 & 244.76 & RW2 \\ 
  Chad & ALL & 2017 & 132.52 & 56.54 & 279.90 & RW2 \\ 
  Chad & ALL & 2018 & 128.70 & 44.01 & 324.35 & RW2 \\ 
  Chad & ALL & 2019 & 124.81 & 32.27 & 371.24 & RW2 \\ 
  Chad & ALL & 80-84 & 219.52 & 246.09 & 195.08 & HT-Direct \\ 
  Chad & ALL & 85-89 & 192.52 & 211.24 & 175.10 & HT-Direct \\ 
  Chad & ALL & 90-94 & 188.47 & 202.46 & 175.23 & HT-Direct \\ 
  Chad & ALL & 95-99 & 198.86 & 211.08 & 187.17 & HT-Direct \\ 
  Chad & ALL & 00-04 & 176.99 & 188.39 & 166.13 & HT-Direct \\ 
  Chad & ALL & 05-09 & 165.84 & 180.90 & 151.80 & HT-Direct \\ 
  Chad & ALL & 10-14 & 131.06 & 138.79 & 123.71 & HT-Direct \\ 
  Chad & ALL & 15-19 & 132.55 & 57.38 & 275.67 & RW2 \\ 
  Chad & ZONE 1 & 1980 & 190.83 & 127.72 & 274.43 & RW2 \\ 
  Chad & ZONE 1 & 1981 & 190.76 & 138.33 & 258.26 & RW2 \\ 
  Chad & ZONE 1 & 1982 & 190.70 & 143.14 & 249.53 & RW2 \\ 
  Chad & ZONE 1 & 1983 & 190.52 & 144.58 & 246.96 & RW2 \\ 
  Chad & ZONE 1 & 1984 & 190.35 & 145.51 & 245.16 & RW2 \\ 
  Chad & ZONE 1 & 1985 & 190.17 & 147.80 & 241.96 & RW2 \\ 
  Chad & ZONE 1 & 1986 & 189.94 & 149.77 & 238.05 & RW2 \\ 
  Chad & ZONE 1 & 1987 & 189.51 & 151.49 & 234.71 & RW2 \\ 
  Chad & ZONE 1 & 1988 & 188.86 & 151.94 & 232.90 & RW2 \\ 
  Chad & ZONE 1 & 1989 & 188.20 & 152.08 & 231.29 & RW2 \\ 
  Chad & ZONE 1 & 1990 & 186.98 & 152.84 & 227.81 & RW2 \\ 
  Chad & ZONE 1 & 1991 & 185.68 & 153.40 & 222.82 & RW2 \\ 
  Chad & ZONE 1 & 1992 & 183.82 & 153.31 & 219.00 & RW2 \\ 
  Chad & ZONE 1 & 1993 & 181.47 & 151.43 & 216.53 & RW2 \\ 
  Chad & ZONE 1 & 1994 & 179.01 & 148.68 & 214.61 & RW2 \\ 
  Chad & ZONE 1 & 1995 & 176.35 & 147.29 & 209.71 & RW2 \\ 
  Chad & ZONE 1 & 1996 & 173.64 & 146.48 & 204.33 & RW2 \\ 
  Chad & ZONE 1 & 1997 & 171.37 & 145.84 & 200.55 & RW2 \\ 
  Chad & ZONE 1 & 1998 & 169.33 & 143.45 & 199.20 & RW2 \\ 
  Chad & ZONE 1 & 1999 & 167.72 & 140.40 & 198.75 & RW2 \\ 
  Chad & ZONE 1 & 2000 & 166.38 & 139.30 & 197.32 & RW2 \\ 
  Chad & ZONE 1 & 2001 & 165.53 & 140.23 & 194.50 & RW2 \\ 
  Chad & ZONE 1 & 2002 & 165.10 & 140.46 & 192.69 & RW2 \\ 
  Chad & ZONE 1 & 2003 & 164.84 & 139.38 & 193.86 & RW2 \\ 
  Chad & ZONE 1 & 2004 & 164.92 & 137.85 & 196.25 & RW2 \\ 
  Chad & ZONE 1 & 2005 & 165.28 & 137.73 & 197.41 & RW2 \\ 
  Chad & ZONE 1 & 2006 & 165.15 & 138.58 & 195.79 & RW2 \\ 
  Chad & ZONE 1 & 2007 & 164.72 & 138.41 & 195.02 & RW2 \\ 
  Chad & ZONE 1 & 2008 & 163.93 & 135.86 & 196.36 & RW2 \\ 
  Chad & ZONE 1 & 2009 & 162.92 & 132.47 & 198.79 & RW2 \\ 
  Chad & ZONE 1 & 2010 & 161.36 & 128.70 & 200.22 & RW2 \\ 
  Chad & ZONE 1 & 2011 & 159.74 & 127.28 & 198.71 & RW2 \\ 
  Chad & ZONE 1 & 2012 & 158.14 & 125.56 & 197.50 & RW2 \\ 
  Chad & ZONE 1 & 2013 & 156.39 & 121.20 & 200.96 & RW2 \\ 
  Chad & ZONE 1 & 2014 & 154.58 & 111.01 & 212.40 & RW2 \\ 
  Chad & ZONE 1 & 2015 & 152.91 & 95.78 & 236.94 & RW2 \\ 
  Chad & ZONE 1 & 2016 & 151.12 & 80.91 & 264.70 & RW2 \\ 
  Chad & ZONE 1 & 2017 & 149.62 & 67.24 & 300.63 & RW2 \\ 
  Chad & ZONE 1 & 2018 & 147.56 & 54.24 & 342.79 & RW2 \\ 
  Chad & ZONE 1 & 2019 & 146.32 & 42.85 & 396.69 & RW2 \\ 
  Chad & ZONE 2 & 1980 & 267.95 & 190.93 & 364.30 & RW2 \\ 
  Chad & ZONE 2 & 1981 & 261.20 & 198.64 & 337.09 & RW2 \\ 
  Chad & ZONE 2 & 1982 & 254.74 & 199.22 & 320.10 & RW2 \\ 
  Chad & ZONE 2 & 1983 & 247.97 & 194.79 & 310.41 & RW2 \\ 
  Chad & ZONE 2 & 1984 & 241.50 & 189.44 & 303.38 & RW2 \\ 
  Chad & ZONE 2 & 1985 & 235.18 & 186.26 & 292.28 & RW2 \\ 
  Chad & ZONE 2 & 1986 & 228.89 & 183.81 & 281.54 & RW2 \\ 
  Chad & ZONE 2 & 1987 & 222.71 & 180.14 & 271.98 & RW2 \\ 
  Chad & ZONE 2 & 1988 & 216.74 & 175.92 & 263.85 & RW2 \\ 
  Chad & ZONE 2 & 1989 & 210.76 & 170.98 & 256.54 & RW2 \\ 
  Chad & ZONE 2 & 1990 & 204.94 & 167.44 & 247.61 & RW2 \\ 
  Chad & ZONE 2 & 1991 & 198.97 & 164.36 & 238.53 & RW2 \\ 
  Chad & ZONE 2 & 1992 & 192.82 & 160.75 & 230.25 & RW2 \\ 
  Chad & ZONE 2 & 1993 & 186.85 & 155.64 & 223.44 & RW2 \\ 
  Chad & ZONE 2 & 1994 & 180.60 & 149.00 & 216.72 & RW2 \\ 
  Chad & ZONE 2 & 1995 & 174.60 & 144.91 & 208.53 & RW2 \\ 
  Chad & ZONE 2 & 1996 & 168.72 & 141.71 & 199.90 & RW2 \\ 
  Chad & ZONE 2 & 1997 & 162.83 & 137.73 & 191.73 & RW2 \\ 
  Chad & ZONE 2 & 1998 & 157.44 & 132.19 & 186.33 & RW2 \\ 
  Chad & ZONE 2 & 1999 & 152.06 & 126.20 & 181.58 & RW2 \\ 
  Chad & ZONE 2 & 2000 & 147.03 & 122.11 & 175.77 & RW2 \\ 
  Chad & ZONE 2 & 2001 & 142.40 & 119.16 & 169.41 & RW2 \\ 
  Chad & ZONE 2 & 2002 & 138.37 & 116.07 & 163.92 & RW2 \\ 
  Chad & ZONE 2 & 2003 & 134.45 & 111.66 & 160.31 & RW2 \\ 
  Chad & ZONE 2 & 2004 & 131.04 & 107.44 & 158.31 & RW2 \\ 
  Chad & ZONE 2 & 2005 & 127.97 & 104.74 & 155.51 & RW2 \\ 
  Chad & ZONE 2 & 2006 & 124.75 & 102.67 & 150.63 & RW2 \\ 
  Chad & ZONE 2 & 2007 & 121.58 & 100.16 & 146.34 & RW2 \\ 
  Chad & ZONE 2 & 2008 & 118.58 & 96.61 & 144.31 & RW2 \\ 
  Chad & ZONE 2 & 2009 & 115.39 & 92.16 & 143.51 & RW2 \\ 
  Chad & ZONE 2 & 2010 & 112.14 & 88.27 & 141.73 & RW2 \\ 
  Chad & ZONE 2 & 2011 & 108.84 & 85.62 & 137.64 & RW2 \\ 
  Chad & ZONE 2 & 2012 & 105.74 & 83.28 & 133.85 & RW2 \\ 
  Chad & ZONE 2 & 2013 & 102.69 & 78.71 & 133.95 & RW2 \\ 
  Chad & ZONE 2 & 2014 & 99.77 & 70.73 & 139.87 & RW2 \\ 
  Chad & ZONE 2 & 2015 & 96.86 & 59.82 & 154.89 & RW2 \\ 
  Chad & ZONE 2 & 2016 & 94.09 & 49.34 & 171.98 & RW2 \\ 
  Chad & ZONE 2 & 2017 & 91.29 & 39.90 & 195.60 & RW2 \\ 
  Chad & ZONE 2 & 2018 & 88.63 & 31.74 & 226.24 & RW2 \\ 
  Chad & ZONE 2 & 2019 & 86.13 & 24.16 & 261.25 & RW2 \\ 
  Chad & ZONE 3 & 1980 & 310.43 & 225.28 & 413.26 & RW2 \\ 
  Chad & ZONE 3 & 1981 & 303.22 & 234.40 & 384.93 & RW2 \\ 
  Chad & ZONE 3 & 1982 & 295.81 & 234.90 & 367.52 & RW2 \\ 
  Chad & ZONE 3 & 1983 & 289.39 & 229.90 & 357.48 & RW2 \\ 
  Chad & ZONE 3 & 1984 & 282.22 & 223.56 & 349.00 & RW2 \\ 
  Chad & ZONE 3 & 1985 & 275.61 & 220.52 & 337.92 & RW2 \\ 
  Chad & ZONE 3 & 1986 & 268.78 & 217.73 & 326.26 & RW2 \\ 
  Chad & ZONE 3 & 1987 & 262.22 & 214.44 & 316.31 & RW2 \\ 
  Chad & ZONE 3 & 1988 & 255.68 & 210.24 & 306.62 & RW2 \\ 
  Chad & ZONE 3 & 1989 & 249.39 & 205.41 & 299.61 & RW2 \\ 
  Chad & ZONE 3 & 1990 & 243.24 & 202.04 & 290.00 & RW2 \\ 
  Chad & ZONE 3 & 1991 & 236.78 & 199.05 & 279.36 & RW2 \\ 
  Chad & ZONE 3 & 1992 & 230.31 & 194.82 & 269.87 & RW2 \\ 
  Chad & ZONE 3 & 1993 & 223.97 & 189.08 & 263.13 & RW2 \\ 
  Chad & ZONE 3 & 1994 & 217.48 & 182.29 & 257.41 & RW2 \\ 
  Chad & ZONE 3 & 1995 & 211.19 & 177.46 & 248.82 & RW2 \\ 
  Chad & ZONE 3 & 1996 & 204.92 & 174.03 & 239.55 & RW2 \\ 
  Chad & ZONE 3 & 1997 & 198.86 & 169.67 & 231.57 & RW2 \\ 
  Chad & ZONE 3 & 1998 & 193.01 & 164.20 & 225.74 & RW2 \\ 
  Chad & ZONE 3 & 1999 & 187.39 & 157.41 & 221.37 & RW2 \\ 
  Chad & ZONE 3 & 2000 & 182.31 & 153.41 & 215.21 & RW2 \\ 
  Chad & ZONE 3 & 2001 & 177.55 & 150.74 & 208.12 & RW2 \\ 
  Chad & ZONE 3 & 2002 & 173.20 & 147.78 & 202.24 & RW2 \\ 
  Chad & ZONE 3 & 2003 & 169.13 & 143.17 & 198.41 & RW2 \\ 
  Chad & ZONE 3 & 2004 & 165.32 & 138.60 & 195.97 & RW2 \\ 
  Chad & ZONE 3 & 2005 & 161.97 & 135.51 & 192.31 & RW2 \\ 
  Chad & ZONE 3 & 2006 & 158.26 & 133.56 & 186.75 & RW2 \\ 
  Chad & ZONE 3 & 2007 & 154.79 & 130.95 & 181.69 & RW2 \\ 
  Chad & ZONE 3 & 2008 & 151.01 & 126.44 & 179.87 & RW2 \\ 
  Chad & ZONE 3 & 2009 & 147.03 & 120.55 & 178.49 & RW2 \\ 
  Chad & ZONE 3 & 2010 & 142.87 & 114.83 & 176.40 & RW2 \\ 
  Chad & ZONE 3 & 2011 & 139.02 & 112.09 & 171.02 & RW2 \\ 
  Chad & ZONE 3 & 2012 & 135.04 & 109.06 & 166.38 & RW2 \\ 
  Chad & ZONE 3 & 2013 & 130.91 & 102.99 & 165.68 & RW2 \\ 
  Chad & ZONE 3 & 2014 & 127.25 & 92.71 & 173.52 & RW2 \\ 
  Chad & ZONE 3 & 2015 & 123.51 & 78.22 & 190.18 & RW2 \\ 
  Chad & ZONE 3 & 2016 & 119.96 & 64.76 & 212.86 & RW2 \\ 
  Chad & ZONE 3 & 2017 & 116.69 & 52.50 & 239.79 & RW2 \\ 
  Chad & ZONE 3 & 2018 & 113.30 & 41.54 & 273.02 & RW2 \\ 
  Chad & ZONE 3 & 2019 & 109.52 & 31.86 & 316.51 & RW2 \\ 
  Chad & ZONE 4 & 1980 & 222.05 & 148.48 & 318.73 & RW2 \\ 
  Chad & ZONE 4 & 1981 & 217.00 & 156.04 & 293.43 & RW2 \\ 
  Chad & ZONE 4 & 1982 & 211.77 & 157.38 & 278.57 & RW2 \\ 
  Chad & ZONE 4 & 1983 & 206.48 & 154.95 & 269.76 & RW2 \\ 
  Chad & ZONE 4 & 1984 & 201.37 & 152.08 & 261.51 & RW2 \\ 
  Chad & ZONE 4 & 1985 & 196.33 & 149.85 & 252.31 & RW2 \\ 
  Chad & ZONE 4 & 1986 & 191.56 & 148.36 & 243.04 & RW2 \\ 
  Chad & ZONE 4 & 1987 & 186.88 & 146.83 & 234.40 & RW2 \\ 
  Chad & ZONE 4 & 1988 & 182.25 & 143.74 & 227.78 & RW2 \\ 
  Chad & ZONE 4 & 1989 & 177.70 & 140.64 & 221.62 & RW2 \\ 
  Chad & ZONE 4 & 1990 & 173.20 & 138.67 & 214.43 & RW2 \\ 
  Chad & ZONE 4 & 1991 & 168.77 & 136.88 & 206.62 & RW2 \\ 
  Chad & ZONE 4 & 1992 & 164.24 & 134.06 & 199.58 & RW2 \\ 
  Chad & ZONE 4 & 1993 & 159.55 & 130.57 & 193.65 & RW2 \\ 
  Chad & ZONE 4 & 1994 & 154.81 & 126.09 & 188.99 & RW2 \\ 
  Chad & ZONE 4 & 1995 & 150.01 & 122.78 & 181.91 & RW2 \\ 
  Chad & ZONE 4 & 1996 & 145.36 & 120.30 & 174.54 & RW2 \\ 
  Chad & ZONE 4 & 1997 & 140.75 & 117.44 & 167.85 & RW2 \\ 
  Chad & ZONE 4 & 1998 & 136.24 & 113.46 & 163.47 & RW2 \\ 
  Chad & ZONE 4 & 1999 & 131.92 & 108.72 & 159.69 & RW2 \\ 
  Chad & ZONE 4 & 2000 & 127.74 & 105.06 & 154.04 & RW2 \\ 
  Chad & ZONE 4 & 2001 & 123.93 & 103.33 & 148.01 & RW2 \\ 
  Chad & ZONE 4 & 2002 & 120.45 & 100.88 & 143.18 & RW2 \\ 
  Chad & ZONE 4 & 2003 & 117.12 & 97.57 & 139.95 & RW2 \\ 
  Chad & ZONE 4 & 2004 & 114.15 & 93.97 & 137.60 & RW2 \\ 
  Chad & ZONE 4 & 2005 & 111.54 & 91.45 & 135.36 & RW2 \\ 
  Chad & ZONE 4 & 2006 & 108.83 & 89.97 & 130.99 & RW2 \\ 
  Chad & ZONE 4 & 2007 & 106.17 & 88.07 & 127.19 & RW2 \\ 
  Chad & ZONE 4 & 2008 & 103.60 & 85.15 & 125.34 & RW2 \\ 
  Chad & ZONE 4 & 2009 & 100.92 & 81.60 & 124.33 & RW2 \\ 
  Chad & ZONE 4 & 2010 & 98.34 & 77.70 & 123.31 & RW2 \\ 
  Chad & ZONE 4 & 2011 & 95.69 & 76.01 & 119.91 & RW2 \\ 
  Chad & ZONE 4 & 2012 & 93.25 & 74.01 & 117.02 & RW2 \\ 
  Chad & ZONE 4 & 2013 & 90.72 & 69.80 & 117.76 & RW2 \\ 
  Chad & ZONE 4 & 2014 & 88.34 & 62.60 & 123.47 & RW2 \\ 
  Chad & ZONE 4 & 2015 & 86.07 & 52.87 & 138.43 & RW2 \\ 
  Chad & ZONE 4 & 2016 & 83.77 & 44.00 & 155.59 & RW2 \\ 
  Chad & ZONE 4 & 2017 & 81.62 & 35.46 & 179.21 & RW2 \\ 
  Chad & ZONE 4 & 2018 & 79.25 & 27.62 & 207.57 & RW2 \\ 
  Chad & ZONE 4 & 2019 & 77.06 & 21.63 & 242.00 & RW2 \\ 
  Chad & ZONE 5 & 1980 & 191.26 & 128.15 & 274.00 & RW2 \\ 
  Chad & ZONE 5 & 1981 & 190.23 & 136.70 & 257.29 & RW2 \\ 
  Chad & ZONE 5 & 1982 & 189.23 & 141.12 & 249.39 & RW2 \\ 
  Chad & ZONE 5 & 1983 & 188.42 & 142.02 & 246.11 & RW2 \\ 
  Chad & ZONE 5 & 1984 & 187.61 & 141.41 & 243.65 & RW2 \\ 
  Chad & ZONE 5 & 1985 & 186.56 & 143.24 & 239.79 & RW2 \\ 
  Chad & ZONE 5 & 1986 & 185.56 & 144.59 & 234.96 & RW2 \\ 
  Chad & ZONE 5 & 1987 & 184.51 & 145.81 & 230.85 & RW2 \\ 
  Chad & ZONE 5 & 1988 & 183.32 & 145.48 & 227.98 & RW2 \\ 
  Chad & ZONE 5 & 1989 & 181.71 & 145.22 & 225.78 & RW2 \\ 
  Chad & ZONE 5 & 1990 & 180.38 & 145.60 & 221.11 & RW2 \\ 
  Chad & ZONE 5 & 1991 & 178.54 & 145.84 & 216.76 & RW2 \\ 
  Chad & ZONE 5 & 1992 & 176.55 & 145.60 & 212.28 & RW2 \\ 
  Chad & ZONE 5 & 1993 & 174.14 & 143.27 & 210.21 & RW2 \\ 
  Chad & ZONE 5 & 1994 & 171.58 & 140.55 & 207.93 & RW2 \\ 
  Chad & ZONE 5 & 1995 & 168.83 & 138.62 & 203.68 & RW2 \\ 
  Chad & ZONE 5 & 1996 & 166.28 & 138.54 & 198.88 & RW2 \\ 
  Chad & ZONE 5 & 1997 & 163.83 & 136.88 & 194.46 & RW2 \\ 
  Chad & ZONE 5 & 1998 & 161.27 & 134.51 & 192.36 & RW2 \\ 
  Chad & ZONE 5 & 1999 & 159.09 & 131.37 & 190.75 & RW2 \\ 
  Chad & ZONE 5 & 2000 & 157.05 & 129.97 & 188.43 & RW2 \\ 
  Chad & ZONE 5 & 2001 & 155.23 & 129.99 & 184.50 & RW2 \\ 
  Chad & ZONE 5 & 2002 & 153.77 & 129.51 & 181.91 & RW2 \\ 
  Chad & ZONE 5 & 2003 & 152.45 & 127.52 & 180.68 & RW2 \\ 
  Chad & ZONE 5 & 2004 & 151.12 & 124.77 & 181.35 & RW2 \\ 
  Chad & ZONE 5 & 2005 & 150.11 & 123.85 & 181.18 & RW2 \\ 
  Chad & ZONE 5 & 2006 & 148.61 & 123.29 & 177.97 & RW2 \\ 
  Chad & ZONE 5 & 2007 & 146.95 & 122.30 & 175.90 & RW2 \\ 
  Chad & ZONE 5 & 2008 & 144.86 & 119.41 & 175.21 & RW2 \\ 
  Chad & ZONE 5 & 2009 & 142.69 & 115.14 & 175.48 & RW2 \\ 
  Chad & ZONE 5 & 2010 & 140.04 & 111.49 & 174.52 & RW2 \\ 
  Chad & ZONE 5 & 2011 & 137.52 & 110.04 & 170.10 & RW2 \\ 
  Chad & ZONE 5 & 2012 & 134.87 & 108.89 & 166.48 & RW2 \\ 
  Chad & ZONE 5 & 2013 & 132.18 & 104.11 & 166.22 & RW2 \\ 
  Chad & ZONE 5 & 2014 & 129.53 & 95.04 & 174.69 & RW2 \\ 
  Chad & ZONE 5 & 2015 & 127.13 & 80.65 & 194.80 & RW2 \\ 
  Chad & ZONE 5 & 2016 & 124.38 & 68.28 & 216.88 & RW2 \\ 
  Chad & ZONE 5 & 2017 & 121.82 & 55.19 & 246.77 & RW2 \\ 
  Chad & ZONE 5 & 2018 & 119.51 & 44.28 & 281.69 & RW2 \\ 
  Chad & ZONE 5 & 2019 & 117.18 & 34.50 & 322.77 & RW2 \\ 
  Chad & ZONE 6 & 1980 & 176.59 & 116.13 & 264.08 & RW2 \\ 
  Chad & ZONE 6 & 1981 & 175.82 & 124.90 & 246.43 & RW2 \\ 
  Chad & ZONE 6 & 1982 & 174.98 & 128.43 & 236.59 & RW2 \\ 
  Chad & ZONE 6 & 1983 & 174.40 & 130.20 & 232.01 & RW2 \\ 
  Chad & ZONE 6 & 1984 & 173.69 & 130.49 & 228.94 & RW2 \\ 
  Chad & ZONE 6 & 1985 & 172.99 & 132.36 & 223.92 & RW2 \\ 
  Chad & ZONE 6 & 1986 & 172.15 & 133.98 & 218.43 & RW2 \\ 
  Chad & ZONE 6 & 1987 & 171.26 & 135.38 & 214.72 & RW2 \\ 
  Chad & ZONE 6 & 1988 & 170.24 & 135.66 & 211.69 & RW2 \\ 
  Chad & ZONE 6 & 1989 & 169.21 & 135.09 & 209.67 & RW2 \\ 
  Chad & ZONE 6 & 1990 & 168.19 & 135.59 & 206.82 & RW2 \\ 
  Chad & ZONE 6 & 1991 & 166.95 & 136.34 & 203.39 & RW2 \\ 
  Chad & ZONE 6 & 1992 & 165.47 & 135.98 & 199.34 & RW2 \\ 
  Chad & ZONE 6 & 1993 & 163.84 & 134.46 & 197.66 & RW2 \\ 
  Chad & ZONE 6 & 1994 & 162.20 & 132.22 & 196.92 & RW2 \\ 
  Chad & ZONE 6 & 1995 & 160.66 & 131.57 & 193.78 & RW2 \\ 
  Chad & ZONE 6 & 1996 & 159.32 & 131.88 & 189.87 & RW2 \\ 
  Chad & ZONE 6 & 1997 & 158.42 & 132.22 & 187.22 & RW2 \\ 
  Chad & ZONE 6 & 1998 & 157.67 & 131.55 & 186.95 & RW2 \\ 
  Chad & ZONE 6 & 1999 & 157.45 & 130.25 & 188.74 & RW2 \\ 
  Chad & ZONE 6 & 2000 & 157.56 & 130.90 & 187.84 & RW2 \\ 
  Chad & ZONE 6 & 2001 & 158.06 & 133.21 & 185.86 & RW2 \\ 
  Chad & ZONE 6 & 2002 & 158.93 & 135.01 & 185.89 & RW2 \\ 
  Chad & ZONE 6 & 2003 & 159.92 & 135.53 & 187.45 & RW2 \\ 
  Chad & ZONE 6 & 2004 & 161.14 & 135.14 & 190.68 & RW2 \\ 
  Chad & ZONE 6 & 2005 & 162.34 & 135.97 & 192.52 & RW2 \\ 
  Chad & ZONE 6 & 2006 & 163.17 & 138.10 & 191.88 & RW2 \\ 
  Chad & ZONE 6 & 2007 & 163.35 & 138.63 & 191.58 & RW2 \\ 
  Chad & ZONE 6 & 2008 & 163.11 & 137.31 & 192.72 & RW2 \\ 
  Chad & ZONE 6 & 2009 & 162.65 & 134.43 & 195.67 & RW2 \\ 
  Chad & ZONE 6 & 2010 & 161.37 & 130.64 & 196.84 & RW2 \\ 
  Chad & ZONE 6 & 2011 & 160.04 & 130.72 & 194.78 & RW2 \\ 
  Chad & ZONE 6 & 2012 & 158.57 & 129.68 & 193.01 & RW2 \\ 
  Chad & ZONE 6 & 2013 & 157.06 & 124.84 & 195.65 & RW2 \\ 
  Chad & ZONE 6 & 2014 & 155.69 & 114.38 & 207.71 & RW2 \\ 
  Chad & ZONE 6 & 2015 & 154.16 & 99.26 & 232.97 & RW2 \\ 
  Chad & ZONE 6 & 2016 & 152.58 & 83.76 & 260.86 & RW2 \\ 
  Chad & ZONE 6 & 2017 & 151.37 & 69.44 & 296.71 & RW2 \\ 
  Chad & ZONE 6 & 2018 & 149.15 & 56.32 & 341.17 & RW2 \\ 
  Chad & ZONE 6 & 2019 & 147.70 & 44.47 & 390.61 & RW2 \\ 
  Chad & ZONE 7 & 1980 & 233.44 & 158.75 & 325.25 & RW2 \\ 
  Chad & ZONE 7 & 1981 & 236.12 & 172.75 & 309.26 & RW2 \\ 
  Chad & ZONE 7 & 1982 & 238.63 & 181.59 & 302.95 & RW2 \\ 
  Chad & ZONE 7 & 1983 & 241.06 & 186.34 & 303.75 & RW2 \\ 
  Chad & ZONE 7 & 1984 & 243.40 & 189.39 & 305.30 & RW2 \\ 
  Chad & ZONE 7 & 1985 & 245.86 & 194.57 & 304.26 & RW2 \\ 
  Chad & ZONE 7 & 1986 & 248.27 & 199.55 & 302.42 & RW2 \\ 
  Chad & ZONE 7 & 1987 & 250.23 & 204.23 & 301.39 & RW2 \\ 
  Chad & ZONE 7 & 1988 & 252.55 & 207.32 & 302.48 & RW2 \\ 
  Chad & ZONE 7 & 1989 & 254.41 & 209.71 & 304.24 & RW2 \\ 
  Chad & ZONE 7 & 1990 & 256.14 & 213.74 & 303.26 & RW2 \\ 
  Chad & ZONE 7 & 1991 & 257.27 & 217.66 & 301.62 & RW2 \\ 
  Chad & ZONE 7 & 1992 & 258.04 & 220.01 & 300.01 & RW2 \\ 
  Chad & ZONE 7 & 1993 & 258.39 & 220.53 & 300.34 & RW2 \\ 
  Chad & ZONE 7 & 1994 & 258.14 & 219.16 & 301.49 & RW2 \\ 
  Chad & ZONE 7 & 1995 & 257.50 & 219.43 & 299.04 & RW2 \\ 
  Chad & ZONE 7 & 1996 & 256.68 & 221.28 & 294.91 & RW2 \\ 
  Chad & ZONE 7 & 1997 & 255.62 & 221.60 & 292.96 & RW2 \\ 
  Chad & ZONE 7 & 1998 & 254.45 & 220.22 & 292.81 & RW2 \\ 
  Chad & ZONE 7 & 1999 & 253.25 & 216.43 & 294.40 & RW2 \\ 
  Chad & ZONE 7 & 2000 & 251.75 & 214.67 & 292.64 & RW2 \\ 
  Chad & ZONE 7 & 2001 & 250.50 & 215.66 & 289.53 & RW2 \\ 
  Chad & ZONE 7 & 2002 & 249.03 & 215.29 & 286.69 & RW2 \\ 
  Chad & ZONE 7 & 2003 & 247.92 & 212.80 & 287.71 & RW2 \\ 
  Chad & ZONE 7 & 2004 & 246.22 & 208.73 & 288.38 & RW2 \\ 
  Chad & ZONE 7 & 2005 & 244.59 & 207.25 & 286.91 & RW2 \\ 
  Chad & ZONE 7 & 2006 & 242.09 & 206.54 & 282.00 & RW2 \\ 
  Chad & ZONE 7 & 2007 & 239.05 & 204.33 & 277.49 & RW2 \\ 
  Chad & ZONE 7 & 2008 & 235.10 & 199.46 & 274.90 & RW2 \\ 
  Chad & ZONE 7 & 2009 & 230.52 & 192.68 & 273.33 & RW2 \\ 
  Chad & ZONE 7 & 2010 & 225.05 & 185.75 & 269.93 & RW2 \\ 
  Chad & ZONE 7 & 2011 & 219.84 & 182.14 & 260.98 & RW2 \\ 
  Chad & ZONE 7 & 2012 & 214.21 & 179.42 & 253.46 & RW2 \\ 
  Chad & ZONE 7 & 2013 & 208.73 & 171.14 & 251.85 & RW2 \\ 
  Chad & ZONE 7 & 2014 & 203.10 & 154.20 & 261.75 & RW2 \\ 
  Chad & ZONE 7 & 2015 & 197.98 & 131.11 & 287.61 & RW2 \\ 
  Chad & ZONE 7 & 2016 & 193.00 & 109.78 & 315.94 & RW2 \\ 
  Chad & ZONE 7 & 2017 & 187.49 & 89.22 & 349.28 & RW2 \\ 
  Chad & ZONE 7 & 2018 & 181.99 & 71.40 & 390.03 & RW2 \\ 
  Chad & ZONE 7 & 2019 & 177.34 & 55.59 & 441.25 & RW2 \\ 
  Chad & ZONE 8 & 1980 & 263.83 & 182.12 & 363.60 & RW2 \\ 
  Chad & ZONE 8 & 1981 & 262.60 & 194.80 & 341.96 & RW2 \\ 
  Chad & ZONE 8 & 1982 & 261.29 & 200.50 & 331.11 & RW2 \\ 
  Chad & ZONE 8 & 1983 & 259.67 & 202.38 & 325.72 & RW2 \\ 
  Chad & ZONE 8 & 1984 & 258.23 & 202.50 & 322.80 & RW2 \\ 
  Chad & ZONE 8 & 1985 & 256.99 & 204.37 & 316.14 & RW2 \\ 
  Chad & ZONE 8 & 1986 & 255.21 & 206.72 & 309.94 & RW2 \\ 
  Chad & ZONE 8 & 1987 & 253.46 & 207.53 & 305.20 & RW2 \\ 
  Chad & ZONE 8 & 1988 & 251.74 & 206.75 & 302.40 & RW2 \\ 
  Chad & ZONE 8 & 1989 & 249.74 & 204.82 & 300.01 & RW2 \\ 
  Chad & ZONE 8 & 1990 & 247.50 & 204.98 & 296.50 & RW2 \\ 
  Chad & ZONE 8 & 1991 & 245.07 & 204.24 & 291.09 & RW2 \\ 
  Chad & ZONE 8 & 1992 & 242.08 & 203.11 & 285.79 & RW2 \\ 
  Chad & ZONE 8 & 1993 & 238.89 & 200.42 & 282.43 & RW2 \\ 
  Chad & ZONE 8 & 1994 & 235.28 & 196.39 & 279.30 & RW2 \\ 
  Chad & ZONE 8 & 1995 & 231.39 & 194.63 & 272.92 & RW2 \\ 
  Chad & ZONE 8 & 1996 & 227.51 & 193.42 & 265.75 & RW2 \\ 
  Chad & ZONE 8 & 1997 & 223.71 & 191.45 & 259.92 & RW2 \\ 
  Chad & ZONE 8 & 1998 & 219.90 & 186.93 & 256.67 & RW2 \\ 
  Chad & ZONE 8 & 1999 & 216.04 & 182.04 & 254.21 & RW2 \\ 
  Chad & ZONE 8 & 2000 & 212.40 & 179.19 & 250.02 & RW2 \\ 
  Chad & ZONE 8 & 2001 & 208.88 & 177.47 & 244.44 & RW2 \\ 
  Chad & ZONE 8 & 2002 & 205.68 & 175.89 & 239.68 & RW2 \\ 
  Chad & ZONE 8 & 2003 & 202.32 & 171.72 & 236.68 & RW2 \\ 
  Chad & ZONE 8 & 2004 & 199.07 & 167.52 & 234.87 & RW2 \\ 
  Chad & ZONE 8 & 2005 & 195.77 & 164.68 & 231.44 & RW2 \\ 
  Chad & ZONE 8 & 2006 & 192.03 & 163.22 & 224.90 & RW2 \\ 
  Chad & ZONE 8 & 2007 & 187.81 & 160.08 & 219.14 & RW2 \\ 
  Chad & ZONE 8 & 2008 & 183.17 & 154.66 & 215.44 & RW2 \\ 
  Chad & ZONE 8 & 2009 & 178.13 & 147.66 & 213.17 & RW2 \\ 
  Chad & ZONE 8 & 2010 & 172.49 & 140.57 & 209.64 & RW2 \\ 
  Chad & ZONE 8 & 2011 & 167.15 & 136.65 & 202.39 & RW2 \\ 
  Chad & ZONE 8 & 2012 & 161.63 & 132.84 & 195.02 & RW2 \\ 
  Chad & ZONE 8 & 2013 & 156.24 & 125.01 & 193.52 & RW2 \\ 
  Chad & ZONE 8 & 2014 & 150.91 & 111.82 & 200.03 & RW2 \\ 
  Chad & ZONE 8 & 2015 & 145.91 & 93.74 & 219.43 & RW2 \\ 
  Chad & ZONE 8 & 2016 & 140.76 & 77.11 & 241.77 & RW2 \\ 
  Chad & ZONE 8 & 2017 & 136.02 & 61.98 & 273.92 & RW2 \\ 
  Chad & ZONE 8 & 2018 & 131.39 & 48.70 & 307.44 & RW2 \\ 
  Chad & ZONE 8 & 2019 & 127.03 & 38.01 & 346.53 & RW2 \\ 
  Comoros & ALL & 1980 & 165.11 & 155.04 & 175.33 & IHME \\ 
  Comoros & ALL & 1980 & 177.18 & 111.68 & 269.23 & RW2 \\ 
  Comoros & ALL & 1980 & 175.30 & 157.10 & 197.90 & UN \\ 
  Comoros & ALL & 1981 & 160.33 & 150.36 & 170.52 & IHME \\ 
  Comoros & ALL & 1981 & 170.59 & 121.83 & 232.97 & RW2 \\ 
  Comoros & ALL & 1981 & 169.40 & 151.80 & 191.20 & UN \\ 
  Comoros & ALL & 1982 & 155.16 & 145.18 & 165.51 & IHME \\ 
  Comoros & ALL & 1982 & 164.25 & 122.01 & 217.64 & RW2 \\ 
  Comoros & ALL & 1982 & 163.60 & 146.30 & 184.30 & UN \\ 
  Comoros & ALL & 1983 & 150.29 & 140.39 & 160.46 & IHME \\ 
  Comoros & ALL & 1983 & 157.82 & 114.43 & 213.88 & RW2 \\ 
  Comoros & ALL & 1983 & 157.90 & 140.90 & 177.70 & UN \\ 
  Comoros & ALL & 1984 & 144.93 & 135.22 & 154.46 & IHME \\ 
  Comoros & ALL & 1984 & 152.15 & 106.46 & 212.21 & RW2 \\ 
  Comoros & ALL & 1984 & 152.40 & 135.40 & 171.40 & UN \\ 
  Comoros & ALL & 1985 & 140.05 & 130.92 & 149.56 & IHME \\ 
  Comoros & ALL & 1985 & 146.13 & 103.77 & 201.39 & RW2 \\ 
  Comoros & ALL & 1985 & 147.20 & 130.70 & 165.40 & UN \\ 
  Comoros & ALL & 1986 & 135.34 & 126.46 & 144.77 & IHME \\ 
  Comoros & ALL & 1986 & 140.95 & 101.63 & 191.59 & RW2 \\ 
  Comoros & ALL & 1986 & 142.20 & 126.10 & 159.60 & UN \\ 
  Comoros & ALL & 1987 & 131.37 & 122.67 & 140.81 & IHME \\ 
  Comoros & ALL & 1987 & 136.31 & 99.64 & 184.06 & RW2 \\ 
  Comoros & ALL & 1987 & 137.60 & 122.00 & 154.30 & UN \\ 
  Comoros & ALL & 1988 & 127.32 & 119.08 & 136.23 & IHME \\ 
  Comoros & ALL & 1988 & 131.92 & 95.20 & 179.34 & RW2 \\ 
  Comoros & ALL & 1988 & 133.20 & 118.40 & 149.30 & UN \\ 
  Comoros & ALL & 1989 & 123.98 & 116.13 & 132.47 & IHME \\ 
  Comoros & ALL & 1989 & 127.93 & 91.27 & 175.98 & RW2 \\ 
  Comoros & ALL & 1989 & 129.10 & 114.70 & 144.20 & UN \\ 
  Comoros & ALL & 1990 & 120.20 & 112.10 & 128.60 & IHME \\ 
  Comoros & ALL & 1990 & 124.46 & 89.14 & 172.18 & RW2 \\ 
  Comoros & ALL & 1990 & 125.10 & 111.30 & 139.70 & UN \\ 
  Comoros & ALL & 1991 & 116.70 & 108.59 & 125.21 & IHME \\ 
  Comoros & ALL & 1991 & 120.98 & 87.83 & 164.18 & RW2 \\ 
  Comoros & ALL & 1991 & 121.20 & 107.50 & 135.50 & UN \\ 
  Comoros & ALL & 1992 & 113.00 & 104.84 & 121.89 & IHME \\ 
  Comoros & ALL & 1992 & 117.61 & 85.81 & 158.84 & RW2 \\ 
  Comoros & ALL & 1992 & 117.20 & 103.50 & 131.70 & UN \\ 
  Comoros & ALL & 1993 & 109.36 & 101.34 & 118.52 & IHME \\ 
  Comoros & ALL & 1993 & 114.38 & 82.53 & 156.58 & RW2 \\ 
  Comoros & ALL & 1993 & 113.60 & 99.00 & 128.30 & UN \\ 
  Comoros & ALL & 1994 & 106.08 & 97.59 & 116.15 & IHME \\ 
  Comoros & ALL & 1994 & 111.16 & 77.78 & 157.41 & RW2 \\ 
  Comoros & ALL & 1994 & 110.50 & 94.70 & 125.40 & UN \\ 
  Comoros & ALL & 1995 & 102.70 & 93.77 & 113.09 & IHME \\ 
  Comoros & ALL & 1995 & 108.10 & 72.48 & 157.29 & RW2 \\ 
  Comoros & ALL & 1995 & 107.80 & 90.90 & 123.00 & UN \\ 
  Comoros & ALL & 1996 & 99.82 & 89.87 & 111.18 & IHME \\ 
  Comoros & ALL & 1996 & 105.21 & 69.07 & 157.03 & RW2 \\ 
  Comoros & ALL & 1996 & 105.60 & 88.00 & 121.00 & UN \\ 
  Comoros & ALL & 1997 & 96.14 & 84.77 & 108.43 & IHME \\ 
  Comoros & ALL & 1997 & 102.61 & 65.46 & 156.83 & RW2 \\ 
  Comoros & ALL & 1997 & 104.00 & 85.70 & 119.70 & UN \\ 
  Comoros & ALL & 1998 & 92.70 & 80.14 & 105.80 & IHME \\ 
  Comoros & ALL & 1998 & 100.48 & 60.68 & 162.11 & RW2 \\ 
  Comoros & ALL & 1998 & 102.70 & 83.80 & 118.90 & UN \\ 
  Comoros & ALL & 1999 & 88.89 & 75.51 & 102.87 & IHME \\ 
  Comoros & ALL & 1999 & 98.53 & 54.86 & 168.71 & RW2 \\ 
  Comoros & ALL & 1999 & 101.80 & 82.40 & 118.70 & UN \\ 
  Comoros & ALL & 2000 & 85.41 & 71.38 & 99.61 & IHME \\ 
  Comoros & ALL & 2000 & 96.87 & 48.06 & 183.93 & RW2 \\ 
  Comoros & ALL & 2000 & 101.10 & 80.90 & 119.10 & UN \\ 
  Comoros & ALL & 2001 & 81.91 & 67.26 & 96.94 & IHME \\ 
  Comoros & ALL & 2001 & 95.43 & 42.56 & 198.23 & RW2 \\ 
  Comoros & ALL & 2001 & 100.80 & 78.70 & 119.80 & UN \\ 
  Comoros & ALL & 2002 & 78.46 & 63.37 & 94.41 & IHME \\ 
  Comoros & ALL & 2002 & 94.07 & 38.55 & 210.67 & RW2 \\ 
  Comoros & ALL & 2002 & 100.80 & 76.40 & 121.80 & UN \\ 
  Comoros & ALL & 2003 & 75.05 & 59.58 & 92.22 & IHME \\ 
  Comoros & ALL & 2003 & 93.24 & 35.83 & 219.85 & RW2 \\ 
  Comoros & ALL & 2003 & 100.40 & 73.70 & 125.20 & UN \\ 
  Comoros & ALL & 2004 & 71.85 & 56.39 & 90.08 & IHME \\ 
  Comoros & ALL & 2004 & 91.74 & 33.22 & 226.59 & RW2 \\ 
  Comoros & ALL & 2004 & 99.40 & 70.90 & 129.60 & UN \\ 
  Comoros & ALL & 2005 & 68.79 & 53.04 & 87.68 & IHME \\ 
  Comoros & ALL & 2005 & 91.20 & 32.09 & 232.02 & RW2 \\ 
  Comoros & ALL & 2005 & 97.80 & 68.10 & 133.80 & UN \\ 
  Comoros & ALL & 2006 & 65.81 & 49.97 & 84.90 & IHME \\ 
  Comoros & ALL & 2006 & 90.20 & 31.53 & 230.59 & RW2 \\ 
  Comoros & ALL & 2006 & 95.70 & 65.30 & 137.00 & UN \\ 
  Comoros & ALL & 2007 & 63.01 & 47.33 & 82.07 & IHME \\ 
  Comoros & ALL & 2007 & 89.52 & 32.31 & 224.22 & RW2 \\ 
  Comoros & ALL & 2007 & 93.50 & 62.20 & 139.00 & UN \\ 
  Comoros & ALL & 2008 & 60.12 & 44.66 & 79.21 & IHME \\ 
  Comoros & ALL & 2008 & 89.47 & 34.07 & 215.11 & RW2 \\ 
  Comoros & ALL & 2008 & 91.20 & 59.30 & 139.90 & UN \\ 
  Comoros & ALL & 2009 & 57.91 & 42.55 & 77.15 & IHME \\ 
  Comoros & ALL & 2009 & 88.58 & 36.50 & 201.76 & RW2 \\ 
  Comoros & ALL & 2009 & 88.50 & 56.10 & 141.90 & UN \\ 
  Comoros & ALL & 2010 & 55.06 & 39.80 & 74.18 & IHME \\ 
  Comoros & ALL & 2010 & 88.09 & 40.35 & 184.39 & RW2 \\ 
  Comoros & ALL & 2010 & 86.00 & 52.80 & 142.10 & UN \\ 
  Comoros & ALL & 2011 & 52.94 & 38.06 & 72.08 & IHME \\ 
  Comoros & ALL & 2011 & 87.87 & 42.18 & 174.46 & RW2 \\ 
  Comoros & ALL & 2011 & 83.30 & 50.20 & 143.30 & UN \\ 
  Comoros & ALL & 2012 & 50.48 & 35.91 & 69.24 & IHME \\ 
  Comoros & ALL & 2012 & 87.42 & 41.62 & 174.49 & RW2 \\ 
  Comoros & ALL & 2012 & 80.90 & 47.30 & 143.30 & UN \\ 
  Comoros & ALL & 2013 & 48.25 & 34.33 & 66.85 & IHME \\ 
  Comoros & ALL & 2013 & 87.23 & 37.71 & 188.52 & RW2 \\ 
  Comoros & ALL & 2013 & 78.30 & 44.50 & 143.90 & UN \\ 
  Comoros & ALL & 2014 & 46.05 & 32.70 & 63.80 & IHME \\ 
  Comoros & ALL & 2014 & 86.71 & 31.01 & 220.31 & RW2 \\ 
  Comoros & ALL & 2014 & 75.90 & 42.00 & 144.10 & UN \\ 
  Comoros & ALL & 2015 & 44.02 & 30.70 & 61.97 & IHME \\ 
  Comoros & ALL & 2015 & 85.96 & 23.35 & 272.72 & RW2 \\ 
  Comoros & ALL & 2015 & 73.50 & 39.40 & 144.60 & UN \\ 
  Comoros & ALL & 2016 & 85.99 & 17.19 & 343.20 & RW2 \\ 
  Comoros & ALL & 2017 & 85.11 & 11.90 & 428.65 & RW2 \\ 
  Comoros & ALL & 2018 & 84.61 & 8.19 & 530.50 & RW2 \\ 
  Comoros & ALL & 2019 & 83.91 & 5.05 & 624.13 & RW2 \\ 
  Comoros & ALL & 80-84 & 154.54 & 177.06 & 134.41 & HT-Direct \\ 
  Comoros & ALL & 85-89 & 117.34 & 135.28 & 101.50 & HT-Direct \\ 
  Comoros & ALL & 90-94 & 97.44 & 110.20 & 86.02 & HT-Direct \\ 
  Comoros & ALL & 95-99 & 70.08 & 86.12 & 56.84 & HT-Direct \\ 
  Comoros & ALL & 00-04 & 51.26 & 64.86 & 40.39 & HT-Direct \\ 
  Comoros & ALL & 05-09 & 47.04 & 59.62 & 37.01 & HT-Direct \\ 
  Comoros & ALL & 10-14 & 57.99 & 78.48 & 42.60 & HT-Direct \\ 
  Comoros & ALL & 15-19 & 85.10 & 12.19 & 420.77 & RW2 \\ 
  Comoros & MOHELI & 1980 & 187.03 & 103.42 & 317.17 & RW2 \\ 
  Comoros & MOHELI & 1981 & 177.08 & 106.57 & 280.26 & RW2 \\ 
  Comoros & MOHELI & 1982 & 167.66 & 105.57 & 256.48 & RW2 \\ 
  Comoros & MOHELI & 1983 & 158.60 & 101.26 & 240.42 & RW2 \\ 
  Comoros & MOHELI & 1984 & 150.39 & 96.36 & 227.24 & RW2 \\ 
  Comoros & MOHELI & 1985 & 142.06 & 92.40 & 211.72 & RW2 \\ 
  Comoros & MOHELI & 1986 & 135.28 & 89.66 & 198.05 & RW2 \\ 
  Comoros & MOHELI & 1987 & 128.77 & 86.33 & 188.12 & RW2 \\ 
  Comoros & MOHELI & 1988 & 123.50 & 82.48 & 179.66 & RW2 \\ 
  Comoros & MOHELI & 1989 & 118.76 & 79.20 & 173.82 & RW2 \\ 
  Comoros & MOHELI & 1990 & 114.82 & 76.43 & 168.73 & RW2 \\ 
  Comoros & MOHELI & 1991 & 110.97 & 73.94 & 163.36 & RW2 \\ 
  Comoros & MOHELI & 1992 & 107.39 & 71.08 & 158.42 & RW2 \\ 
  Comoros & MOHELI & 1993 & 104.10 & 68.05 & 155.71 & RW2 \\ 
  Comoros & MOHELI & 1994 & 101.05 & 64.14 & 155.54 & RW2 \\ 
  Comoros & MOHELI & 1995 & 97.80 & 60.28 & 154.22 & RW2 \\ 
  Comoros & MOHELI & 1996 & 95.20 & 57.15 & 154.24 & RW2 \\ 
  Comoros & MOHELI & 1997 & 92.69 & 53.81 & 154.83 & RW2 \\ 
  Comoros & MOHELI & 1998 & 90.49 & 49.83 & 158.32 & RW2 \\ 
  Comoros & MOHELI & 1999 & 88.60 & 45.53 & 165.70 & RW2 \\ 
  Comoros & MOHELI & 2000 & 86.60 & 40.39 & 175.52 & RW2 \\ 
  Comoros & MOHELI & 2001 & 85.35 & 36.97 & 186.02 & RW2 \\ 
  Comoros & MOHELI & 2002 & 84.33 & 33.15 & 196.68 & RW2 \\ 
  Comoros & MOHELI & 2003 & 83.04 & 30.16 & 205.56 & RW2 \\ 
  Comoros & MOHELI & 2004 & 82.16 & 28.12 & 214.58 & RW2 \\ 
  Comoros & MOHELI & 2005 & 81.26 & 26.55 & 220.72 & RW2 \\ 
  Comoros & MOHELI & 2006 & 80.52 & 25.42 & 224.90 & RW2 \\ 
  Comoros & MOHELI & 2007 & 79.75 & 25.04 & 230.59 & RW2 \\ 
  Comoros & MOHELI & 2008 & 79.36 & 24.47 & 227.34 & RW2 \\ 
  Comoros & MOHELI & 2009 & 78.61 & 24.50 & 227.01 & RW2 \\ 
  Comoros & MOHELI & 2010 & 78.47 & 24.36 & 225.34 & RW2 \\ 
  Comoros & MOHELI & 2011 & 77.84 & 23.68 & 229.60 & RW2 \\ 
  Comoros & MOHELI & 2012 & 77.56 & 22.60 & 238.69 & RW2 \\ 
  Comoros & MOHELI & 2013 & 77.14 & 20.04 & 254.34 & RW2 \\ 
  Comoros & MOHELI & 2014 & 76.51 & 17.09 & 286.40 & RW2 \\ 
  Comoros & MOHELI & 2015 & 76.31 & 13.96 & 330.40 & RW2 \\ 
  Comoros & MOHELI & 2016 & 75.70 & 11.00 & 381.33 & RW2 \\ 
  Comoros & MOHELI & 2017 & 75.05 & 8.33 & 447.43 & RW2 \\ 
  Comoros & MOHELI & 2018 & 74.71 & 6.07 & 526.09 & RW2 \\ 
  Comoros & MOHELI & 2019 & 73.95 & 4.34 & 609.49 & RW2 \\ 
  Comoros & NDZOUANI & 1980 & 206.07 & 132.09 & 306.67 & RW2 \\ 
  Comoros & NDZOUANI & 1981 & 195.74 & 139.31 & 267.28 & RW2 \\ 
  Comoros & NDZOUANI & 1982 & 185.62 & 138.08 & 244.94 & RW2 \\ 
  Comoros & NDZOUANI & 1983 & 175.53 & 131.31 & 231.73 & RW2 \\ 
  Comoros & NDZOUANI & 1984 & 166.64 & 123.78 & 220.65 & RW2 \\ 
  Comoros & NDZOUANI & 1985 & 157.94 & 118.46 & 207.18 & RW2 \\ 
  Comoros & NDZOUANI & 1986 & 150.30 & 114.60 & 194.32 & RW2 \\ 
  Comoros & NDZOUANI & 1987 & 143.62 & 110.40 & 184.33 & RW2 \\ 
  Comoros & NDZOUANI & 1988 & 137.55 & 105.40 & 176.62 & RW2 \\ 
  Comoros & NDZOUANI & 1989 & 132.31 & 100.55 & 171.65 & RW2 \\ 
  Comoros & NDZOUANI & 1990 & 127.64 & 96.78 & 166.33 & RW2 \\ 
  Comoros & NDZOUANI & 1991 & 123.21 & 94.38 & 159.73 & RW2 \\ 
  Comoros & NDZOUANI & 1992 & 119.18 & 90.99 & 154.67 & RW2 \\ 
  Comoros & NDZOUANI & 1993 & 115.17 & 85.91 & 152.89 & RW2 \\ 
  Comoros & NDZOUANI & 1994 & 111.47 & 80.37 & 152.72 & RW2 \\ 
  Comoros & NDZOUANI & 1995 & 107.64 & 74.70 & 152.74 & RW2 \\ 
  Comoros & NDZOUANI & 1996 & 104.36 & 69.87 & 152.35 & RW2 \\ 
  Comoros & NDZOUANI & 1997 & 101.27 & 65.30 & 153.65 & RW2 \\ 
  Comoros & NDZOUANI & 1998 & 98.59 & 60.08 & 157.14 & RW2 \\ 
  Comoros & NDZOUANI & 1999 & 95.71 & 53.91 & 163.40 & RW2 \\ 
  Comoros & NDZOUANI & 2000 & 93.71 & 48.08 & 172.37 & RW2 \\ 
  Comoros & NDZOUANI & 2001 & 91.62 & 43.29 & 181.81 & RW2 \\ 
  Comoros & NDZOUANI & 2002 & 90.09 & 39.51 & 191.46 & RW2 \\ 
  Comoros & NDZOUANI & 2003 & 88.33 & 35.85 & 198.15 & RW2 \\ 
  Comoros & NDZOUANI & 2004 & 87.04 & 33.58 & 205.86 & RW2 \\ 
  Comoros & NDZOUANI & 2005 & 85.46 & 32.14 & 207.76 & RW2 \\ 
  Comoros & NDZOUANI & 2006 & 84.28 & 31.03 & 207.08 & RW2 \\ 
  Comoros & NDZOUANI & 2007 & 83.24 & 31.13 & 202.34 & RW2 \\ 
  Comoros & NDZOUANI & 2008 & 81.92 & 31.69 & 197.06 & RW2 \\ 
  Comoros & NDZOUANI & 2009 & 81.05 & 32.33 & 188.27 & RW2 \\ 
  Comoros & NDZOUANI & 2010 & 80.08 & 33.09 & 180.90 & RW2 \\ 
  Comoros & NDZOUANI & 2011 & 79.31 & 33.56 & 177.27 & RW2 \\ 
  Comoros & NDZOUANI & 2012 & 78.46 & 32.27 & 178.49 & RW2 \\ 
  Comoros & NDZOUANI & 2013 & 77.56 & 29.23 & 190.53 & RW2 \\ 
  Comoros & NDZOUANI & 2014 & 76.66 & 24.83 & 213.89 & RW2 \\ 
  Comoros & NDZOUANI & 2015 & 75.93 & 19.62 & 252.42 & RW2 \\ 
  Comoros & NDZOUANI & 2016 & 74.97 & 15.06 & 301.74 & RW2 \\ 
  Comoros & NDZOUANI & 2017 & 73.91 & 11.26 & 362.51 & RW2 \\ 
  Comoros & NDZOUANI & 2018 & 73.16 & 8.14 & 439.48 & RW2 \\ 
  Comoros & NDZOUANI & 2019 & 72.39 & 5.59 & 524.74 & RW2 \\ 
  Comoros & NGAZIDJA & 1980 & 157.54 & 106.70 & 227.81 & RW2 \\ 
  Comoros & NGAZIDJA & 1981 & 152.02 & 115.87 & 197.55 & RW2 \\ 
  Comoros & NGAZIDJA & 1982 & 146.52 & 115.49 & 184.27 & RW2 \\ 
  Comoros & NGAZIDJA & 1983 & 141.15 & 109.16 & 181.30 & RW2 \\ 
  Comoros & NGAZIDJA & 1984 & 136.09 & 102.32 & 178.01 & RW2 \\ 
  Comoros & NGAZIDJA & 1985 & 131.38 & 98.96 & 172.12 & RW2 \\ 
  Comoros & NGAZIDJA & 1986 & 127.41 & 96.70 & 165.33 & RW2 \\ 
  Comoros & NGAZIDJA & 1987 & 123.90 & 95.52 & 159.37 & RW2 \\ 
  Comoros & NGAZIDJA & 1988 & 121.13 & 92.68 & 156.49 & RW2 \\ 
  Comoros & NGAZIDJA & 1989 & 118.67 & 89.76 & 154.36 & RW2 \\ 
  Comoros & NGAZIDJA & 1990 & 116.80 & 88.54 & 152.62 & RW2 \\ 
  Comoros & NGAZIDJA & 1991 & 114.98 & 88.30 & 148.82 & RW2 \\ 
  Comoros & NGAZIDJA & 1992 & 113.39 & 87.35 & 146.45 & RW2 \\ 
  Comoros & NGAZIDJA & 1993 & 111.87 & 84.57 & 146.62 & RW2 \\ 
  Comoros & NGAZIDJA & 1994 & 110.27 & 80.89 & 149.22 & RW2 \\ 
  Comoros & NGAZIDJA & 1995 & 108.75 & 76.57 & 152.39 & RW2 \\ 
  Comoros & NGAZIDJA & 1996 & 107.46 & 73.67 & 154.41 & RW2 \\ 
  Comoros & NGAZIDJA & 1997 & 106.19 & 70.32 & 157.25 & RW2 \\ 
  Comoros & NGAZIDJA & 1998 & 105.19 & 65.86 & 163.38 & RW2 \\ 
  Comoros & NGAZIDJA & 1999 & 104.43 & 60.98 & 172.63 & RW2 \\ 
  Comoros & NGAZIDJA & 2000 & 103.70 & 55.25 & 186.42 & RW2 \\ 
  Comoros & NGAZIDJA & 2001 & 103.55 & 50.65 & 198.91 & RW2 \\ 
  Comoros & NGAZIDJA & 2002 & 102.73 & 46.49 & 211.03 & RW2 \\ 
  Comoros & NGAZIDJA & 2003 & 102.73 & 43.50 & 223.51 & RW2 \\ 
  Comoros & NGAZIDJA & 2004 & 102.45 & 41.55 & 232.50 & RW2 \\ 
  Comoros & NGAZIDJA & 2005 & 102.41 & 39.88 & 237.49 & RW2 \\ 
  Comoros & NGAZIDJA & 2006 & 102.78 & 39.66 & 237.21 & RW2 \\ 
  Comoros & NGAZIDJA & 2007 & 102.79 & 40.43 & 236.55 & RW2 \\ 
  Comoros & NGAZIDJA & 2008 & 102.74 & 41.62 & 233.14 & RW2 \\ 
  Comoros & NGAZIDJA & 2009 & 103.28 & 42.95 & 226.94 & RW2 \\ 
  Comoros & NGAZIDJA & 2010 & 103.47 & 45.26 & 218.75 & RW2 \\ 
  Comoros & NGAZIDJA & 2011 & 104.00 & 46.47 & 215.97 & RW2 \\ 
  Comoros & NGAZIDJA & 2012 & 104.46 & 45.20 & 221.54 & RW2 \\ 
  Comoros & NGAZIDJA & 2013 & 104.97 & 41.58 & 240.35 & RW2 \\ 
  Comoros & NGAZIDJA & 2014 & 104.54 & 35.42 & 270.39 & RW2 \\ 
  Comoros & NGAZIDJA & 2015 & 105.30 & 28.83 & 317.78 & RW2 \\ 
  Comoros & NGAZIDJA & 2016 & 105.67 & 22.22 & 383.84 & RW2 \\ 
  Comoros & NGAZIDJA & 2017 & 105.64 & 16.78 & 454.35 & RW2 \\ 
  Comoros & NGAZIDJA & 2018 & 106.48 & 12.24 & 536.06 & RW2 \\ 
  Comoros & NGAZIDJA & 2019 & 106.63 & 8.44 & 627.24 & RW2 \\ 
  Congo & ALL & 1980 & 121.14 & 105.35 & 137.76 & IHME \\ 
  Congo & ALL & 1980 & 112.71 & 71.97 & 172.45 & RW2 \\ 
  Congo & ALL & 1980 & 111.90 & 89.00 & 142.90 & UN \\ 
  Congo & ALL & 1981 & 116.44 & 103.66 & 129.86 & IHME \\ 
  Congo & ALL & 1981 & 109.60 & 78.12 & 151.55 & RW2 \\ 
  Congo & ALL & 1981 & 109.10 & 88.50 & 135.70 & UN \\ 
  Congo & ALL & 1982 & 111.97 & 102.21 & 123.06 & IHME \\ 
  Congo & ALL & 1982 & 106.64 & 80.18 & 140.59 & RW2 \\ 
  Congo & ALL & 1982 & 106.20 & 87.90 & 128.80 & UN \\ 
  Congo & ALL & 1983 & 107.73 & 99.41 & 116.90 & IHME \\ 
  Congo & ALL & 1983 & 103.55 & 78.28 & 136.25 & RW2 \\ 
  Congo & ALL & 1983 & 103.20 & 86.80 & 123.30 & UN \\ 
  Congo & ALL & 1984 & 103.61 & 95.89 & 111.62 & IHME \\ 
  Congo & ALL & 1984 & 100.98 & 75.39 & 134.22 & RW2 \\ 
  Congo & ALL & 1984 & 100.20 & 85.60 & 118.10 & UN \\ 
  Congo & ALL & 1985 & 100.42 & 93.09 & 107.69 & IHME \\ 
  Congo & ALL & 1985 & 98.24 & 74.77 & 127.99 & RW2 \\ 
  Congo & ALL & 1985 & 97.50 & 84.00 & 113.80 & UN \\ 
  Congo & ALL & 1986 & 97.87 & 91.00 & 104.95 & IHME \\ 
  Congo & ALL & 1986 & 96.19 & 74.42 & 123.35 & RW2 \\ 
  Congo & ALL & 1986 & 95.30 & 82.60 & 110.40 & UN \\ 
  Congo & ALL & 1987 & 95.92 & 89.35 & 102.92 & IHME \\ 
  Congo & ALL & 1987 & 94.66 & 74.20 & 120.49 & RW2 \\ 
  Congo & ALL & 1987 & 93.70 & 81.60 & 107.80 & UN \\ 
  Congo & ALL & 1988 & 94.43 & 88.36 & 101.16 & IHME \\ 
  Congo & ALL & 1988 & 93.59 & 72.89 & 119.72 & RW2 \\ 
  Congo & ALL & 1988 & 92.80 & 81.10 & 106.10 & UN \\ 
  Congo & ALL & 1989 & 93.44 & 87.32 & 99.89 & IHME \\ 
  Congo & ALL & 1989 & 93.21 & 72.09 & 120.26 & RW2 \\ 
  Congo & ALL & 1989 & 92.80 & 81.30 & 105.80 & UN \\ 
  Congo & ALL & 1990 & 93.35 & 87.29 & 99.37 & IHME \\ 
  Congo & ALL & 1990 & 93.39 & 72.21 & 119.58 & RW2 \\ 
  Congo & ALL & 1990 & 93.60 & 82.40 & 106.40 & UN \\ 
  Congo & ALL & 1991 & 93.82 & 87.70 & 99.83 & IHME \\ 
  Congo & ALL & 1991 & 94.68 & 74.18 & 119.37 & RW2 \\ 
  Congo & ALL & 1991 & 95.20 & 84.00 & 107.60 & UN \\ 
  Congo & ALL & 1992 & 94.67 & 88.50 & 100.78 & IHME \\ 
  Congo & ALL & 1992 & 96.91 & 76.38 & 121.51 & RW2 \\ 
  Congo & ALL & 1992 & 97.40 & 86.50 & 109.70 & UN \\ 
  Congo & ALL & 1993 & 96.37 & 90.11 & 102.51 & IHME \\ 
  Congo & ALL & 1993 & 100.08 & 78.54 & 125.90 & RW2 \\ 
  Congo & ALL & 1993 & 100.20 & 89.40 & 112.30 & UN \\ 
  Congo & ALL & 1994 & 98.31 & 91.74 & 105.07 & IHME \\ 
  Congo & ALL & 1994 & 103.94 & 80.68 & 132.10 & RW2 \\ 
  Congo & ALL & 1994 & 103.80 & 93.00 & 115.80 & UN \\ 
  Congo & ALL & 1995 & 100.37 & 93.35 & 107.96 & IHME \\ 
  Congo & ALL & 1995 & 108.81 & 85.97 & 138.08 & RW2 \\ 
  Congo & ALL & 1995 & 107.70 & 96.70 & 119.90 & UN \\ 
  Congo & ALL & 1996 & 102.79 & 95.19 & 111.15 & IHME \\ 
  Congo & ALL & 1996 & 113.16 & 90.58 & 141.83 & RW2 \\ 
  Congo & ALL & 1996 & 111.90 & 100.70 & 124.20 & UN \\ 
  Congo & ALL & 1997 & 115.77 & 103.91 & 129.70 & IHME \\ 
  Congo & ALL & 1997 & 117.11 & 94.24 & 144.75 & RW2 \\ 
  Congo & ALL & 1997 & 116.00 & 104.60 & 128.50 & UN \\ 
  Congo & ALL & 1998 & 110.97 & 102.53 & 120.36 & IHME \\ 
  Congo & ALL & 1998 & 120.21 & 95.84 & 149.66 & RW2 \\ 
  Congo & ALL & 1998 & 119.30 & 107.80 & 132.00 & UN \\ 
  Congo & ALL & 1999 & 109.84 & 101.56 & 118.87 & IHME \\ 
  Congo & ALL & 1999 & 121.85 & 95.73 & 152.35 & RW2 \\ 
  Congo & ALL & 1999 & 121.40 & 109.60 & 134.40 & UN \\ 
  Congo & ALL & 2000 & 109.38 & 101.70 & 118.22 & IHME \\ 
  Congo & ALL & 2000 & 122.08 & 96.75 & 153.08 & RW2 \\ 
  Congo & ALL & 2000 & 121.60 & 109.80 & 134.80 & UN \\ 
  Congo & ALL & 2001 & 107.56 & 100.29 & 115.81 & IHME \\ 
  Congo & ALL & 2001 & 119.94 & 95.91 & 148.87 & RW2 \\ 
  Congo & ALL & 2001 & 119.70 & 108.00 & 132.50 & UN \\ 
  Congo & ALL & 2002 & 104.95 & 98.62 & 111.97 & IHME \\ 
  Congo & ALL & 2002 & 115.69 & 93.30 & 142.85 & RW2 \\ 
  Congo & ALL & 2002 & 115.60 & 104.60 & 127.90 & UN \\ 
  Congo & ALL & 2003 & 101.02 & 95.14 & 107.61 & IHME \\ 
  Congo & ALL & 2003 & 109.72 & 88.21 & 136.62 & RW2 \\ 
  Congo & ALL & 2003 & 109.80 & 99.30 & 121.40 & UN \\ 
  Congo & ALL & 2004 & 95.77 & 90.00 & 102.06 & IHME \\ 
  Congo & ALL & 2004 & 102.32 & 80.71 & 130.06 & RW2 \\ 
  Congo & ALL & 2004 & 102.90 & 92.80 & 113.90 & UN \\ 
  Congo & ALL & 2005 & 90.04 & 84.01 & 96.37 & IHME \\ 
  Congo & ALL & 2005 & 93.82 & 72.11 & 119.04 & RW2 \\ 
  Congo & ALL & 2005 & 95.30 & 85.30 & 105.90 & UN \\ 
  Congo & ALL & 2006 & 84.32 & 78.20 & 90.52 & IHME \\ 
  Congo & ALL & 2006 & 86.09 & 66.78 & 108.82 & RW2 \\ 
  Congo & ALL & 2006 & 87.40 & 77.80 & 98.10 & UN \\ 
  Congo & ALL & 2007 & 79.33 & 73.07 & 85.46 & IHME \\ 
  Congo & ALL & 2007 & 78.97 & 61.71 & 99.83 & RW2 \\ 
  Congo & ALL & 2007 & 79.90 & 70.30 & 90.50 & UN \\ 
  Congo & ALL & 2008 & 74.72 & 68.71 & 80.79 & IHME \\ 
  Congo & ALL & 2008 & 72.74 & 56.27 & 93.10 & RW2 \\ 
  Congo & ALL & 2008 & 72.60 & 62.90 & 83.80 & UN \\ 
  Congo & ALL & 2009 & 71.09 & 64.96 & 77.65 & IHME \\ 
  Congo & ALL & 2009 & 67.11 & 50.92 & 87.69 & RW2 \\ 
  Congo & ALL & 2009 & 66.10 & 55.80 & 78.10 & UN \\ 
  Congo & ALL & 2010 & 67.99 & 61.02 & 75.35 & IHME \\ 
  Congo & ALL & 2010 & 62.48 & 46.97 & 83.61 & RW2 \\ 
  Congo & ALL & 2010 & 60.60 & 49.50 & 73.90 & UN \\ 
  Congo & ALL & 2011 & 65.07 & 57.13 & 73.58 & IHME \\ 
  Congo & ALL & 2011 & 58.12 & 43.99 & 76.88 & RW2 \\ 
  Congo & ALL & 2011 & 56.10 & 44.20 & 70.80 & UN \\ 
  Congo & ALL & 2012 & 62.80 & 53.72 & 72.67 & IHME \\ 
  Congo & ALL & 2012 & 54.11 & 41.04 & 71.26 & RW2 \\ 
  Congo & ALL & 2012 & 52.60 & 39.40 & 69.00 & UN \\ 
  Congo & ALL & 2013 & 60.16 & 49.94 & 71.68 & IHME \\ 
  Congo & ALL & 2013 & 50.47 & 36.61 & 69.17 & RW2 \\ 
  Congo & ALL & 2013 & 49.60 & 35.60 & 67.80 & UN \\ 
  Congo & ALL & 2014 & 58.15 & 47.10 & 71.48 & IHME \\ 
  Congo & ALL & 2014 & 46.97 & 30.36 & 72.35 & RW2 \\ 
  Congo & ALL & 2014 & 47.10 & 32.30 & 67.00 & UN \\ 
  Congo & ALL & 2015 & 56.13 & 44.35 & 70.14 & IHME \\ 
  Congo & ALL & 2015 & 43.60 & 23.17 & 81.75 & RW2 \\ 
  Congo & ALL & 2015 & 45.00 & 29.60 & 66.50 & UN \\ 
  Congo & ALL & 2016 & 40.66 & 17.60 & 93.89 & RW2 \\ 
  Congo & ALL & 2017 & 37.69 & 12.82 & 110.17 & RW2 \\ 
  Congo & ALL & 2018 & 35.01 & 9.23 & 132.76 & RW2 \\ 
  Congo & ALL & 2019 & 32.46 & 6.21 & 158.38 & RW2 \\ 
  Congo & ALL & 80-84 & 102.46 & 124.52 & 83.94 & HT-Direct \\ 
  Congo & ALL & 85-89 & 93.29 & 106.80 & 81.33 & HT-Direct \\ 
  Congo & ALL & 90-94 & 95.32 & 106.80 & 84.96 & HT-Direct \\ 
  Congo & ALL & 95-99 & 126.25 & 138.06 & 115.32 & HT-Direct \\ 
  Congo & ALL & 00-04 & 112.73 & 121.45 & 104.56 & HT-Direct \\ 
  Congo & ALL & 05-09 & 70.20 & 78.63 & 62.61 & HT-Direct \\ 
  Congo & ALL & 10-14 & 66.38 & 81.82 & 53.68 & HT-Direct \\ 
  Congo & ALL & 15-19 & 37.68 & 13.06 & 108.16 & RW2 \\ 
  Congo & BRAZZAVILLE & 1980 & 78.24 & 46.77 & 128.88 & RW2 \\ 
  Congo & BRAZZAVILLE & 1981 & 76.62 & 50.50 & 114.85 & RW2 \\ 
  Congo & BRAZZAVILLE & 1982 & 75.09 & 52.23 & 106.99 & RW2 \\ 
  Congo & BRAZZAVILLE & 1983 & 73.55 & 52.19 & 102.89 & RW2 \\ 
  Congo & BRAZZAVILLE & 1984 & 72.17 & 51.69 & 100.05 & RW2 \\ 
  Congo & BRAZZAVILLE & 1985 & 70.74 & 51.89 & 95.74 & RW2 \\ 
  Congo & BRAZZAVILLE & 1986 & 69.61 & 52.33 & 91.80 & RW2 \\ 
  Congo & BRAZZAVILLE & 1987 & 68.51 & 52.38 & 89.47 & RW2 \\ 
  Congo & BRAZZAVILLE & 1988 & 67.93 & 52.07 & 88.23 & RW2 \\ 
  Congo & BRAZZAVILLE & 1989 & 67.88 & 52.17 & 88.39 & RW2 \\ 
  Congo & BRAZZAVILLE & 1990 & 68.23 & 52.62 & 87.48 & RW2 \\ 
  Congo & BRAZZAVILLE & 1991 & 70.17 & 55.08 & 88.78 & RW2 \\ 
  Congo & BRAZZAVILLE & 1992 & 73.51 & 58.28 & 91.87 & RW2 \\ 
  Congo & BRAZZAVILLE & 1993 & 78.37 & 62.15 & 97.64 & RW2 \\ 
  Congo & BRAZZAVILLE & 1994 & 84.56 & 66.42 & 105.81 & RW2 \\ 
  Congo & BRAZZAVILLE & 1995 & 92.38 & 74.25 & 115.45 & RW2 \\ 
  Congo & BRAZZAVILLE & 1996 & 99.34 & 81.22 & 121.92 & RW2 \\ 
  Congo & BRAZZAVILLE & 1997 & 105.26 & 86.78 & 127.55 & RW2 \\ 
  Congo & BRAZZAVILLE & 1998 & 109.45 & 89.60 & 133.33 & RW2 \\ 
  Congo & BRAZZAVILLE & 1999 & 111.54 & 90.24 & 138.01 & RW2 \\ 
  Congo & BRAZZAVILLE & 2000 & 110.87 & 88.89 & 136.46 & RW2 \\ 
  Congo & BRAZZAVILLE & 2001 & 109.41 & 89.16 & 133.35 & RW2 \\ 
  Congo & BRAZZAVILLE & 2002 & 106.78 & 87.04 & 129.81 & RW2 \\ 
  Congo & BRAZZAVILLE & 2003 & 103.19 & 82.99 & 126.88 & RW2 \\ 
  Congo & BRAZZAVILLE & 2004 & 99.05 & 78.09 & 124.27 & RW2 \\ 
  Congo & BRAZZAVILLE & 2005 & 94.51 & 73.60 & 120.01 & RW2 \\ 
  Congo & BRAZZAVILLE & 2006 & 89.35 & 69.56 & 113.46 & RW2 \\ 
  Congo & BRAZZAVILLE & 2007 & 83.77 & 65.07 & 107.49 & RW2 \\ 
  Congo & BRAZZAVILLE & 2008 & 77.98 & 59.33 & 101.47 & RW2 \\ 
  Congo & BRAZZAVILLE & 2009 & 71.98 & 53.51 & 96.38 & RW2 \\ 
  Congo & BRAZZAVILLE & 2010 & 66.07 & 47.76 & 90.43 & RW2 \\ 
  Congo & BRAZZAVILLE & 2011 & 60.40 & 43.43 & 83.59 & RW2 \\ 
  Congo & BRAZZAVILLE & 2012 & 55.11 & 39.35 & 77.48 & RW2 \\ 
  Congo & BRAZZAVILLE & 2013 & 50.17 & 34.11 & 73.82 & RW2 \\ 
  Congo & BRAZZAVILLE & 2014 & 45.63 & 28.00 & 74.48 & RW2 \\ 
  Congo & BRAZZAVILLE & 2015 & 41.62 & 21.53 & 80.03 & RW2 \\ 
  Congo & BRAZZAVILLE & 2016 & 37.85 & 16.13 & 86.82 & RW2 \\ 
  Congo & BRAZZAVILLE & 2017 & 34.39 & 11.68 & 97.30 & RW2 \\ 
  Congo & BRAZZAVILLE & 2018 & 31.30 & 8.21 & 112.02 & RW2 \\ 
  Congo & BRAZZAVILLE & 2019 & 28.38 & 5.67 & 131.06 & RW2 \\ 
  Congo & NORD & 1980 & 132.78 & 84.06 & 204.92 & RW2 \\ 
  Congo & NORD & 1981 & 128.12 & 89.73 & 180.52 & RW2 \\ 
  Congo & NORD & 1982 & 123.71 & 91.07 & 165.78 & RW2 \\ 
  Congo & NORD & 1983 & 119.20 & 89.32 & 158.25 & RW2 \\ 
  Congo & NORD & 1984 & 115.09 & 86.75 & 151.69 & RW2 \\ 
  Congo & NORD & 1985 & 111.17 & 85.16 & 143.49 & RW2 \\ 
  Congo & NORD & 1986 & 107.35 & 84.29 & 136.38 & RW2 \\ 
  Congo & NORD & 1987 & 104.04 & 82.95 & 130.19 & RW2 \\ 
  Congo & NORD & 1988 & 101.21 & 80.76 & 126.84 & RW2 \\ 
  Congo & NORD & 1989 & 99.15 & 78.75 & 125.28 & RW2 \\ 
  Congo & NORD & 1990 & 97.71 & 77.52 & 121.76 & RW2 \\ 
  Congo & NORD & 1991 & 98.60 & 79.38 & 120.67 & RW2 \\ 
  Congo & NORD & 1992 & 101.14 & 82.22 & 122.88 & RW2 \\ 
  Congo & NORD & 1993 & 105.66 & 85.68 & 128.57 & RW2 \\ 
  Congo & NORD & 1994 & 111.84 & 89.51 & 136.72 & RW2 \\ 
  Congo & NORD & 1995 & 119.91 & 97.98 & 147.45 & RW2 \\ 
  Congo & NORD & 1996 & 126.75 & 105.36 & 152.12 & RW2 \\ 
  Congo & NORD & 1997 & 132.38 & 110.96 & 157.72 & RW2 \\ 
  Congo & NORD & 1998 & 136.02 & 113.42 & 162.76 & RW2 \\ 
  Congo & NORD & 1999 & 137.07 & 113.03 & 165.86 & RW2 \\ 
  Congo & NORD & 2000 & 135.15 & 111.07 & 162.73 & RW2 \\ 
  Congo & NORD & 2001 & 132.20 & 110.24 & 157.18 & RW2 \\ 
  Congo & NORD & 2002 & 127.79 & 107.69 & 151.22 & RW2 \\ 
  Congo & NORD & 2003 & 122.37 & 102.13 & 146.23 & RW2 \\ 
  Congo & NORD & 2004 & 116.21 & 95.04 & 141.06 & RW2 \\ 
  Congo & NORD & 2005 & 109.48 & 88.85 & 134.52 & RW2 \\ 
  Congo & NORD & 2006 & 101.91 & 83.62 & 124.48 & RW2 \\ 
  Congo & NORD & 2007 & 93.98 & 77.29 & 114.47 & RW2 \\ 
  Congo & NORD & 2008 & 85.74 & 69.41 & 105.80 & RW2 \\ 
  Congo & NORD & 2009 & 77.35 & 61.21 & 97.51 & RW2 \\ 
  Congo & NORD & 2010 & 69.20 & 53.52 & 88.84 & RW2 \\ 
  Congo & NORD & 2011 & 61.62 & 48.07 & 78.61 & RW2 \\ 
  Congo & NORD & 2012 & 54.66 & 42.74 & 69.47 & RW2 \\ 
  Congo & NORD & 2013 & 48.38 & 36.27 & 63.98 & RW2 \\ 
  Congo & NORD & 2014 & 42.72 & 28.52 & 62.79 & RW2 \\ 
  Congo & NORD & 2015 & 37.83 & 20.98 & 66.59 & RW2 \\ 
  Congo & NORD & 2016 & 33.36 & 15.00 & 71.17 & RW2 \\ 
  Congo & NORD & 2017 & 29.51 & 10.49 & 78.45 & RW2 \\ 
  Congo & NORD & 2018 & 25.94 & 7.16 & 87.65 & RW2 \\ 
  Congo & NORD & 2019 & 22.96 & 4.81 & 100.21 & RW2 \\ 
  Congo & POINTE NOIRE & 1980 & 119.63 & 71.60 & 192.52 & RW2 \\ 
  Congo & POINTE NOIRE & 1981 & 115.03 & 76.02 & 170.15 & RW2 \\ 
  Congo & POINTE NOIRE & 1982 & 110.52 & 76.78 & 156.90 & RW2 \\ 
  Congo & POINTE NOIRE & 1983 & 105.82 & 75.28 & 147.94 & RW2 \\ 
  Congo & POINTE NOIRE & 1984 & 101.75 & 73.12 & 140.47 & RW2 \\ 
  Congo & POINTE NOIRE & 1985 & 97.72 & 71.75 & 132.44 & RW2 \\ 
  Congo & POINTE NOIRE & 1986 & 93.89 & 70.45 & 124.68 & RW2 \\ 
  Congo & POINTE NOIRE & 1987 & 90.45 & 68.84 & 118.60 & RW2 \\ 
  Congo & POINTE NOIRE & 1988 & 87.35 & 66.92 & 113.85 & RW2 \\ 
  Congo & POINTE NOIRE & 1989 & 85.05 & 65.29 & 111.00 & RW2 \\ 
  Congo & POINTE NOIRE & 1990 & 83.06 & 63.92 & 106.78 & RW2 \\ 
  Congo & POINTE NOIRE & 1991 & 83.01 & 65.17 & 105.20 & RW2 \\ 
  Congo & POINTE NOIRE & 1992 & 84.54 & 66.98 & 105.95 & RW2 \\ 
  Congo & POINTE NOIRE & 1993 & 87.47 & 68.91 & 109.68 & RW2 \\ 
  Congo & POINTE NOIRE & 1994 & 91.70 & 71.69 & 115.08 & RW2 \\ 
  Congo & POINTE NOIRE & 1995 & 97.43 & 77.93 & 122.26 & RW2 \\ 
  Congo & POINTE NOIRE & 1996 & 102.07 & 82.65 & 125.66 & RW2 \\ 
  Congo & POINTE NOIRE & 1997 & 105.58 & 86.08 & 128.83 & RW2 \\ 
  Congo & POINTE NOIRE & 1998 & 107.40 & 86.89 & 131.88 & RW2 \\ 
  Congo & POINTE NOIRE & 1999 & 107.05 & 85.26 & 133.48 & RW2 \\ 
  Congo & POINTE NOIRE & 2000 & 104.56 & 82.78 & 129.54 & RW2 \\ 
  Congo & POINTE NOIRE & 2001 & 101.24 & 81.05 & 124.47 & RW2 \\ 
  Congo & POINTE NOIRE & 2002 & 97.09 & 78.15 & 119.25 & RW2 \\ 
  Congo & POINTE NOIRE & 2003 & 92.22 & 73.21 & 114.53 & RW2 \\ 
  Congo & POINTE NOIRE & 2004 & 87.04 & 67.84 & 110.50 & RW2 \\ 
  Congo & POINTE NOIRE & 2005 & 81.62 & 63.09 & 104.73 & RW2 \\ 
  Congo & POINTE NOIRE & 2006 & 75.91 & 58.68 & 97.23 & RW2 \\ 
  Congo & POINTE NOIRE & 2007 & 70.06 & 54.14 & 89.91 & RW2 \\ 
  Congo & POINTE NOIRE & 2008 & 64.11 & 48.88 & 83.95 & RW2 \\ 
  Congo & POINTE NOIRE & 2009 & 58.34 & 43.32 & 78.28 & RW2 \\ 
  Congo & POINTE NOIRE & 2010 & 52.67 & 37.98 & 72.65 & RW2 \\ 
  Congo & POINTE NOIRE & 2011 & 47.46 & 34.19 & 66.28 & RW2 \\ 
  Congo & POINTE NOIRE & 2012 & 42.62 & 30.42 & 60.60 & RW2 \\ 
  Congo & POINTE NOIRE & 2013 & 38.20 & 26.10 & 57.44 & RW2 \\ 
  Congo & POINTE NOIRE & 2014 & 34.24 & 21.12 & 57.08 & RW2 \\ 
  Congo & POINTE NOIRE & 2015 & 30.78 & 15.89 & 60.67 & RW2 \\ 
  Congo & POINTE NOIRE & 2016 & 27.58 & 11.69 & 65.46 & RW2 \\ 
  Congo & POINTE NOIRE & 2017 & 24.68 & 8.38 & 72.27 & RW2 \\ 
  Congo & POINTE NOIRE & 2018 & 22.14 & 5.85 & 82.31 & RW2 \\ 
  Congo & POINTE NOIRE & 2019 & 19.84 & 3.93 & 95.26 & RW2 \\ 
  Congo & SUD & 1980 & 150.51 & 98.47 & 222.33 & RW2 \\ 
  Congo & SUD & 1981 & 144.99 & 104.16 & 197.43 & RW2 \\ 
  Congo & SUD & 1982 & 139.30 & 104.65 & 182.04 & RW2 \\ 
  Congo & SUD & 1983 & 133.85 & 101.49 & 174.22 & RW2 \\ 
  Congo & SUD & 1984 & 128.60 & 97.25 & 168.06 & RW2 \\ 
  Congo & SUD & 1985 & 123.80 & 95.47 & 159.29 & RW2 \\ 
  Congo & SUD & 1986 & 119.23 & 93.62 & 150.51 & RW2 \\ 
  Congo & SUD & 1987 & 115.02 & 91.64 & 143.88 & RW2 \\ 
  Congo & SUD & 1988 & 111.45 & 88.92 & 139.68 & RW2 \\ 
  Congo & SUD & 1989 & 108.78 & 86.69 & 136.72 & RW2 \\ 
  Congo & SUD & 1990 & 106.66 & 85.18 & 132.08 & RW2 \\ 
  Congo & SUD & 1991 & 106.99 & 86.99 & 130.47 & RW2 \\ 
  Congo & SUD & 1992 & 109.09 & 89.31 & 131.90 & RW2 \\ 
  Congo & SUD & 1993 & 112.99 & 92.06 & 136.66 & RW2 \\ 
  Congo & SUD & 1994 & 118.57 & 95.32 & 144.00 & RW2 \\ 
  Congo & SUD & 1995 & 125.94 & 103.39 & 154.13 & RW2 \\ 
  Congo & SUD & 1996 & 131.87 & 109.77 & 158.52 & RW2 \\ 
  Congo & SUD & 1997 & 136.28 & 114.25 & 162.51 & RW2 \\ 
  Congo & SUD & 1998 & 138.45 & 115.66 & 165.80 & RW2 \\ 
  Congo & SUD & 1999 & 138.06 & 113.25 & 167.41 & RW2 \\ 
  Congo & SUD & 2000 & 134.50 & 109.68 & 162.52 & RW2 \\ 
  Congo & SUD & 2001 & 129.96 & 107.45 & 155.39 & RW2 \\ 
  Congo & SUD & 2002 & 124.06 & 103.56 & 147.87 & RW2 \\ 
  Congo & SUD & 2003 & 117.23 & 97.12 & 141.34 & RW2 \\ 
  Congo & SUD & 2004 & 109.88 & 89.24 & 134.64 & RW2 \\ 
  Congo & SUD & 2005 & 102.21 & 82.45 & 126.47 & RW2 \\ 
  Congo & SUD & 2006 & 94.10 & 76.52 & 115.59 & RW2 \\ 
  Congo & SUD & 2007 & 85.74 & 69.57 & 105.07 & RW2 \\ 
  Congo & SUD & 2008 & 77.46 & 61.97 & 96.36 & RW2 \\ 
  Congo & SUD & 2009 & 69.28 & 54.29 & 88.11 & RW2 \\ 
  Congo & SUD & 2010 & 61.53 & 47.40 & 79.50 & RW2 \\ 
  Congo & SUD & 2011 & 54.41 & 42.27 & 69.72 & RW2 \\ 
  Congo & SUD & 2012 & 47.99 & 37.60 & 60.97 & RW2 \\ 
  Congo & SUD & 2013 & 42.24 & 31.92 & 55.82 & RW2 \\ 
  Congo & SUD & 2014 & 37.16 & 25.15 & 54.30 & RW2 \\ 
  Congo & SUD & 2015 & 32.69 & 18.38 & 57.08 & RW2 \\ 
  Congo & SUD & 2016 & 28.66 & 13.15 & 60.60 & RW2 \\ 
  Congo & SUD & 2017 & 25.20 & 9.17 & 66.40 & RW2 \\ 
  Congo & SUD & 2018 & 22.05 & 6.19 & 74.46 & RW2 \\ 
  Congo & SUD & 2019 & 19.35 & 4.10 & 85.69 & RW2 \\ 
  C\^{o}te d'Ivoire & ALL & 1980 & 160.49 & 154.91 & 166.29 & IHME \\ 
  C\^{o}te d'Ivoire & ALL & 1980 & 159.38 & 90.08 & 265.92 & RW2 \\ 
  C\^{o}te d'Ivoire & ALL & 1980 & 166.60 & 153.90 & 180.30 & UN \\ 
  C\^{o}te d'Ivoire & ALL & 1981 & 157.08 & 151.78 & 162.96 & IHME \\ 
  C\^{o}te d'Ivoire & ALL & 1981 & 158.81 & 100.89 & 240.69 & RW2 \\ 
  C\^{o}te d'Ivoire & ALL & 1981 & 162.80 & 150.30 & 176.00 & UN \\ 
  C\^{o}te d'Ivoire & ALL & 1982 & 154.31 & 149.18 & 160.08 & IHME \\ 
  C\^{o}te d'Ivoire & ALL & 1982 & 158.46 & 109.07 & 224.40 & RW2 \\ 
  C\^{o}te d'Ivoire & ALL & 1982 & 159.70 & 147.50 & 172.40 & UN \\ 
  C\^{o}te d'Ivoire & ALL & 1983 & 152.90 & 147.98 & 158.54 & IHME \\ 
  C\^{o}te d'Ivoire & ALL & 1983 & 157.60 & 113.07 & 215.53 & RW2 \\ 
  C\^{o}te d'Ivoire & ALL & 1983 & 157.30 & 145.40 & 169.70 & UN \\ 
  C\^{o}te d'Ivoire & ALL & 1984 & 151.80 & 146.87 & 157.20 & IHME \\ 
  C\^{o}te d'Ivoire & ALL & 1984 & 157.30 & 114.48 & 211.58 & RW2 \\ 
  C\^{o}te d'Ivoire & ALL & 1984 & 155.50 & 143.80 & 167.70 & UN \\ 
  C\^{o}te d'Ivoire & ALL & 1985 & 150.82 & 145.79 & 156.12 & IHME \\ 
  C\^{o}te d'Ivoire & ALL & 1985 & 156.37 & 117.20 & 205.96 & RW2 \\ 
  C\^{o}te d'Ivoire & ALL & 1985 & 154.20 & 142.80 & 166.20 & UN \\ 
  C\^{o}te d'Ivoire & ALL & 1986 & 149.99 & 145.25 & 155.15 & IHME \\ 
  C\^{o}te d'Ivoire & ALL & 1986 & 155.70 & 118.53 & 202.00 & RW2 \\ 
  C\^{o}te d'Ivoire & ALL & 1986 & 153.30 & 142.00 & 165.10 & UN \\ 
  C\^{o}te d'Ivoire & ALL & 1987 & 149.48 & 144.72 & 154.46 & IHME \\ 
  C\^{o}te d'Ivoire & ALL & 1987 & 155.11 & 119.66 & 199.30 & RW2 \\ 
  C\^{o}te d'Ivoire & ALL & 1987 & 152.80 & 141.70 & 164.40 & UN \\ 
  C\^{o}te d'Ivoire & ALL & 1988 & 149.16 & 144.36 & 154.11 & IHME \\ 
  C\^{o}te d'Ivoire & ALL & 1988 & 154.33 & 119.16 & 197.70 & RW2 \\ 
  C\^{o}te d'Ivoire & ALL & 1988 & 152.50 & 141.70 & 163.90 & UN \\ 
  C\^{o}te d'Ivoire & ALL & 1989 & 148.92 & 144.14 & 153.73 & IHME \\ 
  C\^{o}te d'Ivoire & ALL & 1989 & 153.65 & 118.98 & 196.47 & RW2 \\ 
  C\^{o}te d'Ivoire & ALL & 1989 & 152.50 & 141.80 & 163.70 & UN \\ 
  C\^{o}te d'Ivoire & ALL & 1990 & 148.88 & 144.20 & 153.65 & IHME \\ 
  C\^{o}te d'Ivoire & ALL & 1990 & 152.97 & 119.76 & 193.23 & RW2 \\ 
  C\^{o}te d'Ivoire & ALL & 1990 & 152.60 & 142.10 & 163.60 & UN \\ 
  C\^{o}te d'Ivoire & ALL & 1991 & 148.87 & 144.14 & 154.02 & IHME \\ 
  C\^{o}te d'Ivoire & ALL & 1991 & 152.55 & 120.94 & 189.83 & RW2 \\ 
  C\^{o}te d'Ivoire & ALL & 1991 & 152.70 & 142.40 & 163.60 & UN \\ 
  C\^{o}te d'Ivoire & ALL & 1992 & 148.95 & 144.14 & 154.15 & IHME \\ 
  C\^{o}te d'Ivoire & ALL & 1992 & 152.30 & 121.43 & 188.56 & RW2 \\ 
  C\^{o}te d'Ivoire & ALL & 1992 & 152.70 & 142.40 & 163.80 & UN \\ 
  C\^{o}te d'Ivoire & ALL & 1993 & 148.98 & 143.78 & 154.11 & IHME \\ 
  C\^{o}te d'Ivoire & ALL & 1993 & 152.25 & 121.14 & 189.01 & RW2 \\ 
  C\^{o}te d'Ivoire & ALL & 1993 & 152.80 & 142.40 & 164.10 & UN \\ 
  C\^{o}te d'Ivoire & ALL & 1994 & 148.99 & 143.70 & 154.10 & IHME \\ 
  C\^{o}te d'Ivoire & ALL & 1994 & 152.14 & 120.23 & 190.61 & RW2 \\ 
  C\^{o}te d'Ivoire & ALL & 1994 & 152.90 & 142.20 & 164.30 & UN \\ 
  C\^{o}te d'Ivoire & ALL & 1995 & 148.69 & 143.30 & 153.93 & IHME \\ 
  C\^{o}te d'Ivoire & ALL & 1995 & 152.33 & 121.56 & 189.60 & RW2 \\ 
  C\^{o}te d'Ivoire & ALL & 1995 & 152.70 & 142.00 & 164.30 & UN \\ 
  C\^{o}te d'Ivoire & ALL & 1996 & 147.77 & 142.39 & 153.22 & IHME \\ 
  C\^{o}te d'Ivoire & ALL & 1996 & 151.89 & 122.47 & 187.78 & RW2 \\ 
  C\^{o}te d'Ivoire & ALL & 1996 & 152.20 & 141.20 & 164.20 & UN \\ 
  C\^{o}te d'Ivoire & ALL & 1997 & 146.46 & 141.00 & 151.90 & IHME \\ 
  C\^{o}te d'Ivoire & ALL & 1997 & 151.11 & 122.51 & 185.31 & RW2 \\ 
  C\^{o}te d'Ivoire & ALL & 1997 & 151.30 & 140.10 & 163.50 & UN \\ 
  C\^{o}te d'Ivoire & ALL & 1998 & 144.80 & 139.25 & 150.39 & IHME \\ 
  C\^{o}te d'Ivoire & ALL & 1998 & 149.95 & 120.91 & 185.41 & RW2 \\ 
  C\^{o}te d'Ivoire & ALL & 1998 & 149.80 & 138.40 & 162.40 & UN \\ 
  C\^{o}te d'Ivoire & ALL & 1999 & 142.81 & 137.13 & 148.73 & IHME \\ 
  C\^{o}te d'Ivoire & ALL & 1999 & 148.20 & 118.31 & 183.79 & RW2 \\ 
  C\^{o}te d'Ivoire & ALL & 1999 & 147.90 & 136.20 & 160.80 & UN \\ 
  C\^{o}te d'Ivoire & ALL & 2000 & 140.78 & 134.98 & 146.61 & IHME \\ 
  C\^{o}te d'Ivoire & ALL & 2000 & 145.86 & 116.83 & 180.39 & RW2 \\ 
  C\^{o}te d'Ivoire & ALL & 2000 & 145.60 & 133.80 & 158.60 & UN \\ 
  C\^{o}te d'Ivoire & ALL & 2001 & 138.48 & 132.81 & 144.28 & IHME \\ 
  C\^{o}te d'Ivoire & ALL & 2001 & 143.17 & 115.54 & 175.86 & RW2 \\ 
  C\^{o}te d'Ivoire & ALL & 2001 & 142.80 & 130.80 & 156.00 & UN \\ 
  C\^{o}te d'Ivoire & ALL & 2002 & 136.26 & 130.21 & 142.21 & IHME \\ 
  C\^{o}te d'Ivoire & ALL & 2002 & 140.02 & 113.74 & 171.30 & RW2 \\ 
  C\^{o}te d'Ivoire & ALL & 2002 & 139.70 & 127.80 & 152.70 & UN \\ 
  C\^{o}te d'Ivoire & ALL & 2003 & 133.74 & 127.53 & 139.81 & IHME \\ 
  C\^{o}te d'Ivoire & ALL & 2003 & 136.65 & 110.65 & 167.81 & RW2 \\ 
  C\^{o}te d'Ivoire & ALL & 2003 & 136.30 & 124.50 & 149.10 & UN \\ 
  C\^{o}te d'Ivoire & ALL & 2004 & 131.06 & 124.80 & 137.40 & IHME \\ 
  C\^{o}te d'Ivoire & ALL & 2004 & 132.80 & 106.02 & 165.02 & RW2 \\ 
  C\^{o}te d'Ivoire & ALL & 2004 & 132.70 & 121.10 & 145.50 & UN \\ 
  C\^{o}te d'Ivoire & ALL & 2005 & 128.05 & 121.53 & 134.56 & IHME \\ 
  C\^{o}te d'Ivoire & ALL & 2005 & 128.95 & 102.83 & 160.37 & RW2 \\ 
  C\^{o}te d'Ivoire & ALL & 2005 & 128.80 & 117.50 & 141.50 & UN \\ 
  C\^{o}te d'Ivoire & ALL & 2006 & 124.82 & 118.13 & 131.44 & IHME \\ 
  C\^{o}te d'Ivoire & ALL & 2006 & 124.96 & 100.78 & 153.68 & RW2 \\ 
  C\^{o}te d'Ivoire & ALL & 2006 & 125.10 & 113.70 & 137.80 & UN \\ 
  C\^{o}te d'Ivoire & ALL & 2007 & 121.45 & 114.75 & 128.31 & IHME \\ 
  C\^{o}te d'Ivoire & ALL & 2007 & 121.03 & 98.36 & 147.98 & RW2 \\ 
  C\^{o}te d'Ivoire & ALL & 2007 & 121.10 & 109.60 & 134.20 & UN \\ 
  C\^{o}te d'Ivoire & ALL & 2008 & 117.73 & 110.50 & 125.14 & IHME \\ 
  C\^{o}te d'Ivoire & ALL & 2008 & 117.29 & 94.48 & 144.85 & RW2 \\ 
  C\^{o}te d'Ivoire & ALL & 2008 & 116.50 & 104.70 & 129.80 & UN \\ 
  C\^{o}te d'Ivoire & ALL & 2009 & 114.16 & 106.60 & 122.19 & IHME \\ 
  C\^{o}te d'Ivoire & ALL & 2009 & 113.40 & 89.52 & 143.11 & RW2 \\ 
  C\^{o}te d'Ivoire & ALL & 2009 & 112.70 & 100.40 & 126.60 & UN \\ 
  C\^{o}te d'Ivoire & ALL & 2010 & 110.66 & 102.85 & 119.56 & IHME \\ 
  C\^{o}te d'Ivoire & ALL & 2010 & 109.71 & 84.76 & 141.69 & RW2 \\ 
  C\^{o}te d'Ivoire & ALL & 2010 & 109.10 & 95.80 & 124.10 & UN \\ 
  C\^{o}te d'Ivoire & ALL & 2011 & 107.72 & 99.30 & 117.31 & IHME \\ 
  C\^{o}te d'Ivoire & ALL & 2011 & 106.19 & 81.76 & 136.89 & RW2 \\ 
  C\^{o}te d'Ivoire & ALL & 2011 & 105.90 & 91.70 & 122.60 & UN \\ 
  C\^{o}te d'Ivoire & ALL & 2012 & 104.40 & 95.74 & 114.71 & IHME \\ 
  C\^{o}te d'Ivoire & ALL & 2012 & 102.67 & 78.35 & 133.41 & RW2 \\ 
  C\^{o}te d'Ivoire & ALL & 2012 & 102.50 & 87.00 & 120.90 & UN \\ 
  C\^{o}te d'Ivoire & ALL & 2013 & 100.83 & 91.72 & 111.51 & IHME \\ 
  C\^{o}te d'Ivoire & ALL & 2013 & 99.37 & 72.41 & 134.66 & RW2 \\ 
  C\^{o}te d'Ivoire & ALL & 2013 & 99.00 & 82.00 & 119.50 & UN \\ 
  C\^{o}te d'Ivoire & ALL & 2014 & 97.26 & 87.95 & 108.81 & IHME \\ 
  C\^{o}te d'Ivoire & ALL & 2014 & 96.04 & 63.08 & 143.39 & RW2 \\ 
  C\^{o}te d'Ivoire & ALL & 2014 & 95.50 & 76.80 & 119.20 & UN \\ 
  C\^{o}te d'Ivoire & ALL & 2015 & 93.49 & 83.69 & 105.67 & IHME \\ 
  C\^{o}te d'Ivoire & ALL & 2015 & 92.70 & 51.15 & 162.46 & RW2 \\ 
  C\^{o}te d'Ivoire & ALL & 2015 & 92.60 & 72.30 & 118.80 & UN \\ 
  C\^{o}te d'Ivoire & ALL & 2016 & 89.83 & 41.18 & 186.09 & RW2 \\ 
  C\^{o}te d'Ivoire & ALL & 2017 & 86.59 & 31.87 & 216.39 & RW2 \\ 
  C\^{o}te d'Ivoire & ALL & 2018 & 83.63 & 24.37 & 256.13 & RW2 \\ 
  C\^{o}te d'Ivoire & ALL & 2019 & 80.65 & 17.50 & 299.39 & RW2 \\ 
  C\^{o}te d'Ivoire & ALL & 80-84 & 131.42 & 171.48 & 99.59 & HT-Direct \\ 
  C\^{o}te d'Ivoire & ALL & 85-89 & 142.74 & 171.14 & 118.38 & HT-Direct \\ 
  C\^{o}te d'Ivoire & ALL & 90-94 & 138.27 & 155.75 & 122.46 & HT-Direct \\ 
  C\^{o}te d'Ivoire & ALL & 95-99 & 142.64 & 157.39 & 129.07 & HT-Direct \\ 
  C\^{o}te d'Ivoire & ALL & 00-04 & 137.05 & 150.12 & 124.95 & HT-Direct \\ 
  C\^{o}te d'Ivoire & ALL & 05-09 & 118.20 & 128.24 & 108.85 & HT-Direct \\ 
  C\^{o}te d'Ivoire & ALL & 10-14 & 97.02 & 115.98 & 80.88 & HT-Direct \\ 
  C\^{o}te d'Ivoire & ALL & 15-19 & 86.59 & 32.38 & 212.92 & RW2 \\ 
  C\^{o}te d'Ivoire & CENTRE & 1980 & 184.17 & 99.04 & 319.01 & RW2 \\ 
  C\^{o}te d'Ivoire & CENTRE & 1981 & 176.16 & 103.25 & 284.73 & RW2 \\ 
  C\^{o}te d'Ivoire & CENTRE & 1982 & 168.08 & 103.61 & 261.23 & RW2 \\ 
  C\^{o}te d'Ivoire & CENTRE & 1983 & 160.33 & 101.70 & 244.47 & RW2 \\ 
  C\^{o}te d'Ivoire & CENTRE & 1984 & 153.43 & 99.49 & 229.66 & RW2 \\ 
  C\^{o}te d'Ivoire & CENTRE & 1985 & 146.86 & 96.62 & 215.80 & RW2 \\ 
  C\^{o}te d'Ivoire & CENTRE & 1986 & 142.30 & 95.55 & 205.49 & RW2 \\ 
  C\^{o}te d'Ivoire & CENTRE & 1987 & 139.05 & 95.51 & 197.29 & RW2 \\ 
  C\^{o}te d'Ivoire & CENTRE & 1988 & 137.11 & 95.34 & 192.28 & RW2 \\ 
  C\^{o}te d'Ivoire & CENTRE & 1989 & 136.42 & 96.23 & 188.97 & RW2 \\ 
  C\^{o}te d'Ivoire & CENTRE & 1990 & 137.12 & 99.01 & 187.35 & RW2 \\ 
  C\^{o}te d'Ivoire & CENTRE & 1991 & 138.25 & 101.87 & 185.50 & RW2 \\ 
  C\^{o}te d'Ivoire & CENTRE & 1992 & 139.76 & 104.43 & 184.58 & RW2 \\ 
  C\^{o}te d'Ivoire & CENTRE & 1993 & 141.37 & 107.06 & 184.49 & RW2 \\ 
  C\^{o}te d'Ivoire & CENTRE & 1994 & 143.09 & 109.00 & 185.85 & RW2 \\ 
  C\^{o}te d'Ivoire & CENTRE & 1995 & 144.76 & 111.91 & 184.97 & RW2 \\ 
  C\^{o}te d'Ivoire & CENTRE & 1996 & 146.76 & 115.32 & 184.49 & RW2 \\ 
  C\^{o}te d'Ivoire & CENTRE & 1997 & 148.78 & 118.46 & 184.65 & RW2 \\ 
  C\^{o}te d'Ivoire & CENTRE & 1998 & 150.74 & 120.47 & 187.10 & RW2 \\ 
  C\^{o}te d'Ivoire & CENTRE & 1999 & 152.51 & 121.21 & 189.92 & RW2 \\ 
  C\^{o}te d'Ivoire & CENTRE & 2000 & 154.13 & 122.93 & 190.95 & RW2 \\ 
  C\^{o}te d'Ivoire & CENTRE & 2001 & 154.69 & 125.08 & 189.88 & RW2 \\ 
  C\^{o}te d'Ivoire & CENTRE & 2002 & 154.36 & 125.20 & 189.02 & RW2 \\ 
  C\^{o}te d'Ivoire & CENTRE & 2003 & 152.84 & 123.17 & 188.74 & RW2 \\ 
  C\^{o}te d'Ivoire & CENTRE & 2004 & 150.64 & 119.75 & 188.27 & RW2 \\ 
  C\^{o}te d'Ivoire & CENTRE & 2005 & 147.43 & 115.59 & 185.78 & RW2 \\ 
  C\^{o}te d'Ivoire & CENTRE & 2006 & 144.42 & 113.16 & 182.39 & RW2 \\ 
  C\^{o}te d'Ivoire & CENTRE & 2007 & 141.44 & 110.00 & 179.92 & RW2 \\ 
  C\^{o}te d'Ivoire & CENTRE & 2008 & 138.87 & 105.83 & 180.12 & RW2 \\ 
  C\^{o}te d'Ivoire & CENTRE & 2009 & 136.40 & 101.16 & 181.70 & RW2 \\ 
  C\^{o}te d'Ivoire & CENTRE & 2010 & 134.85 & 96.34 & 185.39 & RW2 \\ 
  C\^{o}te d'Ivoire & CENTRE & 2011 & 132.81 & 93.13 & 186.36 & RW2 \\ 
  C\^{o}te d'Ivoire & CENTRE & 2012 & 131.32 & 89.30 & 188.48 & RW2 \\ 
  C\^{o}te d'Ivoire & CENTRE & 2013 & 129.60 & 83.58 & 194.84 & RW2 \\ 
  C\^{o}te d'Ivoire & CENTRE & 2014 & 128.05 & 75.64 & 206.51 & RW2 \\ 
  C\^{o}te d'Ivoire & CENTRE & 2015 & 126.47 & 65.21 & 231.41 & RW2 \\ 
  C\^{o}te d'Ivoire & CENTRE & 2016 & 124.80 & 55.11 & 260.43 & RW2 \\ 
  C\^{o}te d'Ivoire & CENTRE & 2017 & 123.32 & 44.94 & 299.63 & RW2 \\ 
  C\^{o}te d'Ivoire & CENTRE & 2018 & 121.33 & 35.25 & 345.25 & RW2 \\ 
  C\^{o}te d'Ivoire & CENTRE & 2019 & 119.61 & 27.79 & 398.41 & RW2 \\ 
  C\^{o}te d'Ivoire & CENTRE-EST & 1980 & 168.30 & 86.26 & 306.63 & RW2 \\ 
  C\^{o}te d'Ivoire & CENTRE-EST & 1981 & 159.84 & 88.96 & 273.52 & RW2 \\ 
  C\^{o}te d'Ivoire & CENTRE-EST & 1982 & 151.81 & 89.80 & 249.72 & RW2 \\ 
  C\^{o}te d'Ivoire & CENTRE-EST & 1983 & 144.58 & 88.49 & 231.07 & RW2 \\ 
  C\^{o}te d'Ivoire & CENTRE-EST & 1984 & 138.11 & 85.89 & 215.44 & RW2 \\ 
  C\^{o}te d'Ivoire & CENTRE-EST & 1985 & 131.76 & 84.28 & 201.27 & RW2 \\ 
  C\^{o}te d'Ivoire & CENTRE-EST & 1986 & 127.25 & 83.33 & 189.70 & RW2 \\ 
  C\^{o}te d'Ivoire & CENTRE-EST & 1987 & 124.06 & 83.41 & 180.66 & RW2 \\ 
  C\^{o}te d'Ivoire & CENTRE-EST & 1988 & 122.10 & 83.58 & 174.15 & RW2 \\ 
  C\^{o}te d'Ivoire & CENTRE-EST & 1989 & 120.99 & 84.98 & 169.88 & RW2 \\ 
  C\^{o}te d'Ivoire & CENTRE-EST & 1990 & 121.66 & 87.83 & 165.98 & RW2 \\ 
  C\^{o}te d'Ivoire & CENTRE-EST & 1991 & 122.38 & 90.39 & 163.75 & RW2 \\ 
  C\^{o}te d'Ivoire & CENTRE-EST & 1992 & 123.62 & 93.15 & 162.09 & RW2 \\ 
  C\^{o}te d'Ivoire & CENTRE-EST & 1993 & 124.93 & 94.76 & 163.05 & RW2 \\ 
  C\^{o}te d'Ivoire & CENTRE-EST & 1994 & 126.45 & 96.29 & 164.23 & RW2 \\ 
  C\^{o}te d'Ivoire & CENTRE-EST & 1995 & 127.96 & 98.13 & 164.54 & RW2 \\ 
  C\^{o}te d'Ivoire & CENTRE-EST & 1996 & 129.92 & 101.33 & 165.10 & RW2 \\ 
  C\^{o}te d'Ivoire & CENTRE-EST & 1997 & 132.05 & 103.32 & 165.96 & RW2 \\ 
  C\^{o}te d'Ivoire & CENTRE-EST & 1998 & 133.98 & 104.98 & 168.64 & RW2 \\ 
  C\^{o}te d'Ivoire & CENTRE-EST & 1999 & 136.02 & 105.83 & 171.40 & RW2 \\ 
  C\^{o}te d'Ivoire & CENTRE-EST & 2000 & 138.03 & 108.13 & 174.11 & RW2 \\ 
  C\^{o}te d'Ivoire & CENTRE-EST & 2001 & 139.03 & 110.09 & 174.02 & RW2 \\ 
  C\^{o}te d'Ivoire & CENTRE-EST & 2002 & 139.41 & 110.75 & 174.28 & RW2 \\ 
  C\^{o}te d'Ivoire & CENTRE-EST & 2003 & 138.92 & 109.30 & 174.62 & RW2 \\ 
  C\^{o}te d'Ivoire & CENTRE-EST & 2004 & 137.54 & 106.46 & 176.01 & RW2 \\ 
  C\^{o}te d'Ivoire & CENTRE-EST & 2005 & 135.40 & 103.61 & 174.89 & RW2 \\ 
  C\^{o}te d'Ivoire & CENTRE-EST & 2006 & 133.32 & 101.41 & 173.00 & RW2 \\ 
  C\^{o}te d'Ivoire & CENTRE-EST & 2007 & 131.42 & 99.00 & 172.95 & RW2 \\ 
  C\^{o}te d'Ivoire & CENTRE-EST & 2008 & 129.53 & 95.41 & 174.59 & RW2 \\ 
  C\^{o}te d'Ivoire & CENTRE-EST & 2009 & 128.21 & 90.94 & 178.03 & RW2 \\ 
  C\^{o}te d'Ivoire & CENTRE-EST & 2010 & 127.20 & 87.57 & 182.54 & RW2 \\ 
  C\^{o}te d'Ivoire & CENTRE-EST & 2011 & 126.29 & 84.69 & 184.82 & RW2 \\ 
  C\^{o}te d'Ivoire & CENTRE-EST & 2012 & 125.40 & 81.92 & 189.56 & RW2 \\ 
  C\^{o}te d'Ivoire & CENTRE-EST & 2013 & 124.50 & 76.87 & 196.05 & RW2 \\ 
  C\^{o}te d'Ivoire & CENTRE-EST & 2014 & 123.65 & 70.37 & 210.58 & RW2 \\ 
  C\^{o}te d'Ivoire & CENTRE-EST & 2015 & 122.97 & 60.38 & 235.95 & RW2 \\ 
  C\^{o}te d'Ivoire & CENTRE-EST & 2016 & 121.77 & 51.73 & 264.70 & RW2 \\ 
  C\^{o}te d'Ivoire & CENTRE-EST & 2017 & 120.82 & 41.98 & 303.85 & RW2 \\ 
  C\^{o}te d'Ivoire & CENTRE-EST & 2018 & 120.14 & 33.83 & 349.27 & RW2 \\ 
  C\^{o}te d'Ivoire & CENTRE-EST & 2019 & 119.37 & 26.35 & 402.10 & RW2 \\ 
  C\^{o}te d'Ivoire & CENTRE-NORD & 1980 & 184.79 & 106.76 & 303.16 & RW2 \\ 
  C\^{o}te d'Ivoire & CENTRE-NORD & 1981 & 173.08 & 108.67 & 265.46 & RW2 \\ 
  C\^{o}te d'Ivoire & CENTRE-NORD & 1982 & 161.81 & 105.81 & 238.87 & RW2 \\ 
  C\^{o}te d'Ivoire & CENTRE-NORD & 1983 & 151.61 & 101.82 & 220.20 & RW2 \\ 
  C\^{o}te d'Ivoire & CENTRE-NORD & 1984 & 142.26 & 96.38 & 205.36 & RW2 \\ 
  C\^{o}te d'Ivoire & CENTRE-NORD & 1985 & 133.60 & 91.49 & 190.23 & RW2 \\ 
  C\^{o}te d'Ivoire & CENTRE-NORD & 1986 & 126.74 & 87.76 & 178.14 & RW2 \\ 
  C\^{o}te d'Ivoire & CENTRE-NORD & 1987 & 121.27 & 85.03 & 169.46 & RW2 \\ 
  C\^{o}te d'Ivoire & CENTRE-NORD & 1988 & 117.08 & 82.79 & 162.41 & RW2 \\ 
  C\^{o}te d'Ivoire & CENTRE-NORD & 1989 & 114.14 & 81.03 & 157.50 & RW2 \\ 
  C\^{o}te d'Ivoire & CENTRE-NORD & 1990 & 112.54 & 81.02 & 154.30 & RW2 \\ 
  C\^{o}te d'Ivoire & CENTRE-NORD & 1991 & 111.16 & 81.12 & 151.20 & RW2 \\ 
  C\^{o}te d'Ivoire & CENTRE-NORD & 1992 & 110.04 & 80.95 & 147.60 & RW2 \\ 
  C\^{o}te d'Ivoire & CENTRE-NORD & 1993 & 109.05 & 80.51 & 145.94 & RW2 \\ 
  C\^{o}te d'Ivoire & CENTRE-NORD & 1994 & 108.16 & 79.76 & 145.02 & RW2 \\ 
  C\^{o}te d'Ivoire & CENTRE-NORD & 1995 & 107.27 & 79.85 & 142.38 & RW2 \\ 
  C\^{o}te d'Ivoire & CENTRE-NORD & 1996 & 106.43 & 80.09 & 139.58 & RW2 \\ 
  C\^{o}te d'Ivoire & CENTRE-NORD & 1997 & 105.75 & 80.16 & 137.44 & RW2 \\ 
  C\^{o}te d'Ivoire & CENTRE-NORD & 1998 & 104.75 & 79.40 & 136.47 & RW2 \\ 
  C\^{o}te d'Ivoire & CENTRE-NORD & 1999 & 103.69 & 77.82 & 136.50 & RW2 \\ 
  C\^{o}te d'Ivoire & CENTRE-NORD & 2000 & 102.51 & 77.20 & 134.58 & RW2 \\ 
  C\^{o}te d'Ivoire & CENTRE-NORD & 2001 & 100.49 & 76.27 & 130.79 & RW2 \\ 
  C\^{o}te d'Ivoire & CENTRE-NORD & 2002 & 97.92 & 74.40 & 127.66 & RW2 \\ 
  C\^{o}te d'Ivoire & CENTRE-NORD & 2003 & 94.69 & 71.51 & 124.36 & RW2 \\ 
  C\^{o}te d'Ivoire & CENTRE-NORD & 2004 & 91.18 & 67.89 & 121.50 & RW2 \\ 
  C\^{o}te d'Ivoire & CENTRE-NORD & 2005 & 87.00 & 63.90 & 116.67 & RW2 \\ 
  C\^{o}te d'Ivoire & CENTRE-NORD & 2006 & 83.31 & 61.09 & 112.22 & RW2 \\ 
  C\^{o}te d'Ivoire & CENTRE-NORD & 2007 & 79.58 & 57.82 & 108.48 & RW2 \\ 
  C\^{o}te d'Ivoire & CENTRE-NORD & 2008 & 76.26 & 54.48 & 105.74 & RW2 \\ 
  C\^{o}te d'Ivoire & CENTRE-NORD & 2009 & 73.53 & 51.10 & 104.68 & RW2 \\ 
  C\^{o}te d'Ivoire & CENTRE-NORD & 2010 & 70.98 & 47.78 & 103.85 & RW2 \\ 
  C\^{o}te d'Ivoire & CENTRE-NORD & 2011 & 68.56 & 45.60 & 102.31 & RW2 \\ 
  C\^{o}te d'Ivoire & CENTRE-NORD & 2012 & 66.29 & 42.90 & 101.58 & RW2 \\ 
  C\^{o}te d'Ivoire & CENTRE-NORD & 2013 & 64.15 & 39.51 & 102.27 & RW2 \\ 
  C\^{o}te d'Ivoire & CENTRE-NORD & 2014 & 62.19 & 35.16 & 107.15 & RW2 \\ 
  C\^{o}te d'Ivoire & CENTRE-NORD & 2015 & 60.06 & 29.89 & 118.59 & RW2 \\ 
  C\^{o}te d'Ivoire & CENTRE-NORD & 2016 & 58.02 & 24.41 & 132.39 & RW2 \\ 
  C\^{o}te d'Ivoire & CENTRE-NORD & 2017 & 56.25 & 19.54 & 151.96 & RW2 \\ 
  C\^{o}te d'Ivoire & CENTRE-NORD & 2018 & 53.99 & 15.25 & 178.52 & RW2 \\ 
  C\^{o}te d'Ivoire & CENTRE-NORD & 2019 & 52.20 & 11.54 & 210.37 & RW2 \\ 
  C\^{o}te d'Ivoire & CENTRE-OUEST & 1980 & 196.39 & 95.02 & 361.15 & RW2 \\ 
  C\^{o}te d'Ivoire & CENTRE-OUEST & 1981 & 185.29 & 97.18 & 321.64 & RW2 \\ 
  C\^{o}te d'Ivoire & CENTRE-OUEST & 1982 & 174.53 & 97.13 & 291.20 & RW2 \\ 
  C\^{o}te d'Ivoire & CENTRE-OUEST & 1983 & 164.44 & 95.23 & 268.85 & RW2 \\ 
  C\^{o}te d'Ivoire & CENTRE-OUEST & 1984 & 155.11 & 92.24 & 248.72 & RW2 \\ 
  C\^{o}te d'Ivoire & CENTRE-OUEST & 1985 & 146.51 & 89.61 & 229.31 & RW2 \\ 
  C\^{o}te d'Ivoire & CENTRE-OUEST & 1986 & 139.91 & 87.81 & 213.49 & RW2 \\ 
  C\^{o}te d'Ivoire & CENTRE-OUEST & 1987 & 134.29 & 86.97 & 200.57 & RW2 \\ 
  C\^{o}te d'Ivoire & CENTRE-OUEST & 1988 & 130.69 & 86.54 & 191.10 & RW2 \\ 
  C\^{o}te d'Ivoire & CENTRE-OUEST & 1989 & 128.00 & 86.82 & 183.86 & RW2 \\ 
  C\^{o}te d'Ivoire & CENTRE-OUEST & 1990 & 126.73 & 88.78 & 177.56 & RW2 \\ 
  C\^{o}te d'Ivoire & CENTRE-OUEST & 1991 & 125.50 & 90.29 & 172.40 & RW2 \\ 
  C\^{o}te d'Ivoire & CENTRE-OUEST & 1992 & 124.68 & 91.42 & 167.87 & RW2 \\ 
  C\^{o}te d'Ivoire & CENTRE-OUEST & 1993 & 124.02 & 92.20 & 164.93 & RW2 \\ 
  C\^{o}te d'Ivoire & CENTRE-OUEST & 1994 & 123.23 & 92.13 & 163.02 & RW2 \\ 
  C\^{o}te d'Ivoire & CENTRE-OUEST & 1995 & 122.34 & 92.44 & 159.87 & RW2 \\ 
  C\^{o}te d'Ivoire & CENTRE-OUEST & 1996 & 121.58 & 93.38 & 156.30 & RW2 \\ 
  C\^{o}te d'Ivoire & CENTRE-OUEST & 1997 & 120.76 & 93.72 & 154.11 & RW2 \\ 
  C\^{o}te d'Ivoire & CENTRE-OUEST & 1998 & 119.83 & 93.53 & 152.43 & RW2 \\ 
  C\^{o}te d'Ivoire & CENTRE-OUEST & 1999 & 118.68 & 91.96 & 151.52 & RW2 \\ 
  C\^{o}te d'Ivoire & CENTRE-OUEST & 2000 & 117.22 & 91.28 & 149.09 & RW2 \\ 
  C\^{o}te d'Ivoire & CENTRE-OUEST & 2001 & 114.98 & 90.53 & 145.11 & RW2 \\ 
  C\^{o}te d'Ivoire & CENTRE-OUEST & 2002 & 111.84 & 88.46 & 140.55 & RW2 \\ 
  C\^{o}te d'Ivoire & CENTRE-OUEST & 2003 & 108.33 & 84.97 & 137.67 & RW2 \\ 
  C\^{o}te d'Ivoire & CENTRE-OUEST & 2004 & 104.00 & 80.42 & 133.99 & RW2 \\ 
  C\^{o}te d'Ivoire & CENTRE-OUEST & 2005 & 99.19 & 76.06 & 128.03 & RW2 \\ 
  C\^{o}te d'Ivoire & CENTRE-OUEST & 2006 & 94.76 & 72.62 & 122.52 & RW2 \\ 
  C\^{o}te d'Ivoire & CENTRE-OUEST & 2007 & 90.69 & 68.86 & 118.21 & RW2 \\ 
  C\^{o}te d'Ivoire & CENTRE-OUEST & 2008 & 86.85 & 64.65 & 115.61 & RW2 \\ 
  C\^{o}te d'Ivoire & CENTRE-OUEST & 2009 & 83.43 & 60.27 & 114.32 & RW2 \\ 
  C\^{o}te d'Ivoire & CENTRE-OUEST & 2010 & 80.40 & 56.35 & 113.96 & RW2 \\ 
  C\^{o}te d'Ivoire & CENTRE-OUEST & 2011 & 77.70 & 52.91 & 111.69 & RW2 \\ 
  C\^{o}te d'Ivoire & CENTRE-OUEST & 2012 & 74.95 & 49.90 & 111.38 & RW2 \\ 
  C\^{o}te d'Ivoire & CENTRE-OUEST & 2013 & 72.45 & 45.94 & 112.88 & RW2 \\ 
  C\^{o}te d'Ivoire & CENTRE-OUEST & 2014 & 69.88 & 40.45 & 118.29 & RW2 \\ 
  C\^{o}te d'Ivoire & CENTRE-OUEST & 2015 & 67.64 & 33.96 & 131.24 & RW2 \\ 
  C\^{o}te d'Ivoire & CENTRE-OUEST & 2016 & 65.48 & 27.88 & 147.20 & RW2 \\ 
  C\^{o}te d'Ivoire & CENTRE-OUEST & 2017 & 63.03 & 22.14 & 167.38 & RW2 \\ 
  C\^{o}te d'Ivoire & CENTRE-OUEST & 2018 & 60.63 & 17.30 & 194.44 & RW2 \\ 
  C\^{o}te d'Ivoire & CENTRE-OUEST & 2019 & 58.67 & 13.11 & 232.18 & RW2 \\ 
  C\^{o}te d'Ivoire & NORD & 1980 & 266.98 & 146.69 & 437.69 & RW2 \\ 
  C\^{o}te d'Ivoire & NORD & 1981 & 256.99 & 153.07 & 399.21 & RW2 \\ 
  C\^{o}te d'Ivoire & NORD & 1982 & 247.18 & 154.44 & 371.62 & RW2 \\ 
  C\^{o}te d'Ivoire & NORD & 1983 & 237.48 & 153.73 & 349.30 & RW2 \\ 
  C\^{o}te d'Ivoire & NORD & 1984 & 229.00 & 151.41 & 332.25 & RW2 \\ 
  C\^{o}te d'Ivoire & NORD & 1985 & 221.29 & 148.90 & 313.33 & RW2 \\ 
  C\^{o}te d'Ivoire & NORD & 1986 & 215.32 & 148.84 & 299.83 & RW2 \\ 
  C\^{o}te d'Ivoire & NORD & 1987 & 211.40 & 149.00 & 290.14 & RW2 \\ 
  C\^{o}te d'Ivoire & NORD & 1988 & 209.65 & 150.05 & 284.00 & RW2 \\ 
  C\^{o}te d'Ivoire & NORD & 1989 & 209.51 & 151.87 & 280.13 & RW2 \\ 
  C\^{o}te d'Ivoire & NORD & 1990 & 211.30 & 157.30 & 279.20 & RW2 \\ 
  C\^{o}te d'Ivoire & NORD & 1991 & 213.75 & 161.51 & 277.57 & RW2 \\ 
  C\^{o}te d'Ivoire & NORD & 1992 & 216.48 & 166.45 & 276.77 & RW2 \\ 
  C\^{o}te d'Ivoire & NORD & 1993 & 219.73 & 170.94 & 278.27 & RW2 \\ 
  C\^{o}te d'Ivoire & NORD & 1994 & 222.91 & 174.51 & 280.41 & RW2 \\ 
  C\^{o}te d'Ivoire & NORD & 1995 & 225.97 & 180.02 & 280.05 & RW2 \\ 
  C\^{o}te d'Ivoire & NORD & 1996 & 229.34 & 185.70 & 279.84 & RW2 \\ 
  C\^{o}te d'Ivoire & NORD & 1997 & 232.84 & 190.83 & 280.84 & RW2 \\ 
  C\^{o}te d'Ivoire & NORD & 1998 & 236.13 & 193.38 & 284.01 & RW2 \\ 
  C\^{o}te d'Ivoire & NORD & 1999 & 238.79 & 195.16 & 287.46 & RW2 \\ 
  C\^{o}te d'Ivoire & NORD & 2000 & 241.38 & 199.15 & 289.47 & RW2 \\ 
  C\^{o}te d'Ivoire & NORD & 2001 & 242.17 & 201.70 & 287.94 & RW2 \\ 
  C\^{o}te d'Ivoire & NORD & 2002 & 241.79 & 202.98 & 285.89 & RW2 \\ 
  C\^{o}te d'Ivoire & NORD & 2003 & 239.63 & 199.92 & 284.54 & RW2 \\ 
  C\^{o}te d'Ivoire & NORD & 2004 & 236.43 & 195.72 & 283.47 & RW2 \\ 
  C\^{o}te d'Ivoire & NORD & 2005 & 231.65 & 190.47 & 277.87 & RW2 \\ 
  C\^{o}te d'Ivoire & NORD & 2006 & 227.49 & 188.58 & 271.44 & RW2 \\ 
  C\^{o}te d'Ivoire & NORD & 2007 & 223.48 & 185.31 & 266.78 & RW2 \\ 
  C\^{o}te d'Ivoire & NORD & 2008 & 220.03 & 180.25 & 265.40 & RW2 \\ 
  C\^{o}te d'Ivoire & NORD & 2009 & 217.17 & 174.16 & 266.89 & RW2 \\ 
  C\^{o}te d'Ivoire & NORD & 2010 & 214.98 & 169.11 & 269.83 & RW2 \\ 
  C\^{o}te d'Ivoire & NORD & 2011 & 213.17 & 166.23 & 269.26 & RW2 \\ 
  C\^{o}te d'Ivoire & NORD & 2012 & 211.30 & 162.99 & 269.92 & RW2 \\ 
  C\^{o}te d'Ivoire & NORD & 2013 & 209.64 & 155.27 & 277.86 & RW2 \\ 
  C\^{o}te d'Ivoire & NORD & 2014 & 207.87 & 141.92 & 294.70 & RW2 \\ 
  C\^{o}te d'Ivoire & NORD & 2015 & 206.30 & 122.17 & 327.90 & RW2 \\ 
  C\^{o}te d'Ivoire & NORD & 2016 & 204.29 & 102.87 & 366.59 & RW2 \\ 
  C\^{o}te d'Ivoire & NORD & 2017 & 202.74 & 84.50 & 418.63 & RW2 \\ 
  C\^{o}te d'Ivoire & NORD & 2018 & 201.13 & 67.69 & 470.92 & RW2 \\ 
  C\^{o}te d'Ivoire & NORD & 2019 & 199.77 & 53.80 & 528.11 & RW2 \\ 
  C\^{o}te d'Ivoire & NORD-EST & 1980 & 243.90 & 127.08 & 422.77 & RW2 \\ 
  C\^{o}te d'Ivoire & NORD-EST & 1981 & 229.25 & 128.80 & 381.51 & RW2 \\ 
  C\^{o}te d'Ivoire & NORD-EST & 1982 & 216.30 & 126.79 & 347.37 & RW2 \\ 
  C\^{o}te d'Ivoire & NORD-EST & 1983 & 203.51 & 123.83 & 321.13 & RW2 \\ 
  C\^{o}te d'Ivoire & NORD-EST & 1984 & 191.73 & 118.85 & 296.91 & RW2 \\ 
  C\^{o}te d'Ivoire & NORD-EST & 1985 & 180.93 & 114.62 & 274.34 & RW2 \\ 
  C\^{o}te d'Ivoire & NORD-EST & 1986 & 172.33 & 112.12 & 256.60 & RW2 \\ 
  C\^{o}te d'Ivoire & NORD-EST & 1987 & 165.74 & 109.93 & 241.57 & RW2 \\ 
  C\^{o}te d'Ivoire & NORD-EST & 1988 & 160.95 & 109.16 & 230.80 & RW2 \\ 
  C\^{o}te d'Ivoire & NORD-EST & 1989 & 157.47 & 108.37 & 222.71 & RW2 \\ 
  C\^{o}te d'Ivoire & NORD-EST & 1990 & 155.55 & 109.99 & 215.72 & RW2 \\ 
  C\^{o}te d'Ivoire & NORD-EST & 1991 & 154.26 & 111.90 & 209.48 & RW2 \\ 
  C\^{o}te d'Ivoire & NORD-EST & 1992 & 153.29 & 112.74 & 204.71 & RW2 \\ 
  C\^{o}te d'Ivoire & NORD-EST & 1993 & 152.72 & 114.10 & 201.72 & RW2 \\ 
  C\^{o}te d'Ivoire & NORD-EST & 1994 & 151.99 & 114.02 & 199.52 & RW2 \\ 
  C\^{o}te d'Ivoire & NORD-EST & 1995 & 151.64 & 115.25 & 196.32 & RW2 \\ 
  C\^{o}te d'Ivoire & NORD-EST & 1996 & 151.27 & 117.05 & 192.87 & RW2 \\ 
  C\^{o}te d'Ivoire & NORD-EST & 1997 & 151.17 & 117.86 & 190.83 & RW2 \\ 
  C\^{o}te d'Ivoire & NORD-EST & 1998 & 151.02 & 117.91 & 190.81 & RW2 \\ 
  C\^{o}te d'Ivoire & NORD-EST & 1999 & 150.71 & 117.20 & 190.77 & RW2 \\ 
  C\^{o}te d'Ivoire & NORD-EST & 2000 & 150.46 & 117.34 & 190.52 & RW2 \\ 
  C\^{o}te d'Ivoire & NORD-EST & 2001 & 149.11 & 117.14 & 187.57 & RW2 \\ 
  C\^{o}te d'Ivoire & NORD-EST & 2002 & 146.79 & 115.82 & 184.66 & RW2 \\ 
  C\^{o}te d'Ivoire & NORD-EST & 2003 & 143.75 & 111.82 & 182.97 & RW2 \\ 
  C\^{o}te d'Ivoire & NORD-EST & 2004 & 140.05 & 107.78 & 180.70 & RW2 \\ 
  C\^{o}te d'Ivoire & NORD-EST & 2005 & 135.28 & 102.56 & 176.14 & RW2 \\ 
  C\^{o}te d'Ivoire & NORD-EST & 2006 & 131.18 & 99.30 & 171.54 & RW2 \\ 
  C\^{o}te d'Ivoire & NORD-EST & 2007 & 127.21 & 95.45 & 167.79 & RW2 \\ 
  C\^{o}te d'Ivoire & NORD-EST & 2008 & 123.49 & 90.83 & 165.73 & RW2 \\ 
  C\^{o}te d'Ivoire & NORD-EST & 2009 & 120.34 & 86.21 & 165.46 & RW2 \\ 
  C\^{o}te d'Ivoire & NORD-EST & 2010 & 117.48 & 82.28 & 165.93 & RW2 \\ 
  C\^{o}te d'Ivoire & NORD-EST & 2011 & 114.93 & 79.28 & 163.78 & RW2 \\ 
  C\^{o}te d'Ivoire & NORD-EST & 2012 & 112.40 & 76.05 & 163.45 & RW2 \\ 
  C\^{o}te d'Ivoire & NORD-EST & 2013 & 110.15 & 71.38 & 166.25 & RW2 \\ 
  C\^{o}te d'Ivoire & NORD-EST & 2014 & 107.76 & 64.55 & 175.48 & RW2 \\ 
  C\^{o}te d'Ivoire & NORD-EST & 2015 & 105.29 & 54.94 & 193.73 & RW2 \\ 
  C\^{o}te d'Ivoire & NORD-EST & 2016 & 102.97 & 45.80 & 216.57 & RW2 \\ 
  C\^{o}te d'Ivoire & NORD-EST & 2017 & 100.57 & 37.24 & 247.60 & RW2 \\ 
  C\^{o}te d'Ivoire & NORD-EST & 2018 & 98.26 & 29.06 & 287.95 & RW2 \\ 
  C\^{o}te d'Ivoire & NORD-EST & 2019 & 96.26 & 22.36 & 336.25 & RW2 \\ 
  C\^{o}te d'Ivoire & NORD-OUEST & 1980 & 243.85 & 144.88 & 381.07 & RW2 \\ 
  C\^{o}te d'Ivoire & NORD-OUEST & 1981 & 234.63 & 151.60 & 343.59 & RW2 \\ 
  C\^{o}te d'Ivoire & NORD-OUEST & 1982 & 226.17 & 153.64 & 318.68 & RW2 \\ 
  C\^{o}te d'Ivoire & NORD-OUEST & 1983 & 217.32 & 152.14 & 300.72 & RW2 \\ 
  C\^{o}te d'Ivoire & NORD-OUEST & 1984 & 209.74 & 149.84 & 285.96 & RW2 \\ 
  C\^{o}te d'Ivoire & NORD-OUEST & 1985 & 202.56 & 146.63 & 271.61 & RW2 \\ 
  C\^{o}te d'Ivoire & NORD-OUEST & 1986 & 197.52 & 145.34 & 261.14 & RW2 \\ 
  C\^{o}te d'Ivoire & NORD-OUEST & 1987 & 194.47 & 145.74 & 254.52 & RW2 \\ 
  C\^{o}te d'Ivoire & NORD-OUEST & 1988 & 192.97 & 145.46 & 250.67 & RW2 \\ 
  C\^{o}te d'Ivoire & NORD-OUEST & 1989 & 193.10 & 146.44 & 248.84 & RW2 \\ 
  C\^{o}te d'Ivoire & NORD-OUEST & 1990 & 195.23 & 150.98 & 248.65 & RW2 \\ 
  C\^{o}te d'Ivoire & NORD-OUEST & 1991 & 197.63 & 155.13 & 248.91 & RW2 \\ 
  C\^{o}te d'Ivoire & NORD-OUEST & 1992 & 200.47 & 159.45 & 249.63 & RW2 \\ 
  C\^{o}te d'Ivoire & NORD-OUEST & 1993 & 203.64 & 162.18 & 252.91 & RW2 \\ 
  C\^{o}te d'Ivoire & NORD-OUEST & 1994 & 206.89 & 165.47 & 255.74 & RW2 \\ 
  C\^{o}te d'Ivoire & NORD-OUEST & 1995 & 209.89 & 169.58 & 256.91 & RW2 \\ 
  C\^{o}te d'Ivoire & NORD-OUEST & 1996 & 213.17 & 175.21 & 257.56 & RW2 \\ 
  C\^{o}te d'Ivoire & NORD-OUEST & 1997 & 216.12 & 179.41 & 258.62 & RW2 \\ 
  C\^{o}te d'Ivoire & NORD-OUEST & 1998 & 219.01 & 181.61 & 261.54 & RW2 \\ 
  C\^{o}te d'Ivoire & NORD-OUEST & 1999 & 221.29 & 182.54 & 265.32 & RW2 \\ 
  C\^{o}te d'Ivoire & NORD-OUEST & 2000 & 222.89 & 185.30 & 266.36 & RW2 \\ 
  C\^{o}te d'Ivoire & NORD-OUEST & 2001 & 222.90 & 187.29 & 263.52 & RW2 \\ 
  C\^{o}te d'Ivoire & NORD-OUEST & 2002 & 221.60 & 187.36 & 260.84 & RW2 \\ 
  C\^{o}te d'Ivoire & NORD-OUEST & 2003 & 218.63 & 183.35 & 258.47 & RW2 \\ 
  C\^{o}te d'Ivoire & NORD-OUEST & 2004 & 214.66 & 177.84 & 257.25 & RW2 \\ 
  C\^{o}te d'Ivoire & NORD-OUEST & 2005 & 209.24 & 171.73 & 251.69 & RW2 \\ 
  C\^{o}te d'Ivoire & NORD-OUEST & 2006 & 204.52 & 168.41 & 245.05 & RW2 \\ 
  C\^{o}te d'Ivoire & NORD-OUEST & 2007 & 199.66 & 164.36 & 240.37 & RW2 \\ 
  C\^{o}te d'Ivoire & NORD-OUEST & 2008 & 195.62 & 158.21 & 238.73 & RW2 \\ 
  C\^{o}te d'Ivoire & NORD-OUEST & 2009 & 191.87 & 151.26 & 239.99 & RW2 \\ 
  C\^{o}te d'Ivoire & NORD-OUEST & 2010 & 189.28 & 145.45 & 242.54 & RW2 \\ 
  C\^{o}te d'Ivoire & NORD-OUEST & 2011 & 186.39 & 141.74 & 240.29 & RW2 \\ 
  C\^{o}te d'Ivoire & NORD-OUEST & 2012 & 183.85 & 137.73 & 241.51 & RW2 \\ 
  C\^{o}te d'Ivoire & NORD-OUEST & 2013 & 181.52 & 129.71 & 247.17 & RW2 \\ 
  C\^{o}te d'Ivoire & NORD-OUEST & 2014 & 179.09 & 117.29 & 262.31 & RW2 \\ 
  C\^{o}te d'Ivoire & NORD-OUEST & 2015 & 176.61 & 100.83 & 290.21 & RW2 \\ 
  C\^{o}te d'Ivoire & NORD-OUEST & 2016 & 174.19 & 84.21 & 325.76 & RW2 \\ 
  C\^{o}te d'Ivoire & NORD-OUEST & 2017 & 171.63 & 69.06 & 369.29 & RW2 \\ 
  C\^{o}te d'Ivoire & NORD-OUEST & 2018 & 168.71 & 55.65 & 417.86 & RW2 \\ 
  C\^{o}te d'Ivoire & NORD-OUEST & 2019 & 166.63 & 43.04 & 476.51 & RW2 \\ 
  C\^{o}te d'Ivoire & OUEST & 1980 & 351.80 & 189.95 & 540.58 & RW2 \\ 
  C\^{o}te d'Ivoire & OUEST & 1981 & 335.46 & 196.79 & 492.09 & RW2 \\ 
  C\^{o}te d'Ivoire & OUEST & 1982 & 318.31 & 196.87 & 454.51 & RW2 \\ 
  C\^{o}te d'Ivoire & OUEST & 1983 & 302.11 & 194.34 & 424.49 & RW2 \\ 
  C\^{o}te d'Ivoire & OUEST & 1984 & 287.02 & 190.70 & 398.14 & RW2 \\ 
  C\^{o}te d'Ivoire & OUEST & 1985 & 272.85 & 187.42 & 371.92 & RW2 \\ 
  C\^{o}te d'Ivoire & OUEST & 1986 & 261.42 & 184.66 & 349.44 & RW2 \\ 
  C\^{o}te d'Ivoire & OUEST & 1987 & 252.03 & 183.31 & 332.54 & RW2 \\ 
  C\^{o}te d'Ivoire & OUEST & 1988 & 244.79 & 181.75 & 318.03 & RW2 \\ 
  C\^{o}te d'Ivoire & OUEST & 1989 & 239.50 & 181.74 & 307.19 & RW2 \\ 
  C\^{o}te d'Ivoire & OUEST & 1990 & 236.71 & 183.30 & 299.52 & RW2 \\ 
  C\^{o}te d'Ivoire & OUEST & 1991 & 233.92 & 184.86 & 291.13 & RW2 \\ 
  C\^{o}te d'Ivoire & OUEST & 1992 & 231.44 & 185.39 & 285.30 & RW2 \\ 
  C\^{o}te d'Ivoire & OUEST & 1993 & 229.05 & 184.97 & 280.85 & RW2 \\ 
  C\^{o}te d'Ivoire & OUEST & 1994 & 226.56 & 183.73 & 277.31 & RW2 \\ 
  C\^{o}te d'Ivoire & OUEST & 1995 & 223.86 & 182.27 & 271.30 & RW2 \\ 
  C\^{o}te d'Ivoire & OUEST & 1996 & 220.83 & 182.98 & 265.03 & RW2 \\ 
  C\^{o}te d'Ivoire & OUEST & 1997 & 217.76 & 181.34 & 259.02 & RW2 \\ 
  C\^{o}te d'Ivoire & OUEST & 1998 & 213.96 & 177.59 & 255.08 & RW2 \\ 
  C\^{o}te d'Ivoire & OUEST & 1999 & 209.78 & 173.15 & 251.97 & RW2 \\ 
  C\^{o}te d'Ivoire & OUEST & 2000 & 205.09 & 169.14 & 245.97 & RW2 \\ 
  C\^{o}te d'Ivoire & OUEST & 2001 & 198.93 & 165.81 & 237.15 & RW2 \\ 
  C\^{o}te d'Ivoire & OUEST & 2002 & 191.47 & 160.24 & 227.71 & RW2 \\ 
  C\^{o}te d'Ivoire & OUEST & 2003 & 182.99 & 151.87 & 219.54 & RW2 \\ 
  C\^{o}te d'Ivoire & OUEST & 2004 & 173.81 & 141.68 & 211.74 & RW2 \\ 
  C\^{o}te d'Ivoire & OUEST & 2005 & 163.77 & 132.07 & 200.71 & RW2 \\ 
  C\^{o}te d'Ivoire & OUEST & 2006 & 154.62 & 124.70 & 189.64 & RW2 \\ 
  C\^{o}te d'Ivoire & OUEST & 2007 & 145.97 & 116.99 & 180.60 & RW2 \\ 
  C\^{o}te d'Ivoire & OUEST & 2008 & 138.10 & 108.55 & 173.88 & RW2 \\ 
  C\^{o}te d'Ivoire & OUEST & 2009 & 130.91 & 99.64 & 169.89 & RW2 \\ 
  C\^{o}te d'Ivoire & OUEST & 2010 & 124.78 & 91.87 & 166.69 & RW2 \\ 
  C\^{o}te d'Ivoire & OUEST & 2011 & 118.89 & 85.93 & 161.41 & RW2 \\ 
  C\^{o}te d'Ivoire & OUEST & 2012 & 113.28 & 79.61 & 157.68 & RW2 \\ 
  C\^{o}te d'Ivoire & OUEST & 2013 & 107.93 & 72.24 & 157.11 & RW2 \\ 
  C\^{o}te d'Ivoire & OUEST & 2014 & 102.80 & 62.99 & 162.54 & RW2 \\ 
  C\^{o}te d'Ivoire & OUEST & 2015 & 97.91 & 51.40 & 177.09 & RW2 \\ 
  C\^{o}te d'Ivoire & OUEST & 2016 & 93.41 & 41.70 & 195.63 & RW2 \\ 
  C\^{o}te d'Ivoire & OUEST & 2017 & 88.55 & 32.51 & 222.32 & RW2 \\ 
  C\^{o}te d'Ivoire & OUEST & 2018 & 84.35 & 24.86 & 252.53 & RW2 \\ 
  C\^{o}te d'Ivoire & OUEST & 2019 & 80.32 & 18.60 & 294.67 & RW2 \\ 
  C\^{o}te d'Ivoire & SUD SANS ABIDJAN & 1980 & 215.08 & 106.20 & 385.92 & RW2 \\ 
  C\^{o}te d'Ivoire & SUD SANS ABIDJAN & 1981 & 203.70 & 109.93 & 346.09 & RW2 \\ 
  C\^{o}te d'Ivoire & SUD SANS ABIDJAN & 1982 & 192.76 & 109.49 & 313.45 & RW2 \\ 
  C\^{o}te d'Ivoire & SUD SANS ABIDJAN & 1983 & 182.15 & 107.77 & 289.56 & RW2 \\ 
  C\^{o}te d'Ivoire & SUD SANS ABIDJAN & 1984 & 172.64 & 105.58 & 269.71 & RW2 \\ 
  C\^{o}te d'Ivoire & SUD SANS ABIDJAN & 1985 & 163.42 & 102.62 & 249.03 & RW2 \\ 
  C\^{o}te d'Ivoire & SUD SANS ABIDJAN & 1986 & 156.46 & 101.38 & 233.83 & RW2 \\ 
  C\^{o}te d'Ivoire & SUD SANS ABIDJAN & 1987 & 151.15 & 100.55 & 221.18 & RW2 \\ 
  C\^{o}te d'Ivoire & SUD SANS ABIDJAN & 1988 & 147.19 & 99.90 & 210.83 & RW2 \\ 
  C\^{o}te d'Ivoire & SUD SANS ABIDJAN & 1989 & 144.56 & 100.33 & 204.14 & RW2 \\ 
  C\^{o}te d'Ivoire & SUD SANS ABIDJAN & 1990 & 143.60 & 101.99 & 198.12 & RW2 \\ 
  C\^{o}te d'Ivoire & SUD SANS ABIDJAN & 1991 & 142.97 & 104.04 & 193.60 & RW2 \\ 
  C\^{o}te d'Ivoire & SUD SANS ABIDJAN & 1992 & 142.31 & 105.58 & 188.78 & RW2 \\ 
  C\^{o}te d'Ivoire & SUD SANS ABIDJAN & 1993 & 142.08 & 106.71 & 187.04 & RW2 \\ 
  C\^{o}te d'Ivoire & SUD SANS ABIDJAN & 1994 & 142.02 & 107.45 & 185.45 & RW2 \\ 
  C\^{o}te d'Ivoire & SUD SANS ABIDJAN & 1995 & 141.56 & 108.35 & 182.63 & RW2 \\ 
  C\^{o}te d'Ivoire & SUD SANS ABIDJAN & 1996 & 141.59 & 110.40 & 179.27 & RW2 \\ 
  C\^{o}te d'Ivoire & SUD SANS ABIDJAN & 1997 & 141.25 & 111.40 & 177.27 & RW2 \\ 
  C\^{o}te d'Ivoire & SUD SANS ABIDJAN & 1998 & 140.89 & 111.39 & 176.23 & RW2 \\ 
  C\^{o}te d'Ivoire & SUD SANS ABIDJAN & 1999 & 140.54 & 110.89 & 176.34 & RW2 \\ 
  C\^{o}te d'Ivoire & SUD SANS ABIDJAN & 2000 & 139.84 & 111.13 & 174.59 & RW2 \\ 
  C\^{o}te d'Ivoire & SUD SANS ABIDJAN & 2001 & 138.08 & 111.34 & 170.47 & RW2 \\ 
  C\^{o}te d'Ivoire & SUD SANS ABIDJAN & 2002 & 135.56 & 109.47 & 166.67 & RW2 \\ 
  C\^{o}te d'Ivoire & SUD SANS ABIDJAN & 2003 & 132.14 & 106.29 & 163.75 & RW2 \\ 
  C\^{o}te d'Ivoire & SUD SANS ABIDJAN & 2004 & 128.15 & 101.91 & 160.28 & RW2 \\ 
  C\^{o}te d'Ivoire & SUD SANS ABIDJAN & 2005 & 123.19 & 97.43 & 154.30 & RW2 \\ 
  C\^{o}te d'Ivoire & SUD SANS ABIDJAN & 2006 & 118.85 & 94.42 & 148.59 & RW2 \\ 
  C\^{o}te d'Ivoire & SUD SANS ABIDJAN & 2007 & 114.45 & 90.59 & 143.16 & RW2 \\ 
  C\^{o}te d'Ivoire & SUD SANS ABIDJAN & 2008 & 110.41 & 85.79 & 141.18 & RW2 \\ 
  C\^{o}te d'Ivoire & SUD SANS ABIDJAN & 2009 & 106.95 & 80.58 & 140.56 & RW2 \\ 
  C\^{o}te d'Ivoire & SUD SANS ABIDJAN & 2010 & 103.88 & 75.82 & 141.08 & RW2 \\ 
  C\^{o}te d'Ivoire & SUD SANS ABIDJAN & 2011 & 100.93 & 72.07 & 139.33 & RW2 \\ 
  C\^{o}te d'Ivoire & SUD SANS ABIDJAN & 2012 & 98.00 & 68.04 & 138.95 & RW2 \\ 
  C\^{o}te d'Ivoire & SUD SANS ABIDJAN & 2013 & 95.31 & 62.81 & 141.53 & RW2 \\ 
  C\^{o}te d'Ivoire & SUD SANS ABIDJAN & 2014 & 92.65 & 55.55 & 149.29 & RW2 \\ 
  C\^{o}te d'Ivoire & SUD SANS ABIDJAN & 2015 & 89.96 & 46.66 & 166.28 & RW2 \\ 
  C\^{o}te d'Ivoire & SUD SANS ABIDJAN & 2016 & 87.19 & 38.51 & 187.28 & RW2 \\ 
  C\^{o}te d'Ivoire & SUD SANS ABIDJAN & 2017 & 84.87 & 30.53 & 214.95 & RW2 \\ 
  C\^{o}te d'Ivoire & SUD SANS ABIDJAN & 2018 & 82.16 & 23.74 & 248.44 & RW2 \\ 
  C\^{o}te d'Ivoire & SUD SANS ABIDJAN & 2019 & 79.66 & 18.28 & 291.81 & RW2 \\ 
  C\^{o}te d'Ivoire & SUD-OUEST & 1980 & 224.17 & 105.08 & 414.28 & RW2 \\ 
  C\^{o}te d'Ivoire & SUD-OUEST & 1981 & 209.77 & 105.66 & 370.70 & RW2 \\ 
  C\^{o}te d'Ivoire & SUD-OUEST & 1982 & 196.08 & 105.67 & 335.12 & RW2 \\ 
  C\^{o}te d'Ivoire & SUD-OUEST & 1983 & 183.24 & 102.72 & 304.57 & RW2 \\ 
  C\^{o}te d'Ivoire & SUD-OUEST & 1984 & 171.16 & 99.06 & 280.09 & RW2 \\ 
  C\^{o}te d'Ivoire & SUD-OUEST & 1985 & 160.32 & 95.78 & 255.51 & RW2 \\ 
  C\^{o}te d'Ivoire & SUD-OUEST & 1986 & 151.69 & 93.56 & 235.42 & RW2 \\ 
  C\^{o}te d'Ivoire & SUD-OUEST & 1987 & 144.67 & 91.38 & 219.79 & RW2 \\ 
  C\^{o}te d'Ivoire & SUD-OUEST & 1988 & 139.03 & 90.00 & 207.53 & RW2 \\ 
  C\^{o}te d'Ivoire & SUD-OUEST & 1989 & 134.66 & 88.86 & 198.66 & RW2 \\ 
  C\^{o}te d'Ivoire & SUD-OUEST & 1990 & 131.78 & 89.13 & 191.18 & RW2 \\ 
  C\^{o}te d'Ivoire & SUD-OUEST & 1991 & 129.11 & 89.10 & 183.84 & RW2 \\ 
  C\^{o}te d'Ivoire & SUD-OUEST & 1992 & 127.06 & 88.73 & 178.22 & RW2 \\ 
  C\^{o}te d'Ivoire & SUD-OUEST & 1993 & 124.72 & 88.03 & 174.30 & RW2 \\ 
  C\^{o}te d'Ivoire & SUD-OUEST & 1994 & 122.78 & 87.49 & 170.14 & RW2 \\ 
  C\^{o}te d'Ivoire & SUD-OUEST & 1995 & 120.44 & 86.85 & 165.31 & RW2 \\ 
  C\^{o}te d'Ivoire & SUD-OUEST & 1996 & 118.47 & 86.42 & 160.13 & RW2 \\ 
  C\^{o}te d'Ivoire & SUD-OUEST & 1997 & 116.36 & 85.84 & 155.24 & RW2 \\ 
  C\^{o}te d'Ivoire & SUD-OUEST & 1998 & 114.08 & 84.38 & 151.98 & RW2 \\ 
  C\^{o}te d'Ivoire & SUD-OUEST & 1999 & 111.67 & 82.91 & 148.27 & RW2 \\ 
  C\^{o}te d'Ivoire & SUD-OUEST & 2000 & 109.27 & 81.45 & 144.27 & RW2 \\ 
  C\^{o}te d'Ivoire & SUD-OUEST & 2001 & 105.97 & 80.37 & 138.32 & RW2 \\ 
  C\^{o}te d'Ivoire & SUD-OUEST & 2002 & 102.21 & 78.13 & 132.27 & RW2 \\ 
  C\^{o}te d'Ivoire & SUD-OUEST & 2003 & 97.80 & 74.72 & 126.98 & RW2 \\ 
  C\^{o}te d'Ivoire & SUD-OUEST & 2004 & 93.15 & 70.60 & 122.23 & RW2 \\ 
  C\^{o}te d'Ivoire & SUD-OUEST & 2005 & 88.06 & 66.56 & 115.38 & RW2 \\ 
  C\^{o}te d'Ivoire & SUD-OUEST & 2006 & 83.45 & 63.57 & 109.12 & RW2 \\ 
  C\^{o}te d'Ivoire & SUD-OUEST & 2007 & 79.11 & 59.99 & 103.49 & RW2 \\ 
  C\^{o}te d'Ivoire & SUD-OUEST & 2008 & 75.24 & 56.01 & 100.61 & RW2 \\ 
  C\^{o}te d'Ivoire & SUD-OUEST & 2009 & 71.74 & 51.84 & 98.70 & RW2 \\ 
  C\^{o}te d'Ivoire & SUD-OUEST & 2010 & 68.77 & 47.95 & 98.47 & RW2 \\ 
  C\^{o}te d'Ivoire & SUD-OUEST & 2011 & 65.80 & 44.74 & 96.37 & RW2 \\ 
  C\^{o}te d'Ivoire & SUD-OUEST & 2012 & 63.14 & 41.52 & 95.48 & RW2 \\ 
  C\^{o}te d'Ivoire & SUD-OUEST & 2013 & 60.60 & 37.63 & 97.01 & RW2 \\ 
  C\^{o}te d'Ivoire & SUD-OUEST & 2014 & 58.18 & 32.96 & 101.90 & RW2 \\ 
  C\^{o}te d'Ivoire & SUD-OUEST & 2015 & 55.62 & 27.13 & 111.61 & RW2 \\ 
  C\^{o}te d'Ivoire & SUD-OUEST & 2016 & 53.33 & 22.14 & 125.19 & RW2 \\ 
  C\^{o}te d'Ivoire & SUD-OUEST & 2017 & 51.22 & 17.57 & 143.59 & RW2 \\ 
  C\^{o}te d'Ivoire & SUD-OUEST & 2018 & 49.12 & 13.65 & 168.29 & RW2 \\ 
  C\^{o}te d'Ivoire & SUD-OUEST & 2019 & 47.03 & 10.03 & 200.27 & RW2 \\ 
  C\^{o}te d'Ivoire & VILLE D'ABIDJAN & 1980 & 170.32 & 77.28 & 346.89 & RW2 \\ 
  C\^{o}te d'Ivoire & VILLE D'ABIDJAN & 1981 & 160.61 & 79.98 & 305.05 & RW2 \\ 
  C\^{o}te d'Ivoire & VILLE D'ABIDJAN & 1982 & 150.85 & 80.86 & 272.29 & RW2 \\ 
  C\^{o}te d'Ivoire & VILLE D'ABIDJAN & 1983 & 141.92 & 79.76 & 244.95 & RW2 \\ 
  C\^{o}te d'Ivoire & VILLE D'ABIDJAN & 1984 & 133.78 & 77.94 & 223.93 & RW2 \\ 
  C\^{o}te d'Ivoire & VILLE D'ABIDJAN & 1985 & 126.14 & 76.39 & 203.26 & RW2 \\ 
  C\^{o}te d'Ivoire & VILLE D'ABIDJAN & 1986 & 120.08 & 76.09 & 187.36 & RW2 \\ 
  C\^{o}te d'Ivoire & VILLE D'ABIDJAN & 1987 & 115.29 & 75.07 & 174.60 & RW2 \\ 
  C\^{o}te d'Ivoire & VILLE D'ABIDJAN & 1988 & 111.97 & 74.21 & 165.87 & RW2 \\ 
  C\^{o}te d'Ivoire & VILLE D'ABIDJAN & 1989 & 109.66 & 74.56 & 158.98 & RW2 \\ 
  C\^{o}te d'Ivoire & VILLE D'ABIDJAN & 1990 & 108.51 & 75.71 & 153.84 & RW2 \\ 
  C\^{o}te d'Ivoire & VILLE D'ABIDJAN & 1991 & 107.48 & 76.59 & 149.13 & RW2 \\ 
  C\^{o}te d'Ivoire & VILLE D'ABIDJAN & 1992 & 106.80 & 77.48 & 145.96 & RW2 \\ 
  C\^{o}te d'Ivoire & VILLE D'ABIDJAN & 1993 & 106.61 & 77.95 & 143.99 & RW2 \\ 
  C\^{o}te d'Ivoire & VILLE D'ABIDJAN & 1994 & 106.51 & 78.52 & 143.05 & RW2 \\ 
  C\^{o}te d'Ivoire & VILLE D'ABIDJAN & 1995 & 106.12 & 79.01 & 140.54 & RW2 \\ 
  C\^{o}te d'Ivoire & VILLE D'ABIDJAN & 1996 & 106.22 & 80.55 & 138.14 & RW2 \\ 
  C\^{o}te d'Ivoire & VILLE D'ABIDJAN & 1997 & 106.16 & 81.11 & 137.28 & RW2 \\ 
  C\^{o}te d'Ivoire & VILLE D'ABIDJAN & 1998 & 106.36 & 81.20 & 137.17 & RW2 \\ 
  C\^{o}te d'Ivoire & VILLE D'ABIDJAN & 1999 & 106.45 & 80.89 & 137.73 & RW2 \\ 
  C\^{o}te d'Ivoire & VILLE D'ABIDJAN & 2000 & 106.41 & 81.19 & 137.45 & RW2 \\ 
  C\^{o}te d'Ivoire & VILLE D'ABIDJAN & 2001 & 105.84 & 81.63 & 135.42 & RW2 \\ 
  C\^{o}te d'Ivoire & VILLE D'ABIDJAN & 2002 & 104.67 & 81.08 & 133.25 & RW2 \\ 
  C\^{o}te d'Ivoire & VILLE D'ABIDJAN & 2003 & 102.92 & 79.55 & 131.58 & RW2 \\ 
  C\^{o}te d'Ivoire & VILLE D'ABIDJAN & 2004 & 100.86 & 77.73 & 129.94 & RW2 \\ 
  C\^{o}te d'Ivoire & VILLE D'ABIDJAN & 2005 & 98.07 & 75.05 & 127.11 & RW2 \\ 
  C\^{o}te d'Ivoire & VILLE D'ABIDJAN & 2006 & 95.68 & 73.79 & 123.10 & RW2 \\ 
  C\^{o}te d'Ivoire & VILLE D'ABIDJAN & 2007 & 93.35 & 71.91 & 120.57 & RW2 \\ 
  C\^{o}te d'Ivoire & VILLE D'ABIDJAN & 2008 & 91.28 & 68.84 & 120.24 & RW2 \\ 
  C\^{o}te d'Ivoire & VILLE D'ABIDJAN & 2009 & 89.46 & 65.68 & 121.18 & RW2 \\ 
  C\^{o}te d'Ivoire & VILLE D'ABIDJAN & 2010 & 87.93 & 62.50 & 123.70 & RW2 \\ 
  C\^{o}te d'Ivoire & VILLE D'ABIDJAN & 2011 & 86.55 & 59.86 & 124.62 & RW2 \\ 
  C\^{o}te d'Ivoire & VILLE D'ABIDJAN & 2012 & 85.21 & 57.15 & 126.41 & RW2 \\ 
  C\^{o}te d'Ivoire & VILLE D'ABIDJAN & 2013 & 83.86 & 53.12 & 131.27 & RW2 \\ 
  C\^{o}te d'Ivoire & VILLE D'ABIDJAN & 2014 & 82.52 & 47.46 & 141.62 & RW2 \\ 
  C\^{o}te d'Ivoire & VILLE D'ABIDJAN & 2015 & 81.18 & 40.33 & 159.93 & RW2 \\ 
  C\^{o}te d'Ivoire & VILLE D'ABIDJAN & 2016 & 80.18 & 33.09 & 182.73 & RW2 \\ 
  C\^{o}te d'Ivoire & VILLE D'ABIDJAN & 2017 & 78.50 & 26.94 & 213.21 & RW2 \\ 
  C\^{o}te d'Ivoire & VILLE D'ABIDJAN & 2018 & 77.22 & 21.26 & 250.19 & RW2 \\ 
  C\^{o}te d'Ivoire & VILLE D'ABIDJAN & 2019 & 76.37 & 16.22 & 296.78 & RW2 \\ 
  DRC & ALL & 1980 & 185.30 & 167.38 & 206.56 & IHME \\ 
  DRC & ALL & 1980 & 214.12 & 143.93 & 307.10 & RW2 \\ 
  DRC & ALL & 1980 & 212.00 & 184.00 & 242.60 & UN \\ 
  DRC & ALL & 1981 & 182.37 & 164.97 & 201.67 & IHME \\ 
  DRC & ALL & 1981 & 210.21 & 154.92 & 278.80 & RW2 \\ 
  DRC & ALL & 1981 & 208.80 & 182.30 & 237.40 & UN \\ 
  DRC & ALL & 1982 & 178.65 & 163.24 & 196.21 & IHME \\ 
  DRC & ALL & 1982 & 206.47 & 159.95 & 262.56 & RW2 \\ 
  DRC & ALL & 1982 & 205.70 & 180.50 & 232.50 & UN \\ 
  DRC & ALL & 1983 & 173.80 & 158.27 & 188.44 & IHME \\ 
  DRC & ALL & 1983 & 202.42 & 158.58 & 255.21 & RW2 \\ 
  DRC & ALL & 1983 & 202.80 & 178.90 & 228.20 & UN \\ 
  DRC & ALL & 1984 & 166.76 & 153.41 & 180.69 & IHME \\ 
  DRC & ALL & 1984 & 199.14 & 154.80 & 252.02 & RW2 \\ 
  DRC & ALL & 1984 & 200.10 & 177.20 & 223.90 & UN \\ 
  DRC & ALL & 1985 & 162.34 & 150.44 & 176.52 & IHME \\ 
  DRC & ALL & 1985 & 195.28 & 153.84 & 244.07 & RW2 \\ 
  DRC & ALL & 1985 & 197.70 & 175.70 & 220.30 & UN \\ 
  DRC & ALL & 1986 & 159.41 & 147.86 & 172.17 & IHME \\ 
  DRC & ALL & 1986 & 192.49 & 153.20 & 238.22 & RW2 \\ 
  DRC & ALL & 1986 & 195.30 & 174.20 & 216.70 & UN \\ 
  DRC & ALL & 1987 & 159.06 & 147.29 & 171.15 & IHME \\ 
  DRC & ALL & 1987 & 190.32 & 152.91 & 234.07 & RW2 \\ 
  DRC & ALL & 1987 & 193.00 & 172.90 & 213.30 & UN \\ 
  DRC & ALL & 1988 & 160.82 & 149.39 & 172.78 & IHME \\ 
  DRC & ALL & 1988 & 188.46 & 150.92 & 231.83 & RW2 \\ 
  DRC & ALL & 1988 & 190.80 & 171.70 & 210.40 & UN \\ 
  DRC & ALL & 1989 & 163.58 & 151.82 & 176.31 & IHME \\ 
  DRC & ALL & 1989 & 187.07 & 149.52 & 230.55 & RW2 \\ 
  DRC & ALL & 1989 & 188.50 & 170.40 & 207.60 & UN \\ 
  DRC & ALL & 1990 & 166.62 & 154.87 & 178.44 & IHME \\ 
  DRC & ALL & 1990 & 186.30 & 150.62 & 228.72 & RW2 \\ 
  DRC & ALL & 1990 & 186.50 & 169.10 & 204.70 & UN \\ 
  DRC & ALL & 1991 & 167.40 & 155.46 & 179.60 & IHME \\ 
  DRC & ALL & 1991 & 185.19 & 151.37 & 224.35 & RW2 \\ 
  DRC & ALL & 1991 & 184.30 & 167.30 & 202.00 & UN \\ 
  DRC & ALL & 1992 & 166.23 & 154.71 & 178.44 & IHME \\ 
  DRC & ALL & 1992 & 183.92 & 150.98 & 221.90 & RW2 \\ 
  DRC & ALL & 1992 & 182.30 & 165.90 & 199.50 & UN \\ 
  DRC & ALL & 1993 & 165.32 & 153.73 & 177.38 & IHME \\ 
  DRC & ALL & 1993 & 182.39 & 149.27 & 221.04 & RW2 \\ 
  DRC & ALL & 1993 & 180.30 & 164.50 & 197.30 & UN \\ 
  DRC & ALL & 1994 & 164.57 & 153.03 & 176.94 & IHME \\ 
  DRC & ALL & 1994 & 180.38 & 146.42 & 221.02 & RW2 \\ 
  DRC & ALL & 1994 & 178.30 & 162.90 & 195.30 & UN \\ 
  DRC & ALL & 1995 & 164.58 & 152.23 & 176.41 & IHME \\ 
  DRC & ALL & 1995 & 178.07 & 145.14 & 216.57 & RW2 \\ 
  DRC & ALL & 1995 & 176.40 & 161.20 & 193.30 & UN \\ 
  DRC & ALL & 1996 & 166.48 & 154.69 & 178.94 & IHME \\ 
  DRC & ALL & 1996 & 175.25 & 144.24 & 211.87 & RW2 \\ 
  DRC & ALL & 1996 & 174.10 & 159.00 & 191.10 & UN \\ 
  DRC & ALL & 1997 & 164.22 & 152.62 & 175.57 & IHME \\ 
  DRC & ALL & 1997 & 172.09 & 142.39 & 206.67 & RW2 \\ 
  DRC & ALL & 1997 & 171.50 & 156.50 & 188.70 & UN \\ 
  DRC & ALL & 1998 & 161.33 & 150.49 & 172.82 & IHME \\ 
  DRC & ALL & 1998 & 168.67 & 138.74 & 204.39 & RW2 \\ 
  DRC & ALL & 1998 & 168.40 & 153.50 & 185.70 & UN \\ 
  DRC & ALL & 1999 & 157.09 & 145.78 & 168.19 & IHME \\ 
  DRC & ALL & 1999 & 164.87 & 134.17 & 200.63 & RW2 \\ 
  DRC & ALL & 1999 & 164.90 & 150.20 & 181.90 & UN \\ 
  DRC & ALL & 2000 & 151.83 & 140.18 & 163.69 & IHME \\ 
  DRC & ALL & 2000 & 160.75 & 131.23 & 195.10 & RW2 \\ 
  DRC & ALL & 2000 & 161.00 & 146.50 & 177.80 & UN \\ 
  DRC & ALL & 2001 & 147.58 & 136.66 & 159.57 & IHME \\ 
  DRC & ALL & 2001 & 156.51 & 128.85 & 188.60 & RW2 \\ 
  DRC & ALL & 2001 & 156.80 & 142.50 & 173.10 & UN \\ 
  DRC & ALL & 2002 & 143.83 & 133.08 & 155.57 & IHME \\ 
  DRC & ALL & 2002 & 152.08 & 126.02 & 182.49 & RW2 \\ 
  DRC & ALL & 2002 & 152.40 & 138.30 & 168.20 & UN \\ 
  DRC & ALL & 2003 & 138.43 & 128.24 & 149.22 & IHME \\ 
  DRC & ALL & 2003 & 147.66 & 121.83 & 177.95 & RW2 \\ 
  DRC & ALL & 2003 & 147.80 & 133.90 & 163.10 & UN \\ 
  DRC & ALL & 2004 & 132.63 & 122.08 & 142.97 & IHME \\ 
  DRC & ALL & 2004 & 142.94 & 116.09 & 174.53 & RW2 \\ 
  DRC & ALL & 2004 & 143.20 & 129.30 & 158.50 & UN \\ 
  DRC & ALL & 2005 & 128.84 & 118.02 & 139.01 & IHME \\ 
  DRC & ALL & 2005 & 138.45 & 112.14 & 169.80 & RW2 \\ 
  DRC & ALL & 2005 & 138.40 & 124.40 & 153.90 & UN \\ 
  DRC & ALL & 2006 & 125.91 & 115.74 & 137.62 & IHME \\ 
  DRC & ALL & 2006 & 133.81 & 109.44 & 162.47 & RW2 \\ 
  DRC & ALL & 2006 & 133.50 & 119.40 & 149.80 & UN \\ 
  DRC & ALL & 2007 & 121.97 & 111.49 & 133.14 & IHME \\ 
  DRC & ALL & 2007 & 129.27 & 106.42 & 156.11 & RW2 \\ 
  DRC & ALL & 2007 & 128.90 & 114.00 & 146.30 & UN \\ 
  DRC & ALL & 2008 & 116.64 & 106.11 & 128.71 & IHME \\ 
  DRC & ALL & 2008 & 124.93 & 102.01 & 152.26 & RW2 \\ 
  DRC & ALL & 2008 & 124.50 & 108.20 & 143.00 & UN \\ 
  DRC & ALL & 2009 & 111.73 & 101.21 & 122.96 & IHME \\ 
  DRC & ALL & 2009 & 120.42 & 96.66 & 149.52 & RW2 \\ 
  DRC & ALL & 2009 & 120.10 & 102.70 & 140.10 & UN \\ 
  DRC & ALL & 2010 & 107.20 & 96.35 & 118.74 & IHME \\ 
  DRC & ALL & 2010 & 116.09 & 91.87 & 146.38 & RW2 \\ 
  DRC & ALL & 2010 & 116.10 & 96.80 & 137.60 & UN \\ 
  DRC & ALL & 2011 & 103.57 & 93.48 & 114.68 & IHME \\ 
  DRC & ALL & 2011 & 111.97 & 89.52 & 139.17 & RW2 \\ 
  DRC & ALL & 2011 & 112.20 & 91.40 & 135.40 & UN \\ 
  DRC & ALL & 2012 & 100.15 & 89.88 & 112.15 & IHME \\ 
  DRC & ALL & 2012 & 107.88 & 87.45 & 132.28 & RW2 \\ 
  DRC & ALL & 2012 & 108.50 & 86.10 & 133.00 & UN \\ 
  DRC & ALL & 2013 & 96.70 & 84.40 & 109.85 & IHME \\ 
  DRC & ALL & 2013 & 104.00 & 82.46 & 130.16 & RW2 \\ 
  DRC & ALL & 2013 & 104.80 & 81.00 & 131.30 & UN \\ 
  DRC & ALL & 2014 & 92.43 & 77.08 & 109.25 & IHME \\ 
  DRC & ALL & 2014 & 100.15 & 72.24 & 137.07 & RW2 \\ 
  DRC & ALL & 2014 & 101.70 & 76.10 & 130.30 & UN \\ 
  DRC & ALL & 2015 & 88.04 & 70.19 & 108.92 & IHME \\ 
  DRC & ALL & 2015 & 96.32 & 58.41 & 154.94 & RW2 \\ 
  DRC & ALL & 2015 & 98.30 & 71.30 & 129.80 & UN \\ 
  DRC & ALL & 2016 & 92.94 & 47.14 & 176.12 & RW2 \\ 
  DRC & ALL & 2017 & 89.27 & 36.65 & 203.26 & RW2 \\ 
  DRC & ALL & 2018 & 85.90 & 28.16 & 238.81 & RW2 \\ 
  DRC & ALL & 2019 & 82.53 & 20.38 & 277.54 & RW2 \\ 
  DRC & ALL & 80-84 & 168.47 & 195.52 & 144.49 & HT-Direct \\ 
  DRC & ALL & 85-89 & 156.77 & 175.63 & 139.59 & HT-Direct \\ 
  DRC & ALL & 90-94 & 181.11 & 199.80 & 163.82 & HT-Direct \\ 
  DRC & ALL & 95-99 & 173.76 & 188.56 & 159.88 & HT-Direct \\ 
  DRC & ALL & 00-04 & 144.53 & 153.83 & 135.69 & HT-Direct \\ 
  DRC & ALL & 05-09 & 122.01 & 130.86 & 113.68 & HT-Direct \\ 
  DRC & ALL & 10-14 & 103.96 & 113.81 & 94.87 & HT-Direct \\ 
  DRC & ALL & 15-19 & 89.28 & 37.24 & 199.95 & RW2 \\ 
  DRC & BANDUNDU & 1980 & 249.40 & 163.10 & 359.84 & RW2 \\ 
  DRC & BANDUNDU & 1981 & 243.97 & 171.48 & 333.15 & RW2 \\ 
  DRC & BANDUNDU & 1982 & 238.31 & 173.10 & 317.26 & RW2 \\ 
  DRC & BANDUNDU & 1983 & 232.43 & 170.68 & 307.65 & RW2 \\ 
  DRC & BANDUNDU & 1984 & 226.56 & 167.64 & 298.32 & RW2 \\ 
  DRC & BANDUNDU & 1985 & 220.80 & 165.14 & 288.37 & RW2 \\ 
  DRC & BANDUNDU & 1986 & 214.55 & 162.74 & 277.05 & RW2 \\ 
  DRC & BANDUNDU & 1987 & 207.93 & 160.15 & 265.63 & RW2 \\ 
  DRC & BANDUNDU & 1988 & 201.06 & 155.83 & 255.85 & RW2 \\ 
  DRC & BANDUNDU & 1989 & 194.24 & 151.61 & 246.26 & RW2 \\ 
  DRC & BANDUNDU & 1990 & 187.03 & 147.82 & 233.92 & RW2 \\ 
  DRC & BANDUNDU & 1991 & 181.52 & 145.77 & 224.04 & RW2 \\ 
  DRC & BANDUNDU & 1992 & 176.92 & 143.26 & 216.34 & RW2 \\ 
  DRC & BANDUNDU & 1993 & 173.16 & 140.60 & 211.10 & RW2 \\ 
  DRC & BANDUNDU & 1994 & 170.12 & 137.33 & 208.32 & RW2 \\ 
  DRC & BANDUNDU & 1995 & 168.19 & 136.94 & 205.54 & RW2 \\ 
  DRC & BANDUNDU & 1996 & 165.42 & 136.04 & 199.99 & RW2 \\ 
  DRC & BANDUNDU & 1997 & 162.08 & 134.26 & 194.36 & RW2 \\ 
  DRC & BANDUNDU & 1998 & 157.92 & 130.59 & 190.35 & RW2 \\ 
  DRC & BANDUNDU & 1999 & 152.91 & 125.21 & 185.89 & RW2 \\ 
  DRC & BANDUNDU & 2000 & 146.93 & 120.02 & 177.84 & RW2 \\ 
  DRC & BANDUNDU & 2001 & 140.59 & 116.42 & 168.66 & RW2 \\ 
  DRC & BANDUNDU & 2002 & 133.89 & 111.39 & 160.17 & RW2 \\ 
  DRC & BANDUNDU & 2003 & 126.87 & 104.98 & 152.97 & RW2 \\ 
  DRC & BANDUNDU & 2004 & 120.01 & 98.03 & 146.39 & RW2 \\ 
  DRC & BANDUNDU & 2005 & 113.20 & 91.51 & 139.02 & RW2 \\ 
  DRC & BANDUNDU & 2006 & 106.95 & 86.95 & 130.84 & RW2 \\ 
  DRC & BANDUNDU & 2007 & 101.17 & 82.27 & 123.72 & RW2 \\ 
  DRC & BANDUNDU & 2008 & 96.01 & 77.11 & 118.86 & RW2 \\ 
  DRC & BANDUNDU & 2009 & 91.23 & 71.90 & 115.12 & RW2 \\ 
  DRC & BANDUNDU & 2010 & 87.19 & 67.20 & 112.34 & RW2 \\ 
  DRC & BANDUNDU & 2011 & 83.12 & 64.30 & 107.06 & RW2 \\ 
  DRC & BANDUNDU & 2012 & 79.47 & 61.33 & 102.29 & RW2 \\ 
  DRC & BANDUNDU & 2013 & 75.89 & 56.80 & 100.60 & RW2 \\ 
  DRC & BANDUNDU & 2014 & 72.51 & 50.09 & 103.28 & RW2 \\ 
  DRC & BANDUNDU & 2015 & 69.25 & 41.44 & 113.88 & RW2 \\ 
  DRC & BANDUNDU & 2016 & 66.10 & 33.77 & 126.39 & RW2 \\ 
  DRC & BANDUNDU & 2017 & 63.15 & 26.61 & 144.30 & RW2 \\ 
  DRC & BANDUNDU & 2018 & 60.11 & 20.23 & 166.09 & RW2 \\ 
  DRC & BANDUNDU & 2019 & 57.29 & 15.47 & 193.01 & RW2 \\ 
  DRC & BAS-CONGO & 1980 & 266.73 & 165.19 & 396.33 & RW2 \\ 
  DRC & BAS-CONGO & 1981 & 262.27 & 174.53 & 370.85 & RW2 \\ 
  DRC & BAS-CONGO & 1982 & 257.97 & 179.85 & 354.67 & RW2 \\ 
  DRC & BAS-CONGO & 1983 & 253.95 & 181.12 & 343.42 & RW2 \\ 
  DRC & BAS-CONGO & 1984 & 249.80 & 179.95 & 333.56 & RW2 \\ 
  DRC & BAS-CONGO & 1985 & 245.46 & 181.48 & 323.60 & RW2 \\ 
  DRC & BAS-CONGO & 1986 & 240.53 & 181.10 & 312.05 & RW2 \\ 
  DRC & BAS-CONGO & 1987 & 235.25 & 180.27 & 301.29 & RW2 \\ 
  DRC & BAS-CONGO & 1988 & 229.65 & 177.42 & 291.94 & RW2 \\ 
  DRC & BAS-CONGO & 1989 & 223.68 & 174.81 & 283.69 & RW2 \\ 
  DRC & BAS-CONGO & 1990 & 217.79 & 171.78 & 271.41 & RW2 \\ 
  DRC & BAS-CONGO & 1991 & 213.28 & 170.68 & 262.98 & RW2 \\ 
  DRC & BAS-CONGO & 1992 & 209.92 & 170.13 & 255.78 & RW2 \\ 
  DRC & BAS-CONGO & 1993 & 207.35 & 168.15 & 252.66 & RW2 \\ 
  DRC & BAS-CONGO & 1994 & 205.64 & 166.51 & 250.40 & RW2 \\ 
  DRC & BAS-CONGO & 1995 & 205.14 & 167.86 & 248.78 & RW2 \\ 
  DRC & BAS-CONGO & 1996 & 203.70 & 169.47 & 244.20 & RW2 \\ 
  DRC & BAS-CONGO & 1997 & 201.61 & 168.44 & 239.18 & RW2 \\ 
  DRC & BAS-CONGO & 1998 & 198.25 & 165.46 & 235.90 & RW2 \\ 
  DRC & BAS-CONGO & 1999 & 194.00 & 160.32 & 232.00 & RW2 \\ 
  DRC & BAS-CONGO & 2000 & 188.48 & 155.80 & 225.70 & RW2 \\ 
  DRC & BAS-CONGO & 2001 & 182.26 & 152.21 & 216.62 & RW2 \\ 
  DRC & BAS-CONGO & 2002 & 175.57 & 147.22 & 208.37 & RW2 \\ 
  DRC & BAS-CONGO & 2003 & 168.49 & 139.98 & 201.13 & RW2 \\ 
  DRC & BAS-CONGO & 2004 & 161.19 & 131.75 & 195.75 & RW2 \\ 
  DRC & BAS-CONGO & 2005 & 153.96 & 124.67 & 188.48 & RW2 \\ 
  DRC & BAS-CONGO & 2006 & 147.22 & 119.17 & 180.12 & RW2 \\ 
  DRC & BAS-CONGO & 2007 & 141.12 & 113.84 & 173.88 & RW2 \\ 
  DRC & BAS-CONGO & 2008 & 135.46 & 107.53 & 169.74 & RW2 \\ 
  DRC & BAS-CONGO & 2009 & 130.59 & 100.74 & 167.31 & RW2 \\ 
  DRC & BAS-CONGO & 2010 & 126.27 & 95.67 & 165.37 & RW2 \\ 
  DRC & BAS-CONGO & 2011 & 122.20 & 91.92 & 160.47 & RW2 \\ 
  DRC & BAS-CONGO & 2012 & 118.30 & 88.49 & 157.13 & RW2 \\ 
  DRC & BAS-CONGO & 2013 & 114.52 & 82.56 & 156.35 & RW2 \\ 
  DRC & BAS-CONGO & 2014 & 110.85 & 74.27 & 162.90 & RW2 \\ 
  DRC & BAS-CONGO & 2015 & 107.40 & 62.19 & 179.35 & RW2 \\ 
  DRC & BAS-CONGO & 2016 & 103.71 & 51.92 & 198.13 & RW2 \\ 
  DRC & BAS-CONGO & 2017 & 100.28 & 41.23 & 224.56 & RW2 \\ 
  DRC & BAS-CONGO & 2018 & 97.13 & 32.47 & 255.93 & RW2 \\ 
  DRC & BAS-CONGO & 2019 & 94.01 & 24.78 & 293.48 & RW2 \\ 
  DRC & EQUATEUR & 1980 & 191.16 & 120.63 & 289.49 & RW2 \\ 
  DRC & EQUATEUR & 1981 & 190.53 & 130.27 & 270.81 & RW2 \\ 
  DRC & EQUATEUR & 1982 & 189.79 & 134.47 & 260.35 & RW2 \\ 
  DRC & EQUATEUR & 1983 & 189.25 & 137.04 & 255.51 & RW2 \\ 
  DRC & EQUATEUR & 1984 & 188.42 & 137.75 & 252.26 & RW2 \\ 
  DRC & EQUATEUR & 1985 & 187.70 & 140.03 & 247.25 & RW2 \\ 
  DRC & EQUATEUR & 1986 & 186.16 & 141.16 & 240.94 & RW2 \\ 
  DRC & EQUATEUR & 1987 & 184.20 & 141.80 & 236.16 & RW2 \\ 
  DRC & EQUATEUR & 1988 & 181.87 & 141.21 & 231.64 & RW2 \\ 
  DRC & EQUATEUR & 1989 & 179.55 & 139.85 & 228.03 & RW2 \\ 
  DRC & EQUATEUR & 1990 & 176.85 & 138.76 & 222.25 & RW2 \\ 
  DRC & EQUATEUR & 1991 & 175.53 & 139.73 & 218.56 & RW2 \\ 
  DRC & EQUATEUR & 1992 & 174.98 & 140.48 & 215.18 & RW2 \\ 
  DRC & EQUATEUR & 1993 & 175.32 & 140.91 & 215.29 & RW2 \\ 
  DRC & EQUATEUR & 1994 & 176.46 & 141.21 & 217.25 & RW2 \\ 
  DRC & EQUATEUR & 1995 & 178.88 & 144.94 & 219.09 & RW2 \\ 
  DRC & EQUATEUR & 1996 & 180.27 & 147.92 & 217.69 & RW2 \\ 
  DRC & EQUATEUR & 1997 & 181.25 & 150.01 & 216.60 & RW2 \\ 
  DRC & EQUATEUR & 1998 & 181.07 & 149.73 & 216.82 & RW2 \\ 
  DRC & EQUATEUR & 1999 & 179.98 & 147.37 & 217.85 & RW2 \\ 
  DRC & EQUATEUR & 2000 & 177.63 & 145.56 & 214.13 & RW2 \\ 
  DRC & EQUATEUR & 2001 & 174.55 & 144.67 & 208.10 & RW2 \\ 
  DRC & EQUATEUR & 2002 & 170.82 & 142.43 & 203.17 & RW2 \\ 
  DRC & EQUATEUR & 2003 & 166.49 & 138.37 & 198.98 & RW2 \\ 
  DRC & EQUATEUR & 2004 & 162.08 & 133.18 & 196.02 & RW2 \\ 
  DRC & EQUATEUR & 2005 & 157.22 & 128.21 & 190.53 & RW2 \\ 
  DRC & EQUATEUR & 2006 & 152.97 & 125.66 & 184.65 & RW2 \\ 
  DRC & EQUATEUR & 2007 & 148.82 & 122.24 & 179.92 & RW2 \\ 
  DRC & EQUATEUR & 2008 & 145.20 & 118.09 & 177.22 & RW2 \\ 
  DRC & EQUATEUR & 2009 & 142.31 & 113.53 & 176.88 & RW2 \\ 
  DRC & EQUATEUR & 2010 & 139.73 & 109.43 & 176.46 & RW2 \\ 
  DRC & EQUATEUR & 2011 & 137.26 & 108.26 & 173.12 & RW2 \\ 
  DRC & EQUATEUR & 2012 & 134.95 & 106.36 & 170.33 & RW2 \\ 
  DRC & EQUATEUR & 2013 & 132.75 & 101.73 & 171.20 & RW2 \\ 
  DRC & EQUATEUR & 2014 & 130.71 & 92.77 & 180.47 & RW2 \\ 
  DRC & EQUATEUR & 2015 & 128.43 & 79.84 & 201.70 & RW2 \\ 
  DRC & EQUATEUR & 2016 & 126.19 & 66.70 & 225.99 & RW2 \\ 
  DRC & EQUATEUR & 2017 & 124.31 & 54.67 & 258.07 & RW2 \\ 
  DRC & EQUATEUR & 2018 & 121.56 & 43.80 & 298.81 & RW2 \\ 
  DRC & EQUATEUR & 2019 & 119.51 & 34.12 & 345.04 & RW2 \\ 
  DRC & KASAI-OCCIDENTAL & 1980 & 265.89 & 173.36 & 384.63 & RW2 \\ 
  DRC & KASAI-OCCIDENTAL & 1981 & 262.14 & 183.67 & 359.06 & RW2 \\ 
  DRC & KASAI-OCCIDENTAL & 1982 & 258.35 & 188.48 & 342.82 & RW2 \\ 
  DRC & KASAI-OCCIDENTAL & 1983 & 254.47 & 189.08 & 333.79 & RW2 \\ 
  DRC & KASAI-OCCIDENTAL & 1984 & 250.31 & 187.66 & 325.59 & RW2 \\ 
  DRC & KASAI-OCCIDENTAL & 1985 & 246.37 & 188.23 & 315.91 & RW2 \\ 
  DRC & KASAI-OCCIDENTAL & 1986 & 241.67 & 187.44 & 304.99 & RW2 \\ 
  DRC & KASAI-OCCIDENTAL & 1987 & 236.12 & 186.04 & 294.80 & RW2 \\ 
  DRC & KASAI-OCCIDENTAL & 1988 & 230.87 & 183.16 & 286.58 & RW2 \\ 
  DRC & KASAI-OCCIDENTAL & 1989 & 225.34 & 179.89 & 279.14 & RW2 \\ 
  DRC & KASAI-OCCIDENTAL & 1990 & 219.48 & 177.09 & 267.97 & RW2 \\ 
  DRC & KASAI-OCCIDENTAL & 1991 & 215.17 & 176.38 & 259.87 & RW2 \\ 
  DRC & KASAI-OCCIDENTAL & 1992 & 212.05 & 175.49 & 253.49 & RW2 \\ 
  DRC & KASAI-OCCIDENTAL & 1993 & 210.10 & 174.24 & 250.43 & RW2 \\ 
  DRC & KASAI-OCCIDENTAL & 1994 & 208.90 & 172.37 & 249.57 & RW2 \\ 
  DRC & KASAI-OCCIDENTAL & 1995 & 209.04 & 174.28 & 248.90 & RW2 \\ 
  DRC & KASAI-OCCIDENTAL & 1996 & 208.15 & 175.81 & 244.48 & RW2 \\ 
  DRC & KASAI-OCCIDENTAL & 1997 & 206.57 & 175.51 & 241.42 & RW2 \\ 
  DRC & KASAI-OCCIDENTAL & 1998 & 203.97 & 173.03 & 239.13 & RW2 \\ 
  DRC & KASAI-OCCIDENTAL & 1999 & 200.23 & 167.57 & 237.39 & RW2 \\ 
  DRC & KASAI-OCCIDENTAL & 2000 & 194.98 & 162.43 & 231.29 & RW2 \\ 
  DRC & KASAI-OCCIDENTAL & 2001 & 189.14 & 158.93 & 223.33 & RW2 \\ 
  DRC & KASAI-OCCIDENTAL & 2002 & 182.36 & 153.70 & 214.96 & RW2 \\ 
  DRC & KASAI-OCCIDENTAL & 2003 & 175.45 & 146.49 & 209.43 & RW2 \\ 
  DRC & KASAI-OCCIDENTAL & 2004 & 167.98 & 138.07 & 203.42 & RW2 \\ 
  DRC & KASAI-OCCIDENTAL & 2005 & 160.42 & 130.86 & 194.78 & RW2 \\ 
  DRC & KASAI-OCCIDENTAL & 2006 & 153.42 & 125.45 & 186.08 & RW2 \\ 
  DRC & KASAI-OCCIDENTAL & 2007 & 147.01 & 119.63 & 179.01 & RW2 \\ 
  DRC & KASAI-OCCIDENTAL & 2008 & 140.99 & 112.98 & 174.40 & RW2 \\ 
  DRC & KASAI-OCCIDENTAL & 2009 & 135.54 & 106.10 & 171.43 & RW2 \\ 
  DRC & KASAI-OCCIDENTAL & 2010 & 130.63 & 100.28 & 169.05 & RW2 \\ 
  DRC & KASAI-OCCIDENTAL & 2011 & 126.15 & 95.87 & 163.13 & RW2 \\ 
  DRC & KASAI-OCCIDENTAL & 2012 & 121.64 & 92.20 & 159.04 & RW2 \\ 
  DRC & KASAI-OCCIDENTAL & 2013 & 117.46 & 86.44 & 157.95 & RW2 \\ 
  DRC & KASAI-OCCIDENTAL & 2014 & 113.24 & 77.12 & 163.20 & RW2 \\ 
  DRC & KASAI-OCCIDENTAL & 2015 & 109.39 & 64.96 & 179.04 & RW2 \\ 
  DRC & KASAI-OCCIDENTAL & 2016 & 105.71 & 53.71 & 198.07 & RW2 \\ 
  DRC & KASAI-OCCIDENTAL & 2017 & 101.71 & 43.00 & 221.78 & RW2 \\ 
  DRC & KASAI-OCCIDENTAL & 2018 & 97.78 & 33.88 & 252.61 & RW2 \\ 
  DRC & KASAI-OCCIDENTAL & 2019 & 94.45 & 25.96 & 294.14 & RW2 \\ 
  DRC & KASAI-ORIENTAL & 1980 & 208.34 & 135.63 & 306.19 & RW2 \\ 
  DRC & KASAI-ORIENTAL & 1981 & 206.56 & 145.87 & 284.73 & RW2 \\ 
  DRC & KASAI-ORIENTAL & 1982 & 204.80 & 150.55 & 272.72 & RW2 \\ 
  DRC & KASAI-ORIENTAL & 1983 & 202.71 & 152.13 & 265.39 & RW2 \\ 
  DRC & KASAI-ORIENTAL & 1984 & 200.63 & 152.05 & 260.56 & RW2 \\ 
  DRC & KASAI-ORIENTAL & 1985 & 198.80 & 153.38 & 253.10 & RW2 \\ 
  DRC & KASAI-ORIENTAL & 1986 & 195.84 & 154.33 & 245.41 & RW2 \\ 
  DRC & KASAI-ORIENTAL & 1987 & 192.59 & 153.82 & 238.72 & RW2 \\ 
  DRC & KASAI-ORIENTAL & 1988 & 189.20 & 152.07 & 233.51 & RW2 \\ 
  DRC & KASAI-ORIENTAL & 1989 & 185.69 & 149.57 & 228.67 & RW2 \\ 
  DRC & KASAI-ORIENTAL & 1990 & 181.73 & 148.03 & 221.44 & RW2 \\ 
  DRC & KASAI-ORIENTAL & 1991 & 179.36 & 147.66 & 215.78 & RW2 \\ 
  DRC & KASAI-ORIENTAL & 1992 & 177.77 & 147.82 & 211.84 & RW2 \\ 
  DRC & KASAI-ORIENTAL & 1993 & 177.26 & 147.41 & 211.12 & RW2 \\ 
  DRC & KASAI-ORIENTAL & 1994 & 177.44 & 146.55 & 211.95 & RW2 \\ 
  DRC & KASAI-ORIENTAL & 1995 & 178.76 & 149.75 & 212.86 & RW2 \\ 
  DRC & KASAI-ORIENTAL & 1996 & 179.22 & 151.93 & 210.63 & RW2 \\ 
  DRC & KASAI-ORIENTAL & 1997 & 179.15 & 152.86 & 208.88 & RW2 \\ 
  DRC & KASAI-ORIENTAL & 1998 & 178.12 & 150.73 & 208.65 & RW2 \\ 
  DRC & KASAI-ORIENTAL & 1999 & 175.89 & 147.31 & 208.07 & RW2 \\ 
  DRC & KASAI-ORIENTAL & 2000 & 172.54 & 144.42 & 204.26 & RW2 \\ 
  DRC & KASAI-ORIENTAL & 2001 & 168.39 & 141.82 & 198.32 & RW2 \\ 
  DRC & KASAI-ORIENTAL & 2002 & 163.65 & 138.58 & 192.44 & RW2 \\ 
  DRC & KASAI-ORIENTAL & 2003 & 158.19 & 132.60 & 187.66 & RW2 \\ 
  DRC & KASAI-ORIENTAL & 2004 & 152.57 & 126.32 & 183.59 & RW2 \\ 
  DRC & KASAI-ORIENTAL & 2005 & 146.57 & 120.27 & 177.11 & RW2 \\ 
  DRC & KASAI-ORIENTAL & 2006 & 141.14 & 116.57 & 169.89 & RW2 \\ 
  DRC & KASAI-ORIENTAL & 2007 & 136.03 & 112.14 & 164.08 & RW2 \\ 
  DRC & KASAI-ORIENTAL & 2008 & 131.41 & 106.82 & 160.47 & RW2 \\ 
  DRC & KASAI-ORIENTAL & 2009 & 127.22 & 101.19 & 158.48 & RW2 \\ 
  DRC & KASAI-ORIENTAL & 2010 & 123.48 & 96.63 & 156.89 & RW2 \\ 
  DRC & KASAI-ORIENTAL & 2011 & 120.04 & 93.90 & 152.32 & RW2 \\ 
  DRC & KASAI-ORIENTAL & 2012 & 116.63 & 91.48 & 147.88 & RW2 \\ 
  DRC & KASAI-ORIENTAL & 2013 & 113.40 & 86.57 & 147.73 & RW2 \\ 
  DRC & KASAI-ORIENTAL & 2014 & 110.21 & 78.06 & 153.59 & RW2 \\ 
  DRC & KASAI-ORIENTAL & 2015 & 107.17 & 65.73 & 170.16 & RW2 \\ 
  DRC & KASAI-ORIENTAL & 2016 & 104.02 & 54.27 & 190.29 & RW2 \\ 
  DRC & KASAI-ORIENTAL & 2017 & 101.15 & 43.75 & 219.82 & RW2 \\ 
  DRC & KASAI-ORIENTAL & 2018 & 98.33 & 34.45 & 251.56 & RW2 \\ 
  DRC & KASAI-ORIENTAL & 2019 & 95.69 & 26.96 & 289.49 & RW2 \\ 
  DRC & KATANGA & 1980 & 212.80 & 137.43 & 312.52 & RW2 \\ 
  DRC & KATANGA & 1981 & 211.19 & 147.52 & 293.02 & RW2 \\ 
  DRC & KATANGA & 1982 & 210.20 & 152.45 & 281.12 & RW2 \\ 
  DRC & KATANGA & 1983 & 208.72 & 154.82 & 275.53 & RW2 \\ 
  DRC & KATANGA & 1984 & 207.02 & 154.86 & 270.48 & RW2 \\ 
  DRC & KATANGA & 1985 & 205.53 & 156.86 & 264.42 & RW2 \\ 
  DRC & KATANGA & 1986 & 203.19 & 158.18 & 257.58 & RW2 \\ 
  DRC & KATANGA & 1987 & 200.59 & 158.09 & 250.88 & RW2 \\ 
  DRC & KATANGA & 1988 & 197.72 & 157.40 & 246.06 & RW2 \\ 
  DRC & KATANGA & 1989 & 194.67 & 155.41 & 241.93 & RW2 \\ 
  DRC & KATANGA & 1990 & 191.08 & 154.10 & 233.98 & RW2 \\ 
  DRC & KATANGA & 1991 & 189.19 & 155.23 & 228.53 & RW2 \\ 
  DRC & KATANGA & 1992 & 188.13 & 155.28 & 225.32 & RW2 \\ 
  DRC & KATANGA & 1993 & 188.16 & 155.69 & 225.17 & RW2 \\ 
  DRC & KATANGA & 1994 & 188.75 & 154.90 & 226.62 & RW2 \\ 
  DRC & KATANGA & 1995 & 190.83 & 158.23 & 228.67 & RW2 \\ 
  DRC & KATANGA & 1996 & 191.64 & 161.18 & 226.73 & RW2 \\ 
  DRC & KATANGA & 1997 & 191.89 & 162.17 & 225.29 & RW2 \\ 
  DRC & KATANGA & 1998 & 191.03 & 160.97 & 225.36 & RW2 \\ 
  DRC & KATANGA & 1999 & 188.96 & 157.94 & 224.47 & RW2 \\ 
  DRC & KATANGA & 2000 & 185.60 & 154.92 & 220.45 & RW2 \\ 
  DRC & KATANGA & 2001 & 181.37 & 152.97 & 213.52 & RW2 \\ 
  DRC & KATANGA & 2002 & 176.22 & 149.86 & 206.69 & RW2 \\ 
  DRC & KATANGA & 2003 & 170.58 & 143.64 & 201.75 & RW2 \\ 
  DRC & KATANGA & 2004 & 164.65 & 137.52 & 196.78 & RW2 \\ 
  DRC & KATANGA & 2005 & 158.27 & 131.16 & 189.50 & RW2 \\ 
  DRC & KATANGA & 2006 & 152.54 & 127.70 & 181.39 & RW2 \\ 
  DRC & KATANGA & 2007 & 147.14 & 123.63 & 174.42 & RW2 \\ 
  DRC & KATANGA & 2008 & 142.17 & 118.22 & 169.87 & RW2 \\ 
  DRC & KATANGA & 2009 & 137.77 & 112.56 & 167.33 & RW2 \\ 
  DRC & KATANGA & 2010 & 133.80 & 108.18 & 165.02 & RW2 \\ 
  DRC & KATANGA & 2011 & 130.11 & 106.37 & 158.09 & RW2 \\ 
  DRC & KATANGA & 2012 & 126.55 & 104.81 & 151.93 & RW2 \\ 
  DRC & KATANGA & 2013 & 123.20 & 99.81 & 150.72 & RW2 \\ 
  DRC & KATANGA & 2014 & 119.82 & 89.45 & 158.71 & RW2 \\ 
  DRC & KATANGA & 2015 & 116.40 & 74.58 & 177.17 & RW2 \\ 
  DRC & KATANGA & 2016 & 113.14 & 61.55 & 198.43 & RW2 \\ 
  DRC & KATANGA & 2017 & 109.87 & 49.73 & 226.42 & RW2 \\ 
  DRC & KATANGA & 2018 & 106.70 & 38.76 & 262.21 & RW2 \\ 
  DRC & KATANGA & 2019 & 103.84 & 29.86 & 304.82 & RW2 \\ 
  DRC & KINSHASA & 1980 & 132.20 & 81.86 & 209.23 & RW2 \\ 
  DRC & KINSHASA & 1981 & 130.46 & 87.95 & 191.93 & RW2 \\ 
  DRC & KINSHASA & 1982 & 129.08 & 90.84 & 182.26 & RW2 \\ 
  DRC & KINSHASA & 1983 & 127.22 & 91.43 & 176.23 & RW2 \\ 
  DRC & KINSHASA & 1984 & 125.51 & 91.57 & 171.06 & RW2 \\ 
  DRC & KINSHASA & 1985 & 123.85 & 92.04 & 165.74 & RW2 \\ 
  DRC & KINSHASA & 1986 & 121.70 & 92.18 & 159.66 & RW2 \\ 
  DRC & KINSHASA & 1987 & 119.51 & 92.35 & 154.62 & RW2 \\ 
  DRC & KINSHASA & 1988 & 117.07 & 91.10 & 150.26 & RW2 \\ 
  DRC & KINSHASA & 1989 & 114.69 & 89.77 & 146.20 & RW2 \\ 
  DRC & KINSHASA & 1990 & 112.17 & 89.04 & 140.16 & RW2 \\ 
  DRC & KINSHASA & 1991 & 110.60 & 89.30 & 136.29 & RW2 \\ 
  DRC & KINSHASA & 1992 & 109.64 & 89.64 & 133.54 & RW2 \\ 
  DRC & KINSHASA & 1993 & 109.40 & 89.03 & 133.34 & RW2 \\ 
  DRC & KINSHASA & 1994 & 109.72 & 88.90 & 133.84 & RW2 \\ 
  DRC & KINSHASA & 1995 & 110.84 & 90.66 & 135.17 & RW2 \\ 
  DRC & KINSHASA & 1996 & 111.50 & 92.54 & 134.28 & RW2 \\ 
  DRC & KINSHASA & 1997 & 111.76 & 93.32 & 133.45 & RW2 \\ 
  DRC & KINSHASA & 1998 & 111.60 & 92.64 & 133.62 & RW2 \\ 
  DRC & KINSHASA & 1999 & 110.77 & 91.02 & 134.08 & RW2 \\ 
  DRC & KINSHASA & 2000 & 109.08 & 89.64 & 131.84 & RW2 \\ 
  DRC & KINSHASA & 2001 & 107.05 & 88.83 & 128.13 & RW2 \\ 
  DRC & KINSHASA & 2002 & 104.65 & 87.41 & 124.91 & RW2 \\ 
  DRC & KINSHASA & 2003 & 101.86 & 84.31 & 122.33 & RW2 \\ 
  DRC & KINSHASA & 2004 & 99.03 & 80.88 & 120.83 & RW2 \\ 
  DRC & KINSHASA & 2005 & 96.03 & 77.78 & 117.73 & RW2 \\ 
  DRC & KINSHASA & 2006 & 93.52 & 76.16 & 113.95 & RW2 \\ 
  DRC & KINSHASA & 2007 & 91.12 & 74.42 & 111.26 & RW2 \\ 
  DRC & KINSHASA & 2008 & 89.23 & 71.84 & 110.15 & RW2 \\ 
  DRC & KINSHASA & 2009 & 87.53 & 69.04 & 110.43 & RW2 \\ 
  DRC & KINSHASA & 2010 & 86.40 & 67.01 & 111.05 & RW2 \\ 
  DRC & KINSHASA & 2011 & 85.16 & 66.35 & 108.73 & RW2 \\ 
  DRC & KINSHASA & 2012 & 84.08 & 65.80 & 107.79 & RW2 \\ 
  DRC & KINSHASA & 2013 & 83.10 & 63.02 & 109.45 & RW2 \\ 
  DRC & KINSHASA & 2014 & 82.11 & 57.33 & 116.77 & RW2 \\ 
  DRC & KINSHASA & 2015 & 81.11 & 49.03 & 131.86 & RW2 \\ 
  DRC & KINSHASA & 2016 & 80.14 & 40.94 & 151.55 & RW2 \\ 
  DRC & KINSHASA & 2017 & 79.12 & 33.62 & 177.09 & RW2 \\ 
  DRC & KINSHASA & 2018 & 77.92 & 27.19 & 207.67 & RW2 \\ 
  DRC & KINSHASA & 2019 & 77.11 & 21.18 & 247.78 & RW2 \\ 
  DRC & MANIEMA & 1980 & 186.41 & 115.09 & 285.23 & RW2 \\ 
  DRC & MANIEMA & 1981 & 186.92 & 125.07 & 267.21 & RW2 \\ 
  DRC & MANIEMA & 1982 & 186.94 & 130.17 & 258.21 & RW2 \\ 
  DRC & MANIEMA & 1983 & 187.00 & 132.77 & 254.43 & RW2 \\ 
  DRC & MANIEMA & 1984 & 186.87 & 134.51 & 251.71 & RW2 \\ 
  DRC & MANIEMA & 1985 & 186.92 & 137.79 & 247.86 & RW2 \\ 
  DRC & MANIEMA & 1986 & 186.17 & 139.79 & 242.01 & RW2 \\ 
  DRC & MANIEMA & 1987 & 184.91 & 141.57 & 237.72 & RW2 \\ 
  DRC & MANIEMA & 1988 & 183.33 & 142.01 & 233.25 & RW2 \\ 
  DRC & MANIEMA & 1989 & 181.70 & 142.64 & 229.55 & RW2 \\ 
  DRC & MANIEMA & 1990 & 179.91 & 142.60 & 223.72 & RW2 \\ 
  DRC & MANIEMA & 1991 & 179.38 & 144.79 & 219.53 & RW2 \\ 
  DRC & MANIEMA & 1992 & 179.69 & 146.68 & 218.00 & RW2 \\ 
  DRC & MANIEMA & 1993 & 180.94 & 148.25 & 218.73 & RW2 \\ 
  DRC & MANIEMA & 1994 & 182.97 & 149.79 & 221.25 & RW2 \\ 
  DRC & MANIEMA & 1995 & 186.29 & 153.47 & 224.36 & RW2 \\ 
  DRC & MANIEMA & 1996 & 188.33 & 157.77 & 224.37 & RW2 \\ 
  DRC & MANIEMA & 1997 & 189.78 & 159.63 & 224.05 & RW2 \\ 
  DRC & MANIEMA & 1998 & 189.83 & 158.92 & 225.06 & RW2 \\ 
  DRC & MANIEMA & 1999 & 188.75 & 156.82 & 226.22 & RW2 \\ 
  DRC & MANIEMA & 2000 & 186.09 & 153.47 & 222.98 & RW2 \\ 
  DRC & MANIEMA & 2001 & 182.54 & 152.08 & 217.79 & RW2 \\ 
  DRC & MANIEMA & 2002 & 177.93 & 149.05 & 211.81 & RW2 \\ 
  DRC & MANIEMA & 2003 & 172.67 & 143.83 & 206.84 & RW2 \\ 
  DRC & MANIEMA & 2004 & 167.01 & 137.19 & 202.27 & RW2 \\ 
  DRC & MANIEMA & 2005 & 160.85 & 131.61 & 195.07 & RW2 \\ 
  DRC & MANIEMA & 2006 & 155.22 & 127.77 & 187.22 & RW2 \\ 
  DRC & MANIEMA & 2007 & 149.82 & 123.44 & 180.96 & RW2 \\ 
  DRC & MANIEMA & 2008 & 144.86 & 117.97 & 176.74 & RW2 \\ 
  DRC & MANIEMA & 2009 & 140.25 & 111.65 & 174.88 & RW2 \\ 
  DRC & MANIEMA & 2010 & 136.38 & 106.39 & 173.19 & RW2 \\ 
  DRC & MANIEMA & 2011 & 132.53 & 103.44 & 168.17 & RW2 \\ 
  DRC & MANIEMA & 2012 & 128.81 & 100.18 & 163.80 & RW2 \\ 
  DRC & MANIEMA & 2013 & 125.21 & 94.79 & 163.07 & RW2 \\ 
  DRC & MANIEMA & 2014 & 121.68 & 85.54 & 169.82 & RW2 \\ 
  DRC & MANIEMA & 2015 & 118.24 & 71.84 & 187.32 & RW2 \\ 
  DRC & MANIEMA & 2016 & 115.07 & 60.06 & 208.76 & RW2 \\ 
  DRC & MANIEMA & 2017 & 111.45 & 48.39 & 238.56 & RW2 \\ 
  DRC & MANIEMA & 2018 & 108.36 & 38.29 & 272.25 & RW2 \\ 
  DRC & MANIEMA & 2019 & 105.34 & 29.69 & 317.68 & RW2 \\ 
  DRC & NORD-KIVU & 1980 & 286.79 & 184.32 & 416.21 & RW2 \\ 
  DRC & NORD-KIVU & 1981 & 276.74 & 190.46 & 384.37 & RW2 \\ 
  DRC & NORD-KIVU & 1982 & 266.95 & 189.16 & 360.61 & RW2 \\ 
  DRC & NORD-KIVU & 1983 & 256.99 & 184.74 & 344.46 & RW2 \\ 
  DRC & NORD-KIVU & 1984 & 247.25 & 179.46 & 330.43 & RW2 \\ 
  DRC & NORD-KIVU & 1985 & 237.58 & 174.77 & 314.27 & RW2 \\ 
  DRC & NORD-KIVU & 1986 & 227.53 & 170.09 & 298.26 & RW2 \\ 
  DRC & NORD-KIVU & 1987 & 217.45 & 164.59 & 282.50 & RW2 \\ 
  DRC & NORD-KIVU & 1988 & 207.26 & 157.95 & 267.36 & RW2 \\ 
  DRC & NORD-KIVU & 1989 & 197.37 & 151.69 & 254.29 & RW2 \\ 
  DRC & NORD-KIVU & 1990 & 187.55 & 145.29 & 237.77 & RW2 \\ 
  DRC & NORD-KIVU & 1991 & 179.64 & 141.42 & 225.29 & RW2 \\ 
  DRC & NORD-KIVU & 1992 & 172.52 & 137.32 & 213.75 & RW2 \\ 
  DRC & NORD-KIVU & 1993 & 166.76 & 133.05 & 206.61 & RW2 \\ 
  DRC & NORD-KIVU & 1994 & 161.96 & 128.84 & 200.70 & RW2 \\ 
  DRC & NORD-KIVU & 1995 & 158.05 & 126.77 & 195.62 & RW2 \\ 
  DRC & NORD-KIVU & 1996 & 153.59 & 124.76 & 187.53 & RW2 \\ 
  DRC & NORD-KIVU & 1997 & 148.31 & 121.14 & 180.33 & RW2 \\ 
  DRC & NORD-KIVU & 1998 & 142.38 & 115.82 & 173.71 & RW2 \\ 
  DRC & NORD-KIVU & 1999 & 135.92 & 109.61 & 167.65 & RW2 \\ 
  DRC & NORD-KIVU & 2000 & 128.48 & 103.29 & 158.49 & RW2 \\ 
  DRC & NORD-KIVU & 2001 & 120.72 & 98.03 & 148.02 & RW2 \\ 
  DRC & NORD-KIVU & 2002 & 112.81 & 91.46 & 138.30 & RW2 \\ 
  DRC & NORD-KIVU & 2003 & 104.80 & 84.44 & 129.86 & RW2 \\ 
  DRC & NORD-KIVU & 2004 & 97.07 & 77.16 & 121.61 & RW2 \\ 
  DRC & NORD-KIVU & 2005 & 89.40 & 70.67 & 112.41 & RW2 \\ 
  DRC & NORD-KIVU & 2006 & 82.60 & 65.52 & 103.74 & RW2 \\ 
  DRC & NORD-KIVU & 2007 & 76.20 & 60.22 & 95.70 & RW2 \\ 
  DRC & NORD-KIVU & 2008 & 70.41 & 54.72 & 90.27 & RW2 \\ 
  DRC & NORD-KIVU & 2009 & 65.31 & 49.40 & 85.79 & RW2 \\ 
  DRC & NORD-KIVU & 2010 & 60.71 & 44.82 & 81.92 & RW2 \\ 
  DRC & NORD-KIVU & 2011 & 56.43 & 41.23 & 76.62 & RW2 \\ 
  DRC & NORD-KIVU & 2012 & 52.41 & 37.82 & 72.07 & RW2 \\ 
  DRC & NORD-KIVU & 2013 & 48.74 & 33.93 & 69.27 & RW2 \\ 
  DRC & NORD-KIVU & 2014 & 45.30 & 29.06 & 69.34 & RW2 \\ 
  DRC & NORD-KIVU & 2015 & 42.05 & 23.47 & 74.01 & RW2 \\ 
  DRC & NORD-KIVU & 2016 & 38.97 & 18.64 & 80.14 & RW2 \\ 
  DRC & NORD-KIVU & 2017 & 36.25 & 14.24 & 88.97 & RW2 \\ 
  DRC & NORD-KIVU & 2018 & 33.55 & 10.67 & 100.08 & RW2 \\ 
  DRC & NORD-KIVU & 2019 & 31.08 & 7.92 & 115.44 & RW2 \\ 
  DRC & ORIENTALE & 1980 & 233.57 & 146.13 & 350.85 & RW2 \\ 
  DRC & ORIENTALE & 1981 & 230.35 & 154.12 & 328.51 & RW2 \\ 
  DRC & ORIENTALE & 1982 & 227.23 & 159.14 & 314.11 & RW2 \\ 
  DRC & ORIENTALE & 1983 & 224.01 & 159.80 & 304.03 & RW2 \\ 
  DRC & ORIENTALE & 1984 & 220.34 & 159.16 & 296.76 & RW2 \\ 
  DRC & ORIENTALE & 1985 & 217.16 & 160.26 & 287.50 & RW2 \\ 
  DRC & ORIENTALE & 1986 & 213.23 & 160.55 & 277.11 & RW2 \\ 
  DRC & ORIENTALE & 1987 & 208.88 & 159.44 & 268.01 & RW2 \\ 
  DRC & ORIENTALE & 1988 & 204.15 & 157.62 & 260.14 & RW2 \\ 
  DRC & ORIENTALE & 1989 & 199.38 & 155.03 & 253.60 & RW2 \\ 
  DRC & ORIENTALE & 1990 & 194.24 & 152.63 & 243.81 & RW2 \\ 
  DRC & ORIENTALE & 1991 & 190.70 & 152.17 & 236.06 & RW2 \\ 
  DRC & ORIENTALE & 1992 & 188.40 & 151.58 & 230.88 & RW2 \\ 
  DRC & ORIENTALE & 1993 & 186.76 & 150.65 & 228.87 & RW2 \\ 
  DRC & ORIENTALE & 1994 & 186.26 & 150.32 & 227.81 & RW2 \\ 
  DRC & ORIENTALE & 1995 & 186.70 & 152.51 & 227.61 & RW2 \\ 
  DRC & ORIENTALE & 1996 & 186.32 & 153.94 & 223.97 & RW2 \\ 
  DRC & ORIENTALE & 1997 & 185.17 & 154.25 & 220.35 & RW2 \\ 
  DRC & ORIENTALE & 1998 & 182.94 & 151.85 & 218.62 & RW2 \\ 
  DRC & ORIENTALE & 1999 & 179.63 & 148.31 & 216.01 & RW2 \\ 
  DRC & ORIENTALE & 2000 & 175.12 & 143.92 & 210.58 & RW2 \\ 
  DRC & ORIENTALE & 2001 & 169.78 & 141.37 & 202.65 & RW2 \\ 
  DRC & ORIENTALE & 2002 & 163.84 & 137.02 & 194.83 & RW2 \\ 
  DRC & ORIENTALE & 2003 & 157.28 & 130.69 & 188.52 & RW2 \\ 
  DRC & ORIENTALE & 2004 & 150.63 & 123.42 & 183.20 & RW2 \\ 
  DRC & ORIENTALE & 2005 & 143.74 & 117.04 & 175.14 & RW2 \\ 
  DRC & ORIENTALE & 2006 & 137.38 & 112.81 & 166.74 & RW2 \\ 
  DRC & ORIENTALE & 2007 & 131.37 & 107.84 & 158.94 & RW2 \\ 
  DRC & ORIENTALE & 2008 & 125.93 & 102.10 & 154.60 & RW2 \\ 
  DRC & ORIENTALE & 2009 & 120.94 & 96.11 & 151.25 & RW2 \\ 
  DRC & ORIENTALE & 2010 & 116.60 & 90.96 & 149.12 & RW2 \\ 
  DRC & ORIENTALE & 2011 & 112.27 & 87.63 & 143.05 & RW2 \\ 
  DRC & ORIENTALE & 2012 & 108.30 & 84.41 & 137.90 & RW2 \\ 
  DRC & ORIENTALE & 2013 & 104.47 & 79.07 & 136.69 & RW2 \\ 
  DRC & ORIENTALE & 2014 & 100.76 & 70.60 & 141.83 & RW2 \\ 
  DRC & ORIENTALE & 2015 & 96.92 & 58.67 & 155.34 & RW2 \\ 
  DRC & ORIENTALE & 2016 & 93.38 & 48.42 & 172.91 & RW2 \\ 
  DRC & ORIENTALE & 2017 & 90.07 & 38.91 & 196.18 & RW2 \\ 
  DRC & ORIENTALE & 2018 & 86.79 & 30.68 & 226.37 & RW2 \\ 
  DRC & ORIENTALE & 2019 & 83.53 & 22.97 & 264.12 & RW2 \\ 
  DRC & SUD-KIVU & 1980 & 274.56 & 180.72 & 393.76 & RW2 \\ 
  DRC & SUD-KIVU & 1981 & 271.44 & 190.94 & 369.89 & RW2 \\ 
  DRC & SUD-KIVU & 1982 & 267.86 & 195.26 & 355.80 & RW2 \\ 
  DRC & SUD-KIVU & 1983 & 264.32 & 194.80 & 346.96 & RW2 \\ 
  DRC & SUD-KIVU & 1984 & 260.58 & 192.88 & 341.17 & RW2 \\ 
  DRC & SUD-KIVU & 1985 & 256.99 & 193.15 & 333.00 & RW2 \\ 
  DRC & SUD-KIVU & 1986 & 252.39 & 193.62 & 323.42 & RW2 \\ 
  DRC & SUD-KIVU & 1987 & 247.39 & 191.69 & 313.91 & RW2 \\ 
  DRC & SUD-KIVU & 1988 & 242.43 & 188.56 & 306.39 & RW2 \\ 
  DRC & SUD-KIVU & 1989 & 237.41 & 186.46 & 298.70 & RW2 \\ 
  DRC & SUD-KIVU & 1990 & 231.88 & 184.09 & 287.44 & RW2 \\ 
  DRC & SUD-KIVU & 1991 & 228.01 & 183.61 & 279.00 & RW2 \\ 
  DRC & SUD-KIVU & 1992 & 225.25 & 183.37 & 273.44 & RW2 \\ 
  DRC & SUD-KIVU & 1993 & 223.97 & 182.36 & 270.99 & RW2 \\ 
  DRC & SUD-KIVU & 1994 & 223.56 & 181.82 & 270.94 & RW2 \\ 
  DRC & SUD-KIVU & 1995 & 224.07 & 183.72 & 270.73 & RW2 \\ 
  DRC & SUD-KIVU & 1996 & 223.70 & 186.15 & 266.68 & RW2 \\ 
  DRC & SUD-KIVU & 1997 & 222.14 & 185.86 & 263.84 & RW2 \\ 
  DRC & SUD-KIVU & 1998 & 219.68 & 183.18 & 261.36 & RW2 \\ 
  DRC & SUD-KIVU & 1999 & 215.79 & 178.54 & 258.75 & RW2 \\ 
  DRC & SUD-KIVU & 2000 & 210.14 & 173.59 & 252.00 & RW2 \\ 
  DRC & SUD-KIVU & 2001 & 203.83 & 169.89 & 242.85 & RW2 \\ 
  DRC & SUD-KIVU & 2002 & 196.55 & 164.17 & 233.65 & RW2 \\ 
  DRC & SUD-KIVU & 2003 & 188.64 & 156.45 & 225.83 & RW2 \\ 
  DRC & SUD-KIVU & 2004 & 180.64 & 148.42 & 218.57 & RW2 \\ 
  DRC & SUD-KIVU & 2005 & 172.19 & 140.06 & 210.05 & RW2 \\ 
  DRC & SUD-KIVU & 2006 & 164.44 & 134.64 & 199.26 & RW2 \\ 
  DRC & SUD-KIVU & 2007 & 157.09 & 128.70 & 190.57 & RW2 \\ 
  DRC & SUD-KIVU & 2008 & 150.37 & 121.29 & 184.88 & RW2 \\ 
  DRC & SUD-KIVU & 2009 & 144.20 & 114.17 & 180.64 & RW2 \\ 
  DRC & SUD-KIVU & 2010 & 138.65 & 107.85 & 177.48 & RW2 \\ 
  DRC & SUD-KIVU & 2011 & 133.45 & 103.47 & 170.66 & RW2 \\ 
  DRC & SUD-KIVU & 2012 & 128.46 & 99.50 & 164.15 & RW2 \\ 
  DRC & SUD-KIVU & 2013 & 123.61 & 93.07 & 161.85 & RW2 \\ 
  DRC & SUD-KIVU & 2014 & 118.88 & 82.91 & 167.20 & RW2 \\ 
  DRC & SUD-KIVU & 2015 & 114.27 & 69.44 & 182.76 & RW2 \\ 
  DRC & SUD-KIVU & 2016 & 110.13 & 56.48 & 201.76 & RW2 \\ 
  DRC & SUD-KIVU & 2017 & 105.44 & 45.43 & 227.39 & RW2 \\ 
  DRC & SUD-KIVU & 2018 & 101.28 & 35.51 & 258.09 & RW2 \\ 
  DRC & SUD-KIVU & 2019 & 97.68 & 26.87 & 296.36 & RW2 \\ 
  Egypt & ALL & 1980 & 160.16 & 152.80 & 169.05 & IHME \\ 
  Egypt & ALL & 1980 & 171.28 & 126.90 & 225.78 & RW2 \\ 
  Egypt & ALL & 1980 & 167.60 & 160.00 & 175.50 & UN \\ 
  Egypt & ALL & 1981 & 151.88 & 144.32 & 159.82 & IHME \\ 
  Egypt & ALL & 1981 & 159.01 & 128.76 & 193.51 & RW2 \\ 
  Egypt & ALL & 1981 & 158.10 & 150.90 & 165.60 & UN \\ 
  Egypt & ALL & 1982 & 142.96 & 135.26 & 150.54 & IHME \\ 
  Egypt & ALL & 1982 & 147.46 & 121.47 & 177.97 & RW2 \\ 
  Egypt & ALL & 1982 & 148.40 & 141.70 & 155.30 & UN \\ 
  Egypt & ALL & 1983 & 133.18 & 126.87 & 140.14 & IHME \\ 
  Egypt & ALL & 1983 & 136.57 & 109.54 & 169.61 & RW2 \\ 
  Egypt & ALL & 1983 & 138.50 & 132.20 & 144.90 & UN \\ 
  Egypt & ALL & 1984 & 123.58 & 117.31 & 129.78 & IHME \\ 
  Egypt & ALL & 1984 & 126.83 & 99.05 & 161.62 & RW2 \\ 
  Egypt & ALL & 1984 & 128.50 & 122.80 & 134.30 & UN \\ 
  Egypt & ALL & 1985 & 113.82 & 108.22 & 119.91 & IHME \\ 
  Egypt & ALL & 1985 & 117.43 & 93.52 & 146.54 & RW2 \\ 
  Egypt & ALL & 1985 & 118.80 & 113.60 & 124.20 & UN \\ 
  Egypt & ALL & 1986 & 104.22 & 98.87 & 110.25 & IHME \\ 
  Egypt & ALL & 1986 & 109.54 & 88.90 & 134.37 & RW2 \\ 
  Egypt & ALL & 1986 & 110.10 & 105.10 & 115.20 & UN \\ 
  Egypt & ALL & 1987 & 96.12 & 91.06 & 101.40 & IHME \\ 
  Egypt & ALL & 1987 & 102.67 & 84.22 & 125.18 & RW2 \\ 
  Egypt & ALL & 1987 & 102.50 & 97.70 & 107.30 & UN \\ 
  Egypt & ALL & 1988 & 89.59 & 84.80 & 94.71 & IHME \\ 
  Egypt & ALL & 1988 & 96.63 & 78.05 & 118.86 & RW2 \\ 
  Egypt & ALL & 1988 & 96.00 & 91.50 & 100.60 & UN \\ 
  Egypt & ALL & 1989 & 85.69 & 80.95 & 90.53 & IHME \\ 
  Egypt & ALL & 1989 & 91.31 & 72.50 & 113.95 & RW2 \\ 
  Egypt & ALL & 1989 & 90.60 & 86.50 & 94.90 & UN \\ 
  Egypt & ALL & 1990 & 82.12 & 77.46 & 86.68 & IHME \\ 
  Egypt & ALL & 1990 & 86.80 & 69.30 & 109.01 & RW2 \\ 
  Egypt & ALL & 1990 & 85.90 & 82.10 & 90.00 & UN \\ 
  Egypt & ALL & 1991 & 79.13 & 74.55 & 83.71 & IHME \\ 
  Egypt & ALL & 1991 & 82.26 & 66.49 & 101.31 & RW2 \\ 
  Egypt & ALL & 1991 & 81.70 & 78.00 & 85.60 & UN \\ 
  Egypt & ALL & 1992 & 75.30 & 70.68 & 79.69 & IHME \\ 
  Egypt & ALL & 1992 & 77.84 & 63.16 & 95.25 & RW2 \\ 
  Egypt & ALL & 1992 & 77.50 & 73.90 & 81.40 & UN \\ 
  Egypt & ALL & 1993 & 70.35 & 66.19 & 74.81 & IHME \\ 
  Egypt & ALL & 1993 & 73.49 & 59.17 & 90.70 & RW2 \\ 
  Egypt & ALL & 1993 & 73.30 & 69.70 & 77.10 & UN \\ 
  Egypt & ALL & 1994 & 65.89 & 61.57 & 70.40 & IHME \\ 
  Egypt & ALL & 1994 & 69.12 & 54.75 & 87.11 & RW2 \\ 
  Egypt & ALL & 1994 & 68.90 & 65.50 & 72.70 & UN \\ 
  Egypt & ALL & 1995 & 61.06 & 57.05 & 65.29 & IHME \\ 
  Egypt & ALL & 1995 & 64.81 & 51.22 & 81.33 & RW2 \\ 
  Egypt & ALL & 1995 & 64.60 & 61.40 & 68.20 & UN \\ 
  Egypt & ALL & 1996 & 55.24 & 51.48 & 59.02 & IHME \\ 
  Egypt & ALL & 1996 & 60.67 & 48.62 & 75.53 & RW2 \\ 
  Egypt & ALL & 1996 & 60.40 & 57.40 & 63.80 & UN \\ 
  Egypt & ALL & 1997 & 50.84 & 47.39 & 54.82 & IHME \\ 
  Egypt & ALL & 1997 & 56.74 & 45.88 & 70.02 & RW2 \\ 
  Egypt & ALL & 1997 & 56.50 & 53.50 & 59.70 & UN \\ 
  Egypt & ALL & 1998 & 48.54 & 45.14 & 52.24 & IHME \\ 
  Egypt & ALL & 1998 & 53.08 & 42.61 & 66.49 & RW2 \\ 
  Egypt & ALL & 1998 & 52.80 & 49.90 & 56.00 & UN \\ 
  Egypt & ALL & 1999 & 46.51 & 43.20 & 50.05 & IHME \\ 
  Egypt & ALL & 1999 & 49.69 & 39.21 & 62.88 & RW2 \\ 
  Egypt & ALL & 1999 & 49.50 & 46.60 & 52.60 & UN \\ 
  Egypt & ALL & 2000 & 43.25 & 39.93 & 46.83 & IHME \\ 
  Egypt & ALL & 2000 & 46.53 & 36.52 & 58.93 & RW2 \\ 
  Egypt & ALL & 2000 & 46.50 & 43.60 & 49.50 & UN \\ 
  Egypt & ALL & 2001 & 39.57 & 36.34 & 42.55 & IHME \\ 
  Egypt & ALL & 2001 & 43.78 & 34.77 & 54.90 & RW2 \\ 
  Egypt & ALL & 2001 & 43.80 & 41.00 & 46.80 & UN \\ 
  Egypt & ALL & 2002 & 37.03 & 34.04 & 40.15 & IHME \\ 
  Egypt & ALL & 2002 & 41.34 & 33.18 & 51.44 & RW2 \\ 
  Egypt & ALL & 2002 & 41.40 & 38.60 & 44.30 & UN \\ 
  Egypt & ALL & 2003 & 35.94 & 32.92 & 39.18 & IHME \\ 
  Egypt & ALL & 2003 & 39.23 & 31.26 & 49.13 & RW2 \\ 
  Egypt & ALL & 2003 & 39.20 & 36.50 & 42.20 & UN \\ 
  Egypt & ALL & 2004 & 35.26 & 32.09 & 38.43 & IHME \\ 
  Egypt & ALL & 2004 & 37.31 & 28.93 & 47.79 & RW2 \\ 
  Egypt & ALL & 2004 & 37.30 & 34.60 & 40.30 & UN \\ 
  Egypt & ALL & 2005 & 34.06 & 30.66 & 37.25 & IHME \\ 
  Egypt & ALL & 2005 & 35.73 & 27.21 & 47.04 & RW2 \\ 
  Egypt & ALL & 2005 & 35.60 & 32.80 & 38.60 & UN \\ 
  Egypt & ALL & 2006 & 32.60 & 29.41 & 36.22 & IHME \\ 
  Egypt & ALL & 2006 & 34.18 & 26.47 & 44.14 & RW2 \\ 
  Egypt & ALL & 2006 & 34.10 & 31.20 & 37.10 & UN \\ 
  Egypt & ALL & 2007 & 31.19 & 27.87 & 35.06 & IHME \\ 
  Egypt & ALL & 2007 & 32.77 & 25.99 & 41.26 & RW2 \\ 
  Egypt & ALL & 2007 & 32.70 & 29.80 & 35.90 & UN \\ 
  Egypt & ALL & 2008 & 30.04 & 26.49 & 33.66 & IHME \\ 
  Egypt & ALL & 2008 & 31.47 & 24.40 & 40.47 & RW2 \\ 
  Egypt & ALL & 2008 & 31.40 & 28.40 & 34.80 & UN \\ 
  Egypt & ALL & 2009 & 29.20 & 25.84 & 32.95 & IHME \\ 
  Egypt & ALL & 2009 & 30.13 & 21.03 & 43.19 & RW2 \\ 
  Egypt & ALL & 2009 & 30.20 & 27.00 & 33.70 & UN \\ 
  Egypt & ALL & 2010 & 28.46 & 25.21 & 32.15 & IHME \\ 
  Egypt & ALL & 2010 & 28.83 & 16.52 & 50.34 & RW2 \\ 
  Egypt & ALL & 2010 & 29.00 & 25.60 & 32.80 & UN \\ 
  Egypt & ALL & 2011 & 27.45 & 24.21 & 30.95 & IHME \\ 
  Egypt & ALL & 2011 & 27.65 & 12.77 & 58.94 & RW2 \\ 
  Egypt & ALL & 2011 & 27.90 & 24.20 & 32.00 & UN \\ 
  Egypt & ALL & 2012 & 26.07 & 22.94 & 29.63 & IHME \\ 
  Egypt & ALL & 2012 & 26.46 & 9.55 & 71.22 & RW2 \\ 
  Egypt & ALL & 2012 & 26.80 & 22.90 & 31.40 & UN \\ 
  Egypt & ALL & 2013 & 24.45 & 21.13 & 28.41 & IHME \\ 
  Egypt & ALL & 2013 & 25.43 & 7.07 & 87.08 & RW2 \\ 
  Egypt & ALL & 2013 & 25.80 & 21.60 & 30.90 & UN \\ 
  Egypt & ALL & 2014 & 22.82 & 18.44 & 28.28 & IHME \\ 
  Egypt & ALL & 2014 & 24.33 & 5.10 & 108.54 & RW2 \\ 
  Egypt & ALL & 2014 & 24.80 & 20.30 & 30.50 & UN \\ 
  Egypt & ALL & 2015 & 21.54 & 16.27 & 28.31 & IHME \\ 
  Egypt & ALL & 2015 & 23.22 & 3.66 & 134.77 & RW2 \\ 
  Egypt & ALL & 2015 & 24.00 & 19.10 & 30.30 & UN \\ 
  Egypt & ALL & 2016 & 22.41 & 2.60 & 173.27 & RW2 \\ 
  Egypt & ALL & 2017 & 21.34 & 1.76 & 220.17 & RW2 \\ 
  Egypt & ALL & 2018 & 20.43 & 1.21 & 282.60 & RW2 \\ 
  Egypt & ALL & 2019 & 19.51 & 0.75 & 344.02 & RW2 \\ 
  Egypt & ALL & 80-84 & 144.41 & 148.75 & 140.17 & HT-Direct \\ 
  Egypt & ALL & 85-89 & 98.18 & 101.15 & 95.28 & HT-Direct \\ 
  Egypt & ALL & 90-94 & 77.83 & 80.62 & 75.13 & HT-Direct \\ 
  Egypt & ALL & 95-99 & 54.31 & 56.84 & 51.88 & HT-Direct \\ 
  Egypt & ALL & 00-04 & 38.94 & 41.38 & 36.64 & HT-Direct \\ 
  Egypt & ALL & 05-09 & 32.00 & 35.02 & 29.24 & HT-Direct \\ 
  Egypt & ALL & 15-19 & 21.33 & 1.77 & 216.74 & RW2 \\ 
  Egypt & FRONTIER GOVERNORATES & 1980 & 113.53 & 82.70 & 154.40 & RW2 \\ 
  Egypt & FRONTIER GOVERNORATES & 1981 & 106.16 & 83.86 & 133.59 & RW2 \\ 
  Egypt & FRONTIER GOVERNORATES & 1982 & 99.26 & 80.08 & 122.13 & RW2 \\ 
  Egypt & FRONTIER GOVERNORATES & 1983 & 92.68 & 74.01 & 115.93 & RW2 \\ 
  Egypt & FRONTIER GOVERNORATES & 1984 & 86.75 & 68.64 & 109.33 & RW2 \\ 
  Egypt & FRONTIER GOVERNORATES & 1985 & 81.17 & 65.11 & 100.23 & RW2 \\ 
  Egypt & FRONTIER GOVERNORATES & 1986 & 76.26 & 62.73 & 92.64 & RW2 \\ 
  Egypt & FRONTIER GOVERNORATES & 1987 & 72.03 & 59.92 & 86.41 & RW2 \\ 
  Egypt & FRONTIER GOVERNORATES & 1988 & 68.29 & 56.29 & 82.59 & RW2 \\ 
  Egypt & FRONTIER GOVERNORATES & 1989 & 64.97 & 52.74 & 79.80 & RW2 \\ 
  Egypt & FRONTIER GOVERNORATES & 1990 & 62.14 & 50.52 & 76.45 & RW2 \\ 
  Egypt & FRONTIER GOVERNORATES & 1991 & 59.51 & 48.94 & 71.91 & RW2 \\ 
  Egypt & FRONTIER GOVERNORATES & 1992 & 56.94 & 47.11 & 68.43 & RW2 \\ 
  Egypt & FRONTIER GOVERNORATES & 1993 & 54.52 & 44.81 & 66.18 & RW2 \\ 
  Egypt & FRONTIER GOVERNORATES & 1994 & 52.20 & 42.10 & 64.41 & RW2 \\ 
  Egypt & FRONTIER GOVERNORATES & 1995 & 49.85 & 40.05 & 61.81 & RW2 \\ 
  Egypt & FRONTIER GOVERNORATES & 1996 & 47.57 & 38.67 & 58.13 & RW2 \\ 
  Egypt & FRONTIER GOVERNORATES & 1997 & 45.35 & 37.06 & 55.47 & RW2 \\ 
  Egypt & FRONTIER GOVERNORATES & 1998 & 43.26 & 35.02 & 53.43 & RW2 \\ 
  Egypt & FRONTIER GOVERNORATES & 1999 & 41.24 & 32.81 & 51.77 & RW2 \\ 
  Egypt & FRONTIER GOVERNORATES & 2000 & 39.34 & 31.04 & 49.75 & RW2 \\ 
  Egypt & FRONTIER GOVERNORATES & 2001 & 37.81 & 29.96 & 47.59 & RW2 \\ 
  Egypt & FRONTIER GOVERNORATES & 2002 & 36.52 & 28.97 & 46.04 & RW2 \\ 
  Egypt & FRONTIER GOVERNORATES & 2003 & 35.50 & 27.68 & 45.45 & RW2 \\ 
  Egypt & FRONTIER GOVERNORATES & 2004 & 34.74 & 26.34 & 45.48 & RW2 \\ 
  Egypt & FRONTIER GOVERNORATES & 2005 & 34.24 & 25.42 & 46.13 & RW2 \\ 
  Egypt & FRONTIER GOVERNORATES & 2006 & 33.74 & 25.11 & 45.63 & RW2 \\ 
  Egypt & FRONTIER GOVERNORATES & 2007 & 33.39 & 24.76 & 45.33 & RW2 \\ 
  Egypt & FRONTIER GOVERNORATES & 2008 & 33.06 & 23.67 & 46.39 & RW2 \\ 
  Egypt & FRONTIER GOVERNORATES & 2009 & 32.67 & 21.52 & 49.72 & RW2 \\ 
  Egypt & FRONTIER GOVERNORATES & 2010 & 32.33 & 18.23 & 57.15 & RW2 \\ 
  Egypt & FRONTIER GOVERNORATES & 2011 & 31.98 & 15.24 & 66.86 & RW2 \\ 
  Egypt & FRONTIER GOVERNORATES & 2012 & 31.63 & 12.30 & 80.08 & RW2 \\ 
  Egypt & FRONTIER GOVERNORATES & 2013 & 31.29 & 9.74 & 97.55 & RW2 \\ 
  Egypt & FRONTIER GOVERNORATES & 2014 & 30.84 & 7.39 & 119.90 & RW2 \\ 
  Egypt & FRONTIER GOVERNORATES & 2015 & 30.64 & 5.68 & 151.09 & RW2 \\ 
  Egypt & FRONTIER GOVERNORATES & 2016 & 30.22 & 4.19 & 188.86 & RW2 \\ 
  Egypt & FRONTIER GOVERNORATES & 2017 & 30.04 & 3.11 & 238.81 & RW2 \\ 
  Egypt & FRONTIER GOVERNORATES & 2018 & 29.52 & 2.27 & 296.03 & RW2 \\ 
  Egypt & FRONTIER GOVERNORATES & 2019 & 29.40 & 1.65 & 366.90 & RW2 \\ 
  Egypt & LOWER EGYPT & 1980 & 146.55 & 112.97 & 186.97 & RW2 \\ 
  Egypt & LOWER EGYPT & 1981 & 135.86 & 114.36 & 159.94 & RW2 \\ 
  Egypt & LOWER EGYPT & 1982 & 125.76 & 107.22 & 146.99 & RW2 \\ 
  Egypt & LOWER EGYPT & 1983 & 116.23 & 96.79 & 139.54 & RW2 \\ 
  Egypt & LOWER EGYPT & 1984 & 107.65 & 88.07 & 131.33 & RW2 \\ 
  Egypt & LOWER EGYPT & 1985 & 99.60 & 82.63 & 119.69 & RW2 \\ 
  Egypt & LOWER EGYPT & 1986 & 92.53 & 78.31 & 109.12 & RW2 \\ 
  Egypt & LOWER EGYPT & 1987 & 86.27 & 73.49 & 101.12 & RW2 \\ 
  Egypt & LOWER EGYPT & 1988 & 80.66 & 67.99 & 95.18 & RW2 \\ 
  Egypt & LOWER EGYPT & 1989 & 75.69 & 62.84 & 90.74 & RW2 \\ 
  Egypt & LOWER EGYPT & 1990 & 71.25 & 59.24 & 85.52 & RW2 \\ 
  Egypt & LOWER EGYPT & 1991 & 67.06 & 56.70 & 79.31 & RW2 \\ 
  Egypt & LOWER EGYPT & 1992 & 63.15 & 53.69 & 74.08 & RW2 \\ 
  Egypt & LOWER EGYPT & 1993 & 59.38 & 49.85 & 70.46 & RW2 \\ 
  Egypt & LOWER EGYPT & 1994 & 55.79 & 46.08 & 67.29 & RW2 \\ 
  Egypt & LOWER EGYPT & 1995 & 52.30 & 43.22 & 63.12 & RW2 \\ 
  Egypt & LOWER EGYPT & 1996 & 49.01 & 40.95 & 58.34 & RW2 \\ 
  Egypt & LOWER EGYPT & 1997 & 45.85 & 38.64 & 54.33 & RW2 \\ 
  Egypt & LOWER EGYPT & 1998 & 42.90 & 35.84 & 51.35 & RW2 \\ 
  Egypt & LOWER EGYPT & 1999 & 40.11 & 32.86 & 48.94 & RW2 \\ 
  Egypt & LOWER EGYPT & 2000 & 37.58 & 30.61 & 45.73 & RW2 \\ 
  Egypt & LOWER EGYPT & 2001 & 35.41 & 29.26 & 42.65 & RW2 \\ 
  Egypt & LOWER EGYPT & 2002 & 33.57 & 27.99 & 40.25 & RW2 \\ 
  Egypt & LOWER EGYPT & 2003 & 31.99 & 26.28 & 38.75 & RW2 \\ 
  Egypt & LOWER EGYPT & 2004 & 30.69 & 24.61 & 38.15 & RW2 \\ 
  Egypt & LOWER EGYPT & 2005 & 29.61 & 23.39 & 37.63 & RW2 \\ 
  Egypt & LOWER EGYPT & 2006 & 28.62 & 22.79 & 35.95 & RW2 \\ 
  Egypt & LOWER EGYPT & 2007 & 27.72 & 22.38 & 34.31 & RW2 \\ 
  Egypt & LOWER EGYPT & 2008 & 26.85 & 21.09 & 34.24 & RW2 \\ 
  Egypt & LOWER EGYPT & 2009 & 26.03 & 18.54 & 36.43 & RW2 \\ 
  Egypt & LOWER EGYPT & 2010 & 25.18 & 15.02 & 41.82 & RW2 \\ 
  Egypt & LOWER EGYPT & 2011 & 24.40 & 12.23 & 48.58 & RW2 \\ 
  Egypt & LOWER EGYPT & 2012 & 23.64 & 9.62 & 57.10 & RW2 \\ 
  Egypt & LOWER EGYPT & 2013 & 22.88 & 7.39 & 69.13 & RW2 \\ 
  Egypt & LOWER EGYPT & 2014 & 22.14 & 5.56 & 84.83 & RW2 \\ 
  Egypt & LOWER EGYPT & 2015 & 21.51 & 4.09 & 105.74 & RW2 \\ 
  Egypt & LOWER EGYPT & 2016 & 20.80 & 2.99 & 132.05 & RW2 \\ 
  Egypt & LOWER EGYPT & 2017 & 20.09 & 2.18 & 164.11 & RW2 \\ 
  Egypt & LOWER EGYPT & 2018 & 19.48 & 1.57 & 207.94 & RW2 \\ 
  Egypt & LOWER EGYPT & 2019 & 18.89 & 1.07 & 261.09 & RW2 \\ 
  Egypt & UPPER EGYPT & 1980 & 230.29 & 182.73 & 285.50 & RW2 \\ 
  Egypt & UPPER EGYPT & 1981 & 213.76 & 183.18 & 247.75 & RW2 \\ 
  Egypt & UPPER EGYPT & 1982 & 197.89 & 171.45 & 227.21 & RW2 \\ 
  Egypt & UPPER EGYPT & 1983 & 183.12 & 154.64 & 215.91 & RW2 \\ 
  Egypt & UPPER EGYPT & 1984 & 169.46 & 140.01 & 203.87 & RW2 \\ 
  Egypt & UPPER EGYPT & 1985 & 156.91 & 131.68 & 185.88 & RW2 \\ 
  Egypt & UPPER EGYPT & 1986 & 145.95 & 124.73 & 169.71 & RW2 \\ 
  Egypt & UPPER EGYPT & 1987 & 136.31 & 117.43 & 157.71 & RW2 \\ 
  Egypt & UPPER EGYPT & 1988 & 127.89 & 108.85 & 149.60 & RW2 \\ 
  Egypt & UPPER EGYPT & 1989 & 120.48 & 100.97 & 142.88 & RW2 \\ 
  Egypt & UPPER EGYPT & 1990 & 114.00 & 95.76 & 135.59 & RW2 \\ 
  Egypt & UPPER EGYPT & 1991 & 107.89 & 91.97 & 126.47 & RW2 \\ 
  Egypt & UPPER EGYPT & 1992 & 102.00 & 87.26 & 118.76 & RW2 \\ 
  Egypt & UPPER EGYPT & 1993 & 96.22 & 81.63 & 112.88 & RW2 \\ 
  Egypt & UPPER EGYPT & 1994 & 90.64 & 75.51 & 107.96 & RW2 \\ 
  Egypt & UPPER EGYPT & 1995 & 85.06 & 70.77 & 101.70 & RW2 \\ 
  Egypt & UPPER EGYPT & 1996 & 79.62 & 67.12 & 93.98 & RW2 \\ 
  Egypt & UPPER EGYPT & 1997 & 74.28 & 63.18 & 87.32 & RW2 \\ 
  Egypt & UPPER EGYPT & 1998 & 69.16 & 58.57 & 81.88 & RW2 \\ 
  Egypt & UPPER EGYPT & 1999 & 64.36 & 53.28 & 77.44 & RW2 \\ 
  Egypt & UPPER EGYPT & 2000 & 59.78 & 49.12 & 72.13 & RW2 \\ 
  Egypt & UPPER EGYPT & 2001 & 55.87 & 46.54 & 66.65 & RW2 \\ 
  Egypt & UPPER EGYPT & 2002 & 52.44 & 44.11 & 62.19 & RW2 \\ 
  Egypt & UPPER EGYPT & 2003 & 49.47 & 41.22 & 59.35 & RW2 \\ 
  Egypt & UPPER EGYPT & 2004 & 46.95 & 38.15 & 57.49 & RW2 \\ 
  Egypt & UPPER EGYPT & 2005 & 44.85 & 35.93 & 56.15 & RW2 \\ 
  Egypt & UPPER EGYPT & 2006 & 42.90 & 34.89 & 52.88 & RW2 \\ 
  Egypt & UPPER EGYPT & 2007 & 41.08 & 33.81 & 49.71 & RW2 \\ 
  Egypt & UPPER EGYPT & 2008 & 39.40 & 31.54 & 48.95 & RW2 \\ 
  Egypt & UPPER EGYPT & 2009 & 37.71 & 27.37 & 51.67 & RW2 \\ 
  Egypt & UPPER EGYPT & 2010 & 36.11 & 22.13 & 58.62 & RW2 \\ 
  Egypt & UPPER EGYPT & 2011 & 34.53 & 17.60 & 67.18 & RW2 \\ 
  Egypt & UPPER EGYPT & 2012 & 33.06 & 13.70 & 78.39 & RW2 \\ 
  Egypt & UPPER EGYPT & 2013 & 31.71 & 10.45 & 94.63 & RW2 \\ 
  Egypt & UPPER EGYPT & 2014 & 30.41 & 7.76 & 112.76 & RW2 \\ 
  Egypt & UPPER EGYPT & 2015 & 29.12 & 5.70 & 137.01 & RW2 \\ 
  Egypt & UPPER EGYPT & 2016 & 27.78 & 4.10 & 167.00 & RW2 \\ 
  Egypt & UPPER EGYPT & 2017 & 26.70 & 2.94 & 207.00 & RW2 \\ 
  Egypt & UPPER EGYPT & 2018 & 25.41 & 2.03 & 256.81 & RW2 \\ 
  Egypt & UPPER EGYPT & 2019 & 24.37 & 1.40 & 320.23 & RW2 \\ 
  Egypt & URBAN GOVERNORATES & 1980 & 103.01 & 78.04 & 135.03 & RW2 \\ 
  Egypt & URBAN GOVERNORATES & 1981 & 96.29 & 79.55 & 115.92 & RW2 \\ 
  Egypt & URBAN GOVERNORATES & 1982 & 89.96 & 75.64 & 106.85 & RW2 \\ 
  Egypt & URBAN GOVERNORATES & 1983 & 84.04 & 69.04 & 102.31 & RW2 \\ 
  Egypt & URBAN GOVERNORATES & 1984 & 78.63 & 63.46 & 97.47 & RW2 \\ 
  Egypt & URBAN GOVERNORATES & 1985 & 73.44 & 60.19 & 89.29 & RW2 \\ 
  Egypt & URBAN GOVERNORATES & 1986 & 68.99 & 57.83 & 82.07 & RW2 \\ 
  Egypt & URBAN GOVERNORATES & 1987 & 64.93 & 54.87 & 76.94 & RW2 \\ 
  Egypt & URBAN GOVERNORATES & 1988 & 61.40 & 51.19 & 73.30 & RW2 \\ 
  Egypt & URBAN GOVERNORATES & 1989 & 58.23 & 47.86 & 70.63 & RW2 \\ 
  Egypt & URBAN GOVERNORATES & 1990 & 55.49 & 45.73 & 67.42 & RW2 \\ 
  Egypt & URBAN GOVERNORATES & 1991 & 52.86 & 44.12 & 63.43 & RW2 \\ 
  Egypt & URBAN GOVERNORATES & 1992 & 50.40 & 42.23 & 59.93 & RW2 \\ 
  Egypt & URBAN GOVERNORATES & 1993 & 48.07 & 39.95 & 57.62 & RW2 \\ 
  Egypt & URBAN GOVERNORATES & 1994 & 45.84 & 37.37 & 56.02 & RW2 \\ 
  Egypt & URBAN GOVERNORATES & 1995 & 43.64 & 35.49 & 53.34 & RW2 \\ 
  Egypt & URBAN GOVERNORATES & 1996 & 41.56 & 34.20 & 50.28 & RW2 \\ 
  Egypt & URBAN GOVERNORATES & 1997 & 39.52 & 32.75 & 47.52 & RW2 \\ 
  Egypt & URBAN GOVERNORATES & 1998 & 37.60 & 30.87 & 45.69 & RW2 \\ 
  Egypt & URBAN GOVERNORATES & 1999 & 35.82 & 28.95 & 44.43 & RW2 \\ 
  Egypt & URBAN GOVERNORATES & 2000 & 34.10 & 27.28 & 42.37 & RW2 \\ 
  Egypt & URBAN GOVERNORATES & 2001 & 32.75 & 26.59 & 40.36 & RW2 \\ 
  Egypt & URBAN GOVERNORATES & 2002 & 31.65 & 25.70 & 38.89 & RW2 \\ 
  Egypt & URBAN GOVERNORATES & 2003 & 30.76 & 24.59 & 38.28 & RW2 \\ 
  Egypt & URBAN GOVERNORATES & 2004 & 30.10 & 23.48 & 38.41 & RW2 \\ 
  Egypt & URBAN GOVERNORATES & 2005 & 29.67 & 22.70 & 38.80 & RW2 \\ 
  Egypt & URBAN GOVERNORATES & 2006 & 29.28 & 22.45 & 38.22 & RW2 \\ 
  Egypt & URBAN GOVERNORATES & 2007 & 28.94 & 22.30 & 38.00 & RW2 \\ 
  Egypt & URBAN GOVERNORATES & 2008 & 28.66 & 21.33 & 38.70 & RW2 \\ 
  Egypt & URBAN GOVERNORATES & 2009 & 28.35 & 19.36 & 41.88 & RW2 \\ 
  Egypt & URBAN GOVERNORATES & 2010 & 28.09 & 16.28 & 48.36 & RW2 \\ 
  Egypt & URBAN GOVERNORATES & 2011 & 27.78 & 13.50 & 57.00 & RW2 \\ 
  Egypt & URBAN GOVERNORATES & 2012 & 27.54 & 11.04 & 68.57 & RW2 \\ 
  Egypt & URBAN GOVERNORATES & 2013 & 27.27 & 8.56 & 83.15 & RW2 \\ 
  Egypt & URBAN GOVERNORATES & 2014 & 26.94 & 6.56 & 105.15 & RW2 \\ 
  Egypt & URBAN GOVERNORATES & 2015 & 26.74 & 5.02 & 132.63 & RW2 \\ 
  Egypt & URBAN GOVERNORATES & 2016 & 26.44 & 3.77 & 165.29 & RW2 \\ 
  Egypt & URBAN GOVERNORATES & 2017 & 26.11 & 2.77 & 209.25 & RW2 \\ 
  Egypt & URBAN GOVERNORATES & 2018 & 25.88 & 2.00 & 266.60 & RW2 \\ 
  Egypt & URBAN GOVERNORATES & 2019 & 25.52 & 1.44 & 335.29 & RW2 \\ 
  Ethiopia & ADDIS ABABA & 1980 & 90.78 & 63.18 & 127.92 & RW2 \\ 
  Ethiopia & ADDIS ABABA & 1981 & 90.31 & 67.40 & 119.03 & RW2 \\ 
  Ethiopia & ADDIS ABABA & 1982 & 89.83 & 69.36 & 115.43 & RW2 \\ 
  Ethiopia & ADDIS ABABA & 1983 & 89.38 & 69.23 & 114.45 & RW2 \\ 
  Ethiopia & ADDIS ABABA & 1984 & 88.96 & 68.95 & 114.20 & RW2 \\ 
  Ethiopia & ADDIS ABABA & 1985 & 88.67 & 70.07 & 111.49 & RW2 \\ 
  Ethiopia & ADDIS ABABA & 1986 & 88.78 & 71.67 & 109.22 & RW2 \\ 
  Ethiopia & ADDIS ABABA & 1987 & 89.16 & 72.68 & 108.38 & RW2 \\ 
  Ethiopia & ADDIS ABABA & 1988 & 89.70 & 73.07 & 109.11 & RW2 \\ 
  Ethiopia & ADDIS ABABA & 1989 & 90.29 & 73.04 & 110.74 & RW2 \\ 
  Ethiopia & ADDIS ABABA & 1990 & 91.11 & 74.29 & 111.52 & RW2 \\ 
  Ethiopia & ADDIS ABABA & 1991 & 91.18 & 75.24 & 110.18 & RW2 \\ 
  Ethiopia & ADDIS ABABA & 1992 & 90.75 & 75.22 & 108.95 & RW2 \\ 
  Ethiopia & ADDIS ABABA & 1993 & 89.47 & 73.86 & 108.42 & RW2 \\ 
  Ethiopia & ADDIS ABABA & 1994 & 87.70 & 71.82 & 107.34 & RW2 \\ 
  Ethiopia & ADDIS ABABA & 1995 & 84.98 & 69.35 & 104.09 & RW2 \\ 
  Ethiopia & ADDIS ABABA & 1996 & 82.38 & 67.74 & 100.18 & RW2 \\ 
  Ethiopia & ADDIS ABABA & 1997 & 79.60 & 65.74 & 96.42 & RW2 \\ 
  Ethiopia & ADDIS ABABA & 1998 & 76.72 & 62.78 & 93.85 & RW2 \\ 
  Ethiopia & ADDIS ABABA & 1999 & 73.81 & 59.71 & 91.21 & RW2 \\ 
  Ethiopia & ADDIS ABABA & 2000 & 71.06 & 57.08 & 88.33 & RW2 \\ 
  Ethiopia & ADDIS ABABA & 2001 & 67.85 & 54.94 & 83.90 & RW2 \\ 
  Ethiopia & ADDIS ABABA & 2002 & 64.50 & 52.13 & 79.64 & RW2 \\ 
  Ethiopia & ADDIS ABABA & 2003 & 60.92 & 48.65 & 76.08 & RW2 \\ 
  Ethiopia & ADDIS ABABA & 2004 & 57.32 & 44.90 & 72.94 & RW2 \\ 
  Ethiopia & ADDIS ABABA & 2005 & 53.61 & 41.52 & 68.77 & RW2 \\ 
  Ethiopia & ADDIS ABABA & 2006 & 50.21 & 38.89 & 64.60 & RW2 \\ 
  Ethiopia & ADDIS ABABA & 2007 & 47.00 & 36.06 & 60.73 & RW2 \\ 
  Ethiopia & ADDIS ABABA & 2008 & 44.09 & 33.14 & 58.30 & RW2 \\ 
  Ethiopia & ADDIS ABABA & 2009 & 41.43 & 30.25 & 56.26 & RW2 \\ 
  Ethiopia & ADDIS ABABA & 2010 & 39.08 & 27.62 & 54.83 & RW2 \\ 
  Ethiopia & ADDIS ABABA & 2011 & 36.81 & 25.50 & 52.29 & RW2 \\ 
  Ethiopia & ADDIS ABABA & 2012 & 34.77 & 23.48 & 50.23 & RW2 \\ 
  Ethiopia & ADDIS ABABA & 2013 & 32.83 & 21.19 & 49.35 & RW2 \\ 
  Ethiopia & ADDIS ABABA & 2014 & 31.00 & 18.52 & 50.19 & RW2 \\ 
  Ethiopia & ADDIS ABABA & 2015 & 29.16 & 15.24 & 53.50 & RW2 \\ 
  Ethiopia & ADDIS ABABA & 2016 & 27.49 & 12.45 & 58.43 & RW2 \\ 
  Ethiopia & ADDIS ABABA & 2017 & 25.96 & 9.89 & 65.43 & RW2 \\ 
  Ethiopia & ADDIS ABABA & 2018 & 24.48 & 7.71 & 75.09 & RW2 \\ 
  Ethiopia & ADDIS ABABA & 2019 & 23.06 & 5.71 & 87.92 & RW2 \\ 
  Ethiopia & AFFAR & 1980 & 338.93 & 268.27 & 418.62 & RW2 \\ 
  Ethiopia & AFFAR & 1981 & 328.63 & 275.04 & 388.49 & RW2 \\ 
  Ethiopia & AFFAR & 1982 & 318.60 & 271.43 & 372.14 & RW2 \\ 
  Ethiopia & AFFAR & 1983 & 309.04 & 261.49 & 362.84 & RW2 \\ 
  Ethiopia & AFFAR & 1984 & 299.86 & 251.34 & 353.45 & RW2 \\ 
  Ethiopia & AFFAR & 1985 & 290.59 & 247.35 & 338.81 & RW2 \\ 
  Ethiopia & AFFAR & 1986 & 282.55 & 244.04 & 324.92 & RW2 \\ 
  Ethiopia & AFFAR & 1987 & 275.26 & 239.75 & 314.22 & RW2 \\ 
  Ethiopia & AFFAR & 1988 & 268.56 & 232.17 & 307.61 & RW2 \\ 
  Ethiopia & AFFAR & 1989 & 261.92 & 224.63 & 302.91 & RW2 \\ 
  Ethiopia & AFFAR & 1990 & 256.29 & 220.46 & 295.79 & RW2 \\ 
  Ethiopia & AFFAR & 1991 & 249.12 & 215.79 & 285.52 & RW2 \\ 
  Ethiopia & AFFAR & 1992 & 241.15 & 209.88 & 274.79 & RW2 \\ 
  Ethiopia & AFFAR & 1993 & 232.08 & 200.32 & 267.04 & RW2 \\ 
  Ethiopia & AFFAR & 1994 & 222.35 & 189.77 & 258.87 & RW2 \\ 
  Ethiopia & AFFAR & 1995 & 211.67 & 179.56 & 246.39 & RW2 \\ 
  Ethiopia & AFFAR & 1996 & 202.11 & 173.57 & 234.02 & RW2 \\ 
  Ethiopia & AFFAR & 1997 & 193.16 & 166.05 & 222.63 & RW2 \\ 
  Ethiopia & AFFAR & 1998 & 184.68 & 157.91 & 214.50 & RW2 \\ 
  Ethiopia & AFFAR & 1999 & 177.00 & 149.19 & 207.42 & RW2 \\ 
  Ethiopia & AFFAR & 2000 & 170.15 & 143.48 & 200.68 & RW2 \\ 
  Ethiopia & AFFAR & 2001 & 162.81 & 138.63 & 190.34 & RW2 \\ 
  Ethiopia & AFFAR & 2002 & 155.44 & 132.96 & 181.01 & RW2 \\ 
  Ethiopia & AFFAR & 2003 & 147.83 & 125.50 & 172.82 & RW2 \\ 
  Ethiopia & AFFAR & 2004 & 140.00 & 117.24 & 166.18 & RW2 \\ 
  Ethiopia & AFFAR & 2005 & 132.08 & 110.15 & 157.43 & RW2 \\ 
  Ethiopia & AFFAR & 2006 & 124.62 & 104.70 & 147.43 & RW2 \\ 
  Ethiopia & AFFAR & 2007 & 117.67 & 99.30 & 139.27 & RW2 \\ 
  Ethiopia & AFFAR & 2008 & 111.14 & 92.72 & 133.23 & RW2 \\ 
  Ethiopia & AFFAR & 2009 & 105.25 & 85.59 & 128.84 & RW2 \\ 
  Ethiopia & AFFAR & 2010 & 99.85 & 79.88 & 124.73 & RW2 \\ 
  Ethiopia & AFFAR & 2011 & 94.75 & 76.00 & 117.85 & RW2 \\ 
  Ethiopia & AFFAR & 2012 & 89.89 & 72.59 & 112.00 & RW2 \\ 
  Ethiopia & AFFAR & 2013 & 85.27 & 66.97 & 108.69 & RW2 \\ 
  Ethiopia & AFFAR & 2014 & 80.88 & 58.95 & 111.24 & RW2 \\ 
  Ethiopia & AFFAR & 2015 & 76.79 & 48.15 & 121.15 & RW2 \\ 
  Ethiopia & AFFAR & 2016 & 72.67 & 39.33 & 132.15 & RW2 \\ 
  Ethiopia & AFFAR & 2017 & 68.83 & 30.69 & 148.05 & RW2 \\ 
  Ethiopia & AFFAR & 2018 & 65.30 & 23.78 & 167.12 & RW2 \\ 
  Ethiopia & AFFAR & 2019 & 61.89 & 17.91 & 190.41 & RW2 \\ 
  Ethiopia & ALL & 1980 & 234.70 & 226.49 & 242.99 & IHME \\ 
  Ethiopia & ALL & 1980 & 244.89 & 189.68 & 309.51 & RW2 \\ 
  Ethiopia & ALL & 1980 & 241.60 & 222.40 & 263.00 & UN \\ 
  Ethiopia & ALL & 1981 & 231.96 & 224.28 & 239.72 & IHME \\ 
  Ethiopia & ALL & 1981 & 240.35 & 200.48 & 284.57 & RW2 \\ 
  Ethiopia & ALL & 1981 & 239.30 & 220.80 & 260.10 & UN \\ 
  Ethiopia & ALL & 1982 & 229.35 & 221.93 & 237.11 & IHME \\ 
  Ethiopia & ALL & 1982 & 235.88 & 200.23 & 275.81 & RW2 \\ 
  Ethiopia & ALL & 1982 & 236.00 & 218.10 & 255.90 & UN \\ 
  Ethiopia & ALL & 1983 & 282.13 & 221.71 & 366.04 & IHME \\ 
  Ethiopia & ALL & 1983 & 231.26 & 192.90 & 275.00 & RW2 \\ 
  Ethiopia & ALL & 1983 & 232.00 & 214.70 & 251.00 & UN \\ 
  Ethiopia & ALL & 1984 & 276.95 & 218.69 & 356.99 & IHME \\ 
  Ethiopia & ALL & 1984 & 227.19 & 185.77 & 274.62 & RW2 \\ 
  Ethiopia & ALL & 1984 & 227.60 & 210.80 & 245.80 & UN \\ 
  Ethiopia & ALL & 1985 & 219.13 & 212.03 & 225.49 & IHME \\ 
  Ethiopia & ALL & 1985 & 222.63 & 185.29 & 265.39 & RW2 \\ 
  Ethiopia & ALL & 1985 & 223.40 & 206.90 & 241.00 & UN \\ 
  Ethiopia & ALL & 1986 & 215.46 & 208.72 & 221.66 & IHME \\ 
  Ethiopia & ALL & 1986 & 218.82 & 184.66 & 257.50 & RW2 \\ 
  Ethiopia & ALL & 1986 & 219.30 & 203.20 & 236.50 & UN \\ 
  Ethiopia & ALL & 1987 & 212.19 & 205.78 & 218.43 & IHME \\ 
  Ethiopia & ALL & 1987 & 215.35 & 183.15 & 252.00 & RW2 \\ 
  Ethiopia & ALL & 1987 & 215.50 & 200.00 & 232.60 & UN \\ 
  Ethiopia & ALL & 1988 & 208.26 & 201.59 & 214.06 & IHME \\ 
  Ethiopia & ALL & 1988 & 211.91 & 178.28 & 249.25 & RW2 \\ 
  Ethiopia & ALL & 1988 & 212.00 & 196.90 & 228.50 & UN \\ 
  Ethiopia & ALL & 1989 & 205.09 & 198.72 & 211.53 & IHME \\ 
  Ethiopia & ALL & 1989 & 208.43 & 173.28 & 247.53 & RW2 \\ 
  Ethiopia & ALL & 1989 & 208.40 & 193.90 & 224.70 & UN \\ 
  Ethiopia & ALL & 1990 & 202.24 & 195.85 & 208.88 & IHME \\ 
  Ethiopia & ALL & 1990 & 205.23 & 171.29 & 244.40 & RW2 \\ 
  Ethiopia & ALL & 1990 & 204.60 & 190.20 & 220.60 & UN \\ 
  Ethiopia & ALL & 1991 & 196.20 & 190.36 & 202.28 & IHME \\ 
  Ethiopia & ALL & 1991 & 200.75 & 169.11 & 235.98 & RW2 \\ 
  Ethiopia & ALL & 1991 & 200.20 & 186.10 & 215.90 & UN \\ 
  Ethiopia & ALL & 1992 & 191.69 & 186.07 & 197.46 & IHME \\ 
  Ethiopia & ALL & 1992 & 195.40 & 165.23 & 229.05 & RW2 \\ 
  Ethiopia & ALL & 1992 & 195.00 & 181.10 & 210.40 & UN \\ 
  Ethiopia & ALL & 1993 & 186.88 & 181.35 & 192.60 & IHME \\ 
  Ethiopia & ALL & 1993 & 189.24 & 159.38 & 223.69 & RW2 \\ 
  Ethiopia & ALL & 1993 & 188.90 & 175.40 & 203.60 & UN \\ 
  Ethiopia & ALL & 1994 & 181.58 & 176.20 & 187.06 & IHME \\ 
  Ethiopia & ALL & 1994 & 182.40 & 151.90 & 219.25 & RW2 \\ 
  Ethiopia & ALL & 1994 & 182.00 & 169.20 & 196.10 & UN \\ 
  Ethiopia & ALL & 1995 & 175.72 & 170.40 & 181.23 & IHME \\ 
  Ethiopia & ALL & 1995 & 174.69 & 144.78 & 208.63 & RW2 \\ 
  Ethiopia & ALL & 1995 & 175.00 & 162.50 & 188.40 & UN \\ 
  Ethiopia & ALL & 1996 & 169.09 & 163.83 & 174.58 & IHME \\ 
  Ethiopia & ALL & 1996 & 167.84 & 140.63 & 199.59 & RW2 \\ 
  Ethiopia & ALL & 1996 & 168.30 & 156.10 & 181.00 & UN \\ 
  Ethiopia & ALL & 1997 & 162.37 & 157.30 & 167.72 & IHME \\ 
  Ethiopia & ALL & 1997 & 161.49 & 136.05 & 190.88 & RW2 \\ 
  Ethiopia & ALL & 1997 & 162.00 & 150.20 & 174.20 & UN \\ 
  Ethiopia & ALL & 1998 & 156.20 & 151.15 & 161.29 & IHME \\ 
  Ethiopia & ALL & 1998 & 155.71 & 130.07 & 185.72 & RW2 \\ 
  Ethiopia & ALL & 1998 & 156.20 & 144.90 & 168.10 & UN \\ 
  Ethiopia & ALL & 1999 & 151.30 & 146.11 & 156.53 & IHME \\ 
  Ethiopia & ALL & 1999 & 150.17 & 123.50 & 180.09 & RW2 \\ 
  Ethiopia & ALL & 1999 & 150.80 & 139.60 & 162.70 & UN \\ 
  Ethiopia & ALL & 2000 & 143.12 & 138.13 & 147.83 & IHME \\ 
  Ethiopia & ALL & 2000 & 145.37 & 120.23 & 175.71 & RW2 \\ 
  Ethiopia & ALL & 2000 & 145.10 & 134.10 & 157.20 & UN \\ 
  Ethiopia & ALL & 2001 & 136.23 & 131.23 & 140.75 & IHME \\ 
  Ethiopia & ALL & 2001 & 139.35 & 115.89 & 167.06 & RW2 \\ 
  Ethiopia & ALL & 2001 & 139.00 & 128.10 & 151.20 & UN \\ 
  Ethiopia & ALL & 2002 & 129.44 & 124.59 & 134.04 & IHME \\ 
  Ethiopia & ALL & 2002 & 132.64 & 110.79 & 158.17 & RW2 \\ 
  Ethiopia & ALL & 2002 & 132.20 & 121.70 & 144.20 & UN \\ 
  Ethiopia & ALL & 2003 & 123.07 & 118.11 & 127.65 & IHME \\ 
  Ethiopia & ALL & 2003 & 125.32 & 104.19 & 150.23 & RW2 \\ 
  Ethiopia & ALL & 2003 & 124.80 & 114.70 & 136.30 & UN \\ 
  Ethiopia & ALL & 2004 & 116.70 & 111.86 & 121.51 & IHME \\ 
  Ethiopia & ALL & 2004 & 117.35 & 96.03 & 142.89 & RW2 \\ 
  Ethiopia & ALL & 2004 & 117.00 & 107.20 & 128.10 & UN \\ 
  Ethiopia & ALL & 2005 & 110.17 & 104.90 & 115.19 & IHME \\ 
  Ethiopia & ALL & 2005 & 108.92 & 88.16 & 132.79 & RW2 \\ 
  Ethiopia & ALL & 2005 & 109.10 & 99.60 & 119.90 & UN \\ 
  Ethiopia & ALL & 2006 & 103.72 & 98.39 & 108.99 & IHME \\ 
  Ethiopia & ALL & 2006 & 101.21 & 82.80 & 122.41 & RW2 \\ 
  Ethiopia & ALL & 2006 & 101.30 & 91.30 & 112.30 & UN \\ 
  Ethiopia & ALL & 2007 & 97.51 & 91.90 & 103.18 & IHME \\ 
  Ethiopia & ALL & 2007 & 94.07 & 77.60 & 113.39 & RW2 \\ 
  Ethiopia & ALL & 2007 & 93.80 & 83.10 & 105.60 & UN \\ 
  Ethiopia & ALL & 2008 & 91.82 & 86.06 & 97.92 & IHME \\ 
  Ethiopia & ALL & 2008 & 87.67 & 71.76 & 106.81 & RW2 \\ 
  Ethiopia & ALL & 2008 & 86.90 & 75.00 & 99.80 & UN \\ 
  Ethiopia & ALL & 2009 & 86.48 & 80.60 & 92.94 & IHME \\ 
  Ethiopia & ALL & 2009 & 81.76 & 65.70 & 101.58 & RW2 \\ 
  Ethiopia & ALL & 2009 & 80.80 & 67.80 & 94.90 & UN \\ 
  Ethiopia & ALL & 2010 & 81.63 & 75.47 & 88.41 & IHME \\ 
  Ethiopia & ALL & 2010 & 76.66 & 60.74 & 97.19 & RW2 \\ 
  Ethiopia & ALL & 2010 & 75.70 & 61.50 & 90.80 & UN \\ 
  Ethiopia & ALL & 2011 & 76.99 & 70.63 & 84.07 & IHME \\ 
  Ethiopia & ALL & 2011 & 71.86 & 57.39 & 89.95 & RW2 \\ 
  Ethiopia & ALL & 2011 & 71.30 & 56.10 & 87.90 & UN \\ 
  Ethiopia & ALL & 2012 & 72.56 & 66.10 & 80.04 & IHME \\ 
  Ethiopia & ALL & 2012 & 67.38 & 54.34 & 83.35 & RW2 \\ 
  Ethiopia & ALL & 2012 & 67.70 & 51.40 & 85.90 & UN \\ 
  Ethiopia & ALL & 2013 & 68.29 & 61.63 & 76.01 & IHME \\ 
  Ethiopia & ALL & 2013 & 63.24 & 49.66 & 80.07 & RW2 \\ 
  Ethiopia & ALL & 2013 & 64.60 & 47.20 & 84.50 & UN \\ 
  Ethiopia & ALL & 2014 & 64.20 & 57.48 & 72.16 & IHME \\ 
  Ethiopia & ALL & 2014 & 59.27 & 42.36 & 82.38 & RW2 \\ 
  Ethiopia & ALL & 2014 & 61.80 & 43.80 & 83.30 & UN \\ 
  Ethiopia & ALL & 2015 & 60.27 & 53.72 & 68.47 & IHME \\ 
  Ethiopia & ALL & 2015 & 55.44 & 33.42 & 91.02 & RW2 \\ 
  Ethiopia & ALL & 2015 & 59.20 & 40.60 & 83.00 & UN \\ 
  Ethiopia & ALL & 2016 & 52.05 & 26.34 & 101.62 & RW2 \\ 
  Ethiopia & ALL & 2017 & 48.62 & 20.02 & 115.60 & RW2 \\ 
  Ethiopia & ALL & 2018 & 45.50 & 15.06 & 134.58 & RW2 \\ 
  Ethiopia & ALL & 2019 & 42.51 & 10.69 & 155.58 & RW2 \\ 
  Ethiopia & ALL & 80-84 & 237.38 & 251.21 & 224.10 & HT-Direct \\ 
  Ethiopia & ALL & 85-89 & 214.10 & 223.65 & 204.86 & HT-Direct \\ 
  Ethiopia & ALL & 90-94 & 202.75 & 211.15 & 194.59 & HT-Direct \\ 
  Ethiopia & ALL & 95-99 & 160.00 & 166.44 & 153.76 & HT-Direct \\ 
  Ethiopia & ALL & 00-04 & 131.67 & 138.63 & 125.00 & HT-Direct \\ 
  Ethiopia & ALL & 05-09 & 96.77 & 103.94 & 90.05 & HT-Direct \\ 
  Ethiopia & ALL & 10-14 & 76.09 & 85.93 & 67.30 & HT-Direct \\ 
  Ethiopia & ALL & 15-19 & 48.61 & 20.34 & 113.73 & RW2 \\ 
  Ethiopia & AMHARA & 1980 & 244.29 & 193.42 & 303.66 & RW2 \\ 
  Ethiopia & AMHARA & 1981 & 239.11 & 202.97 & 279.57 & RW2 \\ 
  Ethiopia & AMHARA & 1982 & 233.89 & 201.05 & 269.76 & RW2 \\ 
  Ethiopia & AMHARA & 1983 & 228.97 & 194.65 & 267.70 & RW2 \\ 
  Ethiopia & AMHARA & 1984 & 224.26 & 188.41 & 265.28 & RW2 \\ 
  Ethiopia & AMHARA & 1985 & 219.64 & 186.97 & 256.33 & RW2 \\ 
  Ethiopia & AMHARA & 1986 & 215.81 & 186.50 & 247.90 & RW2 \\ 
  Ethiopia & AMHARA & 1987 & 212.51 & 185.11 & 242.78 & RW2 \\ 
  Ethiopia & AMHARA & 1988 & 209.62 & 181.42 & 240.31 & RW2 \\ 
  Ethiopia & AMHARA & 1989 & 206.93 & 176.82 & 239.37 & RW2 \\ 
  Ethiopia & AMHARA & 1990 & 204.77 & 175.88 & 237.35 & RW2 \\ 
  Ethiopia & AMHARA & 1991 & 201.37 & 174.86 & 231.40 & RW2 \\ 
  Ethiopia & AMHARA & 1992 & 197.04 & 171.85 & 224.33 & RW2 \\ 
  Ethiopia & AMHARA & 1993 & 191.74 & 166.31 & 219.86 & RW2 \\ 
  Ethiopia & AMHARA & 1994 & 185.70 & 159.08 & 216.04 & RW2 \\ 
  Ethiopia & AMHARA & 1995 & 178.71 & 152.61 & 207.41 & RW2 \\ 
  Ethiopia & AMHARA & 1996 & 172.26 & 148.69 & 198.15 & RW2 \\ 
  Ethiopia & AMHARA & 1997 & 166.15 & 144.37 & 190.25 & RW2 \\ 
  Ethiopia & AMHARA & 1998 & 160.24 & 138.31 & 184.91 & RW2 \\ 
  Ethiopia & AMHARA & 1999 & 154.66 & 131.22 & 181.47 & RW2 \\ 
  Ethiopia & AMHARA & 2000 & 149.48 & 126.76 & 175.90 & RW2 \\ 
  Ethiopia & AMHARA & 2001 & 143.54 & 122.82 & 167.10 & RW2 \\ 
  Ethiopia & AMHARA & 2002 & 137.17 & 117.72 & 159.41 & RW2 \\ 
  Ethiopia & AMHARA & 2003 & 130.23 & 110.98 & 152.36 & RW2 \\ 
  Ethiopia & AMHARA & 2004 & 123.04 & 103.31 & 146.10 & RW2 \\ 
  Ethiopia & AMHARA & 2005 & 115.32 & 95.96 & 137.42 & RW2 \\ 
  Ethiopia & AMHARA & 2006 & 108.08 & 90.72 & 128.10 & RW2 \\ 
  Ethiopia & AMHARA & 2007 & 100.98 & 84.85 & 119.75 & RW2 \\ 
  Ethiopia & AMHARA & 2008 & 94.36 & 78.39 & 113.13 & RW2 \\ 
  Ethiopia & AMHARA & 2009 & 88.36 & 71.72 & 108.30 & RW2 \\ 
  Ethiopia & AMHARA & 2010 & 82.74 & 65.57 & 103.39 & RW2 \\ 
  Ethiopia & AMHARA & 2011 & 77.45 & 61.58 & 96.68 & RW2 \\ 
  Ethiopia & AMHARA & 2012 & 72.53 & 57.32 & 90.47 & RW2 \\ 
  Ethiopia & AMHARA & 2013 & 67.92 & 51.88 & 86.68 & RW2 \\ 
  Ethiopia & AMHARA & 2014 & 63.58 & 44.80 & 87.56 & RW2 \\ 
  Ethiopia & AMHARA & 2015 & 59.37 & 36.57 & 94.37 & RW2 \\ 
  Ethiopia & AMHARA & 2016 & 55.43 & 29.00 & 102.09 & RW2 \\ 
  Ethiopia & AMHARA & 2017 & 51.88 & 22.59 & 113.17 & RW2 \\ 
  Ethiopia & AMHARA & 2018 & 48.18 & 17.22 & 128.20 & RW2 \\ 
  Ethiopia & AMHARA & 2019 & 44.98 & 12.78 & 145.95 & RW2 \\ 
  Ethiopia & BENISHANGUL-GUMUZ & 1980 & 279.93 & 215.62 & 354.51 & RW2 \\ 
  Ethiopia & BENISHANGUL-GUMUZ & 1981 & 274.94 & 225.34 & 331.55 & RW2 \\ 
  Ethiopia & BENISHANGUL-GUMUZ & 1982 & 270.42 & 225.60 & 320.25 & RW2 \\ 
  Ethiopia & BENISHANGUL-GUMUZ & 1983 & 265.78 & 221.50 & 316.44 & RW2 \\ 
  Ethiopia & BENISHANGUL-GUMUZ & 1984 & 261.36 & 216.60 & 311.97 & RW2 \\ 
  Ethiopia & BENISHANGUL-GUMUZ & 1985 & 257.09 & 216.26 & 302.38 & RW2 \\ 
  Ethiopia & BENISHANGUL-GUMUZ & 1986 & 253.68 & 217.36 & 294.06 & RW2 \\ 
  Ethiopia & BENISHANGUL-GUMUZ & 1987 & 251.00 & 216.64 & 288.12 & RW2 \\ 
  Ethiopia & BENISHANGUL-GUMUZ & 1988 & 248.84 & 214.19 & 286.47 & RW2 \\ 
  Ethiopia & BENISHANGUL-GUMUZ & 1989 & 246.78 & 210.08 & 286.38 & RW2 \\ 
  Ethiopia & BENISHANGUL-GUMUZ & 1990 & 245.17 & 210.06 & 284.10 & RW2 \\ 
  Ethiopia & BENISHANGUL-GUMUZ & 1991 & 242.31 & 210.08 & 277.74 & RW2 \\ 
  Ethiopia & BENISHANGUL-GUMUZ & 1992 & 238.30 & 207.00 & 271.70 & RW2 \\ 
  Ethiopia & BENISHANGUL-GUMUZ & 1993 & 233.22 & 202.13 & 267.52 & RW2 \\ 
  Ethiopia & BENISHANGUL-GUMUZ & 1994 & 226.96 & 194.15 & 263.47 & RW2 \\ 
  Ethiopia & BENISHANGUL-GUMUZ & 1995 & 219.70 & 187.15 & 254.93 & RW2 \\ 
  Ethiopia & BENISHANGUL-GUMUZ & 1996 & 212.80 & 183.35 & 245.23 & RW2 \\ 
  Ethiopia & BENISHANGUL-GUMUZ & 1997 & 206.20 & 178.24 & 237.06 & RW2 \\ 
  Ethiopia & BENISHANGUL-GUMUZ & 1998 & 199.83 & 171.70 & 231.71 & RW2 \\ 
  Ethiopia & BENISHANGUL-GUMUZ & 1999 & 193.64 & 164.34 & 226.74 & RW2 \\ 
  Ethiopia & BENISHANGUL-GUMUZ & 2000 & 187.88 & 159.26 & 221.20 & RW2 \\ 
  Ethiopia & BENISHANGUL-GUMUZ & 2001 & 181.03 & 154.60 & 211.61 & RW2 \\ 
  Ethiopia & BENISHANGUL-GUMUZ & 2002 & 173.33 & 148.76 & 202.13 & RW2 \\ 
  Ethiopia & BENISHANGUL-GUMUZ & 2003 & 164.96 & 139.80 & 194.20 & RW2 \\ 
  Ethiopia & BENISHANGUL-GUMUZ & 2004 & 155.99 & 130.80 & 185.89 & RW2 \\ 
  Ethiopia & BENISHANGUL-GUMUZ & 2005 & 146.23 & 121.47 & 175.11 & RW2 \\ 
  Ethiopia & BENISHANGUL-GUMUZ & 2006 & 137.02 & 114.71 & 163.25 & RW2 \\ 
  Ethiopia & BENISHANGUL-GUMUZ & 2007 & 128.06 & 107.30 & 152.47 & RW2 \\ 
  Ethiopia & BENISHANGUL-GUMUZ & 2008 & 119.52 & 98.73 & 143.96 & RW2 \\ 
  Ethiopia & BENISHANGUL-GUMUZ & 2009 & 111.61 & 90.09 & 137.32 & RW2 \\ 
  Ethiopia & BENISHANGUL-GUMUZ & 2010 & 104.20 & 82.46 & 131.06 & RW2 \\ 
  Ethiopia & BENISHANGUL-GUMUZ & 2011 & 97.32 & 76.78 & 121.96 & RW2 \\ 
  Ethiopia & BENISHANGUL-GUMUZ & 2012 & 90.84 & 71.24 & 114.02 & RW2 \\ 
  Ethiopia & BENISHANGUL-GUMUZ & 2013 & 84.83 & 64.28 & 109.26 & RW2 \\ 
  Ethiopia & BENISHANGUL-GUMUZ & 2014 & 79.06 & 55.38 & 109.97 & RW2 \\ 
  Ethiopia & BENISHANGUL-GUMUZ & 2015 & 73.55 & 44.69 & 117.15 & RW2 \\ 
  Ethiopia & BENISHANGUL-GUMUZ & 2016 & 68.44 & 35.53 & 125.93 & RW2 \\ 
  Ethiopia & BENISHANGUL-GUMUZ & 2017 & 63.61 & 27.64 & 138.63 & RW2 \\ 
  Ethiopia & BENISHANGUL-GUMUZ & 2018 & 59.11 & 20.75 & 155.83 & RW2 \\ 
  Ethiopia & BENISHANGUL-GUMUZ & 2019 & 55.04 & 15.39 & 177.06 & RW2 \\ 
  Ethiopia & DIRE DAWA & 1980 & 230.30 & 171.06 & 301.31 & RW2 \\ 
  Ethiopia & DIRE DAWA & 1981 & 227.13 & 179.84 & 280.85 & RW2 \\ 
  Ethiopia & DIRE DAWA & 1982 & 223.65 & 181.78 & 271.45 & RW2 \\ 
  Ethiopia & DIRE DAWA & 1983 & 220.34 & 179.12 & 267.43 & RW2 \\ 
  Ethiopia & DIRE DAWA & 1984 & 217.24 & 176.26 & 264.64 & RW2 \\ 
  Ethiopia & DIRE DAWA & 1985 & 214.15 & 176.92 & 256.51 & RW2 \\ 
  Ethiopia & DIRE DAWA & 1986 & 211.73 & 179.09 & 249.43 & RW2 \\ 
  Ethiopia & DIRE DAWA & 1987 & 209.75 & 178.94 & 244.59 & RW2 \\ 
  Ethiopia & DIRE DAWA & 1988 & 208.24 & 176.42 & 243.62 & RW2 \\ 
  Ethiopia & DIRE DAWA & 1989 & 206.72 & 174.05 & 243.38 & RW2 \\ 
  Ethiopia & DIRE DAWA & 1990 & 205.44 & 173.93 & 241.56 & RW2 \\ 
  Ethiopia & DIRE DAWA & 1991 & 202.56 & 173.02 & 235.70 & RW2 \\ 
  Ethiopia & DIRE DAWA & 1992 & 198.48 & 170.40 & 230.06 & RW2 \\ 
  Ethiopia & DIRE DAWA & 1993 & 193.36 & 164.91 & 225.41 & RW2 \\ 
  Ethiopia & DIRE DAWA & 1994 & 187.15 & 158.16 & 220.94 & RW2 \\ 
  Ethiopia & DIRE DAWA & 1995 & 179.37 & 150.67 & 211.52 & RW2 \\ 
  Ethiopia & DIRE DAWA & 1996 & 172.23 & 146.56 & 201.29 & RW2 \\ 
  Ethiopia & DIRE DAWA & 1997 & 164.97 & 140.92 & 192.80 & RW2 \\ 
  Ethiopia & DIRE DAWA & 1998 & 158.17 & 134.18 & 185.83 & RW2 \\ 
  Ethiopia & DIRE DAWA & 1999 & 151.58 & 127.02 & 180.01 & RW2 \\ 
  Ethiopia & DIRE DAWA & 2000 & 145.35 & 121.73 & 173.21 & RW2 \\ 
  Ethiopia & DIRE DAWA & 2001 & 138.68 & 117.21 & 163.81 & RW2 \\ 
  Ethiopia & DIRE DAWA & 2002 & 131.70 & 111.50 & 154.87 & RW2 \\ 
  Ethiopia & DIRE DAWA & 2003 & 124.45 & 104.50 & 147.36 & RW2 \\ 
  Ethiopia & DIRE DAWA & 2004 & 117.18 & 97.37 & 140.44 & RW2 \\ 
  Ethiopia & DIRE DAWA & 2005 & 109.64 & 90.26 & 132.51 & RW2 \\ 
  Ethiopia & DIRE DAWA & 2006 & 102.74 & 85.39 & 122.97 & RW2 \\ 
  Ethiopia & DIRE DAWA & 2007 & 96.24 & 80.28 & 115.03 & RW2 \\ 
  Ethiopia & DIRE DAWA & 2008 & 90.28 & 74.23 & 109.38 & RW2 \\ 
  Ethiopia & DIRE DAWA & 2009 & 84.83 & 68.39 & 104.90 & RW2 \\ 
  Ethiopia & DIRE DAWA & 2010 & 79.89 & 63.13 & 101.10 & RW2 \\ 
  Ethiopia & DIRE DAWA & 2011 & 75.32 & 59.41 & 95.09 & RW2 \\ 
  Ethiopia & DIRE DAWA & 2012 & 71.01 & 56.04 & 89.43 & RW2 \\ 
  Ethiopia & DIRE DAWA & 2013 & 66.93 & 51.32 & 86.47 & RW2 \\ 
  Ethiopia & DIRE DAWA & 2014 & 63.03 & 44.67 & 87.99 & RW2 \\ 
  Ethiopia & DIRE DAWA & 2015 & 59.34 & 36.55 & 95.16 & RW2 \\ 
  Ethiopia & DIRE DAWA & 2016 & 56.00 & 29.13 & 103.96 & RW2 \\ 
  Ethiopia & DIRE DAWA & 2017 & 52.51 & 22.99 & 116.30 & RW2 \\ 
  Ethiopia & DIRE DAWA & 2018 & 49.39 & 17.67 & 131.49 & RW2 \\ 
  Ethiopia & DIRE DAWA & 2019 & 46.65 & 13.18 & 151.21 & RW2 \\ 
  Ethiopia & GAMBELA & 1980 & 281.25 & 215.98 & 356.22 & RW2 \\ 
  Ethiopia & GAMBELA & 1981 & 277.25 & 227.01 & 332.36 & RW2 \\ 
  Ethiopia & GAMBELA & 1982 & 272.82 & 227.81 & 322.13 & RW2 \\ 
  Ethiopia & GAMBELA & 1983 & 268.62 & 223.41 & 319.08 & RW2 \\ 
  Ethiopia & GAMBELA & 1984 & 264.65 & 219.16 & 315.92 & RW2 \\ 
  Ethiopia & GAMBELA & 1985 & 260.72 & 219.20 & 307.27 & RW2 \\ 
  Ethiopia & GAMBELA & 1986 & 257.52 & 219.70 & 298.79 & RW2 \\ 
  Ethiopia & GAMBELA & 1987 & 254.67 & 219.28 & 293.90 & RW2 \\ 
  Ethiopia & GAMBELA & 1988 & 252.05 & 215.89 & 291.37 & RW2 \\ 
  Ethiopia & GAMBELA & 1989 & 249.33 & 212.23 & 290.13 & RW2 \\ 
  Ethiopia & GAMBELA & 1990 & 246.99 & 210.72 & 287.70 & RW2 \\ 
  Ethiopia & GAMBELA & 1991 & 242.58 & 208.94 & 279.93 & RW2 \\ 
  Ethiopia & GAMBELA & 1992 & 236.56 & 204.52 & 272.34 & RW2 \\ 
  Ethiopia & GAMBELA & 1993 & 228.94 & 197.16 & 265.02 & RW2 \\ 
  Ethiopia & GAMBELA & 1994 & 219.97 & 187.81 & 257.31 & RW2 \\ 
  Ethiopia & GAMBELA & 1995 & 209.56 & 177.41 & 244.78 & RW2 \\ 
  Ethiopia & GAMBELA & 1996 & 199.35 & 171.01 & 231.65 & RW2 \\ 
  Ethiopia & GAMBELA & 1997 & 189.45 & 162.86 & 219.08 & RW2 \\ 
  Ethiopia & GAMBELA & 1998 & 179.69 & 153.25 & 209.22 & RW2 \\ 
  Ethiopia & GAMBELA & 1999 & 170.49 & 143.70 & 201.07 & RW2 \\ 
  Ethiopia & GAMBELA & 2000 & 161.96 & 135.64 & 191.67 & RW2 \\ 
  Ethiopia & GAMBELA & 2001 & 153.11 & 129.25 & 180.23 & RW2 \\ 
  Ethiopia & GAMBELA & 2002 & 144.17 & 122.08 & 169.23 & RW2 \\ 
  Ethiopia & GAMBELA & 2003 & 135.26 & 113.64 & 160.04 & RW2 \\ 
  Ethiopia & GAMBELA & 2004 & 126.50 & 104.60 & 151.80 & RW2 \\ 
  Ethiopia & GAMBELA & 2005 & 117.72 & 96.85 & 141.90 & RW2 \\ 
  Ethiopia & GAMBELA & 2006 & 109.82 & 90.99 & 131.61 & RW2 \\ 
  Ethiopia & GAMBELA & 2007 & 102.47 & 85.14 & 122.87 & RW2 \\ 
  Ethiopia & GAMBELA & 2008 & 95.77 & 78.78 & 115.99 & RW2 \\ 
  Ethiopia & GAMBELA & 2009 & 89.64 & 72.14 & 111.01 & RW2 \\ 
  Ethiopia & GAMBELA & 2010 & 84.23 & 66.42 & 106.24 & RW2 \\ 
  Ethiopia & GAMBELA & 2011 & 79.11 & 62.59 & 99.56 & RW2 \\ 
  Ethiopia & GAMBELA & 2012 & 74.29 & 58.81 & 93.58 & RW2 \\ 
  Ethiopia & GAMBELA & 2013 & 69.77 & 53.85 & 90.15 & RW2 \\ 
  Ethiopia & GAMBELA & 2014 & 65.50 & 46.86 & 91.22 & RW2 \\ 
  Ethiopia & GAMBELA & 2015 & 61.50 & 37.90 & 98.24 & RW2 \\ 
  Ethiopia & GAMBELA & 2016 & 57.81 & 30.56 & 107.04 & RW2 \\ 
  Ethiopia & GAMBELA & 2017 & 54.07 & 23.79 & 120.14 & RW2 \\ 
  Ethiopia & GAMBELA & 2018 & 50.76 & 18.22 & 135.21 & RW2 \\ 
  Ethiopia & GAMBELA & 2019 & 47.64 & 13.69 & 156.97 & RW2 \\ 
  Ethiopia & HARARI & 1980 & 263.93 & 193.26 & 348.42 & RW2 \\ 
  Ethiopia & HARARI & 1981 & 256.70 & 200.82 & 321.82 & RW2 \\ 
  Ethiopia & HARARI & 1982 & 249.56 & 199.98 & 305.38 & RW2 \\ 
  Ethiopia & HARARI & 1983 & 242.49 & 195.37 & 296.30 & RW2 \\ 
  Ethiopia & HARARI & 1984 & 235.92 & 190.84 & 288.38 & RW2 \\ 
  Ethiopia & HARARI & 1985 & 229.39 & 188.74 & 275.54 & RW2 \\ 
  Ethiopia & HARARI & 1986 & 224.04 & 188.17 & 264.94 & RW2 \\ 
  Ethiopia & HARARI & 1987 & 219.47 & 186.42 & 256.88 & RW2 \\ 
  Ethiopia & HARARI & 1988 & 215.37 & 182.33 & 251.95 & RW2 \\ 
  Ethiopia & HARARI & 1989 & 211.46 & 177.89 & 249.32 & RW2 \\ 
  Ethiopia & HARARI & 1990 & 208.01 & 175.69 & 244.55 & RW2 \\ 
  Ethiopia & HARARI & 1991 & 203.04 & 173.40 & 236.77 & RW2 \\ 
  Ethiopia & HARARI & 1992 & 196.51 & 168.66 & 227.52 & RW2 \\ 
  Ethiopia & HARARI & 1993 & 188.78 & 161.17 & 220.48 & RW2 \\ 
  Ethiopia & HARARI & 1994 & 180.02 & 151.94 & 212.63 & RW2 \\ 
  Ethiopia & HARARI & 1995 & 169.80 & 142.33 & 200.77 & RW2 \\ 
  Ethiopia & HARARI & 1996 & 160.32 & 135.73 & 187.99 & RW2 \\ 
  Ethiopia & HARARI & 1997 & 150.90 & 128.10 & 176.85 & RW2 \\ 
  Ethiopia & HARARI & 1998 & 142.03 & 119.53 & 167.58 & RW2 \\ 
  Ethiopia & HARARI & 1999 & 133.90 & 111.11 & 160.15 & RW2 \\ 
  Ethiopia & HARARI & 2000 & 126.44 & 104.60 & 151.79 & RW2 \\ 
  Ethiopia & HARARI & 2001 & 118.88 & 99.10 & 141.61 & RW2 \\ 
  Ethiopia & HARARI & 2002 & 111.58 & 92.70 & 132.70 & RW2 \\ 
  Ethiopia & HARARI & 2003 & 104.41 & 86.06 & 125.51 & RW2 \\ 
  Ethiopia & HARARI & 2004 & 97.58 & 79.34 & 118.75 & RW2 \\ 
  Ethiopia & HARARI & 2005 & 90.82 & 73.52 & 111.03 & RW2 \\ 
  Ethiopia & HARARI & 2006 & 84.99 & 69.35 & 103.42 & RW2 \\ 
  Ethiopia & HARARI & 2007 & 79.58 & 65.12 & 96.44 & RW2 \\ 
  Ethiopia & HARARI & 2008 & 74.78 & 60.55 & 92.07 & RW2 \\ 
  Ethiopia & HARARI & 2009 & 70.62 & 55.95 & 88.80 & RW2 \\ 
  Ethiopia & HARARI & 2010 & 66.91 & 51.95 & 86.09 & RW2 \\ 
  Ethiopia & HARARI & 2011 & 63.46 & 49.12 & 81.89 & RW2 \\ 
  Ethiopia & HARARI & 2012 & 60.16 & 46.37 & 78.51 & RW2 \\ 
  Ethiopia & HARARI & 2013 & 57.11 & 42.66 & 77.24 & RW2 \\ 
  Ethiopia & HARARI & 2014 & 54.22 & 37.32 & 79.30 & RW2 \\ 
  Ethiopia & HARARI & 2015 & 51.45 & 30.74 & 86.58 & RW2 \\ 
  Ethiopia & HARARI & 2016 & 48.74 & 24.98 & 95.58 & RW2 \\ 
  Ethiopia & HARARI & 2017 & 46.33 & 19.57 & 107.83 & RW2 \\ 
  Ethiopia & HARARI & 2018 & 43.84 & 15.07 & 122.97 & RW2 \\ 
  Ethiopia & HARARI & 2019 & 41.53 & 11.51 & 143.31 & RW2 \\ 
  Ethiopia & OROMIYA & 1980 & 245.61 & 195.61 & 303.62 & RW2 \\ 
  Ethiopia & OROMIYA & 1981 & 240.32 & 204.85 & 279.44 & RW2 \\ 
  Ethiopia & OROMIYA & 1982 & 235.00 & 203.54 & 269.72 & RW2 \\ 
  Ethiopia & OROMIYA & 1983 & 229.84 & 196.36 & 268.12 & RW2 \\ 
  Ethiopia & OROMIYA & 1984 & 224.93 & 189.71 & 265.37 & RW2 \\ 
  Ethiopia & OROMIYA & 1985 & 220.17 & 187.99 & 256.33 & RW2 \\ 
  Ethiopia & OROMIYA & 1986 & 216.31 & 187.29 & 248.07 & RW2 \\ 
  Ethiopia & OROMIYA & 1987 & 212.85 & 185.63 & 242.41 & RW2 \\ 
  Ethiopia & OROMIYA & 1988 & 210.10 & 181.69 & 240.57 & RW2 \\ 
  Ethiopia & OROMIYA & 1989 & 207.37 & 177.26 & 239.98 & RW2 \\ 
  Ethiopia & OROMIYA & 1990 & 205.07 & 176.44 & 237.29 & RW2 \\ 
  Ethiopia & OROMIYA & 1991 & 201.31 & 175.02 & 230.86 & RW2 \\ 
  Ethiopia & OROMIYA & 1992 & 196.54 & 171.57 & 223.99 & RW2 \\ 
  Ethiopia & OROMIYA & 1993 & 190.65 & 165.67 & 218.62 & RW2 \\ 
  Ethiopia & OROMIYA & 1994 & 183.71 & 157.60 & 213.56 & RW2 \\ 
  Ethiopia & OROMIYA & 1995 & 175.65 & 149.80 & 203.87 & RW2 \\ 
  Ethiopia & OROMIYA & 1996 & 168.08 & 144.97 & 193.32 & RW2 \\ 
  Ethiopia & OROMIYA & 1997 & 160.69 & 139.35 & 184.64 & RW2 \\ 
  Ethiopia & OROMIYA & 1998 & 153.66 & 132.78 & 177.55 & RW2 \\ 
  Ethiopia & OROMIYA & 1999 & 146.96 & 124.89 & 171.99 & RW2 \\ 
  Ethiopia & OROMIYA & 2000 & 140.69 & 119.45 & 165.12 & RW2 \\ 
  Ethiopia & OROMIYA & 2001 & 134.03 & 115.01 & 155.93 & RW2 \\ 
  Ethiopia & OROMIYA & 2002 & 127.06 & 109.51 & 146.87 & RW2 \\ 
  Ethiopia & OROMIYA & 2003 & 120.08 & 102.65 & 140.34 & RW2 \\ 
  Ethiopia & OROMIYA & 2004 & 112.82 & 95.01 & 133.70 & RW2 \\ 
  Ethiopia & OROMIYA & 2005 & 105.51 & 88.41 & 125.05 & RW2 \\ 
  Ethiopia & OROMIYA & 2006 & 98.74 & 83.55 & 116.07 & RW2 \\ 
  Ethiopia & OROMIYA & 2007 & 92.47 & 78.54 & 108.38 & RW2 \\ 
  Ethiopia & OROMIYA & 2008 & 86.64 & 72.87 & 102.69 & RW2 \\ 
  Ethiopia & OROMIYA & 2009 & 81.33 & 67.01 & 98.35 & RW2 \\ 
  Ethiopia & OROMIYA & 2010 & 76.49 & 61.98 & 94.34 & RW2 \\ 
  Ethiopia & OROMIYA & 2011 & 72.04 & 58.56 & 87.81 & RW2 \\ 
  Ethiopia & OROMIYA & 2012 & 67.78 & 55.81 & 82.23 & RW2 \\ 
  Ethiopia & OROMIYA & 2013 & 63.83 & 51.39 & 79.18 & RW2 \\ 
  Ethiopia & OROMIYA & 2014 & 60.02 & 44.45 & 80.60 & RW2 \\ 
  Ethiopia & OROMIYA & 2015 & 56.53 & 36.07 & 87.96 & RW2 \\ 
  Ethiopia & OROMIYA & 2016 & 53.26 & 28.94 & 96.41 & RW2 \\ 
  Ethiopia & OROMIYA & 2017 & 49.98 & 22.56 & 107.06 & RW2 \\ 
  Ethiopia & OROMIYA & 2018 & 46.86 & 17.36 & 121.31 & RW2 \\ 
  Ethiopia & OROMIYA & 2019 & 44.12 & 13.02 & 141.41 & RW2 \\ 
  Ethiopia & SNNP & 1980 & 252.39 & 196.23 & 317.16 & RW2 \\ 
  Ethiopia & SNNP & 1981 & 248.72 & 206.99 & 294.61 & RW2 \\ 
  Ethiopia & SNNP & 1982 & 245.23 & 208.18 & 286.06 & RW2 \\ 
  Ethiopia & SNNP & 1983 & 241.47 & 203.62 & 284.39 & RW2 \\ 
  Ethiopia & SNNP & 1984 & 238.10 & 199.87 & 281.94 & RW2 \\ 
  Ethiopia & SNNP & 1985 & 234.66 & 199.46 & 274.01 & RW2 \\ 
  Ethiopia & SNNP & 1986 & 231.88 & 200.31 & 266.71 & RW2 \\ 
  Ethiopia & SNNP & 1987 & 229.60 & 200.43 & 262.46 & RW2 \\ 
  Ethiopia & SNNP & 1988 & 227.42 & 196.80 & 261.14 & RW2 \\ 
  Ethiopia & SNNP & 1989 & 225.14 & 192.47 & 260.50 & RW2 \\ 
  Ethiopia & SNNP & 1990 & 223.11 & 192.13 & 257.92 & RW2 \\ 
  Ethiopia & SNNP & 1991 & 219.36 & 190.74 & 251.35 & RW2 \\ 
  Ethiopia & SNNP & 1992 & 214.24 & 187.47 & 243.96 & RW2 \\ 
  Ethiopia & SNNP & 1993 & 207.80 & 180.23 & 238.56 & RW2 \\ 
  Ethiopia & SNNP & 1994 & 200.23 & 171.99 & 232.20 & RW2 \\ 
  Ethiopia & SNNP & 1995 & 191.29 & 163.47 & 221.79 & RW2 \\ 
  Ethiopia & SNNP & 1996 & 182.88 & 158.27 & 210.63 & RW2 \\ 
  Ethiopia & SNNP & 1997 & 174.52 & 151.81 & 200.28 & RW2 \\ 
  Ethiopia & SNNP & 1998 & 166.63 & 143.75 & 192.36 & RW2 \\ 
  Ethiopia & SNNP & 1999 & 159.04 & 135.24 & 185.98 & RW2 \\ 
  Ethiopia & SNNP & 2000 & 151.91 & 129.19 & 178.41 & RW2 \\ 
  Ethiopia & SNNP & 2001 & 144.39 & 123.72 & 168.00 & RW2 \\ 
  Ethiopia & SNNP & 2002 & 136.82 & 117.73 & 158.62 & RW2 \\ 
  Ethiopia & SNNP & 2003 & 128.99 & 109.86 & 150.42 & RW2 \\ 
  Ethiopia & SNNP & 2004 & 121.25 & 101.85 & 143.68 & RW2 \\ 
  Ethiopia & SNNP & 2005 & 113.38 & 94.55 & 134.99 & RW2 \\ 
  Ethiopia & SNNP & 2006 & 106.28 & 89.32 & 125.55 & RW2 \\ 
  Ethiopia & SNNP & 2007 & 99.52 & 84.03 & 117.61 & RW2 \\ 
  Ethiopia & SNNP & 2008 & 93.44 & 77.80 & 111.73 & RW2 \\ 
  Ethiopia & SNNP & 2009 & 87.78 & 71.43 & 107.52 & RW2 \\ 
  Ethiopia & SNNP & 2010 & 82.82 & 66.01 & 103.65 & RW2 \\ 
  Ethiopia & SNNP & 2011 & 78.00 & 62.38 & 97.10 & RW2 \\ 
  Ethiopia & SNNP & 2012 & 73.54 & 59.05 & 91.87 & RW2 \\ 
  Ethiopia & SNNP & 2013 & 69.37 & 53.98 & 89.02 & RW2 \\ 
  Ethiopia & SNNP & 2014 & 65.40 & 46.87 & 90.67 & RW2 \\ 
  Ethiopia & SNNP & 2015 & 61.63 & 38.30 & 97.88 & RW2 \\ 
  Ethiopia & SNNP & 2016 & 58.07 & 30.61 & 107.55 & RW2 \\ 
  Ethiopia & SNNP & 2017 & 54.65 & 24.07 & 120.43 & RW2 \\ 
  Ethiopia & SNNP & 2018 & 51.32 & 18.65 & 135.83 & RW2 \\ 
  Ethiopia & SNNP & 2019 & 48.39 & 13.94 & 156.73 & RW2 \\ 
  Ethiopia & SOMALI & 1980 & 182.94 & 133.38 & 246.41 & RW2 \\ 
  Ethiopia & SOMALI & 1981 & 180.82 & 140.82 & 229.83 & RW2 \\ 
  Ethiopia & SOMALI & 1982 & 178.72 & 142.11 & 222.84 & RW2 \\ 
  Ethiopia & SOMALI & 1983 & 176.55 & 140.65 & 219.97 & RW2 \\ 
  Ethiopia & SOMALI & 1984 & 174.69 & 139.03 & 217.99 & RW2 \\ 
  Ethiopia & SOMALI & 1985 & 172.99 & 139.83 & 211.39 & RW2 \\ 
  Ethiopia & SOMALI & 1986 & 171.74 & 141.91 & 206.20 & RW2 \\ 
  Ethiopia & SOMALI & 1987 & 171.05 & 142.88 & 203.28 & RW2 \\ 
  Ethiopia & SOMALI & 1988 & 170.84 & 142.39 & 203.17 & RW2 \\ 
  Ethiopia & SOMALI & 1989 & 170.72 & 141.14 & 203.95 & RW2 \\ 
  Ethiopia & SOMALI & 1990 & 170.94 & 142.81 & 204.24 & RW2 \\ 
  Ethiopia & SOMALI & 1991 & 170.07 & 143.25 & 200.73 & RW2 \\ 
  Ethiopia & SOMALI & 1992 & 168.13 & 142.72 & 196.82 & RW2 \\ 
  Ethiopia & SOMALI & 1993 & 165.25 & 139.98 & 194.25 & RW2 \\ 
  Ethiopia & SOMALI & 1994 & 161.39 & 135.40 & 191.57 & RW2 \\ 
  Ethiopia & SOMALI & 1995 & 156.42 & 131.34 & 184.95 & RW2 \\ 
  Ethiopia & SOMALI & 1996 & 151.87 & 128.93 & 177.93 & RW2 \\ 
  Ethiopia & SOMALI & 1997 & 147.51 & 126.13 & 171.91 & RW2 \\ 
  Ethiopia & SOMALI & 1998 & 143.37 & 121.59 & 167.96 & RW2 \\ 
  Ethiopia & SOMALI & 1999 & 139.33 & 116.85 & 164.74 & RW2 \\ 
  Ethiopia & SOMALI & 2000 & 135.82 & 114.27 & 160.99 & RW2 \\ 
  Ethiopia & SOMALI & 2001 & 131.64 & 111.58 & 154.76 & RW2 \\ 
  Ethiopia & SOMALI & 2002 & 127.22 & 108.53 & 148.79 & RW2 \\ 
  Ethiopia & SOMALI & 2003 & 122.26 & 103.35 & 143.88 & RW2 \\ 
  Ethiopia & SOMALI & 2004 & 117.08 & 97.92 & 139.58 & RW2 \\ 
  Ethiopia & SOMALI & 2005 & 111.48 & 92.64 & 133.22 & RW2 \\ 
  Ethiopia & SOMALI & 2006 & 106.33 & 89.35 & 125.99 & RW2 \\ 
  Ethiopia & SOMALI & 2007 & 101.40 & 85.53 & 119.83 & RW2 \\ 
  Ethiopia & SOMALI & 2008 & 96.85 & 80.88 & 115.50 & RW2 \\ 
  Ethiopia & SOMALI & 2009 & 92.65 & 75.85 & 112.59 & RW2 \\ 
  Ethiopia & SOMALI & 2010 & 88.81 & 71.58 & 109.90 & RW2 \\ 
  Ethiopia & SOMALI & 2011 & 85.24 & 69.09 & 104.79 & RW2 \\ 
  Ethiopia & SOMALI & 2012 & 81.77 & 66.92 & 99.73 & RW2 \\ 
  Ethiopia & SOMALI & 2013 & 78.50 & 62.71 & 98.10 & RW2 \\ 
  Ethiopia & SOMALI & 2014 & 75.30 & 55.65 & 101.26 & RW2 \\ 
  Ethiopia & SOMALI & 2015 & 72.28 & 46.06 & 111.81 & RW2 \\ 
  Ethiopia & SOMALI & 2016 & 69.25 & 37.54 & 124.38 & RW2 \\ 
  Ethiopia & SOMALI & 2017 & 66.47 & 29.93 & 143.18 & RW2 \\ 
  Ethiopia & SOMALI & 2018 & 63.77 & 23.36 & 163.56 & RW2 \\ 
  Ethiopia & SOMALI & 2019 & 61.25 & 18.14 & 188.46 & RW2 \\ 
  Ethiopia & TIGRAY & 1980 & 277.45 & 217.68 & 345.51 & RW2 \\ 
  Ethiopia & TIGRAY & 1981 & 268.43 & 225.16 & 315.68 & RW2 \\ 
  Ethiopia & TIGRAY & 1982 & 259.36 & 221.49 & 300.87 & RW2 \\ 
  Ethiopia & TIGRAY & 1983 & 250.48 & 211.76 & 294.02 & RW2 \\ 
  Ethiopia & TIGRAY & 1984 & 242.01 & 203.12 & 286.13 & RW2 \\ 
  Ethiopia & TIGRAY & 1985 & 233.64 & 198.14 & 272.88 & RW2 \\ 
  Ethiopia & TIGRAY & 1986 & 226.26 & 194.91 & 260.79 & RW2 \\ 
  Ethiopia & TIGRAY & 1987 & 219.41 & 190.68 & 251.06 & RW2 \\ 
  Ethiopia & TIGRAY & 1988 & 212.93 & 183.40 & 245.18 & RW2 \\ 
  Ethiopia & TIGRAY & 1989 & 206.58 & 175.83 & 240.21 & RW2 \\ 
  Ethiopia & TIGRAY & 1990 & 200.60 & 171.59 & 233.90 & RW2 \\ 
  Ethiopia & TIGRAY & 1991 & 193.45 & 167.04 & 223.41 & RW2 \\ 
  Ethiopia & TIGRAY & 1992 & 185.41 & 160.51 & 213.13 & RW2 \\ 
  Ethiopia & TIGRAY & 1993 & 176.46 & 152.10 & 203.98 & RW2 \\ 
  Ethiopia & TIGRAY & 1994 & 166.92 & 142.01 & 195.82 & RW2 \\ 
  Ethiopia & TIGRAY & 1995 & 156.68 & 132.63 & 183.42 & RW2 \\ 
  Ethiopia & TIGRAY & 1996 & 147.32 & 125.87 & 171.28 & RW2 \\ 
  Ethiopia & TIGRAY & 1997 & 138.49 & 119.03 & 160.37 & RW2 \\ 
  Ethiopia & TIGRAY & 1998 & 130.31 & 111.29 & 152.38 & RW2 \\ 
  Ethiopia & TIGRAY & 1999 & 122.75 & 103.16 & 145.44 & RW2 \\ 
  Ethiopia & TIGRAY & 2000 & 115.95 & 97.10 & 137.67 & RW2 \\ 
  Ethiopia & TIGRAY & 2001 & 109.04 & 92.37 & 128.29 & RW2 \\ 
  Ethiopia & TIGRAY & 2002 & 102.26 & 86.87 & 119.89 & RW2 \\ 
  Ethiopia & TIGRAY & 2003 & 95.44 & 80.51 & 112.82 & RW2 \\ 
  Ethiopia & TIGRAY & 2004 & 88.82 & 73.92 & 106.35 & RW2 \\ 
  Ethiopia & TIGRAY & 2005 & 82.25 & 67.86 & 99.12 & RW2 \\ 
  Ethiopia & TIGRAY & 2006 & 76.21 & 63.47 & 91.18 & RW2 \\ 
  Ethiopia & TIGRAY & 2007 & 70.59 & 59.04 & 84.16 & RW2 \\ 
  Ethiopia & TIGRAY & 2008 & 65.47 & 54.20 & 78.93 & RW2 \\ 
  Ethiopia & TIGRAY & 2009 & 60.73 & 49.33 & 74.66 & RW2 \\ 
  Ethiopia & TIGRAY & 2010 & 56.55 & 44.90 & 71.00 & RW2 \\ 
  Ethiopia & TIGRAY & 2011 & 52.51 & 42.00 & 65.70 & RW2 \\ 
  Ethiopia & TIGRAY & 2012 & 48.87 & 39.19 & 60.87 & RW2 \\ 
  Ethiopia & TIGRAY & 2013 & 45.43 & 35.39 & 58.23 & RW2 \\ 
  Ethiopia & TIGRAY & 2014 & 42.24 & 30.30 & 58.42 & RW2 \\ 
  Ethiopia & TIGRAY & 2015 & 39.26 & 24.33 & 63.10 & RW2 \\ 
  Ethiopia & TIGRAY & 2016 & 36.46 & 19.30 & 68.52 & RW2 \\ 
  Ethiopia & TIGRAY & 2017 & 33.89 & 14.84 & 76.62 & RW2 \\ 
  Ethiopia & TIGRAY & 2018 & 31.39 & 11.04 & 86.55 & RW2 \\ 
  Ethiopia & TIGRAY & 2019 & 29.11 & 8.26 & 98.99 & RW2 \\ 
  Gabon & ALL & 1980 & 113.20 & 106.00 & 120.94 & IHME \\ 
  Gabon & ALL & 1980 & 118.13 & 80.31 & 170.99 & RW2 \\ 
  Gabon & ALL & 1980 & 117.30 & 97.50 & 141.50 & UN \\ 
  Gabon & ALL & 1981 & 109.24 & 102.74 & 116.37 & IHME \\ 
  Gabon & ALL & 1981 & 114.16 & 85.72 & 150.32 & RW2 \\ 
  Gabon & ALL & 1981 & 113.20 & 95.10 & 134.60 & UN \\ 
  Gabon & ALL & 1982 & 105.77 & 99.66 & 112.18 & IHME \\ 
  Gabon & ALL & 1982 & 110.34 & 86.11 & 140.62 & RW2 \\ 
  Gabon & ALL & 1982 & 109.30 & 92.50 & 128.90 & UN \\ 
  Gabon & ALL & 1983 & 102.17 & 96.15 & 107.99 & IHME \\ 
  Gabon & ALL & 1983 & 106.51 & 82.33 & 137.29 & RW2 \\ 
  Gabon & ALL & 1983 & 106.00 & 90.10 & 124.40 & UN \\ 
  Gabon & ALL & 1984 & 98.45 & 92.95 & 104.03 & IHME \\ 
  Gabon & ALL & 1984 & 103.29 & 78.06 & 135.64 & RW2 \\ 
  Gabon & ALL & 1984 & 103.00 & 88.00 & 120.40 & UN \\ 
  Gabon & ALL & 1985 & 94.98 & 89.57 & 100.25 & IHME \\ 
  Gabon & ALL & 1985 & 99.83 & 76.22 & 128.97 & RW2 \\ 
  Gabon & ALL & 1985 & 100.40 & 85.90 & 117.20 & UN \\ 
  Gabon & ALL & 1986 & 91.78 & 86.84 & 96.95 & IHME \\ 
  Gabon & ALL & 1986 & 97.41 & 75.34 & 124.37 & RW2 \\ 
  Gabon & ALL & 1986 & 98.20 & 84.30 & 114.40 & UN \\ 
  Gabon & ALL & 1987 & 89.43 & 84.72 & 94.33 & IHME \\ 
  Gabon & ALL & 1987 & 95.64 & 74.80 & 121.32 & RW2 \\ 
  Gabon & ALL & 1987 & 96.50 & 82.90 & 112.10 & UN \\ 
  Gabon & ALL & 1988 & 87.47 & 83.10 & 92.21 & IHME \\ 
  Gabon & ALL & 1988 & 94.37 & 73.23 & 120.03 & RW2 \\ 
  Gabon & ALL & 1988 & 95.10 & 82.10 & 110.10 & UN \\ 
  Gabon & ALL & 1989 & 85.34 & 81.13 & 89.90 & IHME \\ 
  Gabon & ALL & 1989 & 93.61 & 72.11 & 119.82 & RW2 \\ 
  Gabon & ALL & 1989 & 93.90 & 81.50 & 108.70 & UN \\ 
  Gabon & ALL & 1990 & 83.27 & 79.32 & 87.80 & IHME \\ 
  Gabon & ALL & 1990 & 93.54 & 72.84 & 120.20 & RW2 \\ 
  Gabon & ALL & 1990 & 92.90 & 80.80 & 107.50 & UN \\ 
  Gabon & ALL & 1991 & 81.41 & 77.53 & 85.91 & IHME \\ 
  Gabon & ALL & 1991 & 93.24 & 73.40 & 117.89 & RW2 \\ 
  Gabon & ALL & 1991 & 92.00 & 80.10 & 106.60 & UN \\ 
  Gabon & ALL & 1992 & 79.64 & 75.80 & 83.77 & IHME \\ 
  Gabon & ALL & 1992 & 92.90 & 73.43 & 116.87 & RW2 \\ 
  Gabon & ALL & 1992 & 91.30 & 79.60 & 105.70 & UN \\ 
  Gabon & ALL & 1993 & 78.13 & 74.33 & 82.16 & IHME \\ 
  Gabon & ALL & 1993 & 92.43 & 72.72 & 117.02 & RW2 \\ 
  Gabon & ALL & 1993 & 90.60 & 79.10 & 104.90 & UN \\ 
  Gabon & ALL & 1994 & 76.74 & 72.99 & 80.71 & IHME \\ 
  Gabon & ALL & 1994 & 91.72 & 71.26 & 118.03 & RW2 \\ 
  Gabon & ALL & 1994 & 89.80 & 78.50 & 104.20 & UN \\ 
  Gabon & ALL & 1995 & 75.36 & 71.71 & 79.30 & IHME \\ 
  Gabon & ALL & 1995 & 90.75 & 70.02 & 116.31 & RW2 \\ 
  Gabon & ALL & 1995 & 89.10 & 77.90 & 103.70 & UN \\ 
  Gabon & ALL & 1996 & 74.63 & 71.00 & 78.45 & IHME \\ 
  Gabon & ALL & 1996 & 89.80 & 69.90 & 114.69 & RW2 \\ 
  Gabon & ALL & 1996 & 88.50 & 77.30 & 103.10 & UN \\ 
  Gabon & ALL & 1997 & 73.34 & 69.78 & 77.19 & IHME \\ 
  Gabon & ALL & 1997 & 88.85 & 69.53 & 112.75 & RW2 \\ 
  Gabon & ALL & 1997 & 87.90 & 76.60 & 102.30 & UN \\ 
  Gabon & ALL & 1998 & 72.43 & 68.75 & 76.37 & IHME \\ 
  Gabon & ALL & 1998 & 87.93 & 68.16 & 113.13 & RW2 \\ 
  Gabon & ALL & 1998 & 87.20 & 76.00 & 101.50 & UN \\ 
  Gabon & ALL & 1999 & 71.78 & 68.13 & 75.69 & IHME \\ 
  Gabon & ALL & 1999 & 86.87 & 65.96 & 113.34 & RW2 \\ 
  Gabon & ALL & 1999 & 86.30 & 75.20 & 100.70 & UN \\ 
  Gabon & ALL & 2000 & 71.39 & 67.71 & 75.29 & IHME \\ 
  Gabon & ALL & 2000 & 85.85 & 63.70 & 115.69 & RW2 \\ 
  Gabon & ALL & 2000 & 85.30 & 74.10 & 100.00 & UN \\ 
  Gabon & ALL & 2001 & 70.97 & 67.20 & 75.08 & IHME \\ 
  Gabon & ALL & 2001 & 84.40 & 61.79 & 115.30 & RW2 \\ 
  Gabon & ALL & 2001 & 84.00 & 72.90 & 98.80 & UN \\ 
  Gabon & ALL & 2002 & 70.22 & 66.39 & 74.40 & IHME \\ 
  Gabon & ALL & 2002 & 82.61 & 60.22 & 113.53 & RW2 \\ 
  Gabon & ALL & 2002 & 82.50 & 71.60 & 97.10 & UN \\ 
  Gabon & ALL & 2003 & 69.14 & 65.13 & 73.28 & IHME \\ 
  Gabon & ALL & 2003 & 80.62 & 58.72 & 110.73 & RW2 \\ 
  Gabon & ALL & 2003 & 80.80 & 69.90 & 95.00 & UN \\ 
  Gabon & ALL & 2004 & 68.06 & 63.89 & 72.19 & IHME \\ 
  Gabon & ALL & 2004 & 78.10 & 56.76 & 107.33 & RW2 \\ 
  Gabon & ALL & 2004 & 78.80 & 68.00 & 92.40 & UN \\ 
  Gabon & ALL & 2005 & 66.78 & 62.57 & 70.99 & IHME \\ 
  Gabon & ALL & 2005 & 75.48 & 55.41 & 102.08 & RW2 \\ 
  Gabon & ALL & 2005 & 76.70 & 66.00 & 89.60 & UN \\ 
  Gabon & ALL & 2006 & 65.45 & 61.12 & 69.91 & IHME \\ 
  Gabon & ALL & 2006 & 72.81 & 53.99 & 97.32 & RW2 \\ 
  Gabon & ALL & 2006 & 74.50 & 63.90 & 86.70 & UN \\ 
  Gabon & ALL & 2007 & 63.63 & 59.18 & 68.35 & IHME \\ 
  Gabon & ALL & 2007 & 70.19 & 52.37 & 93.31 & RW2 \\ 
  Gabon & ALL & 2007 & 71.90 & 61.50 & 83.50 & UN \\ 
  Gabon & ALL & 2008 & 61.32 & 56.73 & 66.11 & IHME \\ 
  Gabon & ALL & 2008 & 67.77 & 50.31 & 90.50 & RW2 \\ 
  Gabon & ALL & 2008 & 68.90 & 58.70 & 80.70 & UN \\ 
  Gabon & ALL & 2009 & 59.39 & 54.78 & 64.53 & IHME \\ 
  Gabon & ALL & 2009 & 65.24 & 48.14 & 87.86 & RW2 \\ 
  Gabon & ALL & 2009 & 66.10 & 55.60 & 78.10 & UN \\ 
  Gabon & ALL & 2010 & 57.50 & 52.58 & 62.80 & IHME \\ 
  Gabon & ALL & 2010 & 62.92 & 46.72 & 84.61 & RW2 \\ 
  Gabon & ALL & 2010 & 63.30 & 52.30 & 76.00 & UN \\ 
  Gabon & ALL & 2011 & 55.08 & 49.94 & 60.55 & IHME \\ 
  Gabon & ALL & 2011 & 60.71 & 45.94 & 79.77 & RW2 \\ 
  Gabon & ALL & 2011 & 60.60 & 48.80 & 74.20 & UN \\ 
  Gabon & ALL & 2012 & 52.33 & 46.96 & 58.09 & IHME \\ 
  Gabon & ALL & 2012 & 58.53 & 44.87 & 75.93 & RW2 \\ 
  Gabon & ALL & 2012 & 57.70 & 45.10 & 72.30 & UN \\ 
  Gabon & ALL & 2013 & 49.87 & 44.42 & 55.98 & IHME \\ 
  Gabon & ALL & 2013 & 56.48 & 41.74 & 75.86 & RW2 \\ 
  Gabon & ALL & 2013 & 54.80 & 41.50 & 70.60 & UN \\ 
  Gabon & ALL & 2014 & 47.53 & 41.85 & 53.83 & IHME \\ 
  Gabon & ALL & 2014 & 54.42 & 35.86 & 81.81 & RW2 \\ 
  Gabon & ALL & 2014 & 52.30 & 38.40 & 69.50 & UN \\ 
  Gabon & ALL & 2015 & 46.11 & 40.23 & 52.69 & IHME \\ 
  Gabon & ALL & 2015 & 52.36 & 28.38 & 95.15 & RW2 \\ 
  Gabon & ALL & 2015 & 50.80 & 35.90 & 70.10 & UN \\ 
  Gabon & ALL & 2016 & 50.59 & 22.41 & 111.76 & RW2 \\ 
  Gabon & ALL & 2017 & 48.61 & 17.01 & 133.64 & RW2 \\ 
  Gabon & ALL & 2018 & 46.82 & 12.78 & 163.40 & RW2 \\ 
  Gabon & ALL & 2019 & 45.01 & 8.99 & 197.28 & RW2 \\ 
  Gabon & ALL & 80-84 & 105.93 & 121.81 & 91.90 & HT-Direct \\ 
  Gabon & ALL & 85-89 & 82.19 & 92.07 & 73.28 & HT-Direct \\ 
  Gabon & ALL & 90-94 & 89.79 & 99.87 & 80.62 & HT-Direct \\ 
  Gabon & ALL & 95-99 & 77.78 & 87.16 & 69.33 & HT-Direct \\ 
  Gabon & ALL & 00-04 & 63.12 & 77.25 & 51.43 & HT-Direct \\ 
  Gabon & ALL & 05-09 & 54.35 & 63.32 & 46.59 & HT-Direct \\ 
  Gabon & ALL & 10-14 & 81.07 & 98.47 & 66.52 & HT-Direct \\ 
  Gabon & ALL & 15-19 & 48.61 & 17.33 & 130.99 & RW2 \\ 
  Gabon & EAST & 1980 & 80.46 & 50.50 & 126.53 & RW2 \\ 
  Gabon & EAST & 1981 & 77.85 & 53.36 & 112.83 & RW2 \\ 
  Gabon & EAST & 1982 & 75.11 & 53.74 & 104.24 & RW2 \\ 
  Gabon & EAST & 1983 & 72.51 & 52.40 & 99.73 & RW2 \\ 
  Gabon & EAST & 1984 & 69.98 & 50.51 & 96.19 & RW2 \\ 
  Gabon & EAST & 1985 & 67.66 & 49.83 & 91.42 & RW2 \\ 
  Gabon & EAST & 1986 & 65.41 & 49.04 & 86.59 & RW2 \\ 
  Gabon & EAST & 1987 & 63.30 & 48.18 & 82.87 & RW2 \\ 
  Gabon & EAST & 1988 & 61.45 & 46.93 & 80.25 & RW2 \\ 
  Gabon & EAST & 1989 & 59.94 & 45.88 & 78.18 & RW2 \\ 
  Gabon & EAST & 1990 & 58.67 & 45.15 & 75.57 & RW2 \\ 
  Gabon & EAST & 1991 & 58.35 & 45.66 & 73.99 & RW2 \\ 
  Gabon & EAST & 1992 & 58.68 & 46.23 & 73.86 & RW2 \\ 
  Gabon & EAST & 1993 & 59.75 & 46.93 & 75.45 & RW2 \\ 
  Gabon & EAST & 1994 & 61.67 & 47.78 & 78.51 & RW2 \\ 
  Gabon & EAST & 1995 & 64.33 & 49.95 & 82.61 & RW2 \\ 
  Gabon & EAST & 1996 & 67.23 & 52.60 & 85.62 & RW2 \\ 
  Gabon & EAST & 1997 & 70.26 & 55.23 & 89.33 & RW2 \\ 
  Gabon & EAST & 1998 & 73.25 & 57.28 & 93.66 & RW2 \\ 
  Gabon & EAST & 1999 & 75.91 & 57.96 & 98.68 & RW2 \\ 
  Gabon & EAST & 2000 & 78.06 & 58.70 & 103.88 & RW2 \\ 
  Gabon & EAST & 2001 & 79.41 & 59.22 & 106.46 & RW2 \\ 
  Gabon & EAST & 2002 & 79.75 & 59.27 & 107.57 & RW2 \\ 
  Gabon & EAST & 2003 & 79.06 & 58.56 & 107.39 & RW2 \\ 
  Gabon & EAST & 2004 & 77.60 & 57.09 & 105.67 & RW2 \\ 
  Gabon & EAST & 2005 & 75.12 & 55.25 & 101.66 & RW2 \\ 
  Gabon & EAST & 2006 & 72.59 & 53.64 & 97.81 & RW2 \\ 
  Gabon & EAST & 2007 & 69.81 & 51.19 & 94.28 & RW2 \\ 
  Gabon & EAST & 2008 & 67.09 & 48.56 & 91.72 & RW2 \\ 
  Gabon & EAST & 2009 & 64.29 & 45.93 & 89.02 & RW2 \\ 
  Gabon & EAST & 2010 & 61.82 & 43.95 & 86.33 & RW2 \\ 
  Gabon & EAST & 2011 & 59.26 & 42.28 & 82.57 & RW2 \\ 
  Gabon & EAST & 2012 & 56.87 & 40.40 & 79.50 & RW2 \\ 
  Gabon & EAST & 2013 & 54.60 & 37.21 & 79.59 & RW2 \\ 
  Gabon & EAST & 2014 & 52.41 & 32.40 & 83.37 & RW2 \\ 
  Gabon & EAST & 2015 & 50.27 & 26.42 & 93.40 & RW2 \\ 
  Gabon & EAST & 2016 & 48.10 & 21.05 & 106.29 & RW2 \\ 
  Gabon & EAST & 2017 & 46.19 & 16.35 & 124.58 & RW2 \\ 
  Gabon & EAST & 2018 & 44.12 & 12.28 & 148.81 & RW2 \\ 
  Gabon & EAST & 2019 & 42.31 & 9.06 & 181.16 & RW2 \\ 
  Gabon & LIBREVILLE,PORT-GENTIL & 1980 & 118.66 & 75.62 & 176.84 & RW2 \\ 
  Gabon & LIBREVILLE,PORT-GENTIL & 1981 & 116.15 & 81.99 & 158.76 & RW2 \\ 
  Gabon & LIBREVILLE,PORT-GENTIL & 1982 & 113.59 & 84.27 & 149.60 & RW2 \\ 
  Gabon & LIBREVILLE,PORT-GENTIL & 1983 & 110.78 & 83.64 & 144.93 & RW2 \\ 
  Gabon & LIBREVILLE,PORT-GENTIL & 1984 & 108.39 & 82.13 & 141.31 & RW2 \\ 
  Gabon & LIBREVILLE,PORT-GENTIL & 1985 & 105.93 & 81.70 & 136.33 & RW2 \\ 
  Gabon & LIBREVILLE,PORT-GENTIL & 1986 & 103.45 & 81.23 & 131.01 & RW2 \\ 
  Gabon & LIBREVILLE,PORT-GENTIL & 1987 & 101.07 & 80.13 & 127.10 & RW2 \\ 
  Gabon & LIBREVILLE,PORT-GENTIL & 1988 & 98.65 & 78.26 & 124.21 & RW2 \\ 
  Gabon & LIBREVILLE,PORT-GENTIL & 1989 & 96.51 & 76.27 & 122.62 & RW2 \\ 
  Gabon & LIBREVILLE,PORT-GENTIL & 1990 & 94.25 & 74.34 & 119.01 & RW2 \\ 
  Gabon & LIBREVILLE,PORT-GENTIL & 1991 & 92.83 & 74.29 & 116.29 & RW2 \\ 
  Gabon & LIBREVILLE,PORT-GENTIL & 1992 & 92.04 & 73.96 & 114.78 & RW2 \\ 
  Gabon & LIBREVILLE,PORT-GENTIL & 1993 & 91.71 & 72.95 & 115.30 & RW2 \\ 
  Gabon & LIBREVILLE,PORT-GENTIL & 1994 & 91.94 & 72.16 & 116.63 & RW2 \\ 
  Gabon & LIBREVILLE,PORT-GENTIL & 1995 & 92.66 & 72.87 & 118.20 & RW2 \\ 
  Gabon & LIBREVILLE,PORT-GENTIL & 1996 & 93.30 & 73.55 & 118.09 & RW2 \\ 
  Gabon & LIBREVILLE,PORT-GENTIL & 1997 & 93.82 & 73.85 & 118.69 & RW2 \\ 
  Gabon & LIBREVILLE,PORT-GENTIL & 1998 & 94.12 & 72.77 & 120.37 & RW2 \\ 
  Gabon & LIBREVILLE,PORT-GENTIL & 1999 & 93.69 & 70.30 & 122.66 & RW2 \\ 
  Gabon & LIBREVILLE,PORT-GENTIL & 2000 & 93.07 & 68.11 & 124.46 & RW2 \\ 
  Gabon & LIBREVILLE,PORT-GENTIL & 2001 & 91.36 & 65.86 & 123.67 & RW2 \\ 
  Gabon & LIBREVILLE,PORT-GENTIL & 2002 & 88.98 & 63.48 & 121.36 & RW2 \\ 
  Gabon & LIBREVILLE,PORT-GENTIL & 2003 & 85.67 & 60.39 & 117.33 & RW2 \\ 
  Gabon & LIBREVILLE,PORT-GENTIL & 2004 & 81.89 & 57.57 & 112.84 & RW2 \\ 
  Gabon & LIBREVILLE,PORT-GENTIL & 2005 & 77.33 & 54.72 & 105.27 & RW2 \\ 
  Gabon & LIBREVILLE,PORT-GENTIL & 2006 & 73.14 & 52.19 & 98.74 & RW2 \\ 
  Gabon & LIBREVILLE,PORT-GENTIL & 2007 & 69.11 & 49.85 & 92.89 & RW2 \\ 
  Gabon & LIBREVILLE,PORT-GENTIL & 2008 & 65.26 & 47.26 & 88.43 & RW2 \\ 
  Gabon & LIBREVILLE,PORT-GENTIL & 2009 & 61.83 & 44.56 & 84.22 & RW2 \\ 
  Gabon & LIBREVILLE,PORT-GENTIL & 2010 & 58.72 & 42.25 & 80.53 & RW2 \\ 
  Gabon & LIBREVILLE,PORT-GENTIL & 2011 & 55.83 & 40.85 & 75.87 & RW2 \\ 
  Gabon & LIBREVILLE,PORT-GENTIL & 2012 & 53.10 & 38.86 & 72.01 & RW2 \\ 
  Gabon & LIBREVILLE,PORT-GENTIL & 2013 & 50.50 & 35.42 & 71.54 & RW2 \\ 
  Gabon & LIBREVILLE,PORT-GENTIL & 2014 & 47.99 & 30.41 & 75.41 & RW2 \\ 
  Gabon & LIBREVILLE,PORT-GENTIL & 2015 & 45.66 & 24.30 & 85.07 & RW2 \\ 
  Gabon & LIBREVILLE,PORT-GENTIL & 2016 & 43.35 & 19.08 & 97.31 & RW2 \\ 
  Gabon & LIBREVILLE,PORT-GENTIL & 2017 & 41.10 & 14.63 & 113.42 & RW2 \\ 
  Gabon & LIBREVILLE,PORT-GENTIL & 2018 & 39.06 & 10.95 & 135.58 & RW2 \\ 
  Gabon & LIBREVILLE,PORT-GENTIL & 2019 & 37.11 & 7.90 & 163.68 & RW2 \\ 
  Gabon & NORTH & 1980 & 148.95 & 97.96 & 220.77 & RW2 \\ 
  Gabon & NORTH & 1981 & 142.75 & 102.63 & 195.27 & RW2 \\ 
  Gabon & NORTH & 1982 & 136.85 & 102.45 & 179.78 & RW2 \\ 
  Gabon & NORTH & 1983 & 130.91 & 99.07 & 171.46 & RW2 \\ 
  Gabon & NORTH & 1984 & 125.44 & 95.02 & 163.86 & RW2 \\ 
  Gabon & NORTH & 1985 & 120.19 & 92.16 & 154.54 & RW2 \\ 
  Gabon & NORTH & 1986 & 115.01 & 90.08 & 146.10 & RW2 \\ 
  Gabon & NORTH & 1987 & 110.33 & 87.53 & 138.31 & RW2 \\ 
  Gabon & NORTH & 1988 & 106.01 & 84.11 & 133.08 & RW2 \\ 
  Gabon & NORTH & 1989 & 102.23 & 80.76 & 129.35 & RW2 \\ 
  Gabon & NORTH & 1990 & 98.97 & 78.18 & 124.15 & RW2 \\ 
  Gabon & NORTH & 1991 & 97.25 & 77.75 & 120.08 & RW2 \\ 
  Gabon & NORTH & 1992 & 96.51 & 77.77 & 118.71 & RW2 \\ 
  Gabon & NORTH & 1993 & 97.06 & 77.97 & 120.18 & RW2 \\ 
  Gabon & NORTH & 1994 & 98.78 & 78.15 & 123.65 & RW2 \\ 
  Gabon & NORTH & 1995 & 101.54 & 80.38 & 128.20 & RW2 \\ 
  Gabon & NORTH & 1996 & 104.51 & 83.53 & 130.27 & RW2 \\ 
  Gabon & NORTH & 1997 & 107.64 & 86.39 & 134.31 & RW2 \\ 
  Gabon & NORTH & 1998 & 110.69 & 87.97 & 139.13 & RW2 \\ 
  Gabon & NORTH & 1999 & 112.97 & 88.15 & 144.44 & RW2 \\ 
  Gabon & NORTH & 2000 & 114.61 & 88.04 & 150.19 & RW2 \\ 
  Gabon & NORTH & 2001 & 115.04 & 87.63 & 151.68 & RW2 \\ 
  Gabon & NORTH & 2002 & 114.07 & 86.74 & 150.81 & RW2 \\ 
  Gabon & NORTH & 2003 & 111.81 & 84.80 & 147.92 & RW2 \\ 
  Gabon & NORTH & 2004 & 108.51 & 82.22 & 142.96 & RW2 \\ 
  Gabon & NORTH & 2005 & 103.96 & 79.33 & 135.21 & RW2 \\ 
  Gabon & NORTH & 2006 & 99.31 & 76.95 & 127.77 & RW2 \\ 
  Gabon & NORTH & 2007 & 94.69 & 73.73 & 121.15 & RW2 \\ 
  Gabon & NORTH & 2008 & 90.12 & 69.58 & 115.81 & RW2 \\ 
  Gabon & NORTH & 2009 & 85.58 & 65.45 & 110.69 & RW2 \\ 
  Gabon & NORTH & 2010 & 81.48 & 62.21 & 105.76 & RW2 \\ 
  Gabon & NORTH & 2011 & 77.42 & 60.30 & 98.94 & RW2 \\ 
  Gabon & NORTH & 2012 & 73.55 & 57.87 & 93.07 & RW2 \\ 
  Gabon & NORTH & 2013 & 69.88 & 53.00 & 91.56 & RW2 \\ 
  Gabon & NORTH & 2014 & 66.28 & 45.14 & 96.07 & RW2 \\ 
  Gabon & NORTH & 2015 & 62.98 & 36.03 & 108.28 & RW2 \\ 
  Gabon & NORTH & 2016 & 59.67 & 28.09 & 122.86 & RW2 \\ 
  Gabon & NORTH & 2017 & 56.71 & 21.46 & 143.03 & RW2 \\ 
  Gabon & NORTH & 2018 & 53.61 & 16.02 & 167.85 & RW2 \\ 
  Gabon & NORTH & 2019 & 51.00 & 11.77 & 200.10 & RW2 \\ 
  Gabon & SOUTH & 1980 & 137.58 & 91.13 & 201.83 & RW2 \\ 
  Gabon & SOUTH & 1981 & 131.12 & 95.37 & 178.43 & RW2 \\ 
  Gabon & SOUTH & 1982 & 124.93 & 93.97 & 164.06 & RW2 \\ 
  Gabon & SOUTH & 1983 & 118.93 & 89.58 & 156.12 & RW2 \\ 
  Gabon & SOUTH & 1984 & 113.26 & 85.17 & 149.07 & RW2 \\ 
  Gabon & SOUTH & 1985 & 108.00 & 82.32 & 140.74 & RW2 \\ 
  Gabon & SOUTH & 1986 & 103.04 & 79.59 & 132.23 & RW2 \\ 
  Gabon & SOUTH & 1987 & 98.42 & 76.94 & 125.00 & RW2 \\ 
  Gabon & SOUTH & 1988 & 94.24 & 73.85 & 119.84 & RW2 \\ 
  Gabon & SOUTH & 1989 & 90.80 & 71.16 & 115.77 & RW2 \\ 
  Gabon & SOUTH & 1990 & 87.59 & 69.00 & 110.44 & RW2 \\ 
  Gabon & SOUTH & 1991 & 85.86 & 68.42 & 106.47 & RW2 \\ 
  Gabon & SOUTH & 1992 & 85.03 & 68.41 & 104.85 & RW2 \\ 
  Gabon & SOUTH & 1993 & 85.13 & 68.17 & 105.68 & RW2 \\ 
  Gabon & SOUTH & 1994 & 86.21 & 68.13 & 108.32 & RW2 \\ 
  Gabon & SOUTH & 1995 & 88.29 & 69.86 & 111.59 & RW2 \\ 
  Gabon & SOUTH & 1996 & 90.29 & 71.82 & 113.44 & RW2 \\ 
  Gabon & SOUTH & 1997 & 92.54 & 74.04 & 116.00 & RW2 \\ 
  Gabon & SOUTH & 1998 & 94.49 & 74.87 & 119.69 & RW2 \\ 
  Gabon & SOUTH & 1999 & 95.96 & 74.40 & 123.60 & RW2 \\ 
  Gabon & SOUTH & 2000 & 96.69 & 73.65 & 127.71 & RW2 \\ 
  Gabon & SOUTH & 2001 & 96.25 & 72.89 & 128.17 & RW2 \\ 
  Gabon & SOUTH & 2002 & 94.71 & 71.26 & 126.42 & RW2 \\ 
  Gabon & SOUTH & 2003 & 91.91 & 68.85 & 123.72 & RW2 \\ 
  Gabon & SOUTH & 2004 & 88.32 & 65.94 & 119.47 & RW2 \\ 
  Gabon & SOUTH & 2005 & 83.77 & 62.43 & 112.19 & RW2 \\ 
  Gabon & SOUTH & 2006 & 79.09 & 59.28 & 105.20 & RW2 \\ 
  Gabon & SOUTH & 2007 & 74.33 & 55.70 & 98.80 & RW2 \\ 
  Gabon & SOUTH & 2008 & 69.69 & 51.62 & 93.26 & RW2 \\ 
  Gabon & SOUTH & 2009 & 65.35 & 47.90 & 88.14 & RW2 \\ 
  Gabon & SOUTH & 2010 & 61.25 & 44.81 & 82.92 & RW2 \\ 
  Gabon & SOUTH & 2011 & 57.30 & 42.33 & 76.96 & RW2 \\ 
  Gabon & SOUTH & 2012 & 53.66 & 39.62 & 71.94 & RW2 \\ 
  Gabon & SOUTH & 2013 & 50.21 & 35.79 & 69.84 & RW2 \\ 
  Gabon & SOUTH & 2014 & 46.91 & 30.19 & 71.84 & RW2 \\ 
  Gabon & SOUTH & 2015 & 43.84 & 23.75 & 79.63 & RW2 \\ 
  Gabon & SOUTH & 2016 & 40.91 & 18.21 & 89.02 & RW2 \\ 
  Gabon & SOUTH & 2017 & 38.26 & 13.74 & 102.40 & RW2 \\ 
  Gabon & SOUTH & 2018 & 35.58 & 10.03 & 119.35 & RW2 \\ 
  Gabon & SOUTH & 2019 & 33.33 & 7.16 & 144.02 & RW2 \\ 
  Gabon & WEST & 1980 & 159.75 & 107.71 & 235.30 & RW2 \\ 
  Gabon & WEST & 1981 & 150.79 & 110.83 & 206.10 & RW2 \\ 
  Gabon & WEST & 1982 & 142.44 & 108.46 & 187.05 & RW2 \\ 
  Gabon & WEST & 1983 & 134.25 & 102.56 & 174.93 & RW2 \\ 
  Gabon & WEST & 1984 & 126.58 & 96.46 & 165.45 & RW2 \\ 
  Gabon & WEST & 1985 & 119.43 & 92.39 & 153.26 & RW2 \\ 
  Gabon & WEST & 1986 & 112.65 & 88.86 & 142.14 & RW2 \\ 
  Gabon & WEST & 1987 & 106.40 & 84.62 & 133.09 & RW2 \\ 
  Gabon & WEST & 1988 & 100.84 & 80.10 & 126.40 & RW2 \\ 
  Gabon & WEST & 1989 & 96.04 & 75.61 & 121.33 & RW2 \\ 
  Gabon & WEST & 1990 & 91.83 & 71.96 & 115.20 & RW2 \\ 
  Gabon & WEST & 1991 & 89.08 & 70.39 & 111.08 & RW2 \\ 
  Gabon & WEST & 1992 & 87.41 & 69.53 & 108.83 & RW2 \\ 
  Gabon & WEST & 1993 & 87.00 & 68.77 & 108.87 & RW2 \\ 
  Gabon & WEST & 1994 & 87.45 & 67.91 & 110.26 & RW2 \\ 
  Gabon & WEST & 1995 & 89.21 & 69.62 & 113.11 & RW2 \\ 
  Gabon & WEST & 1996 & 91.13 & 71.74 & 114.77 & RW2 \\ 
  Gabon & WEST & 1997 & 93.13 & 73.62 & 116.52 & RW2 \\ 
  Gabon & WEST & 1998 & 95.27 & 74.33 & 120.16 & RW2 \\ 
  Gabon & WEST & 1999 & 96.78 & 74.00 & 124.32 & RW2 \\ 
  Gabon & WEST & 2000 & 97.89 & 73.58 & 128.97 & RW2 \\ 
  Gabon & WEST & 2001 & 97.95 & 72.95 & 130.64 & RW2 \\ 
  Gabon & WEST & 2002 & 97.30 & 72.07 & 130.14 & RW2 \\ 
  Gabon & WEST & 2003 & 95.39 & 70.31 & 127.92 & RW2 \\ 
  Gabon & WEST & 2004 & 92.93 & 68.54 & 124.87 & RW2 \\ 
  Gabon & WEST & 2005 & 89.50 & 66.55 & 118.70 & RW2 \\ 
  Gabon & WEST & 2006 & 86.04 & 64.68 & 113.04 & RW2 \\ 
  Gabon & WEST & 2007 & 82.59 & 62.40 & 107.91 & RW2 \\ 
  Gabon & WEST & 2008 & 79.45 & 59.74 & 104.29 & RW2 \\ 
  Gabon & WEST & 2009 & 76.36 & 56.88 & 101.37 & RW2 \\ 
  Gabon & WEST & 2010 & 73.62 & 55.11 & 97.87 & RW2 \\ 
  Gabon & WEST & 2011 & 70.78 & 53.71 & 92.89 & RW2 \\ 
  Gabon & WEST & 2012 & 68.20 & 52.29 & 88.74 & RW2 \\ 
  Gabon & WEST & 2013 & 65.71 & 48.73 & 88.63 & RW2 \\ 
  Gabon & WEST & 2014 & 63.31 & 42.56 & 94.02 & RW2 \\ 
  Gabon & WEST & 2015 & 60.92 & 34.68 & 107.01 & RW2 \\ 
  Gabon & WEST & 2016 & 58.66 & 27.49 & 122.27 & RW2 \\ 
  Gabon & WEST & 2017 & 56.39 & 21.35 & 143.71 & RW2 \\ 
  Gabon & WEST & 2018 & 54.26 & 16.30 & 172.38 & RW2 \\ 
  Gabon & WEST & 2019 & 52.27 & 11.85 & 206.24 & RW2 \\ 
  Gambia & ALL & 1980 & 142.42 & 132.90 & 152.19 & IHME \\ 
  Gambia & ALL & 1980 & 249.93 & 119.90 & 449.68 & RW2 \\ 
  Gambia & ALL & 1980 & 238.50 & 208.60 & 279.20 & UN \\ 
  Gambia & ALL & 1981 & 142.11 & 132.03 & 152.49 & IHME \\ 
  Gambia & ALL & 1981 & 240.58 & 131.55 & 398.47 & RW2 \\ 
  Gambia & ALL & 1981 & 231.90 & 203.20 & 271.50 & UN \\ 
  Gambia & ALL & 1982 & 135.26 & 126.66 & 143.69 & IHME \\ 
  Gambia & ALL & 1982 & 231.87 & 139.45 & 359.91 & RW2 \\ 
  Gambia & ALL & 1982 & 225.30 & 197.70 & 263.00 & UN \\ 
  Gambia & ALL & 1983 & 131.71 & 123.76 & 139.78 & IHME \\ 
  Gambia & ALL & 1983 & 222.45 & 140.85 & 333.46 & RW2 \\ 
  Gambia & ALL & 1983 & 218.30 & 192.00 & 254.50 & UN \\ 
  Gambia & ALL & 1984 & 128.79 & 121.31 & 136.75 & IHME \\ 
  Gambia & ALL & 1984 & 214.46 & 137.70 & 316.83 & RW2 \\ 
  Gambia & ALL & 1984 & 211.20 & 186.40 & 245.10 & UN \\ 
  Gambia & ALL & 1985 & 126.10 & 119.08 & 133.51 & IHME \\ 
  Gambia & ALL & 1985 & 205.45 & 134.33 & 301.06 & RW2 \\ 
  Gambia & ALL & 1985 & 204.00 & 180.60 & 235.40 & UN \\ 
  Gambia & ALL & 1986 & 123.16 & 116.45 & 130.24 & IHME \\ 
  Gambia & ALL & 1986 & 197.57 & 129.95 & 288.06 & RW2 \\ 
  Gambia & ALL & 1986 & 196.80 & 174.90 & 225.30 & UN \\ 
  Gambia & ALL & 1987 & 120.38 & 113.97 & 127.42 & IHME \\ 
  Gambia & ALL & 1987 & 190.30 & 126.41 & 276.27 & RW2 \\ 
  Gambia & ALL & 1987 & 189.90 & 168.90 & 216.20 & UN \\ 
  Gambia & ALL & 1988 & 117.64 & 111.48 & 124.38 & IHME \\ 
  Gambia & ALL & 1988 & 183.05 & 121.84 & 264.92 & RW2 \\ 
  Gambia & ALL & 1988 & 183.10 & 163.00 & 207.00 & UN \\ 
  Gambia & ALL & 1989 & 115.16 & 109.29 & 121.60 & IHME \\ 
  Gambia & ALL & 1989 & 176.29 & 118.38 & 253.90 & RW2 \\ 
  Gambia & ALL & 1989 & 176.60 & 157.40 & 198.50 & UN \\ 
  Gambia & ALL & 1990 & 112.71 & 107.01 & 118.98 & IHME \\ 
  Gambia & ALL & 1990 & 170.09 & 115.81 & 243.11 & RW2 \\ 
  Gambia & ALL & 1990 & 170.20 & 152.10 & 190.50 & UN \\ 
  Gambia & ALL & 1991 & 110.54 & 105.12 & 116.33 & IHME \\ 
  Gambia & ALL & 1991 & 163.98 & 112.86 & 231.49 & RW2 \\ 
  Gambia & ALL & 1991 & 164.20 & 146.90 & 182.90 & UN \\ 
  Gambia & ALL & 1992 & 108.50 & 103.17 & 113.93 & IHME \\ 
  Gambia & ALL & 1992 & 158.10 & 109.68 & 222.00 & RW2 \\ 
  Gambia & ALL & 1992 & 158.50 & 142.10 & 176.30 & UN \\ 
  Gambia & ALL & 1993 & 105.84 & 100.76 & 111.04 & IHME \\ 
  Gambia & ALL & 1993 & 152.52 & 106.59 & 213.45 & RW2 \\ 
  Gambia & ALL & 1993 & 152.90 & 137.20 & 170.00 & UN \\ 
  Gambia & ALL & 1994 & 103.33 & 98.20 & 108.09 & IHME \\ 
  Gambia & ALL & 1994 & 147.00 & 103.21 & 206.03 & RW2 \\ 
  Gambia & ALL & 1994 & 147.50 & 132.30 & 164.10 & UN \\ 
  Gambia & ALL & 1995 & 100.67 & 95.64 & 105.20 & IHME \\ 
  Gambia & ALL & 1995 & 141.84 & 100.97 & 195.18 & RW2 \\ 
  Gambia & ALL & 1995 & 142.40 & 127.50 & 158.60 & UN \\ 
  Gambia & ALL & 1996 & 97.70 & 92.87 & 102.22 & IHME \\ 
  Gambia & ALL & 1996 & 136.83 & 98.49 & 187.32 & RW2 \\ 
  Gambia & ALL & 1996 & 137.40 & 122.80 & 153.40 & UN \\ 
  Gambia & ALL & 1997 & 94.66 & 90.02 & 98.99 & IHME \\ 
  Gambia & ALL & 1997 & 132.14 & 95.86 & 179.25 & RW2 \\ 
  Gambia & ALL & 1997 & 132.50 & 118.20 & 148.60 & UN \\ 
  Gambia & ALL & 1998 & 91.55 & 87.16 & 95.68 & IHME \\ 
  Gambia & ALL & 1998 & 127.84 & 92.68 & 174.20 & RW2 \\ 
  Gambia & ALL & 1998 & 127.90 & 113.70 & 144.00 & UN \\ 
  Gambia & ALL & 1999 & 88.56 & 84.20 & 92.68 & IHME \\ 
  Gambia & ALL & 1999 & 123.61 & 89.35 & 167.57 & RW2 \\ 
  Gambia & ALL & 1999 & 123.20 & 109.20 & 139.50 & UN \\ 
  Gambia & ALL & 2000 & 85.34 & 81.09 & 89.56 & IHME \\ 
  Gambia & ALL & 2000 & 119.62 & 87.77 & 161.25 & RW2 \\ 
  Gambia & ALL & 2000 & 118.80 & 104.90 & 135.20 & UN \\ 
  Gambia & ALL & 2001 & 82.37 & 78.33 & 86.46 & IHME \\ 
  Gambia & ALL & 2001 & 115.51 & 85.37 & 154.78 & RW2 \\ 
  Gambia & ALL & 2001 & 114.50 & 100.60 & 131.10 & UN \\ 
  Gambia & ALL & 2002 & 79.62 & 75.54 & 83.75 & IHME \\ 
  Gambia & ALL & 2002 & 111.31 & 82.83 & 148.56 & RW2 \\ 
  Gambia & ALL & 2002 & 110.20 & 96.20 & 126.80 & UN \\ 
  Gambia & ALL & 2003 & 76.80 & 72.75 & 80.88 & IHME \\ 
  Gambia & ALL & 2003 & 107.21 & 79.63 & 143.39 & RW2 \\ 
  Gambia & ALL & 2003 & 106.00 & 91.80 & 123.30 & UN \\ 
  Gambia & ALL & 2004 & 73.82 & 69.90 & 77.76 & IHME \\ 
  Gambia & ALL & 2004 & 102.83 & 75.37 & 139.06 & RW2 \\ 
  Gambia & ALL & 2004 & 101.90 & 87.40 & 119.90 & UN \\ 
  Gambia & ALL & 2005 & 70.92 & 66.79 & 74.87 & IHME \\ 
  Gambia & ALL & 2005 & 98.66 & 71.91 & 133.54 & RW2 \\ 
  Gambia & ALL & 2005 & 97.90 & 82.70 & 116.70 & UN \\ 
  Gambia & ALL & 2006 & 68.10 & 63.90 & 72.03 & IHME \\ 
  Gambia & ALL & 2006 & 94.64 & 69.36 & 127.35 & RW2 \\ 
  Gambia & ALL & 2006 & 94.20 & 78.20 & 113.80 & UN \\ 
  Gambia & ALL & 2007 & 65.33 & 61.18 & 69.35 & IHME \\ 
  Gambia & ALL & 2007 & 90.92 & 66.85 & 122.17 & RW2 \\ 
  Gambia & ALL & 2007 & 90.60 & 74.00 & 111.00 & UN \\ 
  Gambia & ALL & 2008 & 62.47 & 58.35 & 66.50 & IHME \\ 
  Gambia & ALL & 2008 & 87.61 & 63.72 & 119.08 & RW2 \\ 
  Gambia & ALL & 2008 & 87.40 & 70.00 & 108.60 & UN \\ 
  Gambia & ALL & 2009 & 59.70 & 55.60 & 63.67 & IHME \\ 
  Gambia & ALL & 2009 & 84.26 & 60.15 & 116.99 & RW2 \\ 
  Gambia & ALL & 2009 & 84.30 & 65.90 & 106.80 & UN \\ 
  Gambia & ALL & 2010 & 56.96 & 52.79 & 61.08 & IHME \\ 
  Gambia & ALL & 2010 & 81.29 & 56.98 & 115.63 & RW2 \\ 
  Gambia & ALL & 2010 & 81.40 & 62.00 & 105.60 & UN \\ 
  Gambia & ALL & 2011 & 54.44 & 50.19 & 58.59 & IHME \\ 
  Gambia & ALL & 2011 & 78.50 & 54.11 & 112.77 & RW2 \\ 
  Gambia & ALL & 2011 & 78.60 & 58.20 & 104.80 & UN \\ 
  Gambia & ALL & 2012 & 51.97 & 47.65 & 56.38 & IHME \\ 
  Gambia & ALL & 2012 & 75.75 & 50.45 & 112.32 & RW2 \\ 
  Gambia & ALL & 2012 & 76.10 & 54.60 & 104.30 & UN \\ 
  Gambia & ALL & 2013 & 49.68 & 45.29 & 54.27 & IHME \\ 
  Gambia & ALL & 2013 & 73.24 & 45.20 & 116.40 & RW2 \\ 
  Gambia & ALL & 2013 & 73.60 & 51.00 & 103.90 & UN \\ 
  Gambia & ALL & 2014 & 47.55 & 43.18 & 52.41 & IHME \\ 
  Gambia & ALL & 2014 & 70.65 & 38.19 & 127.26 & RW2 \\ 
  Gambia & ALL & 2014 & 71.10 & 47.90 & 104.20 & UN \\ 
  Gambia & ALL & 2015 & 45.39 & 40.89 & 50.31 & IHME \\ 
  Gambia & ALL & 2015 & 68.00 & 29.99 & 148.03 & RW2 \\ 
  Gambia & ALL & 2015 & 68.90 & 44.70 & 104.10 & UN \\ 
  Gambia & ALL & 2016 & 65.85 & 23.35 & 175.00 & RW2 \\ 
  Gambia & ALL & 2017 & 63.30 & 17.37 & 210.03 & RW2 \\ 
  Gambia & ALL & 2018 & 61.03 & 12.79 & 256.94 & RW2 \\ 
  Gambia & ALL & 2019 & 58.72 & 8.73 & 308.51 & RW2 \\ 
  Gambia & ALL & 80-84 & 180.05 & 246.58 & 128.42 & HT-Direct \\ 
  Gambia & ALL & 85-89 & 126.99 & 156.51 & 102.36 & HT-Direct \\ 
  Gambia & ALL & 90-94 & 105.96 & 124.30 & 90.05 & HT-Direct \\ 
  Gambia & ALL & 95-99 & 91.88 & 105.60 & 79.78 & HT-Direct \\ 
  Gambia & ALL & 00-04 & 84.69 & 96.76 & 74.00 & HT-Direct \\ 
  Gambia & ALL & 05-09 & 66.52 & 75.79 & 58.31 & HT-Direct \\ 
  Gambia & ALL & 10-14 & 51.06 & 60.35 & 43.13 & HT-Direct \\ 
  Gambia & ALL & 15-19 & 63.29 & 17.74 & 205.75 & RW2 \\ 
  Gambia & BANJUL & 1980 & 137.94 & 43.66 & 369.53 & RW2 \\ 
  Gambia & BANJUL & 1981 & 134.03 & 47.22 & 334.08 & RW2 \\ 
  Gambia & BANJUL & 1982 & 129.30 & 49.87 & 300.98 & RW2 \\ 
  Gambia & BANJUL & 1983 & 124.96 & 51.46 & 277.43 & RW2 \\ 
  Gambia & BANJUL & 1984 & 120.68 & 51.81 & 258.25 & RW2 \\ 
  Gambia & BANJUL & 1985 & 116.78 & 52.79 & 240.66 & RW2 \\ 
  Gambia & BANJUL & 1986 & 113.22 & 53.08 & 224.87 & RW2 \\ 
  Gambia & BANJUL & 1987 & 109.84 & 53.59 & 212.35 & RW2 \\ 
  Gambia & BANJUL & 1988 & 106.82 & 53.85 & 200.86 & RW2 \\ 
  Gambia & BANJUL & 1989 & 104.06 & 54.50 & 189.52 & RW2 \\ 
  Gambia & BANJUL & 1990 & 101.57 & 55.05 & 180.22 & RW2 \\ 
  Gambia & BANJUL & 1991 & 99.42 & 55.73 & 171.23 & RW2 \\ 
  Gambia & BANJUL & 1992 & 97.08 & 55.78 & 163.31 & RW2 \\ 
  Gambia & BANJUL & 1993 & 94.70 & 56.01 & 155.84 & RW2 \\ 
  Gambia & BANJUL & 1994 & 92.92 & 55.97 & 149.41 & RW2 \\ 
  Gambia & BANJUL & 1995 & 90.83 & 56.19 & 142.85 & RW2 \\ 
  Gambia & BANJUL & 1996 & 89.29 & 56.46 & 137.27 & RW2 \\ 
  Gambia & BANJUL & 1997 & 87.86 & 56.68 & 133.09 & RW2 \\ 
  Gambia & BANJUL & 1998 & 86.68 & 56.90 & 129.22 & RW2 \\ 
  Gambia & BANJUL & 1999 & 85.68 & 56.16 & 126.40 & RW2 \\ 
  Gambia & BANJUL & 2000 & 84.79 & 56.45 & 124.16 & RW2 \\ 
  Gambia & BANJUL & 2001 & 83.84 & 56.47 & 121.42 & RW2 \\ 
  Gambia & BANJUL & 2002 & 82.69 & 56.27 & 119.11 & RW2 \\ 
  Gambia & BANJUL & 2003 & 81.28 & 55.53 & 117.41 & RW2 \\ 
  Gambia & BANJUL & 2004 & 79.86 & 54.24 & 115.70 & RW2 \\ 
  Gambia & BANJUL & 2005 & 78.15 & 53.10 & 113.44 & RW2 \\ 
  Gambia & BANJUL & 2006 & 76.61 & 52.15 & 111.21 & RW2 \\ 
  Gambia & BANJUL & 2007 & 74.99 & 50.41 & 109.69 & RW2 \\ 
  Gambia & BANJUL & 2008 & 73.57 & 48.48 & 109.83 & RW2 \\ 
  Gambia & BANJUL & 2009 & 72.01 & 46.30 & 110.33 & RW2 \\ 
  Gambia & BANJUL & 2010 & 70.77 & 44.26 & 111.78 & RW2 \\ 
  Gambia & BANJUL & 2011 & 69.40 & 42.11 & 113.05 & RW2 \\ 
  Gambia & BANJUL & 2012 & 68.15 & 39.66 & 115.67 & RW2 \\ 
  Gambia & BANJUL & 2013 & 67.00 & 36.31 & 122.43 & RW2 \\ 
  Gambia & BANJUL & 2014 & 65.89 & 31.89 & 132.71 & RW2 \\ 
  Gambia & BANJUL & 2015 & 64.70 & 26.54 & 151.21 & RW2 \\ 
  Gambia & BANJUL & 2016 & 63.38 & 21.56 & 174.66 & RW2 \\ 
  Gambia & BANJUL & 2017 & 62.39 & 17.07 & 206.99 & RW2 \\ 
  Gambia & BANJUL & 2018 & 61.02 & 13.04 & 248.67 & RW2 \\ 
  Gambia & BANJUL & 2019 & 59.99 & 9.76 & 302.40 & RW2 \\ 
  Gambia & CENTRAL RIVER & 1980 & 242.60 & 98.11 & 484.18 & RW2 \\ 
  Gambia & CENTRAL RIVER & 1981 & 232.90 & 105.10 & 443.39 & RW2 \\ 
  Gambia & CENTRAL RIVER & 1982 & 223.61 & 107.19 & 408.12 & RW2 \\ 
  Gambia & CENTRAL RIVER & 1983 & 214.32 & 106.97 & 382.44 & RW2 \\ 
  Gambia & CENTRAL RIVER & 1984 & 205.55 & 106.20 & 359.54 & RW2 \\ 
  Gambia & CENTRAL RIVER & 1985 & 197.08 & 104.53 & 340.67 & RW2 \\ 
  Gambia & CENTRAL RIVER & 1986 & 189.44 & 102.42 & 322.45 & RW2 \\ 
  Gambia & CENTRAL RIVER & 1987 & 182.17 & 101.11 & 305.01 & RW2 \\ 
  Gambia & CENTRAL RIVER & 1988 & 175.30 & 99.83 & 289.31 & RW2 \\ 
  Gambia & CENTRAL RIVER & 1989 & 169.40 & 99.28 & 273.97 & RW2 \\ 
  Gambia & CENTRAL RIVER & 1990 & 163.22 & 98.70 & 258.95 & RW2 \\ 
  Gambia & CENTRAL RIVER & 1991 & 158.00 & 97.47 & 244.10 & RW2 \\ 
  Gambia & CENTRAL RIVER & 1992 & 152.57 & 96.65 & 231.57 & RW2 \\ 
  Gambia & CENTRAL RIVER & 1993 & 147.00 & 95.35 & 220.10 & RW2 \\ 
  Gambia & CENTRAL RIVER & 1994 & 141.94 & 93.75 & 209.56 & RW2 \\ 
  Gambia & CENTRAL RIVER & 1995 & 136.92 & 92.50 & 197.44 & RW2 \\ 
  Gambia & CENTRAL RIVER & 1996 & 131.94 & 90.78 & 187.17 & RW2 \\ 
  Gambia & CENTRAL RIVER & 1997 & 127.67 & 89.72 & 178.47 & RW2 \\ 
  Gambia & CENTRAL RIVER & 1998 & 123.46 & 87.73 & 171.17 & RW2 \\ 
  Gambia & CENTRAL RIVER & 1999 & 119.50 & 85.19 & 164.21 & RW2 \\ 
  Gambia & CENTRAL RIVER & 2000 & 115.64 & 83.63 & 157.91 & RW2 \\ 
  Gambia & CENTRAL RIVER & 2001 & 111.51 & 81.76 & 150.82 & RW2 \\ 
  Gambia & CENTRAL RIVER & 2002 & 107.26 & 78.89 & 144.15 & RW2 \\ 
  Gambia & CENTRAL RIVER & 2003 & 102.65 & 75.11 & 139.27 & RW2 \\ 
  Gambia & CENTRAL RIVER & 2004 & 98.08 & 70.93 & 134.85 & RW2 \\ 
  Gambia & CENTRAL RIVER & 2005 & 93.36 & 66.60 & 129.32 & RW2 \\ 
  Gambia & CENTRAL RIVER & 2006 & 88.63 & 62.76 & 123.57 & RW2 \\ 
  Gambia & CENTRAL RIVER & 2007 & 83.98 & 58.70 & 118.71 & RW2 \\ 
  Gambia & CENTRAL RIVER & 2008 & 79.51 & 54.17 & 114.94 & RW2 \\ 
  Gambia & CENTRAL RIVER & 2009 & 75.37 & 49.96 & 111.89 & RW2 \\ 
  Gambia & CENTRAL RIVER & 2010 & 71.38 & 45.93 & 109.09 & RW2 \\ 
  Gambia & CENTRAL RIVER & 2011 & 67.53 & 42.21 & 106.22 & RW2 \\ 
  Gambia & CENTRAL RIVER & 2012 & 64.01 & 38.31 & 104.58 & RW2 \\ 
  Gambia & CENTRAL RIVER & 2013 & 60.58 & 34.15 & 105.63 & RW2 \\ 
  Gambia & CENTRAL RIVER & 2014 & 57.24 & 28.84 & 110.48 & RW2 \\ 
  Gambia & CENTRAL RIVER & 2015 & 54.14 & 23.01 & 122.57 & RW2 \\ 
  Gambia & CENTRAL RIVER & 2016 & 51.12 & 17.83 & 137.09 & RW2 \\ 
  Gambia & CENTRAL RIVER & 2017 & 48.42 & 13.60 & 157.77 & RW2 \\ 
  Gambia & CENTRAL RIVER & 2018 & 45.56 & 10.00 & 183.75 & RW2 \\ 
  Gambia & CENTRAL RIVER & 2019 & 43.24 & 7.18 & 221.23 & RW2 \\ 
  Gambia & LOWER RIVER & 1980 & 432.20 & 216.24 & 680.43 & RW2 \\ 
  Gambia & LOWER RIVER & 1981 & 410.87 & 221.20 & 632.01 & RW2 \\ 
  Gambia & LOWER RIVER & 1982 & 390.31 & 220.67 & 588.98 & RW2 \\ 
  Gambia & LOWER RIVER & 1983 & 368.91 & 215.65 & 555.93 & RW2 \\ 
  Gambia & LOWER RIVER & 1984 & 349.33 & 207.44 & 523.84 & RW2 \\ 
  Gambia & LOWER RIVER & 1985 & 329.86 & 196.48 & 494.20 & RW2 \\ 
  Gambia & LOWER RIVER & 1986 & 311.01 & 188.28 & 468.67 & RW2 \\ 
  Gambia & LOWER RIVER & 1987 & 294.17 & 180.39 & 440.64 & RW2 \\ 
  Gambia & LOWER RIVER & 1988 & 277.95 & 172.32 & 415.28 & RW2 \\ 
  Gambia & LOWER RIVER & 1989 & 262.29 & 164.83 & 391.43 & RW2 \\ 
  Gambia & LOWER RIVER & 1990 & 248.07 & 158.37 & 367.57 & RW2 \\ 
  Gambia & LOWER RIVER & 1991 & 234.85 & 151.99 & 342.95 & RW2 \\ 
  Gambia & LOWER RIVER & 1992 & 221.24 & 145.46 & 321.78 & RW2 \\ 
  Gambia & LOWER RIVER & 1993 & 208.70 & 139.27 & 302.56 & RW2 \\ 
  Gambia & LOWER RIVER & 1994 & 197.07 & 131.95 & 284.56 & RW2 \\ 
  Gambia & LOWER RIVER & 1995 & 185.20 & 125.25 & 265.72 & RW2 \\ 
  Gambia & LOWER RIVER & 1996 & 174.51 & 119.62 & 246.99 & RW2 \\ 
  Gambia & LOWER RIVER & 1997 & 164.56 & 113.83 & 233.04 & RW2 \\ 
  Gambia & LOWER RIVER & 1998 & 155.56 & 107.83 & 219.53 & RW2 \\ 
  Gambia & LOWER RIVER & 1999 & 146.64 & 101.61 & 206.84 & RW2 \\ 
  Gambia & LOWER RIVER & 2000 & 138.47 & 96.81 & 195.59 & RW2 \\ 
  Gambia & LOWER RIVER & 2001 & 130.38 & 91.38 & 183.57 & RW2 \\ 
  Gambia & LOWER RIVER & 2002 & 122.24 & 85.79 & 172.36 & RW2 \\ 
  Gambia & LOWER RIVER & 2003 & 114.17 & 79.43 & 162.15 & RW2 \\ 
  Gambia & LOWER RIVER & 2004 & 106.34 & 72.91 & 152.33 & RW2 \\ 
  Gambia & LOWER RIVER & 2005 & 98.49 & 66.89 & 142.72 & RW2 \\ 
  Gambia & LOWER RIVER & 2006 & 91.05 & 61.63 & 133.34 & RW2 \\ 
  Gambia & LOWER RIVER & 2007 & 84.27 & 56.14 & 125.22 & RW2 \\ 
  Gambia & LOWER RIVER & 2008 & 77.89 & 50.38 & 118.70 & RW2 \\ 
  Gambia & LOWER RIVER & 2009 & 71.74 & 44.89 & 112.78 & RW2 \\ 
  Gambia & LOWER RIVER & 2010 & 66.27 & 39.80 & 108.32 & RW2 \\ 
  Gambia & LOWER RIVER & 2011 & 61.09 & 35.48 & 103.53 & RW2 \\ 
  Gambia & LOWER RIVER & 2012 & 56.30 & 31.10 & 99.92 & RW2 \\ 
  Gambia & LOWER RIVER & 2013 & 51.87 & 26.63 & 98.65 & RW2 \\ 
  Gambia & LOWER RIVER & 2014 & 47.68 & 21.76 & 100.55 & RW2 \\ 
  Gambia & LOWER RIVER & 2015 & 44.01 & 17.10 & 108.22 & RW2 \\ 
  Gambia & LOWER RIVER & 2016 & 40.40 & 12.97 & 117.98 & RW2 \\ 
  Gambia & LOWER RIVER & 2017 & 37.28 & 9.64 & 132.77 & RW2 \\ 
  Gambia & LOWER RIVER & 2018 & 34.13 & 6.97 & 151.24 & RW2 \\ 
  Gambia & LOWER RIVER & 2019 & 31.52 & 4.96 & 176.13 & RW2 \\ 
  Gambia & NORTH BANK & 1980 & 307.61 & 142.53 & 546.40 & RW2 \\ 
  Gambia & NORTH BANK & 1981 & 293.49 & 149.91 & 497.20 & RW2 \\ 
  Gambia & NORTH BANK & 1982 & 280.24 & 153.24 & 455.33 & RW2 \\ 
  Gambia & NORTH BANK & 1983 & 266.28 & 152.31 & 422.74 & RW2 \\ 
  Gambia & NORTH BANK & 1984 & 253.44 & 149.20 & 397.90 & RW2 \\ 
  Gambia & NORTH BANK & 1985 & 241.10 & 145.00 & 372.13 & RW2 \\ 
  Gambia & NORTH BANK & 1986 & 229.46 & 141.27 & 350.28 & RW2 \\ 
  Gambia & NORTH BANK & 1987 & 218.59 & 136.48 & 330.59 & RW2 \\ 
  Gambia & NORTH BANK & 1988 & 208.68 & 132.59 & 312.00 & RW2 \\ 
  Gambia & NORTH BANK & 1989 & 199.26 & 128.43 & 294.93 & RW2 \\ 
  Gambia & NORTH BANK & 1990 & 190.65 & 124.87 & 278.59 & RW2 \\ 
  Gambia & NORTH BANK & 1991 & 182.21 & 121.25 & 264.18 & RW2 \\ 
  Gambia & NORTH BANK & 1992 & 173.84 & 117.88 & 250.40 & RW2 \\ 
  Gambia & NORTH BANK & 1993 & 166.22 & 114.08 & 237.27 & RW2 \\ 
  Gambia & NORTH BANK & 1994 & 158.40 & 108.96 & 224.01 & RW2 \\ 
  Gambia & NORTH BANK & 1995 & 151.28 & 105.76 & 211.19 & RW2 \\ 
  Gambia & NORTH BANK & 1996 & 144.63 & 102.62 & 200.21 & RW2 \\ 
  Gambia & NORTH BANK & 1997 & 138.10 & 99.24 & 189.25 & RW2 \\ 
  Gambia & NORTH BANK & 1998 & 132.41 & 95.23 & 180.63 & RW2 \\ 
  Gambia & NORTH BANK & 1999 & 126.52 & 91.02 & 172.24 & RW2 \\ 
  Gambia & NORTH BANK & 2000 & 121.09 & 88.18 & 164.30 & RW2 \\ 
  Gambia & NORTH BANK & 2001 & 115.34 & 84.45 & 156.42 & RW2 \\ 
  Gambia & NORTH BANK & 2002 & 109.80 & 80.23 & 148.82 & RW2 \\ 
  Gambia & NORTH BANK & 2003 & 103.75 & 74.80 & 142.06 & RW2 \\ 
  Gambia & NORTH BANK & 2004 & 97.95 & 69.41 & 136.54 & RW2 \\ 
  Gambia & NORTH BANK & 2005 & 91.96 & 64.18 & 130.08 & RW2 \\ 
  Gambia & NORTH BANK & 2006 & 86.11 & 59.18 & 123.56 & RW2 \\ 
  Gambia & NORTH BANK & 2007 & 80.50 & 54.11 & 117.68 & RW2 \\ 
  Gambia & NORTH BANK & 2008 & 75.46 & 49.05 & 113.65 & RW2 \\ 
  Gambia & NORTH BANK & 2009 & 70.48 & 43.94 & 110.80 & RW2 \\ 
  Gambia & NORTH BANK & 2010 & 65.93 & 39.60 & 107.90 & RW2 \\ 
  Gambia & NORTH BANK & 2011 & 61.38 & 35.18 & 104.75 & RW2 \\ 
  Gambia & NORTH BANK & 2012 & 57.33 & 31.15 & 102.54 & RW2 \\ 
  Gambia & NORTH BANK & 2013 & 53.55 & 26.87 & 103.16 & RW2 \\ 
  Gambia & NORTH BANK & 2014 & 50.06 & 22.33 & 106.98 & RW2 \\ 
  Gambia & NORTH BANK & 2015 & 46.65 & 17.72 & 116.72 & RW2 \\ 
  Gambia & NORTH BANK & 2016 & 43.52 & 13.54 & 128.16 & RW2 \\ 
  Gambia & NORTH BANK & 2017 & 40.50 & 10.16 & 145.54 & RW2 \\ 
  Gambia & NORTH BANK & 2018 & 37.75 & 7.48 & 169.54 & RW2 \\ 
  Gambia & NORTH BANK & 2019 & 35.23 & 5.20 & 197.53 & RW2 \\ 
  Gambia & UPPER RIVER & 1980 & 359.35 & 180.65 & 587.20 & RW2 \\ 
  Gambia & UPPER RIVER & 1981 & 346.95 & 190.92 & 544.28 & RW2 \\ 
  Gambia & UPPER RIVER & 1982 & 333.84 & 194.56 & 510.29 & RW2 \\ 
  Gambia & UPPER RIVER & 1983 & 323.20 & 192.96 & 485.32 & RW2 \\ 
  Gambia & UPPER RIVER & 1984 & 310.92 & 187.91 & 465.20 & RW2 \\ 
  Gambia & UPPER RIVER & 1985 & 300.03 & 182.84 & 447.49 & RW2 \\ 
  Gambia & UPPER RIVER & 1986 & 289.13 & 178.12 & 430.08 & RW2 \\ 
  Gambia & UPPER RIVER & 1987 & 279.23 & 174.11 & 414.65 & RW2 \\ 
  Gambia & UPPER RIVER & 1988 & 269.77 & 171.30 & 395.93 & RW2 \\ 
  Gambia & UPPER RIVER & 1989 & 261.25 & 168.91 & 381.40 & RW2 \\ 
  Gambia & UPPER RIVER & 1990 & 253.32 & 167.35 & 364.54 & RW2 \\ 
  Gambia & UPPER RIVER & 1991 & 244.91 & 165.47 & 347.47 & RW2 \\ 
  Gambia & UPPER RIVER & 1992 & 236.68 & 163.04 & 330.69 & RW2 \\ 
  Gambia & UPPER RIVER & 1993 & 228.92 & 160.41 & 316.06 & RW2 \\ 
  Gambia & UPPER RIVER & 1994 & 220.90 & 156.81 & 302.54 & RW2 \\ 
  Gambia & UPPER RIVER & 1995 & 213.39 & 154.09 & 287.10 & RW2 \\ 
  Gambia & UPPER RIVER & 1996 & 206.01 & 151.10 & 274.49 & RW2 \\ 
  Gambia & UPPER RIVER & 1997 & 199.05 & 147.12 & 263.69 & RW2 \\ 
  Gambia & UPPER RIVER & 1998 & 192.39 & 142.70 & 254.73 & RW2 \\ 
  Gambia & UPPER RIVER & 1999 & 185.85 & 136.65 & 247.55 & RW2 \\ 
  Gambia & UPPER RIVER & 2000 & 180.13 & 132.77 & 240.70 & RW2 \\ 
  Gambia & UPPER RIVER & 2001 & 173.73 & 127.89 & 232.93 & RW2 \\ 
  Gambia & UPPER RIVER & 2002 & 167.08 & 122.47 & 225.30 & RW2 \\ 
  Gambia & UPPER RIVER & 2003 & 160.01 & 115.75 & 217.80 & RW2 \\ 
  Gambia & UPPER RIVER & 2004 & 152.64 & 109.27 & 210.49 & RW2 \\ 
  Gambia & UPPER RIVER & 2005 & 145.14 & 102.60 & 201.04 & RW2 \\ 
  Gambia & UPPER RIVER & 2006 & 137.45 & 96.76 & 192.11 & RW2 \\ 
  Gambia & UPPER RIVER & 2007 & 130.49 & 90.92 & 183.17 & RW2 \\ 
  Gambia & UPPER RIVER & 2008 & 123.38 & 84.67 & 177.03 & RW2 \\ 
  Gambia & UPPER RIVER & 2009 & 116.38 & 78.05 & 170.36 & RW2 \\ 
  Gambia & UPPER RIVER & 2010 & 109.81 & 72.00 & 164.17 & RW2 \\ 
  Gambia & UPPER RIVER & 2011 & 103.83 & 66.73 & 157.41 & RW2 \\ 
  Gambia & UPPER RIVER & 2012 & 97.90 & 60.77 & 153.21 & RW2 \\ 
  Gambia & UPPER RIVER & 2013 & 91.95 & 53.59 & 152.27 & RW2 \\ 
  Gambia & UPPER RIVER & 2014 & 86.84 & 45.46 & 159.17 & RW2 \\ 
  Gambia & UPPER RIVER & 2015 & 81.69 & 36.21 & 172.45 & RW2 \\ 
  Gambia & UPPER RIVER & 2016 & 76.95 & 28.00 & 193.92 & RW2 \\ 
  Gambia & UPPER RIVER & 2017 & 72.65 & 21.14 & 219.57 & RW2 \\ 
  Gambia & UPPER RIVER & 2018 & 68.37 & 15.48 & 252.46 & RW2 \\ 
  Gambia & UPPER RIVER & 2019 & 63.90 & 10.91 & 297.79 & RW2 \\ 
  Gambia & WESTERN & 1980 & 214.64 & 86.60 & 443.79 & RW2 \\ 
  Gambia & WESTERN & 1981 & 206.95 & 93.54 & 399.01 & RW2 \\ 
  Gambia & WESTERN & 1982 & 198.95 & 96.78 & 366.20 & RW2 \\ 
  Gambia & WESTERN & 1983 & 190.87 & 97.40 & 341.60 & RW2 \\ 
  Gambia & WESTERN & 1984 & 183.26 & 97.11 & 319.32 & RW2 \\ 
  Gambia & WESTERN & 1985 & 175.87 & 94.99 & 301.46 & RW2 \\ 
  Gambia & WESTERN & 1986 & 169.38 & 93.62 & 285.57 & RW2 \\ 
  Gambia & WESTERN & 1987 & 163.31 & 92.89 & 270.28 & RW2 \\ 
  Gambia & WESTERN & 1988 & 157.62 & 91.29 & 257.39 & RW2 \\ 
  Gambia & WESTERN & 1989 & 152.37 & 90.32 & 244.93 & RW2 \\ 
  Gambia & WESTERN & 1990 & 147.50 & 89.87 & 233.39 & RW2 \\ 
  Gambia & WESTERN & 1991 & 143.07 & 89.17 & 222.38 & RW2 \\ 
  Gambia & WESTERN & 1992 & 138.74 & 87.95 & 211.88 & RW2 \\ 
  Gambia & WESTERN & 1993 & 134.38 & 87.14 & 201.58 & RW2 \\ 
  Gambia & WESTERN & 1994 & 130.21 & 85.63 & 193.27 & RW2 \\ 
  Gambia & WESTERN & 1995 & 126.19 & 84.59 & 183.40 & RW2 \\ 
  Gambia & WESTERN & 1996 & 122.67 & 83.70 & 175.36 & RW2 \\ 
  Gambia & WESTERN & 1997 & 119.42 & 82.98 & 167.95 & RW2 \\ 
  Gambia & WESTERN & 1998 & 116.38 & 82.02 & 162.63 & RW2 \\ 
  Gambia & WESTERN & 1999 & 113.57 & 80.50 & 157.63 & RW2 \\ 
  Gambia & WESTERN & 2000 & 110.98 & 79.65 & 151.82 & RW2 \\ 
  Gambia & WESTERN & 2001 & 108.17 & 79.16 & 146.13 & RW2 \\ 
  Gambia & WESTERN & 2002 & 105.26 & 77.64 & 141.25 & RW2 \\ 
  Gambia & WESTERN & 2003 & 101.94 & 75.33 & 136.99 & RW2 \\ 
  Gambia & WESTERN & 2004 & 98.63 & 72.49 & 132.99 & RW2 \\ 
  Gambia & WESTERN & 2005 & 95.13 & 69.43 & 128.89 & RW2 \\ 
  Gambia & WESTERN & 2006 & 91.68 & 66.91 & 124.32 & RW2 \\ 
  Gambia & WESTERN & 2007 & 88.28 & 63.95 & 120.47 & RW2 \\ 
  Gambia & WESTERN & 2008 & 85.11 & 60.49 & 118.34 & RW2 \\ 
  Gambia & WESTERN & 2009 & 81.92 & 56.75 & 117.00 & RW2 \\ 
  Gambia & WESTERN & 2010 & 79.14 & 52.65 & 116.97 & RW2 \\ 
  Gambia & WESTERN & 2011 & 76.15 & 49.38 & 116.15 & RW2 \\ 
  Gambia & WESTERN & 2012 & 73.54 & 45.50 & 116.94 & RW2 \\ 
  Gambia & WESTERN & 2013 & 70.83 & 40.57 & 121.31 & RW2 \\ 
  Gambia & WESTERN & 2014 & 68.32 & 34.80 & 129.37 & RW2 \\ 
  Gambia & WESTERN & 2015 & 65.92 & 28.40 & 147.23 & RW2 \\ 
  Gambia & WESTERN & 2016 & 63.52 & 22.68 & 168.19 & RW2 \\ 
  Gambia & WESTERN & 2017 & 61.30 & 17.43 & 197.73 & RW2 \\ 
  Gambia & WESTERN & 2018 & 58.84 & 12.84 & 233.36 & RW2 \\ 
  Gambia & WESTERN & 2019 & 56.62 & 9.56 & 277.07 & RW2 \\ 
  Ghana & ALL & 1980 & 152.49 & 147.24 & 157.26 & IHME \\ 
  Ghana & ALL & 1980 & 169.14 & 128.15 & 219.74 & RW2 \\ 
  Ghana & ALL & 1980 & 166.20 & 157.60 & 175.10 & UN \\ 
  Ghana & ALL & 1981 & 150.80 & 145.97 & 155.41 & IHME \\ 
  Ghana & ALL & 1981 & 165.99 & 136.55 & 200.47 & RW2 \\ 
  Ghana & ALL & 1981 & 164.60 & 155.80 & 173.70 & UN \\ 
  Ghana & ALL & 1982 & 149.26 & 144.87 & 153.79 & IHME \\ 
  Ghana & ALL & 1982 & 162.99 & 136.07 & 193.81 & RW2 \\ 
  Ghana & ALL & 1982 & 163.10 & 154.60 & 172.20 & UN \\ 
  Ghana & ALL & 1983 & 147.70 & 143.54 & 152.23 & IHME \\ 
  Ghana & ALL & 1983 & 159.69 & 130.37 & 193.37 & RW2 \\ 
  Ghana & ALL & 1983 & 161.30 & 152.90 & 170.40 & UN \\ 
  Ghana & ALL & 1984 & 145.24 & 141.23 & 149.57 & IHME \\ 
  Ghana & ALL & 1984 & 156.49 & 124.71 & 193.00 & RW2 \\ 
  Ghana & ALL & 1984 & 158.50 & 150.30 & 167.30 & UN \\ 
  Ghana & ALL & 1985 & 141.92 & 137.83 & 146.09 & IHME \\ 
  Ghana & ALL & 1985 & 153.04 & 124.77 & 186.54 & RW2 \\ 
  Ghana & ALL & 1985 & 154.60 & 146.60 & 163.20 & UN \\ 
  Ghana & ALL & 1986 & 138.17 & 134.42 & 142.19 & IHME \\ 
  Ghana & ALL & 1986 & 148.86 & 123.02 & 178.76 & RW2 \\ 
  Ghana & ALL & 1986 & 149.70 & 142.00 & 158.00 & UN \\ 
  Ghana & ALL & 1987 & 134.43 & 130.59 & 138.39 & IHME \\ 
  Ghana & ALL & 1987 & 144.21 & 120.34 & 172.34 & RW2 \\ 
  Ghana & ALL & 1987 & 144.10 & 136.70 & 152.20 & UN \\ 
  Ghana & ALL & 1988 & 130.38 & 126.68 & 134.38 & IHME \\ 
  Ghana & ALL & 1988 & 139.04 & 115.03 & 167.69 & RW2 \\ 
  Ghana & ALL & 1988 & 138.20 & 131.10 & 146.10 & UN \\ 
  Ghana & ALL & 1989 & 126.41 & 123.06 & 130.40 & IHME \\ 
  Ghana & ALL & 1989 & 133.76 & 109.49 & 163.63 & RW2 \\ 
  Ghana & ALL & 1989 & 132.60 & 125.50 & 140.00 & UN \\ 
  Ghana & ALL & 1990 & 122.71 & 119.50 & 126.55 & IHME \\ 
  Ghana & ALL & 1990 & 128.03 & 104.00 & 155.94 & RW2 \\ 
  Ghana & ALL & 1990 & 127.40 & 120.50 & 134.60 & UN \\ 
  Ghana & ALL & 1991 & 119.16 & 115.92 & 122.90 & IHME \\ 
  Ghana & ALL & 1991 & 123.54 & 101.56 & 148.77 & RW2 \\ 
  Ghana & ALL & 1991 & 123.10 & 116.20 & 130.20 & UN \\ 
  Ghana & ALL & 1992 & 116.09 & 112.75 & 119.81 & IHME \\ 
  Ghana & ALL & 1992 & 119.83 & 99.00 & 143.91 & RW2 \\ 
  Ghana & ALL & 1992 & 119.70 & 113.00 & 126.60 & UN \\ 
  Ghana & ALL & 1993 & 113.54 & 110.21 & 117.01 & IHME \\ 
  Ghana & ALL & 1993 & 116.96 & 96.04 & 141.27 & RW2 \\ 
  Ghana & ALL & 1993 & 117.10 & 110.50 & 124.00 & UN \\ 
  Ghana & ALL & 1994 & 111.64 & 108.30 & 115.11 & IHME \\ 
  Ghana & ALL & 1994 & 114.66 & 92.83 & 140.45 & RW2 \\ 
  Ghana & ALL & 1994 & 115.10 & 108.70 & 122.10 & UN \\ 
  Ghana & ALL & 1995 & 109.31 & 105.98 & 112.80 & IHME \\ 
  Ghana & ALL & 1995 & 113.48 & 92.50 & 139.66 & RW2 \\ 
  Ghana & ALL & 1995 & 113.40 & 107.00 & 120.30 & UN \\ 
  Ghana & ALL & 1996 & 107.36 & 104.09 & 110.87 & IHME \\ 
  Ghana & ALL & 1996 & 111.51 & 91.60 & 136.20 & RW2 \\ 
  Ghana & ALL & 1996 & 111.60 & 105.20 & 118.50 & UN \\ 
  Ghana & ALL & 1997 & 105.28 & 102.09 & 108.71 & IHME \\ 
  Ghana & ALL & 1997 & 109.26 & 90.12 & 132.27 & RW2 \\ 
  Ghana & ALL & 1997 & 109.50 & 103.20 & 116.40 & UN \\ 
  Ghana & ALL & 1998 & 103.09 & 99.82 & 106.42 & IHME \\ 
  Ghana & ALL & 1998 & 106.66 & 87.41 & 130.40 & RW2 \\ 
  Ghana & ALL & 1998 & 106.90 & 100.70 & 113.80 & UN \\ 
  Ghana & ALL & 1999 & 100.75 & 97.58 & 104.06 & IHME \\ 
  Ghana & ALL & 1999 & 103.64 & 83.94 & 127.53 & RW2 \\ 
  Ghana & ALL & 1999 & 104.00 & 97.80 & 110.60 & UN \\ 
  Ghana & ALL & 2000 & 98.36 & 95.08 & 101.70 & IHME \\ 
  Ghana & ALL & 2000 & 99.96 & 80.51 & 122.53 & RW2 \\ 
  Ghana & ALL & 2000 & 100.70 & 94.60 & 107.30 & UN \\ 
  Ghana & ALL & 2001 & 96.04 & 92.83 & 99.34 & IHME \\ 
  Ghana & ALL & 2001 & 96.87 & 78.70 & 118.04 & RW2 \\ 
  Ghana & ALL & 2001 & 97.30 & 91.30 & 104.00 & UN \\ 
  Ghana & ALL & 2002 & 93.49 & 90.37 & 96.80 & IHME \\ 
  Ghana & ALL & 2002 & 93.98 & 76.95 & 114.13 & RW2 \\ 
  Ghana & ALL & 2002 & 94.20 & 88.00 & 100.80 & UN \\ 
  Ghana & ALL & 2003 & 90.84 & 87.59 & 94.26 & IHME \\ 
  Ghana & ALL & 2003 & 91.48 & 74.60 & 111.53 & RW2 \\ 
  Ghana & ALL & 2003 & 91.30 & 85.10 & 98.00 & UN \\ 
  Ghana & ALL & 2004 & 88.10 & 84.82 & 91.48 & IHME \\ 
  Ghana & ALL & 2004 & 89.08 & 71.46 & 110.03 & RW2 \\ 
  Ghana & ALL & 2004 & 88.70 & 82.50 & 95.50 & UN \\ 
  Ghana & ALL & 2005 & 85.37 & 82.05 & 88.79 & IHME \\ 
  Ghana & ALL & 2005 & 87.30 & 70.02 & 109.02 & RW2 \\ 
  Ghana & ALL & 2005 & 86.50 & 80.10 & 93.50 & UN \\ 
  Ghana & ALL & 2006 & 82.57 & 79.31 & 85.90 & IHME \\ 
  Ghana & ALL & 2006 & 85.07 & 68.76 & 105.20 & RW2 \\ 
  Ghana & ALL & 2006 & 84.40 & 77.80 & 91.80 & UN \\ 
  Ghana & ALL & 2007 & 79.70 & 76.38 & 83.19 & IHME \\ 
  Ghana & ALL & 2007 & 82.75 & 67.20 & 101.73 & RW2 \\ 
  Ghana & ALL & 2007 & 82.20 & 75.40 & 90.00 & UN \\ 
  Ghana & ALL & 2008 & 76.79 & 73.15 & 80.36 & IHME \\ 
  Ghana & ALL & 2008 & 80.32 & 64.66 & 99.61 & RW2 \\ 
  Ghana & ALL & 2008 & 79.90 & 72.50 & 88.10 & UN \\ 
  Ghana & ALL & 2009 & 74.06 & 70.16 & 77.83 & IHME \\ 
  Ghana & ALL & 2009 & 77.50 & 61.38 & 97.97 & RW2 \\ 
  Ghana & ALL & 2009 & 77.40 & 69.50 & 86.10 & UN \\ 
  Ghana & ALL & 2010 & 70.99 & 66.90 & 75.16 & IHME \\ 
  Ghana & ALL & 2010 & 74.46 & 57.88 & 95.51 & RW2 \\ 
  Ghana & ALL & 2010 & 74.70 & 66.00 & 84.10 & UN \\ 
  Ghana & ALL & 2011 & 67.40 & 63.10 & 72.03 & IHME \\ 
  Ghana & ALL & 2011 & 71.59 & 55.83 & 91.11 & RW2 \\ 
  Ghana & ALL & 2011 & 71.90 & 62.30 & 82.50 & UN \\ 
  Ghana & ALL & 2012 & 63.86 & 59.28 & 69.14 & IHME \\ 
  Ghana & ALL & 2012 & 68.68 & 53.58 & 87.41 & RW2 \\ 
  Ghana & ALL & 2012 & 69.20 & 58.50 & 81.10 & UN \\ 
  Ghana & ALL & 2013 & 60.95 & 55.92 & 66.83 & IHME \\ 
  Ghana & ALL & 2013 & 65.91 & 49.51 & 86.98 & RW2 \\ 
  Ghana & ALL & 2013 & 66.50 & 54.80 & 80.00 & UN \\ 
  Ghana & ALL & 2014 & 58.32 & 52.95 & 65.02 & IHME \\ 
  Ghana & ALL & 2014 & 63.18 & 42.92 & 91.64 & RW2 \\ 
  Ghana & ALL & 2014 & 64.00 & 51.30 & 78.90 & UN \\ 
  Ghana & ALL & 2015 & 55.52 & 49.76 & 62.70 & IHME \\ 
  Ghana & ALL & 2015 & 60.52 & 34.62 & 103.23 & RW2 \\ 
  Ghana & ALL & 2015 & 61.60 & 48.10 & 78.20 & UN \\ 
  Ghana & ALL & 2016 & 58.18 & 27.75 & 117.44 & RW2 \\ 
  Ghana & ALL & 2017 & 55.64 & 21.44 & 136.00 & RW2 \\ 
  Ghana & ALL & 2018 & 53.32 & 16.38 & 160.95 & RW2 \\ 
  Ghana & ALL & 2019 & 51.02 & 11.80 & 188.89 & RW2 \\ 
  Ghana & ALL & 80-84 & 159.12 & 167.12 & 151.43 & HT-Direct \\ 
  Ghana & ALL & 85-89 & 138.07 & 145.08 & 131.35 & HT-Direct \\ 
  Ghana & ALL & 90-94 & 114.61 & 121.08 & 108.44 & HT-Direct \\ 
  Ghana & ALL & 95-99 & 104.55 & 111.69 & 97.81 & HT-Direct \\ 
  Ghana & ALL & 00-04 & 89.95 & 96.67 & 83.66 & HT-Direct \\ 
  Ghana & ALL & 05-09 & 80.90 & 89.07 & 73.42 & HT-Direct \\ 
  Ghana & ALL & 10-14 & 60.21 & 69.15 & 52.36 & HT-Direct \\ 
  Ghana & ALL & 15-19 & 55.65 & 21.75 & 133.68 & RW2 \\ 
  Ghana & ASHANTI & 1980 & 137.24 & 104.05 & 179.10 & RW2 \\ 
  Ghana & ASHANTI & 1981 & 135.41 & 111.01 & 164.57 & RW2 \\ 
  Ghana & ASHANTI & 1982 & 133.55 & 111.06 & 159.25 & RW2 \\ 
  Ghana & ASHANTI & 1983 & 131.72 & 108.32 & 159.00 & RW2 \\ 
  Ghana & ASHANTI & 1984 & 129.75 & 105.36 & 158.39 & RW2 \\ 
  Ghana & ASHANTI & 1985 & 127.82 & 105.51 & 154.09 & RW2 \\ 
  Ghana & ASHANTI & 1986 & 125.36 & 105.09 & 148.39 & RW2 \\ 
  Ghana & ASHANTI & 1987 & 122.60 & 103.75 & 144.29 & RW2 \\ 
  Ghana & ASHANTI & 1988 & 119.57 & 100.73 & 141.50 & RW2 \\ 
  Ghana & ASHANTI & 1989 & 116.50 & 96.97 & 139.50 & RW2 \\ 
  Ghana & ASHANTI & 1990 & 113.26 & 93.96 & 135.43 & RW2 \\ 
  Ghana & ASHANTI & 1991 & 110.77 & 92.96 & 131.63 & RW2 \\ 
  Ghana & ASHANTI & 1992 & 108.76 & 91.75 & 128.07 & RW2 \\ 
  Ghana & ASHANTI & 1993 & 107.29 & 90.00 & 127.07 & RW2 \\ 
  Ghana & ASHANTI & 1994 & 106.33 & 88.03 & 127.49 & RW2 \\ 
  Ghana & ASHANTI & 1995 & 106.06 & 88.05 & 127.57 & RW2 \\ 
  Ghana & ASHANTI & 1996 & 105.47 & 88.28 & 125.58 & RW2 \\ 
  Ghana & ASHANTI & 1997 & 104.84 & 88.20 & 124.04 & RW2 \\ 
  Ghana & ASHANTI & 1998 & 103.85 & 86.92 & 123.70 & RW2 \\ 
  Ghana & ASHANTI & 1999 & 102.68 & 84.71 & 124.27 & RW2 \\ 
  Ghana & ASHANTI & 2000 & 101.11 & 83.05 & 122.19 & RW2 \\ 
  Ghana & ASHANTI & 2001 & 99.68 & 82.61 & 119.38 & RW2 \\ 
  Ghana & ASHANTI & 2002 & 98.32 & 81.72 & 117.73 & RW2 \\ 
  Ghana & ASHANTI & 2003 & 96.97 & 80.04 & 116.88 & RW2 \\ 
  Ghana & ASHANTI & 2004 & 95.77 & 77.87 & 117.05 & RW2 \\ 
  Ghana & ASHANTI & 2005 & 94.62 & 76.31 & 116.69 & RW2 \\ 
  Ghana & ASHANTI & 2006 & 93.22 & 75.38 & 114.82 & RW2 \\ 
  Ghana & ASHANTI & 2007 & 91.42 & 73.64 & 113.04 & RW2 \\ 
  Ghana & ASHANTI & 2008 & 89.44 & 71.26 & 111.75 & RW2 \\ 
  Ghana & ASHANTI & 2009 & 87.41 & 68.35 & 111.30 & RW2 \\ 
  Ghana & ASHANTI & 2010 & 84.90 & 64.93 & 109.72 & RW2 \\ 
  Ghana & ASHANTI & 2011 & 82.48 & 63.21 & 107.06 & RW2 \\ 
  Ghana & ASHANTI & 2012 & 80.04 & 60.70 & 105.05 & RW2 \\ 
  Ghana & ASHANTI & 2013 & 77.65 & 56.65 & 105.27 & RW2 \\ 
  Ghana & ASHANTI & 2014 & 75.40 & 50.55 & 110.28 & RW2 \\ 
  Ghana & ASHANTI & 2015 & 73.12 & 42.88 & 122.43 & RW2 \\ 
  Ghana & ASHANTI & 2016 & 70.86 & 35.18 & 136.39 & RW2 \\ 
  Ghana & ASHANTI & 2017 & 68.87 & 28.35 & 155.48 & RW2 \\ 
  Ghana & ASHANTI & 2018 & 66.39 & 22.35 & 180.71 & RW2 \\ 
  Ghana & ASHANTI & 2019 & 64.38 & 17.16 & 210.47 & RW2 \\ 
  Ghana & BRONG AHAFO & 1980 & 145.60 & 109.21 & 191.45 & RW2 \\ 
  Ghana & BRONG AHAFO & 1981 & 142.56 & 114.91 & 175.38 & RW2 \\ 
  Ghana & BRONG AHAFO & 1982 & 139.51 & 114.68 & 168.43 & RW2 \\ 
  Ghana & BRONG AHAFO & 1983 & 136.43 & 111.05 & 166.64 & RW2 \\ 
  Ghana & BRONG AHAFO & 1984 & 133.23 & 107.27 & 164.08 & RW2 \\ 
  Ghana & BRONG AHAFO & 1985 & 130.13 & 106.43 & 158.26 & RW2 \\ 
  Ghana & BRONG AHAFO & 1986 & 126.56 & 105.02 & 151.26 & RW2 \\ 
  Ghana & BRONG AHAFO & 1987 & 122.56 & 102.70 & 145.36 & RW2 \\ 
  Ghana & BRONG AHAFO & 1988 & 118.56 & 98.77 & 141.51 & RW2 \\ 
  Ghana & BRONG AHAFO & 1989 & 114.40 & 94.39 & 138.31 & RW2 \\ 
  Ghana & BRONG AHAFO & 1990 & 110.08 & 90.69 & 132.59 & RW2 \\ 
  Ghana & BRONG AHAFO & 1991 & 106.42 & 88.63 & 127.44 & RW2 \\ 
  Ghana & BRONG AHAFO & 1992 & 103.30 & 86.43 & 123.08 & RW2 \\ 
  Ghana & BRONG AHAFO & 1993 & 100.73 & 83.85 & 120.58 & RW2 \\ 
  Ghana & BRONG AHAFO & 1994 & 98.53 & 80.91 & 119.33 & RW2 \\ 
  Ghana & BRONG AHAFO & 1995 & 96.96 & 79.53 & 117.96 & RW2 \\ 
  Ghana & BRONG AHAFO & 1996 & 95.14 & 78.65 & 114.69 & RW2 \\ 
  Ghana & BRONG AHAFO & 1997 & 93.23 & 77.31 & 112.22 & RW2 \\ 
  Ghana & BRONG AHAFO & 1998 & 91.16 & 75.41 & 110.25 & RW2 \\ 
  Ghana & BRONG AHAFO & 1999 & 88.95 & 72.54 & 108.84 & RW2 \\ 
  Ghana & BRONG AHAFO & 2000 & 86.34 & 70.05 & 105.59 & RW2 \\ 
  Ghana & BRONG AHAFO & 2001 & 84.02 & 68.76 & 102.24 & RW2 \\ 
  Ghana & BRONG AHAFO & 2002 & 81.65 & 67.01 & 99.11 & RW2 \\ 
  Ghana & BRONG AHAFO & 2003 & 79.55 & 64.63 & 97.72 & RW2 \\ 
  Ghana & BRONG AHAFO & 2004 & 77.32 & 61.80 & 96.32 & RW2 \\ 
  Ghana & BRONG AHAFO & 2005 & 75.31 & 59.81 & 94.56 & RW2 \\ 
  Ghana & BRONG AHAFO & 2006 & 73.04 & 57.99 & 91.65 & RW2 \\ 
  Ghana & BRONG AHAFO & 2007 & 70.68 & 55.80 & 88.97 & RW2 \\ 
  Ghana & BRONG AHAFO & 2008 & 68.08 & 53.09 & 86.83 & RW2 \\ 
  Ghana & BRONG AHAFO & 2009 & 65.36 & 50.04 & 84.94 & RW2 \\ 
  Ghana & BRONG AHAFO & 2010 & 62.41 & 46.94 & 82.48 & RW2 \\ 
  Ghana & BRONG AHAFO & 2011 & 59.71 & 44.51 & 78.74 & RW2 \\ 
  Ghana & BRONG AHAFO & 2012 & 56.93 & 42.26 & 76.10 & RW2 \\ 
  Ghana & BRONG AHAFO & 2013 & 54.30 & 38.96 & 75.12 & RW2 \\ 
  Ghana & BRONG AHAFO & 2014 & 51.70 & 34.12 & 77.24 & RW2 \\ 
  Ghana & BRONG AHAFO & 2015 & 49.39 & 28.31 & 84.64 & RW2 \\ 
  Ghana & BRONG AHAFO & 2016 & 47.18 & 23.03 & 93.54 & RW2 \\ 
  Ghana & BRONG AHAFO & 2017 & 44.85 & 18.16 & 104.89 & RW2 \\ 
  Ghana & BRONG AHAFO & 2018 & 42.62 & 14.12 & 120.21 & RW2 \\ 
  Ghana & BRONG AHAFO & 2019 & 40.69 & 10.69 & 141.88 & RW2 \\ 
  Ghana & CENTRAL & 1980 & 196.37 & 150.91 & 252.60 & RW2 \\ 
  Ghana & CENTRAL & 1981 & 190.55 & 156.91 & 230.53 & RW2 \\ 
  Ghana & CENTRAL & 1982 & 184.96 & 154.79 & 219.57 & RW2 \\ 
  Ghana & CENTRAL & 1983 & 179.18 & 148.03 & 214.98 & RW2 \\ 
  Ghana & CENTRAL & 1984 & 173.47 & 141.71 & 210.68 & RW2 \\ 
  Ghana & CENTRAL & 1985 & 167.92 & 139.17 & 201.18 & RW2 \\ 
  Ghana & CENTRAL & 1986 & 161.83 & 136.46 & 190.97 & RW2 \\ 
  Ghana & CENTRAL & 1987 & 155.49 & 131.88 & 182.36 & RW2 \\ 
  Ghana & CENTRAL & 1988 & 149.07 & 125.81 & 175.89 & RW2 \\ 
  Ghana & CENTRAL & 1989 & 142.68 & 118.99 & 170.32 & RW2 \\ 
  Ghana & CENTRAL & 1990 & 136.21 & 113.03 & 162.14 & RW2 \\ 
  Ghana & CENTRAL & 1991 & 130.74 & 109.47 & 154.80 & RW2 \\ 
  Ghana & CENTRAL & 1992 & 125.93 & 106.13 & 148.90 & RW2 \\ 
  Ghana & CENTRAL & 1993 & 121.99 & 102.14 & 145.15 & RW2 \\ 
  Ghana & CENTRAL & 1994 & 118.54 & 97.49 & 142.43 & RW2 \\ 
  Ghana & CENTRAL & 1995 & 116.07 & 95.54 & 140.36 & RW2 \\ 
  Ghana & CENTRAL & 1996 & 113.38 & 93.95 & 136.47 & RW2 \\ 
  Ghana & CENTRAL & 1997 & 110.56 & 91.93 & 132.49 & RW2 \\ 
  Ghana & CENTRAL & 1998 & 107.83 & 88.85 & 130.07 & RW2 \\ 
  Ghana & CENTRAL & 1999 & 104.78 & 85.30 & 127.69 & RW2 \\ 
  Ghana & CENTRAL & 2000 & 101.52 & 82.48 & 123.95 & RW2 \\ 
  Ghana & CENTRAL & 2001 & 98.53 & 80.53 & 120.05 & RW2 \\ 
  Ghana & CENTRAL & 2002 & 95.86 & 78.43 & 116.58 & RW2 \\ 
  Ghana & CENTRAL & 2003 & 93.14 & 75.38 & 114.24 & RW2 \\ 
  Ghana & CENTRAL & 2004 & 90.73 & 72.34 & 113.00 & RW2 \\ 
  Ghana & CENTRAL & 2005 & 88.49 & 70.10 & 111.44 & RW2 \\ 
  Ghana & CENTRAL & 2006 & 85.91 & 67.93 & 108.31 & RW2 \\ 
  Ghana & CENTRAL & 2007 & 83.20 & 65.35 & 105.34 & RW2 \\ 
  Ghana & CENTRAL & 2008 & 80.48 & 62.14 & 103.60 & RW2 \\ 
  Ghana & CENTRAL & 2009 & 77.46 & 58.39 & 102.45 & RW2 \\ 
  Ghana & CENTRAL & 2010 & 74.30 & 54.88 & 100.57 & RW2 \\ 
  Ghana & CENTRAL & 2011 & 71.06 & 51.79 & 97.67 & RW2 \\ 
  Ghana & CENTRAL & 2012 & 68.01 & 48.82 & 95.16 & RW2 \\ 
  Ghana & CENTRAL & 2013 & 65.07 & 44.84 & 95.01 & RW2 \\ 
  Ghana & CENTRAL & 2014 & 62.29 & 39.51 & 98.03 & RW2 \\ 
  Ghana & CENTRAL & 2015 & 59.58 & 33.04 & 106.72 & RW2 \\ 
  Ghana & CENTRAL & 2016 & 57.02 & 26.83 & 116.89 & RW2 \\ 
  Ghana & CENTRAL & 2017 & 54.48 & 21.36 & 131.73 & RW2 \\ 
  Ghana & CENTRAL & 2018 & 52.10 & 16.73 & 151.69 & RW2 \\ 
  Ghana & CENTRAL & 2019 & 49.88 & 12.51 & 175.00 & RW2 \\ 
  Ghana & EASTERN & 1980 & 128.08 & 96.64 & 167.64 & RW2 \\ 
  Ghana & EASTERN & 1981 & 126.59 & 103.36 & 153.99 & RW2 \\ 
  Ghana & EASTERN & 1982 & 124.99 & 103.83 & 149.50 & RW2 \\ 
  Ghana & EASTERN & 1983 & 123.29 & 100.90 & 149.61 & RW2 \\ 
  Ghana & EASTERN & 1984 & 121.50 & 98.46 & 148.66 & RW2 \\ 
  Ghana & EASTERN & 1985 & 119.71 & 98.25 & 144.82 & RW2 \\ 
  Ghana & EASTERN & 1986 & 117.50 & 98.02 & 139.89 & RW2 \\ 
  Ghana & EASTERN & 1987 & 114.97 & 96.94 & 135.73 & RW2 \\ 
  Ghana & EASTERN & 1988 & 112.16 & 93.91 & 133.49 & RW2 \\ 
  Ghana & EASTERN & 1989 & 109.23 & 90.51 & 131.60 & RW2 \\ 
  Ghana & EASTERN & 1990 & 106.01 & 87.65 & 127.58 & RW2 \\ 
  Ghana & EASTERN & 1991 & 103.48 & 86.47 & 123.60 & RW2 \\ 
  Ghana & EASTERN & 1992 & 101.34 & 84.96 & 120.57 & RW2 \\ 
  Ghana & EASTERN & 1993 & 99.60 & 83.12 & 118.99 & RW2 \\ 
  Ghana & EASTERN & 1994 & 98.23 & 80.82 & 118.87 & RW2 \\ 
  Ghana & EASTERN & 1995 & 97.40 & 80.10 & 118.33 & RW2 \\ 
  Ghana & EASTERN & 1996 & 96.32 & 79.65 & 116.22 & RW2 \\ 
  Ghana & EASTERN & 1997 & 95.11 & 78.92 & 114.24 & RW2 \\ 
  Ghana & EASTERN & 1998 & 93.68 & 77.30 & 113.57 & RW2 \\ 
  Ghana & EASTERN & 1999 & 92.08 & 74.93 & 113.01 & RW2 \\ 
  Ghana & EASTERN & 2000 & 90.14 & 72.75 & 110.55 & RW2 \\ 
  Ghana & EASTERN & 2001 & 88.44 & 72.08 & 107.91 & RW2 \\ 
  Ghana & EASTERN & 2002 & 86.87 & 70.88 & 105.93 & RW2 \\ 
  Ghana & EASTERN & 2003 & 85.29 & 69.10 & 104.85 & RW2 \\ 
  Ghana & EASTERN & 2004 & 83.89 & 67.07 & 104.23 & RW2 \\ 
  Ghana & EASTERN & 2005 & 82.68 & 65.61 & 103.84 & RW2 \\ 
  Ghana & EASTERN & 2006 & 81.16 & 64.53 & 101.69 & RW2 \\ 
  Ghana & EASTERN & 2007 & 79.44 & 63.02 & 99.59 & RW2 \\ 
  Ghana & EASTERN & 2008 & 77.59 & 60.82 & 98.42 & RW2 \\ 
  Ghana & EASTERN & 2009 & 75.44 & 58.14 & 97.47 & RW2 \\ 
  Ghana & EASTERN & 2010 & 73.21 & 54.95 & 96.43 & RW2 \\ 
  Ghana & EASTERN & 2011 & 70.83 & 53.02 & 93.99 & RW2 \\ 
  Ghana & EASTERN & 2012 & 68.62 & 50.69 & 92.05 & RW2 \\ 
  Ghana & EASTERN & 2013 & 66.32 & 47.09 & 92.56 & RW2 \\ 
  Ghana & EASTERN & 2014 & 64.15 & 41.93 & 96.23 & RW2 \\ 
  Ghana & EASTERN & 2015 & 62.10 & 35.38 & 106.78 & RW2 \\ 
  Ghana & EASTERN & 2016 & 60.04 & 29.31 & 119.13 & RW2 \\ 
  Ghana & EASTERN & 2017 & 58.11 & 23.48 & 136.65 & RW2 \\ 
  Ghana & EASTERN & 2018 & 56.03 & 18.17 & 158.01 & RW2 \\ 
  Ghana & EASTERN & 2019 & 54.11 & 14.14 & 184.47 & RW2 \\ 
  Ghana & GREATER ACCRA & 1980 & 136.52 & 101.82 & 180.96 & RW2 \\ 
  Ghana & GREATER ACCRA & 1981 & 132.59 & 106.40 & 164.52 & RW2 \\ 
  Ghana & GREATER ACCRA & 1982 & 128.63 & 105.22 & 156.68 & RW2 \\ 
  Ghana & GREATER ACCRA & 1983 & 125.02 & 100.90 & 153.45 & RW2 \\ 
  Ghana & GREATER ACCRA & 1984 & 121.06 & 96.60 & 149.92 & RW2 \\ 
  Ghana & GREATER ACCRA & 1985 & 117.38 & 95.20 & 143.52 & RW2 \\ 
  Ghana & GREATER ACCRA & 1986 & 113.19 & 93.33 & 136.25 & RW2 \\ 
  Ghana & GREATER ACCRA & 1987 & 108.86 & 90.49 & 130.40 & RW2 \\ 
  Ghana & GREATER ACCRA & 1988 & 104.38 & 86.44 & 125.46 & RW2 \\ 
  Ghana & GREATER ACCRA & 1989 & 99.98 & 81.94 & 122.01 & RW2 \\ 
  Ghana & GREATER ACCRA & 1990 & 95.47 & 77.88 & 116.34 & RW2 \\ 
  Ghana & GREATER ACCRA & 1991 & 91.58 & 75.40 & 110.78 & RW2 \\ 
  Ghana & GREATER ACCRA & 1992 & 88.18 & 72.85 & 106.22 & RW2 \\ 
  Ghana & GREATER ACCRA & 1993 & 85.36 & 70.01 & 103.52 & RW2 \\ 
  Ghana & GREATER ACCRA & 1994 & 82.89 & 67.04 & 101.84 & RW2 \\ 
  Ghana & GREATER ACCRA & 1995 & 81.08 & 65.49 & 100.04 & RW2 \\ 
  Ghana & GREATER ACCRA & 1996 & 79.02 & 64.17 & 97.03 & RW2 \\ 
  Ghana & GREATER ACCRA & 1997 & 76.96 & 62.51 & 94.36 & RW2 \\ 
  Ghana & GREATER ACCRA & 1998 & 74.80 & 60.40 & 92.36 & RW2 \\ 
  Ghana & GREATER ACCRA & 1999 & 72.53 & 57.65 & 90.79 & RW2 \\ 
  Ghana & GREATER ACCRA & 2000 & 70.23 & 55.55 & 88.20 & RW2 \\ 
  Ghana & GREATER ACCRA & 2001 & 68.12 & 54.02 & 85.48 & RW2 \\ 
  Ghana & GREATER ACCRA & 2002 & 66.14 & 52.44 & 83.17 & RW2 \\ 
  Ghana & GREATER ACCRA & 2003 & 64.27 & 50.37 & 81.53 & RW2 \\ 
  Ghana & GREATER ACCRA & 2004 & 62.49 & 48.39 & 80.40 & RW2 \\ 
  Ghana & GREATER ACCRA & 2005 & 60.95 & 46.78 & 78.98 & RW2 \\ 
  Ghana & GREATER ACCRA & 2006 & 59.11 & 45.29 & 76.98 & RW2 \\ 
  Ghana & GREATER ACCRA & 2007 & 57.36 & 43.55 & 75.02 & RW2 \\ 
  Ghana & GREATER ACCRA & 2008 & 55.36 & 41.30 & 74.18 & RW2 \\ 
  Ghana & GREATER ACCRA & 2009 & 53.20 & 38.63 & 73.19 & RW2 \\ 
  Ghana & GREATER ACCRA & 2010 & 50.92 & 35.92 & 72.03 & RW2 \\ 
  Ghana & GREATER ACCRA & 2011 & 48.84 & 33.86 & 70.14 & RW2 \\ 
  Ghana & GREATER ACCRA & 2012 & 46.69 & 31.65 & 68.84 & RW2 \\ 
  Ghana & GREATER ACCRA & 2013 & 44.49 & 28.90 & 68.52 & RW2 \\ 
  Ghana & GREATER ACCRA & 2014 & 42.61 & 25.56 & 71.22 & RW2 \\ 
  Ghana & GREATER ACCRA & 2015 & 40.70 & 21.37 & 76.68 & RW2 \\ 
  Ghana & GREATER ACCRA & 2016 & 38.93 & 17.44 & 85.44 & RW2 \\ 
  Ghana & GREATER ACCRA & 2017 & 37.31 & 13.93 & 96.19 & RW2 \\ 
  Ghana & GREATER ACCRA & 2018 & 35.67 & 10.84 & 110.34 & RW2 \\ 
  Ghana & GREATER ACCRA & 2019 & 33.92 & 8.17 & 130.45 & RW2 \\ 
  Ghana & UPPER W,E \& NORTHERN & 1980 & 268.70 & 213.56 & 332.08 & RW2 \\ 
  Ghana & UPPER W,E \& NORTHERN & 1981 & 261.29 & 222.74 & 304.25 & RW2 \\ 
  Ghana & UPPER W,E \& NORTHERN & 1982 & 253.97 & 219.37 & 291.92 & RW2 \\ 
  Ghana & UPPER W,E \& NORTHERN & 1983 & 246.53 & 209.99 & 286.91 & RW2 \\ 
  Ghana & UPPER W,E \& NORTHERN & 1984 & 239.12 & 200.95 & 281.75 & RW2 \\ 
  Ghana & UPPER W,E \& NORTHERN & 1985 & 232.00 & 197.25 & 270.06 & RW2 \\ 
  Ghana & UPPER W,E \& NORTHERN & 1986 & 223.96 & 193.52 & 257.26 & RW2 \\ 
  Ghana & UPPER W,E \& NORTHERN & 1987 & 215.65 & 187.41 & 246.62 & RW2 \\ 
  Ghana & UPPER W,E \& NORTHERN & 1988 & 207.18 & 178.81 & 238.99 & RW2 \\ 
  Ghana & UPPER W,E \& NORTHERN & 1989 & 198.69 & 169.25 & 232.04 & RW2 \\ 
  Ghana & UPPER W,E \& NORTHERN & 1990 & 189.94 & 161.65 & 221.97 & RW2 \\ 
  Ghana & UPPER W,E \& NORTHERN & 1991 & 182.52 & 156.29 & 211.79 & RW2 \\ 
  Ghana & UPPER W,E \& NORTHERN & 1992 & 175.83 & 151.43 & 203.06 & RW2 \\ 
  Ghana & UPPER W,E \& NORTHERN & 1993 & 170.13 & 145.72 & 197.61 & RW2 \\ 
  Ghana & UPPER W,E \& NORTHERN & 1994 & 165.14 & 139.51 & 193.86 & RW2 \\ 
  Ghana & UPPER W,E \& NORTHERN & 1995 & 161.18 & 136.64 & 189.95 & RW2 \\ 
  Ghana & UPPER W,E \& NORTHERN & 1996 & 157.03 & 134.07 & 183.48 & RW2 \\ 
  Ghana & UPPER W,E \& NORTHERN & 1997 & 152.98 & 131.16 & 177.87 & RW2 \\ 
  Ghana & UPPER W,E \& NORTHERN & 1998 & 148.82 & 126.48 & 174.07 & RW2 \\ 
  Ghana & UPPER W,E \& NORTHERN & 1999 & 144.47 & 121.50 & 170.71 & RW2 \\ 
  Ghana & UPPER W,E \& NORTHERN & 2000 & 139.94 & 117.51 & 165.33 & RW2 \\ 
  Ghana & UPPER W,E \& NORTHERN & 2001 & 135.76 & 114.79 & 159.47 & RW2 \\ 
  Ghana & UPPER W,E \& NORTHERN & 2002 & 131.89 & 112.22 & 154.46 & RW2 \\ 
  Ghana & UPPER W,E \& NORTHERN & 2003 & 128.06 & 107.89 & 151.00 & RW2 \\ 
  Ghana & UPPER W,E \& NORTHERN & 2004 & 124.44 & 103.58 & 148.65 & RW2 \\ 
  Ghana & UPPER W,E \& NORTHERN & 2005 & 120.97 & 100.13 & 145.78 & RW2 \\ 
  Ghana & UPPER W,E \& NORTHERN & 2006 & 117.22 & 97.58 & 140.56 & RW2 \\ 
  Ghana & UPPER W,E \& NORTHERN & 2007 & 113.19 & 94.09 & 135.77 & RW2 \\ 
  Ghana & UPPER W,E \& NORTHERN & 2008 & 108.91 & 89.48 & 132.01 & RW2 \\ 
  Ghana & UPPER W,E \& NORTHERN & 2009 & 104.40 & 84.17 & 128.91 & RW2 \\ 
  Ghana & UPPER W,E \& NORTHERN & 2010 & 99.53 & 78.78 & 124.94 & RW2 \\ 
  Ghana & UPPER W,E \& NORTHERN & 2011 & 94.96 & 75.01 & 119.36 & RW2 \\ 
  Ghana & UPPER W,E \& NORTHERN & 2012 & 90.35 & 71.08 & 114.37 & RW2 \\ 
  Ghana & UPPER W,E \& NORTHERN & 2013 & 85.93 & 65.07 & 113.15 & RW2 \\ 
  Ghana & UPPER W,E \& NORTHERN & 2014 & 81.67 & 56.71 & 116.26 & RW2 \\ 
  Ghana & UPPER W,E \& NORTHERN & 2015 & 77.75 & 46.45 & 127.01 & RW2 \\ 
  Ghana & UPPER W,E \& NORTHERN & 2016 & 73.81 & 37.27 & 139.89 & RW2 \\ 
  Ghana & UPPER W,E \& NORTHERN & 2017 & 70.21 & 29.24 & 159.43 & RW2 \\ 
  Ghana & UPPER W,E \& NORTHERN & 2018 & 66.75 & 22.45 & 180.29 & RW2 \\ 
  Ghana & UPPER W,E \& NORTHERN & 2019 & 63.53 & 17.14 & 205.71 & RW2 \\ 
  Ghana & VOLTA & 1980 & 163.11 & 124.04 & 210.87 & RW2 \\ 
  Ghana & VOLTA & 1981 & 158.42 & 129.70 & 191.83 & RW2 \\ 
  Ghana & VOLTA & 1982 & 153.87 & 128.58 & 183.35 & RW2 \\ 
  Ghana & VOLTA & 1983 & 149.43 & 123.33 & 180.04 & RW2 \\ 
  Ghana & VOLTA & 1984 & 144.97 & 117.88 & 176.11 & RW2 \\ 
  Ghana & VOLTA & 1985 & 140.55 & 116.46 & 169.01 & RW2 \\ 
  Ghana & VOLTA & 1986 & 135.76 & 114.18 & 160.60 & RW2 \\ 
  Ghana & VOLTA & 1987 & 130.78 & 110.96 & 153.62 & RW2 \\ 
  Ghana & VOLTA & 1988 & 125.65 & 105.71 & 148.74 & RW2 \\ 
  Ghana & VOLTA & 1989 & 120.43 & 100.39 & 144.75 & RW2 \\ 
  Ghana & VOLTA & 1990 & 115.28 & 95.55 & 137.88 & RW2 \\ 
  Ghana & VOLTA & 1991 & 110.86 & 92.58 & 132.00 & RW2 \\ 
  Ghana & VOLTA & 1992 & 107.05 & 89.87 & 126.86 & RW2 \\ 
  Ghana & VOLTA & 1993 & 103.73 & 86.21 & 124.24 & RW2 \\ 
  Ghana & VOLTA & 1994 & 100.92 & 82.75 & 122.29 & RW2 \\ 
  Ghana & VOLTA & 1995 & 98.75 & 80.73 & 120.44 & RW2 \\ 
  Ghana & VOLTA & 1996 & 96.45 & 79.55 & 117.10 & RW2 \\ 
  Ghana & VOLTA & 1997 & 94.12 & 77.42 & 113.80 & RW2 \\ 
  Ghana & VOLTA & 1998 & 91.54 & 74.80 & 111.71 & RW2 \\ 
  Ghana & VOLTA & 1999 & 88.95 & 71.62 & 109.63 & RW2 \\ 
  Ghana & VOLTA & 2000 & 86.11 & 68.96 & 106.76 & RW2 \\ 
  Ghana & VOLTA & 2001 & 83.49 & 67.27 & 103.15 & RW2 \\ 
  Ghana & VOLTA & 2002 & 81.08 & 65.37 & 100.25 & RW2 \\ 
  Ghana & VOLTA & 2003 & 78.79 & 62.85 & 97.95 & RW2 \\ 
  Ghana & VOLTA & 2004 & 76.52 & 60.07 & 96.71 & RW2 \\ 
  Ghana & VOLTA & 2005 & 74.54 & 58.18 & 95.28 & RW2 \\ 
  Ghana & VOLTA & 2006 & 72.23 & 56.21 & 92.34 & RW2 \\ 
  Ghana & VOLTA & 2007 & 69.85 & 54.07 & 90.03 & RW2 \\ 
  Ghana & VOLTA & 2008 & 67.23 & 51.23 & 88.17 & RW2 \\ 
  Ghana & VOLTA & 2009 & 64.62 & 47.87 & 86.70 & RW2 \\ 
  Ghana & VOLTA & 2010 & 61.75 & 44.75 & 84.78 & RW2 \\ 
  Ghana & VOLTA & 2011 & 59.06 & 42.24 & 81.85 & RW2 \\ 
  Ghana & VOLTA & 2012 & 56.36 & 39.80 & 79.86 & RW2 \\ 
  Ghana & VOLTA & 2013 & 53.74 & 36.30 & 78.93 & RW2 \\ 
  Ghana & VOLTA & 2014 & 51.23 & 32.05 & 81.46 & RW2 \\ 
  Ghana & VOLTA & 2015 & 48.97 & 26.50 & 88.78 & RW2 \\ 
  Ghana & VOLTA & 2016 & 46.58 & 21.81 & 97.13 & RW2 \\ 
  Ghana & VOLTA & 2017 & 44.39 & 17.08 & 109.54 & RW2 \\ 
  Ghana & VOLTA & 2018 & 42.38 & 13.28 & 124.75 & RW2 \\ 
  Ghana & VOLTA & 2019 & 40.43 & 10.02 & 143.77 & RW2 \\ 
  Ghana & WESTERN & 1980 & 163.11 & 121.57 & 214.67 & RW2 \\ 
  Ghana & WESTERN & 1981 & 159.48 & 128.05 & 197.44 & RW2 \\ 
  Ghana & WESTERN & 1982 & 155.90 & 127.46 & 189.07 & RW2 \\ 
  Ghana & WESTERN & 1983 & 152.24 & 123.30 & 186.20 & RW2 \\ 
  Ghana & WESTERN & 1984 & 148.52 & 119.56 & 182.64 & RW2 \\ 
  Ghana & WESTERN & 1985 & 144.90 & 118.70 & 176.09 & RW2 \\ 
  Ghana & WESTERN & 1986 & 140.77 & 117.09 & 168.23 & RW2 \\ 
  Ghana & WESTERN & 1987 & 136.30 & 114.47 & 161.51 & RW2 \\ 
  Ghana & WESTERN & 1988 & 131.55 & 110.02 & 156.96 & RW2 \\ 
  Ghana & WESTERN & 1989 & 126.86 & 105.19 & 152.98 & RW2 \\ 
  Ghana & WESTERN & 1990 & 121.75 & 100.71 & 146.58 & RW2 \\ 
  Ghana & WESTERN & 1991 & 117.65 & 98.00 & 140.16 & RW2 \\ 
  Ghana & WESTERN & 1992 & 113.98 & 95.51 & 135.47 & RW2 \\ 
  Ghana & WESTERN & 1993 & 110.80 & 92.30 & 132.72 & RW2 \\ 
  Ghana & WESTERN & 1994 & 108.18 & 88.97 & 131.17 & RW2 \\ 
  Ghana & WESTERN & 1995 & 106.18 & 87.37 & 129.15 & RW2 \\ 
  Ghana & WESTERN & 1996 & 103.74 & 85.76 & 125.40 & RW2 \\ 
  Ghana & WESTERN & 1997 & 101.34 & 84.20 & 122.16 & RW2 \\ 
  Ghana & WESTERN & 1998 & 98.67 & 81.39 & 119.83 & RW2 \\ 
  Ghana & WESTERN & 1999 & 95.83 & 77.89 & 117.48 & RW2 \\ 
  Ghana & WESTERN & 2000 & 92.58 & 74.84 & 113.92 & RW2 \\ 
  Ghana & WESTERN & 2001 & 89.55 & 72.94 & 109.64 & RW2 \\ 
  Ghana & WESTERN & 2002 & 86.62 & 70.58 & 105.73 & RW2 \\ 
  Ghana & WESTERN & 2003 & 83.69 & 67.59 & 103.29 & RW2 \\ 
  Ghana & WESTERN & 2004 & 80.94 & 64.42 & 101.40 & RW2 \\ 
  Ghana & WESTERN & 2005 & 78.38 & 61.82 & 99.11 & RW2 \\ 
  Ghana & WESTERN & 2006 & 75.49 & 59.46 & 95.44 & RW2 \\ 
  Ghana & WESTERN & 2007 & 72.47 & 56.72 & 92.14 & RW2 \\ 
  Ghana & WESTERN & 2008 & 69.33 & 53.24 & 89.56 & RW2 \\ 
  Ghana & WESTERN & 2009 & 66.17 & 49.71 & 87.30 & RW2 \\ 
  Ghana & WESTERN & 2010 & 62.83 & 46.07 & 84.61 & RW2 \\ 
  Ghana & WESTERN & 2011 & 59.63 & 43.16 & 81.18 & RW2 \\ 
  Ghana & WESTERN & 2012 & 56.59 & 40.17 & 78.18 & RW2 \\ 
  Ghana & WESTERN & 2013 & 53.61 & 36.75 & 77.00 & RW2 \\ 
  Ghana & WESTERN & 2014 & 50.72 & 32.02 & 78.64 & RW2 \\ 
  Ghana & WESTERN & 2015 & 48.04 & 26.37 & 85.25 & RW2 \\ 
  Ghana & WESTERN & 2016 & 45.44 & 21.18 & 93.12 & RW2 \\ 
  Ghana & WESTERN & 2017 & 43.07 & 16.75 & 104.48 & RW2 \\ 
  Ghana & WESTERN & 2018 & 40.63 & 12.84 & 118.79 & RW2 \\ 
  Ghana & WESTERN & 2019 & 38.57 & 9.64 & 139.38 & RW2 \\ 
  Guinea & ALL & 1980 & 275.01 & 264.87 & 286.04 & IHME \\ 
  Guinea & ALL & 1980 & 287.24 & 225.44 & 357.62 & RW2 \\ 
  Guinea & ALL & 1980 & 286.90 & 266.40 & 309.10 & UN \\ 
  Guinea & ALL & 1981 & 270.62 & 261.22 & 281.23 & IHME \\ 
  Guinea & ALL & 1981 & 282.34 & 237.92 & 330.90 & RW2 \\ 
  Guinea & ALL & 1981 & 282.20 & 262.20 & 303.40 & UN \\ 
  Guinea & ALL & 1982 & 266.27 & 257.31 & 276.33 & IHME \\ 
  Guinea & ALL & 1982 & 277.54 & 237.82 & 321.07 & RW2 \\ 
  Guinea & ALL & 1982 & 277.40 & 258.00 & 297.60 & UN \\ 
  Guinea & ALL & 1983 & 261.93 & 253.03 & 271.47 & IHME \\ 
  Guinea & ALL & 1983 & 272.48 & 229.60 & 320.00 & RW2 \\ 
  Guinea & ALL & 1983 & 272.60 & 253.70 & 292.40 & UN \\ 
  Guinea & ALL & 1984 & 257.35 & 248.57 & 266.68 & IHME \\ 
  Guinea & ALL & 1984 & 267.88 & 221.39 & 319.43 & RW2 \\ 
  Guinea & ALL & 1984 & 267.80 & 249.70 & 287.00 & UN \\ 
  Guinea & ALL & 1985 & 252.77 & 244.34 & 261.53 & IHME \\ 
  Guinea & ALL & 1985 & 262.75 & 220.68 & 309.87 & RW2 \\ 
  Guinea & ALL & 1985 & 263.10 & 245.60 & 281.60 & UN \\ 
  Guinea & ALL & 1986 & 248.32 & 240.17 & 256.37 & IHME \\ 
  Guinea & ALL & 1986 & 257.89 & 219.23 & 300.69 & RW2 \\ 
  Guinea & ALL & 1986 & 258.40 & 241.40 & 276.60 & UN \\ 
  Guinea & ALL & 1987 & 243.83 & 235.90 & 251.72 & IHME \\ 
  Guinea & ALL & 1987 & 253.08 & 216.78 & 293.75 & RW2 \\ 
  Guinea & ALL & 1987 & 253.60 & 237.20 & 271.30 & UN \\ 
  Guinea & ALL & 1988 & 239.07 & 231.73 & 246.74 & IHME \\ 
  Guinea & ALL & 1988 & 248.02 & 210.53 & 289.53 & RW2 \\ 
  Guinea & ALL & 1988 & 248.80 & 232.90 & 266.00 & UN \\ 
  Guinea & ALL & 1989 & 234.15 & 227.00 & 241.63 & IHME \\ 
  Guinea & ALL & 1989 & 242.86 & 204.09 & 286.22 & RW2 \\ 
  Guinea & ALL & 1989 & 243.70 & 228.00 & 260.60 & UN \\ 
  Guinea & ALL & 1990 & 229.11 & 222.30 & 236.52 & IHME \\ 
  Guinea & ALL & 1990 & 237.67 & 200.11 & 280.31 & RW2 \\ 
  Guinea & ALL & 1990 & 238.20 & 223.10 & 254.80 & UN \\ 
  Guinea & ALL & 1991 & 223.98 & 217.18 & 231.18 & IHME \\ 
  Guinea & ALL & 1991 & 232.06 & 197.21 & 270.57 & RW2 \\ 
  Guinea & ALL & 1991 & 232.40 & 217.70 & 248.50 & UN \\ 
  Guinea & ALL & 1992 & 219.01 & 212.18 & 226.18 & IHME \\ 
  Guinea & ALL & 1992 & 226.14 & 192.78 & 262.92 & RW2 \\ 
  Guinea & ALL & 1992 & 226.20 & 211.80 & 241.90 & UN \\ 
  Guinea & ALL & 1993 & 213.79 & 207.06 & 220.79 & IHME \\ 
  Guinea & ALL & 1993 & 219.92 & 186.45 & 257.36 & RW2 \\ 
  Guinea & ALL & 1993 & 219.70 & 205.70 & 234.80 & UN \\ 
  Guinea & ALL & 1994 & 208.42 & 201.89 & 215.20 & IHME \\ 
  Guinea & ALL & 1994 & 213.27 & 178.63 & 253.13 & RW2 \\ 
  Guinea & ALL & 1994 & 212.90 & 199.30 & 227.60 & UN \\ 
  Guinea & ALL & 1995 & 202.81 & 196.53 & 209.34 & IHME \\ 
  Guinea & ALL & 1995 & 206.46 & 172.90 & 244.46 & RW2 \\ 
  Guinea & ALL & 1995 & 205.90 & 192.80 & 220.30 & UN \\ 
  Guinea & ALL & 1996 & 196.88 & 190.79 & 203.22 & IHME \\ 
  Guinea & ALL & 1996 & 199.24 & 168.36 & 234.71 & RW2 \\ 
  Guinea & ALL & 1996 & 199.00 & 186.10 & 212.80 & UN \\ 
  Guinea & ALL & 1997 & 191.16 & 185.11 & 197.50 & IHME \\ 
  Guinea & ALL & 1997 & 191.87 & 162.89 & 224.75 & RW2 \\ 
  Guinea & ALL & 1997 & 192.00 & 179.50 & 205.40 & UN \\ 
  Guinea & ALL & 1998 & 185.70 & 179.57 & 192.00 & IHME \\ 
  Guinea & ALL & 1998 & 184.48 & 155.54 & 218.35 & RW2 \\ 
  Guinea & ALL & 1998 & 184.80 & 172.70 & 197.80 & UN \\ 
  Guinea & ALL & 1999 & 180.22 & 173.98 & 186.66 & IHME \\ 
  Guinea & ALL & 1999 & 177.03 & 147.31 & 211.09 & RW2 \\ 
  Guinea & ALL & 1999 & 177.50 & 165.80 & 190.30 & UN \\ 
  Guinea & ALL & 2000 & 175.08 & 168.89 & 181.33 & IHME \\ 
  Guinea & ALL & 2000 & 169.56 & 140.79 & 202.41 & RW2 \\ 
  Guinea & ALL & 2000 & 170.20 & 158.60 & 182.70 & UN \\ 
  Guinea & ALL & 2001 & 169.94 & 163.71 & 176.28 & IHME \\ 
  Guinea & ALL & 2001 & 162.49 & 135.93 & 192.85 & RW2 \\ 
  Guinea & ALL & 2001 & 162.90 & 151.40 & 175.30 & UN \\ 
  Guinea & ALL & 2002 & 164.76 & 158.48 & 171.12 & IHME \\ 
  Guinea & ALL & 2002 & 155.72 & 131.04 & 184.13 & RW2 \\ 
  Guinea & ALL & 2002 & 155.80 & 144.20 & 168.20 & UN \\ 
  Guinea & ALL & 2003 & 159.97 & 153.56 & 166.57 & IHME \\ 
  Guinea & ALL & 2003 & 149.40 & 125.12 & 177.51 & RW2 \\ 
  Guinea & ALL & 2003 & 149.00 & 137.50 & 161.40 & UN \\ 
  Guinea & ALL & 2004 & 155.43 & 148.79 & 162.32 & IHME \\ 
  Guinea & ALL & 2004 & 143.23 & 117.93 & 172.57 & RW2 \\ 
  Guinea & ALL & 2004 & 142.70 & 131.00 & 155.30 & UN \\ 
  Guinea & ALL & 2005 & 151.05 & 144.27 & 158.14 & IHME \\ 
  Guinea & ALL & 2005 & 137.70 & 112.75 & 167.36 & RW2 \\ 
  Guinea & ALL & 2005 & 136.70 & 124.90 & 149.60 & UN \\ 
  Guinea & ALL & 2006 & 146.68 & 139.40 & 153.93 & IHME \\ 
  Guinea & ALL & 2006 & 132.22 & 109.04 & 159.42 & RW2 \\ 
  Guinea & ALL & 2006 & 131.20 & 118.90 & 144.50 & UN \\ 
  Guinea & ALL & 2007 & 142.22 & 134.66 & 149.79 & IHME \\ 
  Guinea & ALL & 2007 & 127.03 & 105.29 & 152.50 & RW2 \\ 
  Guinea & ALL & 2007 & 125.90 & 113.20 & 140.10 & UN \\ 
  Guinea & ALL & 2008 & 137.63 & 129.59 & 145.73 & IHME \\ 
  Guinea & ALL & 2008 & 122.19 & 100.43 & 147.98 & RW2 \\ 
  Guinea & ALL & 2008 & 121.00 & 107.60 & 136.20 & UN \\ 
  Guinea & ALL & 2009 & 133.35 & 125.11 & 141.98 & IHME \\ 
  Guinea & ALL & 2009 & 117.29 & 94.84 & 144.61 & RW2 \\ 
  Guinea & ALL & 2009 & 116.30 & 102.00 & 133.00 & UN \\ 
  Guinea & ALL & 2010 & 129.00 & 120.10 & 138.56 & IHME \\ 
  Guinea & ALL & 2010 & 112.64 & 89.76 & 141.12 & RW2 \\ 
  Guinea & ALL & 2010 & 111.90 & 96.60 & 129.80 & UN \\ 
  Guinea & ALL & 2011 & 124.62 & 115.19 & 134.41 & IHME \\ 
  Guinea & ALL & 2011 & 108.25 & 86.77 & 134.27 & RW2 \\ 
  Guinea & ALL & 2011 & 107.70 & 91.60 & 126.80 & UN \\ 
  Guinea & ALL & 2012 & 120.40 & 110.57 & 130.95 & IHME \\ 
  Guinea & ALL & 2012 & 103.94 & 83.79 & 128.17 & RW2 \\ 
  Guinea & ALL & 2012 & 104.00 & 86.90 & 124.70 & UN \\ 
  Guinea & ALL & 2013 & 116.21 & 105.76 & 127.21 & IHME \\ 
  Guinea & ALL & 2013 & 99.87 & 78.10 & 126.66 & RW2 \\ 
  Guinea & ALL & 2013 & 100.40 & 81.80 & 123.40 & UN \\ 
  Guinea & ALL & 2014 & 112.71 & 101.75 & 124.64 & IHME \\ 
  Guinea & ALL & 2014 & 95.84 & 68.28 & 132.77 & RW2 \\ 
  Guinea & ALL & 2014 & 97.00 & 76.70 & 122.60 & UN \\ 
  Guinea & ALL & 2015 & 108.75 & 97.10 & 121.41 & IHME \\ 
  Guinea & ALL & 2015 & 91.85 & 55.56 & 148.22 & RW2 \\ 
  Guinea & ALL & 2015 & 93.70 & 71.80 & 122.00 & UN \\ 
  Guinea & ALL & 2016 & 88.33 & 45.01 & 166.99 & RW2 \\ 
  Guinea & ALL & 2017 & 84.56 & 35.17 & 191.05 & RW2 \\ 
  Guinea & ALL & 2018 & 81.08 & 27.18 & 222.59 & RW2 \\ 
  Guinea & ALL & 2019 & 77.63 & 19.83 & 256.93 & RW2 \\ 
  Guinea & ALL & 80-84 & 271.25 & 286.80 & 256.24 & HT-Direct \\ 
  Guinea & ALL & 85-89 & 238.17 & 249.94 & 226.79 & HT-Direct \\ 
  Guinea & ALL & 90-94 & 220.60 & 229.75 & 211.71 & HT-Direct \\ 
  Guinea & ALL & 95-99 & 186.23 & 193.90 & 178.80 & HT-Direct \\ 
  Guinea & ALL & 00-04 & 157.68 & 165.81 & 149.89 & HT-Direct \\ 
  Guinea & ALL & 05-09 & 140.23 & 153.27 & 128.13 & HT-Direct \\ 
  Guinea & ALL & 10-14 & 106.07 & 120.46 & 93.21 & HT-Direct \\ 
  Guinea & ALL & 15-19 & 84.56 & 35.67 & 188.23 & RW2 \\ 
  Guinea & CENTRAL GUINEA & 1980 & 278.21 & 218.37 & 347.67 & RW2 \\ 
  Guinea & CENTRAL GUINEA & 1981 & 272.27 & 228.00 & 322.00 & RW2 \\ 
  Guinea & CENTRAL GUINEA & 1982 & 266.05 & 227.44 & 308.58 & RW2 \\ 
  Guinea & CENTRAL GUINEA & 1983 & 260.02 & 220.40 & 304.06 & RW2 \\ 
  Guinea & CENTRAL GUINEA & 1984 & 254.08 & 212.84 & 299.85 & RW2 \\ 
  Guinea & CENTRAL GUINEA & 1985 & 248.38 & 211.00 & 290.01 & RW2 \\ 
  Guinea & CENTRAL GUINEA & 1986 & 242.94 & 209.06 & 279.84 & RW2 \\ 
  Guinea & CENTRAL GUINEA & 1987 & 237.70 & 205.97 & 272.47 & RW2 \\ 
  Guinea & CENTRAL GUINEA & 1988 & 232.71 & 200.37 & 268.13 & RW2 \\ 
  Guinea & CENTRAL GUINEA & 1989 & 227.84 & 194.60 & 264.39 & RW2 \\ 
  Guinea & CENTRAL GUINEA & 1990 & 223.19 & 191.14 & 258.86 & RW2 \\ 
  Guinea & CENTRAL GUINEA & 1991 & 218.46 & 189.20 & 250.80 & RW2 \\ 
  Guinea & CENTRAL GUINEA & 1992 & 213.47 & 185.52 & 244.03 & RW2 \\ 
  Guinea & CENTRAL GUINEA & 1993 & 208.23 & 180.12 & 239.36 & RW2 \\ 
  Guinea & CENTRAL GUINEA & 1994 & 203.09 & 173.57 & 235.78 & RW2 \\ 
  Guinea & CENTRAL GUINEA & 1995 & 197.49 & 168.63 & 229.54 & RW2 \\ 
  Guinea & CENTRAL GUINEA & 1996 & 192.15 & 165.66 & 221.42 & RW2 \\ 
  Guinea & CENTRAL GUINEA & 1997 & 186.73 & 161.98 & 214.63 & RW2 \\ 
  Guinea & CENTRAL GUINEA & 1998 & 181.35 & 156.79 & 209.30 & RW2 \\ 
  Guinea & CENTRAL GUINEA & 1999 & 175.98 & 149.69 & 205.09 & RW2 \\ 
  Guinea & CENTRAL GUINEA & 2000 & 170.52 & 145.04 & 199.13 & RW2 \\ 
  Guinea & CENTRAL GUINEA & 2001 & 165.04 & 141.73 & 191.01 & RW2 \\ 
  Guinea & CENTRAL GUINEA & 2002 & 159.37 & 137.68 & 183.75 & RW2 \\ 
  Guinea & CENTRAL GUINEA & 2003 & 153.51 & 131.71 & 178.52 & RW2 \\ 
  Guinea & CENTRAL GUINEA & 2004 & 147.62 & 124.54 & 174.02 & RW2 \\ 
  Guinea & CENTRAL GUINEA & 2005 & 141.61 & 118.52 & 168.56 & RW2 \\ 
  Guinea & CENTRAL GUINEA & 2006 & 135.66 & 113.97 & 160.94 & RW2 \\ 
  Guinea & CENTRAL GUINEA & 2007 & 129.65 & 108.36 & 154.07 & RW2 \\ 
  Guinea & CENTRAL GUINEA & 2008 & 123.82 & 101.89 & 149.41 & RW2 \\ 
  Guinea & CENTRAL GUINEA & 2009 & 117.96 & 94.92 & 145.44 & RW2 \\ 
  Guinea & CENTRAL GUINEA & 2010 & 112.42 & 88.47 & 141.76 & RW2 \\ 
  Guinea & CENTRAL GUINEA & 2011 & 106.96 & 83.25 & 136.07 & RW2 \\ 
  Guinea & CENTRAL GUINEA & 2012 & 101.79 & 77.86 & 131.22 & RW2 \\ 
  Guinea & CENTRAL GUINEA & 2013 & 96.86 & 70.77 & 130.42 & RW2 \\ 
  Guinea & CENTRAL GUINEA & 2014 & 92.13 & 61.60 & 134.02 & RW2 \\ 
  Guinea & CENTRAL GUINEA & 2015 & 87.54 & 50.81 & 144.96 & RW2 \\ 
  Guinea & CENTRAL GUINEA & 2016 & 83.03 & 41.18 & 158.38 & RW2 \\ 
  Guinea & CENTRAL GUINEA & 2017 & 78.98 & 32.63 & 177.04 & RW2 \\ 
  Guinea & CENTRAL GUINEA & 2018 & 74.78 & 25.13 & 200.90 & RW2 \\ 
  Guinea & CENTRAL GUINEA & 2019 & 71.04 & 19.03 & 231.52 & RW2 \\ 
  Guinea & CONAKRY & 1980 & 232.66 & 171.89 & 308.12 & RW2 \\ 
  Guinea & CONAKRY & 1981 & 223.73 & 177.47 & 278.83 & RW2 \\ 
  Guinea & CONAKRY & 1982 & 215.15 & 175.45 & 261.02 & RW2 \\ 
  Guinea & CONAKRY & 1983 & 206.56 & 168.31 & 250.88 & RW2 \\ 
  Guinea & CONAKRY & 1984 & 198.41 & 160.84 & 242.78 & RW2 \\ 
  Guinea & CONAKRY & 1985 & 190.61 & 156.43 & 230.12 & RW2 \\ 
  Guinea & CONAKRY & 1986 & 183.17 & 152.71 & 218.41 & RW2 \\ 
  Guinea & CONAKRY & 1987 & 176.14 & 147.62 & 208.85 & RW2 \\ 
  Guinea & CONAKRY & 1988 & 169.52 & 141.53 & 201.69 & RW2 \\ 
  Guinea & CONAKRY & 1989 & 163.11 & 134.95 & 195.57 & RW2 \\ 
  Guinea & CONAKRY & 1990 & 157.02 & 130.13 & 187.90 & RW2 \\ 
  Guinea & CONAKRY & 1991 & 150.79 & 126.16 & 179.21 & RW2 \\ 
  Guinea & CONAKRY & 1992 & 144.40 & 121.70 & 171.16 & RW2 \\ 
  Guinea & CONAKRY & 1993 & 138.08 & 115.79 & 164.52 & RW2 \\ 
  Guinea & CONAKRY & 1994 & 131.53 & 108.61 & 158.07 & RW2 \\ 
  Guinea & CONAKRY & 1995 & 125.10 & 103.25 & 150.47 & RW2 \\ 
  Guinea & CONAKRY & 1996 & 118.87 & 98.90 & 142.29 & RW2 \\ 
  Guinea & CONAKRY & 1997 & 112.73 & 94.03 & 134.54 & RW2 \\ 
  Guinea & CONAKRY & 1998 & 107.06 & 88.11 & 129.02 & RW2 \\ 
  Guinea & CONAKRY & 1999 & 101.52 & 82.02 & 124.11 & RW2 \\ 
  Guinea & CONAKRY & 2000 & 96.41 & 77.27 & 118.89 & RW2 \\ 
  Guinea & CONAKRY & 2001 & 91.61 & 73.35 & 113.23 & RW2 \\ 
  Guinea & CONAKRY & 2002 & 87.33 & 69.69 & 108.09 & RW2 \\ 
  Guinea & CONAKRY & 2003 & 83.13 & 65.59 & 104.03 & RW2 \\ 
  Guinea & CONAKRY & 2004 & 79.36 & 61.76 & 101.03 & RW2 \\ 
  Guinea & CONAKRY & 2005 & 75.73 & 58.51 & 97.55 & RW2 \\ 
  Guinea & CONAKRY & 2006 & 72.10 & 55.52 & 93.31 & RW2 \\ 
  Guinea & CONAKRY & 2007 & 68.51 & 52.25 & 89.40 & RW2 \\ 
  Guinea & CONAKRY & 2008 & 65.09 & 48.57 & 86.76 & RW2 \\ 
  Guinea & CONAKRY & 2009 & 61.56 & 44.49 & 84.92 & RW2 \\ 
  Guinea & CONAKRY & 2010 & 58.14 & 40.77 & 83.05 & RW2 \\ 
  Guinea & CONAKRY & 2011 & 54.66 & 37.11 & 80.67 & RW2 \\ 
  Guinea & CONAKRY & 2012 & 51.46 & 33.64 & 78.92 & RW2 \\ 
  Guinea & CONAKRY & 2013 & 48.45 & 29.75 & 79.18 & RW2 \\ 
  Guinea & CONAKRY & 2014 & 45.62 & 25.40 & 81.74 & RW2 \\ 
  Guinea & CONAKRY & 2015 & 42.89 & 20.76 & 88.41 & RW2 \\ 
  Guinea & CONAKRY & 2016 & 40.37 & 16.46 & 96.27 & RW2 \\ 
  Guinea & CONAKRY & 2017 & 37.91 & 12.83 & 107.95 & RW2 \\ 
  Guinea & CONAKRY & 2018 & 35.65 & 9.84 & 123.89 & RW2 \\ 
  Guinea & CONAKRY & 2019 & 33.57 & 7.20 & 142.46 & RW2 \\ 
  Guinea & FOREST GUINEA & 1980 & 285.04 & 221.53 & 356.84 & RW2 \\ 
  Guinea & FOREST GUINEA & 1981 & 286.60 & 240.38 & 338.31 & RW2 \\ 
  Guinea & FOREST GUINEA & 1982 & 288.14 & 247.14 & 332.85 & RW2 \\ 
  Guinea & FOREST GUINEA & 1983 & 289.52 & 246.26 & 336.77 & RW2 \\ 
  Guinea & FOREST GUINEA & 1984 & 290.73 & 245.54 & 340.20 & RW2 \\ 
  Guinea & FOREST GUINEA & 1985 & 291.59 & 249.32 & 338.10 & RW2 \\ 
  Guinea & FOREST GUINEA & 1986 & 292.00 & 252.72 & 334.54 & RW2 \\ 
  Guinea & FOREST GUINEA & 1987 & 291.63 & 254.25 & 332.10 & RW2 \\ 
  Guinea & FOREST GUINEA & 1988 & 290.31 & 252.00 & 332.16 & RW2 \\ 
  Guinea & FOREST GUINEA & 1989 & 288.05 & 248.31 & 331.85 & RW2 \\ 
  Guinea & FOREST GUINEA & 1990 & 284.28 & 245.80 & 327.27 & RW2 \\ 
  Guinea & FOREST GUINEA & 1991 & 279.10 & 242.95 & 317.96 & RW2 \\ 
  Guinea & FOREST GUINEA & 1992 & 272.06 & 237.99 & 308.99 & RW2 \\ 
  Guinea & FOREST GUINEA & 1993 & 263.26 & 229.11 & 301.13 & RW2 \\ 
  Guinea & FOREST GUINEA & 1994 & 253.24 & 217.99 & 292.96 & RW2 \\ 
  Guinea & FOREST GUINEA & 1995 & 242.01 & 208.09 & 279.67 & RW2 \\ 
  Guinea & FOREST GUINEA & 1996 & 230.17 & 199.24 & 264.23 & RW2 \\ 
  Guinea & FOREST GUINEA & 1997 & 218.34 & 190.19 & 250.01 & RW2 \\ 
  Guinea & FOREST GUINEA & 1998 & 206.46 & 178.50 & 238.32 & RW2 \\ 
  Guinea & FOREST GUINEA & 1999 & 194.87 & 165.84 & 227.40 & RW2 \\ 
  Guinea & FOREST GUINEA & 2000 & 183.53 & 155.54 & 215.33 & RW2 \\ 
  Guinea & FOREST GUINEA & 2001 & 172.54 & 147.44 & 201.14 & RW2 \\ 
  Guinea & FOREST GUINEA & 2002 & 162.02 & 138.61 & 188.16 & RW2 \\ 
  Guinea & FOREST GUINEA & 2003 & 151.91 & 128.59 & 178.47 & RW2 \\ 
  Guinea & FOREST GUINEA & 2004 & 142.55 & 118.57 & 170.43 & RW2 \\ 
  Guinea & FOREST GUINEA & 2005 & 133.98 & 110.00 & 161.81 & RW2 \\ 
  Guinea & FOREST GUINEA & 2006 & 126.10 & 103.51 & 152.14 & RW2 \\ 
  Guinea & FOREST GUINEA & 2007 & 119.05 & 97.46 & 144.03 & RW2 \\ 
  Guinea & FOREST GUINEA & 2008 & 112.81 & 91.12 & 138.14 & RW2 \\ 
  Guinea & FOREST GUINEA & 2009 & 107.38 & 85.38 & 133.71 & RW2 \\ 
  Guinea & FOREST GUINEA & 2010 & 102.48 & 80.29 & 129.63 & RW2 \\ 
  Guinea & FOREST GUINEA & 2011 & 98.00 & 76.66 & 124.29 & RW2 \\ 
  Guinea & FOREST GUINEA & 2012 & 93.91 & 72.80 & 120.08 & RW2 \\ 
  Guinea & FOREST GUINEA & 2013 & 89.96 & 67.36 & 119.66 & RW2 \\ 
  Guinea & FOREST GUINEA & 2014 & 86.12 & 59.00 & 124.59 & RW2 \\ 
  Guinea & FOREST GUINEA & 2015 & 82.46 & 48.69 & 137.45 & RW2 \\ 
  Guinea & FOREST GUINEA & 2016 & 78.88 & 39.32 & 152.47 & RW2 \\ 
  Guinea & FOREST GUINEA & 2017 & 75.60 & 31.27 & 172.91 & RW2 \\ 
  Guinea & FOREST GUINEA & 2018 & 72.12 & 24.14 & 197.91 & RW2 \\ 
  Guinea & FOREST GUINEA & 2019 & 69.23 & 18.27 & 232.23 & RW2 \\ 
  Guinea & LOWER GUINEA & 1980 & 290.62 & 226.63 & 363.88 & RW2 \\ 
  Guinea & LOWER GUINEA & 1981 & 283.39 & 236.35 & 335.40 & RW2 \\ 
  Guinea & LOWER GUINEA & 1982 & 276.16 & 235.26 & 321.24 & RW2 \\ 
  Guinea & LOWER GUINEA & 1983 & 268.69 & 227.77 & 314.59 & RW2 \\ 
  Guinea & LOWER GUINEA & 1984 & 261.91 & 220.25 & 308.20 & RW2 \\ 
  Guinea & LOWER GUINEA & 1985 & 255.10 & 217.06 & 297.51 & RW2 \\ 
  Guinea & LOWER GUINEA & 1986 & 248.73 & 214.41 & 286.54 & RW2 \\ 
  Guinea & LOWER GUINEA & 1987 & 242.84 & 210.33 & 278.34 & RW2 \\ 
  Guinea & LOWER GUINEA & 1988 & 237.13 & 204.31 & 272.55 & RW2 \\ 
  Guinea & LOWER GUINEA & 1989 & 231.82 & 198.09 & 268.82 & RW2 \\ 
  Guinea & LOWER GUINEA & 1990 & 226.74 & 193.93 & 262.74 & RW2 \\ 
  Guinea & LOWER GUINEA & 1991 & 221.60 & 191.78 & 254.57 & RW2 \\ 
  Guinea & LOWER GUINEA & 1992 & 216.55 & 188.17 & 247.64 & RW2 \\ 
  Guinea & LOWER GUINEA & 1993 & 211.24 & 182.16 & 243.57 & RW2 \\ 
  Guinea & LOWER GUINEA & 1994 & 205.93 & 175.79 & 239.91 & RW2 \\ 
  Guinea & LOWER GUINEA & 1995 & 200.12 & 170.96 & 232.88 & RW2 \\ 
  Guinea & LOWER GUINEA & 1996 & 194.53 & 167.49 & 224.46 & RW2 \\ 
  Guinea & LOWER GUINEA & 1997 & 188.76 & 163.43 & 217.28 & RW2 \\ 
  Guinea & LOWER GUINEA & 1998 & 182.92 & 157.26 & 211.97 & RW2 \\ 
  Guinea & LOWER GUINEA & 1999 & 176.70 & 149.68 & 207.34 & RW2 \\ 
  Guinea & LOWER GUINEA & 2000 & 170.64 & 144.46 & 200.10 & RW2 \\ 
  Guinea & LOWER GUINEA & 2001 & 164.21 & 140.27 & 191.29 & RW2 \\ 
  Guinea & LOWER GUINEA & 2002 & 157.67 & 135.26 & 183.29 & RW2 \\ 
  Guinea & LOWER GUINEA & 2003 & 150.78 & 127.85 & 176.66 & RW2 \\ 
  Guinea & LOWER GUINEA & 2004 & 143.77 & 119.97 & 171.49 & RW2 \\ 
  Guinea & LOWER GUINEA & 2005 & 136.44 & 112.81 & 164.36 & RW2 \\ 
  Guinea & LOWER GUINEA & 2006 & 129.04 & 106.53 & 155.34 & RW2 \\ 
  Guinea & LOWER GUINEA & 2007 & 121.55 & 100.04 & 146.68 & RW2 \\ 
  Guinea & LOWER GUINEA & 2008 & 113.94 & 92.37 & 139.89 & RW2 \\ 
  Guinea & LOWER GUINEA & 2009 & 106.52 & 84.01 & 133.68 & RW2 \\ 
  Guinea & LOWER GUINEA & 2010 & 99.22 & 75.78 & 128.13 & RW2 \\ 
  Guinea & LOWER GUINEA & 2011 & 92.31 & 69.39 & 121.07 & RW2 \\ 
  Guinea & LOWER GUINEA & 2012 & 85.73 & 62.56 & 114.76 & RW2 \\ 
  Guinea & LOWER GUINEA & 2013 & 79.53 & 54.81 & 111.84 & RW2 \\ 
  Guinea & LOWER GUINEA & 2014 & 73.69 & 46.19 & 112.94 & RW2 \\ 
  Guinea & LOWER GUINEA & 2015 & 68.32 & 36.89 & 119.77 & RW2 \\ 
  Guinea & LOWER GUINEA & 2016 & 63.22 & 29.02 & 128.52 & RW2 \\ 
  Guinea & LOWER GUINEA & 2017 & 58.41 & 22.38 & 140.26 & RW2 \\ 
  Guinea & LOWER GUINEA & 2018 & 54.07 & 16.86 & 156.58 & RW2 \\ 
  Guinea & LOWER GUINEA & 2019 & 50.02 & 12.31 & 176.70 & RW2 \\ 
  Guinea & UPPER GUINEA & 1980 & 325.72 & 257.77 & 403.27 & RW2 \\ 
  Guinea & UPPER GUINEA & 1981 & 320.29 & 269.97 & 375.66 & RW2 \\ 
  Guinea & UPPER GUINEA & 1982 & 314.95 & 270.74 & 362.29 & RW2 \\ 
  Guinea & UPPER GUINEA & 1983 & 309.31 & 264.34 & 358.88 & RW2 \\ 
  Guinea & UPPER GUINEA & 1984 & 303.93 & 257.73 & 354.56 & RW2 \\ 
  Guinea & UPPER GUINEA & 1985 & 298.30 & 255.15 & 344.47 & RW2 \\ 
  Guinea & UPPER GUINEA & 1986 & 292.45 & 254.02 & 334.76 & RW2 \\ 
  Guinea & UPPER GUINEA & 1987 & 286.65 & 250.77 & 325.65 & RW2 \\ 
  Guinea & UPPER GUINEA & 1988 & 280.50 & 244.07 & 320.07 & RW2 \\ 
  Guinea & UPPER GUINEA & 1989 & 273.92 & 236.18 & 315.46 & RW2 \\ 
  Guinea & UPPER GUINEA & 1990 & 267.23 & 230.83 & 307.56 & RW2 \\ 
  Guinea & UPPER GUINEA & 1991 & 260.06 & 226.67 & 295.78 & RW2 \\ 
  Guinea & UPPER GUINEA & 1992 & 252.12 & 220.89 & 285.89 & RW2 \\ 
  Guinea & UPPER GUINEA & 1993 & 243.97 & 212.93 & 278.61 & RW2 \\ 
  Guinea & UPPER GUINEA & 1994 & 235.75 & 203.07 & 272.27 & RW2 \\ 
  Guinea & UPPER GUINEA & 1995 & 227.14 & 195.18 & 262.58 & RW2 \\ 
  Guinea & UPPER GUINEA & 1996 & 219.10 & 190.26 & 250.37 & RW2 \\ 
  Guinea & UPPER GUINEA & 1997 & 211.56 & 184.64 & 241.69 & RW2 \\ 
  Guinea & UPPER GUINEA & 1998 & 204.75 & 177.44 & 235.27 & RW2 \\ 
  Guinea & UPPER GUINEA & 1999 & 198.41 & 169.69 & 230.37 & RW2 \\ 
  Guinea & UPPER GUINEA & 2000 & 192.83 & 164.65 & 224.85 & RW2 \\ 
  Guinea & UPPER GUINEA & 2001 & 187.90 & 161.62 & 217.34 & RW2 \\ 
  Guinea & UPPER GUINEA & 2002 & 183.44 & 158.60 & 211.32 & RW2 \\ 
  Guinea & UPPER GUINEA & 2003 & 179.50 & 153.99 & 208.03 & RW2 \\ 
  Guinea & UPPER GUINEA & 2004 & 176.11 & 148.93 & 206.26 & RW2 \\ 
  Guinea & UPPER GUINEA & 2005 & 173.16 & 145.68 & 204.38 & RW2 \\ 
  Guinea & UPPER GUINEA & 2006 & 170.63 & 144.84 & 200.41 & RW2 \\ 
  Guinea & UPPER GUINEA & 2007 & 168.72 & 143.51 & 197.82 & RW2 \\ 
  Guinea & UPPER GUINEA & 2008 & 167.20 & 140.50 & 198.03 & RW2 \\ 
  Guinea & UPPER GUINEA & 2009 & 165.83 & 136.82 & 199.73 & RW2 \\ 
  Guinea & UPPER GUINEA & 2010 & 164.92 & 133.48 & 202.29 & RW2 \\ 
  Guinea & UPPER GUINEA & 2011 & 164.10 & 132.74 & 201.85 & RW2 \\ 
  Guinea & UPPER GUINEA & 2012 & 163.37 & 131.06 & 202.60 & RW2 \\ 
  Guinea & UPPER GUINEA & 2013 & 162.68 & 125.87 & 208.84 & RW2 \\ 
  Guinea & UPPER GUINEA & 2014 & 161.82 & 115.16 & 223.41 & RW2 \\ 
  Guinea & UPPER GUINEA & 2015 & 161.32 & 100.29 & 250.49 & RW2 \\ 
  Guinea & UPPER GUINEA & 2016 & 160.44 & 85.52 & 281.69 & RW2 \\ 
  Guinea & UPPER GUINEA & 2017 & 160.02 & 71.56 & 321.40 & RW2 \\ 
  Guinea & UPPER GUINEA & 2018 & 158.92 & 58.59 & 367.00 & RW2 \\ 
  Guinea & UPPER GUINEA & 2019 & 158.65 & 47.28 & 420.67 & RW2 \\ 
  Kenya & ALL & 1980 & 107.20 & 103.15 & 111.60 & IHME \\ 
  Kenya & ALL & 1980 & 106.95 & 78.24 & 144.05 & RW2 \\ 
  Kenya & ALL & 1980 & 108.70 & 102.10 & 115.60 & UN \\ 
  Kenya & ALL & 1981 & 104.26 & 100.38 & 108.30 & IHME \\ 
  Kenya & ALL & 1981 & 104.65 & 83.81 & 129.17 & RW2 \\ 
  Kenya & ALL & 1981 & 105.30 & 98.90 & 111.90 & UN \\ 
  Kenya & ALL & 1982 & 101.70 & 97.96 & 105.39 & IHME \\ 
  Kenya & ALL & 1982 & 102.39 & 84.05 & 124.39 & RW2 \\ 
  Kenya & ALL & 1982 & 102.30 & 96.00 & 108.80 & UN \\ 
  Kenya & ALL & 1983 & 99.56 & 95.89 & 103.19 & IHME \\ 
  Kenya & ALL & 1983 & 100.18 & 80.72 & 124.48 & RW2 \\ 
  Kenya & ALL & 1983 & 99.80 & 93.50 & 106.20 & UN \\ 
  Kenya & ALL & 1984 & 97.65 & 94.27 & 101.08 & IHME \\ 
  Kenya & ALL & 1984 & 98.53 & 77.66 & 125.14 & RW2 \\ 
  Kenya & ALL & 1984 & 97.90 & 91.70 & 104.30 & UN \\ 
  Kenya & ALL & 1985 & 95.93 & 92.69 & 99.53 & IHME \\ 
  Kenya & ALL & 1985 & 96.69 & 77.30 & 119.90 & RW2 \\ 
  Kenya & ALL & 1985 & 96.80 & 90.60 & 103.10 & UN \\ 
  Kenya & ALL & 1986 & 94.43 & 91.20 & 97.82 & IHME \\ 
  Kenya & ALL & 1986 & 96.19 & 78.30 & 117.43 & RW2 \\ 
  Kenya & ALL & 1986 & 96.30 & 90.30 & 102.60 & UN \\ 
  Kenya & ALL & 1987 & 93.32 & 90.06 & 96.61 & IHME \\ 
  Kenya & ALL & 1987 & 96.59 & 79.54 & 117.09 & RW2 \\ 
  Kenya & ALL & 1987 & 96.70 & 90.70 & 103.00 & UN \\ 
  Kenya & ALL & 1988 & 92.58 & 89.36 & 96.00 & IHME \\ 
  Kenya & ALL & 1988 & 97.82 & 79.66 & 119.18 & RW2 \\ 
  Kenya & ALL & 1988 & 97.90 & 91.90 & 104.20 & UN \\ 
  Kenya & ALL & 1989 & 92.32 & 89.04 & 95.64 & IHME \\ 
  Kenya & ALL & 1989 & 99.85 & 80.30 & 122.83 & RW2 \\ 
  Kenya & ALL & 1989 & 99.80 & 93.70 & 106.20 & UN \\ 
  Kenya & ALL & 1990 & 92.29 & 88.88 & 95.40 & IHME \\ 
  Kenya & ALL & 1990 & 102.96 & 83.55 & 127.41 & RW2 \\ 
  Kenya & ALL & 1990 & 102.30 & 96.10 & 108.90 & UN \\ 
  Kenya & ALL & 1991 & 92.75 & 89.38 & 95.99 & IHME \\ 
  Kenya & ALL & 1991 & 105.85 & 86.86 & 128.62 & RW2 \\ 
  Kenya & ALL & 1991 & 105.20 & 98.80 & 112.20 & UN \\ 
  Kenya & ALL & 1992 & 93.50 & 90.16 & 96.86 & IHME \\ 
  Kenya & ALL & 1992 & 108.67 & 89.53 & 131.08 & RW2 \\ 
  Kenya & ALL & 1992 & 108.20 & 101.50 & 115.40 & UN \\ 
  Kenya & ALL & 1993 & 94.48 & 91.03 & 97.84 & IHME \\ 
  Kenya & ALL & 1993 & 111.17 & 91.09 & 134.76 & RW2 \\ 
  Kenya & ALL & 1993 & 111.10 & 104.10 & 118.60 & UN \\ 
  Kenya & ALL & 1994 & 95.22 & 91.66 & 98.53 & IHME \\ 
  Kenya & ALL & 1994 & 113.06 & 91.49 & 139.08 & RW2 \\ 
  Kenya & ALL & 1994 & 113.40 & 106.20 & 121.10 & UN \\ 
  Kenya & ALL & 1995 & 95.04 & 91.50 & 98.54 & IHME \\ 
  Kenya & ALL & 1995 & 114.26 & 92.63 & 140.06 & RW2 \\ 
  Kenya & ALL & 1995 & 114.80 & 107.40 & 122.70 & UN \\ 
  Kenya & ALL & 1996 & 94.94 & 91.33 & 98.74 & IHME \\ 
  Kenya & ALL & 1996 & 114.72 & 94.07 & 139.55 & RW2 \\ 
  Kenya & ALL & 1996 & 115.40 & 107.80 & 123.40 & UN \\ 
  Kenya & ALL & 1997 & 94.40 & 90.79 & 98.26 & IHME \\ 
  Kenya & ALL & 1997 & 114.40 & 94.43 & 138.00 & RW2 \\ 
  Kenya & ALL & 1997 & 114.90 & 107.30 & 123.10 & UN \\ 
  Kenya & ALL & 1998 & 93.15 & 89.55 & 96.99 & IHME \\ 
  Kenya & ALL & 1998 & 113.32 & 92.91 & 138.06 & RW2 \\ 
  Kenya & ALL & 1998 & 113.40 & 105.70 & 121.90 & UN \\ 
  Kenya & ALL & 1999 & 91.22 & 87.39 & 95.18 & IHME \\ 
  Kenya & ALL & 1999 & 111.35 & 90.08 & 136.52 & RW2 \\ 
  Kenya & ALL & 1999 & 111.00 & 103.30 & 119.60 & UN \\ 
  Kenya & ALL & 2000 & 88.56 & 84.78 & 92.67 & IHME \\ 
  Kenya & ALL & 2000 & 108.61 & 87.90 & 133.32 & RW2 \\ 
  Kenya & ALL & 2000 & 107.90 & 100.20 & 116.70 & UN \\ 
  Kenya & ALL & 2001 & 85.47 & 81.53 & 89.39 & IHME \\ 
  Kenya & ALL & 2001 & 105.01 & 85.69 & 127.92 & RW2 \\ 
  Kenya & ALL & 2001 & 104.10 & 96.40 & 113.00 & UN \\ 
  Kenya & ALL & 2002 & 82.07 & 78.30 & 86.03 & IHME \\ 
  Kenya & ALL & 2002 & 100.66 & 82.74 & 122.08 & RW2 \\ 
  Kenya & ALL & 2002 & 99.90 & 92.10 & 108.90 & UN \\ 
  Kenya & ALL & 2003 & 78.56 & 74.89 & 82.35 & IHME \\ 
  Kenya & ALL & 2003 & 95.83 & 78.46 & 116.90 & RW2 \\ 
  Kenya & ALL & 2003 & 95.40 & 87.40 & 104.50 & UN \\ 
  Kenya & ALL & 2004 & 75.10 & 71.46 & 79.13 & IHME \\ 
  Kenya & ALL & 2004 & 90.51 & 72.88 & 112.19 & RW2 \\ 
  Kenya & ALL & 2004 & 90.60 & 82.50 & 99.80 & UN \\ 
  Kenya & ALL & 2005 & 71.15 & 67.40 & 75.04 & IHME \\ 
  Kenya & ALL & 2005 & 85.02 & 67.75 & 105.66 & RW2 \\ 
  Kenya & ALL & 2005 & 85.50 & 77.00 & 94.60 & UN \\ 
  Kenya & ALL & 2006 & 67.26 & 63.61 & 70.95 & IHME \\ 
  Kenya & ALL & 2006 & 79.84 & 64.24 & 98.40 & RW2 \\ 
  Kenya & ALL & 2006 & 80.70 & 72.10 & 89.80 & UN \\ 
  Kenya & ALL & 2007 & 63.82 & 60.27 & 67.45 & IHME \\ 
  Kenya & ALL & 2007 & 75.02 & 60.78 & 92.08 & RW2 \\ 
  Kenya & ALL & 2007 & 76.10 & 67.40 & 85.50 & UN \\ 
  Kenya & ALL & 2008 & 60.16 & 56.60 & 63.80 & IHME \\ 
  Kenya & ALL & 2008 & 70.67 & 56.75 & 87.61 & RW2 \\ 
  Kenya & ALL & 2008 & 70.10 & 61.40 & 79.50 & UN \\ 
  Kenya & ALL & 2009 & 57.60 & 53.95 & 61.28 & IHME \\ 
  Kenya & ALL & 2009 & 66.56 & 52.53 & 84.07 & RW2 \\ 
  Kenya & ALL & 2009 & 65.80 & 56.80 & 75.50 & UN \\ 
  Kenya & ALL & 2010 & 55.40 & 51.50 & 59.31 & IHME \\ 
  Kenya & ALL & 2010 & 62.98 & 49.22 & 80.91 & RW2 \\ 
  Kenya & ALL & 2010 & 62.10 & 52.90 & 72.30 & UN \\ 
  Kenya & ALL & 2011 & 53.35 & 49.33 & 57.62 & IHME \\ 
  Kenya & ALL & 2011 & 59.59 & 47.16 & 75.21 & RW2 \\ 
  Kenya & ALL & 2011 & 58.50 & 49.20 & 69.40 & UN \\ 
  Kenya & ALL & 2012 & 52.21 & 48.02 & 56.73 & IHME \\ 
  Kenya & ALL & 2012 & 56.38 & 45.32 & 69.94 & RW2 \\ 
  Kenya & ALL & 2012 & 55.60 & 45.80 & 67.20 & UN \\ 
  Kenya & ALL & 2013 & 51.75 & 47.24 & 57.06 & IHME \\ 
  Kenya & ALL & 2013 & 53.41 & 41.91 & 67.66 & RW2 \\ 
  Kenya & ALL & 2013 & 53.40 & 43.10 & 65.90 & UN \\ 
  Kenya & ALL & 2014 & 50.85 & 45.87 & 56.87 & IHME \\ 
  Kenya & ALL & 2014 & 50.51 & 35.86 & 70.71 & RW2 \\ 
  Kenya & ALL & 2014 & 51.30 & 40.30 & 64.60 & UN \\ 
  Kenya & ALL & 2015 & 50.83 & 45.37 & 57.64 & IHME \\ 
  Kenya & ALL & 2015 & 47.70 & 28.23 & 79.77 & RW2 \\ 
  Kenya & ALL & 2015 & 49.40 & 38.00 & 64.00 & UN \\ 
  Kenya & ALL & 2016 & 45.20 & 22.22 & 90.89 & RW2 \\ 
  Kenya & ALL & 2017 & 42.63 & 16.85 & 105.59 & RW2 \\ 
  Kenya & ALL & 2018 & 40.28 & 12.64 & 125.63 & RW2 \\ 
  Kenya & ALL & 2019 & 38.00 & 8.93 & 148.25 & RW2 \\ 
  Kenya & ALL & 80-84 & 100.25 & 107.65 & 93.30 & HT-Direct \\ 
  Kenya & ALL & 85-89 & 91.17 & 96.80 & 85.84 & HT-Direct \\ 
  Kenya & ALL & 90-94 & 102.12 & 107.95 & 96.58 & HT-Direct \\ 
  Kenya & ALL & 95-99 & 100.52 & 106.10 & 95.20 & HT-Direct \\ 
  Kenya & ALL & 00-04 & 88.05 & 93.37 & 83.01 & HT-Direct \\ 
  Kenya & ALL & 05-09 & 62.02 & 66.78 & 57.58 & HT-Direct \\ 
  Kenya & ALL & 10-14 & 52.20 & 57.09 & 47.71 & HT-Direct \\ 
  Kenya & ALL & 15-19 & 42.62 & 17.13 & 103.78 & RW2 \\ 
  Kenya & CENTRAL & 1980 & 53.30 & 36.05 & 78.60 & RW2 \\ 
  Kenya & CENTRAL & 1981 & 51.07 & 37.70 & 69.10 & RW2 \\ 
  Kenya & CENTRAL & 1982 & 48.98 & 37.36 & 64.10 & RW2 \\ 
  Kenya & CENTRAL & 1983 & 47.00 & 35.82 & 61.73 & RW2 \\ 
  Kenya & CENTRAL & 1984 & 45.36 & 34.46 & 60.13 & RW2 \\ 
  Kenya & CENTRAL & 1985 & 43.90 & 33.81 & 56.58 & RW2 \\ 
  Kenya & CENTRAL & 1986 & 43.27 & 34.16 & 54.64 & RW2 \\ 
  Kenya & CENTRAL & 1987 & 43.31 & 34.55 & 54.13 & RW2 \\ 
  Kenya & CENTRAL & 1988 & 44.09 & 35.05 & 55.22 & RW2 \\ 
  Kenya & CENTRAL & 1989 & 45.58 & 35.82 & 57.53 & RW2 \\ 
  Kenya & CENTRAL & 1990 & 47.93 & 37.89 & 60.46 & RW2 \\ 
  Kenya & CENTRAL & 1991 & 50.64 & 40.50 & 63.14 & RW2 \\ 
  Kenya & CENTRAL & 1992 & 53.71 & 43.33 & 66.55 & RW2 \\ 
  Kenya & CENTRAL & 1993 & 57.10 & 45.76 & 71.04 & RW2 \\ 
  Kenya & CENTRAL & 1994 & 60.43 & 47.52 & 75.88 & RW2 \\ 
  Kenya & CENTRAL & 1995 & 63.84 & 50.34 & 80.60 & RW2 \\ 
  Kenya & CENTRAL & 1996 & 66.67 & 53.03 & 83.61 & RW2 \\ 
  Kenya & CENTRAL & 1997 & 68.76 & 55.00 & 85.69 & RW2 \\ 
  Kenya & CENTRAL & 1998 & 70.19 & 55.67 & 88.01 & RW2 \\ 
  Kenya & CENTRAL & 1999 & 70.60 & 55.34 & 89.42 & RW2 \\ 
  Kenya & CENTRAL & 2000 & 70.07 & 54.83 & 88.76 & RW2 \\ 
  Kenya & CENTRAL & 2001 & 68.98 & 54.41 & 87.09 & RW2 \\ 
  Kenya & CENTRAL & 2002 & 67.48 & 53.39 & 84.92 & RW2 \\ 
  Kenya & CENTRAL & 2003 & 65.42 & 51.26 & 83.03 & RW2 \\ 
  Kenya & CENTRAL & 2004 & 63.27 & 48.89 & 81.57 & RW2 \\ 
  Kenya & CENTRAL & 2005 & 60.89 & 46.71 & 78.98 & RW2 \\ 
  Kenya & CENTRAL & 2006 & 58.55 & 45.02 & 75.81 & RW2 \\ 
  Kenya & CENTRAL & 2007 & 56.33 & 43.18 & 73.02 & RW2 \\ 
  Kenya & CENTRAL & 2008 & 54.39 & 41.03 & 71.52 & RW2 \\ 
  Kenya & CENTRAL & 2009 & 52.50 & 38.66 & 70.82 & RW2 \\ 
  Kenya & CENTRAL & 2010 & 50.87 & 36.91 & 70.09 & RW2 \\ 
  Kenya & CENTRAL & 2011 & 49.15 & 35.38 & 68.16 & RW2 \\ 
  Kenya & CENTRAL & 2012 & 47.62 & 33.97 & 66.58 & RW2 \\ 
  Kenya & CENTRAL & 2013 & 46.16 & 31.64 & 67.17 & RW2 \\ 
  Kenya & CENTRAL & 2014 & 44.77 & 28.02 & 70.95 & RW2 \\ 
  Kenya & CENTRAL & 2015 & 43.34 & 23.30 & 79.98 & RW2 \\ 
  Kenya & CENTRAL & 2016 & 42.01 & 18.78 & 91.04 & RW2 \\ 
  Kenya & CENTRAL & 2017 & 40.66 & 14.81 & 106.99 & RW2 \\ 
  Kenya & CENTRAL & 2018 & 39.39 & 11.47 & 128.77 & RW2 \\ 
  Kenya & CENTRAL & 2019 & 38.20 & 8.45 & 155.14 & RW2 \\ 
  Kenya & COAST & 1980 & 179.02 & 131.83 & 239.05 & RW2 \\ 
  Kenya & COAST & 1981 & 166.07 & 133.06 & 205.73 & RW2 \\ 
  Kenya & COAST & 1982 & 153.78 & 126.47 & 186.54 & RW2 \\ 
  Kenya & COAST & 1983 & 142.94 & 115.93 & 175.57 & RW2 \\ 
  Kenya & COAST & 1984 & 133.12 & 106.55 & 165.82 & RW2 \\ 
  Kenya & COAST & 1985 & 124.53 & 100.63 & 152.00 & RW2 \\ 
  Kenya & COAST & 1986 & 118.45 & 97.63 & 142.25 & RW2 \\ 
  Kenya & COAST & 1987 & 114.51 & 95.25 & 136.82 & RW2 \\ 
  Kenya & COAST & 1988 & 112.59 & 93.09 & 134.92 & RW2 \\ 
  Kenya & COAST & 1989 & 112.56 & 91.92 & 136.84 & RW2 \\ 
  Kenya & COAST & 1990 & 114.49 & 94.01 & 139.58 & RW2 \\ 
  Kenya & COAST & 1991 & 117.02 & 97.29 & 140.66 & RW2 \\ 
  Kenya & COAST & 1992 & 120.21 & 100.42 & 143.12 & RW2 \\ 
  Kenya & COAST & 1993 & 123.67 & 102.47 & 148.11 & RW2 \\ 
  Kenya & COAST & 1994 & 126.75 & 103.37 & 153.96 & RW2 \\ 
  Kenya & COAST & 1995 & 129.48 & 105.59 & 158.03 & RW2 \\ 
  Kenya & COAST & 1996 & 130.58 & 107.34 & 158.30 & RW2 \\ 
  Kenya & COAST & 1997 & 130.11 & 107.25 & 157.24 & RW2 \\ 
  Kenya & COAST & 1998 & 127.92 & 104.94 & 155.55 & RW2 \\ 
  Kenya & COAST & 1999 & 124.07 & 100.22 & 152.86 & RW2 \\ 
  Kenya & COAST & 2000 & 118.89 & 95.70 & 146.42 & RW2 \\ 
  Kenya & COAST & 2001 & 113.00 & 91.63 & 138.46 & RW2 \\ 
  Kenya & COAST & 2002 & 106.65 & 86.82 & 130.48 & RW2 \\ 
  Kenya & COAST & 2003 & 100.10 & 80.72 & 123.44 & RW2 \\ 
  Kenya & COAST & 2004 & 93.66 & 74.70 & 117.14 & RW2 \\ 
  Kenya & COAST & 2005 & 87.49 & 69.21 & 109.55 & RW2 \\ 
  Kenya & COAST & 2006 & 81.72 & 65.08 & 102.13 & RW2 \\ 
  Kenya & COAST & 2007 & 76.64 & 60.98 & 95.62 & RW2 \\ 
  Kenya & COAST & 2008 & 71.93 & 56.41 & 91.56 & RW2 \\ 
  Kenya & COAST & 2009 & 67.61 & 51.68 & 88.12 & RW2 \\ 
  Kenya & COAST & 2010 & 63.80 & 47.70 & 85.20 & RW2 \\ 
  Kenya & COAST & 2011 & 60.32 & 44.87 & 80.79 & RW2 \\ 
  Kenya & COAST & 2012 & 56.98 & 41.92 & 77.31 & RW2 \\ 
  Kenya & COAST & 2013 & 53.71 & 37.93 & 75.96 & RW2 \\ 
  Kenya & COAST & 2014 & 50.82 & 32.70 & 79.08 & RW2 \\ 
  Kenya & COAST & 2015 & 47.96 & 26.32 & 86.46 & RW2 \\ 
  Kenya & COAST & 2016 & 45.31 & 20.68 & 97.88 & RW2 \\ 
  Kenya & COAST & 2017 & 42.90 & 15.87 & 112.05 & RW2 \\ 
  Kenya & COAST & 2018 & 40.51 & 11.86 & 130.64 & RW2 \\ 
  Kenya & COAST & 2019 & 38.04 & 8.57 & 157.00 & RW2 \\ 
  Kenya & EASTERN & 1980 & 84.68 & 60.32 & 117.44 & RW2 \\ 
  Kenya & EASTERN & 1981 & 80.33 & 62.88 & 101.78 & RW2 \\ 
  Kenya & EASTERN & 1982 & 76.11 & 61.24 & 94.25 & RW2 \\ 
  Kenya & EASTERN & 1983 & 72.24 & 57.46 & 90.91 & RW2 \\ 
  Kenya & EASTERN & 1984 & 68.93 & 54.37 & 87.63 & RW2 \\ 
  Kenya & EASTERN & 1985 & 65.97 & 52.43 & 82.08 & RW2 \\ 
  Kenya & EASTERN & 1986 & 64.34 & 52.24 & 78.60 & RW2 \\ 
  Kenya & EASTERN & 1987 & 63.74 & 52.45 & 77.08 & RW2 \\ 
  Kenya & EASTERN & 1988 & 64.22 & 52.36 & 78.21 & RW2 \\ 
  Kenya & EASTERN & 1989 & 65.72 & 52.92 & 80.92 & RW2 \\ 
  Kenya & EASTERN & 1990 & 68.40 & 55.51 & 84.53 & RW2 \\ 
  Kenya & EASTERN & 1991 & 71.61 & 58.88 & 87.23 & RW2 \\ 
  Kenya & EASTERN & 1992 & 75.31 & 62.19 & 90.85 & RW2 \\ 
  Kenya & EASTERN & 1993 & 79.21 & 65.09 & 95.78 & RW2 \\ 
  Kenya & EASTERN & 1994 & 83.03 & 67.22 & 101.76 & RW2 \\ 
  Kenya & EASTERN & 1995 & 86.65 & 70.31 & 106.50 & RW2 \\ 
  Kenya & EASTERN & 1996 & 89.38 & 73.18 & 108.73 & RW2 \\ 
  Kenya & EASTERN & 1997 & 91.03 & 75.06 & 109.97 & RW2 \\ 
  Kenya & EASTERN & 1998 & 91.45 & 75.11 & 111.48 & RW2 \\ 
  Kenya & EASTERN & 1999 & 90.64 & 73.44 & 111.93 & RW2 \\ 
  Kenya & EASTERN & 2000 & 88.51 & 71.15 & 108.90 & RW2 \\ 
  Kenya & EASTERN & 2001 & 85.75 & 69.88 & 104.66 & RW2 \\ 
  Kenya & EASTERN & 2002 & 82.41 & 67.39 & 100.42 & RW2 \\ 
  Kenya & EASTERN & 2003 & 78.60 & 63.76 & 96.77 & RW2 \\ 
  Kenya & EASTERN & 2004 & 74.72 & 59.65 & 93.30 & RW2 \\ 
  Kenya & EASTERN & 2005 & 70.79 & 55.78 & 89.25 & RW2 \\ 
  Kenya & EASTERN & 2006 & 67.07 & 53.07 & 84.33 & RW2 \\ 
  Kenya & EASTERN & 2007 & 63.59 & 50.28 & 79.98 & RW2 \\ 
  Kenya & EASTERN & 2008 & 60.46 & 47.19 & 77.03 & RW2 \\ 
  Kenya & EASTERN & 2009 & 57.53 & 44.09 & 74.72 & RW2 \\ 
  Kenya & EASTERN & 2010 & 55.05 & 41.32 & 72.89 & RW2 \\ 
  Kenya & EASTERN & 2011 & 52.54 & 39.67 & 69.44 & RW2 \\ 
  Kenya & EASTERN & 2012 & 50.30 & 37.90 & 66.45 & RW2 \\ 
  Kenya & EASTERN & 2013 & 48.09 & 34.96 & 65.84 & RW2 \\ 
  Kenya & EASTERN & 2014 & 46.02 & 30.51 & 68.54 & RW2 \\ 
  Kenya & EASTERN & 2015 & 44.03 & 24.90 & 77.15 & RW2 \\ 
  Kenya & EASTERN & 2016 & 42.09 & 19.96 & 87.58 & RW2 \\ 
  Kenya & EASTERN & 2017 & 40.29 & 15.44 & 102.64 & RW2 \\ 
  Kenya & EASTERN & 2018 & 38.40 & 11.50 & 121.42 & RW2 \\ 
  Kenya & EASTERN & 2019 & 36.66 & 8.62 & 145.29 & RW2 \\ 
  Kenya & NAIROBI & 1980 & 73.47 & 46.09 & 115.15 & RW2 \\ 
  Kenya & NAIROBI & 1981 & 70.81 & 48.50 & 103.12 & RW2 \\ 
  Kenya & NAIROBI & 1982 & 68.29 & 48.39 & 95.85 & RW2 \\ 
  Kenya & NAIROBI & 1983 & 65.96 & 47.17 & 91.95 & RW2 \\ 
  Kenya & NAIROBI & 1984 & 64.04 & 46.26 & 88.61 & RW2 \\ 
  Kenya & NAIROBI & 1985 & 62.32 & 45.97 & 83.98 & RW2 \\ 
  Kenya & NAIROBI & 1986 & 61.80 & 46.64 & 81.36 & RW2 \\ 
  Kenya & NAIROBI & 1987 & 62.17 & 47.73 & 80.61 & RW2 \\ 
  Kenya & NAIROBI & 1988 & 63.53 & 48.83 & 82.32 & RW2 \\ 
  Kenya & NAIROBI & 1989 & 65.98 & 50.53 & 85.71 & RW2 \\ 
  Kenya & NAIROBI & 1990 & 69.50 & 53.68 & 90.23 & RW2 \\ 
  Kenya & NAIROBI & 1991 & 73.69 & 57.36 & 94.05 & RW2 \\ 
  Kenya & NAIROBI & 1992 & 78.37 & 61.42 & 99.30 & RW2 \\ 
  Kenya & NAIROBI & 1993 & 83.29 & 65.00 & 106.01 & RW2 \\ 
  Kenya & NAIROBI & 1994 & 88.33 & 68.18 & 113.48 & RW2 \\ 
  Kenya & NAIROBI & 1995 & 93.28 & 72.18 & 119.91 & RW2 \\ 
  Kenya & NAIROBI & 1996 & 97.19 & 75.58 & 124.00 & RW2 \\ 
  Kenya & NAIROBI & 1997 & 100.31 & 78.54 & 127.42 & RW2 \\ 
  Kenya & NAIROBI & 1998 & 102.15 & 79.60 & 130.49 & RW2 \\ 
  Kenya & NAIROBI & 1999 & 102.78 & 79.11 & 132.22 & RW2 \\ 
  Kenya & NAIROBI & 2000 & 101.92 & 78.12 & 131.34 & RW2 \\ 
  Kenya & NAIROBI & 2001 & 100.37 & 77.64 & 128.60 & RW2 \\ 
  Kenya & NAIROBI & 2002 & 98.13 & 76.06 & 125.24 & RW2 \\ 
  Kenya & NAIROBI & 2003 & 95.28 & 73.39 & 122.91 & RW2 \\ 
  Kenya & NAIROBI & 2004 & 92.29 & 70.25 & 120.74 & RW2 \\ 
  Kenya & NAIROBI & 2005 & 89.14 & 67.20 & 117.10 & RW2 \\ 
  Kenya & NAIROBI & 2006 & 86.04 & 65.03 & 112.84 & RW2 \\ 
  Kenya & NAIROBI & 2007 & 83.11 & 62.66 & 109.42 & RW2 \\ 
  Kenya & NAIROBI & 2008 & 80.42 & 59.67 & 107.37 & RW2 \\ 
  Kenya & NAIROBI & 2009 & 78.09 & 56.82 & 106.23 & RW2 \\ 
  Kenya & NAIROBI & 2010 & 76.01 & 54.37 & 105.40 & RW2 \\ 
  Kenya & NAIROBI & 2011 & 73.96 & 52.59 & 103.23 & RW2 \\ 
  Kenya & NAIROBI & 2012 & 72.12 & 50.71 & 101.61 & RW2 \\ 
  Kenya & NAIROBI & 2013 & 70.27 & 47.93 & 102.67 & RW2 \\ 
  Kenya & NAIROBI & 2014 & 68.41 & 42.81 & 108.28 & RW2 \\ 
  Kenya & NAIROBI & 2015 & 66.64 & 35.82 & 121.85 & RW2 \\ 
  Kenya & NAIROBI & 2016 & 64.85 & 29.14 & 138.56 & RW2 \\ 
  Kenya & NAIROBI & 2017 & 63.26 & 23.28 & 161.82 & RW2 \\ 
  Kenya & NAIROBI & 2018 & 61.39 & 17.97 & 191.07 & RW2 \\ 
  Kenya & NAIROBI & 2019 & 60.00 & 13.55 & 231.90 & RW2 \\ 
  Kenya & NORTHEASTERN & 1980 & 242.68 & 158.21 & 351.45 & RW2 \\ 
  Kenya & NORTHEASTERN & 1981 & 225.37 & 158.07 & 309.00 & RW2 \\ 
  Kenya & NORTHEASTERN & 1982 & 208.99 & 151.63 & 279.77 & RW2 \\ 
  Kenya & NORTHEASTERN & 1983 & 193.69 & 142.96 & 257.26 & RW2 \\ 
  Kenya & NORTHEASTERN & 1984 & 180.32 & 134.43 & 238.72 & RW2 \\ 
  Kenya & NORTHEASTERN & 1985 & 168.36 & 127.76 & 216.71 & RW2 \\ 
  Kenya & NORTHEASTERN & 1986 & 159.54 & 124.71 & 201.06 & RW2 \\ 
  Kenya & NORTHEASTERN & 1987 & 153.49 & 122.12 & 190.76 & RW2 \\ 
  Kenya & NORTHEASTERN & 1988 & 150.10 & 119.81 & 186.03 & RW2 \\ 
  Kenya & NORTHEASTERN & 1989 & 148.87 & 118.27 & 184.85 & RW2 \\ 
  Kenya & NORTHEASTERN & 1990 & 149.82 & 120.74 & 186.10 & RW2 \\ 
  Kenya & NORTHEASTERN & 1991 & 151.39 & 123.14 & 185.54 & RW2 \\ 
  Kenya & NORTHEASTERN & 1992 & 153.14 & 125.56 & 185.97 & RW2 \\ 
  Kenya & NORTHEASTERN & 1993 & 154.74 & 126.28 & 188.60 & RW2 \\ 
  Kenya & NORTHEASTERN & 1994 & 155.39 & 125.13 & 191.24 & RW2 \\ 
  Kenya & NORTHEASTERN & 1995 & 155.05 & 125.25 & 191.52 & RW2 \\ 
  Kenya & NORTHEASTERN & 1996 & 152.69 & 124.08 & 187.26 & RW2 \\ 
  Kenya & NORTHEASTERN & 1997 & 148.52 & 121.06 & 181.42 & RW2 \\ 
  Kenya & NORTHEASTERN & 1998 & 142.46 & 114.89 & 175.19 & RW2 \\ 
  Kenya & NORTHEASTERN & 1999 & 134.71 & 107.31 & 167.54 & RW2 \\ 
  Kenya & NORTHEASTERN & 2000 & 125.76 & 99.64 & 156.70 & RW2 \\ 
  Kenya & NORTHEASTERN & 2001 & 116.50 & 92.42 & 144.98 & RW2 \\ 
  Kenya & NORTHEASTERN & 2002 & 107.36 & 85.25 & 133.72 & RW2 \\ 
  Kenya & NORTHEASTERN & 2003 & 98.32 & 77.05 & 123.81 & RW2 \\ 
  Kenya & NORTHEASTERN & 2004 & 89.93 & 69.56 & 115.07 & RW2 \\ 
  Kenya & NORTHEASTERN & 2005 & 82.05 & 62.82 & 105.76 & RW2 \\ 
  Kenya & NORTHEASTERN & 2006 & 75.13 & 57.79 & 96.68 & RW2 \\ 
  Kenya & NORTHEASTERN & 2007 & 68.93 & 52.89 & 89.03 & RW2 \\ 
  Kenya & NORTHEASTERN & 2008 & 63.45 & 48.07 & 83.02 & RW2 \\ 
  Kenya & NORTHEASTERN & 2009 & 58.60 & 43.49 & 78.29 & RW2 \\ 
  Kenya & NORTHEASTERN & 2010 & 54.28 & 39.62 & 74.10 & RW2 \\ 
  Kenya & NORTHEASTERN & 2011 & 50.41 & 36.57 & 69.17 & RW2 \\ 
  Kenya & NORTHEASTERN & 2012 & 46.76 & 33.68 & 64.94 & RW2 \\ 
  Kenya & NORTHEASTERN & 2013 & 43.42 & 29.99 & 63.17 & RW2 \\ 
  Kenya & NORTHEASTERN & 2014 & 40.29 & 25.38 & 64.04 & RW2 \\ 
  Kenya & NORTHEASTERN & 2015 & 37.44 & 20.03 & 69.67 & RW2 \\ 
  Kenya & NORTHEASTERN & 2016 & 34.68 & 15.41 & 77.05 & RW2 \\ 
  Kenya & NORTHEASTERN & 2017 & 32.21 & 11.55 & 89.08 & RW2 \\ 
  Kenya & NORTHEASTERN & 2018 & 29.89 & 8.45 & 102.22 & RW2 \\ 
  Kenya & NORTHEASTERN & 2019 & 27.78 & 6.14 & 118.93 & RW2 \\ 
  Kenya & NYANZA & 1980 & 162.97 & 118.03 & 218.45 & RW2 \\ 
  Kenya & NYANZA & 1981 & 162.92 & 129.31 & 201.46 & RW2 \\ 
  Kenya & NYANZA & 1982 & 162.87 & 133.99 & 196.82 & RW2 \\ 
  Kenya & NYANZA & 1983 & 163.18 & 133.51 & 199.20 & RW2 \\ 
  Kenya & NYANZA & 1984 & 164.04 & 132.74 & 201.89 & RW2 \\ 
  Kenya & NYANZA & 1985 & 164.87 & 135.48 & 198.81 & RW2 \\ 
  Kenya & NYANZA & 1986 & 167.84 & 140.83 & 198.70 & RW2 \\ 
  Kenya & NYANZA & 1987 & 172.45 & 146.48 & 202.34 & RW2 \\ 
  Kenya & NYANZA & 1988 & 178.82 & 150.54 & 210.86 & RW2 \\ 
  Kenya & NYANZA & 1989 & 186.71 & 155.76 & 222.60 & RW2 \\ 
  Kenya & NYANZA & 1990 & 196.95 & 165.28 & 234.15 & RW2 \\ 
  Kenya & NYANZA & 1991 & 207.34 & 175.94 & 243.62 & RW2 \\ 
  Kenya & NYANZA & 1992 & 218.13 & 186.53 & 253.35 & RW2 \\ 
  Kenya & NYANZA & 1993 & 228.40 & 193.69 & 266.79 & RW2 \\ 
  Kenya & NYANZA & 1994 & 237.65 & 199.22 & 279.86 & RW2 \\ 
  Kenya & NYANZA & 1995 & 245.58 & 206.02 & 290.13 & RW2 \\ 
  Kenya & NYANZA & 1996 & 250.93 & 213.02 & 294.22 & RW2 \\ 
  Kenya & NYANZA & 1997 & 253.44 & 215.29 & 295.39 & RW2 \\ 
  Kenya & NYANZA & 1998 & 252.40 & 213.41 & 296.47 & RW2 \\ 
  Kenya & NYANZA & 1999 & 248.30 & 207.10 & 294.47 & RW2 \\ 
  Kenya & NYANZA & 2000 & 240.53 & 199.45 & 286.71 & RW2 \\ 
  Kenya & NYANZA & 2001 & 230.20 & 192.35 & 273.51 & RW2 \\ 
  Kenya & NYANZA & 2002 & 217.62 & 182.34 & 258.83 & RW2 \\ 
  Kenya & NYANZA & 2003 & 203.18 & 168.67 & 243.14 & RW2 \\ 
  Kenya & NYANZA & 2004 & 187.41 & 153.23 & 227.86 & RW2 \\ 
  Kenya & NYANZA & 2005 & 171.17 & 139.14 & 209.23 & RW2 \\ 
  Kenya & NYANZA & 2006 & 155.08 & 126.63 & 188.56 & RW2 \\ 
  Kenya & NYANZA & 2007 & 139.75 & 114.40 & 170.12 & RW2 \\ 
  Kenya & NYANZA & 2008 & 125.25 & 101.30 & 154.32 & RW2 \\ 
  Kenya & NYANZA & 2009 & 112.03 & 88.33 & 140.72 & RW2 \\ 
  Kenya & NYANZA & 2010 & 99.97 & 77.78 & 127.90 & RW2 \\ 
  Kenya & NYANZA & 2011 & 88.99 & 69.44 & 113.11 & RW2 \\ 
  Kenya & NYANZA & 2012 & 79.05 & 61.83 & 100.71 & RW2 \\ 
  Kenya & NYANZA & 2013 & 70.09 & 52.47 & 92.30 & RW2 \\ 
  Kenya & NYANZA & 2014 & 62.05 & 42.12 & 90.19 & RW2 \\ 
  Kenya & NYANZA & 2015 & 54.95 & 31.14 & 94.45 & RW2 \\ 
  Kenya & NYANZA & 2016 & 48.44 & 22.98 & 99.27 & RW2 \\ 
  Kenya & NYANZA & 2017 & 42.72 & 16.08 & 107.70 & RW2 \\ 
  Kenya & NYANZA & 2018 & 37.74 & 11.18 & 118.06 & RW2 \\ 
  Kenya & NYANZA & 2019 & 33.29 & 7.52 & 131.16 & RW2 \\ 
  Kenya & RIFT VALLEY & 1980 & 86.93 & 62.61 & 119.53 & RW2 \\ 
  Kenya & RIFT VALLEY & 1981 & 82.74 & 65.39 & 104.20 & RW2 \\ 
  Kenya & RIFT VALLEY & 1982 & 78.70 & 63.61 & 96.87 & RW2 \\ 
  Kenya & RIFT VALLEY & 1983 & 75.08 & 60.12 & 93.88 & RW2 \\ 
  Kenya & RIFT VALLEY & 1984 & 71.93 & 56.92 & 91.30 & RW2 \\ 
  Kenya & RIFT VALLEY & 1985 & 69.12 & 55.32 & 85.67 & RW2 \\ 
  Kenya & RIFT VALLEY & 1986 & 67.54 & 55.23 & 81.95 & RW2 \\ 
  Kenya & RIFT VALLEY & 1987 & 66.95 & 55.47 & 80.64 & RW2 \\ 
  Kenya & RIFT VALLEY & 1988 & 67.40 & 55.50 & 81.50 & RW2 \\ 
  Kenya & RIFT VALLEY & 1989 & 68.85 & 55.88 & 84.12 & RW2 \\ 
  Kenya & RIFT VALLEY & 1990 & 71.44 & 58.42 & 87.62 & RW2 \\ 
  Kenya & RIFT VALLEY & 1991 & 74.43 & 61.72 & 90.15 & RW2 \\ 
  Kenya & RIFT VALLEY & 1992 & 77.78 & 64.84 & 92.82 & RW2 \\ 
  Kenya & RIFT VALLEY & 1993 & 81.26 & 67.19 & 97.49 & RW2 \\ 
  Kenya & RIFT VALLEY & 1994 & 84.61 & 68.82 & 102.99 & RW2 \\ 
  Kenya & RIFT VALLEY & 1995 & 87.77 & 71.58 & 107.26 & RW2 \\ 
  Kenya & RIFT VALLEY & 1996 & 89.98 & 74.10 & 108.64 & RW2 \\ 
  Kenya & RIFT VALLEY & 1997 & 91.32 & 75.71 & 109.38 & RW2 \\ 
  Kenya & RIFT VALLEY & 1998 & 91.43 & 75.46 & 110.31 & RW2 \\ 
  Kenya & RIFT VALLEY & 1999 & 90.49 & 73.60 & 111.09 & RW2 \\ 
  Kenya & RIFT VALLEY & 2000 & 88.33 & 71.57 & 107.96 & RW2 \\ 
  Kenya & RIFT VALLEY & 2001 & 85.49 & 70.20 & 103.19 & RW2 \\ 
  Kenya & RIFT VALLEY & 2002 & 82.06 & 67.85 & 98.78 & RW2 \\ 
  Kenya & RIFT VALLEY & 2003 & 78.15 & 64.32 & 94.69 & RW2 \\ 
  Kenya & RIFT VALLEY & 2004 & 74.12 & 60.14 & 91.15 & RW2 \\ 
  Kenya & RIFT VALLEY & 2005 & 69.90 & 56.23 & 86.17 & RW2 \\ 
  Kenya & RIFT VALLEY & 2006 & 65.95 & 53.58 & 80.77 & RW2 \\ 
  Kenya & RIFT VALLEY & 2007 & 62.10 & 50.57 & 75.99 & RW2 \\ 
  Kenya & RIFT VALLEY & 2008 & 58.57 & 47.25 & 72.32 & RW2 \\ 
  Kenya & RIFT VALLEY & 2009 & 55.44 & 43.86 & 69.79 & RW2 \\ 
  Kenya & RIFT VALLEY & 2010 & 52.52 & 40.89 & 67.05 & RW2 \\ 
  Kenya & RIFT VALLEY & 2011 & 49.76 & 39.38 & 62.92 & RW2 \\ 
  Kenya & RIFT VALLEY & 2012 & 47.17 & 37.64 & 59.19 & RW2 \\ 
  Kenya & RIFT VALLEY & 2013 & 44.73 & 34.53 & 57.70 & RW2 \\ 
  Kenya & RIFT VALLEY & 2014 & 42.46 & 29.66 & 60.22 & RW2 \\ 
  Kenya & RIFT VALLEY & 2015 & 40.22 & 23.87 & 67.67 & RW2 \\ 
  Kenya & RIFT VALLEY & 2016 & 38.08 & 18.64 & 76.31 & RW2 \\ 
  Kenya & RIFT VALLEY & 2017 & 36.17 & 14.30 & 88.37 & RW2 \\ 
  Kenya & RIFT VALLEY & 2018 & 34.05 & 10.72 & 104.82 & RW2 \\ 
  Kenya & RIFT VALLEY & 2019 & 32.26 & 7.81 & 124.86 & RW2 \\ 
  Kenya & WESTERN & 1980 & 127.50 & 91.32 & 174.51 & RW2 \\ 
  Kenya & WESTERN & 1981 & 125.73 & 98.72 & 157.98 & RW2 \\ 
  Kenya & WESTERN & 1982 & 123.91 & 100.55 & 151.68 & RW2 \\ 
  Kenya & WESTERN & 1983 & 122.32 & 98.63 & 151.78 & RW2 \\ 
  Kenya & WESTERN & 1984 & 121.18 & 96.74 & 152.02 & RW2 \\ 
  Kenya & WESTERN & 1985 & 120.22 & 97.03 & 147.59 & RW2 \\ 
  Kenya & WESTERN & 1986 & 120.99 & 99.54 & 145.72 & RW2 \\ 
  Kenya & WESTERN & 1987 & 122.95 & 102.41 & 146.78 & RW2 \\ 
  Kenya & WESTERN & 1988 & 126.60 & 104.69 & 151.87 & RW2 \\ 
  Kenya & WESTERN & 1989 & 131.57 & 107.62 & 159.41 & RW2 \\ 
  Kenya & WESTERN & 1990 & 138.28 & 114.20 & 167.32 & RW2 \\ 
  Kenya & WESTERN & 1991 & 145.28 & 121.66 & 173.53 & RW2 \\ 
  Kenya & WESTERN & 1992 & 152.74 & 128.61 & 180.43 & RW2 \\ 
  Kenya & WESTERN & 1993 & 160.15 & 134.02 & 189.71 & RW2 \\ 
  Kenya & WESTERN & 1994 & 166.86 & 137.52 & 199.99 & RW2 \\ 
  Kenya & WESTERN & 1995 & 172.88 & 142.36 & 208.16 & RW2 \\ 
  Kenya & WESTERN & 1996 & 177.06 & 147.05 & 211.04 & RW2 \\ 
  Kenya & WESTERN & 1997 & 179.30 & 149.54 & 213.25 & RW2 \\ 
  Kenya & WESTERN & 1998 & 179.33 & 149.31 & 214.41 & RW2 \\ 
  Kenya & WESTERN & 1999 & 177.08 & 145.41 & 214.40 & RW2 \\ 
  Kenya & WESTERN & 2000 & 172.07 & 140.50 & 208.20 & RW2 \\ 
  Kenya & WESTERN & 2001 & 165.47 & 136.46 & 199.39 & RW2 \\ 
  Kenya & WESTERN & 2002 & 156.90 & 129.89 & 188.85 & RW2 \\ 
  Kenya & WESTERN & 2003 & 147.34 & 120.72 & 179.85 & RW2 \\ 
  Kenya & WESTERN & 2004 & 136.57 & 110.01 & 169.38 & RW2 \\ 
  Kenya & WESTERN & 2005 & 125.40 & 100.12 & 156.19 & RW2 \\ 
  Kenya & WESTERN & 2006 & 114.44 & 91.69 & 142.21 & RW2 \\ 
  Kenya & WESTERN & 2007 & 104.00 & 83.15 & 129.31 & RW2 \\ 
  Kenya & WESTERN & 2008 & 94.07 & 74.34 & 118.30 & RW2 \\ 
  Kenya & WESTERN & 2009 & 84.89 & 65.82 & 108.66 & RW2 \\ 
  Kenya & WESTERN & 2010 & 76.49 & 58.58 & 99.54 & RW2 \\ 
  Kenya & WESTERN & 2011 & 68.88 & 52.76 & 88.71 & RW2 \\ 
  Kenya & WESTERN & 2012 & 61.84 & 47.61 & 79.90 & RW2 \\ 
  Kenya & WESTERN & 2013 & 55.53 & 41.21 & 74.26 & RW2 \\ 
  Kenya & WESTERN & 2014 & 49.73 & 33.34 & 73.13 & RW2 \\ 
  Kenya & WESTERN & 2015 & 44.62 & 25.23 & 77.58 & RW2 \\ 
  Kenya & WESTERN & 2016 & 40.04 & 18.73 & 83.19 & RW2 \\ 
  Kenya & WESTERN & 2017 & 35.74 & 13.45 & 90.64 & RW2 \\ 
  Kenya & WESTERN & 2018 & 31.85 & 9.51 & 101.23 & RW2 \\ 
  Kenya & WESTERN & 2019 & 28.54 & 6.54 & 117.04 & RW2 \\ 
  Lesotho & ALL & 1980 & 99.98 & 93.97 & 106.25 & IHME \\ 
  Lesotho & ALL & 1980 & 119.18 & 76.50 & 182.26 & RW2 \\ 
  Lesotho & ALL & 1980 & 120.50 & 109.00 & 133.40 & UN \\ 
  Lesotho & ALL & 1981 & 99.25 & 93.58 & 105.31 & IHME \\ 
  Lesotho & ALL & 1981 & 114.11 & 81.23 & 158.29 & RW2 \\ 
  Lesotho & ALL & 1981 & 114.90 & 103.50 & 127.10 & UN \\ 
  Lesotho & ALL & 1982 & 98.60 & 92.83 & 104.73 & IHME \\ 
  Lesotho & ALL & 1982 & 109.28 & 82.17 & 144.36 & RW2 \\ 
  Lesotho & ALL & 1982 & 109.60 & 98.60 & 121.30 & UN \\ 
  Lesotho & ALL & 1983 & 98.17 & 92.47 & 103.97 & IHME \\ 
  Lesotho & ALL & 1983 & 104.48 & 79.36 & 137.15 & RW2 \\ 
  Lesotho & ALL & 1983 & 105.00 & 94.30 & 116.20 & UN \\ 
  Lesotho & ALL & 1984 & 97.41 & 91.79 & 102.98 & IHME \\ 
  Lesotho & ALL & 1984 & 100.42 & 75.53 & 132.73 & RW2 \\ 
  Lesotho & ALL & 1984 & 100.90 & 90.80 & 111.60 & UN \\ 
  Lesotho & ALL & 1985 & 96.39 & 91.11 & 101.71 & IHME \\ 
  Lesotho & ALL & 1985 & 96.28 & 73.57 & 124.61 & RW2 \\ 
  Lesotho & ALL & 1985 & 97.40 & 87.80 & 107.70 & UN \\ 
  Lesotho & ALL & 1986 & 95.57 & 90.37 & 100.64 & IHME \\ 
  Lesotho & ALL & 1986 & 93.28 & 72.33 & 119.02 & RW2 \\ 
  Lesotho & ALL & 1986 & 94.40 & 85.30 & 104.20 & UN \\ 
  Lesotho & ALL & 1987 & 94.71 & 89.55 & 99.71 & IHME \\ 
  Lesotho & ALL & 1987 & 91.10 & 71.47 & 115.37 & RW2 \\ 
  Lesotho & ALL & 1987 & 92.10 & 83.40 & 101.40 & UN \\ 
  Lesotho & ALL & 1988 & 93.63 & 88.65 & 98.39 & IHME \\ 
  Lesotho & ALL & 1988 & 89.62 & 69.89 & 113.78 & RW2 \\ 
  Lesotho & ALL & 1988 & 90.20 & 81.80 & 99.20 & UN \\ 
  Lesotho & ALL & 1989 & 92.51 & 87.54 & 97.44 & IHME \\ 
  Lesotho & ALL & 1989 & 88.95 & 69.01 & 113.54 & RW2 \\ 
  Lesotho & ALL & 1989 & 88.90 & 80.80 & 97.50 & UN \\ 
  Lesotho & ALL & 1990 & 91.58 & 86.94 & 96.68 & IHME \\ 
  Lesotho & ALL & 1990 & 89.17 & 69.83 & 113.42 & RW2 \\ 
  Lesotho & ALL & 1990 & 88.10 & 80.20 & 96.40 & UN \\ 
  Lesotho & ALL & 1991 & 90.83 & 86.17 & 95.88 & IHME \\ 
  Lesotho & ALL & 1991 & 89.89 & 71.38 & 112.41 & RW2 \\ 
  Lesotho & ALL & 1991 & 87.90 & 80.30 & 96.10 & UN \\ 
  Lesotho & ALL & 1992 & 90.23 & 85.61 & 95.11 & IHME \\ 
  Lesotho & ALL & 1992 & 91.20 & 72.84 & 113.39 & RW2 \\ 
  Lesotho & ALL & 1992 & 88.70 & 81.10 & 96.90 & UN \\ 
  Lesotho & ALL & 1993 & 90.26 & 85.71 & 95.01 & IHME \\ 
  Lesotho & ALL & 1993 & 93.06 & 73.98 & 116.42 & RW2 \\ 
  Lesotho & ALL & 1993 & 90.80 & 83.10 & 99.20 & UN \\ 
  Lesotho & ALL & 1994 & 90.92 & 86.45 & 95.56 & IHME \\ 
  Lesotho & ALL & 1994 & 95.35 & 74.81 & 121.18 & RW2 \\ 
  Lesotho & ALL & 1994 & 94.30 & 86.50 & 102.90 & UN \\ 
  Lesotho & ALL & 1995 & 92.62 & 88.15 & 97.08 & IHME \\ 
  Lesotho & ALL & 1995 & 98.14 & 76.69 & 124.66 & RW2 \\ 
  Lesotho & ALL & 1995 & 98.90 & 90.80 & 107.80 & UN \\ 
  Lesotho & ALL & 1996 & 94.44 & 89.87 & 99.15 & IHME \\ 
  Lesotho & ALL & 1996 & 101.27 & 79.68 & 128.16 & RW2 \\ 
  Lesotho & ALL & 1996 & 103.70 & 95.10 & 112.80 & UN \\ 
  Lesotho & ALL & 1997 & 96.78 & 92.23 & 101.67 & IHME \\ 
  Lesotho & ALL & 1997 & 104.78 & 82.87 & 131.52 & RW2 \\ 
  Lesotho & ALL & 1997 & 107.90 & 99.10 & 117.20 & UN \\ 
  Lesotho & ALL & 1998 & 99.45 & 94.77 & 104.68 & IHME \\ 
  Lesotho & ALL & 1998 & 108.61 & 85.55 & 137.06 & RW2 \\ 
  Lesotho & ALL & 1998 & 111.40 & 102.50 & 120.70 & UN \\ 
  Lesotho & ALL & 1999 & 102.15 & 97.35 & 107.35 & IHME \\ 
  Lesotho & ALL & 1999 & 112.44 & 87.96 & 141.58 & RW2 \\ 
  Lesotho & ALL & 1999 & 114.30 & 105.50 & 123.60 & UN \\ 
  Lesotho & ALL & 2000 & 106.34 & 101.16 & 111.92 & IHME \\ 
  Lesotho & ALL & 2000 & 116.32 & 92.43 & 145.71 & RW2 \\ 
  Lesotho & ALL & 2000 & 116.80 & 108.00 & 126.20 & UN \\ 
  Lesotho & ALL & 2001 & 110.11 & 104.76 & 116.03 & IHME \\ 
  Lesotho & ALL & 2001 & 119.51 & 96.08 & 147.92 & RW2 \\ 
  Lesotho & ALL & 2001 & 118.60 & 109.80 & 128.10 & UN \\ 
  Lesotho & ALL & 2002 & 112.79 & 107.27 & 118.88 & IHME \\ 
  Lesotho & ALL & 2002 & 121.94 & 98.82 & 149.76 & RW2 \\ 
  Lesotho & ALL & 2002 & 120.40 & 111.50 & 130.10 & UN \\ 
  Lesotho & ALL & 2003 & 114.97 & 109.16 & 121.27 & IHME \\ 
  Lesotho & ALL & 2003 & 123.51 & 99.70 & 152.16 & RW2 \\ 
  Lesotho & ALL & 2003 & 122.00 & 112.90 & 132.00 & UN \\ 
  Lesotho & ALL & 2004 & 116.97 & 110.73 & 123.56 & IHME \\ 
  Lesotho & ALL & 2004 & 123.78 & 98.33 & 154.54 & RW2 \\ 
  Lesotho & ALL & 2004 & 123.00 & 113.80 & 133.30 & UN \\ 
  Lesotho & ALL & 2005 & 119.61 & 112.74 & 126.78 & IHME \\ 
  Lesotho & ALL & 2005 & 123.12 & 97.46 & 154.39 & RW2 \\ 
  Lesotho & ALL & 2005 & 123.40 & 113.80 & 134.20 & UN \\ 
  Lesotho & ALL & 2006 & 120.46 & 113.06 & 128.57 & IHME \\ 
  Lesotho & ALL & 2006 & 121.27 & 96.91 & 150.53 & RW2 \\ 
  Lesotho & ALL & 2006 & 123.20 & 113.10 & 134.60 & UN \\ 
  Lesotho & ALL & 2007 & 117.42 & 109.53 & 126.29 & IHME \\ 
  Lesotho & ALL & 2007 & 118.47 & 95.33 & 146.34 & RW2 \\ 
  Lesotho & ALL & 2007 & 119.30 & 108.80 & 131.20 & UN \\ 
  Lesotho & ALL & 2008 & 114.75 & 106.16 & 124.25 & IHME \\ 
  Lesotho & ALL & 2008 & 114.96 & 91.73 & 143.55 & RW2 \\ 
  Lesotho & ALL & 2008 & 116.50 & 105.20 & 129.20 & UN \\ 
  Lesotho & ALL & 2009 & 110.40 & 101.16 & 120.67 & IHME \\ 
  Lesotho & ALL & 2009 & 110.55 & 86.64 & 140.96 & RW2 \\ 
  Lesotho & ALL & 2009 & 106.50 & 95.10 & 119.30 & UN \\ 
  Lesotho & ALL & 2010 & 104.63 & 94.58 & 116.38 & IHME \\ 
  Lesotho & ALL & 2010 & 105.56 & 80.94 & 136.42 & RW2 \\ 
  Lesotho & ALL & 2010 & 100.70 & 88.80 & 114.60 & UN \\ 
  Lesotho & ALL & 2011 & 97.41 & 86.76 & 109.66 & IHME \\ 
  Lesotho & ALL & 2011 & 100.72 & 77.57 & 129.19 & RW2 \\ 
  Lesotho & ALL & 2011 & 96.80 & 83.70 & 112.00 & UN \\ 
  Lesotho & ALL & 2012 & 95.67 & 84.35 & 108.98 & IHME \\ 
  Lesotho & ALL & 2012 & 95.79 & 73.92 & 122.91 & RW2 \\ 
  Lesotho & ALL & 2012 & 94.10 & 79.40 & 111.30 & UN \\ 
  Lesotho & ALL & 2013 & 98.87 & 85.59 & 113.95 & IHME \\ 
  Lesotho & ALL & 2013 & 91.08 & 67.51 & 121.44 & RW2 \\ 
  Lesotho & ALL & 2013 & 93.60 & 77.00 & 113.30 & UN \\ 
  Lesotho & ALL & 2014 & 100.84 & 86.18 & 117.34 & IHME \\ 
  Lesotho & ALL & 2014 & 86.49 & 57.44 & 127.54 & RW2 \\ 
  Lesotho & ALL & 2014 & 92.00 & 73.60 & 114.10 & UN \\ 
  Lesotho & ALL & 2015 & 83.67 & 70.34 & 99.44 & IHME \\ 
  Lesotho & ALL & 2015 & 82.14 & 45.32 & 143.83 & RW2 \\ 
  Lesotho & ALL & 2015 & 90.20 & 70.20 & 115.00 & UN \\ 
  Lesotho & ALL & 2016 & 78.22 & 35.40 & 163.54 & RW2 \\ 
  Lesotho & ALL & 2017 & 74.07 & 26.54 & 189.19 & RW2 \\ 
  Lesotho & ALL & 2018 & 70.28 & 19.66 & 223.41 & RW2 \\ 
  Lesotho & ALL & 2019 & 66.57 & 13.64 & 260.96 & RW2 \\ 
  Lesotho & ALL & 80-84 & 90.78 & 107.31 & 76.58 & HT-Direct \\ 
  Lesotho & ALL & 85-89 & 78.72 & 88.97 & 69.57 & HT-Direct \\ 
  Lesotho & ALL & 90-94 & 89.11 & 98.70 & 80.37 & HT-Direct \\ 
  Lesotho & ALL & 95-99 & 87.20 & 97.23 & 78.12 & HT-Direct \\ 
  Lesotho & ALL & 00-04 & 115.28 & 125.24 & 106.02 & HT-Direct \\ 
  Lesotho & ALL & 05-09 & 113.15 & 124.93 & 102.34 & HT-Direct \\ 
  Lesotho & ALL & 10-14 & 86.94 & 101.03 & 74.65 & HT-Direct \\ 
  Lesotho & ALL & 15-19 & 74.10 & 26.93 & 185.85 & RW2 \\ 
  Lesotho & BEREA & 1980 & 176.42 & 109.38 & 272.98 & RW2 \\ 
  Lesotho & BEREA & 1981 & 164.14 & 110.05 & 239.18 & RW2 \\ 
  Lesotho & BEREA & 1982 & 153.07 & 106.38 & 215.22 & RW2 \\ 
  Lesotho & BEREA & 1983 & 142.50 & 101.00 & 198.69 & RW2 \\ 
  Lesotho & BEREA & 1984 & 132.95 & 94.59 & 184.12 & RW2 \\ 
  Lesotho & BEREA & 1985 & 124.28 & 89.52 & 169.21 & RW2 \\ 
  Lesotho & BEREA & 1986 & 117.36 & 86.17 & 157.69 & RW2 \\ 
  Lesotho & BEREA & 1987 & 112.02 & 83.13 & 148.54 & RW2 \\ 
  Lesotho & BEREA & 1988 & 108.12 & 80.93 & 142.64 & RW2 \\ 
  Lesotho & BEREA & 1989 & 105.41 & 79.00 & 138.77 & RW2 \\ 
  Lesotho & BEREA & 1990 & 103.94 & 79.23 & 135.17 & RW2 \\ 
  Lesotho & BEREA & 1991 & 103.33 & 80.35 & 132.13 & RW2 \\ 
  Lesotho & BEREA & 1992 & 103.35 & 80.94 & 130.64 & RW2 \\ 
  Lesotho & BEREA & 1993 & 103.96 & 81.71 & 131.30 & RW2 \\ 
  Lesotho & BEREA & 1994 & 104.73 & 81.48 & 133.19 & RW2 \\ 
  Lesotho & BEREA & 1995 & 106.08 & 82.51 & 135.21 & RW2 \\ 
  Lesotho & BEREA & 1996 & 107.10 & 84.13 & 135.47 & RW2 \\ 
  Lesotho & BEREA & 1997 & 108.09 & 85.20 & 136.01 & RW2 \\ 
  Lesotho & BEREA & 1998 & 108.81 & 85.83 & 137.37 & RW2 \\ 
  Lesotho & BEREA & 1999 & 109.29 & 86.00 & 138.16 & RW2 \\ 
  Lesotho & BEREA & 2000 & 109.38 & 86.44 & 137.31 & RW2 \\ 
  Lesotho & BEREA & 2001 & 109.47 & 87.57 & 135.93 & RW2 \\ 
  Lesotho & BEREA & 2002 & 109.31 & 88.09 & 135.20 & RW2 \\ 
  Lesotho & BEREA & 2003 & 108.99 & 86.64 & 136.02 & RW2 \\ 
  Lesotho & BEREA & 2004 & 108.29 & 85.18 & 136.67 & RW2 \\ 
  Lesotho & BEREA & 2005 & 107.23 & 83.88 & 136.71 & RW2 \\ 
  Lesotho & BEREA & 2006 & 105.09 & 82.43 & 133.65 & RW2 \\ 
  Lesotho & BEREA & 2007 & 101.90 & 79.53 & 130.00 & RW2 \\ 
  Lesotho & BEREA & 2008 & 97.67 & 74.91 & 126.55 & RW2 \\ 
  Lesotho & BEREA & 2009 & 92.76 & 69.41 & 123.22 & RW2 \\ 
  Lesotho & BEREA & 2010 & 86.80 & 63.27 & 117.86 & RW2 \\ 
  Lesotho & BEREA & 2011 & 81.25 & 58.70 & 110.86 & RW2 \\ 
  Lesotho & BEREA & 2012 & 75.68 & 53.97 & 105.01 & RW2 \\ 
  Lesotho & BEREA & 2013 & 70.43 & 48.13 & 101.75 & RW2 \\ 
  Lesotho & BEREA & 2014 & 65.43 & 40.88 & 103.04 & RW2 \\ 
  Lesotho & BEREA & 2015 & 60.84 & 32.49 & 110.55 & RW2 \\ 
  Lesotho & BEREA & 2016 & 56.51 & 25.20 & 119.84 & RW2 \\ 
  Lesotho & BEREA & 2017 & 52.42 & 19.06 & 133.54 & RW2 \\ 
  Lesotho & BEREA & 2018 & 48.62 & 13.83 & 152.47 & RW2 \\ 
  Lesotho & BEREA & 2019 & 45.22 & 9.91 & 176.24 & RW2 \\ 
  Lesotho & BUTHA-BUTHE & 1980 & 101.59 & 55.42 & 179.56 & RW2 \\ 
  Lesotho & BUTHA-BUTHE & 1981 & 96.15 & 57.30 & 157.29 & RW2 \\ 
  Lesotho & BUTHA-BUTHE & 1982 & 90.79 & 57.14 & 141.94 & RW2 \\ 
  Lesotho & BUTHA-BUTHE & 1983 & 86.35 & 55.70 & 131.20 & RW2 \\ 
  Lesotho & BUTHA-BUTHE & 1984 & 81.90 & 53.89 & 122.37 & RW2 \\ 
  Lesotho & BUTHA-BUTHE & 1985 & 78.03 & 52.72 & 113.20 & RW2 \\ 
  Lesotho & BUTHA-BUTHE & 1986 & 75.07 & 52.16 & 106.33 & RW2 \\ 
  Lesotho & BUTHA-BUTHE & 1987 & 73.02 & 51.85 & 101.74 & RW2 \\ 
  Lesotho & BUTHA-BUTHE & 1988 & 71.78 & 51.79 & 98.29 & RW2 \\ 
  Lesotho & BUTHA-BUTHE & 1989 & 71.43 & 52.05 & 97.29 & RW2 \\ 
  Lesotho & BUTHA-BUTHE & 1990 & 71.94 & 53.32 & 96.59 & RW2 \\ 
  Lesotho & BUTHA-BUTHE & 1991 & 72.81 & 55.05 & 95.94 & RW2 \\ 
  Lesotho & BUTHA-BUTHE & 1992 & 74.19 & 56.78 & 96.35 & RW2 \\ 
  Lesotho & BUTHA-BUTHE & 1993 & 76.04 & 58.28 & 98.58 & RW2 \\ 
  Lesotho & BUTHA-BUTHE & 1994 & 78.07 & 59.42 & 101.92 & RW2 \\ 
  Lesotho & BUTHA-BUTHE & 1995 & 80.51 & 61.27 & 105.12 & RW2 \\ 
  Lesotho & BUTHA-BUTHE & 1996 & 82.76 & 63.48 & 107.41 & RW2 \\ 
  Lesotho & BUTHA-BUTHE & 1997 & 84.95 & 65.48 & 109.69 & RW2 \\ 
  Lesotho & BUTHA-BUTHE & 1998 & 86.98 & 67.25 & 112.18 & RW2 \\ 
  Lesotho & BUTHA-BUTHE & 1999 & 88.81 & 68.27 & 114.85 & RW2 \\ 
  Lesotho & BUTHA-BUTHE & 2000 & 90.47 & 70.14 & 115.81 & RW2 \\ 
  Lesotho & BUTHA-BUTHE & 2001 & 92.13 & 72.24 & 116.81 & RW2 \\ 
  Lesotho & BUTHA-BUTHE & 2002 & 93.71 & 73.86 & 118.31 & RW2 \\ 
  Lesotho & BUTHA-BUTHE & 2003 & 95.07 & 74.25 & 120.65 & RW2 \\ 
  Lesotho & BUTHA-BUTHE & 2004 & 96.11 & 74.25 & 123.41 & RW2 \\ 
  Lesotho & BUTHA-BUTHE & 2005 & 97.09 & 74.67 & 125.40 & RW2 \\ 
  Lesotho & BUTHA-BUTHE & 2006 & 96.73 & 74.51 & 125.07 & RW2 \\ 
  Lesotho & BUTHA-BUTHE & 2007 & 95.74 & 73.25 & 123.93 & RW2 \\ 
  Lesotho & BUTHA-BUTHE & 2008 & 93.58 & 70.37 & 123.96 & RW2 \\ 
  Lesotho & BUTHA-BUTHE & 2009 & 90.46 & 66.19 & 122.97 & RW2 \\ 
  Lesotho & BUTHA-BUTHE & 2010 & 86.40 & 61.23 & 120.30 & RW2 \\ 
  Lesotho & BUTHA-BUTHE & 2011 & 82.59 & 57.69 & 116.38 & RW2 \\ 
  Lesotho & BUTHA-BUTHE & 2012 & 78.51 & 53.72 & 113.21 & RW2 \\ 
  Lesotho & BUTHA-BUTHE & 2013 & 74.30 & 48.62 & 111.99 & RW2 \\ 
  Lesotho & BUTHA-BUTHE & 2014 & 70.62 & 42.25 & 116.13 & RW2 \\ 
  Lesotho & BUTHA-BUTHE & 2015 & 67.06 & 34.51 & 125.71 & RW2 \\ 
  Lesotho & BUTHA-BUTHE & 2016 & 63.68 & 27.35 & 140.45 & RW2 \\ 
  Lesotho & BUTHA-BUTHE & 2017 & 60.60 & 21.16 & 158.54 & RW2 \\ 
  Lesotho & BUTHA-BUTHE & 2018 & 57.52 & 15.90 & 182.10 & RW2 \\ 
  Lesotho & BUTHA-BUTHE & 2019 & 54.28 & 11.53 & 215.07 & RW2 \\ 
  Lesotho & LERIBE & 1980 & 112.47 & 64.84 & 188.73 & RW2 \\ 
  Lesotho & LERIBE & 1981 & 107.02 & 67.25 & 166.31 & RW2 \\ 
  Lesotho & LERIBE & 1982 & 101.87 & 67.70 & 151.78 & RW2 \\ 
  Lesotho & LERIBE & 1983 & 97.22 & 66.36 & 141.51 & RW2 \\ 
  Lesotho & LERIBE & 1984 & 93.09 & 64.26 & 133.08 & RW2 \\ 
  Lesotho & LERIBE & 1985 & 89.12 & 63.19 & 124.43 & RW2 \\ 
  Lesotho & LERIBE & 1986 & 86.41 & 62.64 & 117.92 & RW2 \\ 
  Lesotho & LERIBE & 1987 & 84.65 & 62.58 & 113.53 & RW2 \\ 
  Lesotho & LERIBE & 1988 & 83.84 & 62.36 & 111.41 & RW2 \\ 
  Lesotho & LERIBE & 1989 & 83.81 & 62.94 & 111.07 & RW2 \\ 
  Lesotho & LERIBE & 1990 & 85.06 & 64.75 & 110.89 & RW2 \\ 
  Lesotho & LERIBE & 1991 & 86.78 & 67.12 & 111.60 & RW2 \\ 
  Lesotho & LERIBE & 1992 & 89.23 & 69.91 & 113.08 & RW2 \\ 
  Lesotho & LERIBE & 1993 & 92.09 & 71.99 & 117.07 & RW2 \\ 
  Lesotho & LERIBE & 1994 & 95.38 & 74.11 & 121.84 & RW2 \\ 
  Lesotho & LERIBE & 1995 & 99.06 & 76.91 & 126.75 & RW2 \\ 
  Lesotho & LERIBE & 1996 & 102.85 & 80.85 & 130.67 & RW2 \\ 
  Lesotho & LERIBE & 1997 & 106.64 & 84.00 & 134.26 & RW2 \\ 
  Lesotho & LERIBE & 1998 & 110.11 & 86.95 & 138.85 & RW2 \\ 
  Lesotho & LERIBE & 1999 & 113.58 & 89.28 & 143.11 & RW2 \\ 
  Lesotho & LERIBE & 2000 & 116.64 & 92.33 & 146.03 & RW2 \\ 
  Lesotho & LERIBE & 2001 & 119.77 & 96.22 & 148.13 & RW2 \\ 
  Lesotho & LERIBE & 2002 & 122.92 & 99.40 & 151.20 & RW2 \\ 
  Lesotho & LERIBE & 2003 & 125.86 & 100.99 & 155.01 & RW2 \\ 
  Lesotho & LERIBE & 2004 & 128.28 & 101.51 & 160.08 & RW2 \\ 
  Lesotho & LERIBE & 2005 & 130.65 & 103.50 & 164.34 & RW2 \\ 
  Lesotho & LERIBE & 2006 & 131.37 & 104.37 & 164.21 & RW2 \\ 
  Lesotho & LERIBE & 2007 & 130.91 & 103.99 & 164.02 & RW2 \\ 
  Lesotho & LERIBE & 2008 & 128.95 & 101.13 & 163.87 & RW2 \\ 
  Lesotho & LERIBE & 2009 & 126.01 & 96.24 & 163.56 & RW2 \\ 
  Lesotho & LERIBE & 2010 & 121.52 & 90.71 & 160.50 & RW2 \\ 
  Lesotho & LERIBE & 2011 & 117.11 & 86.84 & 155.25 & RW2 \\ 
  Lesotho & LERIBE & 2012 & 112.39 & 82.81 & 151.34 & RW2 \\ 
  Lesotho & LERIBE & 2013 & 107.63 & 75.94 & 150.08 & RW2 \\ 
  Lesotho & LERIBE & 2014 & 103.04 & 66.65 & 156.11 & RW2 \\ 
  Lesotho & LERIBE & 2015 & 99.00 & 54.36 & 172.40 & RW2 \\ 
  Lesotho & LERIBE & 2016 & 94.60 & 43.99 & 190.38 & RW2 \\ 
  Lesotho & LERIBE & 2017 & 90.53 & 33.72 & 216.02 & RW2 \\ 
  Lesotho & LERIBE & 2018 & 86.81 & 25.64 & 246.62 & RW2 \\ 
  Lesotho & LERIBE & 2019 & 83.16 & 18.85 & 283.56 & RW2 \\ 
  Lesotho & MAFETENG & 1980 & 107.51 & 59.17 & 188.87 & RW2 \\ 
  Lesotho & MAFETENG & 1981 & 102.49 & 61.77 & 165.73 & RW2 \\ 
  Lesotho & MAFETENG & 1982 & 97.54 & 61.87 & 150.77 & RW2 \\ 
  Lesotho & MAFETENG & 1983 & 92.82 & 60.56 & 140.43 & RW2 \\ 
  Lesotho & MAFETENG & 1984 & 88.64 & 59.19 & 131.26 & RW2 \\ 
  Lesotho & MAFETENG & 1985 & 84.74 & 57.76 & 122.17 & RW2 \\ 
  Lesotho & MAFETENG & 1986 & 82.06 & 57.36 & 115.56 & RW2 \\ 
  Lesotho & MAFETENG & 1987 & 80.22 & 57.42 & 110.73 & RW2 \\ 
  Lesotho & MAFETENG & 1988 & 79.26 & 57.25 & 108.39 & RW2 \\ 
  Lesotho & MAFETENG & 1989 & 79.18 & 57.64 & 107.49 & RW2 \\ 
  Lesotho & MAFETENG & 1990 & 80.04 & 59.28 & 107.56 & RW2 \\ 
  Lesotho & MAFETENG & 1991 & 81.53 & 61.43 & 107.75 & RW2 \\ 
  Lesotho & MAFETENG & 1992 & 83.60 & 63.55 & 109.11 & RW2 \\ 
  Lesotho & MAFETENG & 1993 & 86.08 & 65.72 & 111.82 & RW2 \\ 
  Lesotho & MAFETENG & 1994 & 88.91 & 67.47 & 116.19 & RW2 \\ 
  Lesotho & MAFETENG & 1995 & 92.14 & 70.03 & 120.21 & RW2 \\ 
  Lesotho & MAFETENG & 1996 & 95.40 & 73.03 & 123.41 & RW2 \\ 
  Lesotho & MAFETENG & 1997 & 98.61 & 76.15 & 126.47 & RW2 \\ 
  Lesotho & MAFETENG & 1998 & 101.70 & 78.86 & 130.61 & RW2 \\ 
  Lesotho & MAFETENG & 1999 & 104.67 & 80.95 & 134.60 & RW2 \\ 
  Lesotho & MAFETENG & 2000 & 107.25 & 83.33 & 136.09 & RW2 \\ 
  Lesotho & MAFETENG & 2001 & 110.02 & 87.04 & 137.83 & RW2 \\ 
  Lesotho & MAFETENG & 2002 & 112.76 & 89.80 & 140.40 & RW2 \\ 
  Lesotho & MAFETENG & 2003 & 115.16 & 91.37 & 144.00 & RW2 \\ 
  Lesotho & MAFETENG & 2004 & 117.34 & 92.08 & 147.83 & RW2 \\ 
  Lesotho & MAFETENG & 2005 & 119.35 & 93.38 & 151.73 & RW2 \\ 
  Lesotho & MAFETENG & 2006 & 119.88 & 94.11 & 151.79 & RW2 \\ 
  Lesotho & MAFETENG & 2007 & 119.17 & 93.20 & 151.12 & RW2 \\ 
  Lesotho & MAFETENG & 2008 & 117.30 & 90.21 & 151.22 & RW2 \\ 
  Lesotho & MAFETENG & 2009 & 114.11 & 85.64 & 150.94 & RW2 \\ 
  Lesotho & MAFETENG & 2010 & 109.89 & 79.31 & 149.24 & RW2 \\ 
  Lesotho & MAFETENG & 2011 & 105.29 & 75.07 & 145.68 & RW2 \\ 
  Lesotho & MAFETENG & 2012 & 100.80 & 70.15 & 142.65 & RW2 \\ 
  Lesotho & MAFETENG & 2013 & 96.13 & 63.57 & 143.01 & RW2 \\ 
  Lesotho & MAFETENG & 2014 & 91.75 & 55.20 & 147.52 & RW2 \\ 
  Lesotho & MAFETENG & 2015 & 87.75 & 45.43 & 162.50 & RW2 \\ 
  Lesotho & MAFETENG & 2016 & 83.72 & 36.48 & 179.66 & RW2 \\ 
  Lesotho & MAFETENG & 2017 & 79.98 & 28.21 & 204.26 & RW2 \\ 
  Lesotho & MAFETENG & 2018 & 76.05 & 20.96 & 233.87 & RW2 \\ 
  Lesotho & MAFETENG & 2019 & 72.45 & 15.67 & 270.02 & RW2 \\ 
  Lesotho & MASERU & 1980 & 81.64 & 47.11 & 137.81 & RW2 \\ 
  Lesotho & MASERU & 1981 & 78.84 & 49.72 & 122.38 & RW2 \\ 
  Lesotho & MASERU & 1982 & 76.12 & 50.26 & 113.30 & RW2 \\ 
  Lesotho & MASERU & 1983 & 73.46 & 49.82 & 107.17 & RW2 \\ 
  Lesotho & MASERU & 1984 & 71.22 & 48.97 & 102.85 & RW2 \\ 
  Lesotho & MASERU & 1985 & 69.26 & 48.41 & 97.14 & RW2 \\ 
  Lesotho & MASERU & 1986 & 67.98 & 48.82 & 93.42 & RW2 \\ 
  Lesotho & MASERU & 1987 & 67.46 & 49.32 & 91.31 & RW2 \\ 
  Lesotho & MASERU & 1988 & 67.78 & 50.04 & 90.87 & RW2 \\ 
  Lesotho & MASERU & 1989 & 68.84 & 51.12 & 91.52 & RW2 \\ 
  Lesotho & MASERU & 1990 & 70.73 & 53.76 & 92.99 & RW2 \\ 
  Lesotho & MASERU & 1991 & 73.22 & 56.53 & 94.41 & RW2 \\ 
  Lesotho & MASERU & 1992 & 76.20 & 59.73 & 96.73 & RW2 \\ 
  Lesotho & MASERU & 1993 & 79.74 & 62.70 & 100.95 & RW2 \\ 
  Lesotho & MASERU & 1994 & 83.59 & 65.31 & 106.34 & RW2 \\ 
  Lesotho & MASERU & 1995 & 87.86 & 68.98 & 111.73 & RW2 \\ 
  Lesotho & MASERU & 1996 & 92.18 & 72.98 & 116.16 & RW2 \\ 
  Lesotho & MASERU & 1997 & 96.59 & 77.05 & 120.82 & RW2 \\ 
  Lesotho & MASERU & 1998 & 100.92 & 80.27 & 126.25 & RW2 \\ 
  Lesotho & MASERU & 1999 & 105.04 & 83.38 & 131.62 & RW2 \\ 
  Lesotho & MASERU & 2000 & 108.93 & 87.22 & 135.15 & RW2 \\ 
  Lesotho & MASERU & 2001 & 112.87 & 91.34 & 138.68 & RW2 \\ 
  Lesotho & MASERU & 2002 & 116.86 & 95.20 & 142.86 & RW2 \\ 
  Lesotho & MASERU & 2003 & 120.47 & 97.01 & 148.26 & RW2 \\ 
  Lesotho & MASERU & 2004 & 123.78 & 98.37 & 154.32 & RW2 \\ 
  Lesotho & MASERU & 2005 & 126.76 & 100.27 & 159.59 & RW2 \\ 
  Lesotho & MASERU & 2006 & 128.37 & 101.88 & 161.03 & RW2 \\ 
  Lesotho & MASERU & 2007 & 128.66 & 101.61 & 161.82 & RW2 \\ 
  Lesotho & MASERU & 2008 & 127.61 & 99.31 & 162.61 & RW2 \\ 
  Lesotho & MASERU & 2009 & 125.31 & 95.41 & 163.08 & RW2 \\ 
  Lesotho & MASERU & 2010 & 121.42 & 90.30 & 160.75 & RW2 \\ 
  Lesotho & MASERU & 2011 & 117.71 & 86.90 & 156.71 & RW2 \\ 
  Lesotho & MASERU & 2012 & 113.52 & 83.22 & 152.72 & RW2 \\ 
  Lesotho & MASERU & 2013 & 109.35 & 77.13 & 152.98 & RW2 \\ 
  Lesotho & MASERU & 2014 & 105.26 & 68.08 & 158.51 & RW2 \\ 
  Lesotho & MASERU & 2015 & 101.64 & 56.28 & 175.20 & RW2 \\ 
  Lesotho & MASERU & 2016 & 97.76 & 45.19 & 195.31 & RW2 \\ 
  Lesotho & MASERU & 2017 & 94.25 & 35.34 & 225.47 & RW2 \\ 
  Lesotho & MASERU & 2018 & 90.82 & 26.94 & 257.80 & RW2 \\ 
  Lesotho & MASERU & 2019 & 87.62 & 20.40 & 296.72 & RW2 \\ 
  Lesotho & MOHALE'S HOEK & 1980 & 146.72 & 89.12 & 233.24 & RW2 \\ 
  Lesotho & MOHALE'S HOEK & 1981 & 139.03 & 91.82 & 205.19 & RW2 \\ 
  Lesotho & MOHALE'S HOEK & 1982 & 131.96 & 91.03 & 187.54 & RW2 \\ 
  Lesotho & MOHALE'S HOEK & 1983 & 124.89 & 88.03 & 175.11 & RW2 \\ 
  Lesotho & MOHALE'S HOEK & 1984 & 118.78 & 84.98 & 164.53 & RW2 \\ 
  Lesotho & MOHALE'S HOEK & 1985 & 113.11 & 82.09 & 153.47 & RW2 \\ 
  Lesotho & MOHALE'S HOEK & 1986 & 108.88 & 80.55 & 145.09 & RW2 \\ 
  Lesotho & MOHALE'S HOEK & 1987 & 105.98 & 79.93 & 139.59 & RW2 \\ 
  Lesotho & MOHALE'S HOEK & 1988 & 104.18 & 78.85 & 136.39 & RW2 \\ 
  Lesotho & MOHALE'S HOEK & 1989 & 103.55 & 78.55 & 134.83 & RW2 \\ 
  Lesotho & MOHALE'S HOEK & 1990 & 104.21 & 80.43 & 134.01 & RW2 \\ 
  Lesotho & MOHALE'S HOEK & 1991 & 105.52 & 82.81 & 133.86 & RW2 \\ 
  Lesotho & MOHALE'S HOEK & 1992 & 107.51 & 85.46 & 134.68 & RW2 \\ 
  Lesotho & MOHALE'S HOEK & 1993 & 110.11 & 87.15 & 138.13 & RW2 \\ 
  Lesotho & MOHALE'S HOEK & 1994 & 113.11 & 89.11 & 142.36 & RW2 \\ 
  Lesotho & MOHALE'S HOEK & 1995 & 116.50 & 91.78 & 147.06 & RW2 \\ 
  Lesotho & MOHALE'S HOEK & 1996 & 119.89 & 95.43 & 150.28 & RW2 \\ 
  Lesotho & MOHALE'S HOEK & 1997 & 123.07 & 98.51 & 153.35 & RW2 \\ 
  Lesotho & MOHALE'S HOEK & 1998 & 126.19 & 100.78 & 157.25 & RW2 \\ 
  Lesotho & MOHALE'S HOEK & 1999 & 129.02 & 102.56 & 161.58 & RW2 \\ 
  Lesotho & MOHALE'S HOEK & 2000 & 131.16 & 104.75 & 163.09 & RW2 \\ 
  Lesotho & MOHALE'S HOEK & 2001 & 133.44 & 107.67 & 164.25 & RW2 \\ 
  Lesotho & MOHALE'S HOEK & 2002 & 135.63 & 109.99 & 166.40 & RW2 \\ 
  Lesotho & MOHALE'S HOEK & 2003 & 137.36 & 110.23 & 169.22 & RW2 \\ 
  Lesotho & MOHALE'S HOEK & 2004 & 138.72 & 109.85 & 173.12 & RW2 \\ 
  Lesotho & MOHALE'S HOEK & 2005 & 139.70 & 110.44 & 175.81 & RW2 \\ 
  Lesotho & MOHALE'S HOEK & 2006 & 139.26 & 110.28 & 174.28 & RW2 \\ 
  Lesotho & MOHALE'S HOEK & 2007 & 137.12 & 108.48 & 172.31 & RW2 \\ 
  Lesotho & MOHALE'S HOEK & 2008 & 133.85 & 104.07 & 170.45 & RW2 \\ 
  Lesotho & MOHALE'S HOEK & 2009 & 129.10 & 98.03 & 168.67 & RW2 \\ 
  Lesotho & MOHALE'S HOEK & 2010 & 123.23 & 90.90 & 164.18 & RW2 \\ 
  Lesotho & MOHALE'S HOEK & 2011 & 117.14 & 85.83 & 156.87 & RW2 \\ 
  Lesotho & MOHALE'S HOEK & 2012 & 111.02 & 80.57 & 151.56 & RW2 \\ 
  Lesotho & MOHALE'S HOEK & 2013 & 105.09 & 72.91 & 148.95 & RW2 \\ 
  Lesotho & MOHALE'S HOEK & 2014 & 99.38 & 62.83 & 152.44 & RW2 \\ 
  Lesotho & MOHALE'S HOEK & 2015 & 94.12 & 51.24 & 164.75 & RW2 \\ 
  Lesotho & MOHALE'S HOEK & 2016 & 89.00 & 40.30 & 181.43 & RW2 \\ 
  Lesotho & MOHALE'S HOEK & 2017 & 84.06 & 31.09 & 203.61 & RW2 \\ 
  Lesotho & MOHALE'S HOEK & 2018 & 79.16 & 23.56 & 229.95 & RW2 \\ 
  Lesotho & MOHALE'S HOEK & 2019 & 74.93 & 17.11 & 265.20 & RW2 \\ 
  Lesotho & MOKHOTLONG & 1980 & 167.71 & 97.59 & 273.12 & RW2 \\ 
  Lesotho & MOKHOTLONG & 1981 & 157.74 & 99.90 & 239.73 & RW2 \\ 
  Lesotho & MOKHOTLONG & 1982 & 148.13 & 97.84 & 216.51 & RW2 \\ 
  Lesotho & MOKHOTLONG & 1983 & 139.43 & 94.93 & 200.12 & RW2 \\ 
  Lesotho & MOKHOTLONG & 1984 & 131.49 & 90.96 & 186.64 & RW2 \\ 
  Lesotho & MOKHOTLONG & 1985 & 124.18 & 87.85 & 172.10 & RW2 \\ 
  Lesotho & MOKHOTLONG & 1986 & 118.50 & 85.77 & 160.45 & RW2 \\ 
  Lesotho & MOKHOTLONG & 1987 & 114.14 & 84.40 & 152.33 & RW2 \\ 
  Lesotho & MOKHOTLONG & 1988 & 111.05 & 83.19 & 146.42 & RW2 \\ 
  Lesotho & MOKHOTLONG & 1989 & 109.24 & 82.36 & 142.90 & RW2 \\ 
  Lesotho & MOKHOTLONG & 1990 & 108.76 & 83.49 & 140.57 & RW2 \\ 
  Lesotho & MOKHOTLONG & 1991 & 108.92 & 85.31 & 138.70 & RW2 \\ 
  Lesotho & MOKHOTLONG & 1992 & 109.70 & 86.91 & 137.34 & RW2 \\ 
  Lesotho & MOKHOTLONG & 1993 & 110.93 & 87.89 & 138.94 & RW2 \\ 
  Lesotho & MOKHOTLONG & 1994 & 112.51 & 88.32 & 142.24 & RW2 \\ 
  Lesotho & MOKHOTLONG & 1995 & 114.47 & 89.85 & 145.08 & RW2 \\ 
  Lesotho & MOKHOTLONG & 1996 & 116.19 & 91.76 & 146.14 & RW2 \\ 
  Lesotho & MOKHOTLONG & 1997 & 117.89 & 93.63 & 147.29 & RW2 \\ 
  Lesotho & MOKHOTLONG & 1998 & 119.13 & 94.64 & 149.20 & RW2 \\ 
  Lesotho & MOKHOTLONG & 1999 & 120.24 & 94.83 & 151.87 & RW2 \\ 
  Lesotho & MOKHOTLONG & 2000 & 120.83 & 95.82 & 150.92 & RW2 \\ 
  Lesotho & MOKHOTLONG & 2001 & 121.46 & 97.59 & 149.68 & RW2 \\ 
  Lesotho & MOKHOTLONG & 2002 & 121.99 & 98.45 & 149.99 & RW2 \\ 
  Lesotho & MOKHOTLONG & 2003 & 122.22 & 97.95 & 151.11 & RW2 \\ 
  Lesotho & MOKHOTLONG & 2004 & 122.21 & 96.45 & 153.14 & RW2 \\ 
  Lesotho & MOKHOTLONG & 2005 & 121.85 & 95.62 & 154.12 & RW2 \\ 
  Lesotho & MOKHOTLONG & 2006 & 120.24 & 94.56 & 151.86 & RW2 \\ 
  Lesotho & MOKHOTLONG & 2007 & 117.25 & 91.75 & 148.82 & RW2 \\ 
  Lesotho & MOKHOTLONG & 2008 & 113.24 & 87.49 & 145.60 & RW2 \\ 
  Lesotho & MOKHOTLONG & 2009 & 108.57 & 82.04 & 142.77 & RW2 \\ 
  Lesotho & MOKHOTLONG & 2010 & 102.56 & 75.29 & 137.08 & RW2 \\ 
  Lesotho & MOKHOTLONG & 2011 & 96.77 & 71.03 & 130.48 & RW2 \\ 
  Lesotho & MOKHOTLONG & 2012 & 90.97 & 65.89 & 124.58 & RW2 \\ 
  Lesotho & MOKHOTLONG & 2013 & 85.37 & 59.26 & 121.26 & RW2 \\ 
  Lesotho & MOKHOTLONG & 2014 & 80.19 & 50.72 & 123.52 & RW2 \\ 
  Lesotho & MOKHOTLONG & 2015 & 75.27 & 41.12 & 134.14 & RW2 \\ 
  Lesotho & MOKHOTLONG & 2016 & 70.54 & 31.96 & 146.21 & RW2 \\ 
  Lesotho & MOKHOTLONG & 2017 & 66.29 & 24.32 & 163.65 & RW2 \\ 
  Lesotho & MOKHOTLONG & 2018 & 61.72 & 18.05 & 187.36 & RW2 \\ 
  Lesotho & MOKHOTLONG & 2019 & 57.84 & 12.99 & 215.34 & RW2 \\ 
  Lesotho & QASHA'S NEK & 1980 & 177.29 & 105.95 & 286.35 & RW2 \\ 
  Lesotho & QASHA'S NEK & 1981 & 166.47 & 108.06 & 250.28 & RW2 \\ 
  Lesotho & QASHA'S NEK & 1982 & 155.77 & 105.47 & 225.82 & RW2 \\ 
  Lesotho & QASHA'S NEK & 1983 & 145.97 & 100.82 & 208.19 & RW2 \\ 
  Lesotho & QASHA'S NEK & 1984 & 137.11 & 95.92 & 193.28 & RW2 \\ 
  Lesotho & QASHA'S NEK & 1985 & 129.00 & 92.10 & 177.92 & RW2 \\ 
  Lesotho & QASHA'S NEK & 1986 & 122.62 & 89.25 & 165.33 & RW2 \\ 
  Lesotho & QASHA'S NEK & 1987 & 117.60 & 87.27 & 156.60 & RW2 \\ 
  Lesotho & QASHA'S NEK & 1988 & 113.97 & 85.34 & 150.02 & RW2 \\ 
  Lesotho & QASHA'S NEK & 1989 & 111.67 & 84.55 & 145.89 & RW2 \\ 
  Lesotho & QASHA'S NEK & 1990 & 110.91 & 85.08 & 143.24 & RW2 \\ 
  Lesotho & QASHA'S NEK & 1991 & 110.78 & 86.57 & 140.56 & RW2 \\ 
  Lesotho & QASHA'S NEK & 1992 & 111.35 & 87.95 & 139.98 & RW2 \\ 
  Lesotho & QASHA'S NEK & 1993 & 112.48 & 89.00 & 141.22 & RW2 \\ 
  Lesotho & QASHA'S NEK & 1994 & 114.02 & 89.79 & 143.95 & RW2 \\ 
  Lesotho & QASHA'S NEK & 1995 & 116.06 & 90.81 & 146.73 & RW2 \\ 
  Lesotho & QASHA'S NEK & 1996 & 117.83 & 93.26 & 148.23 & RW2 \\ 
  Lesotho & QASHA'S NEK & 1997 & 119.65 & 94.81 & 149.45 & RW2 \\ 
  Lesotho & QASHA'S NEK & 1998 & 121.05 & 95.69 & 151.62 & RW2 \\ 
  Lesotho & QASHA'S NEK & 1999 & 122.39 & 96.51 & 154.30 & RW2 \\ 
  Lesotho & QASHA'S NEK & 2000 & 123.22 & 96.89 & 154.22 & RW2 \\ 
  Lesotho & QASHA'S NEK & 2001 & 124.23 & 98.86 & 154.50 & RW2 \\ 
  Lesotho & QASHA'S NEK & 2002 & 125.00 & 99.98 & 155.06 & RW2 \\ 
  Lesotho & QASHA'S NEK & 2003 & 125.59 & 99.69 & 156.88 & RW2 \\ 
  Lesotho & QASHA'S NEK & 2004 & 125.81 & 98.23 & 159.05 & RW2 \\ 
  Lesotho & QASHA'S NEK & 2005 & 125.73 & 98.11 & 160.27 & RW2 \\ 
  Lesotho & QASHA'S NEK & 2006 & 124.31 & 97.08 & 158.07 & RW2 \\ 
  Lesotho & QASHA'S NEK & 2007 & 121.63 & 94.52 & 155.55 & RW2 \\ 
  Lesotho & QASHA'S NEK & 2008 & 117.76 & 90.07 & 152.82 & RW2 \\ 
  Lesotho & QASHA'S NEK & 2009 & 112.76 & 83.93 & 150.52 & RW2 \\ 
  Lesotho & QASHA'S NEK & 2010 & 106.75 & 76.85 & 145.64 & RW2 \\ 
  Lesotho & QASHA'S NEK & 2011 & 100.84 & 71.79 & 139.51 & RW2 \\ 
  Lesotho & QASHA'S NEK & 2012 & 94.79 & 66.22 & 134.18 & RW2 \\ 
  Lesotho & QASHA'S NEK & 2013 & 88.91 & 59.59 & 131.31 & RW2 \\ 
  Lesotho & QASHA'S NEK & 2014 & 83.37 & 51.18 & 133.41 & RW2 \\ 
  Lesotho & QASHA'S NEK & 2015 & 78.38 & 41.01 & 143.69 & RW2 \\ 
  Lesotho & QASHA'S NEK & 2016 & 73.70 & 32.48 & 156.70 & RW2 \\ 
  Lesotho & QASHA'S NEK & 2017 & 68.85 & 24.67 & 176.39 & RW2 \\ 
  Lesotho & QASHA'S NEK & 2018 & 64.64 & 18.37 & 199.01 & RW2 \\ 
  Lesotho & QASHA'S NEK & 2019 & 60.65 & 13.38 & 231.75 & RW2 \\ 
  Lesotho & QUTHING & 1980 & 169.18 & 101.06 & 269.52 & RW2 \\ 
  Lesotho & QUTHING & 1981 & 159.46 & 103.71 & 237.55 & RW2 \\ 
  Lesotho & QUTHING & 1982 & 150.18 & 101.74 & 214.76 & RW2 \\ 
  Lesotho & QUTHING & 1983 & 141.36 & 97.85 & 199.56 & RW2 \\ 
  Lesotho & QUTHING & 1984 & 133.51 & 93.73 & 187.29 & RW2 \\ 
  Lesotho & QUTHING & 1985 & 126.10 & 89.84 & 173.37 & RW2 \\ 
  Lesotho & QUTHING & 1986 & 120.50 & 87.78 & 163.20 & RW2 \\ 
  Lesotho & QUTHING & 1987 & 116.31 & 86.16 & 155.31 & RW2 \\ 
  Lesotho & QUTHING & 1988 & 113.39 & 84.67 & 149.71 & RW2 \\ 
  Lesotho & QUTHING & 1989 & 111.74 & 84.19 & 146.88 & RW2 \\ 
  Lesotho & QUTHING & 1990 & 111.49 & 85.32 & 144.10 & RW2 \\ 
  Lesotho & QUTHING & 1991 & 112.00 & 87.43 & 142.54 & RW2 \\ 
  Lesotho & QUTHING & 1992 & 112.98 & 89.27 & 141.62 & RW2 \\ 
  Lesotho & QUTHING & 1993 & 114.62 & 90.57 & 144.06 & RW2 \\ 
  Lesotho & QUTHING & 1994 & 116.68 & 91.46 & 147.57 & RW2 \\ 
  Lesotho & QUTHING & 1995 & 118.88 & 92.80 & 151.19 & RW2 \\ 
  Lesotho & QUTHING & 1996 & 121.14 & 95.24 & 152.81 & RW2 \\ 
  Lesotho & QUTHING & 1997 & 122.95 & 96.96 & 154.93 & RW2 \\ 
  Lesotho & QUTHING & 1998 & 124.53 & 98.02 & 157.17 & RW2 \\ 
  Lesotho & QUTHING & 1999 & 126.06 & 98.85 & 159.99 & RW2 \\ 
  Lesotho & QUTHING & 2000 & 126.87 & 99.83 & 159.70 & RW2 \\ 
  Lesotho & QUTHING & 2001 & 127.72 & 101.84 & 159.24 & RW2 \\ 
  Lesotho & QUTHING & 2002 & 128.49 & 102.30 & 159.84 & RW2 \\ 
  Lesotho & QUTHING & 2003 & 128.90 & 101.85 & 161.83 & RW2 \\ 
  Lesotho & QUTHING & 2004 & 128.97 & 100.40 & 163.51 & RW2 \\ 
  Lesotho & QUTHING & 2005 & 128.59 & 99.94 & 164.50 & RW2 \\ 
  Lesotho & QUTHING & 2006 & 126.99 & 98.73 & 162.38 & RW2 \\ 
  Lesotho & QUTHING & 2007 & 123.86 & 95.67 & 158.51 & RW2 \\ 
  Lesotho & QUTHING & 2008 & 119.53 & 90.69 & 156.32 & RW2 \\ 
  Lesotho & QUTHING & 2009 & 114.42 & 84.44 & 153.46 & RW2 \\ 
  Lesotho & QUTHING & 2010 & 107.97 & 77.16 & 148.49 & RW2 \\ 
  Lesotho & QUTHING & 2011 & 101.76 & 71.51 & 141.93 & RW2 \\ 
  Lesotho & QUTHING & 2012 & 95.38 & 65.61 & 136.32 & RW2 \\ 
  Lesotho & QUTHING & 2013 & 89.34 & 58.62 & 133.46 & RW2 \\ 
  Lesotho & QUTHING & 2014 & 83.61 & 49.87 & 135.35 & RW2 \\ 
  Lesotho & QUTHING & 2015 & 78.32 & 40.17 & 145.87 & RW2 \\ 
  Lesotho & QUTHING & 2016 & 73.10 & 31.59 & 158.82 & RW2 \\ 
  Lesotho & QUTHING & 2017 & 68.52 & 23.79 & 176.86 & RW2 \\ 
  Lesotho & QUTHING & 2018 & 63.85 & 17.55 & 199.14 & RW2 \\ 
  Lesotho & QUTHING & 2019 & 59.58 & 12.82 & 229.13 & RW2 \\ 
  Lesotho & THABA-TSEKA & 1980 & 190.40 & 114.40 & 301.20 & RW2 \\ 
  Lesotho & THABA-TSEKA & 1981 & 177.98 & 115.23 & 264.43 & RW2 \\ 
  Lesotho & THABA-TSEKA & 1982 & 166.12 & 112.36 & 238.50 & RW2 \\ 
  Lesotho & THABA-TSEKA & 1983 & 155.07 & 107.06 & 220.42 & RW2 \\ 
  Lesotho & THABA-TSEKA & 1984 & 145.02 & 101.09 & 204.48 & RW2 \\ 
  Lesotho & THABA-TSEKA & 1985 & 135.82 & 96.29 & 187.82 & RW2 \\ 
  Lesotho & THABA-TSEKA & 1986 & 128.60 & 92.75 & 174.55 & RW2 \\ 
  Lesotho & THABA-TSEKA & 1987 & 122.61 & 90.06 & 164.29 & RW2 \\ 
  Lesotho & THABA-TSEKA & 1988 & 118.53 & 87.68 & 157.51 & RW2 \\ 
  Lesotho & THABA-TSEKA & 1989 & 115.62 & 86.13 & 152.97 & RW2 \\ 
  Lesotho & THABA-TSEKA & 1990 & 114.15 & 86.59 & 149.00 & RW2 \\ 
  Lesotho & THABA-TSEKA & 1991 & 113.32 & 87.64 & 145.77 & RW2 \\ 
  Lesotho & THABA-TSEKA & 1992 & 113.30 & 88.71 & 143.63 & RW2 \\ 
  Lesotho & THABA-TSEKA & 1993 & 113.84 & 89.50 & 143.72 & RW2 \\ 
  Lesotho & THABA-TSEKA & 1994 & 114.62 & 89.61 & 145.39 & RW2 \\ 
  Lesotho & THABA-TSEKA & 1995 & 115.75 & 90.61 & 146.79 & RW2 \\ 
  Lesotho & THABA-TSEKA & 1996 & 116.73 & 92.28 & 146.47 & RW2 \\ 
  Lesotho & THABA-TSEKA & 1997 & 117.51 & 93.54 & 146.89 & RW2 \\ 
  Lesotho & THABA-TSEKA & 1998 & 118.04 & 94.37 & 147.40 & RW2 \\ 
  Lesotho & THABA-TSEKA & 1999 & 118.29 & 93.93 & 148.35 & RW2 \\ 
  Lesotho & THABA-TSEKA & 2000 & 117.89 & 93.95 & 146.46 & RW2 \\ 
  Lesotho & THABA-TSEKA & 2001 & 117.63 & 94.99 & 144.78 & RW2 \\ 
  Lesotho & THABA-TSEKA & 2002 & 117.00 & 94.92 & 143.30 & RW2 \\ 
  Lesotho & THABA-TSEKA & 2003 & 116.38 & 93.40 & 144.14 & RW2 \\ 
  Lesotho & THABA-TSEKA & 2004 & 115.03 & 90.74 & 144.44 & RW2 \\ 
  Lesotho & THABA-TSEKA & 2005 & 113.54 & 89.33 & 143.63 & RW2 \\ 
  Lesotho & THABA-TSEKA & 2006 & 110.71 & 87.07 & 139.92 & RW2 \\ 
  Lesotho & THABA-TSEKA & 2007 & 106.95 & 83.40 & 135.91 & RW2 \\ 
  Lesotho & THABA-TSEKA & 2008 & 102.06 & 78.14 & 132.23 & RW2 \\ 
  Lesotho & THABA-TSEKA & 2009 & 96.42 & 71.83 & 128.39 & RW2 \\ 
  Lesotho & THABA-TSEKA & 2010 & 89.81 & 64.87 & 122.64 & RW2 \\ 
  Lesotho & THABA-TSEKA & 2011 & 83.71 & 59.26 & 115.29 & RW2 \\ 
  Lesotho & THABA-TSEKA & 2012 & 77.52 & 54.12 & 109.59 & RW2 \\ 
  Lesotho & THABA-TSEKA & 2013 & 71.73 & 47.93 & 105.99 & RW2 \\ 
  Lesotho & THABA-TSEKA & 2014 & 66.22 & 40.24 & 106.36 & RW2 \\ 
  Lesotho & THABA-TSEKA & 2015 & 61.44 & 32.03 & 113.94 & RW2 \\ 
  Lesotho & THABA-TSEKA & 2016 & 56.92 & 24.81 & 123.07 & RW2 \\ 
  Lesotho & THABA-TSEKA & 2017 & 52.44 & 18.55 & 135.09 & RW2 \\ 
  Lesotho & THABA-TSEKA & 2018 & 48.27 & 13.65 & 151.90 & RW2 \\ 
  Lesotho & THABA-TSEKA & 2019 & 44.67 & 9.73 & 176.36 & RW2 \\ 
  Liberia & ALL & 1980 & 243.13 & 233.47 & 252.25 & IHME \\ 
  Liberia & ALL & 1980 & 237.64 & 169.20 & 324.10 & RW2 \\ 
  Liberia & ALL & 1980 & 241.10 & 220.10 & 264.40 & UN \\ 
  Liberia & ALL & 1981 & 239.34 & 230.37 & 248.02 & IHME \\ 
  Liberia & ALL & 1981 & 237.24 & 184.57 & 299.36 & RW2 \\ 
  Liberia & ALL & 1981 & 238.00 & 217.50 & 260.30 & UN \\ 
  Liberia & ALL & 1982 & 236.32 & 227.61 & 244.77 & IHME \\ 
  Liberia & ALL & 1982 & 236.87 & 191.40 & 290.15 & RW2 \\ 
  Liberia & ALL & 1982 & 235.60 & 215.80 & 257.00 & UN \\ 
  Liberia & ALL & 1983 & 233.41 & 225.51 & 241.36 & IHME \\ 
  Liberia & ALL & 1983 & 236.32 & 190.34 & 290.96 & RW2 \\ 
  Liberia & ALL & 1983 & 234.00 & 214.80 & 254.90 & UN \\ 
  Liberia & ALL & 1984 & 231.63 & 224.02 & 239.08 & IHME \\ 
  Liberia & ALL & 1984 & 236.77 & 187.95 & 294.61 & RW2 \\ 
  Liberia & ALL & 1984 & 233.70 & 215.00 & 253.90 & UN \\ 
  Liberia & ALL & 1985 & 230.15 & 222.87 & 237.75 & IHME \\ 
  Liberia & ALL & 1985 & 236.37 & 190.35 & 288.38 & RW2 \\ 
  Liberia & ALL & 1985 & 235.00 & 216.40 & 255.00 & UN \\ 
  Liberia & ALL & 1986 & 228.87 & 221.40 & 236.68 & IHME \\ 
  Liberia & ALL & 1986 & 238.29 & 194.85 & 286.89 & RW2 \\ 
  Liberia & ALL & 1986 & 237.80 & 219.10 & 257.70 & UN \\ 
  Liberia & ALL & 1987 & 227.51 & 219.94 & 235.21 & IHME \\ 
  Liberia & ALL & 1987 & 241.53 & 199.52 & 288.49 & RW2 \\ 
  Liberia & ALL & 1987 & 242.10 & 223.40 & 262.20 & UN \\ 
  Liberia & ALL & 1988 & 225.42 & 217.75 & 232.94 & IHME \\ 
  Liberia & ALL & 1988 & 245.61 & 201.12 & 293.49 & RW2 \\ 
  Liberia & ALL & 1988 & 246.90 & 227.60 & 267.90 & UN \\ 
  Liberia & ALL & 1989 & 223.01 & 215.21 & 230.78 & IHME \\ 
  Liberia & ALL & 1989 & 250.10 & 203.01 & 300.09 & RW2 \\ 
  Liberia & ALL & 1989 & 251.50 & 231.40 & 273.70 & UN \\ 
  Liberia & ALL & 1990 & 225.87 & 216.95 & 235.55 & IHME \\ 
  Liberia & ALL & 1990 & 255.91 & 211.43 & 309.16 & RW2 \\ 
  Liberia & ALL & 1990 & 255.00 & 234.20 & 278.30 & UN \\ 
  Liberia & ALL & 1991 & 220.40 & 212.85 & 228.28 & IHME \\ 
  Liberia & ALL & 1991 & 258.08 & 215.27 & 307.05 & RW2 \\ 
  Liberia & ALL & 1991 & 256.70 & 235.20 & 280.20 & UN \\ 
  Liberia & ALL & 1992 & 218.11 & 210.67 & 225.54 & IHME \\ 
  Liberia & ALL & 1992 & 257.54 & 215.45 & 304.91 & RW2 \\ 
  Liberia & ALL & 1992 & 255.40 & 234.00 & 278.90 & UN \\ 
  Liberia & ALL & 1993 & 218.39 & 210.68 & 227.29 & IHME \\ 
  Liberia & ALL & 1993 & 253.87 & 211.53 & 302.59 & RW2 \\ 
  Liberia & ALL & 1993 & 251.60 & 231.00 & 274.50 & UN \\ 
  Liberia & ALL & 1994 & 217.16 & 207.91 & 228.23 & IHME \\ 
  Liberia & ALL & 1994 & 247.07 & 203.72 & 299.10 & RW2 \\ 
  Liberia & ALL & 1994 & 245.50 & 225.60 & 267.40 & UN \\ 
  Liberia & ALL & 1995 & 210.56 & 203.09 & 218.84 & IHME \\ 
  Liberia & ALL & 1995 & 236.75 & 193.44 & 284.13 & RW2 \\ 
  Liberia & ALL & 1995 & 237.40 & 218.30 & 258.20 & UN \\ 
  Liberia & ALL & 1996 & 207.53 & 199.62 & 215.90 & IHME \\ 
  Liberia & ALL & 1996 & 226.20 & 186.54 & 270.87 & RW2 \\ 
  Liberia & ALL & 1996 & 227.70 & 210.10 & 247.40 & UN \\ 
  Liberia & ALL & 1997 & 197.05 & 190.82 & 203.75 & IHME \\ 
  Liberia & ALL & 1997 & 215.00 & 178.21 & 256.65 & RW2 \\ 
  Liberia & ALL & 1997 & 217.10 & 200.50 & 235.50 & UN \\ 
  Liberia & ALL & 1998 & 188.78 & 182.66 & 195.16 & IHME \\ 
  Liberia & ALL & 1998 & 203.74 & 167.52 & 246.09 & RW2 \\ 
  Liberia & ALL & 1998 & 205.80 & 190.30 & 223.20 & UN \\ 
  Liberia & ALL & 1999 & 178.78 & 173.22 & 184.66 & IHME \\ 
  Liberia & ALL & 1999 & 192.40 & 155.88 & 234.00 & RW2 \\ 
  Liberia & ALL & 1999 & 193.90 & 179.30 & 210.40 & UN \\ 
  Liberia & ALL & 2000 & 167.18 & 161.91 & 172.69 & IHME \\ 
  Liberia & ALL & 2000 & 181.64 & 147.55 & 222.13 & RW2 \\ 
  Liberia & ALL & 2000 & 181.80 & 167.80 & 197.50 & UN \\ 
  Liberia & ALL & 2001 & 156.16 & 150.92 & 161.52 & IHME \\ 
  Liberia & ALL & 2001 & 170.12 & 138.98 & 206.94 & RW2 \\ 
  Liberia & ALL & 2001 & 169.70 & 156.30 & 184.20 & UN \\ 
  Liberia & ALL & 2002 & 146.90 & 141.40 & 152.93 & IHME \\ 
  Liberia & ALL & 2002 & 158.45 & 130.02 & 192.16 & RW2 \\ 
  Liberia & ALL & 2002 & 157.60 & 145.10 & 171.10 & UN \\ 
  Liberia & ALL & 2003 & 138.97 & 133.17 & 145.12 & IHME \\ 
  Liberia & ALL & 2003 & 146.93 & 119.89 & 179.45 & RW2 \\ 
  Liberia & ALL & 2003 & 145.90 & 134.00 & 158.40 & UN \\ 
  Liberia & ALL & 2004 & 127.87 & 122.97 & 132.59 & IHME \\ 
  Liberia & ALL & 2004 & 135.53 & 108.53 & 168.47 & RW2 \\ 
  Liberia & ALL & 2004 & 134.80 & 123.50 & 146.70 & UN \\ 
  Liberia & ALL & 2005 & 119.91 & 115.13 & 124.58 & IHME \\ 
  Liberia & ALL & 2005 & 124.54 & 98.08 & 155.48 & RW2 \\ 
  Liberia & ALL & 2005 & 124.70 & 113.80 & 136.20 & UN \\ 
  Liberia & ALL & 2006 & 112.75 & 107.96 & 117.34 & IHME \\ 
  Liberia & ALL & 2006 & 115.08 & 91.21 & 143.04 & RW2 \\ 
  Liberia & ALL & 2006 & 115.70 & 105.30 & 126.90 & UN \\ 
  Liberia & ALL & 2007 & 105.92 & 101.08 & 110.54 & IHME \\ 
  Liberia & ALL & 2007 & 106.92 & 85.17 & 132.75 & RW2 \\ 
  Liberia & ALL & 2007 & 107.80 & 97.60 & 118.70 & UN \\ 
  Liberia & ALL & 2008 & 100.20 & 95.27 & 104.92 & IHME \\ 
  Liberia & ALL & 2008 & 100.15 & 78.99 & 125.62 & RW2 \\ 
  Liberia & ALL & 2008 & 100.90 & 91.00 & 111.60 & UN \\ 
  Liberia & ALL & 2009 & 95.53 & 90.28 & 100.82 & IHME \\ 
  Liberia & ALL & 2009 & 94.24 & 73.06 & 120.28 & RW2 \\ 
  Liberia & ALL & 2009 & 94.70 & 84.80 & 105.70 & UN \\ 
  Liberia & ALL & 2010 & 91.28 & 85.86 & 97.11 & IHME \\ 
  Liberia & ALL & 2010 & 89.64 & 69.16 & 116.76 & RW2 \\ 
  Liberia & ALL & 2010 & 89.30 & 78.90 & 101.20 & UN \\ 
  Liberia & ALL & 2011 & 86.90 & 81.23 & 93.16 & IHME \\ 
  Liberia & ALL & 2011 & 85.32 & 66.53 & 109.41 & RW2 \\ 
  Liberia & ALL & 2011 & 84.50 & 73.30 & 97.70 & UN \\ 
  Liberia & ALL & 2012 & 82.85 & 76.96 & 89.73 & IHME \\ 
  Liberia & ALL & 2012 & 81.35 & 64.03 & 103.05 & RW2 \\ 
  Liberia & ALL & 2012 & 80.30 & 67.90 & 95.20 & UN \\ 
  Liberia & ALL & 2013 & 78.45 & 72.25 & 85.93 & IHME \\ 
  Liberia & ALL & 2013 & 77.72 & 59.20 & 101.25 & RW2 \\ 
  Liberia & ALL & 2013 & 76.30 & 62.90 & 93.50 & UN \\ 
  Liberia & ALL & 2014 & 75.72 & 69.00 & 83.58 & IHME \\ 
  Liberia & ALL & 2014 & 74.12 & 50.88 & 107.18 & RW2 \\ 
  Liberia & ALL & 2014 & 72.90 & 58.20 & 92.40 & UN \\ 
  Liberia & ALL & 2015 & 71.39 & 64.64 & 79.71 & IHME \\ 
  Liberia & ALL & 2015 & 70.47 & 40.17 & 121.88 & RW2 \\ 
  Liberia & ALL & 2015 & 69.90 & 53.90 & 91.80 & UN \\ 
  Liberia & ALL & 2016 & 67.33 & 31.75 & 140.37 & RW2 \\ 
  Liberia & ALL & 2017 & 63.99 & 24.12 & 164.51 & RW2 \\ 
  Liberia & ALL & 2018 & 60.94 & 18.11 & 196.87 & RW2 \\ 
  Liberia & ALL & 2019 & 57.93 & 12.76 & 232.86 & RW2 \\ 
  Liberia & ALL & 80-84 & 227.15 & 257.62 & 199.32 & HT-Direct \\ 
  Liberia & ALL & 85-89 & 231.40 & 251.73 & 212.25 & HT-Direct \\ 
  Liberia & ALL & 90-94 & 264.15 & 282.64 & 246.45 & HT-Direct \\ 
  Liberia & ALL & 95-99 & 206.41 & 220.49 & 193.00 & HT-Direct \\ 
  Liberia & ALL & 00-04 & 152.33 & 163.17 & 142.09 & HT-Direct \\ 
  Liberia & ALL & 05-09 & 100.25 & 111.64 & 89.90 & HT-Direct \\ 
  Liberia & ALL & 10-14 & 92.88 & 107.44 & 80.12 & HT-Direct \\ 
  Liberia & ALL & 15-19 & 63.96 & 24.56 & 161.81 & RW2 \\ 
  Liberia & NORTH CENTRAL & 1980 & 229.99 & 162.59 & 314.89 & RW2 \\ 
  Liberia & NORTH CENTRAL & 1981 & 230.09 & 176.00 & 294.23 & RW2 \\ 
  Liberia & NORTH CENTRAL & 1982 & 230.03 & 182.12 & 286.36 & RW2 \\ 
  Liberia & NORTH CENTRAL & 1983 & 229.71 & 183.51 & 284.90 & RW2 \\ 
  Liberia & NORTH CENTRAL & 1984 & 230.48 & 184.29 & 284.82 & RW2 \\ 
  Liberia & NORTH CENTRAL & 1985 & 230.90 & 187.30 & 280.64 & RW2 \\ 
  Liberia & NORTH CENTRAL & 1986 & 233.40 & 192.87 & 279.09 & RW2 \\ 
  Liberia & NORTH CENTRAL & 1987 & 237.38 & 198.12 & 280.89 & RW2 \\ 
  Liberia & NORTH CENTRAL & 1988 & 242.25 & 202.07 & 285.35 & RW2 \\ 
  Liberia & NORTH CENTRAL & 1989 & 247.66 & 205.80 & 292.25 & RW2 \\ 
  Liberia & NORTH CENTRAL & 1990 & 253.89 & 213.86 & 299.60 & RW2 \\ 
  Liberia & NORTH CENTRAL & 1991 & 256.41 & 219.03 & 298.77 & RW2 \\ 
  Liberia & NORTH CENTRAL & 1992 & 255.93 & 219.66 & 296.05 & RW2 \\ 
  Liberia & NORTH CENTRAL & 1993 & 251.56 & 214.49 & 293.21 & RW2 \\ 
  Liberia & NORTH CENTRAL & 1994 & 243.79 & 205.98 & 287.29 & RW2 \\ 
  Liberia & NORTH CENTRAL & 1995 & 231.83 & 194.86 & 272.10 & RW2 \\ 
  Liberia & NORTH CENTRAL & 1996 & 219.47 & 185.87 & 255.91 & RW2 \\ 
  Liberia & NORTH CENTRAL & 1997 & 206.16 & 175.66 & 240.47 & RW2 \\ 
  Liberia & NORTH CENTRAL & 1998 & 192.80 & 163.07 & 226.92 & RW2 \\ 
  Liberia & NORTH CENTRAL & 1999 & 179.54 & 149.47 & 214.35 & RW2 \\ 
  Liberia & NORTH CENTRAL & 2000 & 167.24 & 139.03 & 199.31 & RW2 \\ 
  Liberia & NORTH CENTRAL & 2001 & 155.52 & 130.58 & 184.17 & RW2 \\ 
  Liberia & NORTH CENTRAL & 2002 & 144.74 & 122.10 & 171.00 & RW2 \\ 
  Liberia & NORTH CENTRAL & 2003 & 134.64 & 112.18 & 160.23 & RW2 \\ 
  Liberia & NORTH CENTRAL & 2004 & 125.38 & 102.83 & 151.83 & RW2 \\ 
  Liberia & NORTH CENTRAL & 2005 & 116.82 & 95.51 & 142.52 & RW2 \\ 
  Liberia & NORTH CENTRAL & 2006 & 108.79 & 89.26 & 131.88 & RW2 \\ 
  Liberia & NORTH CENTRAL & 2007 & 101.28 & 83.24 & 122.51 & RW2 \\ 
  Liberia & NORTH CENTRAL & 2008 & 94.15 & 76.41 & 115.70 & RW2 \\ 
  Liberia & NORTH CENTRAL & 2009 & 87.54 & 69.29 & 109.81 & RW2 \\ 
  Liberia & NORTH CENTRAL & 2010 & 81.27 & 62.79 & 104.31 & RW2 \\ 
  Liberia & NORTH CENTRAL & 2011 & 75.49 & 58.55 & 96.80 & RW2 \\ 
  Liberia & NORTH CENTRAL & 2012 & 70.06 & 54.18 & 89.71 & RW2 \\ 
  Liberia & NORTH CENTRAL & 2013 & 64.97 & 48.41 & 86.18 & RW2 \\ 
  Liberia & NORTH CENTRAL & 2014 & 60.21 & 40.93 & 87.22 & RW2 \\ 
  Liberia & NORTH CENTRAL & 2015 & 55.83 & 32.22 & 94.22 & RW2 \\ 
  Liberia & NORTH CENTRAL & 2016 & 51.68 & 24.99 & 102.98 & RW2 \\ 
  Liberia & NORTH CENTRAL & 2017 & 47.79 & 18.95 & 114.70 & RW2 \\ 
  Liberia & NORTH CENTRAL & 2018 & 44.26 & 14.02 & 130.98 & RW2 \\ 
  Liberia & NORTH CENTRAL & 2019 & 40.97 & 10.03 & 151.35 & RW2 \\ 
  Liberia & NORTH WESTERN & 1980 & 257.37 & 173.68 & 360.06 & RW2 \\ 
  Liberia & NORTH WESTERN & 1981 & 260.35 & 189.34 & 342.45 & RW2 \\ 
  Liberia & NORTH WESTERN & 1982 & 262.67 & 199.45 & 334.08 & RW2 \\ 
  Liberia & NORTH WESTERN & 1983 & 265.32 & 205.32 & 333.97 & RW2 \\ 
  Liberia & NORTH WESTERN & 1984 & 268.46 & 209.66 & 335.82 & RW2 \\ 
  Liberia & NORTH WESTERN & 1985 & 271.69 & 217.51 & 332.39 & RW2 \\ 
  Liberia & NORTH WESTERN & 1986 & 277.17 & 227.02 & 332.23 & RW2 \\ 
  Liberia & NORTH WESTERN & 1987 & 284.02 & 236.45 & 336.23 & RW2 \\ 
  Liberia & NORTH WESTERN & 1988 & 292.10 & 243.65 & 344.24 & RW2 \\ 
  Liberia & NORTH WESTERN & 1989 & 300.47 & 250.36 & 353.49 & RW2 \\ 
  Liberia & NORTH WESTERN & 1990 & 309.81 & 261.72 & 364.35 & RW2 \\ 
  Liberia & NORTH WESTERN & 1991 & 314.93 & 269.29 & 365.78 & RW2 \\ 
  Liberia & NORTH WESTERN & 1992 & 316.01 & 271.31 & 364.88 & RW2 \\ 
  Liberia & NORTH WESTERN & 1993 & 312.49 & 267.65 & 362.24 & RW2 \\ 
  Liberia & NORTH WESTERN & 1994 & 305.04 & 258.79 & 356.62 & RW2 \\ 
  Liberia & NORTH WESTERN & 1995 & 292.33 & 246.61 & 341.42 & RW2 \\ 
  Liberia & NORTH WESTERN & 1996 & 278.73 & 237.03 & 323.87 & RW2 \\ 
  Liberia & NORTH WESTERN & 1997 & 263.60 & 225.31 & 306.61 & RW2 \\ 
  Liberia & NORTH WESTERN & 1998 & 248.03 & 211.13 & 290.31 & RW2 \\ 
  Liberia & NORTH WESTERN & 1999 & 232.59 & 194.55 & 275.26 & RW2 \\ 
  Liberia & NORTH WESTERN & 2000 & 217.54 & 181.38 & 258.34 & RW2 \\ 
  Liberia & NORTH WESTERN & 2001 & 203.37 & 170.72 & 240.21 & RW2 \\ 
  Liberia & NORTH WESTERN & 2002 & 190.00 & 160.04 & 224.10 & RW2 \\ 
  Liberia & NORTH WESTERN & 2003 & 177.46 & 148.25 & 211.26 & RW2 \\ 
  Liberia & NORTH WESTERN & 2004 & 165.93 & 136.32 & 200.04 & RW2 \\ 
  Liberia & NORTH WESTERN & 2005 & 155.33 & 127.16 & 188.82 & RW2 \\ 
  Liberia & NORTH WESTERN & 2006 & 145.33 & 119.76 & 175.56 & RW2 \\ 
  Liberia & NORTH WESTERN & 2007 & 135.86 & 111.71 & 163.80 & RW2 \\ 
  Liberia & NORTH WESTERN & 2008 & 127.04 & 103.23 & 155.07 & RW2 \\ 
  Liberia & NORTH WESTERN & 2009 & 118.56 & 94.67 & 147.33 & RW2 \\ 
  Liberia & NORTH WESTERN & 2010 & 110.73 & 87.33 & 139.65 & RW2 \\ 
  Liberia & NORTH WESTERN & 2011 & 103.28 & 82.08 & 129.32 & RW2 \\ 
  Liberia & NORTH WESTERN & 2012 & 96.34 & 77.16 & 119.66 & RW2 \\ 
  Liberia & NORTH WESTERN & 2013 & 89.85 & 69.84 & 115.12 & RW2 \\ 
  Liberia & NORTH WESTERN & 2014 & 83.75 & 59.29 & 116.70 & RW2 \\ 
  Liberia & NORTH WESTERN & 2015 & 77.98 & 46.88 & 126.44 & RW2 \\ 
  Liberia & NORTH WESTERN & 2016 & 72.48 & 36.46 & 138.24 & RW2 \\ 
  Liberia & NORTH WESTERN & 2017 & 67.54 & 27.67 & 155.06 & RW2 \\ 
  Liberia & NORTH WESTERN & 2018 & 62.64 & 20.35 & 176.92 & RW2 \\ 
  Liberia & NORTH WESTERN & 2019 & 58.28 & 14.70 & 205.52 & RW2 \\ 
  Liberia & SOUTH CENTRAL & 1980 & 245.97 & 180.09 & 328.49 & RW2 \\ 
  Liberia & SOUTH CENTRAL & 1981 & 245.70 & 194.54 & 305.60 & RW2 \\ 
  Liberia & SOUTH CENTRAL & 1982 & 245.44 & 200.79 & 296.18 & RW2 \\ 
  Liberia & SOUTH CENTRAL & 1983 & 245.11 & 201.03 & 296.77 & RW2 \\ 
  Liberia & SOUTH CENTRAL & 1984 & 245.62 & 200.64 & 297.79 & RW2 \\ 
  Liberia & SOUTH CENTRAL & 1985 & 245.97 & 202.59 & 293.17 & RW2 \\ 
  Liberia & SOUTH CENTRAL & 1986 & 248.27 & 208.56 & 292.59 & RW2 \\ 
  Liberia & SOUTH CENTRAL & 1987 & 252.21 & 214.06 & 293.85 & RW2 \\ 
  Liberia & SOUTH CENTRAL & 1988 & 257.20 & 217.41 & 299.62 & RW2 \\ 
  Liberia & SOUTH CENTRAL & 1989 & 262.44 & 220.15 & 307.28 & RW2 \\ 
  Liberia & SOUTH CENTRAL & 1990 & 268.76 & 228.51 & 315.38 & RW2 \\ 
  Liberia & SOUTH CENTRAL & 1991 & 271.39 & 233.13 & 313.50 & RW2 \\ 
  Liberia & SOUTH CENTRAL & 1992 & 270.37 & 233.54 & 310.69 & RW2 \\ 
  Liberia & SOUTH CENTRAL & 1993 & 265.68 & 228.85 & 307.43 & RW2 \\ 
  Liberia & SOUTH CENTRAL & 1994 & 257.48 & 219.06 & 301.63 & RW2 \\ 
  Liberia & SOUTH CENTRAL & 1995 & 244.94 & 206.51 & 286.44 & RW2 \\ 
  Liberia & SOUTH CENTRAL & 1996 & 231.92 & 197.60 & 268.63 & RW2 \\ 
  Liberia & SOUTH CENTRAL & 1997 & 218.08 & 186.85 & 253.25 & RW2 \\ 
  Liberia & SOUTH CENTRAL & 1998 & 204.34 & 173.86 & 239.18 & RW2 \\ 
  Liberia & SOUTH CENTRAL & 1999 & 190.80 & 160.16 & 225.96 & RW2 \\ 
  Liberia & SOUTH CENTRAL & 2000 & 178.04 & 149.19 & 211.52 & RW2 \\ 
  Liberia & SOUTH CENTRAL & 2001 & 166.13 & 140.27 & 195.94 & RW2 \\ 
  Liberia & SOUTH CENTRAL & 2002 & 154.97 & 131.36 & 182.30 & RW2 \\ 
  Liberia & SOUTH CENTRAL & 2003 & 144.64 & 121.19 & 171.64 & RW2 \\ 
  Liberia & SOUTH CENTRAL & 2004 & 135.13 & 111.10 & 162.59 & RW2 \\ 
  Liberia & SOUTH CENTRAL & 2005 & 126.42 & 103.28 & 154.01 & RW2 \\ 
  Liberia & SOUTH CENTRAL & 2006 & 118.12 & 97.18 & 143.41 & RW2 \\ 
  Liberia & SOUTH CENTRAL & 2007 & 110.50 & 90.82 & 134.18 & RW2 \\ 
  Liberia & SOUTH CENTRAL & 2008 & 103.32 & 83.56 & 127.20 & RW2 \\ 
  Liberia & SOUTH CENTRAL & 2009 & 96.44 & 76.36 & 121.08 & RW2 \\ 
  Liberia & SOUTH CENTRAL & 2010 & 90.10 & 70.02 & 115.21 & RW2 \\ 
  Liberia & SOUTH CENTRAL & 2011 & 84.10 & 65.77 & 107.13 & RW2 \\ 
  Liberia & SOUTH CENTRAL & 2012 & 78.47 & 61.50 & 99.70 & RW2 \\ 
  Liberia & SOUTH CENTRAL & 2013 & 73.20 & 55.59 & 95.86 & RW2 \\ 
  Liberia & SOUTH CENTRAL & 2014 & 68.17 & 47.14 & 97.23 & RW2 \\ 
  Liberia & SOUTH CENTRAL & 2015 & 63.61 & 37.48 & 105.66 & RW2 \\ 
  Liberia & SOUTH CENTRAL & 2016 & 59.19 & 29.10 & 115.86 & RW2 \\ 
  Liberia & SOUTH CENTRAL & 2017 & 55.23 & 22.12 & 130.49 & RW2 \\ 
  Liberia & SOUTH CENTRAL & 2018 & 51.27 & 16.42 & 148.53 & RW2 \\ 
  Liberia & SOUTH CENTRAL & 2019 & 47.87 & 12.01 & 172.13 & RW2 \\ 
  Liberia & SOUTH EASTERN A & 1980 & 203.72 & 135.57 & 293.29 & RW2 \\ 
  Liberia & SOUTH EASTERN A & 1981 & 204.93 & 148.32 & 276.27 & RW2 \\ 
  Liberia & SOUTH EASTERN A & 1982 & 206.11 & 154.88 & 268.22 & RW2 \\ 
  Liberia & SOUTH EASTERN A & 1983 & 207.37 & 158.01 & 267.28 & RW2 \\ 
  Liberia & SOUTH EASTERN A & 1984 & 209.14 & 161.19 & 267.22 & RW2 \\ 
  Liberia & SOUTH EASTERN A & 1985 & 210.89 & 166.12 & 263.72 & RW2 \\ 
  Liberia & SOUTH EASTERN A & 1986 & 214.58 & 172.68 & 262.94 & RW2 \\ 
  Liberia & SOUTH EASTERN A & 1987 & 219.44 & 179.42 & 264.97 & RW2 \\ 
  Liberia & SOUTH EASTERN A & 1988 & 225.26 & 184.61 & 270.93 & RW2 \\ 
  Liberia & SOUTH EASTERN A & 1989 & 231.64 & 189.55 & 278.35 & RW2 \\ 
  Liberia & SOUTH EASTERN A & 1990 & 238.58 & 198.16 & 286.57 & RW2 \\ 
  Liberia & SOUTH EASTERN A & 1991 & 242.24 & 203.02 & 286.45 & RW2 \\ 
  Liberia & SOUTH EASTERN A & 1992 & 242.58 & 204.70 & 285.03 & RW2 \\ 
  Liberia & SOUTH EASTERN A & 1993 & 239.09 & 201.22 & 282.54 & RW2 \\ 
  Liberia & SOUTH EASTERN A & 1994 & 232.32 & 193.80 & 277.56 & RW2 \\ 
  Liberia & SOUTH EASTERN A & 1995 & 221.72 & 183.96 & 263.66 & RW2 \\ 
  Liberia & SOUTH EASTERN A & 1996 & 210.31 & 175.75 & 248.76 & RW2 \\ 
  Liberia & SOUTH EASTERN A & 1997 & 198.44 & 167.13 & 234.45 & RW2 \\ 
  Liberia & SOUTH EASTERN A & 1998 & 186.43 & 156.07 & 222.18 & RW2 \\ 
  Liberia & SOUTH EASTERN A & 1999 & 174.92 & 144.33 & 210.56 & RW2 \\ 
  Liberia & SOUTH EASTERN A & 2000 & 163.95 & 134.87 & 197.96 & RW2 \\ 
  Liberia & SOUTH EASTERN A & 2001 & 153.75 & 127.80 & 184.36 & RW2 \\ 
  Liberia & SOUTH EASTERN A & 2002 & 144.37 & 120.33 & 172.21 & RW2 \\ 
  Liberia & SOUTH EASTERN A & 2003 & 135.63 & 112.08 & 163.39 & RW2 \\ 
  Liberia & SOUTH EASTERN A & 2004 & 127.70 & 104.03 & 156.09 & RW2 \\ 
  Liberia & SOUTH EASTERN A & 2005 & 120.54 & 97.66 & 148.28 & RW2 \\ 
  Liberia & SOUTH EASTERN A & 2006 & 113.55 & 92.50 & 138.82 & RW2 \\ 
  Liberia & SOUTH EASTERN A & 2007 & 106.93 & 87.09 & 130.73 & RW2 \\ 
  Liberia & SOUTH EASTERN A & 2008 & 100.64 & 80.65 & 124.72 & RW2 \\ 
  Liberia & SOUTH EASTERN A & 2009 & 94.74 & 74.34 & 119.84 & RW2 \\ 
  Liberia & SOUTH EASTERN A & 2010 & 89.05 & 68.58 & 114.74 & RW2 \\ 
  Liberia & SOUTH EASTERN A & 2011 & 83.67 & 64.45 & 107.77 & RW2 \\ 
  Liberia & SOUTH EASTERN A & 2012 & 78.66 & 60.42 & 101.32 & RW2 \\ 
  Liberia & SOUTH EASTERN A & 2013 & 73.86 & 55.15 & 98.18 & RW2 \\ 
  Liberia & SOUTH EASTERN A & 2014 & 69.28 & 47.30 & 100.07 & RW2 \\ 
  Liberia & SOUTH EASTERN A & 2015 & 64.99 & 37.83 & 109.34 & RW2 \\ 
  Liberia & SOUTH EASTERN A & 2016 & 60.90 & 29.56 & 120.28 & RW2 \\ 
  Liberia & SOUTH EASTERN A & 2017 & 57.17 & 22.71 & 135.83 & RW2 \\ 
  Liberia & SOUTH EASTERN A & 2018 & 53.42 & 16.89 & 155.26 & RW2 \\ 
  Liberia & SOUTH EASTERN A & 2019 & 50.22 & 12.30 & 182.99 & RW2 \\ 
  Liberia & SOUTH EASTERN B & 1980 & 168.78 & 110.94 & 251.81 & RW2 \\ 
  Liberia & SOUTH EASTERN B & 1981 & 171.30 & 122.73 & 236.68 & RW2 \\ 
  Liberia & SOUTH EASTERN B & 1982 & 173.99 & 130.48 & 229.99 & RW2 \\ 
  Liberia & SOUTH EASTERN B & 1983 & 176.57 & 134.82 & 229.59 & RW2 \\ 
  Liberia & SOUTH EASTERN B & 1984 & 179.71 & 138.62 & 231.87 & RW2 \\ 
  Liberia & SOUTH EASTERN B & 1985 & 182.81 & 144.06 & 229.09 & RW2 \\ 
  Liberia & SOUTH EASTERN B & 1986 & 187.55 & 151.80 & 229.79 & RW2 \\ 
  Liberia & SOUTH EASTERN B & 1987 & 193.43 & 158.80 & 233.39 & RW2 \\ 
  Liberia & SOUTH EASTERN B & 1988 & 200.35 & 164.96 & 240.14 & RW2 \\ 
  Liberia & SOUTH EASTERN B & 1989 & 207.57 & 170.19 & 248.53 & RW2 \\ 
  Liberia & SOUTH EASTERN B & 1990 & 215.73 & 179.76 & 257.52 & RW2 \\ 
  Liberia & SOUTH EASTERN B & 1991 & 220.49 & 186.14 & 259.88 & RW2 \\ 
  Liberia & SOUTH EASTERN B & 1992 & 222.26 & 189.21 & 260.28 & RW2 \\ 
  Liberia & SOUTH EASTERN B & 1993 & 220.88 & 187.18 & 259.87 & RW2 \\ 
  Liberia & SOUTH EASTERN B & 1994 & 216.03 & 180.17 & 256.71 & RW2 \\ 
  Liberia & SOUTH EASTERN B & 1995 & 208.02 & 172.08 & 246.93 & RW2 \\ 
  Liberia & SOUTH EASTERN B & 1996 & 199.61 & 166.34 & 236.34 & RW2 \\ 
  Liberia & SOUTH EASTERN B & 1997 & 190.52 & 159.27 & 225.33 & RW2 \\ 
  Liberia & SOUTH EASTERN B & 1998 & 181.94 & 150.43 & 217.17 & RW2 \\ 
  Liberia & SOUTH EASTERN B & 1999 & 173.65 & 141.52 & 209.67 & RW2 \\ 
  Liberia & SOUTH EASTERN B & 2000 & 166.28 & 135.23 & 201.26 & RW2 \\ 
  Liberia & SOUTH EASTERN B & 2001 & 159.71 & 130.95 & 192.37 & RW2 \\ 
  Liberia & SOUTH EASTERN B & 2002 & 154.27 & 127.24 & 184.55 & RW2 \\ 
  Liberia & SOUTH EASTERN B & 2003 & 149.31 & 122.66 & 179.02 & RW2 \\ 
  Liberia & SOUTH EASTERN B & 2004 & 145.33 & 118.70 & 175.67 & RW2 \\ 
  Liberia & SOUTH EASTERN B & 2005 & 142.04 & 116.99 & 171.54 & RW2 \\ 
  Liberia & SOUTH EASTERN B & 2006 & 138.92 & 116.07 & 165.44 & RW2 \\ 
  Liberia & SOUTH EASTERN B & 2007 & 136.13 & 114.54 & 160.62 & RW2 \\ 
  Liberia & SOUTH EASTERN B & 2008 & 133.70 & 111.34 & 159.34 & RW2 \\ 
  Liberia & SOUTH EASTERN B & 2009 & 131.25 & 106.77 & 160.26 & RW2 \\ 
  Liberia & SOUTH EASTERN B & 2010 & 128.91 & 103.06 & 160.72 & RW2 \\ 
  Liberia & SOUTH EASTERN B & 2011 & 126.47 & 100.99 & 157.78 & RW2 \\ 
  Liberia & SOUTH EASTERN B & 2012 & 124.23 & 99.10 & 155.42 & RW2 \\ 
  Liberia & SOUTH EASTERN B & 2013 & 122.04 & 93.79 & 158.57 & RW2 \\ 
  Liberia & SOUTH EASTERN B & 2014 & 119.94 & 83.97 & 169.54 & RW2 \\ 
  Liberia & SOUTH EASTERN B & 2015 & 117.74 & 70.59 & 192.00 & RW2 \\ 
  Liberia & SOUTH EASTERN B & 2016 & 115.68 & 57.88 & 217.64 & RW2 \\ 
  Liberia & SOUTH EASTERN B & 2017 & 113.52 & 46.52 & 252.01 & RW2 \\ 
  Liberia & SOUTH EASTERN B & 2018 & 111.47 & 36.75 & 295.65 & RW2 \\ 
  Liberia & SOUTH EASTERN B & 2019 & 109.59 & 27.67 & 344.70 & RW2 \\ 
  Madagascar & ALL & 1980 & 160.07 & 151.09 & 170.34 & IHME \\ 
  Madagascar & ALL & 1980 & 178.53 & 130.95 & 238.60 & RW2 \\ 
  Madagascar & ALL & 1980 & 174.60 & 162.00 & 188.70 & UN \\ 
  Madagascar & ALL & 1981 & 163.44 & 154.09 & 173.64 & IHME \\ 
  Madagascar & ALL & 1981 & 179.33 & 144.76 & 220.31 & RW2 \\ 
  Madagascar & ALL & 1981 & 177.60 & 165.00 & 191.50 & UN \\ 
  Madagascar & ALL & 1982 & 166.85 & 157.19 & 176.99 & IHME \\ 
  Madagascar & ALL & 1982 & 180.23 & 147.99 & 217.40 & RW2 \\ 
  Madagascar & ALL & 1982 & 180.30 & 167.80 & 194.20 & UN \\ 
  Madagascar & ALL & 1983 & 168.11 & 158.71 & 177.25 & IHME \\ 
  Madagascar & ALL & 1983 & 180.67 & 144.52 & 222.31 & RW2 \\ 
  Madagascar & ALL & 1983 & 182.40 & 169.80 & 196.40 & UN \\ 
  Madagascar & ALL & 1984 & 168.48 & 159.72 & 177.99 & IHME \\ 
  Madagascar & ALL & 1984 & 181.03 & 141.04 & 226.96 & RW2 \\ 
  Madagascar & ALL & 1984 & 183.10 & 170.50 & 197.00 & UN \\ 
  Madagascar & ALL & 1985 & 168.91 & 159.76 & 178.34 & IHME \\ 
  Madagascar & ALL & 1985 & 180.89 & 145.05 & 223.78 & RW2 \\ 
  Madagascar & ALL & 1985 & 182.40 & 170.00 & 196.20 & UN \\ 
  Madagascar & ALL & 1986 & 168.34 & 158.97 & 177.94 & IHME \\ 
  Madagascar & ALL & 1986 & 179.38 & 146.29 & 217.82 & RW2 \\ 
  Madagascar & ALL & 1986 & 180.10 & 168.00 & 193.60 & UN \\ 
  Madagascar & ALL & 1987 & 164.68 & 155.80 & 174.59 & IHME \\ 
  Madagascar & ALL & 1987 & 176.74 & 145.77 & 213.21 & RW2 \\ 
  Madagascar & ALL & 1987 & 176.50 & 164.80 & 189.40 & UN \\ 
  Madagascar & ALL & 1988 & 157.40 & 148.97 & 166.62 & IHME \\ 
  Madagascar & ALL & 1988 & 172.80 & 140.97 & 210.55 & RW2 \\ 
  Madagascar & ALL & 1988 & 171.70 & 160.40 & 184.10 & UN \\ 
  Madagascar & ALL & 1989 & 149.03 & 140.58 & 157.60 & IHME \\ 
  Madagascar & ALL & 1989 & 167.98 & 135.32 & 207.91 & RW2 \\ 
  Madagascar & ALL & 1989 & 166.20 & 155.30 & 177.90 & UN \\ 
  Madagascar & ALL & 1990 & 143.12 & 135.08 & 151.04 & IHME \\ 
  Madagascar & ALL & 1990 & 161.96 & 129.57 & 199.74 & RW2 \\ 
  Madagascar & ALL & 1990 & 160.70 & 150.00 & 172.10 & UN \\ 
  Madagascar & ALL & 1991 & 140.43 & 133.04 & 147.86 & IHME \\ 
  Madagascar & ALL & 1991 & 156.55 & 127.00 & 190.48 & RW2 \\ 
  Madagascar & ALL & 1991 & 155.80 & 145.40 & 166.80 & UN \\ 
  Madagascar & ALL & 1992 & 139.35 & 131.67 & 147.58 & IHME \\ 
  Madagascar & ALL & 1992 & 151.34 & 123.52 & 183.62 & RW2 \\ 
  Madagascar & ALL & 1992 & 151.30 & 141.40 & 161.90 & UN \\ 
  Madagascar & ALL & 1993 & 138.32 & 130.24 & 146.87 & IHME \\ 
  Madagascar & ALL & 1993 & 146.52 & 118.75 & 179.22 & RW2 \\ 
  Madagascar & ALL & 1993 & 147.10 & 137.60 & 157.50 & UN \\ 
  Madagascar & ALL & 1994 & 135.94 & 128.17 & 143.82 & IHME \\ 
  Madagascar & ALL & 1994 & 141.88 & 113.08 & 176.59 & RW2 \\ 
  Madagascar & ALL & 1994 & 142.90 & 133.60 & 153.40 & UN \\ 
  Madagascar & ALL & 1995 & 131.40 & 123.08 & 139.22 & IHME \\ 
  Madagascar & ALL & 1995 & 137.93 & 110.44 & 172.04 & RW2 \\ 
  Madagascar & ALL & 1995 & 138.40 & 129.20 & 149.00 & UN \\ 
  Madagascar & ALL & 1996 & 125.57 & 117.93 & 133.17 & IHME \\ 
  Madagascar & ALL & 1996 & 133.07 & 107.64 & 164.60 & RW2 \\ 
  Madagascar & ALL & 1996 & 133.40 & 124.30 & 143.90 & UN \\ 
  Madagascar & ALL & 1997 & 119.78 & 112.01 & 127.10 & IHME \\ 
  Madagascar & ALL & 1997 & 127.83 & 103.92 & 156.53 & RW2 \\ 
  Madagascar & ALL & 1997 & 127.70 & 118.60 & 138.10 & UN \\ 
  Madagascar & ALL & 1998 & 114.59 & 107.02 & 122.33 & IHME \\ 
  Madagascar & ALL & 1998 & 122.15 & 98.40 & 151.47 & RW2 \\ 
  Madagascar & ALL & 1998 & 121.70 & 112.60 & 132.10 & UN \\ 
  Madagascar & ALL & 1999 & 110.50 & 103.16 & 118.96 & IHME \\ 
  Madagascar & ALL & 1999 & 116.03 & 91.93 & 145.23 & RW2 \\ 
  Madagascar & ALL & 1999 & 115.50 & 106.20 & 125.70 & UN \\ 
  Madagascar & ALL & 2000 & 107.08 & 99.82 & 115.23 & IHME \\ 
  Madagascar & ALL & 2000 & 109.41 & 86.21 & 137.11 & RW2 \\ 
  Madagascar & ALL & 2000 & 109.20 & 99.90 & 119.40 & UN \\ 
  Madagascar & ALL & 2001 & 103.41 & 96.18 & 111.31 & IHME \\ 
  Madagascar & ALL & 2001 & 103.13 & 82.10 & 128.26 & RW2 \\ 
  Madagascar & ALL & 2001 & 103.10 & 93.90 & 112.90 & UN \\ 
  Madagascar & ALL & 2002 & 99.34 & 91.95 & 107.49 & IHME \\ 
  Madagascar & ALL & 2002 & 97.01 & 77.98 & 120.05 & RW2 \\ 
  Madagascar & ALL & 2002 & 97.20 & 88.10 & 106.90 & UN \\ 
  Madagascar & ALL & 2003 & 94.18 & 86.60 & 102.21 & IHME \\ 
  Madagascar & ALL & 2003 & 91.27 & 72.84 & 113.87 & RW2 \\ 
  Madagascar & ALL & 2003 & 91.50 & 82.40 & 101.00 & UN \\ 
  Madagascar & ALL & 2004 & 88.63 & 80.84 & 96.94 & IHME \\ 
  Madagascar & ALL & 2004 & 85.70 & 66.62 & 109.38 & RW2 \\ 
  Madagascar & ALL & 2004 & 85.90 & 76.80 & 95.60 & UN \\ 
  Madagascar & ALL & 2005 & 83.56 & 75.85 & 91.52 & IHME \\ 
  Madagascar & ALL & 2005 & 80.71 & 61.46 & 105.43 & RW2 \\ 
  Madagascar & ALL & 2005 & 80.60 & 71.20 & 90.50 & UN \\ 
  Madagascar & ALL & 2006 & 80.08 & 72.74 & 87.61 & IHME \\ 
  Madagascar & ALL & 2006 & 75.84 & 58.66 & 97.50 & RW2 \\ 
  Madagascar & ALL & 2006 & 75.40 & 65.60 & 86.20 & UN \\ 
  Madagascar & ALL & 2007 & 78.13 & 70.67 & 85.92 & IHME \\ 
  Madagascar & ALL & 2007 & 71.30 & 56.37 & 89.84 & RW2 \\ 
  Madagascar & ALL & 2007 & 70.80 & 60.10 & 82.80 & UN \\ 
  Madagascar & ALL & 2008 & 77.01 & 69.18 & 85.40 & IHME \\ 
  Madagascar & ALL & 2008 & 67.12 & 51.72 & 86.73 & RW2 \\ 
  Madagascar & ALL & 2008 & 66.70 & 55.30 & 79.90 & UN \\ 
  Madagascar & ALL & 2009 & 76.61 & 67.13 & 86.62 & IHME \\ 
  Madagascar & ALL & 2009 & 62.96 & 43.60 & 90.63 & RW2 \\ 
  Madagascar & ALL & 2009 & 63.30 & 50.80 & 78.00 & UN \\ 
  Madagascar & ALL & 2010 & 75.84 & 64.58 & 87.00 & IHME \\ 
  Madagascar & ALL & 2010 & 59.07 & 33.60 & 103.17 & RW2 \\ 
  Madagascar & ALL & 2010 & 60.30 & 46.70 & 76.60 & UN \\ 
  Madagascar & ALL & 2011 & 74.39 & 62.34 & 86.91 & IHME \\ 
  Madagascar & ALL & 2011 & 55.50 & 25.36 & 117.72 & RW2 \\ 
  Madagascar & ALL & 2011 & 57.80 & 43.20 & 75.80 & UN \\ 
  Madagascar & ALL & 2012 & 72.24 & 60.15 & 86.02 & IHME \\ 
  Madagascar & ALL & 2012 & 52.02 & 18.49 & 138.29 & RW2 \\ 
  Madagascar & ALL & 2012 & 55.50 & 40.00 & 75.40 & UN \\ 
  Madagascar & ALL & 2013 & 69.66 & 56.94 & 84.42 & IHME \\ 
  Madagascar & ALL & 2013 & 48.95 & 13.33 & 164.05 & RW2 \\ 
  Madagascar & ALL & 2013 & 53.40 & 37.10 & 75.40 & UN \\ 
  Madagascar & ALL & 2014 & 66.92 & 54.66 & 83.01 & IHME \\ 
  Madagascar & ALL & 2014 & 45.86 & 9.34 & 197.67 & RW2 \\ 
  Madagascar & ALL & 2014 & 51.50 & 34.40 & 74.90 & UN \\ 
  Madagascar & ALL & 2015 & 64.20 & 51.93 & 80.64 & IHME \\ 
  Madagascar & ALL & 2015 & 42.83 & 6.52 & 236.65 & RW2 \\ 
  Madagascar & ALL & 2015 & 49.60 & 32.00 & 74.90 & UN \\ 
  Madagascar & ALL & 2016 & 40.46 & 4.49 & 291.19 & RW2 \\ 
  Madagascar & ALL & 2017 & 37.70 & 2.95 & 352.75 & RW2 \\ 
  Madagascar & ALL & 2018 & 35.32 & 1.97 & 428.43 & RW2 \\ 
  Madagascar & ALL & 2019 & 32.99 & 1.17 & 495.57 & RW2 \\ 
  Madagascar & ALL & 80-84 & 187.79 & 198.41 & 177.60 & HT-Direct \\ 
  Madagascar & ALL & 85-89 & 178.75 & 187.84 & 170.01 & HT-Direct \\ 
  Madagascar & ALL & 90-94 & 146.51 & 154.46 & 138.90 & HT-Direct \\ 
  Madagascar & ALL & 95-99 & 122.89 & 130.62 & 115.55 & HT-Direct \\ 
  Madagascar & ALL & 00-04 & 90.86 & 97.55 & 84.59 & HT-Direct \\ 
  Madagascar & ALL & 05-09 & 66.11 & 72.90 & 59.91 & HT-Direct \\ 
  Madagascar & ALL & 15-19 & 37.70 & 2.97 & 347.99 & RW2 \\ 
  Madagascar & ANTANANARIVO & 1980 & 136.31 & 100.04 & 182.18 & RW2 \\ 
  Madagascar & ANTANANARIVO & 1981 & 138.72 & 112.36 & 169.78 & RW2 \\ 
  Madagascar & ANTANANARIVO & 1982 & 140.90 & 117.37 & 167.96 & RW2 \\ 
  Madagascar & ANTANANARIVO & 1983 & 142.97 & 117.00 & 173.43 & RW2 \\ 
  Madagascar & ANTANANARIVO & 1984 & 144.62 & 116.09 & 178.24 & RW2 \\ 
  Madagascar & ANTANANARIVO & 1985 & 145.74 & 119.30 & 177.51 & RW2 \\ 
  Madagascar & ANTANANARIVO & 1986 & 145.73 & 121.34 & 174.40 & RW2 \\ 
  Madagascar & ANTANANARIVO & 1987 & 144.43 & 121.21 & 171.97 & RW2 \\ 
  Madagascar & ANTANANARIVO & 1988 & 141.84 & 117.74 & 170.43 & RW2 \\ 
  Madagascar & ANTANANARIVO & 1989 & 137.94 & 113.01 & 167.66 & RW2 \\ 
  Madagascar & ANTANANARIVO & 1990 & 132.99 & 108.85 & 161.82 & RW2 \\ 
  Madagascar & ANTANANARIVO & 1991 & 127.29 & 105.51 & 153.04 & RW2 \\ 
  Madagascar & ANTANANARIVO & 1992 & 121.00 & 100.61 & 144.83 & RW2 \\ 
  Madagascar & ANTANANARIVO & 1993 & 114.46 & 94.47 & 138.19 & RW2 \\ 
  Madagascar & ANTANANARIVO & 1994 & 108.20 & 87.77 & 132.53 & RW2 \\ 
  Madagascar & ANTANANARIVO & 1995 & 101.84 & 82.11 & 125.01 & RW2 \\ 
  Madagascar & ANTANANARIVO & 1996 & 96.45 & 78.42 & 117.40 & RW2 \\ 
  Madagascar & ANTANANARIVO & 1997 & 91.64 & 74.81 & 111.40 & RW2 \\ 
  Madagascar & ANTANANARIVO & 1998 & 87.49 & 70.86 & 107.04 & RW2 \\ 
  Madagascar & ANTANANARIVO & 1999 & 83.90 & 66.33 & 104.03 & RW2 \\ 
  Madagascar & ANTANANARIVO & 2000 & 81.01 & 64.05 & 101.49 & RW2 \\ 
  Madagascar & ANTANANARIVO & 2001 & 78.03 & 62.19 & 96.76 & RW2 \\ 
  Madagascar & ANTANANARIVO & 2002 & 75.08 & 60.25 & 92.59 & RW2 \\ 
  Madagascar & ANTANANARIVO & 2003 & 72.06 & 57.48 & 89.89 & RW2 \\ 
  Madagascar & ANTANANARIVO & 2004 & 69.06 & 53.99 & 87.97 & RW2 \\ 
  Madagascar & ANTANANARIVO & 2005 & 65.81 & 50.34 & 85.41 & RW2 \\ 
  Madagascar & ANTANANARIVO & 2006 & 62.90 & 48.21 & 81.81 & RW2 \\ 
  Madagascar & ANTANANARIVO & 2007 & 59.99 & 45.46 & 79.10 & RW2 \\ 
  Madagascar & ANTANANARIVO & 2008 & 57.25 & 41.37 & 79.83 & RW2 \\ 
  Madagascar & ANTANANARIVO & 2009 & 54.51 & 35.51 & 84.48 & RW2 \\ 
  Madagascar & ANTANANARIVO & 2010 & 52.13 & 28.66 & 95.54 & RW2 \\ 
  Madagascar & ANTANANARIVO & 2011 & 49.67 & 22.47 & 109.15 & RW2 \\ 
  Madagascar & ANTANANARIVO & 2012 & 47.40 & 17.19 & 126.91 & RW2 \\ 
  Madagascar & ANTANANARIVO & 2013 & 45.33 & 12.86 & 152.49 & RW2 \\ 
  Madagascar & ANTANANARIVO & 2014 & 43.37 & 9.35 & 180.65 & RW2 \\ 
  Madagascar & ANTANANARIVO & 2015 & 41.41 & 6.71 & 217.59 & RW2 \\ 
  Madagascar & ANTANANARIVO & 2016 & 39.35 & 4.71 & 262.02 & RW2 \\ 
  Madagascar & ANTANANARIVO & 2017 & 37.73 & 3.29 & 318.84 & RW2 \\ 
  Madagascar & ANTANANARIVO & 2018 & 35.77 & 2.22 & 385.78 & RW2 \\ 
  Madagascar & ANTANANARIVO & 2019 & 34.20 & 1.49 & 464.97 & RW2 \\ 
  Madagascar & ANTSIRANANA & 1980 & 144.76 & 102.64 & 201.98 & RW2 \\ 
  Madagascar & ANTSIRANANA & 1981 & 143.60 & 112.39 & 182.14 & RW2 \\ 
  Madagascar & ANTSIRANANA & 1982 & 142.47 & 115.26 & 174.29 & RW2 \\ 
  Madagascar & ANTSIRANANA & 1983 & 141.18 & 113.19 & 174.85 & RW2 \\ 
  Madagascar & ANTSIRANANA & 1984 & 140.16 & 110.91 & 175.06 & RW2 \\ 
  Madagascar & ANTSIRANANA & 1985 & 139.31 & 111.41 & 171.70 & RW2 \\ 
  Madagascar & ANTSIRANANA & 1986 & 138.44 & 112.93 & 168.46 & RW2 \\ 
  Madagascar & ANTSIRANANA & 1987 & 137.90 & 113.48 & 166.09 & RW2 \\ 
  Madagascar & ANTSIRANANA & 1988 & 137.32 & 112.06 & 166.77 & RW2 \\ 
  Madagascar & ANTSIRANANA & 1989 & 136.48 & 109.94 & 168.36 & RW2 \\ 
  Madagascar & ANTSIRANANA & 1990 & 135.32 & 108.81 & 167.34 & RW2 \\ 
  Madagascar & ANTSIRANANA & 1991 & 133.40 & 108.25 & 162.96 & RW2 \\ 
  Madagascar & ANTSIRANANA & 1992 & 130.24 & 106.29 & 158.91 & RW2 \\ 
  Madagascar & ANTSIRANANA & 1993 & 126.29 & 102.55 & 155.66 & RW2 \\ 
  Madagascar & ANTSIRANANA & 1994 & 121.81 & 97.23 & 152.30 & RW2 \\ 
  Madagascar & ANTSIRANANA & 1995 & 116.57 & 92.35 & 146.11 & RW2 \\ 
  Madagascar & ANTSIRANANA & 1996 & 112.09 & 89.54 & 138.93 & RW2 \\ 
  Madagascar & ANTSIRANANA & 1997 & 108.23 & 86.65 & 134.47 & RW2 \\ 
  Madagascar & ANTSIRANANA & 1998 & 105.11 & 83.21 & 131.64 & RW2 \\ 
  Madagascar & ANTSIRANANA & 1999 & 102.23 & 79.51 & 129.89 & RW2 \\ 
  Madagascar & ANTSIRANANA & 2000 & 99.95 & 77.45 & 128.93 & RW2 \\ 
  Madagascar & ANTSIRANANA & 2001 & 96.82 & 74.84 & 124.90 & RW2 \\ 
  Madagascar & ANTSIRANANA & 2002 & 92.84 & 71.28 & 120.64 & RW2 \\ 
  Madagascar & ANTSIRANANA & 2003 & 88.01 & 65.95 & 116.87 & RW2 \\ 
  Madagascar & ANTSIRANANA & 2004 & 82.51 & 59.56 & 113.01 & RW2 \\ 
  Madagascar & ANTSIRANANA & 2005 & 76.21 & 52.71 & 108.55 & RW2 \\ 
  Madagascar & ANTSIRANANA & 2006 & 70.10 & 47.23 & 103.19 & RW2 \\ 
  Madagascar & ANTSIRANANA & 2007 & 64.37 & 41.50 & 98.85 & RW2 \\ 
  Madagascar & ANTSIRANANA & 2008 & 58.98 & 35.14 & 97.21 & RW2 \\ 
  Madagascar & ANTSIRANANA & 2009 & 53.83 & 28.52 & 98.70 & RW2 \\ 
  Madagascar & ANTSIRANANA & 2010 & 49.36 & 21.88 & 106.42 & RW2 \\ 
  Madagascar & ANTSIRANANA & 2011 & 45.11 & 16.44 & 116.47 & RW2 \\ 
  Madagascar & ANTSIRANANA & 2012 & 41.23 & 11.89 & 130.40 & RW2 \\ 
  Madagascar & ANTSIRANANA & 2013 & 37.70 & 8.42 & 148.60 & RW2 \\ 
  Madagascar & ANTSIRANANA & 2014 & 34.27 & 5.69 & 171.13 & RW2 \\ 
  Madagascar & ANTSIRANANA & 2015 & 31.50 & 3.90 & 202.51 & RW2 \\ 
  Madagascar & ANTSIRANANA & 2016 & 28.65 & 2.55 & 238.40 & RW2 \\ 
  Madagascar & ANTSIRANANA & 2017 & 26.33 & 1.68 & 284.89 & RW2 \\ 
  Madagascar & ANTSIRANANA & 2018 & 23.85 & 1.09 & 335.41 & RW2 \\ 
  Madagascar & ANTSIRANANA & 2019 & 21.95 & 0.70 & 397.14 & RW2 \\ 
  Madagascar & FIANARANTSOA & 1980 & 242.47 & 183.11 & 314.83 & RW2 \\ 
  Madagascar & FIANARANTSOA & 1981 & 238.63 & 197.37 & 287.42 & RW2 \\ 
  Madagascar & FIANARANTSOA & 1982 & 234.81 & 198.32 & 276.08 & RW2 \\ 
  Madagascar & FIANARANTSOA & 1983 & 230.89 & 191.57 & 275.46 & RW2 \\ 
  Madagascar & FIANARANTSOA & 1984 & 227.00 & 185.94 & 273.65 & RW2 \\ 
  Madagascar & FIANARANTSOA & 1985 & 223.26 & 185.86 & 266.32 & RW2 \\ 
  Madagascar & FIANARANTSOA & 1986 & 219.37 & 185.46 & 257.30 & RW2 \\ 
  Madagascar & FIANARANTSOA & 1987 & 215.35 & 183.46 & 250.60 & RW2 \\ 
  Madagascar & FIANARANTSOA & 1988 & 211.19 & 178.24 & 247.95 & RW2 \\ 
  Madagascar & FIANARANTSOA & 1989 & 207.09 & 172.63 & 246.20 & RW2 \\ 
  Madagascar & FIANARANTSOA & 1990 & 202.74 & 169.05 & 241.14 & RW2 \\ 
  Madagascar & FIANARANTSOA & 1991 & 198.57 & 167.05 & 233.02 & RW2 \\ 
  Madagascar & FIANARANTSOA & 1992 & 194.19 & 164.69 & 226.91 & RW2 \\ 
  Madagascar & FIANARANTSOA & 1993 & 189.67 & 160.02 & 223.77 & RW2 \\ 
  Madagascar & FIANARANTSOA & 1994 & 185.34 & 154.33 & 222.00 & RW2 \\ 
  Madagascar & FIANARANTSOA & 1995 & 180.65 & 149.90 & 215.88 & RW2 \\ 
  Madagascar & FIANARANTSOA & 1996 & 176.36 & 147.68 & 209.10 & RW2 \\ 
  Madagascar & FIANARANTSOA & 1997 & 172.38 & 145.59 & 203.60 & RW2 \\ 
  Madagascar & FIANARANTSOA & 1998 & 168.27 & 140.86 & 200.29 & RW2 \\ 
  Madagascar & FIANARANTSOA & 1999 & 163.88 & 134.68 & 197.17 & RW2 \\ 
  Madagascar & FIANARANTSOA & 2000 & 159.36 & 131.12 & 193.55 & RW2 \\ 
  Madagascar & FIANARANTSOA & 2001 & 153.17 & 127.49 & 184.01 & RW2 \\ 
  Madagascar & FIANARANTSOA & 2002 & 145.83 & 121.83 & 173.73 & RW2 \\ 
  Madagascar & FIANARANTSOA & 2003 & 137.21 & 113.40 & 165.55 & RW2 \\ 
  Madagascar & FIANARANTSOA & 2004 & 127.84 & 103.53 & 157.75 & RW2 \\ 
  Madagascar & FIANARANTSOA & 2005 & 117.65 & 93.07 & 147.08 & RW2 \\ 
  Madagascar & FIANARANTSOA & 2006 & 107.77 & 86.16 & 133.50 & RW2 \\ 
  Madagascar & FIANARANTSOA & 2007 & 98.19 & 79.25 & 120.89 & RW2 \\ 
  Madagascar & FIANARANTSOA & 2008 & 89.17 & 69.27 & 113.88 & RW2 \\ 
  Madagascar & FIANARANTSOA & 2009 & 80.94 & 56.52 & 114.10 & RW2 \\ 
  Madagascar & FIANARANTSOA & 2010 & 73.40 & 42.80 & 122.29 & RW2 \\ 
  Madagascar & FIANARANTSOA & 2011 & 66.38 & 31.57 & 132.67 & RW2 \\ 
  Madagascar & FIANARANTSOA & 2012 & 60.15 & 22.59 & 147.11 & RW2 \\ 
  Madagascar & FIANARANTSOA & 2013 & 54.35 & 16.23 & 165.60 & RW2 \\ 
  Madagascar & FIANARANTSOA & 2014 & 48.94 & 11.04 & 187.78 & RW2 \\ 
  Madagascar & FIANARANTSOA & 2015 & 44.14 & 7.36 & 219.12 & RW2 \\ 
  Madagascar & FIANARANTSOA & 2016 & 39.69 & 4.74 & 252.18 & RW2 \\ 
  Madagascar & FIANARANTSOA & 2017 & 35.88 & 3.12 & 294.44 & RW2 \\ 
  Madagascar & FIANARANTSOA & 2018 & 32.07 & 1.98 & 340.70 & RW2 \\ 
  Madagascar & FIANARANTSOA & 2019 & 29.08 & 1.23 & 405.71 & RW2 \\ 
  Madagascar & MAHAJANGA & 1980 & 252.80 & 192.28 & 327.32 & RW2 \\ 
  Madagascar & MAHAJANGA & 1981 & 242.99 & 200.68 & 292.51 & RW2 \\ 
  Madagascar & MAHAJANGA & 1982 & 233.49 & 196.94 & 274.73 & RW2 \\ 
  Madagascar & MAHAJANGA & 1983 & 223.99 & 185.70 & 267.43 & RW2 \\ 
  Madagascar & MAHAJANGA & 1984 & 214.91 & 175.37 & 260.80 & RW2 \\ 
  Madagascar & MAHAJANGA & 1985 & 206.32 & 170.64 & 247.13 & RW2 \\ 
  Madagascar & MAHAJANGA & 1986 & 197.90 & 166.62 & 233.38 & RW2 \\ 
  Madagascar & MAHAJANGA & 1987 & 189.83 & 160.49 & 222.50 & RW2 \\ 
  Madagascar & MAHAJANGA & 1988 & 182.09 & 152.61 & 215.09 & RW2 \\ 
  Madagascar & MAHAJANGA & 1989 & 174.54 & 144.11 & 208.93 & RW2 \\ 
  Madagascar & MAHAJANGA & 1990 & 167.26 & 137.97 & 200.03 & RW2 \\ 
  Madagascar & MAHAJANGA & 1991 & 159.93 & 133.50 & 189.68 & RW2 \\ 
  Madagascar & MAHAJANGA & 1992 & 152.55 & 128.48 & 180.47 & RW2 \\ 
  Madagascar & MAHAJANGA & 1993 & 145.44 & 121.61 & 173.68 & RW2 \\ 
  Madagascar & MAHAJANGA & 1994 & 138.43 & 113.35 & 167.64 & RW2 \\ 
  Madagascar & MAHAJANGA & 1995 & 131.78 & 107.30 & 160.03 & RW2 \\ 
  Madagascar & MAHAJANGA & 1996 & 126.18 & 103.78 & 152.58 & RW2 \\ 
  Madagascar & MAHAJANGA & 1997 & 121.22 & 100.24 & 145.98 & RW2 \\ 
  Madagascar & MAHAJANGA & 1998 & 117.14 & 95.70 & 142.31 & RW2 \\ 
  Madagascar & MAHAJANGA & 1999 & 113.19 & 90.88 & 139.40 & RW2 \\ 
  Madagascar & MAHAJANGA & 2000 & 109.75 & 88.26 & 136.61 & RW2 \\ 
  Madagascar & MAHAJANGA & 2001 & 105.23 & 85.17 & 130.55 & RW2 \\ 
  Madagascar & MAHAJANGA & 2002 & 100.11 & 81.13 & 123.72 & RW2 \\ 
  Madagascar & MAHAJANGA & 2003 & 93.92 & 75.01 & 117.34 & RW2 \\ 
  Madagascar & MAHAJANGA & 2004 & 87.25 & 68.20 & 111.45 & RW2 \\ 
  Madagascar & MAHAJANGA & 2005 & 79.88 & 61.08 & 103.69 & RW2 \\ 
  Madagascar & MAHAJANGA & 2006 & 72.79 & 55.79 & 94.30 & RW2 \\ 
  Madagascar & MAHAJANGA & 2007 & 66.01 & 50.48 & 85.70 & RW2 \\ 
  Madagascar & MAHAJANGA & 2008 & 59.84 & 43.85 & 81.01 & RW2 \\ 
  Madagascar & MAHAJANGA & 2009 & 54.10 & 35.63 & 81.11 & RW2 \\ 
  Madagascar & MAHAJANGA & 2010 & 49.01 & 27.29 & 86.63 & RW2 \\ 
  Madagascar & MAHAJANGA & 2011 & 44.11 & 20.03 & 93.60 & RW2 \\ 
  Madagascar & MAHAJANGA & 2012 & 39.83 & 14.45 & 103.13 & RW2 \\ 
  Madagascar & MAHAJANGA & 2013 & 35.98 & 10.16 & 116.83 & RW2 \\ 
  Madagascar & MAHAJANGA & 2014 & 32.54 & 6.97 & 133.74 & RW2 \\ 
  Madagascar & MAHAJANGA & 2015 & 29.30 & 4.75 & 156.65 & RW2 \\ 
  Madagascar & MAHAJANGA & 2016 & 26.45 & 3.08 & 180.86 & RW2 \\ 
  Madagascar & MAHAJANGA & 2017 & 23.79 & 2.02 & 214.49 & RW2 \\ 
  Madagascar & MAHAJANGA & 2018 & 21.44 & 1.32 & 257.96 & RW2 \\ 
  Madagascar & MAHAJANGA & 2019 & 19.37 & 0.80 & 302.70 & RW2 \\ 
  Madagascar & TOAMASINA & 1980 & 167.43 & 121.41 & 225.66 & RW2 \\ 
  Madagascar & TOAMASINA & 1981 & 170.50 & 136.05 & 211.38 & RW2 \\ 
  Madagascar & TOAMASINA & 1982 & 173.44 & 142.77 & 209.28 & RW2 \\ 
  Madagascar & TOAMASINA & 1983 & 176.76 & 143.69 & 214.75 & RW2 \\ 
  Madagascar & TOAMASINA & 1984 & 179.52 & 144.25 & 219.85 & RW2 \\ 
  Madagascar & TOAMASINA & 1985 & 182.32 & 149.55 & 219.69 & RW2 \\ 
  Madagascar & TOAMASINA & 1986 & 184.31 & 154.46 & 217.84 & RW2 \\ 
  Madagascar & TOAMASINA & 1987 & 185.66 & 157.07 & 218.02 & RW2 \\ 
  Madagascar & TOAMASINA & 1988 & 185.98 & 156.36 & 219.39 & RW2 \\ 
  Madagascar & TOAMASINA & 1989 & 185.14 & 153.60 & 222.13 & RW2 \\ 
  Madagascar & TOAMASINA & 1990 & 182.88 & 151.44 & 219.71 & RW2 \\ 
  Madagascar & TOAMASINA & 1991 & 178.56 & 149.45 & 212.54 & RW2 \\ 
  Madagascar & TOAMASINA & 1992 & 172.39 & 144.99 & 204.14 & RW2 \\ 
  Madagascar & TOAMASINA & 1993 & 164.67 & 137.58 & 196.72 & RW2 \\ 
  Madagascar & TOAMASINA & 1994 & 155.59 & 128.12 & 188.78 & RW2 \\ 
  Madagascar & TOAMASINA & 1995 & 145.65 & 119.30 & 176.44 & RW2 \\ 
  Madagascar & TOAMASINA & 1996 & 135.95 & 112.53 & 163.51 & RW2 \\ 
  Madagascar & TOAMASINA & 1997 & 126.65 & 105.08 & 151.85 & RW2 \\ 
  Madagascar & TOAMASINA & 1998 & 117.87 & 96.78 & 142.54 & RW2 \\ 
  Madagascar & TOAMASINA & 1999 & 109.54 & 87.77 & 134.85 & RW2 \\ 
  Madagascar & TOAMASINA & 2000 & 102.17 & 81.54 & 127.42 & RW2 \\ 
  Madagascar & TOAMASINA & 2001 & 94.34 & 75.30 & 117.46 & RW2 \\ 
  Madagascar & TOAMASINA & 2002 & 86.53 & 68.80 & 108.08 & RW2 \\ 
  Madagascar & TOAMASINA & 2003 & 78.68 & 61.33 & 100.02 & RW2 \\ 
  Madagascar & TOAMASINA & 2004 & 70.97 & 53.96 & 92.86 & RW2 \\ 
  Madagascar & TOAMASINA & 2005 & 63.44 & 46.65 & 84.82 & RW2 \\ 
  Madagascar & TOAMASINA & 2006 & 56.53 & 41.15 & 76.68 & RW2 \\ 
  Madagascar & TOAMASINA & 2007 & 50.42 & 35.98 & 69.42 & RW2 \\ 
  Madagascar & TOAMASINA & 2008 & 44.78 & 30.37 & 65.38 & RW2 \\ 
  Madagascar & TOAMASINA & 2009 & 39.69 & 24.21 & 64.06 & RW2 \\ 
  Madagascar & TOAMASINA & 2010 & 35.21 & 18.12 & 66.96 & RW2 \\ 
  Madagascar & TOAMASINA & 2011 & 31.36 & 13.30 & 70.86 & RW2 \\ 
  Madagascar & TOAMASINA & 2012 & 27.82 & 9.42 & 77.52 & RW2 \\ 
  Madagascar & TOAMASINA & 2013 & 24.47 & 6.49 & 85.11 & RW2 \\ 
  Madagascar & TOAMASINA & 2014 & 21.81 & 4.45 & 98.50 & RW2 \\ 
  Madagascar & TOAMASINA & 2015 & 19.29 & 2.96 & 110.95 & RW2 \\ 
  Madagascar & TOAMASINA & 2016 & 17.13 & 1.91 & 133.71 & RW2 \\ 
  Madagascar & TOAMASINA & 2017 & 15.28 & 1.24 & 156.55 & RW2 \\ 
  Madagascar & TOAMASINA & 2018 & 13.55 & 0.78 & 184.87 & RW2 \\ 
  Madagascar & TOAMASINA & 2019 & 11.87 & 0.47 & 226.93 & RW2 \\ 
  Madagascar & TOLIARY & 1980 & 173.56 & 124.15 & 236.79 & RW2 \\ 
  Madagascar & TOLIARY & 1981 & 174.82 & 138.03 & 218.49 & RW2 \\ 
  Madagascar & TOLIARY & 1982 & 175.88 & 143.48 & 213.42 & RW2 \\ 
  Madagascar & TOLIARY & 1983 & 176.74 & 143.04 & 216.17 & RW2 \\ 
  Madagascar & TOLIARY & 1984 & 177.43 & 142.69 & 218.00 & RW2 \\ 
  Madagascar & TOLIARY & 1985 & 177.85 & 145.43 & 215.28 & RW2 \\ 
  Madagascar & TOLIARY & 1986 & 177.68 & 148.33 & 210.99 & RW2 \\ 
  Madagascar & TOLIARY & 1987 & 176.83 & 149.37 & 207.81 & RW2 \\ 
  Madagascar & TOLIARY & 1988 & 175.23 & 146.49 & 207.67 & RW2 \\ 
  Madagascar & TOLIARY & 1989 & 172.94 & 142.59 & 207.68 & RW2 \\ 
  Madagascar & TOLIARY & 1990 & 170.10 & 140.33 & 204.59 & RW2 \\ 
  Madagascar & TOLIARY & 1991 & 166.88 & 139.24 & 198.70 & RW2 \\ 
  Madagascar & TOLIARY & 1992 & 163.31 & 136.85 & 193.48 & RW2 \\ 
  Madagascar & TOLIARY & 1993 & 159.43 & 133.05 & 190.13 & RW2 \\ 
  Madagascar & TOLIARY & 1994 & 155.42 & 127.84 & 188.41 & RW2 \\ 
  Madagascar & TOLIARY & 1995 & 150.97 & 123.54 & 182.86 & RW2 \\ 
  Madagascar & TOLIARY & 1996 & 146.90 & 121.38 & 176.67 & RW2 \\ 
  Madagascar & TOLIARY & 1997 & 142.69 & 118.71 & 170.73 & RW2 \\ 
  Madagascar & TOLIARY & 1998 & 138.24 & 114.30 & 166.90 & RW2 \\ 
  Madagascar & TOLIARY & 1999 & 133.37 & 108.48 & 162.96 & RW2 \\ 
  Madagascar & TOLIARY & 2000 & 128.32 & 104.38 & 157.22 & RW2 \\ 
  Madagascar & TOLIARY & 2001 & 121.81 & 100.56 & 147.42 & RW2 \\ 
  Madagascar & TOLIARY & 2002 & 114.36 & 94.78 & 137.60 & RW2 \\ 
  Madagascar & TOLIARY & 2003 & 105.91 & 86.74 & 128.99 & RW2 \\ 
  Madagascar & TOLIARY & 2004 & 97.02 & 77.49 & 120.89 & RW2 \\ 
  Madagascar & TOLIARY & 2005 & 87.69 & 67.85 & 111.88 & RW2 \\ 
  Madagascar & TOLIARY & 2006 & 78.91 & 61.10 & 100.76 & RW2 \\ 
  Madagascar & TOLIARY & 2007 & 70.64 & 54.46 & 90.59 & RW2 \\ 
  Madagascar & TOLIARY & 2008 & 63.13 & 46.51 & 84.56 & RW2 \\ 
  Madagascar & TOLIARY & 2009 & 56.24 & 37.32 & 83.20 & RW2 \\ 
  Madagascar & TOLIARY & 2010 & 50.31 & 27.66 & 88.15 & RW2 \\ 
  Madagascar & TOLIARY & 2011 & 44.66 & 20.26 & 94.09 & RW2 \\ 
  Madagascar & TOLIARY & 2012 & 39.85 & 14.32 & 102.87 & RW2 \\ 
  Madagascar & TOLIARY & 2013 & 35.36 & 9.80 & 115.41 & RW2 \\ 
  Madagascar & TOLIARY & 2014 & 31.46 & 6.59 & 128.77 & RW2 \\ 
  Madagascar & TOLIARY & 2015 & 28.00 & 4.42 & 152.14 & RW2 \\ 
  Madagascar & TOLIARY & 2016 & 24.85 & 2.91 & 176.71 & RW2 \\ 
  Madagascar & TOLIARY & 2017 & 22.12 & 1.84 & 210.63 & RW2 \\ 
  Madagascar & TOLIARY & 2018 & 19.49 & 1.11 & 246.99 & RW2 \\ 
  Madagascar & TOLIARY & 2019 & 17.26 & 0.71 & 289.38 & RW2 \\ 
  Malawi & ALL & 1980 & 258.10 & 244.90 & 271.86 & IHME \\ 
  Malawi & ALL & 1980 & 241.77 & 188.53 & 304.58 & RW2 \\ 
  Malawi & ALL & 1980 & 256.70 & 241.00 & 272.40 & UN \\ 
  Malawi & ALL & 1981 & 253.00 & 241.01 & 265.60 & IHME \\ 
  Malawi & ALL & 1981 & 244.66 & 206.00 & 288.36 & RW2 \\ 
  Malawi & ALL & 1981 & 249.20 & 234.00 & 264.70 & UN \\ 
  Malawi & ALL & 1982 & 248.57 & 237.06 & 260.79 & IHME \\ 
  Malawi & ALL & 1982 & 247.67 & 211.43 & 287.52 & RW2 \\ 
  Malawi & ALL & 1982 & 244.40 & 229.40 & 259.70 & UN \\ 
  Malawi & ALL & 1983 & 244.80 & 233.54 & 257.50 & IHME \\ 
  Malawi & ALL & 1983 & 250.15 & 209.35 & 294.60 & RW2 \\ 
  Malawi & ALL & 1983 & 242.70 & 228.00 & 258.30 & UN \\ 
  Malawi & ALL & 1984 & 242.80 & 231.12 & 254.44 & IHME \\ 
  Malawi & ALL & 1984 & 252.50 & 207.06 & 301.42 & RW2 \\ 
  Malawi & ALL & 1984 & 244.60 & 230.30 & 260.10 & UN \\ 
  Malawi & ALL & 1985 & 241.29 & 230.06 & 252.83 & IHME \\ 
  Malawi & ALL & 1985 & 254.24 & 213.50 & 299.99 & RW2 \\ 
  Malawi & ALL & 1985 & 248.40 & 234.30 & 264.40 & UN \\ 
  Malawi & ALL & 1986 & 239.43 & 228.72 & 250.57 & IHME \\ 
  Malawi & ALL & 1986 & 254.19 & 216.36 & 295.61 & RW2 \\ 
  Malawi & ALL & 1986 & 252.00 & 237.40 & 268.30 & UN \\ 
  Malawi & ALL & 1987 & 236.73 & 225.95 & 247.75 & IHME \\ 
  Malawi & ALL & 1987 & 252.62 & 216.96 & 292.36 & RW2 \\ 
  Malawi & ALL & 1987 & 253.10 & 238.70 & 268.90 & UN \\ 
  Malawi & ALL & 1988 & 233.05 & 221.99 & 244.01 & IHME \\ 
  Malawi & ALL & 1988 & 249.31 & 212.43 & 290.81 & RW2 \\ 
  Malawi & ALL & 1988 & 251.80 & 237.40 & 267.10 & UN \\ 
  Malawi & ALL & 1989 & 228.63 & 218.19 & 238.98 & IHME \\ 
  Malawi & ALL & 1989 & 244.73 & 206.52 & 289.04 & RW2 \\ 
  Malawi & ALL & 1989 & 248.10 & 234.10 & 262.70 & UN \\ 
  Malawi & ALL & 1990 & 224.22 & 214.23 & 234.60 & IHME \\ 
  Malawi & ALL & 1990 & 238.35 & 200.17 & 280.62 & RW2 \\ 
  Malawi & ALL & 1990 & 242.40 & 229.10 & 256.90 & UN \\ 
  Malawi & ALL & 1991 & 218.71 & 209.51 & 228.54 & IHME \\ 
  Malawi & ALL & 1991 & 232.67 & 197.61 & 271.17 & RW2 \\ 
  Malawi & ALL & 1991 & 235.00 & 221.90 & 249.10 & UN \\ 
  Malawi & ALL & 1992 & 212.71 & 203.55 & 222.62 & IHME \\ 
  Malawi & ALL & 1992 & 227.10 & 193.73 & 264.12 & RW2 \\ 
  Malawi & ALL & 1992 & 226.80 & 213.90 & 239.90 & UN \\ 
  Malawi & ALL & 1993 & 205.74 & 195.98 & 216.09 & IHME \\ 
  Malawi & ALL & 1993 & 221.82 & 188.07 & 259.52 & RW2 \\ 
  Malawi & ALL & 1993 & 219.00 & 206.40 & 231.80 & UN \\ 
  Malawi & ALL & 1994 & 199.18 & 190.01 & 208.21 & IHME \\ 
  Malawi & ALL & 1994 & 216.53 & 181.03 & 256.61 & RW2 \\ 
  Malawi & ALL & 1994 & 212.60 & 200.60 & 225.30 & UN \\ 
  Malawi & ALL & 1995 & 193.54 & 184.48 & 202.51 & IHME \\ 
  Malawi & ALL & 1995 & 212.00 & 178.19 & 251.73 & RW2 \\ 
  Malawi & ALL & 1995 & 207.50 & 195.60 & 219.80 & UN \\ 
  Malawi & ALL & 1996 & 189.35 & 180.81 & 198.67 & IHME \\ 
  Malawi & ALL & 1996 & 205.63 & 174.17 & 242.37 & RW2 \\ 
  Malawi & ALL & 1996 & 203.00 & 191.60 & 215.00 & UN \\ 
  Malawi & ALL & 1997 & 184.87 & 175.48 & 194.29 & IHME \\ 
  Malawi & ALL & 1997 & 198.23 & 168.37 & 231.84 & RW2 \\ 
  Malawi & ALL & 1997 & 198.70 & 187.60 & 210.20 & UN \\ 
  Malawi & ALL & 1998 & 180.13 & 171.47 & 189.74 & IHME \\ 
  Malawi & ALL & 1998 & 189.70 & 159.82 & 224.10 & RW2 \\ 
  Malawi & ALL & 1998 & 193.10 & 182.00 & 204.70 & UN \\ 
  Malawi & ALL & 1999 & 174.30 & 165.82 & 183.87 & IHME \\ 
  Malawi & ALL & 1999 & 180.08 & 149.65 & 214.52 & RW2 \\ 
  Malawi & ALL & 1999 & 185.00 & 174.00 & 196.60 & UN \\ 
  Malawi & ALL & 2000 & 166.84 & 158.71 & 175.22 & IHME \\ 
  Malawi & ALL & 2000 & 169.34 & 140.23 & 201.77 & RW2 \\ 
  Malawi & ALL & 2000 & 174.40 & 163.70 & 185.70 & UN \\ 
  Malawi & ALL & 2001 & 157.76 & 149.94 & 166.57 & IHME \\ 
  Malawi & ALL & 2001 & 158.78 & 132.58 & 188.24 & RW2 \\ 
  Malawi & ALL & 2001 & 161.70 & 151.50 & 172.70 & UN \\ 
  Malawi & ALL & 2002 & 148.08 & 139.18 & 157.36 & IHME \\ 
  Malawi & ALL & 2002 & 148.28 & 124.71 & 175.50 & RW2 \\ 
  Malawi & ALL & 2002 & 148.00 & 138.30 & 158.40 & UN \\ 
  Malawi & ALL & 2003 & 137.14 & 128.58 & 145.70 & IHME \\ 
  Malawi & ALL & 2003 & 138.26 & 115.82 & 164.83 & RW2 \\ 
  Malawi & ALL & 2003 & 135.20 & 125.70 & 145.60 & UN \\ 
  Malawi & ALL & 2004 & 128.13 & 120.31 & 136.12 & IHME \\ 
  Malawi & ALL & 2004 & 128.67 & 105.98 & 155.82 & RW2 \\ 
  Malawi & ALL & 2004 & 124.60 & 115.10 & 134.70 & UN \\ 
  Malawi & ALL & 2005 & 121.08 & 113.11 & 128.84 & IHME \\ 
  Malawi & ALL & 2005 & 119.88 & 98.11 & 145.65 & RW2 \\ 
  Malawi & ALL & 2005 & 116.20 & 107.00 & 126.00 & UN \\ 
  Malawi & ALL & 2006 & 116.15 & 108.35 & 123.23 & IHME \\ 
  Malawi & ALL & 2006 & 111.94 & 92.63 & 134.57 & RW2 \\ 
  Malawi & ALL & 2006 & 109.90 & 101.10 & 119.30 & UN \\ 
  Malawi & ALL & 2007 & 112.53 & 105.22 & 119.91 & IHME \\ 
  Malawi & ALL & 2007 & 104.88 & 87.45 & 125.30 & RW2 \\ 
  Malawi & ALL & 2007 & 104.90 & 96.10 & 114.60 & UN \\ 
  Malawi & ALL & 2008 & 109.33 & 102.30 & 117.08 & IHME \\ 
  Malawi & ALL & 2008 & 98.71 & 81.58 & 119.01 & RW2 \\ 
  Malawi & ALL & 2008 & 100.00 & 90.50 & 110.70 & UN \\ 
  Malawi & ALL & 2009 & 105.56 & 97.93 & 113.73 & IHME \\ 
  Malawi & ALL & 2009 & 93.05 & 75.39 & 114.48 & RW2 \\ 
  Malawi & ALL & 2009 & 95.40 & 84.90 & 107.10 & UN \\ 
  Malawi & ALL & 2010 & 100.88 & 93.46 & 109.28 & IHME \\ 
  Malawi & ALL & 2010 & 88.21 & 70.43 & 111.00 & RW2 \\ 
  Malawi & ALL & 2010 & 90.90 & 79.40 & 104.40 & UN \\ 
  Malawi & ALL & 2011 & 95.21 & 87.09 & 104.12 & IHME \\ 
  Malawi & ALL & 2011 & 83.66 & 67.37 & 103.79 & RW2 \\ 
  Malawi & ALL & 2011 & 84.50 & 72.60 & 100.40 & UN \\ 
  Malawi & ALL & 2012 & 90.01 & 81.40 & 99.35 & IHME \\ 
  Malawi & ALL & 2012 & 79.39 & 64.77 & 96.98 & RW2 \\ 
  Malawi & ALL & 2012 & 77.30 & 64.80 & 95.70 & UN \\ 
  Malawi & ALL & 2013 & 86.65 & 76.14 & 97.34 & IHME \\ 
  Malawi & ALL & 2013 & 75.42 & 60.21 & 93.82 & RW2 \\ 
  Malawi & ALL & 2013 & 71.30 & 57.80 & 92.40 & UN \\ 
  Malawi & ALL & 2014 & 84.78 & 73.18 & 97.84 & IHME \\ 
  Malawi & ALL & 2014 & 71.56 & 52.09 & 97.39 & RW2 \\ 
  Malawi & ALL & 2014 & 66.90 & 51.60 & 90.60 & UN \\ 
  Malawi & ALL & 2015 & 83.35 & 70.24 & 98.43 & IHME \\ 
  Malawi & ALL & 2015 & 67.76 & 41.53 & 108.60 & RW2 \\ 
  Malawi & ALL & 2015 & 64.00 & 47.10 & 90.80 & UN \\ 
  Malawi & ALL & 2016 & 64.40 & 33.14 & 122.24 & RW2 \\ 
  Malawi & ALL & 2017 & 60.93 & 25.50 & 140.05 & RW2 \\ 
  Malawi & ALL & 2018 & 57.74 & 19.41 & 163.94 & RW2 \\ 
  Malawi & ALL & 2019 & 54.64 & 13.94 & 190.38 & RW2 \\ 
  Malawi & ALL & 80-84 & 242.81 & 253.26 & 232.65 & HT-Direct \\ 
  Malawi & ALL & 85-89 & 242.53 & 250.37 & 234.86 & HT-Direct \\ 
  Malawi & ALL & 90-94 & 216.68 & 223.62 & 209.90 & HT-Direct \\ 
  Malawi & ALL & 95-99 & 187.01 & 192.96 & 181.21 & HT-Direct \\ 
  Malawi & ALL & 00-04 & 143.11 & 148.47 & 137.91 & HT-Direct \\ 
  Malawi & ALL & 05-09 & 101.75 & 106.32 & 97.35 & HT-Direct \\ 
  Malawi & ALL & 10-14 & 68.74 & 74.00 & 63.83 & HT-Direct \\ 
  Malawi & ALL & 15-19 & 60.92 & 25.90 & 137.85 & RW2 \\ 
  Malawi & CENTRAL REGION & 1980 & 275.80 & 222.85 & 336.74 & RW2 \\ 
  Malawi & CENTRAL REGION & 1981 & 277.18 & 240.48 & 318.18 & RW2 \\ 
  Malawi & CENTRAL REGION & 1982 & 278.46 & 244.56 & 314.96 & RW2 \\ 
  Malawi & CENTRAL REGION & 1983 & 279.34 & 241.71 & 320.10 & RW2 \\ 
  Malawi & CENTRAL REGION & 1984 & 279.61 & 238.59 & 322.69 & RW2 \\ 
  Malawi & CENTRAL REGION & 1985 & 279.69 & 242.92 & 320.30 & RW2 \\ 
  Malawi & CENTRAL REGION & 1986 & 277.83 & 244.30 & 313.71 & RW2 \\ 
  Malawi & CENTRAL REGION & 1987 & 274.26 & 243.87 & 307.36 & RW2 \\ 
  Malawi & CENTRAL REGION & 1988 & 269.20 & 237.56 & 303.99 & RW2 \\ 
  Malawi & CENTRAL REGION & 1989 & 262.66 & 228.98 & 299.78 & RW2 \\ 
  Malawi & CENTRAL REGION & 1990 & 254.54 & 221.04 & 290.27 & RW2 \\ 
  Malawi & CENTRAL REGION & 1991 & 246.95 & 217.03 & 279.40 & RW2 \\ 
  Malawi & CENTRAL REGION & 1992 & 239.60 & 211.75 & 270.23 & RW2 \\ 
  Malawi & CENTRAL REGION & 1993 & 232.57 & 203.74 & 263.82 & RW2 \\ 
  Malawi & CENTRAL REGION & 1994 & 225.65 & 195.25 & 259.05 & RW2 \\ 
  Malawi & CENTRAL REGION & 1995 & 219.37 & 190.08 & 252.85 & RW2 \\ 
  Malawi & CENTRAL REGION & 1996 & 211.89 & 185.43 & 241.70 & RW2 \\ 
  Malawi & CENTRAL REGION & 1997 & 203.53 & 178.85 & 230.60 & RW2 \\ 
  Malawi & CENTRAL REGION & 1998 & 194.31 & 169.19 & 221.78 & RW2 \\ 
  Malawi & CENTRAL REGION & 1999 & 184.34 & 158.60 & 213.22 & RW2 \\ 
  Malawi & CENTRAL REGION & 2000 & 173.54 & 148.82 & 201.04 & RW2 \\ 
  Malawi & CENTRAL REGION & 2001 & 163.10 & 141.27 & 187.27 & RW2 \\ 
  Malawi & CENTRAL REGION & 2002 & 152.64 & 132.84 & 174.81 & RW2 \\ 
  Malawi & CENTRAL REGION & 2003 & 142.82 & 123.37 & 165.24 & RW2 \\ 
  Malawi & CENTRAL REGION & 2004 & 133.53 & 113.88 & 156.79 & RW2 \\ 
  Malawi & CENTRAL REGION & 2005 & 124.94 & 105.75 & 146.98 & RW2 \\ 
  Malawi & CENTRAL REGION & 2006 & 117.26 & 100.35 & 136.19 & RW2 \\ 
  Malawi & CENTRAL REGION & 2007 & 110.29 & 95.08 & 127.54 & RW2 \\ 
  Malawi & CENTRAL REGION & 2008 & 104.02 & 88.79 & 121.65 & RW2 \\ 
  Malawi & CENTRAL REGION & 2009 & 98.48 & 82.09 & 117.46 & RW2 \\ 
  Malawi & CENTRAL REGION & 2010 & 93.51 & 76.69 & 113.76 & RW2 \\ 
  Malawi & CENTRAL REGION & 2011 & 88.89 & 73.62 & 107.16 & RW2 \\ 
  Malawi & CENTRAL REGION & 2012 & 84.56 & 70.61 & 101.00 & RW2 \\ 
  Malawi & CENTRAL REGION & 2013 & 80.47 & 65.49 & 98.75 & RW2 \\ 
  Malawi & CENTRAL REGION & 2014 & 76.40 & 57.05 & 101.74 & RW2 \\ 
  Malawi & CENTRAL REGION & 2015 & 72.70 & 46.81 & 111.30 & RW2 \\ 
  Malawi & CENTRAL REGION & 2016 & 69.13 & 37.76 & 123.96 & RW2 \\ 
  Malawi & CENTRAL REGION & 2017 & 65.61 & 29.92 & 138.82 & RW2 \\ 
  Malawi & CENTRAL REGION & 2018 & 62.49 & 23.18 & 158.09 & RW2 \\ 
  Malawi & CENTRAL REGION & 2019 & 59.33 & 17.40 & 183.61 & RW2 \\ 
  Malawi & NORTHERN REGION & 1980 & 195.90 & 152.19 & 249.05 & RW2 \\ 
  Malawi & NORTHERN REGION & 1981 & 198.00 & 165.29 & 235.50 & RW2 \\ 
  Malawi & NORTHERN REGION & 1982 & 200.12 & 170.52 & 233.40 & RW2 \\ 
  Malawi & NORTHERN REGION & 1983 & 201.99 & 170.15 & 237.72 & RW2 \\ 
  Malawi & NORTHERN REGION & 1984 & 203.52 & 169.65 & 241.35 & RW2 \\ 
  Malawi & NORTHERN REGION & 1985 & 204.49 & 173.61 & 239.34 & RW2 \\ 
  Malawi & NORTHERN REGION & 1986 & 204.07 & 176.28 & 234.61 & RW2 \\ 
  Malawi & NORTHERN REGION & 1987 & 202.00 & 175.78 & 231.39 & RW2 \\ 
  Malawi & NORTHERN REGION & 1988 & 198.76 & 171.63 & 228.94 & RW2 \\ 
  Malawi & NORTHERN REGION & 1989 & 194.29 & 166.33 & 226.59 & RW2 \\ 
  Malawi & NORTHERN REGION & 1990 & 188.47 & 160.94 & 219.12 & RW2 \\ 
  Malawi & NORTHERN REGION & 1991 & 182.97 & 157.99 & 211.18 & RW2 \\ 
  Malawi & NORTHERN REGION & 1992 & 177.55 & 153.96 & 203.85 & RW2 \\ 
  Malawi & NORTHERN REGION & 1993 & 172.38 & 148.63 & 199.03 & RW2 \\ 
  Malawi & NORTHERN REGION & 1994 & 167.42 & 142.21 & 195.85 & RW2 \\ 
  Malawi & NORTHERN REGION & 1995 & 162.78 & 138.59 & 190.70 & RW2 \\ 
  Malawi & NORTHERN REGION & 1996 & 157.40 & 135.25 & 182.71 & RW2 \\ 
  Malawi & NORTHERN REGION & 1997 & 151.40 & 130.66 & 174.65 & RW2 \\ 
  Malawi & NORTHERN REGION & 1998 & 144.78 & 123.95 & 168.20 & RW2 \\ 
  Malawi & NORTHERN REGION & 1999 & 137.62 & 116.40 & 162.34 & RW2 \\ 
  Malawi & NORTHERN REGION & 2000 & 129.74 & 109.17 & 153.02 & RW2 \\ 
  Malawi & NORTHERN REGION & 2001 & 122.16 & 104.23 & 142.93 & RW2 \\ 
  Malawi & NORTHERN REGION & 2002 & 114.64 & 98.00 & 133.74 & RW2 \\ 
  Malawi & NORTHERN REGION & 2003 & 107.28 & 90.90 & 126.16 & RW2 \\ 
  Malawi & NORTHERN REGION & 2004 & 100.36 & 83.88 & 119.79 & RW2 \\ 
  Malawi & NORTHERN REGION & 2005 & 93.82 & 77.95 & 112.38 & RW2 \\ 
  Malawi & NORTHERN REGION & 2006 & 87.90 & 73.64 & 104.47 & RW2 \\ 
  Malawi & NORTHERN REGION & 2007 & 82.52 & 69.52 & 98.15 & RW2 \\ 
  Malawi & NORTHERN REGION & 2008 & 77.72 & 64.57 & 93.06 & RW2 \\ 
  Malawi & NORTHERN REGION & 2009 & 73.32 & 59.73 & 89.67 & RW2 \\ 
  Malawi & NORTHERN REGION & 2010 & 69.51 & 55.54 & 86.64 & RW2 \\ 
  Malawi & NORTHERN REGION & 2011 & 65.83 & 52.80 & 81.84 & RW2 \\ 
  Malawi & NORTHERN REGION & 2012 & 62.45 & 50.29 & 77.33 & RW2 \\ 
  Malawi & NORTHERN REGION & 2013 & 59.24 & 46.11 & 75.24 & RW2 \\ 
  Malawi & NORTHERN REGION & 2014 & 56.15 & 40.19 & 77.52 & RW2 \\ 
  Malawi & NORTHERN REGION & 2015 & 53.25 & 32.91 & 84.67 & RW2 \\ 
  Malawi & NORTHERN REGION & 2016 & 50.46 & 26.61 & 93.06 & RW2 \\ 
  Malawi & NORTHERN REGION & 2017 & 47.77 & 20.95 & 104.86 & RW2 \\ 
  Malawi & NORTHERN REGION & 2018 & 45.29 & 16.10 & 120.61 & RW2 \\ 
  Malawi & NORTHERN REGION & 2019 & 42.83 & 12.21 & 140.41 & RW2 \\ 
  Malawi & SOUTHERN REGION & 1980 & 223.89 & 178.89 & 276.02 & RW2 \\ 
  Malawi & SOUTHERN REGION & 1981 & 228.09 & 196.57 & 263.03 & RW2 \\ 
  Malawi & SOUTHERN REGION & 1982 & 232.25 & 202.77 & 264.38 & RW2 \\ 
  Malawi & SOUTHERN REGION & 1983 & 235.96 & 202.78 & 272.39 & RW2 \\ 
  Malawi & SOUTHERN REGION & 1984 & 239.62 & 203.35 & 278.83 & RW2 \\ 
  Malawi & SOUTHERN REGION & 1985 & 242.98 & 209.64 & 280.06 & RW2 \\ 
  Malawi & SOUTHERN REGION & 1986 & 244.78 & 214.34 & 277.82 & RW2 \\ 
  Malawi & SOUTHERN REGION & 1987 & 245.36 & 216.13 & 276.93 & RW2 \\ 
  Malawi & SOUTHERN REGION & 1988 & 244.44 & 214.27 & 277.19 & RW2 \\ 
  Malawi & SOUTHERN REGION & 1989 & 242.38 & 210.66 & 278.02 & RW2 \\ 
  Malawi & SOUTHERN REGION & 1990 & 238.54 & 206.65 & 272.49 & RW2 \\ 
  Malawi & SOUTHERN REGION & 1991 & 234.91 & 206.36 & 266.03 & RW2 \\ 
  Malawi & SOUTHERN REGION & 1992 & 231.07 & 204.04 & 260.55 & RW2 \\ 
  Malawi & SOUTHERN REGION & 1993 & 226.86 & 198.53 & 257.84 & RW2 \\ 
  Malawi & SOUTHERN REGION & 1994 & 222.33 & 192.19 & 255.24 & RW2 \\ 
  Malawi & SOUTHERN REGION & 1995 & 217.55 & 188.94 & 250.23 & RW2 \\ 
  Malawi & SOUTHERN REGION & 1996 & 211.10 & 184.75 & 240.19 & RW2 \\ 
  Malawi & SOUTHERN REGION & 1997 & 203.22 & 178.69 & 230.26 & RW2 \\ 
  Malawi & SOUTHERN REGION & 1998 & 194.02 & 169.23 & 221.38 & RW2 \\ 
  Malawi & SOUTHERN REGION & 1999 & 183.49 & 157.52 & 212.35 & RW2 \\ 
  Malawi & SOUTHERN REGION & 2000 & 172.24 & 147.46 & 198.94 & RW2 \\ 
  Malawi & SOUTHERN REGION & 2001 & 161.05 & 139.43 & 184.66 & RW2 \\ 
  Malawi & SOUTHERN REGION & 2002 & 150.20 & 130.93 & 171.86 & RW2 \\ 
  Malawi & SOUTHERN REGION & 2003 & 139.81 & 120.66 & 161.23 & RW2 \\ 
  Malawi & SOUTHERN REGION & 2004 & 130.20 & 110.81 & 152.76 & RW2 \\ 
  Malawi & SOUTHERN REGION & 2005 & 121.22 & 102.78 & 142.51 & RW2 \\ 
  Malawi & SOUTHERN REGION & 2006 & 113.16 & 96.85 & 131.65 & RW2 \\ 
  Malawi & SOUTHERN REGION & 2007 & 105.86 & 91.31 & 122.27 & RW2 \\ 
  Malawi & SOUTHERN REGION & 2008 & 99.20 & 84.83 & 115.86 & RW2 \\ 
  Malawi & SOUTHERN REGION & 2009 & 93.21 & 77.93 & 110.88 & RW2 \\ 
  Malawi & SOUTHERN REGION & 2010 & 87.81 & 71.91 & 106.89 & RW2 \\ 
  Malawi & SOUTHERN REGION & 2011 & 82.76 & 68.46 & 100.00 & RW2 \\ 
  Malawi & SOUTHERN REGION & 2012 & 78.02 & 65.04 & 93.25 & RW2 \\ 
  Malawi & SOUTHERN REGION & 2013 & 73.55 & 59.60 & 90.29 & RW2 \\ 
  Malawi & SOUTHERN REGION & 2014 & 69.30 & 51.51 & 92.49 & RW2 \\ 
  Malawi & SOUTHERN REGION & 2015 & 65.30 & 41.58 & 100.70 & RW2 \\ 
  Malawi & SOUTHERN REGION & 2016 & 61.47 & 33.29 & 110.43 & RW2 \\ 
  Malawi & SOUTHERN REGION & 2017 & 57.80 & 26.13 & 123.06 & RW2 \\ 
  Malawi & SOUTHERN REGION & 2018 & 54.44 & 20.08 & 140.08 & RW2 \\ 
  Malawi & SOUTHERN REGION & 2019 & 51.25 & 14.98 & 161.04 & RW2 \\ 
  Mali & ALL & 1980 & 276.41 & 268.72 & 284.06 & IHME \\ 
  Mali & ALL & 1980 & 324.53 & 251.88 & 405.82 & RW2 \\ 
  Mali & ALL & 1980 & 324.70 & 303.10 & 347.70 & UN \\ 
  Mali & ALL & 1981 & 270.49 & 263.12 & 277.90 & IHME \\ 
  Mali & ALL & 1981 & 316.41 & 265.38 & 371.52 & RW2 \\ 
  Mali & ALL & 1981 & 317.00 & 295.90 & 339.10 & UN \\ 
  Mali & ALL & 1982 & 264.51 & 257.34 & 272.00 & IHME \\ 
  Mali & ALL & 1982 & 308.49 & 262.28 & 358.80 & RW2 \\ 
  Mali & ALL & 1982 & 308.90 & 288.80 & 329.80 & UN \\ 
  Mali & ALL & 1983 & 259.20 & 252.36 & 266.72 & IHME \\ 
  Mali & ALL & 1983 & 300.38 & 249.52 & 356.60 & RW2 \\ 
  Mali & ALL & 1983 & 300.70 & 281.40 & 321.00 & UN \\ 
  Mali & ALL & 1984 & 254.20 & 247.42 & 261.56 & IHME \\ 
  Mali & ALL & 1984 & 292.94 & 237.74 & 354.30 & RW2 \\ 
  Mali & ALL & 1984 & 292.60 & 274.00 & 312.30 & UN \\ 
  Mali & ALL & 1985 & 249.07 & 242.75 & 256.48 & IHME \\ 
  Mali & ALL & 1985 & 285.13 & 235.65 & 339.89 & RW2 \\ 
  Mali & ALL & 1985 & 285.00 & 267.00 & 303.90 & UN \\ 
  Mali & ALL & 1986 & 243.91 & 237.74 & 251.17 & IHME \\ 
  Mali & ALL & 1986 & 277.93 & 233.26 & 326.83 & RW2 \\ 
  Mali & ALL & 1986 & 277.90 & 260.30 & 296.50 & UN \\ 
  Mali & ALL & 1987 & 239.35 & 233.09 & 246.43 & IHME \\ 
  Mali & ALL & 1987 & 271.15 & 229.79 & 317.65 & RW2 \\ 
  Mali & ALL & 1987 & 271.30 & 254.30 & 289.50 & UN \\ 
  Mali & ALL & 1988 & 235.93 & 229.59 & 242.67 & IHME \\ 
  Mali & ALL & 1988 & 264.67 & 221.83 & 312.96 & RW2 \\ 
  Mali & ALL & 1988 & 265.20 & 248.80 & 283.10 & UN \\ 
  Mali & ALL & 1989 & 232.35 & 226.11 & 238.91 & IHME \\ 
  Mali & ALL & 1989 & 258.86 & 214.24 & 310.25 & RW2 \\ 
  Mali & ALL & 1989 & 259.50 & 243.20 & 277.00 & UN \\ 
  Mali & ALL & 1990 & 228.96 & 222.75 & 235.13 & IHME \\ 
  Mali & ALL & 1990 & 253.48 & 209.53 & 303.67 & RW2 \\ 
  Mali & ALL & 1990 & 254.40 & 238.20 & 271.20 & UN \\ 
  Mali & ALL & 1991 & 225.37 & 218.99 & 231.70 & IHME \\ 
  Mali & ALL & 1991 & 249.42 & 208.79 & 294.77 & RW2 \\ 
  Mali & ALL & 1991 & 250.00 & 233.80 & 266.60 & UN \\ 
  Mali & ALL & 1992 & 222.29 & 216.15 & 228.50 & IHME \\ 
  Mali & ALL & 1992 & 246.33 & 207.16 & 289.87 & RW2 \\ 
  Mali & ALL & 1992 & 246.60 & 230.60 & 262.90 & UN \\ 
  Mali & ALL & 1993 & 219.13 & 212.96 & 225.54 & IHME \\ 
  Mali & ALL & 1993 & 244.14 & 203.86 & 288.96 & RW2 \\ 
  Mali & ALL & 1993 & 243.90 & 228.10 & 260.30 & UN \\ 
  Mali & ALL & 1994 & 216.28 & 210.04 & 222.83 & IHME \\ 
  Mali & ALL & 1994 & 242.33 & 199.21 & 290.95 & RW2 \\ 
  Mali & ALL & 1994 & 242.00 & 226.10 & 258.20 & UN \\ 
  Mali & ALL & 1995 & 212.94 & 206.88 & 219.18 & IHME \\ 
  Mali & ALL & 1995 & 241.48 & 199.65 & 290.39 & RW2 \\ 
  Mali & ALL & 1995 & 240.00 & 224.30 & 256.60 & UN \\ 
  Mali & ALL & 1996 & 209.54 & 203.00 & 216.12 & IHME \\ 
  Mali & ALL & 1996 & 239.13 & 199.68 & 285.06 & RW2 \\ 
  Mali & ALL & 1996 & 237.90 & 221.90 & 255.10 & UN \\ 
  Mali & ALL & 1997 & 205.67 & 199.19 & 212.06 & IHME \\ 
  Mali & ALL & 1997 & 235.81 & 197.87 & 278.49 & RW2 \\ 
  Mali & ALL & 1997 & 235.30 & 219.20 & 253.00 & UN \\ 
  Mali & ALL & 1998 & 203.82 & 196.61 & 211.64 & IHME \\ 
  Mali & ALL & 1998 & 231.28 & 192.45 & 276.03 & RW2 \\ 
  Mali & ALL & 1998 & 231.70 & 215.30 & 249.60 & UN \\ 
  Mali & ALL & 1999 & 196.54 & 190.04 & 202.93 & IHME \\ 
  Mali & ALL & 1999 & 225.34 & 184.62 & 271.25 & RW2 \\ 
  Mali & ALL & 1999 & 226.40 & 210.00 & 244.50 & UN \\ 
  Mali & ALL & 2000 & 192.60 & 186.09 & 199.06 & IHME \\ 
  Mali & ALL & 2000 & 217.84 & 177.78 & 262.78 & RW2 \\ 
  Mali & ALL & 2000 & 219.60 & 203.30 & 237.30 & UN \\ 
  Mali & ALL & 2001 & 187.45 & 180.84 & 194.17 & IHME \\ 
  Mali & ALL & 2001 & 209.84 & 172.98 & 251.35 & RW2 \\ 
  Mali & ALL & 2001 & 211.20 & 195.30 & 228.70 & UN \\ 
  Mali & ALL & 2002 & 182.45 & 175.78 & 189.41 & IHME \\ 
  Mali & ALL & 2002 & 201.15 & 167.35 & 239.90 & RW2 \\ 
  Mali & ALL & 2002 & 201.80 & 186.10 & 219.00 & UN \\ 
  Mali & ALL & 2003 & 177.36 & 170.73 & 184.63 & IHME \\ 
  Mali & ALL & 2003 & 192.20 & 158.90 & 231.15 & RW2 \\ 
  Mali & ALL & 2003 & 191.80 & 175.80 & 209.20 & UN \\ 
  Mali & ALL & 2004 & 172.36 & 165.51 & 179.73 & IHME \\ 
  Mali & ALL & 2004 & 182.82 & 147.29 & 224.82 & RW2 \\ 
  Mali & ALL & 2004 & 181.70 & 164.90 & 200.00 & UN \\ 
  Mali & ALL & 2005 & 167.49 & 160.44 & 174.87 & IHME \\ 
  Mali & ALL & 2005 & 173.63 & 136.40 & 218.10 & RW2 \\ 
  Mali & ALL & 2005 & 171.80 & 153.50 & 192.20 & UN \\ 
  Mali & ALL & 2006 & 162.97 & 155.54 & 170.61 & IHME \\ 
  Mali & ALL & 2006 & 164.49 & 130.86 & 204.29 & RW2 \\ 
  Mali & ALL & 2006 & 162.90 & 142.40 & 185.60 & UN \\ 
  Mali & ALL & 2007 & 158.92 & 150.94 & 166.90 & IHME \\ 
  Mali & ALL & 2007 & 155.70 & 126.35 & 190.27 & RW2 \\ 
  Mali & ALL & 2007 & 155.10 & 131.90 & 179.80 & UN \\ 
  Mali & ALL & 2008 & 155.00 & 146.59 & 163.15 & IHME \\ 
  Mali & ALL & 2008 & 147.41 & 117.02 & 184.25 & RW2 \\ 
  Mali & ALL & 2008 & 148.30 & 122.30 & 176.70 & UN \\ 
  Mali & ALL & 2009 & 151.15 & 141.70 & 160.63 & IHME \\ 
  Mali & ALL & 2009 & 139.13 & 100.01 & 191.15 & RW2 \\ 
  Mali & ALL & 2009 & 142.00 & 113.30 & 174.80 & UN \\ 
  Mali & ALL & 2010 & 147.35 & 137.15 & 157.84 & IHME \\ 
  Mali & ALL & 2010 & 131.35 & 78.52 & 213.99 & RW2 \\ 
  Mali & ALL & 2010 & 136.60 & 105.30 & 174.30 & UN \\ 
  Mali & ALL & 2011 & 143.32 & 132.27 & 155.08 & IHME \\ 
  Mali & ALL & 2011 & 124.09 & 60.15 & 239.19 & RW2 \\ 
  Mali & ALL & 2011 & 131.70 & 97.50 & 173.80 & UN \\ 
  Mali & ALL & 2012 & 139.95 & 128.24 & 153.13 & IHME \\ 
  Mali & ALL & 2012 & 116.94 & 44.47 & 273.41 & RW2 \\ 
  Mali & ALL & 2012 & 127.00 & 90.90 & 174.20 & UN \\ 
  Mali & ALL & 2013 & 136.92 & 123.74 & 152.15 & IHME \\ 
  Mali & ALL & 2013 & 110.57 & 32.43 & 314.03 & RW2 \\ 
  Mali & ALL & 2013 & 122.70 & 84.90 & 174.50 & UN \\ 
  Mali & ALL & 2014 & 133.52 & 119.54 & 150.69 & IHME \\ 
  Mali & ALL & 2014 & 104.11 & 22.97 & 363.69 & RW2 \\ 
  Mali & ALL & 2014 & 118.30 & 78.60 & 175.30 & UN \\ 
  Mali & ALL & 2015 & 130.20 & 115.37 & 148.88 & IHME \\ 
  Mali & ALL & 2015 & 97.72 & 16.18 & 417.03 & RW2 \\ 
  Mali & ALL & 2015 & 114.70 & 73.00 & 175.90 & UN \\ 
  Mali & ALL & 2016 & 92.66 & 11.24 & 485.06 & RW2 \\ 
  Mali & ALL & 2017 & 86.74 & 7.42 & 553.96 & RW2 \\ 
  Mali & ALL & 2018 & 81.60 & 5.00 & 629.15 & RW2 \\ 
  Mali & ALL & 2019 & 76.53 & 3.00 & 688.57 & RW2 \\ 
  Mali & ALL & 80-84 & 296.64 & 306.73 & 286.74 & HT-Direct \\ 
  Mali & ALL & 85-89 & 268.90 & 277.16 & 260.79 & HT-Direct \\ 
  Mali & ALL & 90-94 & 245.08 & 251.99 & 238.30 & HT-Direct \\ 
  Mali & ALL & 95-99 & 245.50 & 254.85 & 236.39 & HT-Direct \\ 
  Mali & ALL & 00-04 & 201.49 & 210.20 & 193.05 & HT-Direct \\ 
  Mali & ALL & 05-09 & 168.55 & 185.28 & 153.05 & HT-Direct \\ 
  Mali & ALL & 15-19 & 86.74 & 7.49 & 548.82 & RW2 \\ 
  Mali & BAMAKO & 1980 & 193.50 & 148.44 & 248.90 & RW2 \\ 
  Mali & BAMAKO & 1981 & 187.34 & 155.61 & 223.88 & RW2 \\ 
  Mali & BAMAKO & 1982 & 181.38 & 153.23 & 213.52 & RW2 \\ 
  Mali & BAMAKO & 1983 & 175.53 & 145.40 & 210.52 & RW2 \\ 
  Mali & BAMAKO & 1984 & 169.99 & 138.63 & 206.77 & RW2 \\ 
  Mali & BAMAKO & 1985 & 164.46 & 136.17 & 196.86 & RW2 \\ 
  Mali & BAMAKO & 1986 & 159.40 & 134.67 & 187.03 & RW2 \\ 
  Mali & BAMAKO & 1987 & 154.39 & 131.54 & 180.57 & RW2 \\ 
  Mali & BAMAKO & 1988 & 149.94 & 126.47 & 176.76 & RW2 \\ 
  Mali & BAMAKO & 1989 & 145.93 & 121.78 & 174.67 & RW2 \\ 
  Mali & BAMAKO & 1990 & 142.25 & 118.68 & 169.78 & RW2 \\ 
  Mali & BAMAKO & 1991 & 139.59 & 118.16 & 164.88 & RW2 \\ 
  Mali & BAMAKO & 1992 & 137.70 & 117.25 & 161.32 & RW2 \\ 
  Mali & BAMAKO & 1993 & 136.48 & 115.38 & 160.77 & RW2 \\ 
  Mali & BAMAKO & 1994 & 135.65 & 112.63 & 162.18 & RW2 \\ 
  Mali & BAMAKO & 1995 & 135.24 & 112.81 & 162.11 & RW2 \\ 
  Mali & BAMAKO & 1996 & 133.87 & 112.97 & 158.51 & RW2 \\ 
  Mali & BAMAKO & 1997 & 131.56 & 111.65 & 154.47 & RW2 \\ 
  Mali & BAMAKO & 1998 & 128.22 & 107.77 & 151.83 & RW2 \\ 
  Mali & BAMAKO & 1999 & 123.91 & 102.56 & 149.48 & RW2 \\ 
  Mali & BAMAKO & 2000 & 118.34 & 97.03 & 142.88 & RW2 \\ 
  Mali & BAMAKO & 2001 & 112.87 & 94.03 & 135.06 & RW2 \\ 
  Mali & BAMAKO & 2002 & 107.20 & 89.42 & 127.90 & RW2 \\ 
  Mali & BAMAKO & 2003 & 101.43 & 83.27 & 122.74 & RW2 \\ 
  Mali & BAMAKO & 2004 & 95.87 & 76.54 & 119.26 & RW2 \\ 
  Mali & BAMAKO & 2005 & 90.51 & 70.21 & 115.53 & RW2 \\ 
  Mali & BAMAKO & 2006 & 85.37 & 65.85 & 109.33 & RW2 \\ 
  Mali & BAMAKO & 2007 & 80.45 & 61.71 & 104.10 & RW2 \\ 
  Mali & BAMAKO & 2008 & 75.85 & 55.72 & 101.20 & RW2 \\ 
  Mali & BAMAKO & 2009 & 71.34 & 47.98 & 104.01 & RW2 \\ 
  Mali & BAMAKO & 2010 & 67.27 & 38.52 & 113.69 & RW2 \\ 
  Mali & BAMAKO & 2011 & 63.23 & 30.36 & 126.37 & RW2 \\ 
  Mali & BAMAKO & 2012 & 59.56 & 23.56 & 143.09 & RW2 \\ 
  Mali & BAMAKO & 2013 & 56.03 & 17.31 & 163.17 & RW2 \\ 
  Mali & BAMAKO & 2014 & 52.57 & 12.56 & 193.27 & RW2 \\ 
  Mali & BAMAKO & 2015 & 49.57 & 9.08 & 228.21 & RW2 \\ 
  Mali & BAMAKO & 2016 & 46.53 & 6.44 & 266.43 & RW2 \\ 
  Mali & BAMAKO & 2017 & 43.62 & 4.47 & 315.31 & RW2 \\ 
  Mali & BAMAKO & 2018 & 41.04 & 3.04 & 375.14 & RW2 \\ 
  Mali & BAMAKO & 2019 & 38.39 & 2.06 & 441.57 & RW2 \\ 
  Mali & KAYES, KOULIKORO & 1980 & 320.30 & 257.43 & 389.54 & RW2 \\ 
  Mali & KAYES, KOULIKORO & 1981 & 311.41 & 269.42 & 356.37 & RW2 \\ 
  Mali & KAYES, KOULIKORO & 1982 & 302.57 & 264.50 & 343.53 & RW2 \\ 
  Mali & KAYES, KOULIKORO & 1983 & 293.57 & 251.70 & 339.68 & RW2 \\ 
  Mali & KAYES, KOULIKORO & 1984 & 285.19 & 240.88 & 333.79 & RW2 \\ 
  Mali & KAYES, KOULIKORO & 1985 & 276.91 & 236.80 & 320.77 & RW2 \\ 
  Mali & KAYES, KOULIKORO & 1986 & 268.94 & 233.67 & 306.97 & RW2 \\ 
  Mali & KAYES, KOULIKORO & 1987 & 261.44 & 228.49 & 297.17 & RW2 \\ 
  Mali & KAYES, KOULIKORO & 1988 & 254.34 & 220.65 & 291.10 & RW2 \\ 
  Mali & KAYES, KOULIKORO & 1989 & 248.12 & 212.83 & 287.93 & RW2 \\ 
  Mali & KAYES, KOULIKORO & 1990 & 242.27 & 207.32 & 280.36 & RW2 \\ 
  Mali & KAYES, KOULIKORO & 1991 & 238.22 & 207.12 & 272.72 & RW2 \\ 
  Mali & KAYES, KOULIKORO & 1992 & 235.60 & 206.04 & 268.01 & RW2 \\ 
  Mali & KAYES, KOULIKORO & 1993 & 234.01 & 202.43 & 268.30 & RW2 \\ 
  Mali & KAYES, KOULIKORO & 1994 & 233.25 & 198.90 & 270.29 & RW2 \\ 
  Mali & KAYES, KOULIKORO & 1995 & 233.28 & 200.38 & 271.13 & RW2 \\ 
  Mali & KAYES, KOULIKORO & 1996 & 231.91 & 201.08 & 265.93 & RW2 \\ 
  Mali & KAYES, KOULIKORO & 1997 & 229.19 & 199.96 & 261.29 & RW2 \\ 
  Mali & KAYES, KOULIKORO & 1998 & 224.90 & 194.59 & 258.37 & RW2 \\ 
  Mali & KAYES, KOULIKORO & 1999 & 218.76 & 185.92 & 255.39 & RW2 \\ 
  Mali & KAYES, KOULIKORO & 2000 & 211.07 & 178.34 & 246.11 & RW2 \\ 
  Mali & KAYES, KOULIKORO & 2001 & 202.98 & 173.76 & 234.81 & RW2 \\ 
  Mali & KAYES, KOULIKORO & 2002 & 194.59 & 168.01 & 224.48 & RW2 \\ 
  Mali & KAYES, KOULIKORO & 2003 & 185.95 & 158.31 & 216.91 & RW2 \\ 
  Mali & KAYES, KOULIKORO & 2004 & 177.53 & 147.55 & 212.44 & RW2 \\ 
  Mali & KAYES, KOULIKORO & 2005 & 169.21 & 137.65 & 206.58 & RW2 \\ 
  Mali & KAYES, KOULIKORO & 2006 & 161.17 & 131.54 & 195.79 & RW2 \\ 
  Mali & KAYES, KOULIKORO & 2007 & 153.42 & 126.21 & 185.03 & RW2 \\ 
  Mali & KAYES, KOULIKORO & 2008 & 145.82 & 116.62 & 181.17 & RW2 \\ 
  Mali & KAYES, KOULIKORO & 2009 & 138.63 & 101.16 & 186.78 & RW2 \\ 
  Mali & KAYES, KOULIKORO & 2010 & 131.63 & 81.41 & 205.48 & RW2 \\ 
  Mali & KAYES, KOULIKORO & 2011 & 125.05 & 65.34 & 227.48 & RW2 \\ 
  Mali & KAYES, KOULIKORO & 2012 & 118.69 & 50.59 & 253.66 & RW2 \\ 
  Mali & KAYES, KOULIKORO & 2013 & 112.53 & 38.15 & 288.70 & RW2 \\ 
  Mali & KAYES, KOULIKORO & 2014 & 106.63 & 28.13 & 330.41 & RW2 \\ 
  Mali & KAYES, KOULIKORO & 2015 & 101.34 & 20.22 & 380.12 & RW2 \\ 
  Mali & KAYES, KOULIKORO & 2016 & 95.93 & 14.40 & 434.66 & RW2 \\ 
  Mali & KAYES, KOULIKORO & 2017 & 90.62 & 10.24 & 491.57 & RW2 \\ 
  Mali & KAYES, KOULIKORO & 2018 & 85.96 & 7.14 & 557.57 & RW2 \\ 
  Mali & KAYES, KOULIKORO & 2019 & 81.46 & 4.76 & 623.17 & RW2 \\ 
  Mali & MOPTI, TOMBOUCTOU, GAO, KIDAL & 1980 & 430.33 & 355.60 & 507.66 & RW2 \\ 
  Mali & MOPTI, TOMBOUCTOU, GAO, KIDAL & 1981 & 418.74 & 367.57 & 471.54 & RW2 \\ 
  Mali & MOPTI, TOMBOUCTOU, GAO, KIDAL & 1982 & 406.85 & 361.13 & 453.69 & RW2 \\ 
  Mali & MOPTI, TOMBOUCTOU, GAO, KIDAL & 1983 & 395.09 & 344.70 & 447.60 & RW2 \\ 
  Mali & MOPTI, TOMBOUCTOU, GAO, KIDAL & 1984 & 383.32 & 329.10 & 439.86 & RW2 \\ 
  Mali & MOPTI, TOMBOUCTOU, GAO, KIDAL & 1985 & 371.74 & 323.35 & 422.48 & RW2 \\ 
  Mali & MOPTI, TOMBOUCTOU, GAO, KIDAL & 1986 & 360.04 & 317.40 & 403.99 & RW2 \\ 
  Mali & MOPTI, TOMBOUCTOU, GAO, KIDAL & 1987 & 348.26 & 309.03 & 389.81 & RW2 \\ 
  Mali & MOPTI, TOMBOUCTOU, GAO, KIDAL & 1988 & 336.69 & 296.10 & 380.61 & RW2 \\ 
  Mali & MOPTI, TOMBOUCTOU, GAO, KIDAL & 1989 & 325.56 & 282.99 & 372.37 & RW2 \\ 
  Mali & MOPTI, TOMBOUCTOU, GAO, KIDAL & 1990 & 314.60 & 272.86 & 359.74 & RW2 \\ 
  Mali & MOPTI, TOMBOUCTOU, GAO, KIDAL & 1991 & 305.59 & 268.35 & 346.21 & RW2 \\ 
  Mali & MOPTI, TOMBOUCTOU, GAO, KIDAL & 1992 & 297.73 & 262.28 & 335.89 & RW2 \\ 
  Mali & MOPTI, TOMBOUCTOU, GAO, KIDAL & 1993 & 290.93 & 254.26 & 329.97 & RW2 \\ 
  Mali & MOPTI, TOMBOUCTOU, GAO, KIDAL & 1994 & 285.08 & 245.10 & 326.71 & RW2 \\ 
  Mali & MOPTI, TOMBOUCTOU, GAO, KIDAL & 1995 & 279.94 & 241.63 & 322.86 & RW2 \\ 
  Mali & MOPTI, TOMBOUCTOU, GAO, KIDAL & 1996 & 273.02 & 237.80 & 311.44 & RW2 \\ 
  Mali & MOPTI, TOMBOUCTOU, GAO, KIDAL & 1997 & 264.46 & 231.45 & 300.47 & RW2 \\ 
  Mali & MOPTI, TOMBOUCTOU, GAO, KIDAL & 1998 & 254.15 & 221.15 & 290.56 & RW2 \\ 
  Mali & MOPTI, TOMBOUCTOU, GAO, KIDAL & 1999 & 242.28 & 206.53 & 281.03 & RW2 \\ 
  Mali & MOPTI, TOMBOUCTOU, GAO, KIDAL & 2000 & 228.60 & 193.35 & 266.25 & RW2 \\ 
  Mali & MOPTI, TOMBOUCTOU, GAO, KIDAL & 2001 & 215.31 & 184.03 & 248.96 & RW2 \\ 
  Mali & MOPTI, TOMBOUCTOU, GAO, KIDAL & 2002 & 202.07 & 174.04 & 233.14 & RW2 \\ 
  Mali & MOPTI, TOMBOUCTOU, GAO, KIDAL & 2003 & 189.25 & 161.36 & 221.33 & RW2 \\ 
  Mali & MOPTI, TOMBOUCTOU, GAO, KIDAL & 2004 & 177.21 & 147.28 & 211.83 & RW2 \\ 
  Mali & MOPTI, TOMBOUCTOU, GAO, KIDAL & 2005 & 165.89 & 134.99 & 202.54 & RW2 \\ 
  Mali & MOPTI, TOMBOUCTOU, GAO, KIDAL & 2006 & 155.22 & 127.22 & 188.49 & RW2 \\ 
  Mali & MOPTI, TOMBOUCTOU, GAO, KIDAL & 2007 & 145.03 & 119.27 & 175.15 & RW2 \\ 
  Mali & MOPTI, TOMBOUCTOU, GAO, KIDAL & 2008 & 135.52 & 108.04 & 168.81 & RW2 \\ 
  Mali & MOPTI, TOMBOUCTOU, GAO, KIDAL & 2009 & 126.31 & 91.92 & 171.52 & RW2 \\ 
  Mali & MOPTI, TOMBOUCTOU, GAO, KIDAL & 2010 & 117.88 & 73.20 & 185.80 & RW2 \\ 
  Mali & MOPTI, TOMBOUCTOU, GAO, KIDAL & 2011 & 109.67 & 56.88 & 202.88 & RW2 \\ 
  Mali & MOPTI, TOMBOUCTOU, GAO, KIDAL & 2012 & 102.10 & 43.14 & 224.90 & RW2 \\ 
  Mali & MOPTI, TOMBOUCTOU, GAO, KIDAL & 2013 & 95.15 & 31.96 & 256.24 & RW2 \\ 
  Mali & MOPTI, TOMBOUCTOU, GAO, KIDAL & 2014 & 88.65 & 23.01 & 288.22 & RW2 \\ 
  Mali & MOPTI, TOMBOUCTOU, GAO, KIDAL & 2015 & 82.43 & 16.33 & 328.47 & RW2 \\ 
  Mali & MOPTI, TOMBOUCTOU, GAO, KIDAL & 2016 & 76.30 & 11.35 & 374.05 & RW2 \\ 
  Mali & MOPTI, TOMBOUCTOU, GAO, KIDAL & 2017 & 71.14 & 7.84 & 429.54 & RW2 \\ 
  Mali & MOPTI, TOMBOUCTOU, GAO, KIDAL & 2018 & 65.65 & 5.23 & 491.06 & RW2 \\ 
  Mali & MOPTI, TOMBOUCTOU, GAO, KIDAL & 2019 & 61.04 & 3.46 & 560.20 & RW2 \\ 
  Mali & SIKASSO, SEGOU & 1980 & 299.87 & 240.15 & 367.88 & RW2 \\ 
  Mali & SIKASSO, SEGOU & 1981 & 294.96 & 254.66 & 338.79 & RW2 \\ 
  Mali & SIKASSO, SEGOU & 1982 & 290.14 & 253.46 & 328.85 & RW2 \\ 
  Mali & SIKASSO, SEGOU & 1983 & 285.13 & 244.30 & 330.09 & RW2 \\ 
  Mali & SIKASSO, SEGOU & 1984 & 280.41 & 236.83 & 328.34 & RW2 \\ 
  Mali & SIKASSO, SEGOU & 1985 & 275.72 & 235.31 & 318.45 & RW2 \\ 
  Mali & SIKASSO, SEGOU & 1986 & 270.93 & 235.86 & 309.02 & RW2 \\ 
  Mali & SIKASSO, SEGOU & 1987 & 266.45 & 234.07 & 301.53 & RW2 \\ 
  Mali & SIKASSO, SEGOU & 1988 & 262.17 & 228.45 & 299.51 & RW2 \\ 
  Mali & SIKASSO, SEGOU & 1989 & 258.33 & 222.19 & 299.59 & RW2 \\ 
  Mali & SIKASSO, SEGOU & 1990 & 254.88 & 218.88 & 294.88 & RW2 \\ 
  Mali & SIKASSO, SEGOU & 1991 & 253.24 & 220.34 & 288.76 & RW2 \\ 
  Mali & SIKASSO, SEGOU & 1992 & 252.60 & 221.35 & 286.56 & RW2 \\ 
  Mali & SIKASSO, SEGOU & 1993 & 253.22 & 220.65 & 289.11 & RW2 \\ 
  Mali & SIKASSO, SEGOU & 1994 & 254.67 & 218.19 & 294.06 & RW2 \\ 
  Mali & SIKASSO, SEGOU & 1995 & 256.87 & 220.86 & 298.25 & RW2 \\ 
  Mali & SIKASSO, SEGOU & 1996 & 257.40 & 223.95 & 293.97 & RW2 \\ 
  Mali & SIKASSO, SEGOU & 1997 & 256.46 & 224.41 & 291.91 & RW2 \\ 
  Mali & SIKASSO, SEGOU & 1998 & 253.79 & 220.44 & 290.58 & RW2 \\ 
  Mali & SIKASSO, SEGOU & 1999 & 249.05 & 213.24 & 288.72 & RW2 \\ 
  Mali & SIKASSO, SEGOU & 2000 & 242.23 & 206.26 & 281.59 & RW2 \\ 
  Mali & SIKASSO, SEGOU & 2001 & 235.15 & 202.42 & 270.90 & RW2 \\ 
  Mali & SIKASSO, SEGOU & 2002 & 227.42 & 197.45 & 260.86 & RW2 \\ 
  Mali & SIKASSO, SEGOU & 2003 & 219.50 & 188.58 & 254.26 & RW2 \\ 
  Mali & SIKASSO, SEGOU & 2004 & 211.67 & 177.68 & 249.69 & RW2 \\ 
  Mali & SIKASSO, SEGOU & 2005 & 203.99 & 167.95 & 245.40 & RW2 \\ 
  Mali & SIKASSO, SEGOU & 2006 & 196.32 & 163.71 & 234.36 & RW2 \\ 
  Mali & SIKASSO, SEGOU & 2007 & 189.05 & 159.64 & 223.01 & RW2 \\ 
  Mali & SIKASSO, SEGOU & 2008 & 181.95 & 149.68 & 219.42 & RW2 \\ 
  Mali & SIKASSO, SEGOU & 2009 & 174.76 & 131.66 & 228.21 & RW2 \\ 
  Mali & SIKASSO, SEGOU & 2010 & 168.08 & 108.01 & 252.41 & RW2 \\ 
  Mali & SIKASSO, SEGOU & 2011 & 161.43 & 87.62 & 280.07 & RW2 \\ 
  Mali & SIKASSO, SEGOU & 2012 & 154.97 & 68.68 & 314.69 & RW2 \\ 
  Mali & SIKASSO, SEGOU & 2013 & 148.77 & 52.75 & 355.96 & RW2 \\ 
  Mali & SIKASSO, SEGOU & 2014 & 142.29 & 38.76 & 402.69 & RW2 \\ 
  Mali & SIKASSO, SEGOU & 2015 & 137.02 & 28.80 & 459.95 & RW2 \\ 
  Mali & SIKASSO, SEGOU & 2016 & 131.04 & 20.49 & 518.50 & RW2 \\ 
  Mali & SIKASSO, SEGOU & 2017 & 126.17 & 14.66 & 583.67 & RW2 \\ 
  Mali & SIKASSO, SEGOU & 2018 & 120.25 & 10.30 & 644.77 & RW2 \\ 
  Mali & SIKASSO, SEGOU & 2019 & 115.92 & 7.21 & 707.26 & RW2 \\ 
  Morocco & ALL & 1980 & 124.28 & 120.41 & 128.36 & IHME \\ 
  Morocco & ALL & 1980 & 133.68 & 90.47 & 192.39 & RW2 \\ 
  Morocco & ALL & 1980 & 133.80 & 126.40 & 141.80 & UN \\ 
  Morocco & ALL & 1981 & 117.60 & 113.73 & 121.58 & IHME \\ 
  Morocco & ALL & 1981 & 126.99 & 97.01 & 163.87 & RW2 \\ 
  Morocco & ALL & 1981 & 127.20 & 120.00 & 134.80 & UN \\ 
  Morocco & ALL & 1982 & 110.84 & 107.13 & 114.74 & IHME \\ 
  Morocco & ALL & 1982 & 120.62 & 94.35 & 152.98 & RW2 \\ 
  Morocco & ALL & 1982 & 120.80 & 114.00 & 127.90 & UN \\ 
  Morocco & ALL & 1983 & 104.53 & 100.97 & 108.22 & IHME \\ 
  Morocco & ALL & 1983 & 114.36 & 86.30 & 150.05 & RW2 \\ 
  Morocco & ALL & 1983 & 114.60 & 108.10 & 121.40 & UN \\ 
  Morocco & ALL & 1984 & 98.56 & 95.07 & 102.21 & IHME \\ 
  Morocco & ALL & 1984 & 108.68 & 79.20 & 146.99 & RW2 \\ 
  Morocco & ALL & 1984 & 108.60 & 102.40 & 115.20 & UN \\ 
  Morocco & ALL & 1985 & 92.89 & 89.67 & 96.34 & IHME \\ 
  Morocco & ALL & 1985 & 102.96 & 77.07 & 136.31 & RW2 \\ 
  Morocco & ALL & 1985 & 103.00 & 96.90 & 109.30 & UN \\ 
  Morocco & ALL & 1986 & 87.53 & 84.50 & 90.90 & IHME \\ 
  Morocco & ALL & 1986 & 97.78 & 75.00 & 126.41 & RW2 \\ 
  Morocco & ALL & 1986 & 97.70 & 91.80 & 103.80 & UN \\ 
  Morocco & ALL & 1987 & 82.78 & 79.73 & 85.92 & IHME \\ 
  Morocco & ALL & 1987 & 92.97 & 72.44 & 119.07 & RW2 \\ 
  Morocco & ALL & 1987 & 92.70 & 87.10 & 98.60 & UN \\ 
  Morocco & ALL & 1988 & 78.14 & 75.08 & 81.12 & IHME \\ 
  Morocco & ALL & 1988 & 88.39 & 67.64 & 114.65 & RW2 \\ 
  Morocco & ALL & 1988 & 88.10 & 82.70 & 93.90 & UN \\ 
  Morocco & ALL & 1989 & 73.96 & 70.99 & 76.81 & IHME \\ 
  Morocco & ALL & 1989 & 84.13 & 63.14 & 111.30 & RW2 \\ 
  Morocco & ALL & 1989 & 84.00 & 78.60 & 89.50 & UN \\ 
  Morocco & ALL & 1990 & 70.10 & 67.20 & 72.97 & IHME \\ 
  Morocco & ALL & 1990 & 80.20 & 60.09 & 106.56 & RW2 \\ 
  Morocco & ALL & 1990 & 80.10 & 74.90 & 85.50 & UN \\ 
  Morocco & ALL & 1991 & 66.51 & 63.65 & 69.47 & IHME \\ 
  Morocco & ALL & 1991 & 76.47 & 58.26 & 99.44 & RW2 \\ 
  Morocco & ALL & 1991 & 76.50 & 71.40 & 81.70 & UN \\ 
  Morocco & ALL & 1992 & 63.33 & 60.54 & 66.32 & IHME \\ 
  Morocco & ALL & 1992 & 72.96 & 56.01 & 94.21 & RW2 \\ 
  Morocco & ALL & 1992 & 73.00 & 68.20 & 78.10 & UN \\ 
  Morocco & ALL & 1993 & 60.45 & 57.62 & 63.38 & IHME \\ 
  Morocco & ALL & 1993 & 69.67 & 53.00 & 91.02 & RW2 \\ 
  Morocco & ALL & 1993 & 69.70 & 64.90 & 74.60 & UN \\ 
  Morocco & ALL & 1994 & 57.60 & 54.63 & 60.51 & IHME \\ 
  Morocco & ALL & 1994 & 66.49 & 49.51 & 89.16 & RW2 \\ 
  Morocco & ALL & 1994 & 66.40 & 61.80 & 71.30 & UN \\ 
  Morocco & ALL & 1995 & 55.20 & 52.26 & 58.11 & IHME \\ 
  Morocco & ALL & 1995 & 63.55 & 47.17 & 85.17 & RW2 \\ 
  Morocco & ALL & 1995 & 63.30 & 58.70 & 68.10 & UN \\ 
  Morocco & ALL & 1996 & 52.54 & 49.53 & 55.59 & IHME \\ 
  Morocco & ALL & 1996 & 60.66 & 45.72 & 80.49 & RW2 \\ 
  Morocco & ALL & 1996 & 60.30 & 55.90 & 65.00 & UN \\ 
  Morocco & ALL & 1997 & 50.24 & 47.17 & 53.50 & IHME \\ 
  Morocco & ALL & 1997 & 57.90 & 44.10 & 75.77 & RW2 \\ 
  Morocco & ALL & 1997 & 57.50 & 53.10 & 62.10 & UN \\ 
  Morocco & ALL & 1998 & 47.94 & 44.93 & 51.24 & IHME \\ 
  Morocco & ALL & 1998 & 55.29 & 41.53 & 73.76 & RW2 \\ 
  Morocco & ALL & 1998 & 54.80 & 50.40 & 59.40 & UN \\ 
  Morocco & ALL & 1999 & 45.84 & 42.66 & 49.39 & IHME \\ 
  Morocco & ALL & 1999 & 52.75 & 38.47 & 71.68 & RW2 \\ 
  Morocco & ALL & 1999 & 52.30 & 48.00 & 56.80 & UN \\ 
  Morocco & ALL & 2000 & 43.85 & 40.54 & 47.50 & IHME \\ 
  Morocco & ALL & 2000 & 50.29 & 35.90 & 69.88 & RW2 \\ 
  Morocco & ALL & 2000 & 50.00 & 45.70 & 54.50 & UN \\ 
  Morocco & ALL & 2001 & 41.97 & 38.49 & 45.66 & IHME \\ 
  Morocco & ALL & 2001 & 47.97 & 34.78 & 65.69 & RW2 \\ 
  Morocco & ALL & 2001 & 47.90 & 43.60 & 52.20 & UN \\ 
  Morocco & ALL & 2002 & 40.23 & 36.65 & 44.09 & IHME \\ 
  Morocco & ALL & 2002 & 45.72 & 34.05 & 61.16 & RW2 \\ 
  Morocco & ALL & 2002 & 45.80 & 41.60 & 50.20 & UN \\ 
  Morocco & ALL & 2003 & 38.48 & 34.48 & 42.67 & IHME \\ 
  Morocco & ALL & 2003 & 43.62 & 31.47 & 60.19 & RW2 \\ 
  Morocco & ALL & 2003 & 43.90 & 39.70 & 48.20 & UN \\ 
  Morocco & ALL & 2004 & 36.88 & 32.69 & 41.38 & IHME \\ 
  Morocco & ALL & 2004 & 41.47 & 25.96 & 65.41 & RW2 \\ 
  Morocco & ALL & 2004 & 42.10 & 38.00 & 46.50 & UN \\ 
  Morocco & ALL & 2005 & 35.24 & 30.74 & 40.01 & IHME \\ 
  Morocco & ALL & 2005 & 39.57 & 19.32 & 79.33 & RW2 \\ 
  Morocco & ALL & 2005 & 40.40 & 36.20 & 44.80 & UN \\ 
  Morocco & ALL & 2006 & 33.70 & 29.09 & 38.60 & IHME \\ 
  Morocco & ALL & 2006 & 37.59 & 14.15 & 95.69 & RW2 \\ 
  Morocco & ALL & 2006 & 38.80 & 34.50 & 43.10 & UN \\ 
  Morocco & ALL & 2007 & 32.24 & 27.56 & 37.27 & IHME \\ 
  Morocco & ALL & 2007 & 35.75 & 10.10 & 118.46 & RW2 \\ 
  Morocco & ALL & 2007 & 37.30 & 32.80 & 41.70 & UN \\ 
  Morocco & ALL & 2008 & 30.89 & 26.23 & 36.15 & IHME \\ 
  Morocco & ALL & 2008 & 34.29 & 7.04 & 151.41 & RW2 \\ 
  Morocco & ALL & 2008 & 35.80 & 31.20 & 40.50 & UN \\ 
  Morocco & ALL & 2009 & 29.56 & 24.77 & 34.99 & IHME \\ 
  Morocco & ALL & 2009 & 32.40 & 4.75 & 194.23 & RW2 \\ 
  Morocco & ALL & 2009 & 34.50 & 29.60 & 39.50 & UN \\ 
  Morocco & ALL & 2010 & 28.33 & 23.61 & 33.86 & IHME \\ 
  Morocco & ALL & 2010 & 30.75 & 3.19 & 247.72 & RW2 \\ 
  Morocco & ALL & 2010 & 33.10 & 27.90 & 38.50 & UN \\ 
  Morocco & ALL & 2011 & 27.17 & 22.42 & 32.73 & IHME \\ 
  Morocco & ALL & 2011 & 29.39 & 1.96 & 318.08 & RW2 \\ 
  Morocco & ALL & 2011 & 31.90 & 26.40 & 37.80 & UN \\ 
  Morocco & ALL & 2012 & 25.91 & 21.24 & 31.42 & IHME \\ 
  Morocco & ALL & 2012 & 27.88 & 1.17 & 407.63 & RW2 \\ 
  Morocco & ALL & 2012 & 30.70 & 25.00 & 37.10 & UN \\ 
  Morocco & ALL & 2013 & 24.68 & 20.15 & 30.02 & IHME \\ 
  Morocco & ALL & 2013 & 26.75 & 0.73 & 501.11 & RW2 \\ 
  Morocco & ALL & 2013 & 29.70 & 23.50 & 36.70 & UN \\ 
  Morocco & ALL & 2014 & 23.54 & 19.04 & 28.91 & IHME \\ 
  Morocco & ALL & 2014 & 25.38 & 0.44 & 602.14 & RW2 \\ 
  Morocco & ALL & 2014 & 28.60 & 22.00 & 36.50 & UN \\ 
  Morocco & ALL & 2015 & 22.44 & 18.00 & 27.92 & IHME \\ 
  Morocco & ALL & 2015 & 23.94 & 0.27 & 689.10 & RW2 \\ 
  Morocco & ALL & 2015 & 27.60 & 20.70 & 36.60 & UN \\ 
  Morocco & ALL & 2016 & 23.15 & 0.17 & 782.55 & RW2 \\ 
  Morocco & ALL & 2017 & 21.75 & 0.09 & 850.67 & RW2 \\ 
  Morocco & ALL & 2018 & 20.68 & 0.06 & 905.44 & RW2 \\ 
  Morocco & ALL & 2019 & 19.56 & 0.03 & 934.19 & RW2 \\ 
  Morocco & ALL & 80-84 & 114.95 & 121.53 & 108.69 & HT-Direct \\ 
  Morocco & ALL & 85-89 & 88.60 & 95.00 & 82.59 & HT-Direct \\ 
  Morocco & ALL & 90-94 & 71.10 & 77.68 & 65.03 & HT-Direct \\ 
  Morocco & ALL & 95-99 & 59.09 & 66.35 & 52.57 & HT-Direct \\ 
  Morocco & ALL & 00-04 & 44.49 & 51.51 & 38.39 & HT-Direct \\ 
  Morocco & ALL & 15-19 & 21.74 & 0.09 & 845.96 & RW2 \\ 
  Morocco & CENTRE & 1980 & 93.82 & 66.57 & 131.46 & RW2 \\ 
  Morocco & CENTRE & 1981 & 89.10 & 70.54 & 112.56 & RW2 \\ 
  Morocco & CENTRE & 1982 & 84.63 & 68.46 & 104.27 & RW2 \\ 
  Morocco & CENTRE & 1983 & 80.26 & 63.12 & 101.50 & RW2 \\ 
  Morocco & CENTRE & 1984 & 76.16 & 58.68 & 98.65 & RW2 \\ 
  Morocco & CENTRE & 1985 & 72.32 & 56.67 & 91.73 & RW2 \\ 
  Morocco & CENTRE & 1986 & 68.69 & 55.27 & 85.03 & RW2 \\ 
  Morocco & CENTRE & 1987 & 65.30 & 52.95 & 80.02 & RW2 \\ 
  Morocco & CENTRE & 1988 & 62.19 & 49.86 & 77.05 & RW2 \\ 
  Morocco & CENTRE & 1989 & 59.31 & 46.59 & 74.89 & RW2 \\ 
  Morocco & CENTRE & 1990 & 56.67 & 44.25 & 71.61 & RW2 \\ 
  Morocco & CENTRE & 1991 & 54.35 & 42.98 & 68.04 & RW2 \\ 
  Morocco & CENTRE & 1992 & 52.26 & 41.75 & 65.17 & RW2 \\ 
  Morocco & CENTRE & 1993 & 50.46 & 39.96 & 63.56 & RW2 \\ 
  Morocco & CENTRE & 1994 & 48.78 & 37.67 & 62.42 & RW2 \\ 
  Morocco & CENTRE & 1995 & 47.40 & 36.53 & 61.24 & RW2 \\ 
  Morocco & CENTRE & 1996 & 46.01 & 35.82 & 59.12 & RW2 \\ 
  Morocco & CENTRE & 1997 & 44.59 & 34.88 & 56.99 & RW2 \\ 
  Morocco & CENTRE & 1998 & 43.28 & 33.26 & 56.12 & RW2 \\ 
  Morocco & CENTRE & 1999 & 41.86 & 31.32 & 55.57 & RW2 \\ 
  Morocco & CENTRE & 2000 & 40.42 & 29.71 & 54.72 & RW2 \\ 
  Morocco & CENTRE & 2001 & 39.03 & 28.84 & 52.80 & RW2 \\ 
  Morocco & CENTRE & 2002 & 37.75 & 27.97 & 50.94 & RW2 \\ 
  Morocco & CENTRE & 2003 & 36.38 & 25.80 & 51.18 & RW2 \\ 
  Morocco & CENTRE & 2004 & 35.18 & 22.27 & 55.26 & RW2 \\ 
  Morocco & CENTRE & 2005 & 34.02 & 17.76 & 64.45 & RW2 \\ 
  Morocco & CENTRE & 2006 & 32.78 & 13.76 & 76.04 & RW2 \\ 
  Morocco & CENTRE & 2007 & 31.57 & 10.30 & 91.18 & RW2 \\ 
  Morocco & CENTRE & 2008 & 30.69 & 7.58 & 113.45 & RW2 \\ 
  Morocco & CENTRE & 2009 & 29.61 & 5.37 & 145.31 & RW2 \\ 
  Morocco & CENTRE & 2010 & 28.68 & 3.92 & 184.03 & RW2 \\ 
  Morocco & CENTRE & 2011 & 27.39 & 2.57 & 233.88 & RW2 \\ 
  Morocco & CENTRE & 2012 & 26.47 & 1.75 & 292.53 & RW2 \\ 
  Morocco & CENTRE & 2013 & 25.62 & 1.16 & 369.74 & RW2 \\ 
  Morocco & CENTRE & 2014 & 24.87 & 0.75 & 453.67 & RW2 \\ 
  Morocco & CENTRE & 2015 & 23.94 & 0.50 & 551.62 & RW2 \\ 
  Morocco & CENTRE & 2016 & 23.18 & 0.30 & 632.80 & RW2 \\ 
  Morocco & CENTRE & 2017 & 22.30 & 0.19 & 721.65 & RW2 \\ 
  Morocco & CENTRE & 2018 & 21.55 & 0.12 & 802.76 & RW2 \\ 
  Morocco & CENTRE & 2019 & 20.92 & 0.07 & 857.65 & RW2 \\ 
  Morocco & CENTRE-NORD & 1980 & 142.78 & 100.77 & 197.59 & RW2 \\ 
  Morocco & CENTRE-NORD & 1981 & 136.78 & 107.60 & 172.95 & RW2 \\ 
  Morocco & CENTRE-NORD & 1982 & 130.97 & 105.30 & 161.70 & RW2 \\ 
  Morocco & CENTRE-NORD & 1983 & 125.30 & 98.42 & 158.20 & RW2 \\ 
  Morocco & CENTRE-NORD & 1984 & 119.85 & 92.75 & 153.49 & RW2 \\ 
  Morocco & CENTRE-NORD & 1985 & 114.61 & 90.39 & 144.67 & RW2 \\ 
  Morocco & CENTRE-NORD & 1986 & 109.52 & 88.25 & 135.20 & RW2 \\ 
  Morocco & CENTRE-NORD & 1987 & 104.53 & 85.25 & 127.79 & RW2 \\ 
  Morocco & CENTRE-NORD & 1988 & 99.69 & 80.39 & 123.49 & RW2 \\ 
  Morocco & CENTRE-NORD & 1989 & 95.16 & 75.45 & 120.02 & RW2 \\ 
  Morocco & CENTRE-NORD & 1990 & 90.63 & 71.51 & 114.73 & RW2 \\ 
  Morocco & CENTRE-NORD & 1991 & 86.65 & 68.89 & 108.22 & RW2 \\ 
  Morocco & CENTRE-NORD & 1992 & 82.89 & 66.31 & 103.19 & RW2 \\ 
  Morocco & CENTRE-NORD & 1993 & 79.34 & 62.79 & 100.03 & RW2 \\ 
  Morocco & CENTRE-NORD & 1994 & 76.10 & 59.02 & 97.84 & RW2 \\ 
  Morocco & CENTRE-NORD & 1995 & 73.12 & 56.38 & 94.64 & RW2 \\ 
  Morocco & CENTRE-NORD & 1996 & 69.98 & 54.12 & 90.13 & RW2 \\ 
  Morocco & CENTRE-NORD & 1997 & 67.02 & 51.97 & 86.32 & RW2 \\ 
  Morocco & CENTRE-NORD & 1998 & 63.99 & 48.77 & 83.89 & RW2 \\ 
  Morocco & CENTRE-NORD & 1999 & 61.00 & 45.07 & 81.90 & RW2 \\ 
  Morocco & CENTRE-NORD & 2000 & 57.93 & 41.75 & 79.68 & RW2 \\ 
  Morocco & CENTRE-NORD & 2001 & 55.01 & 39.74 & 75.64 & RW2 \\ 
  Morocco & CENTRE-NORD & 2002 & 52.23 & 37.49 & 71.83 & RW2 \\ 
  Morocco & CENTRE-NORD & 2003 & 49.47 & 33.99 & 71.21 & RW2 \\ 
  Morocco & CENTRE-NORD & 2004 & 46.95 & 28.95 & 75.09 & RW2 \\ 
  Morocco & CENTRE-NORD & 2005 & 44.63 & 22.74 & 85.09 & RW2 \\ 
  Morocco & CENTRE-NORD & 2006 & 42.28 & 17.46 & 97.74 & RW2 \\ 
  Morocco & CENTRE-NORD & 2007 & 39.98 & 12.94 & 115.55 & RW2 \\ 
  Morocco & CENTRE-NORD & 2008 & 37.78 & 9.17 & 139.52 & RW2 \\ 
  Morocco & CENTRE-NORD & 2009 & 35.91 & 6.51 & 170.80 & RW2 \\ 
  Morocco & CENTRE-NORD & 2010 & 33.99 & 4.49 & 210.39 & RW2 \\ 
  Morocco & CENTRE-NORD & 2011 & 32.12 & 2.97 & 263.89 & RW2 \\ 
  Morocco & CENTRE-NORD & 2012 & 30.63 & 1.94 & 328.51 & RW2 \\ 
  Morocco & CENTRE-NORD & 2013 & 29.01 & 1.36 & 402.32 & RW2 \\ 
  Morocco & CENTRE-NORD & 2014 & 27.34 & 0.85 & 479.72 & RW2 \\ 
  Morocco & CENTRE-NORD & 2015 & 25.90 & 0.52 & 576.33 & RW2 \\ 
  Morocco & CENTRE-NORD & 2016 & 24.42 & 0.31 & 655.72 & RW2 \\ 
  Morocco & CENTRE-NORD & 2017 & 23.27 & 0.19 & 736.43 & RW2 \\ 
  Morocco & CENTRE-NORD & 2018 & 21.74 & 0.12 & 801.11 & RW2 \\ 
  Morocco & CENTRE-NORD & 2019 & 20.85 & 0.07 & 867.39 & RW2 \\ 
  Morocco & CENTRE-SUD & 1980 & 115.43 & 79.28 & 164.60 & RW2 \\ 
  Morocco & CENTRE-SUD & 1981 & 111.93 & 85.44 & 144.99 & RW2 \\ 
  Morocco & CENTRE-SUD & 1982 & 108.39 & 84.80 & 137.35 & RW2 \\ 
  Morocco & CENTRE-SUD & 1983 & 104.88 & 80.55 & 135.50 & RW2 \\ 
  Morocco & CENTRE-SUD & 1984 & 101.48 & 77.07 & 132.43 & RW2 \\ 
  Morocco & CENTRE-SUD & 1985 & 98.16 & 75.64 & 126.06 & RW2 \\ 
  Morocco & CENTRE-SUD & 1986 & 94.94 & 74.94 & 119.29 & RW2 \\ 
  Morocco & CENTRE-SUD & 1987 & 91.76 & 73.56 & 113.95 & RW2 \\ 
  Morocco & CENTRE-SUD & 1988 & 88.65 & 70.27 & 111.32 & RW2 \\ 
  Morocco & CENTRE-SUD & 1989 & 85.66 & 66.79 & 109.39 & RW2 \\ 
  Morocco & CENTRE-SUD & 1990 & 82.74 & 64.34 & 106.05 & RW2 \\ 
  Morocco & CENTRE-SUD & 1991 & 80.18 & 63.02 & 101.87 & RW2 \\ 
  Morocco & CENTRE-SUD & 1992 & 77.83 & 61.24 & 98.54 & RW2 \\ 
  Morocco & CENTRE-SUD & 1993 & 75.65 & 58.93 & 96.67 & RW2 \\ 
  Morocco & CENTRE-SUD & 1994 & 73.63 & 56.06 & 96.21 & RW2 \\ 
  Morocco & CENTRE-SUD & 1995 & 71.85 & 54.12 & 94.91 & RW2 \\ 
  Morocco & CENTRE-SUD & 1996 & 70.01 & 52.70 & 92.48 & RW2 \\ 
  Morocco & CENTRE-SUD & 1997 & 68.13 & 51.12 & 90.27 & RW2 \\ 
  Morocco & CENTRE-SUD & 1998 & 66.17 & 48.84 & 89.63 & RW2 \\ 
  Morocco & CENTRE-SUD & 1999 & 64.16 & 45.96 & 89.35 & RW2 \\ 
  Morocco & CENTRE-SUD & 2000 & 62.03 & 43.18 & 87.94 & RW2 \\ 
  Morocco & CENTRE-SUD & 2001 & 59.99 & 41.88 & 85.34 & RW2 \\ 
  Morocco & CENTRE-SUD & 2002 & 58.03 & 40.19 & 83.19 & RW2 \\ 
  Morocco & CENTRE-SUD & 2003 & 55.97 & 37.27 & 83.52 & RW2 \\ 
  Morocco & CENTRE-SUD & 2004 & 54.09 & 32.53 & 88.43 & RW2 \\ 
  Morocco & CENTRE-SUD & 2005 & 52.34 & 26.09 & 101.87 & RW2 \\ 
  Morocco & CENTRE-SUD & 2006 & 50.54 & 20.46 & 118.89 & RW2 \\ 
  Morocco & CENTRE-SUD & 2007 & 48.72 & 15.47 & 141.36 & RW2 \\ 
  Morocco & CENTRE-SUD & 2008 & 47.16 & 11.48 & 172.03 & RW2 \\ 
  Morocco & CENTRE-SUD & 2009 & 45.26 & 8.36 & 209.09 & RW2 \\ 
  Morocco & CENTRE-SUD & 2010 & 43.98 & 5.72 & 261.59 & RW2 \\ 
  Morocco & CENTRE-SUD & 2011 & 41.99 & 3.99 & 322.17 & RW2 \\ 
  Morocco & CENTRE-SUD & 2012 & 40.77 & 2.67 & 393.41 & RW2 \\ 
  Morocco & CENTRE-SUD & 2013 & 39.07 & 1.73 & 478.79 & RW2 \\ 
  Morocco & CENTRE-SUD & 2014 & 37.73 & 1.12 & 552.71 & RW2 \\ 
  Morocco & CENTRE-SUD & 2015 & 36.49 & 0.74 & 658.08 & RW2 \\ 
  Morocco & CENTRE-SUD & 2016 & 35.12 & 0.48 & 736.75 & RW2 \\ 
  Morocco & CENTRE-SUD & 2017 & 34.01 & 0.29 & 812.84 & RW2 \\ 
  Morocco & CENTRE-SUD & 2018 & 32.39 & 0.17 & 865.48 & RW2 \\ 
  Morocco & CENTRE-SUD & 2019 & 31.12 & 0.11 & 905.52 & RW2 \\ 
  Morocco & NORD-OUEST & 1980 & 137.03 & 98.68 & 187.52 & RW2 \\ 
  Morocco & NORD-OUEST & 1981 & 130.16 & 104.34 & 161.09 & RW2 \\ 
  Morocco & NORD-OUEST & 1982 & 123.61 & 101.15 & 149.59 & RW2 \\ 
  Morocco & NORD-OUEST & 1983 & 117.18 & 93.43 & 146.28 & RW2 \\ 
  Morocco & NORD-OUEST & 1984 & 111.16 & 86.85 & 141.18 & RW2 \\ 
  Morocco & NORD-OUEST & 1985 & 105.32 & 83.28 & 131.62 & RW2 \\ 
  Morocco & NORD-OUEST & 1986 & 99.62 & 80.91 & 122.39 & RW2 \\ 
  Morocco & NORD-OUEST & 1987 & 94.25 & 77.45 & 114.38 & RW2 \\ 
  Morocco & NORD-OUEST & 1988 & 89.07 & 72.22 & 109.52 & RW2 \\ 
  Morocco & NORD-OUEST & 1989 & 84.11 & 66.80 & 105.77 & RW2 \\ 
  Morocco & NORD-OUEST & 1990 & 79.46 & 62.66 & 100.31 & RW2 \\ 
  Morocco & NORD-OUEST & 1991 & 75.34 & 60.07 & 93.58 & RW2 \\ 
  Morocco & NORD-OUEST & 1992 & 71.44 & 57.34 & 88.46 & RW2 \\ 
  Morocco & NORD-OUEST & 1993 & 67.98 & 54.12 & 85.25 & RW2 \\ 
  Morocco & NORD-OUEST & 1994 & 64.89 & 50.49 & 82.98 & RW2 \\ 
  Morocco & NORD-OUEST & 1995 & 61.98 & 47.93 & 80.09 & RW2 \\ 
  Morocco & NORD-OUEST & 1996 & 59.19 & 46.22 & 75.41 & RW2 \\ 
  Morocco & NORD-OUEST & 1997 & 56.49 & 44.22 & 72.20 & RW2 \\ 
  Morocco & NORD-OUEST & 1998 & 53.91 & 41.49 & 69.91 & RW2 \\ 
  Morocco & NORD-OUEST & 1999 & 51.27 & 38.37 & 68.20 & RW2 \\ 
  Morocco & NORD-OUEST & 2000 & 48.69 & 35.62 & 66.37 & RW2 \\ 
  Morocco & NORD-OUEST & 2001 & 46.28 & 33.83 & 63.00 & RW2 \\ 
  Morocco & NORD-OUEST & 2002 & 43.91 & 32.06 & 59.99 & RW2 \\ 
  Morocco & NORD-OUEST & 2003 & 41.65 & 29.04 & 59.42 & RW2 \\ 
  Morocco & NORD-OUEST & 2004 & 39.53 & 24.56 & 62.52 & RW2 \\ 
  Morocco & NORD-OUEST & 2005 & 37.49 & 19.20 & 71.54 & RW2 \\ 
  Morocco & NORD-OUEST & 2006 & 35.47 & 14.83 & 82.94 & RW2 \\ 
  Morocco & NORD-OUEST & 2007 & 33.73 & 11.04 & 98.92 & RW2 \\ 
  Morocco & NORD-OUEST & 2008 & 32.09 & 7.89 & 120.75 & RW2 \\ 
  Morocco & NORD-OUEST & 2009 & 30.26 & 5.53 & 146.81 & RW2 \\ 
  Morocco & NORD-OUEST & 2010 & 28.79 & 3.79 & 184.71 & RW2 \\ 
  Morocco & NORD-OUEST & 2011 & 27.28 & 2.59 & 232.35 & RW2 \\ 
  Morocco & NORD-OUEST & 2012 & 25.84 & 1.69 & 290.10 & RW2 \\ 
  Morocco & NORD-OUEST & 2013 & 24.51 & 1.11 & 358.03 & RW2 \\ 
  Morocco & NORD-OUEST & 2014 & 23.03 & 0.67 & 432.67 & RW2 \\ 
  Morocco & NORD-OUEST & 2015 & 22.05 & 0.44 & 526.06 & RW2 \\ 
  Morocco & NORD-OUEST & 2016 & 20.76 & 0.26 & 611.96 & RW2 \\ 
  Morocco & NORD-OUEST & 2017 & 19.87 & 0.16 & 701.42 & RW2 \\ 
  Morocco & NORD-OUEST & 2018 & 18.60 & 0.10 & 772.42 & RW2 \\ 
  Morocco & NORD-OUEST & 2019 & 17.85 & 0.06 & 836.79 & RW2 \\ 
  Morocco & ORIENTAL & 1980 & 112.31 & 77.02 & 161.28 & RW2 \\ 
  Morocco & ORIENTAL & 1981 & 107.56 & 81.86 & 140.52 & RW2 \\ 
  Morocco & ORIENTAL & 1982 & 102.86 & 80.48 & 130.97 & RW2 \\ 
  Morocco & ORIENTAL & 1983 & 98.65 & 75.65 & 127.38 & RW2 \\ 
  Morocco & ORIENTAL & 1984 & 94.31 & 71.13 & 123.49 & RW2 \\ 
  Morocco & ORIENTAL & 1985 & 90.34 & 69.42 & 116.26 & RW2 \\ 
  Morocco & ORIENTAL & 1986 & 86.42 & 67.92 & 108.89 & RW2 \\ 
  Morocco & ORIENTAL & 1987 & 82.74 & 65.58 & 103.71 & RW2 \\ 
  Morocco & ORIENTAL & 1988 & 79.23 & 62.14 & 100.17 & RW2 \\ 
  Morocco & ORIENTAL & 1989 & 76.00 & 58.45 & 98.48 & RW2 \\ 
  Morocco & ORIENTAL & 1990 & 72.95 & 55.54 & 95.20 & RW2 \\ 
  Morocco & ORIENTAL & 1991 & 70.13 & 53.73 & 91.08 & RW2 \\ 
  Morocco & ORIENTAL & 1992 & 67.57 & 51.74 & 87.66 & RW2 \\ 
  Morocco & ORIENTAL & 1993 & 65.30 & 49.35 & 85.85 & RW2 \\ 
  Morocco & ORIENTAL & 1994 & 63.13 & 46.70 & 84.77 & RW2 \\ 
  Morocco & ORIENTAL & 1995 & 61.28 & 44.92 & 83.04 & RW2 \\ 
  Morocco & ORIENTAL & 1996 & 59.27 & 43.47 & 80.44 & RW2 \\ 
  Morocco & ORIENTAL & 1997 & 57.25 & 41.68 & 78.20 & RW2 \\ 
  Morocco & ORIENTAL & 1998 & 55.17 & 39.47 & 76.81 & RW2 \\ 
  Morocco & ORIENTAL & 1999 & 53.01 & 36.61 & 76.14 & RW2 \\ 
  Morocco & ORIENTAL & 2000 & 50.95 & 34.23 & 75.12 & RW2 \\ 
  Morocco & ORIENTAL & 2001 & 48.89 & 32.40 & 73.17 & RW2 \\ 
  Morocco & ORIENTAL & 2002 & 46.90 & 30.53 & 71.49 & RW2 \\ 
  Morocco & ORIENTAL & 2003 & 44.92 & 27.74 & 71.70 & RW2 \\ 
  Morocco & ORIENTAL & 2004 & 43.00 & 24.07 & 75.66 & RW2 \\ 
  Morocco & ORIENTAL & 2005 & 41.28 & 19.32 & 85.02 & RW2 \\ 
  Morocco & ORIENTAL & 2006 & 39.35 & 15.09 & 98.42 & RW2 \\ 
  Morocco & ORIENTAL & 2007 & 37.96 & 11.44 & 115.95 & RW2 \\ 
  Morocco & ORIENTAL & 2008 & 36.34 & 8.41 & 144.51 & RW2 \\ 
  Morocco & ORIENTAL & 2009 & 34.61 & 5.91 & 177.16 & RW2 \\ 
  Morocco & ORIENTAL & 2010 & 33.01 & 4.08 & 219.46 & RW2 \\ 
  Morocco & ORIENTAL & 2011 & 31.90 & 2.80 & 270.71 & RW2 \\ 
  Morocco & ORIENTAL & 2012 & 30.53 & 1.85 & 338.53 & RW2 \\ 
  Morocco & ORIENTAL & 2013 & 28.70 & 1.20 & 404.07 & RW2 \\ 
  Morocco & ORIENTAL & 2014 & 27.81 & 0.80 & 503.73 & RW2 \\ 
  Morocco & ORIENTAL & 2015 & 26.51 & 0.51 & 577.22 & RW2 \\ 
  Morocco & ORIENTAL & 2016 & 25.48 & 0.31 & 684.27 & RW2 \\ 
  Morocco & ORIENTAL & 2017 & 24.68 & 0.20 & 758.36 & RW2 \\ 
  Morocco & ORIENTAL & 2018 & 23.65 & 0.12 & 822.16 & RW2 \\ 
  Morocco & ORIENTAL & 2019 & 22.23 & 0.07 & 883.05 & RW2 \\ 
  Morocco & SUD & 1980 & 193.37 & 138.69 & 263.02 & RW2 \\ 
  Morocco & SUD & 1981 & 181.36 & 143.67 & 226.55 & RW2 \\ 
  Morocco & SUD & 1982 & 169.86 & 136.90 & 207.99 & RW2 \\ 
  Morocco & SUD & 1983 & 159.06 & 125.91 & 198.87 & RW2 \\ 
  Morocco & SUD & 1984 & 148.79 & 115.64 & 189.37 & RW2 \\ 
  Morocco & SUD & 1985 & 139.15 & 109.98 & 174.56 & RW2 \\ 
  Morocco & SUD & 1986 & 130.01 & 104.92 & 159.40 & RW2 \\ 
  Morocco & SUD & 1987 & 121.41 & 99.13 & 147.91 & RW2 \\ 
  Morocco & SUD & 1988 & 113.35 & 91.60 & 139.51 & RW2 \\ 
  Morocco & SUD & 1989 & 105.91 & 83.83 & 132.82 & RW2 \\ 
  Morocco & SUD & 1990 & 99.01 & 77.93 & 124.87 & RW2 \\ 
  Morocco & SUD & 1991 & 92.79 & 73.85 & 116.42 & RW2 \\ 
  Morocco & SUD & 1992 & 87.08 & 69.48 & 108.35 & RW2 \\ 
  Morocco & SUD & 1993 & 81.85 & 64.46 & 103.22 & RW2 \\ 
  Morocco & SUD & 1994 & 77.07 & 59.27 & 99.48 & RW2 \\ 
  Morocco & SUD & 1995 & 72.78 & 55.50 & 95.03 & RW2 \\ 
  Morocco & SUD & 1996 & 68.55 & 52.34 & 89.25 & RW2 \\ 
  Morocco & SUD & 1997 & 64.59 & 49.15 & 84.24 & RW2 \\ 
  Morocco & SUD & 1998 & 60.62 & 45.29 & 80.72 & RW2 \\ 
  Morocco & SUD & 1999 & 56.86 & 41.06 & 78.44 & RW2 \\ 
  Morocco & SUD & 2000 & 53.17 & 37.40 & 74.90 & RW2 \\ 
  Morocco & SUD & 2001 & 49.69 & 34.87 & 70.05 & RW2 \\ 
  Morocco & SUD & 2002 & 46.44 & 32.32 & 66.21 & RW2 \\ 
  Morocco & SUD & 2003 & 43.32 & 28.92 & 64.32 & RW2 \\ 
  Morocco & SUD & 2004 & 40.51 & 24.29 & 66.55 & RW2 \\ 
  Morocco & SUD & 2005 & 37.80 & 18.76 & 73.84 & RW2 \\ 
  Morocco & SUD & 2006 & 35.38 & 14.26 & 84.20 & RW2 \\ 
  Morocco & SUD & 2007 & 32.81 & 10.28 & 98.72 & RW2 \\ 
  Morocco & SUD & 2008 & 30.47 & 7.34 & 116.25 & RW2 \\ 
  Morocco & SUD & 2009 & 28.67 & 5.14 & 142.85 & RW2 \\ 
  Morocco & SUD & 2010 & 26.61 & 3.38 & 172.43 & RW2 \\ 
  Morocco & SUD & 2011 & 24.78 & 2.35 & 215.49 & RW2 \\ 
  Morocco & SUD & 2012 & 23.08 & 1.49 & 272.83 & RW2 \\ 
  Morocco & SUD & 2013 & 21.55 & 0.93 & 327.80 & RW2 \\ 
  Morocco & SUD & 2014 & 20.29 & 0.58 & 402.75 & RW2 \\ 
  Morocco & SUD & 2015 & 18.82 & 0.39 & 495.86 & RW2 \\ 
  Morocco & SUD & 2016 & 17.46 & 0.22 & 572.82 & RW2 \\ 
  Morocco & SUD & 2017 & 16.44 & 0.13 & 657.18 & RW2 \\ 
  Morocco & SUD & 2018 & 14.91 & 0.08 & 741.97 & RW2 \\ 
  Morocco & SUD & 2019 & 13.92 & 0.05 & 805.30 & RW2 \\ 
  Morocco & TENSIFT & 1980 & 159.98 & 113.86 & 219.23 & RW2 \\ 
  Morocco & TENSIFT & 1981 & 151.24 & 119.31 & 189.34 & RW2 \\ 
  Morocco & TENSIFT & 1982 & 142.92 & 115.65 & 175.71 & RW2 \\ 
  Morocco & TENSIFT & 1983 & 135.08 & 106.81 & 169.88 & RW2 \\ 
  Morocco & TENSIFT & 1984 & 127.68 & 98.55 & 163.15 & RW2 \\ 
  Morocco & TENSIFT & 1985 & 120.63 & 94.88 & 152.44 & RW2 \\ 
  Morocco & TENSIFT & 1986 & 114.06 & 91.44 & 141.19 & RW2 \\ 
  Morocco & TENSIFT & 1987 & 107.95 & 87.45 & 132.47 & RW2 \\ 
  Morocco & TENSIFT & 1988 & 102.26 & 81.68 & 126.98 & RW2 \\ 
  Morocco & TENSIFT & 1989 & 96.86 & 76.29 & 122.86 & RW2 \\ 
  Morocco & TENSIFT & 1990 & 92.03 & 72.09 & 116.46 & RW2 \\ 
  Morocco & TENSIFT & 1991 & 87.58 & 69.26 & 110.18 & RW2 \\ 
  Morocco & TENSIFT & 1992 & 83.57 & 66.54 & 104.47 & RW2 \\ 
  Morocco & TENSIFT & 1993 & 79.79 & 62.60 & 101.35 & RW2 \\ 
  Morocco & TENSIFT & 1994 & 76.29 & 58.66 & 98.70 & RW2 \\ 
  Morocco & TENSIFT & 1995 & 73.08 & 55.59 & 95.66 & RW2 \\ 
  Morocco & TENSIFT & 1996 & 69.94 & 53.67 & 91.38 & RW2 \\ 
  Morocco & TENSIFT & 1997 & 66.87 & 50.98 & 87.16 & RW2 \\ 
  Morocco & TENSIFT & 1998 & 63.67 & 47.89 & 84.44 & RW2 \\ 
  Morocco & TENSIFT & 1999 & 60.62 & 44.34 & 82.04 & RW2 \\ 
  Morocco & TENSIFT & 2000 & 57.55 & 41.26 & 79.65 & RW2 \\ 
  Morocco & TENSIFT & 2001 & 54.58 & 39.34 & 75.35 & RW2 \\ 
  Morocco & TENSIFT & 2002 & 51.81 & 37.33 & 71.64 & RW2 \\ 
  Morocco & TENSIFT & 2003 & 49.14 & 33.91 & 70.30 & RW2 \\ 
  Morocco & TENSIFT & 2004 & 46.52 & 28.73 & 74.01 & RW2 \\ 
  Morocco & TENSIFT & 2005 & 44.18 & 22.67 & 84.26 & RW2 \\ 
  Morocco & TENSIFT & 2006 & 41.78 & 17.25 & 96.67 & RW2 \\ 
  Morocco & TENSIFT & 2007 & 39.61 & 12.86 & 115.68 & RW2 \\ 
  Morocco & TENSIFT & 2008 & 37.31 & 9.24 & 139.96 & RW2 \\ 
  Morocco & TENSIFT & 2009 & 35.45 & 6.29 & 172.42 & RW2 \\ 
  Morocco & TENSIFT & 2010 & 33.47 & 4.40 & 212.43 & RW2 \\ 
  Morocco & TENSIFT & 2011 & 31.80 & 2.94 & 259.17 & RW2 \\ 
  Morocco & TENSIFT & 2012 & 30.09 & 2.02 & 329.46 & RW2 \\ 
  Morocco & TENSIFT & 2013 & 28.40 & 1.26 & 390.32 & RW2 \\ 
  Morocco & TENSIFT & 2014 & 26.81 & 0.83 & 480.87 & RW2 \\ 
  Morocco & TENSIFT & 2015 & 25.60 & 0.48 & 580.53 & RW2 \\ 
  Morocco & TENSIFT & 2016 & 23.92 & 0.32 & 654.29 & RW2 \\ 
  Morocco & TENSIFT & 2017 & 22.57 & 0.18 & 736.66 & RW2 \\ 
  Morocco & TENSIFT & 2018 & 21.49 & 0.11 & 798.58 & RW2 \\ 
  Morocco & TENSIFT & 2019 & 20.38 & 0.06 & 849.98 & RW2 \\ 
  Mozambique & ALL & 1980 & 227.57 & 219.89 & 235.60 & IHME \\ 
  Mozambique & ALL & 1980 & 264.22 & 203.01 & 335.64 & RW2 \\ 
  Mozambique & ALL & 1980 & 261.80 & 235.90 & 289.80 & UN \\ 
  Mozambique & ALL & 1981 & 252.26 & 223.19 & 292.18 & IHME \\ 
  Mozambique & ALL & 1981 & 262.08 & 217.50 & 311.59 & RW2 \\ 
  Mozambique & ALL & 1981 & 261.00 & 236.40 & 287.70 & UN \\ 
  Mozambique & ALL & 1982 & 251.16 & 222.83 & 290.16 & IHME \\ 
  Mozambique & ALL & 1982 & 259.99 & 220.25 & 304.10 & RW2 \\ 
  Mozambique & ALL & 1982 & 259.80 & 236.50 & 285.80 & UN \\ 
  Mozambique & ALL & 1983 & 250.51 & 223.07 & 286.46 & IHME \\ 
  Mozambique & ALL & 1983 & 257.60 & 214.81 & 305.64 & RW2 \\ 
  Mozambique & ALL & 1983 & 258.30 & 235.90 & 283.30 & UN \\ 
  Mozambique & ALL & 1984 & 254.05 & 225.32 & 291.23 & IHME \\ 
  Mozambique & ALL & 1984 & 255.66 & 209.00 & 308.06 & RW2 \\ 
  Mozambique & ALL & 1984 & 256.20 & 234.60 & 280.50 & UN \\ 
  Mozambique & ALL & 1985 & 253.46 & 225.24 & 290.83 & IHME \\ 
  Mozambique & ALL & 1985 & 253.09 & 209.98 & 301.75 & RW2 \\ 
  Mozambique & ALL & 1985 & 253.80 & 232.50 & 277.10 & UN \\ 
  Mozambique & ALL & 1986 & 228.91 & 219.91 & 239.58 & IHME \\ 
  Mozambique & ALL & 1986 & 250.87 & 210.66 & 295.75 & RW2 \\ 
  Mozambique & ALL & 1986 & 251.20 & 230.50 & 273.90 & UN \\ 
  Mozambique & ALL & 1987 & 227.57 & 218.92 & 237.53 & IHME \\ 
  Mozambique & ALL & 1987 & 248.69 & 210.57 & 291.62 & RW2 \\ 
  Mozambique & ALL & 1987 & 248.50 & 228.50 & 270.20 & UN \\ 
  Mozambique & ALL & 1988 & 220.79 & 214.88 & 227.24 & IHME \\ 
  Mozambique & ALL & 1988 & 246.21 & 206.81 & 289.99 & RW2 \\ 
  Mozambique & ALL & 1988 & 245.70 & 226.20 & 266.70 & UN \\ 
  Mozambique & ALL & 1989 & 217.72 & 211.96 & 224.06 & IHME \\ 
  Mozambique & ALL & 1989 & 243.51 & 202.71 & 289.16 & RW2 \\ 
  Mozambique & ALL & 1989 & 242.80 & 224.10 & 262.90 & UN \\ 
  Mozambique & ALL & 1990 & 214.22 & 208.68 & 220.13 & IHME \\ 
  Mozambique & ALL & 1990 & 240.70 & 200.93 & 286.30 & RW2 \\ 
  Mozambique & ALL & 1990 & 239.70 & 221.80 & 259.10 & UN \\ 
  Mozambique & ALL & 1991 & 210.03 & 204.58 & 215.93 & IHME \\ 
  Mozambique & ALL & 1991 & 237.03 & 199.56 & 278.81 & RW2 \\ 
  Mozambique & ALL & 1991 & 236.40 & 219.00 & 255.40 & UN \\ 
  Mozambique & ALL & 1992 & 205.87 & 200.37 & 211.37 & IHME \\ 
  Mozambique & ALL & 1992 & 232.63 & 196.41 & 272.81 & RW2 \\ 
  Mozambique & ALL & 1992 & 232.40 & 215.60 & 251.00 & UN \\ 
  Mozambique & ALL & 1993 & 201.13 & 195.86 & 206.50 & IHME \\ 
  Mozambique & ALL & 1993 & 227.46 & 191.10 & 268.29 & RW2 \\ 
  Mozambique & ALL & 1993 & 227.70 & 211.30 & 245.60 & UN \\ 
  Mozambique & ALL & 1994 & 196.05 & 190.79 & 201.32 & IHME \\ 
  Mozambique & ALL & 1994 & 221.34 & 183.99 & 264.43 & RW2 \\ 
  Mozambique & ALL & 1994 & 222.00 & 206.00 & 239.30 & UN \\ 
  Mozambique & ALL & 1995 & 190.73 & 185.64 & 195.76 & IHME \\ 
  Mozambique & ALL & 1995 & 214.58 & 178.67 & 255.31 & RW2 \\ 
  Mozambique & ALL & 1995 & 215.20 & 199.80 & 232.20 & UN \\ 
  Mozambique & ALL & 1996 & 184.30 & 179.41 & 189.45 & IHME \\ 
  Mozambique & ALL & 1996 & 206.95 & 173.91 & 244.93 & RW2 \\ 
  Mozambique & ALL & 1996 & 207.40 & 192.60 & 223.70 & UN \\ 
  Mozambique & ALL & 1997 & 177.59 & 172.75 & 182.57 & IHME \\ 
  Mozambique & ALL & 1997 & 198.75 & 167.81 & 233.96 & RW2 \\ 
  Mozambique & ALL & 1997 & 198.80 & 184.50 & 214.20 & UN \\ 
  Mozambique & ALL & 1998 & 170.60 & 165.88 & 175.59 & IHME \\ 
  Mozambique & ALL & 1998 & 190.21 & 159.45 & 226.37 & RW2 \\ 
  Mozambique & ALL & 1998 & 189.50 & 175.90 & 204.30 & UN \\ 
  Mozambique & ALL & 1999 & 163.72 & 158.98 & 168.63 & IHME \\ 
  Mozambique & ALL & 1999 & 181.34 & 149.95 & 217.55 & RW2 \\ 
  Mozambique & ALL & 1999 & 180.10 & 167.20 & 194.20 & UN \\ 
  Mozambique & ALL & 2000 & 157.46 & 152.70 & 162.31 & IHME \\ 
  Mozambique & ALL & 2000 & 172.21 & 141.77 & 207.00 & RW2 \\ 
  Mozambique & ALL & 2000 & 171.10 & 158.60 & 184.60 & UN \\ 
  Mozambique & ALL & 2001 & 150.85 & 146.21 & 155.53 & IHME \\ 
  Mozambique & ALL & 2001 & 163.49 & 135.42 & 195.68 & RW2 \\ 
  Mozambique & ALL & 2001 & 162.50 & 150.40 & 175.60 & UN \\ 
  Mozambique & ALL & 2002 & 144.27 & 139.73 & 148.79 & IHME \\ 
  Mozambique & ALL & 2002 & 155.06 & 129.13 & 185.10 & RW2 \\ 
  Mozambique & ALL & 2002 & 154.50 & 142.90 & 167.10 & UN \\ 
  Mozambique & ALL & 2003 & 138.07 & 133.48 & 142.69 & IHME \\ 
  Mozambique & ALL & 2003 & 147.14 & 121.94 & 176.55 & RW2 \\ 
  Mozambique & ALL & 2003 & 147.10 & 135.80 & 159.30 & UN \\ 
  Mozambique & ALL & 2004 & 132.19 & 127.58 & 136.82 & IHME \\ 
  Mozambique & ALL & 2004 & 139.46 & 113.67 & 169.66 & RW2 \\ 
  Mozambique & ALL & 2004 & 140.20 & 129.20 & 152.30 & UN \\ 
  Mozambique & ALL & 2005 & 126.33 & 121.44 & 131.12 & IHME \\ 
  Mozambique & ALL & 2005 & 132.52 & 107.47 & 162.53 & RW2 \\ 
  Mozambique & ALL & 2005 & 133.80 & 123.10 & 145.60 & UN \\ 
  Mozambique & ALL & 2006 & 120.54 & 115.40 & 125.56 & IHME \\ 
  Mozambique & ALL & 2006 & 125.79 & 102.74 & 153.09 & RW2 \\ 
  Mozambique & ALL & 2006 & 127.80 & 117.20 & 139.10 & UN \\ 
  Mozambique & ALL & 2007 & 114.94 & 109.50 & 120.22 & IHME \\ 
  Mozambique & ALL & 2007 & 119.51 & 98.07 & 144.86 & RW2 \\ 
  Mozambique & ALL & 2007 & 120.10 & 109.80 & 131.20 & UN \\ 
  Mozambique & ALL & 2008 & 109.78 & 104.03 & 115.48 & IHME \\ 
  Mozambique & ALL & 2008 & 113.71 & 92.46 & 139.16 & RW2 \\ 
  Mozambique & ALL & 2008 & 113.60 & 103.30 & 124.70 & UN \\ 
  Mozambique & ALL & 2009 & 104.80 & 98.91 & 111.03 & IHME \\ 
  Mozambique & ALL & 2009 & 108.00 & 86.26 & 134.76 & RW2 \\ 
  Mozambique & ALL & 2009 & 107.60 & 97.00 & 119.30 & UN \\ 
  Mozambique & ALL & 2010 & 99.66 & 93.41 & 106.32 & IHME \\ 
  Mozambique & ALL & 2010 & 102.69 & 80.62 & 130.68 & RW2 \\ 
  Mozambique & ALL & 2010 & 102.80 & 91.40 & 115.80 & UN \\ 
  Mozambique & ALL & 2011 & 95.10 & 88.58 & 102.07 & IHME \\ 
  Mozambique & ALL & 2011 & 97.70 & 76.67 & 123.79 & RW2 \\ 
  Mozambique & ALL & 2011 & 97.50 & 85.40 & 112.10 & UN \\ 
  Mozambique & ALL & 2012 & 90.59 & 83.75 & 98.26 & IHME \\ 
  Mozambique & ALL & 2012 & 92.87 & 72.51 & 118.17 & RW2 \\ 
  Mozambique & ALL & 2012 & 90.90 & 77.90 & 106.70 & UN \\ 
  Mozambique & ALL & 2013 & 86.26 & 79.20 & 94.50 & IHME \\ 
  Mozambique & ALL & 2013 & 88.36 & 66.17 & 116.85 & RW2 \\ 
  Mozambique & ALL & 2013 & 85.60 & 71.40 & 103.10 & UN \\ 
  Mozambique & ALL & 2014 & 81.97 & 74.54 & 90.72 & IHME \\ 
  Mozambique & ALL & 2014 & 83.94 & 56.94 & 121.97 & RW2 \\ 
  Mozambique & ALL & 2014 & 81.20 & 65.80 & 101.20 & UN \\ 
  Mozambique & ALL & 2015 & 77.93 & 70.40 & 87.26 & IHME \\ 
  Mozambique & ALL & 2015 & 79.61 & 45.70 & 135.32 & RW2 \\ 
  Mozambique & ALL & 2015 & 78.50 & 61.50 & 100.80 & UN \\ 
  Mozambique & ALL & 2016 & 75.80 & 36.44 & 152.03 & RW2 \\ 
  Mozambique & ALL & 2017 & 71.80 & 27.99 & 173.68 & RW2 \\ 
  Mozambique & ALL & 2018 & 68.14 & 21.26 & 202.47 & RW2 \\ 
  Mozambique & ALL & 2019 & 64.56 & 15.22 & 233.92 & RW2 \\ 
  Mozambique & ALL & 80-84 & 254.77 & 272.82 & 237.53 & HT-Direct \\ 
  Mozambique & ALL & 85-89 & 232.33 & 247.89 & 217.46 & HT-Direct \\ 
  Mozambique & ALL & 90-94 & 218.30 & 231.70 & 205.46 & HT-Direct \\ 
  Mozambique & ALL & 95-99 & 188.90 & 198.28 & 179.87 & HT-Direct \\ 
  Mozambique & ALL & 00-04 & 141.87 & 150.93 & 133.26 & HT-Direct \\ 
  Mozambique & ALL & 05-09 & 101.60 & 109.61 & 94.12 & HT-Direct \\ 
  Mozambique & ALL & 10-14 & 87.52 & 102.60 & 74.48 & HT-Direct \\ 
  Mozambique & ALL & 15-19 & 71.80 & 28.39 & 171.09 & RW2 \\ 
  Mozambique & CABO DELGADO & 1980 & 321.00 & 237.97 & 414.35 & RW2 \\ 
  Mozambique & CABO DELGADO & 1981 & 319.21 & 251.80 & 392.81 & RW2 \\ 
  Mozambique & CABO DELGADO & 1982 & 317.44 & 257.72 & 383.36 & RW2 \\ 
  Mozambique & CABO DELGADO & 1983 & 315.87 & 257.71 & 380.34 & RW2 \\ 
  Mozambique & CABO DELGADO & 1984 & 314.32 & 255.93 & 377.74 & RW2 \\ 
  Mozambique & CABO DELGADO & 1985 & 312.52 & 259.17 & 371.50 & RW2 \\ 
  Mozambique & CABO DELGADO & 1986 & 310.79 & 261.62 & 364.29 & RW2 \\ 
  Mozambique & CABO DELGADO & 1987 & 309.00 & 262.98 & 358.94 & RW2 \\ 
  Mozambique & CABO DELGADO & 1988 & 306.96 & 260.81 & 356.53 & RW2 \\ 
  Mozambique & CABO DELGADO & 1989 & 304.26 & 258.11 & 355.40 & RW2 \\ 
  Mozambique & CABO DELGADO & 1990 & 301.60 & 256.68 & 350.35 & RW2 \\ 
  Mozambique & CABO DELGADO & 1991 & 297.92 & 255.64 & 344.01 & RW2 \\ 
  Mozambique & CABO DELGADO & 1992 & 293.45 & 253.33 & 336.99 & RW2 \\ 
  Mozambique & CABO DELGADO & 1993 & 287.72 & 246.87 & 332.78 & RW2 \\ 
  Mozambique & CABO DELGADO & 1994 & 280.89 & 238.84 & 327.53 & RW2 \\ 
  Mozambique & CABO DELGADO & 1995 & 272.88 & 231.63 & 318.69 & RW2 \\ 
  Mozambique & CABO DELGADO & 1996 & 263.68 & 226.07 & 306.77 & RW2 \\ 
  Mozambique & CABO DELGADO & 1997 & 253.30 & 216.99 & 293.85 & RW2 \\ 
  Mozambique & CABO DELGADO & 1998 & 241.55 & 205.62 & 282.68 & RW2 \\ 
  Mozambique & CABO DELGADO & 1999 & 229.20 & 192.22 & 270.91 & RW2 \\ 
  Mozambique & CABO DELGADO & 2000 & 216.09 & 180.12 & 257.20 & RW2 \\ 
  Mozambique & CABO DELGADO & 2001 & 202.85 & 169.93 & 240.83 & RW2 \\ 
  Mozambique & CABO DELGADO & 2002 & 189.86 & 159.01 & 225.72 & RW2 \\ 
  Mozambique & CABO DELGADO & 2003 & 177.12 & 146.54 & 212.07 & RW2 \\ 
  Mozambique & CABO DELGADO & 2004 & 164.70 & 133.69 & 200.88 & RW2 \\ 
  Mozambique & CABO DELGADO & 2005 & 153.11 & 122.75 & 189.39 & RW2 \\ 
  Mozambique & CABO DELGADO & 2006 & 141.95 & 112.85 & 176.50 & RW2 \\ 
  Mozambique & CABO DELGADO & 2007 & 131.62 & 103.38 & 165.79 & RW2 \\ 
  Mozambique & CABO DELGADO & 2008 & 121.84 & 93.44 & 156.87 & RW2 \\ 
  Mozambique & CABO DELGADO & 2009 & 112.98 & 83.43 & 149.52 & RW2 \\ 
  Mozambique & CABO DELGADO & 2010 & 104.66 & 74.88 & 142.66 & RW2 \\ 
  Mozambique & CABO DELGADO & 2011 & 97.03 & 67.46 & 134.42 & RW2 \\ 
  Mozambique & CABO DELGADO & 2012 & 89.87 & 60.64 & 128.14 & RW2 \\ 
  Mozambique & CABO DELGADO & 2013 & 83.14 & 52.83 & 123.67 & RW2 \\ 
  Mozambique & CABO DELGADO & 2014 & 76.84 & 44.89 & 124.35 & RW2 \\ 
  Mozambique & CABO DELGADO & 2015 & 71.08 & 35.83 & 131.30 & RW2 \\ 
  Mozambique & CABO DELGADO & 2016 & 65.41 & 28.55 & 139.24 & RW2 \\ 
  Mozambique & CABO DELGADO & 2017 & 60.29 & 21.60 & 152.07 & RW2 \\ 
  Mozambique & CABO DELGADO & 2018 & 55.68 & 16.25 & 167.62 & RW2 \\ 
  Mozambique & CABO DELGADO & 2019 & 51.36 & 11.84 & 186.93 & RW2 \\ 
  Mozambique & GAZA & 1980 & 222.33 & 164.42 & 293.59 & RW2 \\ 
  Mozambique & GAZA & 1981 & 221.20 & 175.62 & 275.55 & RW2 \\ 
  Mozambique & GAZA & 1982 & 220.03 & 178.40 & 267.62 & RW2 \\ 
  Mozambique & GAZA & 1983 & 218.67 & 177.00 & 266.68 & RW2 \\ 
  Mozambique & GAZA & 1984 & 217.35 & 175.51 & 266.29 & RW2 \\ 
  Mozambique & GAZA & 1985 & 215.78 & 176.44 & 261.00 & RW2 \\ 
  Mozambique & GAZA & 1986 & 214.14 & 178.00 & 255.83 & RW2 \\ 
  Mozambique & GAZA & 1987 & 212.35 & 178.12 & 251.65 & RW2 \\ 
  Mozambique & GAZA & 1988 & 210.22 & 175.82 & 249.09 & RW2 \\ 
  Mozambique & GAZA & 1989 & 207.77 & 172.96 & 247.98 & RW2 \\ 
  Mozambique & GAZA & 1990 & 205.13 & 171.07 & 243.59 & RW2 \\ 
  Mozambique & GAZA & 1991 & 202.07 & 170.34 & 238.12 & RW2 \\ 
  Mozambique & GAZA & 1992 & 198.25 & 167.96 & 231.80 & RW2 \\ 
  Mozambique & GAZA & 1993 & 194.10 & 163.68 & 228.42 & RW2 \\ 
  Mozambique & GAZA & 1994 & 189.56 & 158.27 & 224.93 & RW2 \\ 
  Mozambique & GAZA & 1995 & 184.42 & 153.64 & 219.18 & RW2 \\ 
  Mozambique & GAZA & 1996 & 179.09 & 150.60 & 210.85 & RW2 \\ 
  Mozambique & GAZA & 1997 & 173.12 & 146.02 & 203.62 & RW2 \\ 
  Mozambique & GAZA & 1998 & 166.97 & 139.89 & 197.72 & RW2 \\ 
  Mozambique & GAZA & 1999 & 160.93 & 133.28 & 193.19 & RW2 \\ 
  Mozambique & GAZA & 2000 & 154.58 & 127.31 & 186.03 & RW2 \\ 
  Mozambique & GAZA & 2001 & 148.47 & 123.23 & 177.94 & RW2 \\ 
  Mozambique & GAZA & 2002 & 142.65 & 117.97 & 171.23 & RW2 \\ 
  Mozambique & GAZA & 2003 & 137.02 & 112.24 & 166.51 & RW2 \\ 
  Mozambique & GAZA & 2004 & 131.78 & 106.13 & 162.35 & RW2 \\ 
  Mozambique & GAZA & 2005 & 126.70 & 101.06 & 157.88 & RW2 \\ 
  Mozambique & GAZA & 2006 & 122.12 & 97.25 & 152.61 & RW2 \\ 
  Mozambique & GAZA & 2007 & 117.54 & 93.06 & 147.16 & RW2 \\ 
  Mozambique & GAZA & 2008 & 113.23 & 88.39 & 144.30 & RW2 \\ 
  Mozambique & GAZA & 2009 & 109.35 & 83.56 & 142.02 & RW2 \\ 
  Mozambique & GAZA & 2010 & 105.54 & 79.09 & 139.97 & RW2 \\ 
  Mozambique & GAZA & 2011 & 101.91 & 75.69 & 136.09 & RW2 \\ 
  Mozambique & GAZA & 2012 & 98.28 & 72.02 & 133.39 & RW2 \\ 
  Mozambique & GAZA & 2013 & 94.88 & 66.91 & 133.55 & RW2 \\ 
  Mozambique & GAZA & 2014 & 91.58 & 59.46 & 138.59 & RW2 \\ 
  Mozambique & GAZA & 2015 & 88.33 & 50.09 & 151.92 & RW2 \\ 
  Mozambique & GAZA & 2016 & 85.06 & 41.50 & 168.38 & RW2 \\ 
  Mozambique & GAZA & 2017 & 82.21 & 33.10 & 190.25 & RW2 \\ 
  Mozambique & GAZA & 2018 & 79.07 & 25.92 & 216.76 & RW2 \\ 
  Mozambique & GAZA & 2019 & 76.15 & 20.12 & 251.27 & RW2 \\ 
  Mozambique & INHAMBANE & 1980 & 244.28 & 181.04 & 320.79 & RW2 \\ 
  Mozambique & INHAMBANE & 1981 & 241.17 & 191.14 & 298.83 & RW2 \\ 
  Mozambique & INHAMBANE & 1982 & 237.63 & 192.38 & 289.12 & RW2 \\ 
  Mozambique & INHAMBANE & 1983 & 234.20 & 189.15 & 285.85 & RW2 \\ 
  Mozambique & INHAMBANE & 1984 & 230.73 & 185.65 & 282.61 & RW2 \\ 
  Mozambique & INHAMBANE & 1985 & 227.29 & 185.63 & 275.34 & RW2 \\ 
  Mozambique & INHAMBANE & 1986 & 223.61 & 185.16 & 266.65 & RW2 \\ 
  Mozambique & INHAMBANE & 1987 & 219.65 & 183.91 & 260.25 & RW2 \\ 
  Mozambique & INHAMBANE & 1988 & 215.39 & 180.16 & 255.00 & RW2 \\ 
  Mozambique & INHAMBANE & 1989 & 210.80 & 176.03 & 250.57 & RW2 \\ 
  Mozambique & INHAMBANE & 1990 & 206.05 & 172.50 & 244.28 & RW2 \\ 
  Mozambique & INHAMBANE & 1991 & 200.59 & 169.88 & 235.37 & RW2 \\ 
  Mozambique & INHAMBANE & 1992 & 194.50 & 165.60 & 227.42 & RW2 \\ 
  Mozambique & INHAMBANE & 1993 & 187.82 & 159.44 & 220.35 & RW2 \\ 
  Mozambique & INHAMBANE & 1994 & 180.58 & 152.12 & 213.61 & RW2 \\ 
  Mozambique & INHAMBANE & 1995 & 173.03 & 145.15 & 204.47 & RW2 \\ 
  Mozambique & INHAMBANE & 1996 & 164.80 & 140.05 & 193.48 & RW2 \\ 
  Mozambique & INHAMBANE & 1997 & 156.40 & 133.03 & 182.73 & RW2 \\ 
  Mozambique & INHAMBANE & 1998 & 147.62 & 124.38 & 174.07 & RW2 \\ 
  Mozambique & INHAMBANE & 1999 & 138.96 & 115.47 & 166.66 & RW2 \\ 
  Mozambique & INHAMBANE & 2000 & 130.28 & 106.67 & 157.26 & RW2 \\ 
  Mozambique & INHAMBANE & 2001 & 122.08 & 100.21 & 147.70 & RW2 \\ 
  Mozambique & INHAMBANE & 2002 & 114.21 & 93.43 & 138.92 & RW2 \\ 
  Mozambique & INHAMBANE & 2003 & 106.86 & 86.08 & 131.92 & RW2 \\ 
  Mozambique & INHAMBANE & 2004 & 100.02 & 78.62 & 126.04 & RW2 \\ 
  Mozambique & INHAMBANE & 2005 & 93.64 & 72.47 & 119.96 & RW2 \\ 
  Mozambique & INHAMBANE & 2006 & 87.79 & 67.22 & 113.39 & RW2 \\ 
  Mozambique & INHAMBANE & 2007 & 82.31 & 62.10 & 107.87 & RW2 \\ 
  Mozambique & INHAMBANE & 2008 & 77.23 & 56.97 & 103.17 & RW2 \\ 
  Mozambique & INHAMBANE & 2009 & 72.44 & 51.76 & 99.73 & RW2 \\ 
  Mozambique & INHAMBANE & 2010 & 68.10 & 47.07 & 96.27 & RW2 \\ 
  Mozambique & INHAMBANE & 2011 & 63.99 & 43.36 & 92.04 & RW2 \\ 
  Mozambique & INHAMBANE & 2012 & 60.07 & 39.58 & 88.62 & RW2 \\ 
  Mozambique & INHAMBANE & 2013 & 56.36 & 35.59 & 86.63 & RW2 \\ 
  Mozambique & INHAMBANE & 2014 & 52.89 & 30.93 & 87.61 & RW2 \\ 
  Mozambique & INHAMBANE & 2015 & 49.64 & 25.29 & 93.26 & RW2 \\ 
  Mozambique & INHAMBANE & 2016 & 46.65 & 20.55 & 100.87 & RW2 \\ 
  Mozambique & INHAMBANE & 2017 & 43.57 & 16.06 & 112.76 & RW2 \\ 
  Mozambique & INHAMBANE & 2018 & 40.90 & 12.34 & 126.52 & RW2 \\ 
  Mozambique & INHAMBANE & 2019 & 38.38 & 9.28 & 146.93 & RW2 \\ 
  Mozambique & MANICA & 1980 & 287.13 & 214.16 & 372.57 & RW2 \\ 
  Mozambique & MANICA & 1981 & 283.35 & 224.66 & 351.26 & RW2 \\ 
  Mozambique & MANICA & 1982 & 280.12 & 226.34 & 340.58 & RW2 \\ 
  Mozambique & MANICA & 1983 & 276.53 & 224.05 & 336.85 & RW2 \\ 
  Mozambique & MANICA & 1984 & 272.85 & 220.26 & 332.39 & RW2 \\ 
  Mozambique & MANICA & 1985 & 269.22 & 220.44 & 324.01 & RW2 \\ 
  Mozambique & MANICA & 1986 & 265.30 & 221.01 & 315.17 & RW2 \\ 
  Mozambique & MANICA & 1987 & 261.32 & 219.64 & 307.24 & RW2 \\ 
  Mozambique & MANICA & 1988 & 257.10 & 216.59 & 302.30 & RW2 \\ 
  Mozambique & MANICA & 1989 & 252.38 & 211.42 & 298.17 & RW2 \\ 
  Mozambique & MANICA & 1990 & 247.20 & 208.44 & 290.61 & RW2 \\ 
  Mozambique & MANICA & 1991 & 241.64 & 206.42 & 281.15 & RW2 \\ 
  Mozambique & MANICA & 1992 & 235.38 & 201.64 & 272.38 & RW2 \\ 
  Mozambique & MANICA & 1993 & 228.62 & 195.47 & 265.64 & RW2 \\ 
  Mozambique & MANICA & 1994 & 221.03 & 186.57 & 259.30 & RW2 \\ 
  Mozambique & MANICA & 1995 & 213.28 & 179.86 & 250.52 & RW2 \\ 
  Mozambique & MANICA & 1996 & 204.73 & 174.27 & 238.77 & RW2 \\ 
  Mozambique & MANICA & 1997 & 195.91 & 166.96 & 227.96 & RW2 \\ 
  Mozambique & MANICA & 1998 & 186.77 & 158.10 & 219.42 & RW2 \\ 
  Mozambique & MANICA & 1999 & 177.55 & 148.43 & 210.99 & RW2 \\ 
  Mozambique & MANICA & 2000 & 168.38 & 139.79 & 201.16 & RW2 \\ 
  Mozambique & MANICA & 2001 & 159.59 & 133.16 & 189.94 & RW2 \\ 
  Mozambique & MANICA & 2002 & 151.12 & 126.45 & 180.01 & RW2 \\ 
  Mozambique & MANICA & 2003 & 143.24 & 118.07 & 172.70 & RW2 \\ 
  Mozambique & MANICA & 2004 & 135.90 & 110.55 & 166.27 & RW2 \\ 
  Mozambique & MANICA & 2005 & 128.95 & 103.51 & 159.69 & RW2 \\ 
  Mozambique & MANICA & 2006 & 122.62 & 98.31 & 152.30 & RW2 \\ 
  Mozambique & MANICA & 2007 & 116.66 & 92.84 & 145.90 & RW2 \\ 
  Mozambique & MANICA & 2008 & 111.02 & 86.76 & 141.11 & RW2 \\ 
  Mozambique & MANICA & 2009 & 105.84 & 80.76 & 137.71 & RW2 \\ 
  Mozambique & MANICA & 2010 & 100.78 & 75.27 & 134.54 & RW2 \\ 
  Mozambique & MANICA & 2011 & 96.16 & 70.98 & 129.40 & RW2 \\ 
  Mozambique & MANICA & 2012 & 91.66 & 66.54 & 125.78 & RW2 \\ 
  Mozambique & MANICA & 2013 & 87.48 & 60.90 & 124.82 & RW2 \\ 
  Mozambique & MANICA & 2014 & 83.37 & 53.60 & 128.66 & RW2 \\ 
  Mozambique & MANICA & 2015 & 79.38 & 44.48 & 138.89 & RW2 \\ 
  Mozambique & MANICA & 2016 & 75.61 & 36.24 & 151.72 & RW2 \\ 
  Mozambique & MANICA & 2017 & 71.94 & 28.88 & 169.70 & RW2 \\ 
  Mozambique & MANICA & 2018 & 68.45 & 22.17 & 193.70 & RW2 \\ 
  Mozambique & MANICA & 2019 & 65.30 & 16.81 & 223.13 & RW2 \\ 
  Mozambique & MAPUTO CIDADE & 1980 & 96.47 & 65.00 & 141.88 & RW2 \\ 
  Mozambique & MAPUTO CIDADE & 1981 & 97.62 & 71.14 & 133.11 & RW2 \\ 
  Mozambique & MAPUTO CIDADE & 1982 & 98.61 & 74.56 & 130.02 & RW2 \\ 
  Mozambique & MAPUTO CIDADE & 1983 & 99.63 & 75.80 & 129.94 & RW2 \\ 
  Mozambique & MAPUTO CIDADE & 1984 & 100.61 & 76.60 & 131.01 & RW2 \\ 
  Mozambique & MAPUTO CIDADE & 1985 & 101.53 & 78.82 & 129.86 & RW2 \\ 
  Mozambique & MAPUTO CIDADE & 1986 & 102.30 & 81.37 & 128.55 & RW2 \\ 
  Mozambique & MAPUTO CIDADE & 1987 & 102.96 & 82.66 & 127.89 & RW2 \\ 
  Mozambique & MAPUTO CIDADE & 1988 & 103.59 & 82.82 & 128.81 & RW2 \\ 
  Mozambique & MAPUTO CIDADE & 1989 & 104.05 & 83.03 & 129.82 & RW2 \\ 
  Mozambique & MAPUTO CIDADE & 1990 & 104.35 & 83.64 & 129.47 & RW2 \\ 
  Mozambique & MAPUTO CIDADE & 1991 & 104.36 & 84.39 & 128.06 & RW2 \\ 
  Mozambique & MAPUTO CIDADE & 1992 & 104.15 & 84.67 & 127.18 & RW2 \\ 
  Mozambique & MAPUTO CIDADE & 1993 & 103.89 & 83.95 & 127.19 & RW2 \\ 
  Mozambique & MAPUTO CIDADE & 1994 & 103.43 & 82.98 & 127.76 & RW2 \\ 
  Mozambique & MAPUTO CIDADE & 1995 & 102.64 & 82.22 & 126.65 & RW2 \\ 
  Mozambique & MAPUTO CIDADE & 1996 & 101.86 & 82.65 & 124.26 & RW2 \\ 
  Mozambique & MAPUTO CIDADE & 1997 & 100.79 & 82.17 & 122.79 & RW2 \\ 
  Mozambique & MAPUTO CIDADE & 1998 & 99.77 & 81.05 & 122.03 & RW2 \\ 
  Mozambique & MAPUTO CIDADE & 1999 & 98.67 & 79.49 & 121.90 & RW2 \\ 
  Mozambique & MAPUTO CIDADE & 2000 & 97.30 & 78.20 & 120.41 & RW2 \\ 
  Mozambique & MAPUTO CIDADE & 2001 & 96.09 & 77.88 & 118.28 & RW2 \\ 
  Mozambique & MAPUTO CIDADE & 2002 & 94.80 & 76.86 & 116.59 & RW2 \\ 
  Mozambique & MAPUTO CIDADE & 2003 & 93.45 & 75.02 & 115.97 & RW2 \\ 
  Mozambique & MAPUTO CIDADE & 2004 & 92.20 & 73.07 & 115.95 & RW2 \\ 
  Mozambique & MAPUTO CIDADE & 2005 & 90.85 & 70.98 & 116.14 & RW2 \\ 
  Mozambique & MAPUTO CIDADE & 2006 & 89.56 & 69.85 & 114.41 & RW2 \\ 
  Mozambique & MAPUTO CIDADE & 2007 & 88.22 & 68.26 & 113.66 & RW2 \\ 
  Mozambique & MAPUTO CIDADE & 2008 & 86.94 & 65.70 & 114.38 & RW2 \\ 
  Mozambique & MAPUTO CIDADE & 2009 & 85.69 & 63.04 & 115.83 & RW2 \\ 
  Mozambique & MAPUTO CIDADE & 2010 & 84.41 & 60.10 & 118.11 & RW2 \\ 
  Mozambique & MAPUTO CIDADE & 2011 & 83.30 & 57.59 & 119.15 & RW2 \\ 
  Mozambique & MAPUTO CIDADE & 2012 & 82.18 & 54.96 & 120.82 & RW2 \\ 
  Mozambique & MAPUTO CIDADE & 2013 & 81.01 & 51.17 & 125.09 & RW2 \\ 
  Mozambique & MAPUTO CIDADE & 2014 & 79.84 & 46.00 & 134.08 & RW2 \\ 
  Mozambique & MAPUTO CIDADE & 2015 & 78.69 & 39.57 & 149.94 & RW2 \\ 
  Mozambique & MAPUTO CIDADE & 2016 & 77.83 & 32.91 & 169.51 & RW2 \\ 
  Mozambique & MAPUTO CIDADE & 2017 & 76.32 & 27.20 & 195.61 & RW2 \\ 
  Mozambique & MAPUTO CIDADE & 2018 & 75.19 & 21.84 & 227.27 & RW2 \\ 
  Mozambique & MAPUTO CIDADE & 2019 & 74.47 & 16.98 & 267.22 & RW2 \\ 
  Mozambique & MAPUTO PROVINCIA & 1980 & 129.52 & 87.01 & 189.91 & RW2 \\ 
  Mozambique & MAPUTO PROVINCIA & 1981 & 129.96 & 93.42 & 178.91 & RW2 \\ 
  Mozambique & MAPUTO PROVINCIA & 1982 & 130.42 & 97.24 & 174.09 & RW2 \\ 
  Mozambique & MAPUTO PROVINCIA & 1983 & 130.83 & 98.32 & 172.42 & RW2 \\ 
  Mozambique & MAPUTO PROVINCIA & 1984 & 131.04 & 98.96 & 171.99 & RW2 \\ 
  Mozambique & MAPUTO PROVINCIA & 1985 & 131.38 & 101.29 & 168.95 & RW2 \\ 
  Mozambique & MAPUTO PROVINCIA & 1986 & 131.60 & 103.67 & 165.53 & RW2 \\ 
  Mozambique & MAPUTO PROVINCIA & 1987 & 131.68 & 105.05 & 163.34 & RW2 \\ 
  Mozambique & MAPUTO PROVINCIA & 1988 & 131.54 & 105.44 & 162.60 & RW2 \\ 
  Mozambique & MAPUTO PROVINCIA & 1989 & 131.18 & 104.98 & 162.78 & RW2 \\ 
  Mozambique & MAPUTO PROVINCIA & 1990 & 130.68 & 105.33 & 161.22 & RW2 \\ 
  Mozambique & MAPUTO PROVINCIA & 1991 & 129.92 & 105.88 & 158.31 & RW2 \\ 
  Mozambique & MAPUTO PROVINCIA & 1992 & 129.05 & 105.52 & 156.13 & RW2 \\ 
  Mozambique & MAPUTO PROVINCIA & 1993 & 127.69 & 103.98 & 155.45 & RW2 \\ 
  Mozambique & MAPUTO PROVINCIA & 1994 & 126.31 & 102.16 & 154.70 & RW2 \\ 
  Mozambique & MAPUTO PROVINCIA & 1995 & 124.61 & 100.78 & 152.76 & RW2 \\ 
  Mozambique & MAPUTO PROVINCIA & 1996 & 122.83 & 99.98 & 149.17 & RW2 \\ 
  Mozambique & MAPUTO PROVINCIA & 1997 & 120.84 & 98.92 & 145.83 & RW2 \\ 
  Mozambique & MAPUTO PROVINCIA & 1998 & 118.67 & 96.72 & 144.27 & RW2 \\ 
  Mozambique & MAPUTO PROVINCIA & 1999 & 116.47 & 94.42 & 142.76 & RW2 \\ 
  Mozambique & MAPUTO PROVINCIA & 2000 & 114.21 & 92.07 & 140.20 & RW2 \\ 
  Mozambique & MAPUTO PROVINCIA & 2001 & 111.97 & 91.34 & 136.68 & RW2 \\ 
  Mozambique & MAPUTO PROVINCIA & 2002 & 109.82 & 89.81 & 133.76 & RW2 \\ 
  Mozambique & MAPUTO PROVINCIA & 2003 & 107.61 & 87.22 & 132.30 & RW2 \\ 
  Mozambique & MAPUTO PROVINCIA & 2004 & 105.53 & 84.08 & 131.93 & RW2 \\ 
  Mozambique & MAPUTO PROVINCIA & 2005 & 103.52 & 81.67 & 130.63 & RW2 \\ 
  Mozambique & MAPUTO PROVINCIA & 2006 & 101.55 & 80.13 & 128.53 & RW2 \\ 
  Mozambique & MAPUTO PROVINCIA & 2007 & 99.58 & 77.92 & 126.47 & RW2 \\ 
  Mozambique & MAPUTO PROVINCIA & 2008 & 97.75 & 75.16 & 126.67 & RW2 \\ 
  Mozambique & MAPUTO PROVINCIA & 2009 & 95.94 & 71.95 & 127.32 & RW2 \\ 
  Mozambique & MAPUTO PROVINCIA & 2010 & 94.28 & 68.66 & 129.07 & RW2 \\ 
  Mozambique & MAPUTO PROVINCIA & 2011 & 92.49 & 66.10 & 128.66 & RW2 \\ 
  Mozambique & MAPUTO PROVINCIA & 2012 & 90.91 & 63.19 & 129.58 & RW2 \\ 
  Mozambique & MAPUTO PROVINCIA & 2013 & 89.33 & 59.01 & 133.55 & RW2 \\ 
  Mozambique & MAPUTO PROVINCIA & 2014 & 87.82 & 53.28 & 141.96 & RW2 \\ 
  Mozambique & MAPUTO PROVINCIA & 2015 & 86.04 & 45.41 & 156.79 & RW2 \\ 
  Mozambique & MAPUTO PROVINCIA & 2016 & 84.50 & 38.37 & 176.47 & RW2 \\ 
  Mozambique & MAPUTO PROVINCIA & 2017 & 83.12 & 31.60 & 202.25 & RW2 \\ 
  Mozambique & MAPUTO PROVINCIA & 2018 & 81.66 & 25.53 & 235.59 & RW2 \\ 
  Mozambique & MAPUTO PROVINCIA & 2019 & 80.12 & 19.58 & 277.16 & RW2 \\ 
  Mozambique & NAMPULA & 1980 & 318.48 & 245.82 & 400.76 & RW2 \\ 
  Mozambique & NAMPULA & 1981 & 314.77 & 258.67 & 376.65 & RW2 \\ 
  Mozambique & NAMPULA & 1982 & 310.92 & 259.58 & 366.16 & RW2 \\ 
  Mozambique & NAMPULA & 1983 & 307.30 & 255.76 & 363.84 & RW2 \\ 
  Mozambique & NAMPULA & 1984 & 303.60 & 250.75 & 361.92 & RW2 \\ 
  Mozambique & NAMPULA & 1985 & 299.96 & 250.32 & 354.57 & RW2 \\ 
  Mozambique & NAMPULA & 1986 & 296.20 & 249.88 & 346.00 & RW2 \\ 
  Mozambique & NAMPULA & 1987 & 292.33 & 248.51 & 340.00 & RW2 \\ 
  Mozambique & NAMPULA & 1988 & 288.20 & 244.62 & 335.71 & RW2 \\ 
  Mozambique & NAMPULA & 1989 & 283.84 & 239.27 & 332.47 & RW2 \\ 
  Mozambique & NAMPULA & 1990 & 279.16 & 236.07 & 326.85 & RW2 \\ 
  Mozambique & NAMPULA & 1991 & 273.73 & 233.75 & 318.78 & RW2 \\ 
  Mozambique & NAMPULA & 1992 & 267.36 & 229.33 & 309.12 & RW2 \\ 
  Mozambique & NAMPULA & 1993 & 260.00 & 222.13 & 301.99 & RW2 \\ 
  Mozambique & NAMPULA & 1994 & 251.66 & 212.87 & 295.16 & RW2 \\ 
  Mozambique & NAMPULA & 1995 & 242.46 & 205.42 & 284.11 & RW2 \\ 
  Mozambique & NAMPULA & 1996 & 232.02 & 198.13 & 269.81 & RW2 \\ 
  Mozambique & NAMPULA & 1997 & 220.80 & 189.31 & 255.81 & RW2 \\ 
  Mozambique & NAMPULA & 1998 & 208.57 & 177.58 & 243.69 & RW2 \\ 
  Mozambique & NAMPULA & 1999 & 195.99 & 164.04 & 233.04 & RW2 \\ 
  Mozambique & NAMPULA & 2000 & 183.03 & 151.95 & 218.53 & RW2 \\ 
  Mozambique & NAMPULA & 2001 & 170.30 & 141.87 & 202.62 & RW2 \\ 
  Mozambique & NAMPULA & 2002 & 158.02 & 131.31 & 188.88 & RW2 \\ 
  Mozambique & NAMPULA & 2003 & 146.21 & 120.04 & 176.75 & RW2 \\ 
  Mozambique & NAMPULA & 2004 & 135.24 & 108.80 & 166.56 & RW2 \\ 
  Mozambique & NAMPULA & 2005 & 124.88 & 98.77 & 155.96 & RW2 \\ 
  Mozambique & NAMPULA & 2006 & 115.37 & 90.66 & 145.05 & RW2 \\ 
  Mozambique & NAMPULA & 2007 & 106.34 & 82.49 & 135.30 & RW2 \\ 
  Mozambique & NAMPULA & 2008 & 98.09 & 74.68 & 126.92 & RW2 \\ 
  Mozambique & NAMPULA & 2009 & 90.75 & 67.24 & 120.33 & RW2 \\ 
  Mozambique & NAMPULA & 2010 & 83.79 & 60.21 & 113.74 & RW2 \\ 
  Mozambique & NAMPULA & 2011 & 77.40 & 54.95 & 106.67 & RW2 \\ 
  Mozambique & NAMPULA & 2012 & 71.49 & 49.40 & 100.70 & RW2 \\ 
  Mozambique & NAMPULA & 2013 & 66.01 & 43.42 & 96.64 & RW2 \\ 
  Mozambique & NAMPULA & 2014 & 61.01 & 36.85 & 96.54 & RW2 \\ 
  Mozambique & NAMPULA & 2015 & 56.17 & 29.90 & 101.80 & RW2 \\ 
  Mozambique & NAMPULA & 2016 & 51.71 & 23.38 & 108.21 & RW2 \\ 
  Mozambique & NAMPULA & 2017 & 47.74 & 17.96 & 118.15 & RW2 \\ 
  Mozambique & NAMPULA & 2018 & 43.67 & 13.47 & 132.12 & RW2 \\ 
  Mozambique & NAMPULA & 2019 & 40.21 & 9.82 & 148.65 & RW2 \\ 
  Mozambique & NIASSA & 1980 & 304.94 & 222.48 & 403.14 & RW2 \\ 
  Mozambique & NIASSA & 1981 & 300.40 & 232.88 & 378.38 & RW2 \\ 
  Mozambique & NIASSA & 1982 & 295.63 & 234.26 & 365.50 & RW2 \\ 
  Mozambique & NIASSA & 1983 & 290.71 & 230.78 & 359.07 & RW2 \\ 
  Mozambique & NIASSA & 1984 & 285.89 & 227.49 & 352.22 & RW2 \\ 
  Mozambique & NIASSA & 1985 & 281.06 & 226.30 & 342.46 & RW2 \\ 
  Mozambique & NIASSA & 1986 & 276.42 & 225.96 & 332.50 & RW2 \\ 
  Mozambique & NIASSA & 1987 & 271.74 & 224.91 & 323.69 & RW2 \\ 
  Mozambique & NIASSA & 1988 & 266.93 & 220.87 & 317.92 & RW2 \\ 
  Mozambique & NIASSA & 1989 & 261.95 & 216.26 & 312.80 & RW2 \\ 
  Mozambique & NIASSA & 1990 & 256.69 & 213.23 & 305.68 & RW2 \\ 
  Mozambique & NIASSA & 1991 & 251.08 & 210.79 & 296.26 & RW2 \\ 
  Mozambique & NIASSA & 1992 & 244.83 & 206.54 & 287.12 & RW2 \\ 
  Mozambique & NIASSA & 1993 & 237.76 & 200.55 & 279.08 & RW2 \\ 
  Mozambique & NIASSA & 1994 & 229.94 & 192.41 & 272.16 & RW2 \\ 
  Mozambique & NIASSA & 1995 & 221.44 & 185.61 & 261.60 & RW2 \\ 
  Mozambique & NIASSA & 1996 & 212.17 & 179.27 & 248.89 & RW2 \\ 
  Mozambique & NIASSA & 1997 & 202.16 & 171.64 & 236.29 & RW2 \\ 
  Mozambique & NIASSA & 1998 & 191.56 & 161.71 & 226.13 & RW2 \\ 
  Mozambique & NIASSA & 1999 & 180.62 & 150.17 & 216.24 & RW2 \\ 
  Mozambique & NIASSA & 2000 & 169.37 & 139.38 & 203.51 & RW2 \\ 
  Mozambique & NIASSA & 2001 & 158.45 & 131.25 & 190.04 & RW2 \\ 
  Mozambique & NIASSA & 2002 & 147.91 & 122.34 & 177.82 & RW2 \\ 
  Mozambique & NIASSA & 2003 & 137.66 & 112.75 & 167.31 & RW2 \\ 
  Mozambique & NIASSA & 2004 & 128.09 & 103.23 & 157.85 & RW2 \\ 
  Mozambique & NIASSA & 2005 & 119.18 & 94.80 & 148.83 & RW2 \\ 
  Mozambique & NIASSA & 2006 & 110.77 & 87.90 & 138.73 & RW2 \\ 
  Mozambique & NIASSA & 2007 & 102.94 & 81.09 & 129.72 & RW2 \\ 
  Mozambique & NIASSA & 2008 & 95.78 & 74.10 & 122.79 & RW2 \\ 
  Mozambique & NIASSA & 2009 & 89.00 & 67.24 & 116.93 & RW2 \\ 
  Mozambique & NIASSA & 2010 & 82.96 & 60.58 & 112.20 & RW2 \\ 
  Mozambique & NIASSA & 2011 & 77.06 & 55.45 & 106.19 & RW2 \\ 
  Mozambique & NIASSA & 2012 & 71.78 & 50.27 & 101.16 & RW2 \\ 
  Mozambique & NIASSA & 2013 & 66.70 & 44.41 & 98.72 & RW2 \\ 
  Mozambique & NIASSA & 2014 & 62.04 & 37.87 & 99.12 & RW2 \\ 
  Mozambique & NIASSA & 2015 & 57.65 & 30.73 & 105.77 & RW2 \\ 
  Mozambique & NIASSA & 2016 & 53.53 & 24.47 & 113.79 & RW2 \\ 
  Mozambique & NIASSA & 2017 & 49.76 & 18.84 & 126.09 & RW2 \\ 
  Mozambique & NIASSA & 2018 & 46.04 & 13.99 & 141.09 & RW2 \\ 
  Mozambique & NIASSA & 2019 & 42.68 & 10.45 & 159.82 & RW2 \\ 
  Mozambique & SOFALA & 1980 & 342.12 & 267.31 & 425.64 & RW2 \\ 
  Mozambique & SOFALA & 1981 & 335.46 & 278.17 & 398.00 & RW2 \\ 
  Mozambique & SOFALA & 1982 & 329.15 & 277.75 & 384.93 & RW2 \\ 
  Mozambique & SOFALA & 1983 & 322.34 & 270.67 & 379.16 & RW2 \\ 
  Mozambique & SOFALA & 1984 & 315.78 & 264.08 & 372.87 & RW2 \\ 
  Mozambique & SOFALA & 1985 & 309.16 & 260.96 & 361.94 & RW2 \\ 
  Mozambique & SOFALA & 1986 & 302.33 & 258.29 & 350.06 & RW2 \\ 
  Mozambique & SOFALA & 1987 & 295.53 & 254.89 & 340.58 & RW2 \\ 
  Mozambique & SOFALA & 1988 & 288.31 & 247.23 & 333.54 & RW2 \\ 
  Mozambique & SOFALA & 1989 & 280.76 & 238.65 & 326.88 & RW2 \\ 
  Mozambique & SOFALA & 1990 & 272.98 & 232.91 & 317.19 & RW2 \\ 
  Mozambique & SOFALA & 1991 & 264.56 & 227.46 & 305.77 & RW2 \\ 
  Mozambique & SOFALA & 1992 & 255.52 & 220.69 & 294.14 & RW2 \\ 
  Mozambique & SOFALA & 1993 & 245.97 & 210.43 & 285.17 & RW2 \\ 
  Mozambique & SOFALA & 1994 & 235.87 & 199.88 & 275.66 & RW2 \\ 
  Mozambique & SOFALA & 1995 & 225.33 & 190.55 & 264.08 & RW2 \\ 
  Mozambique & SOFALA & 1996 & 214.44 & 182.96 & 249.97 & RW2 \\ 
  Mozambique & SOFALA & 1997 & 203.11 & 173.74 & 236.23 & RW2 \\ 
  Mozambique & SOFALA & 1998 & 191.81 & 162.68 & 224.66 & RW2 \\ 
  Mozambique & SOFALA & 1999 & 180.50 & 151.02 & 214.40 & RW2 \\ 
  Mozambique & SOFALA & 2000 & 169.10 & 140.73 & 201.70 & RW2 \\ 
  Mozambique & SOFALA & 2001 & 158.34 & 132.34 & 188.17 & RW2 \\ 
  Mozambique & SOFALA & 2002 & 148.22 & 124.05 & 176.37 & RW2 \\ 
  Mozambique & SOFALA & 2003 & 138.55 & 114.49 & 166.32 & RW2 \\ 
  Mozambique & SOFALA & 2004 & 129.65 & 105.35 & 158.44 & RW2 \\ 
  Mozambique & SOFALA & 2005 & 121.34 & 97.42 & 150.24 & RW2 \\ 
  Mozambique & SOFALA & 2006 & 113.78 & 91.08 & 140.95 & RW2 \\ 
  Mozambique & SOFALA & 2007 & 106.50 & 84.97 & 132.96 & RW2 \\ 
  Mozambique & SOFALA & 2008 & 99.93 & 78.41 & 126.43 & RW2 \\ 
  Mozambique & SOFALA & 2009 & 93.64 & 71.97 & 121.21 & RW2 \\ 
  Mozambique & SOFALA & 2010 & 87.97 & 66.28 & 116.17 & RW2 \\ 
  Mozambique & SOFALA & 2011 & 82.45 & 61.88 & 109.17 & RW2 \\ 
  Mozambique & SOFALA & 2012 & 77.33 & 57.61 & 104.09 & RW2 \\ 
  Mozambique & SOFALA & 2013 & 72.56 & 51.91 & 101.14 & RW2 \\ 
  Mozambique & SOFALA & 2014 & 68.03 & 44.76 & 102.41 & RW2 \\ 
  Mozambique & SOFALA & 2015 & 63.77 & 36.58 & 109.17 & RW2 \\ 
  Mozambique & SOFALA & 2016 & 59.76 & 29.08 & 118.94 & RW2 \\ 
  Mozambique & SOFALA & 2017 & 55.93 & 22.72 & 132.17 & RW2 \\ 
  Mozambique & SOFALA & 2018 & 52.21 & 17.48 & 148.03 & RW2 \\ 
  Mozambique & SOFALA & 2019 & 48.97 & 12.95 & 169.78 & RW2 \\ 
  Mozambique & TETE & 1980 & 304.61 & 230.52 & 389.97 & RW2 \\ 
  Mozambique & TETE & 1981 & 303.30 & 244.46 & 369.08 & RW2 \\ 
  Mozambique & TETE & 1982 & 301.92 & 248.46 & 361.07 & RW2 \\ 
  Mozambique & TETE & 1983 & 300.27 & 247.47 & 359.10 & RW2 \\ 
  Mozambique & TETE & 1984 & 298.72 & 245.60 & 358.23 & RW2 \\ 
  Mozambique & TETE & 1985 & 297.29 & 247.42 & 351.40 & RW2 \\ 
  Mozambique & TETE & 1986 & 295.26 & 250.14 & 344.40 & RW2 \\ 
  Mozambique & TETE & 1987 & 293.11 & 250.66 & 339.25 & RW2 \\ 
  Mozambique & TETE & 1988 & 290.78 & 248.49 & 336.83 & RW2 \\ 
  Mozambique & TETE & 1989 & 287.95 & 244.64 & 334.86 & RW2 \\ 
  Mozambique & TETE & 1990 & 284.58 & 243.69 & 330.53 & RW2 \\ 
  Mozambique & TETE & 1991 & 280.65 & 242.00 & 323.07 & RW2 \\ 
  Mozambique & TETE & 1992 & 275.74 & 239.16 & 315.64 & RW2 \\ 
  Mozambique & TETE & 1993 & 270.12 & 233.38 & 310.54 & RW2 \\ 
  Mozambique & TETE & 1994 & 263.52 & 225.20 & 305.62 & RW2 \\ 
  Mozambique & TETE & 1995 & 256.16 & 219.26 & 297.18 & RW2 \\ 
  Mozambique & TETE & 1996 & 247.97 & 213.70 & 285.80 & RW2 \\ 
  Mozambique & TETE & 1997 & 239.16 & 206.82 & 274.90 & RW2 \\ 
  Mozambique & TETE & 1998 & 229.77 & 196.77 & 266.04 & RW2 \\ 
  Mozambique & TETE & 1999 & 219.88 & 186.03 & 257.51 & RW2 \\ 
  Mozambique & TETE & 2000 & 209.87 & 176.93 & 246.65 & RW2 \\ 
  Mozambique & TETE & 2001 & 200.05 & 169.23 & 234.55 & RW2 \\ 
  Mozambique & TETE & 2002 & 190.72 & 161.81 & 223.55 & RW2 \\ 
  Mozambique & TETE & 2003 & 181.58 & 152.29 & 214.74 & RW2 \\ 
  Mozambique & TETE & 2004 & 173.06 & 143.18 & 207.68 & RW2 \\ 
  Mozambique & TETE & 2005 & 164.99 & 135.11 & 199.93 & RW2 \\ 
  Mozambique & TETE & 2006 & 157.51 & 129.14 & 190.88 & RW2 \\ 
  Mozambique & TETE & 2007 & 150.44 & 122.63 & 183.26 & RW2 \\ 
  Mozambique & TETE & 2008 & 143.85 & 115.46 & 177.67 & RW2 \\ 
  Mozambique & TETE & 2009 & 137.64 & 108.00 & 173.68 & RW2 \\ 
  Mozambique & TETE & 2010 & 131.67 & 101.13 & 170.07 & RW2 \\ 
  Mozambique & TETE & 2011 & 126.18 & 95.76 & 164.76 & RW2 \\ 
  Mozambique & TETE & 2012 & 120.74 & 90.38 & 160.31 & RW2 \\ 
  Mozambique & TETE & 2013 & 115.60 & 83.00 & 160.11 & RW2 \\ 
  Mozambique & TETE & 2014 & 110.60 & 73.25 & 164.57 & RW2 \\ 
  Mozambique & TETE & 2015 & 105.94 & 60.98 & 178.57 & RW2 \\ 
  Mozambique & TETE & 2016 & 101.20 & 49.66 & 195.81 & RW2 \\ 
  Mozambique & TETE & 2017 & 96.88 & 39.51 & 221.84 & RW2 \\ 
  Mozambique & TETE & 2018 & 92.69 & 30.71 & 249.22 & RW2 \\ 
  Mozambique & TETE & 2019 & 88.78 & 23.72 & 282.02 & RW2 \\ 
  Mozambique & ZAMBEZIA & 1980 & 277.84 & 210.63 & 357.80 & RW2 \\ 
  Mozambique & ZAMBEZIA & 1981 & 276.63 & 223.44 & 337.63 & RW2 \\ 
  Mozambique & ZAMBEZIA & 1982 & 275.33 & 227.35 & 329.55 & RW2 \\ 
  Mozambique & ZAMBEZIA & 1983 & 273.99 & 225.69 & 329.15 & RW2 \\ 
  Mozambique & ZAMBEZIA & 1984 & 272.53 & 223.24 & 328.40 & RW2 \\ 
  Mozambique & ZAMBEZIA & 1985 & 271.08 & 224.89 & 322.98 & RW2 \\ 
  Mozambique & ZAMBEZIA & 1986 & 269.54 & 226.54 & 316.61 & RW2 \\ 
  Mozambique & ZAMBEZIA & 1987 & 267.53 & 227.18 & 311.72 & RW2 \\ 
  Mozambique & ZAMBEZIA & 1988 & 265.73 & 225.23 & 309.84 & RW2 \\ 
  Mozambique & ZAMBEZIA & 1989 & 263.39 & 222.24 & 308.78 & RW2 \\ 
  Mozambique & ZAMBEZIA & 1990 & 260.74 & 221.37 & 304.25 & RW2 \\ 
  Mozambique & ZAMBEZIA & 1991 & 257.33 & 220.85 & 297.96 & RW2 \\ 
  Mozambique & ZAMBEZIA & 1992 & 253.30 & 218.40 & 291.51 & RW2 \\ 
  Mozambique & ZAMBEZIA & 1993 & 248.55 & 213.55 & 286.99 & RW2 \\ 
  Mozambique & ZAMBEZIA & 1994 & 242.86 & 206.40 & 283.10 & RW2 \\ 
  Mozambique & ZAMBEZIA & 1995 & 236.52 & 200.71 & 275.82 & RW2 \\ 
  Mozambique & ZAMBEZIA & 1996 & 229.35 & 196.26 & 265.26 & RW2 \\ 
  Mozambique & ZAMBEZIA & 1997 & 221.49 & 190.16 & 255.96 & RW2 \\ 
  Mozambique & ZAMBEZIA & 1998 & 213.08 & 182.33 & 247.74 & RW2 \\ 
  Mozambique & ZAMBEZIA & 1999 & 204.33 & 172.24 & 240.68 & RW2 \\ 
  Mozambique & ZAMBEZIA & 2000 & 195.10 & 163.42 & 230.37 & RW2 \\ 
  Mozambique & ZAMBEZIA & 2001 & 186.24 & 157.21 & 219.06 & RW2 \\ 
  Mozambique & ZAMBEZIA & 2002 & 177.42 & 150.24 & 208.18 & RW2 \\ 
  Mozambique & ZAMBEZIA & 2003 & 169.27 & 142.12 & 200.79 & RW2 \\ 
  Mozambique & ZAMBEZIA & 2004 & 161.22 & 133.36 & 193.73 & RW2 \\ 
  Mozambique & ZAMBEZIA & 2005 & 153.75 & 126.44 & 185.69 & RW2 \\ 
  Mozambique & ZAMBEZIA & 2006 & 146.64 & 120.87 & 176.76 & RW2 \\ 
  Mozambique & ZAMBEZIA & 2007 & 140.04 & 115.02 & 169.21 & RW2 \\ 
  Mozambique & ZAMBEZIA & 2008 & 133.67 & 108.37 & 163.78 & RW2 \\ 
  Mozambique & ZAMBEZIA & 2009 & 127.68 & 101.30 & 159.81 & RW2 \\ 
  Mozambique & ZAMBEZIA & 2010 & 121.91 & 94.67 & 156.29 & RW2 \\ 
  Mozambique & ZAMBEZIA & 2011 & 116.62 & 89.28 & 150.29 & RW2 \\ 
  Mozambique & ZAMBEZIA & 2012 & 111.33 & 84.24 & 146.61 & RW2 \\ 
  Mozambique & ZAMBEZIA & 2013 & 106.38 & 77.23 & 145.96 & RW2 \\ 
  Mozambique & ZAMBEZIA & 2014 & 101.48 & 67.51 & 150.63 & RW2 \\ 
  Mozambique & ZAMBEZIA & 2015 & 97.06 & 56.10 & 164.17 & RW2 \\ 
  Mozambique & ZAMBEZIA & 2016 & 92.84 & 45.76 & 180.27 & RW2 \\ 
  Mozambique & ZAMBEZIA & 2017 & 88.38 & 36.21 & 200.22 & RW2 \\ 
  Mozambique & ZAMBEZIA & 2018 & 84.07 & 28.24 & 226.37 & RW2 \\ 
  Mozambique & ZAMBEZIA & 2019 & 80.36 & 21.44 & 262.03 & RW2 \\ 
  Namibia & ALL & 1980 & 98.34 & 90.95 & 106.46 & IHME \\ 
  Namibia & ALL & 1980 & 100.82 & 65.17 & 152.69 & RW2 \\ 
  Namibia & ALL & 1980 & 97.70 & 88.70 & 107.30 & UN \\ 
  Namibia & ALL & 1981 & 96.78 & 89.88 & 104.23 & IHME \\ 
  Namibia & ALL & 1981 & 97.52 & 70.09 & 133.96 & RW2 \\ 
  Namibia & ALL & 1981 & 96.50 & 87.70 & 105.80 & UN \\ 
  Namibia & ALL & 1982 & 93.68 & 87.01 & 100.73 & IHME \\ 
  Namibia & ALL & 1982 & 94.39 & 71.36 & 123.84 & RW2 \\ 
  Namibia & ALL & 1982 & 94.70 & 86.20 & 103.70 & UN \\ 
  Namibia & ALL & 1983 & 89.37 & 83.04 & 96.34 & IHME \\ 
  Namibia & ALL & 1983 & 91.15 & 69.11 & 119.41 & RW2 \\ 
  Namibia & ALL & 1983 & 92.50 & 84.30 & 101.20 & UN \\ 
  Namibia & ALL & 1984 & 85.18 & 78.91 & 92.17 & IHME \\ 
  Namibia & ALL & 1984 & 88.31 & 65.94 & 116.99 & RW2 \\ 
  Namibia & ALL & 1984 & 89.80 & 81.90 & 98.20 & UN \\ 
  Namibia & ALL & 1985 & 81.54 & 75.60 & 88.17 & IHME \\ 
  Namibia & ALL & 1985 & 85.30 & 64.82 & 111.59 & RW2 \\ 
  Namibia & ALL & 1985 & 87.00 & 79.50 & 94.90 & UN \\ 
  Namibia & ALL & 1986 & 78.81 & 73.28 & 85.12 & IHME \\ 
  Namibia & ALL & 1986 & 82.64 & 63.58 & 106.73 & RW2 \\ 
  Namibia & ALL & 1986 & 83.90 & 76.90 & 91.50 & UN \\ 
  Namibia & ALL & 1987 & 76.74 & 70.96 & 82.59 & IHME \\ 
  Namibia & ALL & 1987 & 80.22 & 62.40 & 102.88 & RW2 \\ 
  Namibia & ALL & 1987 & 81.10 & 74.30 & 88.30 & UN \\ 
  Namibia & ALL & 1988 & 75.49 & 70.15 & 80.69 & IHME \\ 
  Namibia & ALL & 1988 & 77.92 & 60.29 & 100.23 & RW2 \\ 
  Namibia & ALL & 1988 & 78.30 & 71.90 & 85.30 & UN \\ 
  Namibia & ALL & 1989 & 73.97 & 68.95 & 78.96 & IHME \\ 
  Namibia & ALL & 1989 & 75.89 & 58.46 & 98.23 & RW2 \\ 
  Namibia & ALL & 1989 & 75.70 & 69.60 & 82.50 & UN \\ 
  Namibia & ALL & 1990 & 71.65 & 67.01 & 76.28 & IHME \\ 
  Namibia & ALL & 1990 & 74.12 & 57.33 & 95.39 & RW2 \\ 
  Namibia & ALL & 1990 & 73.50 & 67.40 & 80.20 & UN \\ 
  Namibia & ALL & 1991 & 68.73 & 64.26 & 73.50 & IHME \\ 
  Namibia & ALL & 1991 & 72.65 & 56.88 & 92.07 & RW2 \\ 
  Namibia & ALL & 1991 & 71.70 & 65.50 & 78.30 & UN \\ 
  Namibia & ALL & 1992 & 65.69 & 61.26 & 70.36 & IHME \\ 
  Namibia & ALL & 1992 & 71.48 & 56.28 & 90.24 & RW2 \\ 
  Namibia & ALL & 1992 & 70.40 & 64.10 & 77.20 & UN \\ 
  Namibia & ALL & 1993 & 63.20 & 59.05 & 67.91 & IHME \\ 
  Namibia & ALL & 1993 & 70.68 & 55.40 & 89.86 & RW2 \\ 
  Namibia & ALL & 1993 & 69.70 & 63.10 & 76.60 & UN \\ 
  Namibia & ALL & 1994 & 61.15 & 56.88 & 65.93 & IHME \\ 
  Namibia & ALL & 1994 & 70.21 & 54.30 & 90.72 & RW2 \\ 
  Namibia & ALL & 1994 & 69.70 & 62.80 & 76.80 & UN \\ 
  Namibia & ALL & 1995 & 60.18 & 55.92 & 65.18 & IHME \\ 
  Namibia & ALL & 1995 & 70.06 & 53.72 & 90.19 & RW2 \\ 
  Namibia & ALL & 1995 & 70.20 & 63.20 & 77.50 & UN \\ 
  Namibia & ALL & 1996 & 59.98 & 55.64 & 65.01 & IHME \\ 
  Namibia & ALL & 1996 & 70.49 & 54.36 & 90.54 & RW2 \\ 
  Namibia & ALL & 1996 & 71.20 & 64.00 & 78.70 & UN \\ 
  Namibia & ALL & 1997 & 60.53 & 56.00 & 65.48 & IHME \\ 
  Namibia & ALL & 1997 & 71.40 & 55.29 & 91.00 & RW2 \\ 
  Namibia & ALL & 1997 & 72.40 & 65.20 & 80.10 & UN \\ 
  Namibia & ALL & 1998 & 61.58 & 57.18 & 66.49 & IHME \\ 
  Namibia & ALL & 1998 & 72.76 & 56.03 & 93.18 & RW2 \\ 
  Namibia & ALL & 1998 & 73.70 & 66.70 & 81.40 & UN \\ 
  Namibia & ALL & 1999 & 62.46 & 57.96 & 67.82 & IHME \\ 
  Namibia & ALL & 1999 & 74.23 & 56.69 & 94.83 & RW2 \\ 
  Namibia & ALL & 1999 & 74.90 & 68.10 & 82.60 & UN \\ 
  Namibia & ALL & 2000 & 63.71 & 59.28 & 68.93 & IHME \\ 
  Namibia & ALL & 2000 & 76.13 & 59.58 & 97.71 & RW2 \\ 
  Namibia & ALL & 2000 & 75.70 & 69.10 & 83.40 & UN \\ 
  Namibia & ALL & 2001 & 64.31 & 59.69 & 69.65 & IHME \\ 
  Namibia & ALL & 2001 & 76.94 & 60.76 & 97.69 & RW2 \\ 
  Namibia & ALL & 2001 & 76.00 & 69.40 & 83.60 & UN \\ 
  Namibia & ALL & 2002 & 64.49 & 59.40 & 70.08 & IHME \\ 
  Namibia & ALL & 2002 & 76.88 & 61.12 & 96.94 & RW2 \\ 
  Namibia & ALL & 2002 & 75.80 & 69.20 & 83.40 & UN \\ 
  Namibia & ALL & 2003 & 64.74 & 59.51 & 70.21 & IHME \\ 
  Namibia & ALL & 2003 & 75.89 & 60.21 & 96.08 & RW2 \\ 
  Namibia & ALL & 2003 & 75.20 & 68.50 & 82.80 & UN \\ 
  Namibia & ALL & 2004 & 65.36 & 60.07 & 71.14 & IHME \\ 
  Namibia & ALL & 2004 & 73.82 & 57.82 & 94.75 & RW2 \\ 
  Namibia & ALL & 2004 & 73.90 & 67.10 & 81.60 & UN \\ 
  Namibia & ALL & 2005 & 64.26 & 59.13 & 70.31 & IHME \\ 
  Namibia & ALL & 2005 & 70.77 & 54.86 & 90.08 & RW2 \\ 
  Namibia & ALL & 2005 & 71.70 & 64.70 & 79.70 & UN \\ 
  Namibia & ALL & 2006 & 62.31 & 56.69 & 68.48 & IHME \\ 
  Namibia & ALL & 2006 & 67.70 & 53.16 & 85.18 & RW2 \\ 
  Namibia & ALL & 2006 & 68.40 & 61.20 & 76.60 & UN \\ 
  Namibia & ALL & 2007 & 61.35 & 55.60 & 68.03 & IHME \\ 
  Namibia & ALL & 2007 & 64.51 & 51.15 & 80.81 & RW2 \\ 
  Namibia & ALL & 2007 & 64.00 & 56.60 & 72.40 & UN \\ 
  Namibia & ALL & 2008 & 60.22 & 54.00 & 67.42 & IHME \\ 
  Namibia & ALL & 2008 & 61.45 & 48.32 & 77.88 & RW2 \\ 
  Namibia & ALL & 2008 & 60.10 & 52.60 & 69.20 & UN \\ 
  Namibia & ALL & 2009 & 58.72 & 52.47 & 65.92 & IHME \\ 
  Namibia & ALL & 2009 & 58.38 & 44.99 & 75.63 & RW2 \\ 
  Namibia & ALL & 2009 & 56.70 & 48.70 & 66.70 & UN \\ 
  Namibia & ALL & 2010 & 55.26 & 49.10 & 62.06 & IHME \\ 
  Namibia & ALL & 2010 & 55.59 & 42.07 & 73.55 & RW2 \\ 
  Namibia & ALL & 2010 & 53.80 & 45.10 & 64.80 & UN \\ 
  Namibia & ALL & 2011 & 51.27 & 45.14 & 58.20 & IHME \\ 
  Namibia & ALL & 2011 & 52.90 & 40.17 & 69.46 & RW2 \\ 
  Namibia & ALL & 2011 & 51.60 & 42.30 & 64.20 & UN \\ 
  Namibia & ALL & 2012 & 48.75 & 42.16 & 56.08 & IHME \\ 
  Namibia & ALL & 2012 & 50.31 & 38.17 & 66.08 & RW2 \\ 
  Namibia & ALL & 2012 & 49.90 & 39.80 & 63.80 & UN \\ 
  Namibia & ALL & 2013 & 46.09 & 38.89 & 53.89 & IHME \\ 
  Namibia & ALL & 2013 & 47.89 & 34.82 & 65.45 & RW2 \\ 
  Namibia & ALL & 2013 & 47.70 & 37.00 & 62.60 & UN \\ 
  Namibia & ALL & 2014 & 44.73 & 36.10 & 54.41 & IHME \\ 
  Namibia & ALL & 2014 & 45.52 & 29.68 & 69.26 & RW2 \\ 
  Namibia & ALL & 2014 & 46.40 & 34.80 & 62.80 & UN \\ 
  Namibia & ALL & 2015 & 40.52 & 31.39 & 51.56 & IHME \\ 
  Namibia & ALL & 2015 & 43.22 & 23.40 & 79.02 & RW2 \\ 
  Namibia & ALL & 2015 & 45.40 & 33.00 & 63.00 & UN \\ 
  Namibia & ALL & 2016 & 41.18 & 18.35 & 91.28 & RW2 \\ 
  Namibia & ALL & 2017 & 39.02 & 13.82 & 107.61 & RW2 \\ 
  Namibia & ALL & 2018 & 37.05 & 10.30 & 130.13 & RW2 \\ 
  Namibia & ALL & 2019 & 35.13 & 7.19 & 155.87 & RW2 \\ 
  Namibia & ALL & 80-84 & 83.11 & 98.70 & 69.80 & HT-Direct \\ 
  Namibia & ALL & 85-89 & 70.39 & 80.42 & 61.53 & HT-Direct \\ 
  Namibia & ALL & 90-94 & 65.80 & 73.48 & 58.87 & HT-Direct \\ 
  Namibia & ALL & 95-99 & 60.22 & 68.37 & 52.98 & HT-Direct \\ 
  Namibia & ALL & 00-04 & 74.81 & 83.32 & 67.10 & HT-Direct \\ 
  Namibia & ALL & 05-09 & 65.73 & 73.01 & 59.14 & HT-Direct \\ 
  Namibia & ALL & 10-14 & 53.16 & 64.30 & 43.85 & HT-Direct \\ 
  Namibia & ALL & 15-19 & 39.02 & 14.06 & 105.54 & RW2 \\ 
  Namibia & CAPRIVI & 1980 & 115.16 & 60.05 & 211.72 & RW2 \\ 
  Namibia & CAPRIVI & 1981 & 111.49 & 63.65 & 190.06 & RW2 \\ 
  Namibia & CAPRIVI & 1982 & 107.78 & 64.46 & 174.93 & RW2 \\ 
  Namibia & CAPRIVI & 1983 & 104.44 & 64.93 & 164.45 & RW2 \\ 
  Namibia & CAPRIVI & 1984 & 101.09 & 64.36 & 155.73 & RW2 \\ 
  Namibia & CAPRIVI & 1985 & 97.97 & 64.36 & 146.65 & RW2 \\ 
  Namibia & CAPRIVI & 1986 & 94.78 & 63.97 & 137.76 & RW2 \\ 
  Namibia & CAPRIVI & 1987 & 91.73 & 63.52 & 130.88 & RW2 \\ 
  Namibia & CAPRIVI & 1988 & 88.87 & 62.64 & 124.86 & RW2 \\ 
  Namibia & CAPRIVI & 1989 & 86.44 & 61.50 & 120.31 & RW2 \\ 
  Namibia & CAPRIVI & 1990 & 84.24 & 60.62 & 115.61 & RW2 \\ 
  Namibia & CAPRIVI & 1991 & 83.10 & 60.82 & 112.77 & RW2 \\ 
  Namibia & CAPRIVI & 1992 & 82.70 & 61.12 & 110.44 & RW2 \\ 
  Namibia & CAPRIVI & 1993 & 83.09 & 61.54 & 110.72 & RW2 \\ 
  Namibia & CAPRIVI & 1994 & 84.25 & 62.07 & 112.71 & RW2 \\ 
  Namibia & CAPRIVI & 1995 & 86.42 & 64.11 & 115.44 & RW2 \\ 
  Namibia & CAPRIVI & 1996 & 88.49 & 66.16 & 116.97 & RW2 \\ 
  Namibia & CAPRIVI & 1997 & 90.76 & 68.29 & 118.89 & RW2 \\ 
  Namibia & CAPRIVI & 1998 & 92.64 & 69.81 & 121.49 & RW2 \\ 
  Namibia & CAPRIVI & 1999 & 94.25 & 70.53 & 124.67 & RW2 \\ 
  Namibia & CAPRIVI & 2000 & 95.34 & 71.74 & 125.46 & RW2 \\ 
  Namibia & CAPRIVI & 2001 & 95.57 & 72.71 & 124.33 & RW2 \\ 
  Namibia & CAPRIVI & 2002 & 95.04 & 72.74 & 123.34 & RW2 \\ 
  Namibia & CAPRIVI & 2003 & 93.65 & 71.66 & 121.73 & RW2 \\ 
  Namibia & CAPRIVI & 2004 & 91.74 & 69.65 & 120.21 & RW2 \\ 
  Namibia & CAPRIVI & 2005 & 88.94 & 67.09 & 116.46 & RW2 \\ 
  Namibia & CAPRIVI & 2006 & 86.31 & 65.39 & 112.85 & RW2 \\ 
  Namibia & CAPRIVI & 2007 & 83.41 & 62.88 & 109.83 & RW2 \\ 
  Namibia & CAPRIVI & 2008 & 80.67 & 59.94 & 107.78 & RW2 \\ 
  Namibia & CAPRIVI & 2009 & 78.30 & 56.65 & 107.31 & RW2 \\ 
  Namibia & CAPRIVI & 2010 & 75.96 & 53.24 & 106.90 & RW2 \\ 
  Namibia & CAPRIVI & 2011 & 73.68 & 50.98 & 105.77 & RW2 \\ 
  Namibia & CAPRIVI & 2012 & 71.51 & 47.99 & 105.62 & RW2 \\ 
  Namibia & CAPRIVI & 2013 & 69.43 & 44.05 & 107.28 & RW2 \\ 
  Namibia & CAPRIVI & 2014 & 67.50 & 38.94 & 113.55 & RW2 \\ 
  Namibia & CAPRIVI & 2015 & 65.43 & 32.90 & 127.02 & RW2 \\ 
  Namibia & CAPRIVI & 2016 & 63.41 & 26.71 & 142.94 & RW2 \\ 
  Namibia & CAPRIVI & 2017 & 61.67 & 21.28 & 165.05 & RW2 \\ 
  Namibia & CAPRIVI & 2018 & 59.38 & 16.55 & 194.65 & RW2 \\ 
  Namibia & CAPRIVI & 2019 & 57.59 & 12.49 & 229.87 & RW2 \\ 
  Namibia & ERONGO & 1980 & 60.52 & 32.73 & 109.76 & RW2 \\ 
  Namibia & ERONGO & 1981 & 59.16 & 34.70 & 99.31 & RW2 \\ 
  Namibia & ERONGO & 1982 & 57.82 & 35.59 & 92.71 & RW2 \\ 
  Namibia & ERONGO & 1983 & 56.49 & 35.66 & 88.85 & RW2 \\ 
  Namibia & ERONGO & 1984 & 55.15 & 35.33 & 85.28 & RW2 \\ 
  Namibia & ERONGO & 1985 & 53.92 & 35.41 & 81.39 & RW2 \\ 
  Namibia & ERONGO & 1986 & 52.68 & 35.30 & 77.54 & RW2 \\ 
  Namibia & ERONGO & 1987 & 51.34 & 35.18 & 74.25 & RW2 \\ 
  Namibia & ERONGO & 1988 & 50.32 & 34.84 & 71.95 & RW2 \\ 
  Namibia & ERONGO & 1989 & 49.38 & 34.53 & 70.24 & RW2 \\ 
  Namibia & ERONGO & 1990 & 48.54 & 34.36 & 67.88 & RW2 \\ 
  Namibia & ERONGO & 1991 & 48.25 & 34.73 & 66.69 & RW2 \\ 
  Namibia & ERONGO & 1992 & 48.45 & 35.24 & 66.10 & RW2 \\ 
  Namibia & ERONGO & 1993 & 49.16 & 35.93 & 66.64 & RW2 \\ 
  Namibia & ERONGO & 1994 & 50.24 & 36.60 & 68.16 & RW2 \\ 
  Namibia & ERONGO & 1995 & 51.94 & 38.06 & 70.25 & RW2 \\ 
  Namibia & ERONGO & 1996 & 53.62 & 39.70 & 71.56 & RW2 \\ 
  Namibia & ERONGO & 1997 & 55.35 & 41.24 & 73.45 & RW2 \\ 
  Namibia & ERONGO & 1998 & 56.99 & 42.69 & 75.38 & RW2 \\ 
  Namibia & ERONGO & 1999 & 58.41 & 43.43 & 77.54 & RW2 \\ 
  Namibia & ERONGO & 2000 & 59.44 & 44.37 & 78.58 & RW2 \\ 
  Namibia & ERONGO & 2001 & 60.05 & 45.30 & 78.84 & RW2 \\ 
  Namibia & ERONGO & 2002 & 60.05 & 45.54 & 78.52 & RW2 \\ 
  Namibia & ERONGO & 2003 & 59.76 & 45.04 & 78.96 & RW2 \\ 
  Namibia & ERONGO & 2004 & 58.87 & 43.88 & 78.67 & RW2 \\ 
  Namibia & ERONGO & 2005 & 57.56 & 42.64 & 76.96 & RW2 \\ 
  Namibia & ERONGO & 2006 & 56.29 & 41.74 & 75.29 & RW2 \\ 
  Namibia & ERONGO & 2007 & 55.06 & 40.56 & 74.04 & RW2 \\ 
  Namibia & ERONGO & 2008 & 53.79 & 39.08 & 73.48 & RW2 \\ 
  Namibia & ERONGO & 2009 & 52.63 & 37.43 & 73.45 & RW2 \\ 
  Namibia & ERONGO & 2010 & 51.55 & 35.98 & 73.59 & RW2 \\ 
  Namibia & ERONGO & 2011 & 50.62 & 34.78 & 72.48 & RW2 \\ 
  Namibia & ERONGO & 2012 & 49.59 & 33.78 & 72.57 & RW2 \\ 
  Namibia & ERONGO & 2013 & 48.67 & 31.94 & 74.02 & RW2 \\ 
  Namibia & ERONGO & 2014 & 47.67 & 28.74 & 78.48 & RW2 \\ 
  Namibia & ERONGO & 2015 & 46.86 & 24.52 & 88.77 & RW2 \\ 
  Namibia & ERONGO & 2016 & 46.07 & 20.45 & 101.60 & RW2 \\ 
  Namibia & ERONGO & 2017 & 45.06 & 16.48 & 118.13 & RW2 \\ 
  Namibia & ERONGO & 2018 & 44.04 & 13.07 & 140.54 & RW2 \\ 
  Namibia & ERONGO & 2019 & 43.29 & 10.06 & 172.28 & RW2 \\ 
  Namibia & HARDAP & 1980 & 142.37 & 82.33 & 235.11 & RW2 \\ 
  Namibia & HARDAP & 1981 & 134.03 & 83.73 & 208.01 & RW2 \\ 
  Namibia & HARDAP & 1982 & 126.14 & 81.76 & 189.71 & RW2 \\ 
  Namibia & HARDAP & 1983 & 118.44 & 78.47 & 175.27 & RW2 \\ 
  Namibia & HARDAP & 1984 & 111.30 & 74.58 & 163.38 & RW2 \\ 
  Namibia & HARDAP & 1985 & 104.80 & 71.40 & 150.42 & RW2 \\ 
  Namibia & HARDAP & 1986 & 98.26 & 68.61 & 138.63 & RW2 \\ 
  Namibia & HARDAP & 1987 & 92.23 & 65.45 & 128.48 & RW2 \\ 
  Namibia & HARDAP & 1988 & 86.80 & 62.18 & 120.03 & RW2 \\ 
  Namibia & HARDAP & 1989 & 81.87 & 58.96 & 112.62 & RW2 \\ 
  Namibia & HARDAP & 1990 & 77.25 & 56.43 & 105.22 & RW2 \\ 
  Namibia & HARDAP & 1991 & 73.85 & 54.44 & 99.31 & RW2 \\ 
  Namibia & HARDAP & 1992 & 71.12 & 53.00 & 94.66 & RW2 \\ 
  Namibia & HARDAP & 1993 & 69.20 & 51.68 & 91.97 & RW2 \\ 
  Namibia & HARDAP & 1994 & 67.82 & 50.32 & 90.43 & RW2 \\ 
  Namibia & HARDAP & 1995 & 67.15 & 50.19 & 89.62 & RW2 \\ 
  Namibia & HARDAP & 1996 & 66.39 & 49.86 & 88.01 & RW2 \\ 
  Namibia & HARDAP & 1997 & 65.68 & 49.45 & 86.68 & RW2 \\ 
  Namibia & HARDAP & 1998 & 64.76 & 48.33 & 85.77 & RW2 \\ 
  Namibia & HARDAP & 1999 & 63.45 & 47.02 & 84.58 & RW2 \\ 
  Namibia & HARDAP & 2000 & 61.89 & 45.97 & 82.67 & RW2 \\ 
  Namibia & HARDAP & 2001 & 59.78 & 44.42 & 79.85 & RW2 \\ 
  Namibia & HARDAP & 2002 & 57.36 & 42.65 & 76.74 & RW2 \\ 
  Namibia & HARDAP & 2003 & 54.44 & 40.00 & 73.58 & RW2 \\ 
  Namibia & HARDAP & 2004 & 51.36 & 37.31 & 70.44 & RW2 \\ 
  Namibia & HARDAP & 2005 & 47.96 & 34.43 & 66.26 & RW2 \\ 
  Namibia & HARDAP & 2006 & 44.84 & 32.14 & 62.21 & RW2 \\ 
  Namibia & HARDAP & 2007 & 41.84 & 29.66 & 58.69 & RW2 \\ 
  Namibia & HARDAP & 2008 & 39.06 & 27.16 & 55.78 & RW2 \\ 
  Namibia & HARDAP & 2009 & 36.49 & 24.72 & 53.44 & RW2 \\ 
  Namibia & HARDAP & 2010 & 34.10 & 22.57 & 51.34 & RW2 \\ 
  Namibia & HARDAP & 2011 & 31.95 & 20.75 & 48.87 & RW2 \\ 
  Namibia & HARDAP & 2012 & 29.87 & 19.07 & 46.66 & RW2 \\ 
  Namibia & HARDAP & 2013 & 27.96 & 17.10 & 45.74 & RW2 \\ 
  Namibia & HARDAP & 2014 & 26.15 & 14.77 & 46.09 & RW2 \\ 
  Namibia & HARDAP & 2015 & 24.50 & 12.00 & 49.56 & RW2 \\ 
  Namibia & HARDAP & 2016 & 22.87 & 9.49 & 54.30 & RW2 \\ 
  Namibia & HARDAP & 2017 & 21.42 & 7.32 & 62.38 & RW2 \\ 
  Namibia & HARDAP & 2018 & 20.04 & 5.50 & 71.23 & RW2 \\ 
  Namibia & HARDAP & 2019 & 18.78 & 4.10 & 82.69 & RW2 \\ 
  Namibia & KARAS & 1980 & 130.88 & 78.38 & 211.15 & RW2 \\ 
  Namibia & KARAS & 1981 & 123.74 & 80.64 & 186.94 & RW2 \\ 
  Namibia & KARAS & 1982 & 117.37 & 79.54 & 169.96 & RW2 \\ 
  Namibia & KARAS & 1983 & 111.02 & 76.96 & 158.58 & RW2 \\ 
  Namibia & KARAS & 1984 & 104.94 & 73.21 & 148.36 & RW2 \\ 
  Namibia & KARAS & 1985 & 99.35 & 70.49 & 138.30 & RW2 \\ 
  Namibia & KARAS & 1986 & 93.90 & 67.90 & 128.87 & RW2 \\ 
  Namibia & KARAS & 1987 & 88.89 & 64.93 & 120.41 & RW2 \\ 
  Namibia & KARAS & 1988 & 84.30 & 62.06 & 113.87 & RW2 \\ 
  Namibia & KARAS & 1989 & 80.13 & 58.91 & 108.43 & RW2 \\ 
  Namibia & KARAS & 1990 & 76.19 & 56.36 & 101.93 & RW2 \\ 
  Namibia & KARAS & 1991 & 73.42 & 55.01 & 97.23 & RW2 \\ 
  Namibia & KARAS & 1992 & 71.34 & 53.44 & 93.99 & RW2 \\ 
  Namibia & KARAS & 1993 & 70.03 & 52.40 & 92.48 & RW2 \\ 
  Namibia & KARAS & 1994 & 69.19 & 51.04 & 92.10 & RW2 \\ 
  Namibia & KARAS & 1995 & 69.28 & 51.03 & 92.69 & RW2 \\ 
  Namibia & KARAS & 1996 & 69.16 & 51.16 & 92.06 & RW2 \\ 
  Namibia & KARAS & 1997 & 69.15 & 51.00 & 91.79 & RW2 \\ 
  Namibia & KARAS & 1998 & 68.96 & 50.69 & 92.08 & RW2 \\ 
  Namibia & KARAS & 1999 & 68.49 & 50.09 & 91.82 & RW2 \\ 
  Namibia & KARAS & 2000 & 67.77 & 49.64 & 90.98 & RW2 \\ 
  Namibia & KARAS & 2001 & 66.50 & 49.07 & 88.76 & RW2 \\ 
  Namibia & KARAS & 2002 & 64.69 & 48.10 & 86.31 & RW2 \\ 
  Namibia & KARAS & 2003 & 62.54 & 46.03 & 84.23 & RW2 \\ 
  Namibia & KARAS & 2004 & 60.09 & 44.03 & 81.74 & RW2 \\ 
  Namibia & KARAS & 2005 & 57.16 & 41.45 & 78.22 & RW2 \\ 
  Namibia & KARAS & 2006 & 54.52 & 39.53 & 74.89 & RW2 \\ 
  Namibia & KARAS & 2007 & 51.91 & 37.26 & 72.06 & RW2 \\ 
  Namibia & KARAS & 2008 & 49.38 & 34.64 & 70.06 & RW2 \\ 
  Namibia & KARAS & 2009 & 47.09 & 32.02 & 68.95 & RW2 \\ 
  Namibia & KARAS & 2010 & 44.88 & 29.64 & 68.17 & RW2 \\ 
  Namibia & KARAS & 2011 & 42.86 & 27.62 & 66.47 & RW2 \\ 
  Namibia & KARAS & 2012 & 40.88 & 25.55 & 65.80 & RW2 \\ 
  Namibia & KARAS & 2013 & 39.07 & 23.12 & 66.44 & RW2 \\ 
  Namibia & KARAS & 2014 & 37.28 & 20.20 & 69.68 & RW2 \\ 
  Namibia & KARAS & 2015 & 35.53 & 16.65 & 76.47 & RW2 \\ 
  Namibia & KARAS & 2016 & 33.89 & 13.43 & 85.19 & RW2 \\ 
  Namibia & KARAS & 2017 & 32.28 & 10.58 & 97.71 & RW2 \\ 
  Namibia & KARAS & 2018 & 30.76 & 8.00 & 115.08 & RW2 \\ 
  Namibia & KARAS & 2019 & 29.41 & 5.97 & 137.31 & RW2 \\ 
  Namibia & KAVANGO & 1980 & 128.25 & 78.03 & 204.93 & RW2 \\ 
  Namibia & KAVANGO & 1981 & 123.84 & 81.98 & 184.06 & RW2 \\ 
  Namibia & KAVANGO & 1982 & 119.85 & 82.72 & 171.42 & RW2 \\ 
  Namibia & KAVANGO & 1983 & 115.55 & 81.37 & 162.65 & RW2 \\ 
  Namibia & KAVANGO & 1984 & 111.58 & 79.76 & 154.83 & RW2 \\ 
  Namibia & KAVANGO & 1985 & 107.82 & 78.51 & 146.76 & RW2 \\ 
  Namibia & KAVANGO & 1986 & 104.03 & 77.23 & 138.83 & RW2 \\ 
  Namibia & KAVANGO & 1987 & 100.54 & 76.11 & 132.50 & RW2 \\ 
  Namibia & KAVANGO & 1988 & 97.20 & 73.89 & 127.43 & RW2 \\ 
  Namibia & KAVANGO & 1989 & 94.26 & 71.77 & 123.20 & RW2 \\ 
  Namibia & KAVANGO & 1990 & 91.60 & 70.32 & 118.07 & RW2 \\ 
  Namibia & KAVANGO & 1991 & 90.02 & 70.08 & 114.80 & RW2 \\ 
  Namibia & KAVANGO & 1992 & 89.25 & 70.34 & 112.55 & RW2 \\ 
  Namibia & KAVANGO & 1993 & 89.39 & 70.24 & 112.65 & RW2 \\ 
  Namibia & KAVANGO & 1994 & 90.30 & 70.82 & 113.65 & RW2 \\ 
  Namibia & KAVANGO & 1995 & 92.14 & 72.90 & 115.88 & RW2 \\ 
  Namibia & KAVANGO & 1996 & 94.02 & 75.41 & 116.91 & RW2 \\ 
  Namibia & KAVANGO & 1997 & 95.87 & 77.36 & 118.22 & RW2 \\ 
  Namibia & KAVANGO & 1998 & 97.62 & 78.40 & 120.52 & RW2 \\ 
  Namibia & KAVANGO & 1999 & 98.92 & 78.73 & 123.23 & RW2 \\ 
  Namibia & KAVANGO & 2000 & 99.58 & 79.34 & 124.29 & RW2 \\ 
  Namibia & KAVANGO & 2001 & 99.48 & 79.75 & 123.31 & RW2 \\ 
  Namibia & KAVANGO & 2002 & 98.66 & 79.41 & 122.13 & RW2 \\ 
  Namibia & KAVANGO & 2003 & 96.94 & 77.41 & 120.60 & RW2 \\ 
  Namibia & KAVANGO & 2004 & 94.69 & 74.87 & 119.33 & RW2 \\ 
  Namibia & KAVANGO & 2005 & 91.69 & 72.12 & 115.60 & RW2 \\ 
  Namibia & KAVANGO & 2006 & 88.90 & 70.25 & 111.45 & RW2 \\ 
  Namibia & KAVANGO & 2007 & 85.87 & 67.93 & 108.17 & RW2 \\ 
  Namibia & KAVANGO & 2008 & 83.12 & 64.63 & 106.16 & RW2 \\ 
  Namibia & KAVANGO & 2009 & 80.36 & 60.98 & 105.36 & RW2 \\ 
  Namibia & KAVANGO & 2010 & 78.01 & 57.78 & 104.85 & RW2 \\ 
  Namibia & KAVANGO & 2011 & 75.51 & 55.56 & 101.99 & RW2 \\ 
  Namibia & KAVANGO & 2012 & 73.17 & 53.25 & 100.83 & RW2 \\ 
  Namibia & KAVANGO & 2013 & 70.96 & 49.31 & 101.72 & RW2 \\ 
  Namibia & KAVANGO & 2014 & 68.76 & 43.62 & 107.19 & RW2 \\ 
  Namibia & KAVANGO & 2015 & 66.63 & 36.54 & 119.17 & RW2 \\ 
  Namibia & KAVANGO & 2016 & 64.56 & 29.70 & 135.43 & RW2 \\ 
  Namibia & KAVANGO & 2017 & 62.48 & 23.73 & 156.85 & RW2 \\ 
  Namibia & KAVANGO & 2018 & 60.30 & 18.64 & 182.69 & RW2 \\ 
  Namibia & KAVANGO & 2019 & 58.51 & 14.07 & 217.41 & RW2 \\ 
  Namibia & KHOMAS & 1980 & 84.77 & 45.68 & 152.43 & RW2 \\ 
  Namibia & KHOMAS & 1981 & 81.68 & 47.75 & 136.20 & RW2 \\ 
  Namibia & KHOMAS & 1982 & 78.38 & 47.70 & 125.93 & RW2 \\ 
  Namibia & KHOMAS & 1983 & 75.30 & 46.78 & 118.84 & RW2 \\ 
  Namibia & KHOMAS & 1984 & 72.33 & 45.66 & 112.54 & RW2 \\ 
  Namibia & KHOMAS & 1985 & 69.60 & 45.07 & 106.12 & RW2 \\ 
  Namibia & KHOMAS & 1986 & 66.85 & 44.18 & 99.35 & RW2 \\ 
  Namibia & KHOMAS & 1987 & 64.18 & 43.41 & 94.03 & RW2 \\ 
  Namibia & KHOMAS & 1988 & 61.70 & 42.34 & 88.99 & RW2 \\ 
  Namibia & KHOMAS & 1989 & 59.51 & 41.56 & 84.89 & RW2 \\ 
  Namibia & KHOMAS & 1990 & 57.58 & 40.60 & 80.74 & RW2 \\ 
  Namibia & KHOMAS & 1991 & 56.31 & 40.48 & 77.57 & RW2 \\ 
  Namibia & KHOMAS & 1992 & 55.53 & 40.45 & 75.74 & RW2 \\ 
  Namibia & KHOMAS & 1993 & 55.27 & 40.58 & 74.89 & RW2 \\ 
  Namibia & KHOMAS & 1994 & 55.48 & 40.85 & 75.02 & RW2 \\ 
  Namibia & KHOMAS & 1995 & 56.33 & 41.53 & 75.92 & RW2 \\ 
  Namibia & KHOMAS & 1996 & 56.98 & 42.70 & 76.11 & RW2 \\ 
  Namibia & KHOMAS & 1997 & 57.70 & 43.40 & 76.17 & RW2 \\ 
  Namibia & KHOMAS & 1998 & 58.10 & 43.71 & 76.65 & RW2 \\ 
  Namibia & KHOMAS & 1999 & 58.30 & 43.82 & 77.27 & RW2 \\ 
  Namibia & KHOMAS & 2000 & 58.11 & 43.60 & 76.69 & RW2 \\ 
  Namibia & KHOMAS & 2001 & 57.43 & 43.55 & 75.35 & RW2 \\ 
  Namibia & KHOMAS & 2002 & 56.20 & 42.88 & 73.50 & RW2 \\ 
  Namibia & KHOMAS & 2003 & 54.56 & 41.45 & 71.72 & RW2 \\ 
  Namibia & KHOMAS & 2004 & 52.57 & 39.43 & 69.86 & RW2 \\ 
  Namibia & KHOMAS & 2005 & 50.17 & 37.44 & 66.82 & RW2 \\ 
  Namibia & KHOMAS & 2006 & 47.93 & 35.77 & 63.84 & RW2 \\ 
  Namibia & KHOMAS & 2007 & 45.69 & 33.88 & 61.43 & RW2 \\ 
  Namibia & KHOMAS & 2008 & 43.57 & 31.73 & 59.57 & RW2 \\ 
  Namibia & KHOMAS & 2009 & 41.54 & 29.35 & 58.56 & RW2 \\ 
  Namibia & KHOMAS & 2010 & 39.76 & 27.21 & 57.64 & RW2 \\ 
  Namibia & KHOMAS & 2011 & 38.01 & 25.57 & 56.07 & RW2 \\ 
  Namibia & KHOMAS & 2012 & 36.32 & 23.81 & 54.99 & RW2 \\ 
  Namibia & KHOMAS & 2013 & 34.70 & 21.76 & 54.91 & RW2 \\ 
  Namibia & KHOMAS & 2014 & 33.16 & 19.12 & 57.00 & RW2 \\ 
  Namibia & KHOMAS & 2015 & 31.71 & 15.72 & 62.69 & RW2 \\ 
  Namibia & KHOMAS & 2016 & 30.38 & 12.85 & 70.22 & RW2 \\ 
  Namibia & KHOMAS & 2017 & 28.90 & 10.08 & 81.55 & RW2 \\ 
  Namibia & KHOMAS & 2018 & 27.65 & 7.76 & 95.06 & RW2 \\ 
  Namibia & KHOMAS & 2019 & 26.45 & 5.85 & 115.14 & RW2 \\ 
  Namibia & KUNENE & 1980 & 125.81 & 72.90 & 208.87 & RW2 \\ 
  Namibia & KUNENE & 1981 & 120.03 & 75.48 & 186.92 & RW2 \\ 
  Namibia & KUNENE & 1982 & 114.52 & 74.67 & 171.22 & RW2 \\ 
  Namibia & KUNENE & 1983 & 109.07 & 72.53 & 160.76 & RW2 \\ 
  Namibia & KUNENE & 1984 & 103.93 & 70.14 & 151.88 & RW2 \\ 
  Namibia & KUNENE & 1985 & 98.94 & 68.04 & 141.84 & RW2 \\ 
  Namibia & KUNENE & 1986 & 94.14 & 66.22 & 132.75 & RW2 \\ 
  Namibia & KUNENE & 1987 & 89.63 & 64.15 & 124.46 & RW2 \\ 
  Namibia & KUNENE & 1988 & 85.35 & 61.66 & 117.15 & RW2 \\ 
  Namibia & KUNENE & 1989 & 81.50 & 59.46 & 111.55 & RW2 \\ 
  Namibia & KUNENE & 1990 & 77.97 & 57.28 & 104.77 & RW2 \\ 
  Namibia & KUNENE & 1991 & 75.51 & 56.43 & 100.23 & RW2 \\ 
  Namibia & KUNENE & 1992 & 73.60 & 55.66 & 96.30 & RW2 \\ 
  Namibia & KUNENE & 1993 & 72.54 & 55.00 & 94.88 & RW2 \\ 
  Namibia & KUNENE & 1994 & 72.16 & 54.48 & 94.41 & RW2 \\ 
  Namibia & KUNENE & 1995 & 72.38 & 54.84 & 94.84 & RW2 \\ 
  Namibia & KUNENE & 1996 & 72.69 & 55.58 & 94.04 & RW2 \\ 
  Namibia & KUNENE & 1997 & 72.80 & 55.84 & 93.85 & RW2 \\ 
  Namibia & KUNENE & 1998 & 72.74 & 55.61 & 93.89 & RW2 \\ 
  Namibia & KUNENE & 1999 & 72.50 & 55.07 & 94.28 & RW2 \\ 
  Namibia & KUNENE & 2000 & 71.74 & 54.56 & 93.18 & RW2 \\ 
  Namibia & KUNENE & 2001 & 70.38 & 54.03 & 90.82 & RW2 \\ 
  Namibia & KUNENE & 2002 & 68.53 & 52.43 & 88.47 & RW2 \\ 
  Namibia & KUNENE & 2003 & 66.14 & 50.35 & 86.34 & RW2 \\ 
  Namibia & KUNENE & 2004 & 63.45 & 47.76 & 83.68 & RW2 \\ 
  Namibia & KUNENE & 2005 & 60.25 & 45.11 & 79.79 & RW2 \\ 
  Namibia & KUNENE & 2006 & 57.40 & 43.07 & 76.06 & RW2 \\ 
  Namibia & KUNENE & 2007 & 54.48 & 40.74 & 72.22 & RW2 \\ 
  Namibia & KUNENE & 2008 & 51.76 & 38.17 & 69.93 & RW2 \\ 
  Namibia & KUNENE & 2009 & 49.35 & 35.58 & 68.04 & RW2 \\ 
  Namibia & KUNENE & 2010 & 47.11 & 33.28 & 66.42 & RW2 \\ 
  Namibia & KUNENE & 2011 & 44.97 & 31.50 & 63.81 & RW2 \\ 
  Namibia & KUNENE & 2012 & 42.88 & 29.67 & 61.85 & RW2 \\ 
  Namibia & KUNENE & 2013 & 40.94 & 27.27 & 61.36 & RW2 \\ 
  Namibia & KUNENE & 2014 & 39.08 & 23.89 & 63.45 & RW2 \\ 
  Namibia & KUNENE & 2015 & 37.28 & 19.74 & 70.05 & RW2 \\ 
  Namibia & KUNENE & 2016 & 35.49 & 16.00 & 78.52 & RW2 \\ 
  Namibia & KUNENE & 2017 & 33.93 & 12.45 & 90.27 & RW2 \\ 
  Namibia & KUNENE & 2018 & 32.27 & 9.49 & 105.14 & RW2 \\ 
  Namibia & KUNENE & 2019 & 30.73 & 7.18 & 125.57 & RW2 \\ 
  Namibia & OHANGWENA & 1980 & 147.17 & 92.09 & 226.57 & RW2 \\ 
  Namibia & OHANGWENA & 1981 & 141.46 & 95.16 & 204.52 & RW2 \\ 
  Namibia & OHANGWENA & 1982 & 135.94 & 95.45 & 190.41 & RW2 \\ 
  Namibia & OHANGWENA & 1983 & 130.58 & 92.68 & 180.63 & RW2 \\ 
  Namibia & OHANGWENA & 1984 & 125.23 & 89.29 & 173.13 & RW2 \\ 
  Namibia & OHANGWENA & 1985 & 120.41 & 87.18 & 164.14 & RW2 \\ 
  Namibia & OHANGWENA & 1986 & 115.67 & 85.21 & 154.94 & RW2 \\ 
  Namibia & OHANGWENA & 1987 & 111.16 & 82.80 & 147.19 & RW2 \\ 
  Namibia & OHANGWENA & 1988 & 106.92 & 80.33 & 140.90 & RW2 \\ 
  Namibia & OHANGWENA & 1989 & 103.13 & 77.81 & 136.01 & RW2 \\ 
  Namibia & OHANGWENA & 1990 & 99.63 & 75.83 & 129.68 & RW2 \\ 
  Namibia & OHANGWENA & 1991 & 97.39 & 75.21 & 124.99 & RW2 \\ 
  Namibia & OHANGWENA & 1992 & 96.22 & 74.83 & 122.28 & RW2 \\ 
  Namibia & OHANGWENA & 1993 & 95.74 & 74.52 & 121.98 & RW2 \\ 
  Namibia & OHANGWENA & 1994 & 96.27 & 74.76 & 122.78 & RW2 \\ 
  Namibia & OHANGWENA & 1995 & 97.61 & 76.26 & 124.82 & RW2 \\ 
  Namibia & OHANGWENA & 1996 & 98.98 & 77.89 & 125.22 & RW2 \\ 
  Namibia & OHANGWENA & 1997 & 100.27 & 79.41 & 125.75 & RW2 \\ 
  Namibia & OHANGWENA & 1998 & 101.21 & 79.94 & 127.40 & RW2 \\ 
  Namibia & OHANGWENA & 1999 & 101.68 & 80.13 & 128.40 & RW2 \\ 
  Namibia & OHANGWENA & 2000 & 101.60 & 80.05 & 128.14 & RW2 \\ 
  Namibia & OHANGWENA & 2001 & 100.52 & 80.22 & 125.77 & RW2 \\ 
  Namibia & OHANGWENA & 2002 & 98.64 & 78.93 & 122.94 & RW2 \\ 
  Namibia & OHANGWENA & 2003 & 95.83 & 76.12 & 120.43 & RW2 \\ 
  Namibia & OHANGWENA & 2004 & 92.45 & 72.35 & 117.97 & RW2 \\ 
  Namibia & OHANGWENA & 2005 & 88.36 & 68.51 & 113.05 & RW2 \\ 
  Namibia & OHANGWENA & 2006 & 84.39 & 65.70 & 108.04 & RW2 \\ 
  Namibia & OHANGWENA & 2007 & 80.41 & 62.21 & 103.15 & RW2 \\ 
  Namibia & OHANGWENA & 2008 & 76.66 & 58.30 & 100.36 & RW2 \\ 
  Namibia & OHANGWENA & 2009 & 73.12 & 54.24 & 97.97 & RW2 \\ 
  Namibia & OHANGWENA & 2010 & 69.93 & 50.50 & 96.44 & RW2 \\ 
  Namibia & OHANGWENA & 2011 & 66.73 & 47.57 & 92.84 & RW2 \\ 
  Namibia & OHANGWENA & 2012 & 63.80 & 44.55 & 90.24 & RW2 \\ 
  Namibia & OHANGWENA & 2013 & 60.99 & 40.62 & 90.12 & RW2 \\ 
  Namibia & OHANGWENA & 2014 & 58.29 & 35.56 & 93.64 & RW2 \\ 
  Namibia & OHANGWENA & 2015 & 55.52 & 29.14 & 102.35 & RW2 \\ 
  Namibia & OHANGWENA & 2016 & 52.99 & 23.65 & 114.38 & RW2 \\ 
  Namibia & OHANGWENA & 2017 & 50.65 & 18.66 & 130.83 & RW2 \\ 
  Namibia & OHANGWENA & 2018 & 48.34 & 14.43 & 152.98 & RW2 \\ 
  Namibia & OHANGWENA & 2019 & 46.08 & 10.55 & 181.75 & RW2 \\ 
  Namibia & OMAHEKE & 1980 & 91.23 & 48.26 & 159.93 & RW2 \\ 
  Namibia & OMAHEKE & 1981 & 89.10 & 51.29 & 144.87 & RW2 \\ 
  Namibia & OMAHEKE & 1982 & 86.72 & 52.90 & 135.22 & RW2 \\ 
  Namibia & OMAHEKE & 1983 & 84.46 & 53.19 & 128.26 & RW2 \\ 
  Namibia & OMAHEKE & 1984 & 82.25 & 53.05 & 123.10 & RW2 \\ 
  Namibia & OMAHEKE & 1985 & 80.19 & 53.56 & 116.94 & RW2 \\ 
  Namibia & OMAHEKE & 1986 & 78.03 & 54.27 & 110.94 & RW2 \\ 
  Namibia & OMAHEKE & 1987 & 75.98 & 54.14 & 105.60 & RW2 \\ 
  Namibia & OMAHEKE & 1988 & 74.21 & 53.60 & 101.84 & RW2 \\ 
  Namibia & OMAHEKE & 1989 & 72.66 & 53.42 & 98.62 & RW2 \\ 
  Namibia & OMAHEKE & 1990 & 71.18 & 53.10 & 94.64 & RW2 \\ 
  Namibia & OMAHEKE & 1991 & 70.46 & 53.55 & 91.99 & RW2 \\ 
  Namibia & OMAHEKE & 1992 & 70.31 & 54.23 & 90.78 & RW2 \\ 
  Namibia & OMAHEKE & 1993 & 70.88 & 54.77 & 91.11 & RW2 \\ 
  Namibia & OMAHEKE & 1994 & 71.97 & 55.55 & 92.81 & RW2 \\ 
  Namibia & OMAHEKE & 1995 & 73.53 & 57.08 & 94.82 & RW2 \\ 
  Namibia & OMAHEKE & 1996 & 75.07 & 59.04 & 95.70 & RW2 \\ 
  Namibia & OMAHEKE & 1997 & 76.36 & 60.18 & 97.37 & RW2 \\ 
  Namibia & OMAHEKE & 1998 & 77.49 & 60.63 & 99.22 & RW2 \\ 
  Namibia & OMAHEKE & 1999 & 78.12 & 60.39 & 101.10 & RW2 \\ 
  Namibia & OMAHEKE & 2000 & 78.11 & 60.01 & 101.77 & RW2 \\ 
  Namibia & OMAHEKE & 2001 & 77.45 & 59.53 & 100.85 & RW2 \\ 
  Namibia & OMAHEKE & 2002 & 76.03 & 58.02 & 99.38 & RW2 \\ 
  Namibia & OMAHEKE & 2003 & 73.94 & 55.67 & 97.77 & RW2 \\ 
  Namibia & OMAHEKE & 2004 & 71.46 & 53.12 & 95.75 & RW2 \\ 
  Namibia & OMAHEKE & 2005 & 68.30 & 49.98 & 92.71 & RW2 \\ 
  Namibia & OMAHEKE & 2006 & 65.31 & 47.71 & 88.67 & RW2 \\ 
  Namibia & OMAHEKE & 2007 & 62.30 & 45.12 & 85.46 & RW2 \\ 
  Namibia & OMAHEKE & 2008 & 59.45 & 42.00 & 83.36 & RW2 \\ 
  Namibia & OMAHEKE & 2009 & 56.76 & 39.08 & 81.71 & RW2 \\ 
  Namibia & OMAHEKE & 2010 & 54.25 & 36.26 & 80.66 & RW2 \\ 
  Namibia & OMAHEKE & 2011 & 51.94 & 33.87 & 78.44 & RW2 \\ 
  Namibia & OMAHEKE & 2012 & 49.72 & 31.63 & 76.48 & RW2 \\ 
  Namibia & OMAHEKE & 2013 & 47.55 & 28.82 & 76.18 & RW2 \\ 
  Namibia & OMAHEKE & 2014 & 45.45 & 25.29 & 78.99 & RW2 \\ 
  Namibia & OMAHEKE & 2015 & 43.43 & 21.09 & 86.37 & RW2 \\ 
  Namibia & OMAHEKE & 2016 & 41.65 & 16.94 & 96.03 & RW2 \\ 
  Namibia & OMAHEKE & 2017 & 39.58 & 13.49 & 109.84 & RW2 \\ 
  Namibia & OMAHEKE & 2018 & 37.79 & 10.41 & 127.29 & RW2 \\ 
  Namibia & OMAHEKE & 2019 & 36.28 & 7.75 & 150.52 & RW2 \\ 
  Namibia & OMUSATI & 1980 & 140.87 & 79.37 & 238.36 & RW2 \\ 
  Namibia & OMUSATI & 1981 & 134.07 & 81.78 & 212.08 & RW2 \\ 
  Namibia & OMUSATI & 1982 & 127.20 & 80.95 & 194.44 & RW2 \\ 
  Namibia & OMUSATI & 1983 & 120.85 & 78.49 & 181.50 & RW2 \\ 
  Namibia & OMUSATI & 1984 & 114.40 & 75.76 & 169.34 & RW2 \\ 
  Namibia & OMUSATI & 1985 & 108.54 & 73.75 & 157.26 & RW2 \\ 
  Namibia & OMUSATI & 1986 & 103.03 & 71.49 & 146.34 & RW2 \\ 
  Namibia & OMUSATI & 1987 & 97.64 & 69.28 & 136.13 & RW2 \\ 
  Namibia & OMUSATI & 1988 & 92.49 & 66.42 & 127.70 & RW2 \\ 
  Namibia & OMUSATI & 1989 & 88.07 & 63.79 & 120.93 & RW2 \\ 
  Namibia & OMUSATI & 1990 & 83.92 & 61.53 & 113.64 & RW2 \\ 
  Namibia & OMUSATI & 1991 & 80.92 & 60.17 & 108.01 & RW2 \\ 
  Namibia & OMUSATI & 1992 & 78.75 & 59.21 & 103.25 & RW2 \\ 
  Namibia & OMUSATI & 1993 & 77.27 & 58.43 & 101.20 & RW2 \\ 
  Namibia & OMUSATI & 1994 & 76.59 & 58.02 & 100.12 & RW2 \\ 
  Namibia & OMUSATI & 1995 & 76.61 & 58.32 & 99.92 & RW2 \\ 
  Namibia & OMUSATI & 1996 & 76.65 & 59.16 & 98.86 & RW2 \\ 
  Namibia & OMUSATI & 1997 & 76.55 & 59.39 & 97.74 & RW2 \\ 
  Namibia & OMUSATI & 1998 & 76.33 & 59.27 & 97.80 & RW2 \\ 
  Namibia & OMUSATI & 1999 & 75.62 & 58.36 & 97.43 & RW2 \\ 
  Namibia & OMUSATI & 2000 & 74.55 & 57.66 & 96.03 & RW2 \\ 
  Namibia & OMUSATI & 2001 & 72.75 & 56.53 & 93.26 & RW2 \\ 
  Namibia & OMUSATI & 2002 & 70.39 & 54.80 & 89.88 & RW2 \\ 
  Namibia & OMUSATI & 2003 & 67.53 & 52.09 & 87.39 & RW2 \\ 
  Namibia & OMUSATI & 2004 & 64.23 & 49.03 & 84.20 & RW2 \\ 
  Namibia & OMUSATI & 2005 & 60.59 & 45.82 & 79.79 & RW2 \\ 
  Namibia & OMUSATI & 2006 & 57.11 & 43.00 & 75.34 & RW2 \\ 
  Namibia & OMUSATI & 2007 & 53.73 & 40.23 & 71.51 & RW2 \\ 
  Namibia & OMUSATI & 2008 & 50.58 & 37.01 & 68.53 & RW2 \\ 
  Namibia & OMUSATI & 2009 & 47.60 & 33.90 & 66.29 & RW2 \\ 
  Namibia & OMUSATI & 2010 & 44.96 & 30.98 & 64.83 & RW2 \\ 
  Namibia & OMUSATI & 2011 & 42.34 & 28.58 & 62.10 & RW2 \\ 
  Namibia & OMUSATI & 2012 & 39.87 & 26.38 & 59.79 & RW2 \\ 
  Namibia & OMUSATI & 2013 & 37.62 & 23.81 & 59.06 & RW2 \\ 
  Namibia & OMUSATI & 2014 & 35.54 & 20.45 & 61.04 & RW2 \\ 
  Namibia & OMUSATI & 2015 & 33.46 & 16.69 & 65.90 & RW2 \\ 
  Namibia & OMUSATI & 2016 & 31.52 & 13.26 & 73.21 & RW2 \\ 
  Namibia & OMUSATI & 2017 & 29.64 & 10.32 & 82.62 & RW2 \\ 
  Namibia & OMUSATI & 2018 & 27.98 & 7.82 & 96.02 & RW2 \\ 
  Namibia & OMUSATI & 2019 & 26.43 & 5.77 & 112.50 & RW2 \\ 
  Namibia & OSHANA & 1980 & 128.01 & 78.05 & 202.15 & RW2 \\ 
  Namibia & OSHANA & 1981 & 123.03 & 80.99 & 182.16 & RW2 \\ 
  Namibia & OSHANA & 1982 & 118.03 & 80.75 & 168.24 & RW2 \\ 
  Namibia & OSHANA & 1983 & 113.27 & 78.77 & 160.22 & RW2 \\ 
  Namibia & OSHANA & 1984 & 108.71 & 76.08 & 153.02 & RW2 \\ 
  Namibia & OSHANA & 1985 & 104.15 & 74.05 & 144.14 & RW2 \\ 
  Namibia & OSHANA & 1986 & 99.89 & 72.44 & 136.50 & RW2 \\ 
  Namibia & OSHANA & 1987 & 95.73 & 70.43 & 129.37 & RW2 \\ 
  Namibia & OSHANA & 1988 & 91.93 & 68.02 & 123.22 & RW2 \\ 
  Namibia & OSHANA & 1989 & 88.57 & 66.07 & 118.59 & RW2 \\ 
  Namibia & OSHANA & 1990 & 85.32 & 64.19 & 112.42 & RW2 \\ 
  Namibia & OSHANA & 1991 & 83.29 & 63.62 & 108.09 & RW2 \\ 
  Namibia & OSHANA & 1992 & 81.93 & 63.35 & 105.37 & RW2 \\ 
  Namibia & OSHANA & 1993 & 81.45 & 62.89 & 104.65 & RW2 \\ 
  Namibia & OSHANA & 1994 & 81.62 & 62.77 & 104.82 & RW2 \\ 
  Namibia & OSHANA & 1995 & 82.67 & 64.03 & 106.20 & RW2 \\ 
  Namibia & OSHANA & 1996 & 83.53 & 65.38 & 106.54 & RW2 \\ 
  Namibia & OSHANA & 1997 & 84.38 & 66.02 & 106.75 & RW2 \\ 
  Namibia & OSHANA & 1998 & 84.97 & 66.63 & 107.91 & RW2 \\ 
  Namibia & OSHANA & 1999 & 85.12 & 66.11 & 108.86 & RW2 \\ 
  Namibia & OSHANA & 2000 & 84.78 & 65.93 & 108.85 & RW2 \\ 
  Namibia & OSHANA & 2001 & 83.67 & 65.27 & 107.05 & RW2 \\ 
  Namibia & OSHANA & 2002 & 81.80 & 63.89 & 104.45 & RW2 \\ 
  Namibia & OSHANA & 2003 & 79.21 & 61.15 & 102.52 & RW2 \\ 
  Namibia & OSHANA & 2004 & 76.26 & 57.89 & 99.87 & RW2 \\ 
  Namibia & OSHANA & 2005 & 72.66 & 54.44 & 95.99 & RW2 \\ 
  Namibia & OSHANA & 2006 & 69.26 & 51.77 & 91.74 & RW2 \\ 
  Namibia & OSHANA & 2007 & 65.90 & 48.87 & 88.37 & RW2 \\ 
  Namibia & OSHANA & 2008 & 62.64 & 45.46 & 85.87 & RW2 \\ 
  Namibia & OSHANA & 2009 & 59.68 & 42.13 & 83.93 & RW2 \\ 
  Namibia & OSHANA & 2010 & 56.84 & 38.81 & 82.40 & RW2 \\ 
  Namibia & OSHANA & 2011 & 54.19 & 36.41 & 79.64 & RW2 \\ 
  Namibia & OSHANA & 2012 & 51.68 & 33.87 & 77.93 & RW2 \\ 
  Namibia & OSHANA & 2013 & 49.20 & 30.73 & 77.56 & RW2 \\ 
  Namibia & OSHANA & 2014 & 46.90 & 26.91 & 80.33 & RW2 \\ 
  Namibia & OSHANA & 2015 & 44.72 & 22.32 & 87.11 & RW2 \\ 
  Namibia & OSHANA & 2016 & 42.65 & 18.12 & 97.88 & RW2 \\ 
  Namibia & OSHANA & 2017 & 40.46 & 14.19 & 111.48 & RW2 \\ 
  Namibia & OSHANA & 2018 & 38.58 & 10.84 & 130.35 & RW2 \\ 
  Namibia & OSHANA & 2019 & 36.81 & 8.13 & 153.71 & RW2 \\ 
  Namibia & OSHIKOTO & 1980 & 91.05 & 51.10 & 154.81 & RW2 \\ 
  Namibia & OSHIKOTO & 1981 & 89.11 & 54.34 & 141.20 & RW2 \\ 
  Namibia & OSHIKOTO & 1982 & 87.47 & 56.15 & 132.09 & RW2 \\ 
  Namibia & OSHIKOTO & 1983 & 85.61 & 56.26 & 126.52 & RW2 \\ 
  Namibia & OSHIKOTO & 1984 & 83.69 & 56.44 & 121.56 & RW2 \\ 
  Namibia & OSHIKOTO & 1985 & 82.02 & 57.07 & 116.51 & RW2 \\ 
  Namibia & OSHIKOTO & 1986 & 80.28 & 56.97 & 111.62 & RW2 \\ 
  Namibia & OSHIKOTO & 1987 & 78.63 & 57.29 & 107.05 & RW2 \\ 
  Namibia & OSHIKOTO & 1988 & 77.10 & 56.83 & 103.87 & RW2 \\ 
  Namibia & OSHIKOTO & 1989 & 75.79 & 56.22 & 101.67 & RW2 \\ 
  Namibia & OSHIKOTO & 1990 & 74.66 & 55.99 & 98.78 & RW2 \\ 
  Namibia & OSHIKOTO & 1991 & 74.30 & 56.54 & 96.58 & RW2 \\ 
  Namibia & OSHIKOTO & 1992 & 74.60 & 57.71 & 95.73 & RW2 \\ 
  Namibia & OSHIKOTO & 1993 & 75.62 & 58.57 & 96.98 & RW2 \\ 
  Namibia & OSHIKOTO & 1994 & 77.20 & 59.59 & 99.00 & RW2 \\ 
  Namibia & OSHIKOTO & 1995 & 79.57 & 62.07 & 101.84 & RW2 \\ 
  Namibia & OSHIKOTO & 1996 & 81.82 & 64.69 & 103.55 & RW2 \\ 
  Namibia & OSHIKOTO & 1997 & 84.15 & 66.91 & 105.49 & RW2 \\ 
  Namibia & OSHIKOTO & 1998 & 86.28 & 68.41 & 108.37 & RW2 \\ 
  Namibia & OSHIKOTO & 1999 & 87.81 & 69.42 & 111.01 & RW2 \\ 
  Namibia & OSHIKOTO & 2000 & 88.89 & 70.31 & 112.03 & RW2 \\ 
  Namibia & OSHIKOTO & 2001 & 89.18 & 71.13 & 111.44 & RW2 \\ 
  Namibia & OSHIKOTO & 2002 & 88.65 & 70.79 & 110.67 & RW2 \\ 
  Namibia & OSHIKOTO & 2003 & 87.33 & 69.08 & 109.89 & RW2 \\ 
  Namibia & OSHIKOTO & 2004 & 85.35 & 66.63 & 108.76 & RW2 \\ 
  Namibia & OSHIKOTO & 2005 & 82.64 & 63.64 & 106.08 & RW2 \\ 
  Namibia & OSHIKOTO & 2006 & 80.12 & 61.81 & 103.31 & RW2 \\ 
  Namibia & OSHIKOTO & 2007 & 77.39 & 59.35 & 100.23 & RW2 \\ 
  Namibia & OSHIKOTO & 2008 & 74.86 & 56.43 & 98.61 & RW2 \\ 
  Namibia & OSHIKOTO & 2009 & 72.50 & 53.34 & 97.64 & RW2 \\ 
  Namibia & OSHIKOTO & 2010 & 70.40 & 50.65 & 96.64 & RW2 \\ 
  Namibia & OSHIKOTO & 2011 & 68.15 & 48.64 & 94.47 & RW2 \\ 
  Namibia & OSHIKOTO & 2012 & 66.19 & 46.66 & 93.05 & RW2 \\ 
  Namibia & OSHIKOTO & 2013 & 64.16 & 43.19 & 93.31 & RW2 \\ 
  Namibia & OSHIKOTO & 2014 & 62.16 & 38.60 & 98.37 & RW2 \\ 
  Namibia & OSHIKOTO & 2015 & 60.27 & 32.13 & 109.01 & RW2 \\ 
  Namibia & OSHIKOTO & 2016 & 58.43 & 26.50 & 123.81 & RW2 \\ 
  Namibia & OSHIKOTO & 2017 & 56.56 & 21.07 & 143.83 & RW2 \\ 
  Namibia & OSHIKOTO & 2018 & 54.80 & 16.28 & 167.53 & RW2 \\ 
  Namibia & OSHIKOTO & 2019 & 53.05 & 12.46 & 199.56 & RW2 \\ 
  Namibia & OTJOZONDJUPA & 1980 & 88.20 & 50.47 & 148.89 & RW2 \\ 
  Namibia & OTJOZONDJUPA & 1981 & 85.63 & 53.25 & 134.26 & RW2 \\ 
  Namibia & OTJOZONDJUPA & 1982 & 82.63 & 53.89 & 124.47 & RW2 \\ 
  Namibia & OTJOZONDJUPA & 1983 & 80.10 & 53.19 & 118.25 & RW2 \\ 
  Namibia & OTJOZONDJUPA & 1984 & 77.47 & 52.38 & 113.31 & RW2 \\ 
  Namibia & OTJOZONDJUPA & 1985 & 75.08 & 52.09 & 107.47 & RW2 \\ 
  Namibia & OTJOZONDJUPA & 1986 & 72.62 & 51.73 & 101.39 & RW2 \\ 
  Namibia & OTJOZONDJUPA & 1987 & 70.36 & 50.96 & 96.56 & RW2 \\ 
  Namibia & OTJOZONDJUPA & 1988 & 68.10 & 49.90 & 92.45 & RW2 \\ 
  Namibia & OTJOZONDJUPA & 1989 & 66.31 & 48.89 & 89.65 & RW2 \\ 
  Namibia & OTJOZONDJUPA & 1990 & 64.54 & 47.88 & 85.96 & RW2 \\ 
  Namibia & OTJOZONDJUPA & 1991 & 63.51 & 48.04 & 83.19 & RW2 \\ 
  Namibia & OTJOZONDJUPA & 1992 & 63.07 & 48.43 & 81.87 & RW2 \\ 
  Namibia & OTJOZONDJUPA & 1993 & 63.29 & 48.54 & 82.13 & RW2 \\ 
  Namibia & OTJOZONDJUPA & 1994 & 64.07 & 48.95 & 83.36 & RW2 \\ 
  Namibia & OTJOZONDJUPA & 1995 & 65.38 & 50.24 & 84.75 & RW2 \\ 
  Namibia & OTJOZONDJUPA & 1996 & 66.62 & 51.64 & 85.57 & RW2 \\ 
  Namibia & OTJOZONDJUPA & 1997 & 67.82 & 52.99 & 86.59 & RW2 \\ 
  Namibia & OTJOZONDJUPA & 1998 & 68.92 & 53.71 & 87.93 & RW2 \\ 
  Namibia & OTJOZONDJUPA & 1999 & 69.53 & 53.94 & 89.25 & RW2 \\ 
  Namibia & OTJOZONDJUPA & 2000 & 69.81 & 54.44 & 89.41 & RW2 \\ 
  Namibia & OTJOZONDJUPA & 2001 & 69.43 & 54.41 & 88.31 & RW2 \\ 
  Namibia & OTJOZONDJUPA & 2002 & 68.36 & 53.76 & 86.66 & RW2 \\ 
  Namibia & OTJOZONDJUPA & 2003 & 66.87 & 52.15 & 85.95 & RW2 \\ 
  Namibia & OTJOZONDJUPA & 2004 & 64.79 & 49.75 & 84.30 & RW2 \\ 
  Namibia & OTJOZONDJUPA & 2005 & 62.23 & 47.31 & 81.38 & RW2 \\ 
  Namibia & OTJOZONDJUPA & 2006 & 59.76 & 45.32 & 78.44 & RW2 \\ 
  Namibia & OTJOZONDJUPA & 2007 & 57.32 & 43.08 & 76.02 & RW2 \\ 
  Namibia & OTJOZONDJUPA & 2008 & 54.97 & 40.71 & 74.05 & RW2 \\ 
  Namibia & OTJOZONDJUPA & 2009 & 52.79 & 38.02 & 72.92 & RW2 \\ 
  Namibia & OTJOZONDJUPA & 2010 & 50.74 & 35.71 & 71.63 & RW2 \\ 
  Namibia & OTJOZONDJUPA & 2011 & 48.83 & 33.96 & 69.65 & RW2 \\ 
  Namibia & OTJOZONDJUPA & 2012 & 47.01 & 32.12 & 68.05 & RW2 \\ 
  Namibia & OTJOZONDJUPA & 2013 & 45.15 & 29.65 & 67.97 & RW2 \\ 
  Namibia & OTJOZONDJUPA & 2014 & 43.41 & 26.27 & 70.88 & RW2 \\ 
  Namibia & OTJOZONDJUPA & 2015 & 41.69 & 21.68 & 78.00 & RW2 \\ 
  Namibia & OTJOZONDJUPA & 2016 & 40.05 & 17.85 & 87.48 & RW2 \\ 
  Namibia & OTJOZONDJUPA & 2017 & 38.40 & 14.06 & 100.31 & RW2 \\ 
  Namibia & OTJOZONDJUPA & 2018 & 37.08 & 10.80 & 119.11 & RW2 \\ 
  Namibia & OTJOZONDJUPA & 2019 & 35.45 & 8.24 & 144.13 & RW2 \\ 
  Niger & ALL & 1980 & 316.20 & 307.31 & 324.93 & IHME \\ 
  Niger & ALL & 1980 & 314.93 & 242.57 & 397.50 & RW2 \\ 
  Niger & ALL & 1980 & 319.60 & 298.00 & 342.80 & UN \\ 
  Niger & ALL & 1981 & 314.66 & 305.98 & 323.35 & IHME \\ 
  Niger & ALL & 1981 & 319.23 & 267.09 & 376.11 & RW2 \\ 
  Niger & ALL & 1981 & 319.10 & 298.00 & 341.80 & UN \\ 
  Niger & ALL & 1982 & 313.57 & 305.25 & 321.79 & IHME \\ 
  Niger & ALL & 1982 & 323.58 & 275.14 & 376.00 & RW2 \\ 
  Niger & ALL & 1982 & 321.00 & 300.30 & 343.40 & UN \\ 
  Niger & ALL & 1983 & 312.14 & 303.94 & 320.16 & IHME \\ 
  Niger & ALL & 1983 & 327.42 & 272.60 & 386.54 & RW2 \\ 
  Niger & ALL & 1983 & 324.50 & 304.20 & 346.70 & UN \\ 
  Niger & ALL & 1984 & 312.09 & 304.16 & 320.09 & IHME \\ 
  Niger & ALL & 1984 & 331.37 & 270.20 & 396.63 & RW2 \\ 
  Niger & ALL & 1984 & 328.90 & 308.60 & 351.00 & UN \\ 
  Niger & ALL & 1985 & 311.07 & 303.08 & 318.84 & IHME \\ 
  Niger & ALL & 1985 & 334.45 & 279.58 & 395.41 & RW2 \\ 
  Niger & ALL & 1985 & 332.90 & 312.60 & 355.20 & UN \\ 
  Niger & ALL & 1986 & 309.49 & 301.57 & 317.13 & IHME \\ 
  Niger & ALL & 1986 & 336.40 & 285.71 & 391.61 & RW2 \\ 
  Niger & ALL & 1986 & 335.80 & 315.20 & 358.00 & UN \\ 
  Niger & ALL & 1987 & 307.24 & 299.77 & 314.34 & IHME \\ 
  Niger & ALL & 1987 & 337.13 & 288.88 & 389.81 & RW2 \\ 
  Niger & ALL & 1987 & 336.90 & 316.30 & 358.70 & UN \\ 
  Niger & ALL & 1988 & 303.90 & 296.92 & 311.24 & IHME \\ 
  Niger & ALL & 1988 & 336.06 & 284.77 & 390.74 & RW2 \\ 
  Niger & ALL & 1988 & 335.70 & 315.50 & 357.10 & UN \\ 
  Niger & ALL & 1989 & 300.35 & 293.09 & 307.96 & IHME \\ 
  Niger & ALL & 1989 & 333.21 & 278.93 & 391.54 & RW2 \\ 
  Niger & ALL & 1989 & 332.90 & 312.90 & 354.00 & UN \\ 
  Niger & ALL & 1990 & 296.08 & 288.57 & 303.94 & IHME \\ 
  Niger & ALL & 1990 & 328.69 & 275.18 & 386.66 & RW2 \\ 
  Niger & ALL & 1990 & 328.20 & 309.00 & 349.00 & UN \\ 
  Niger & ALL & 1991 & 291.11 & 283.84 & 298.92 & IHME \\ 
  Niger & ALL & 1991 & 321.90 & 272.27 & 374.38 & RW2 \\ 
  Niger & ALL & 1991 & 321.90 & 303.00 & 342.30 & UN \\ 
  Niger & ALL & 1992 & 285.98 & 278.99 & 294.29 & IHME \\ 
  Niger & ALL & 1992 & 313.15 & 266.00 & 363.71 & RW2 \\ 
  Niger & ALL & 1992 & 313.70 & 294.90 & 333.50 & UN \\ 
  Niger & ALL & 1993 & 280.04 & 272.94 & 287.98 & IHME \\ 
  Niger & ALL & 1993 & 302.81 & 255.87 & 355.25 & RW2 \\ 
  Niger & ALL & 1993 & 303.80 & 285.50 & 323.00 & UN \\ 
  Niger & ALL & 1994 & 273.08 & 265.69 & 281.04 & IHME \\ 
  Niger & ALL & 1994 & 291.23 & 242.74 & 348.01 & RW2 \\ 
  Niger & ALL & 1994 & 292.20 & 274.60 & 310.80 & UN \\ 
  Niger & ALL & 1995 & 265.81 & 258.47 & 273.61 & IHME \\ 
  Niger & ALL & 1995 & 278.41 & 230.49 & 331.09 & RW2 \\ 
  Niger & ALL & 1995 & 279.50 & 262.50 & 297.20 & UN \\ 
  Niger & ALL & 1996 & 257.80 & 250.28 & 265.32 & IHME \\ 
  Niger & ALL & 1996 & 266.51 & 222.99 & 315.87 & RW2 \\ 
  Niger & ALL & 1996 & 267.10 & 250.20 & 284.20 & UN \\ 
  Niger & ALL & 1997 & 249.88 & 242.38 & 257.49 & IHME \\ 
  Niger & ALL & 1997 & 255.29 & 214.72 & 300.99 & RW2 \\ 
  Niger & ALL & 1997 & 255.90 & 239.40 & 272.80 & UN \\ 
  Niger & ALL & 1998 & 241.19 & 233.76 & 248.71 & IHME \\ 
  Niger & ALL & 1998 & 244.98 & 203.97 & 292.14 & RW2 \\ 
  Niger & ALL & 1998 & 245.80 & 229.70 & 262.70 & UN \\ 
  Niger & ALL & 1999 & 232.41 & 224.78 & 240.15 & IHME \\ 
  Niger & ALL & 1999 & 235.23 & 192.28 & 282.70 & RW2 \\ 
  Niger & ALL & 1999 & 236.40 & 220.60 & 253.60 & UN \\ 
  Niger & ALL & 2000 & 223.68 & 216.23 & 231.35 & IHME \\ 
  Niger & ALL & 2000 & 226.61 & 185.77 & 274.66 & RW2 \\ 
  Niger & ALL & 2000 & 227.30 & 211.90 & 244.80 & UN \\ 
  Niger & ALL & 2001 & 214.43 & 206.95 & 221.79 & IHME \\ 
  Niger & ALL & 2001 & 217.00 & 179.21 & 260.74 & RW2 \\ 
  Niger & ALL & 2001 & 217.60 & 202.50 & 235.20 & UN \\ 
  Niger & ALL & 2002 & 204.70 & 197.55 & 211.79 & IHME \\ 
  Niger & ALL & 2002 & 206.89 & 171.86 & 246.95 & RW2 \\ 
  Niger & ALL & 2002 & 207.00 & 192.30 & 223.90 & UN \\ 
  Niger & ALL & 2003 & 194.96 & 187.91 & 202.03 & IHME \\ 
  Niger & ALL & 2003 & 196.32 & 161.72 & 236.08 & RW2 \\ 
  Niger & ALL & 2003 & 195.70 & 181.60 & 211.70 & UN \\ 
  Niger & ALL & 2004 & 185.18 & 177.97 & 192.46 & IHME \\ 
  Niger & ALL & 2004 & 184.98 & 148.49 & 227.58 & RW2 \\ 
  Niger & ALL & 2004 & 184.10 & 170.60 & 199.20 & UN \\ 
  Niger & ALL & 2005 & 175.88 & 168.56 & 183.36 & IHME \\ 
  Niger & ALL & 2005 & 173.30 & 135.64 & 217.58 & RW2 \\ 
  Niger & ALL & 2005 & 172.70 & 159.60 & 187.40 & UN \\ 
  Niger & ALL & 2006 & 166.58 & 158.94 & 173.96 & IHME \\ 
  Niger & ALL & 2006 & 161.97 & 128.78 & 200.68 & RW2 \\ 
  Niger & ALL & 2006 & 161.80 & 148.60 & 176.10 & UN \\ 
  Niger & ALL & 2007 & 158.17 & 150.05 & 165.88 & IHME \\ 
  Niger & ALL & 2007 & 151.05 & 123.25 & 183.50 & RW2 \\ 
  Niger & ALL & 2007 & 151.30 & 138.00 & 165.80 & UN \\ 
  Niger & ALL & 2008 & 149.98 & 141.85 & 158.26 & IHME \\ 
  Niger & ALL & 2008 & 140.75 & 113.14 & 174.46 & RW2 \\ 
  Niger & ALL & 2008 & 141.30 & 127.60 & 156.80 & UN \\ 
  Niger & ALL & 2009 & 143.16 & 134.56 & 151.69 & IHME \\ 
  Niger & ALL & 2009 & 130.78 & 94.97 & 178.47 & RW2 \\ 
  Niger & ALL & 2009 & 132.10 & 117.30 & 148.80 & UN \\ 
  Niger & ALL & 2010 & 137.04 & 128.09 & 146.66 & IHME \\ 
  Niger & ALL & 2010 & 121.53 & 72.81 & 198.17 & RW2 \\ 
  Niger & ALL & 2010 & 123.60 & 107.30 & 142.40 & UN \\ 
  Niger & ALL & 2011 & 131.73 & 122.32 & 142.41 & IHME \\ 
  Niger & ALL & 2011 & 112.98 & 54.49 & 219.39 & RW2 \\ 
  Niger & ALL & 2011 & 116.10 & 97.80 & 137.20 & UN \\ 
  Niger & ALL & 2012 & 126.84 & 116.69 & 138.36 & IHME \\ 
  Niger & ALL & 2012 & 104.75 & 39.32 & 248.89 & RW2 \\ 
  Niger & ALL & 2012 & 109.60 & 89.90 & 133.60 & UN \\ 
  Niger & ALL & 2013 & 121.99 & 111.28 & 134.74 & IHME \\ 
  Niger & ALL & 2013 & 97.44 & 27.98 & 284.27 & RW2 \\ 
  Niger & ALL & 2013 & 104.10 & 82.50 & 131.10 & UN \\ 
  Niger & ALL & 2014 & 117.20 & 105.88 & 130.55 & IHME \\ 
  Niger & ALL & 2014 & 90.22 & 19.34 & 328.28 & RW2 \\ 
  Niger & ALL & 2014 & 99.60 & 76.20 & 129.50 & UN \\ 
  Niger & ALL & 2015 & 112.77 & 101.27 & 127.06 & IHME \\ 
  Niger & ALL & 2015 & 83.27 & 13.29 & 376.12 & RW2 \\ 
  Niger & ALL & 2015 & 95.50 & 70.50 & 128.30 & UN \\ 
  Niger & ALL & 2016 & 77.64 & 9.01 & 439.12 & RW2 \\ 
  Niger & ALL & 2017 & 71.44 & 5.80 & 504.48 & RW2 \\ 
  Niger & ALL & 2018 & 66.06 & 3.82 & 578.44 & RW2 \\ 
  Niger & ALL & 2019 & 60.88 & 2.23 & 638.18 & RW2 \\ 
  Niger & ALL & 80-84 & 315.24 & 327.65 & 303.10 & HT-Direct \\ 
  Niger & ALL & 85-89 & 327.20 & 337.64 & 316.93 & HT-Direct \\ 
  Niger & ALL & 90-94 & 297.18 & 306.91 & 287.63 & HT-Direct \\ 
  Niger & ALL & 95-99 & 246.40 & 255.27 & 237.73 & HT-Direct \\ 
  Niger & ALL & 00-04 & 206.27 & 215.43 & 197.40 & HT-Direct \\ 
  Niger & ALL & 05-09 & 146.56 & 154.99 & 138.51 & HT-Direct \\ 
  Niger & ALL & 15-19 & 71.45 & 5.85 & 499.15 & RW2 \\ 
  Niger & DOSSO & 1980 & 264.42 & 205.10 & 335.35 & RW2 \\ 
  Niger & DOSSO & 1981 & 268.36 & 225.64 & 316.57 & RW2 \\ 
  Niger & DOSSO & 1982 & 272.35 & 233.14 & 314.69 & RW2 \\ 
  Niger & DOSSO & 1983 & 275.85 & 232.77 & 323.52 & RW2 \\ 
  Niger & DOSSO & 1984 & 279.15 & 232.86 & 329.67 & RW2 \\ 
  Niger & DOSSO & 1985 & 281.97 & 238.90 & 328.87 & RW2 \\ 
  Niger & DOSSO & 1986 & 283.01 & 244.70 & 325.56 & RW2 \\ 
  Niger & DOSSO & 1987 & 282.79 & 246.61 & 322.05 & RW2 \\ 
  Niger & DOSSO & 1988 & 280.86 & 242.79 & 322.08 & RW2 \\ 
  Niger & DOSSO & 1989 & 277.21 & 236.49 & 322.20 & RW2 \\ 
  Niger & DOSSO & 1990 & 272.23 & 231.79 & 316.18 & RW2 \\ 
  Niger & DOSSO & 1991 & 266.48 & 229.33 & 305.55 & RW2 \\ 
  Niger & DOSSO & 1992 & 259.71 & 224.98 & 297.04 & RW2 \\ 
  Niger & DOSSO & 1993 & 252.78 & 218.04 & 292.04 & RW2 \\ 
  Niger & DOSSO & 1994 & 246.09 & 209.05 & 288.48 & RW2 \\ 
  Niger & DOSSO & 1995 & 239.06 & 202.03 & 280.27 & RW2 \\ 
  Niger & DOSSO & 1996 & 233.52 & 199.97 & 270.10 & RW2 \\ 
  Niger & DOSSO & 1997 & 228.99 & 197.52 & 264.39 & RW2 \\ 
  Niger & DOSSO & 1998 & 225.59 & 193.11 & 261.64 & RW2 \\ 
  Niger & DOSSO & 1999 & 222.72 & 187.86 & 260.79 & RW2 \\ 
  Niger & DOSSO & 2000 & 220.95 & 187.14 & 260.29 & RW2 \\ 
  Niger & DOSSO & 2001 & 218.41 & 187.08 & 253.91 & RW2 \\ 
  Niger & DOSSO & 2002 & 215.20 & 185.81 & 248.11 & RW2 \\ 
  Niger & DOSSO & 2003 & 211.18 & 180.38 & 245.48 & RW2 \\ 
  Niger & DOSSO & 2004 & 206.39 & 172.39 & 244.33 & RW2 \\ 
  Niger & DOSSO & 2005 & 200.47 & 164.05 & 241.43 & RW2 \\ 
  Niger & DOSSO & 2006 & 194.51 & 161.75 & 232.11 & RW2 \\ 
  Niger & DOSSO & 2007 & 188.59 & 159.59 & 221.93 & RW2 \\ 
  Niger & DOSSO & 2008 & 182.62 & 151.15 & 219.65 & RW2 \\ 
  Niger & DOSSO & 2009 & 176.55 & 133.50 & 230.20 & RW2 \\ 
  Niger & DOSSO & 2010 & 171.00 & 109.79 & 256.97 & RW2 \\ 
  Niger & DOSSO & 2011 & 165.38 & 89.24 & 287.04 & RW2 \\ 
  Niger & DOSSO & 2012 & 159.88 & 70.06 & 324.39 & RW2 \\ 
  Niger & DOSSO & 2013 & 154.58 & 53.88 & 368.68 & RW2 \\ 
  Niger & DOSSO & 2014 & 148.91 & 39.60 & 418.61 & RW2 \\ 
  Niger & DOSSO & 2015 & 144.44 & 29.46 & 479.11 & RW2 \\ 
  Niger & DOSSO & 2016 & 139.14 & 20.96 & 540.48 & RW2 \\ 
  Niger & DOSSO & 2017 & 134.97 & 15.00 & 607.73 & RW2 \\ 
  Niger & DOSSO & 2018 & 129.59 & 10.54 & 669.92 & RW2 \\ 
  Niger & DOSSO & 2019 & 125.87 & 7.38 & 732.23 & RW2 \\ 
  Niger & MARADI & 1980 & 362.82 & 291.94 & 440.40 & RW2 \\ 
  Niger & MARADI & 1981 & 369.10 & 320.76 & 422.09 & RW2 \\ 
  Niger & MARADI & 1982 & 375.38 & 330.55 & 422.49 & RW2 \\ 
  Niger & MARADI & 1983 & 381.42 & 330.12 & 434.67 & RW2 \\ 
  Niger & MARADI & 1984 & 387.17 & 331.78 & 444.14 & RW2 \\ 
  Niger & MARADI & 1985 & 392.90 & 342.68 & 446.44 & RW2 \\ 
  Niger & MARADI & 1986 & 397.26 & 351.62 & 444.59 & RW2 \\ 
  Niger & MARADI & 1987 & 400.33 & 357.07 & 444.63 & RW2 \\ 
  Niger & MARADI & 1988 & 401.72 & 356.13 & 448.83 & RW2 \\ 
  Niger & MARADI & 1989 & 401.37 & 352.63 & 452.52 & RW2 \\ 
  Niger & MARADI & 1990 & 398.26 & 349.85 & 448.66 & RW2 \\ 
  Niger & MARADI & 1991 & 393.00 & 347.83 & 438.54 & RW2 \\ 
  Niger & MARADI & 1992 & 384.72 & 342.42 & 428.62 & RW2 \\ 
  Niger & MARADI & 1993 & 373.64 & 330.55 & 420.22 & RW2 \\ 
  Niger & MARADI & 1994 & 360.59 & 314.91 & 411.21 & RW2 \\ 
  Niger & MARADI & 1995 & 345.41 & 300.31 & 393.57 & RW2 \\ 
  Niger & MARADI & 1996 & 329.97 & 288.91 & 373.75 & RW2 \\ 
  Niger & MARADI & 1997 & 314.77 & 277.39 & 355.69 & RW2 \\ 
  Niger & MARADI & 1998 & 299.69 & 261.52 & 341.50 & RW2 \\ 
  Niger & MARADI & 1999 & 284.84 & 243.83 & 328.44 & RW2 \\ 
  Niger & MARADI & 2000 & 270.51 & 231.17 & 314.95 & RW2 \\ 
  Niger & MARADI & 2001 & 254.88 & 220.01 & 294.09 & RW2 \\ 
  Niger & MARADI & 2002 & 238.57 & 206.54 & 273.38 & RW2 \\ 
  Niger & MARADI & 2003 & 221.40 & 189.35 & 257.09 & RW2 \\ 
  Niger & MARADI & 2004 & 203.96 & 170.52 & 242.61 & RW2 \\ 
  Niger & MARADI & 2005 & 186.23 & 151.95 & 225.39 & RW2 \\ 
  Niger & MARADI & 2006 & 169.28 & 139.48 & 203.30 & RW2 \\ 
  Niger & MARADI & 2007 & 153.19 & 127.99 & 182.06 & RW2 \\ 
  Niger & MARADI & 2008 & 138.21 & 112.31 & 168.59 & RW2 \\ 
  Niger & MARADI & 2009 & 124.55 & 91.90 & 166.13 & RW2 \\ 
  Niger & MARADI & 2010 & 112.06 & 69.72 & 174.70 & RW2 \\ 
  Niger & MARADI & 2011 & 100.56 & 51.68 & 185.54 & RW2 \\ 
  Niger & MARADI & 2012 & 90.37 & 37.17 & 201.06 & RW2 \\ 
  Niger & MARADI & 2013 & 80.97 & 26.79 & 220.93 & RW2 \\ 
  Niger & MARADI & 2014 & 72.31 & 18.30 & 244.33 & RW2 \\ 
  Niger & MARADI & 2015 & 64.64 & 12.25 & 277.38 & RW2 \\ 
  Niger & MARADI & 2016 & 57.61 & 7.94 & 310.83 & RW2 \\ 
  Niger & MARADI & 2017 & 51.57 & 5.25 & 352.91 & RW2 \\ 
  Niger & MARADI & 2018 & 45.67 & 3.34 & 397.43 & RW2 \\ 
  Niger & MARADI & 2019 & 40.98 & 2.09 & 459.61 & RW2 \\ 
  Niger & NIAMEY & 1980 & 143.08 & 104.71 & 192.94 & RW2 \\ 
  Niger & NIAMEY & 1981 & 147.02 & 117.73 & 182.54 & RW2 \\ 
  Niger & NIAMEY & 1982 & 150.77 & 124.07 & 181.87 & RW2 \\ 
  Niger & NIAMEY & 1983 & 154.49 & 125.59 & 188.26 & RW2 \\ 
  Niger & NIAMEY & 1984 & 157.95 & 126.57 & 194.30 & RW2 \\ 
  Niger & NIAMEY & 1985 & 161.33 & 132.48 & 195.48 & RW2 \\ 
  Niger & NIAMEY & 1986 & 163.67 & 136.98 & 194.09 & RW2 \\ 
  Niger & NIAMEY & 1987 & 165.04 & 139.38 & 194.20 & RW2 \\ 
  Niger & NIAMEY & 1988 & 165.36 & 138.31 & 196.19 & RW2 \\ 
  Niger & NIAMEY & 1989 & 164.57 & 135.99 & 197.49 & RW2 \\ 
  Niger & NIAMEY & 1990 & 162.68 & 134.28 & 195.12 & RW2 \\ 
  Niger & NIAMEY & 1991 & 160.11 & 133.97 & 189.80 & RW2 \\ 
  Niger & NIAMEY & 1992 & 156.73 & 131.77 & 185.03 & RW2 \\ 
  Niger & NIAMEY & 1993 & 152.81 & 127.80 & 181.95 & RW2 \\ 
  Niger & NIAMEY & 1994 & 148.90 & 122.71 & 179.72 & RW2 \\ 
  Niger & NIAMEY & 1995 & 144.46 & 118.67 & 174.45 & RW2 \\ 
  Niger & NIAMEY & 1996 & 140.78 & 117.02 & 168.38 & RW2 \\ 
  Niger & NIAMEY & 1997 & 137.40 & 115.03 & 163.88 & RW2 \\ 
  Niger & NIAMEY & 1998 & 134.43 & 111.89 & 161.12 & RW2 \\ 
  Niger & NIAMEY & 1999 & 131.71 & 107.25 & 159.72 & RW2 \\ 
  Niger & NIAMEY & 2000 & 129.37 & 105.45 & 158.06 & RW2 \\ 
  Niger & NIAMEY & 2001 & 126.36 & 103.95 & 152.75 & RW2 \\ 
  Niger & NIAMEY & 2002 & 122.80 & 101.48 & 147.85 & RW2 \\ 
  Niger & NIAMEY & 2003 & 118.61 & 96.70 & 144.91 & RW2 \\ 
  Niger & NIAMEY & 2004 & 113.98 & 90.30 & 142.76 & RW2 \\ 
  Niger & NIAMEY & 2005 & 108.69 & 83.82 & 139.44 & RW2 \\ 
  Niger & NIAMEY & 2006 & 103.63 & 79.72 & 133.40 & RW2 \\ 
  Niger & NIAMEY & 2007 & 98.51 & 74.86 & 127.78 & RW2 \\ 
  Niger & NIAMEY & 2008 & 93.67 & 68.20 & 126.57 & RW2 \\ 
  Niger & NIAMEY & 2009 & 88.82 & 58.87 & 131.11 & RW2 \\ 
  Niger & NIAMEY & 2010 & 84.49 & 47.78 & 144.92 & RW2 \\ 
  Niger & NIAMEY & 2011 & 80.11 & 37.63 & 162.03 & RW2 \\ 
  Niger & NIAMEY & 2012 & 76.08 & 28.92 & 184.29 & RW2 \\ 
  Niger & NIAMEY & 2013 & 72.37 & 21.70 & 215.95 & RW2 \\ 
  Niger & NIAMEY & 2014 & 68.85 & 15.82 & 249.66 & RW2 \\ 
  Niger & NIAMEY & 2015 & 65.39 & 11.37 & 292.60 & RW2 \\ 
  Niger & NIAMEY & 2016 & 61.83 & 8.00 & 342.33 & RW2 \\ 
  Niger & NIAMEY & 2017 & 58.94 & 5.59 & 403.39 & RW2 \\ 
  Niger & NIAMEY & 2018 & 55.58 & 3.77 & 471.92 & RW2 \\ 
  Niger & NIAMEY & 2019 & 52.85 & 2.53 & 549.07 & RW2 \\ 
  Niger & TASHOUA/AGADEZ & 1980 & 322.20 & 256.37 & 397.36 & RW2 \\ 
  Niger & TASHOUA/AGADEZ & 1981 & 325.61 & 279.09 & 376.98 & RW2 \\ 
  Niger & TASHOUA/AGADEZ & 1982 & 329.06 & 286.22 & 374.80 & RW2 \\ 
  Niger & TASHOUA/AGADEZ & 1983 & 331.99 & 283.92 & 383.03 & RW2 \\ 
  Niger & TASHOUA/AGADEZ & 1984 & 334.55 & 282.65 & 390.35 & RW2 \\ 
  Niger & TASHOUA/AGADEZ & 1985 & 336.80 & 289.53 & 387.95 & RW2 \\ 
  Niger & TASHOUA/AGADEZ & 1986 & 337.23 & 295.32 & 382.39 & RW2 \\ 
  Niger & TASHOUA/AGADEZ & 1987 & 336.08 & 295.87 & 378.58 & RW2 \\ 
  Niger & TASHOUA/AGADEZ & 1988 & 333.15 & 291.23 & 377.54 & RW2 \\ 
  Niger & TASHOUA/AGADEZ & 1989 & 328.21 & 283.16 & 376.11 & RW2 \\ 
  Niger & TASHOUA/AGADEZ & 1990 & 321.28 & 276.60 & 367.70 & RW2 \\ 
  Niger & TASHOUA/AGADEZ & 1991 & 312.58 & 271.88 & 355.45 & RW2 \\ 
  Niger & TASHOUA/AGADEZ & 1992 & 302.11 & 264.81 & 343.05 & RW2 \\ 
  Niger & TASHOUA/AGADEZ & 1993 & 290.61 & 253.19 & 332.79 & RW2 \\ 
  Niger & TASHOUA/AGADEZ & 1994 & 278.11 & 237.98 & 322.29 & RW2 \\ 
  Niger & TASHOUA/AGADEZ & 1995 & 265.22 & 226.28 & 307.48 & RW2 \\ 
  Niger & TASHOUA/AGADEZ & 1996 & 253.30 & 218.64 & 291.92 & RW2 \\ 
  Niger & TASHOUA/AGADEZ & 1997 & 242.02 & 210.19 & 277.35 & RW2 \\ 
  Niger & TASHOUA/AGADEZ & 1998 & 231.92 & 199.11 & 267.83 & RW2 \\ 
  Niger & TASHOUA/AGADEZ & 1999 & 222.24 & 187.38 & 259.84 & RW2 \\ 
  Niger & TASHOUA/AGADEZ & 2000 & 213.62 & 180.41 & 251.77 & RW2 \\ 
  Niger & TASHOUA/AGADEZ & 2001 & 204.14 & 173.90 & 238.81 & RW2 \\ 
  Niger & TASHOUA/AGADEZ & 2002 & 194.43 & 166.30 & 226.01 & RW2 \\ 
  Niger & TASHOUA/AGADEZ & 2003 & 183.84 & 154.98 & 215.99 & RW2 \\ 
  Niger & TASHOUA/AGADEZ & 2004 & 173.02 & 142.61 & 208.22 & RW2 \\ 
  Niger & TASHOUA/AGADEZ & 2005 & 161.58 & 130.43 & 197.96 & RW2 \\ 
  Niger & TASHOUA/AGADEZ & 2006 & 150.52 & 122.71 & 182.88 & RW2 \\ 
  Niger & TASHOUA/AGADEZ & 2007 & 139.83 & 115.72 & 167.56 & RW2 \\ 
  Niger & TASHOUA/AGADEZ & 2008 & 129.85 & 105.12 & 159.26 & RW2 \\ 
  Niger & TASHOUA/AGADEZ & 2009 & 120.32 & 88.61 & 161.48 & RW2 \\ 
  Niger & TASHOUA/AGADEZ & 2010 & 111.58 & 69.97 & 174.55 & RW2 \\ 
  Niger & TASHOUA/AGADEZ & 2011 & 102.95 & 53.34 & 189.66 & RW2 \\ 
  Niger & TASHOUA/AGADEZ & 2012 & 95.19 & 39.96 & 209.22 & RW2 \\ 
  Niger & TASHOUA/AGADEZ & 2013 & 88.02 & 29.16 & 235.80 & RW2 \\ 
  Niger & TASHOUA/AGADEZ & 2014 & 81.45 & 20.74 & 267.24 & RW2 \\ 
  Niger & TASHOUA/AGADEZ & 2015 & 75.06 & 14.66 & 307.46 & RW2 \\ 
  Niger & TASHOUA/AGADEZ & 2016 & 69.33 & 9.84 & 347.94 & RW2 \\ 
  Niger & TASHOUA/AGADEZ & 2017 & 63.81 & 6.68 & 400.05 & RW2 \\ 
  Niger & TASHOUA/AGADEZ & 2018 & 58.82 & 4.53 & 461.57 & RW2 \\ 
  Niger & TASHOUA/AGADEZ & 2019 & 54.32 & 2.84 & 519.38 & RW2 \\ 
  Niger & TILLABERI & 1980 & 264.79 & 203.71 & 336.06 & RW2 \\ 
  Niger & TILLABERI & 1981 & 270.77 & 226.34 & 320.21 & RW2 \\ 
  Niger & TILLABERI & 1982 & 276.52 & 236.31 & 320.81 & RW2 \\ 
  Niger & TILLABERI & 1983 & 282.56 & 237.86 & 330.89 & RW2 \\ 
  Niger & TILLABERI & 1984 & 287.48 & 239.09 & 339.25 & RW2 \\ 
  Niger & TILLABERI & 1985 & 292.05 & 247.96 & 340.27 & RW2 \\ 
  Niger & TILLABERI & 1986 & 294.23 & 254.55 & 337.08 & RW2 \\ 
  Niger & TILLABERI & 1987 & 294.40 & 256.72 & 335.36 & RW2 \\ 
  Niger & TILLABERI & 1988 & 292.21 & 253.25 & 334.02 & RW2 \\ 
  Niger & TILLABERI & 1989 & 287.86 & 246.49 & 333.73 & RW2 \\ 
  Niger & TILLABERI & 1990 & 281.45 & 240.60 & 325.97 & RW2 \\ 
  Niger & TILLABERI & 1991 & 273.42 & 236.34 & 313.72 & RW2 \\ 
  Niger & TILLABERI & 1992 & 264.34 & 229.55 & 301.94 & RW2 \\ 
  Niger & TILLABERI & 1993 & 254.81 & 219.61 & 293.86 & RW2 \\ 
  Niger & TILLABERI & 1994 & 245.05 & 207.98 & 287.16 & RW2 \\ 
  Niger & TILLABERI & 1995 & 235.38 & 198.63 & 275.61 & RW2 \\ 
  Niger & TILLABERI & 1996 & 226.81 & 193.50 & 263.71 & RW2 \\ 
  Niger & TILLABERI & 1997 & 219.22 & 187.94 & 253.83 & RW2 \\ 
  Niger & TILLABERI & 1998 & 212.58 & 181.18 & 247.46 & RW2 \\ 
  Niger & TILLABERI & 1999 & 206.61 & 172.97 & 243.47 & RW2 \\ 
  Niger & TILLABERI & 2000 & 201.90 & 169.84 & 238.98 & RW2 \\ 
  Niger & TILLABERI & 2001 & 196.30 & 167.05 & 229.79 & RW2 \\ 
  Niger & TILLABERI & 2002 & 190.27 & 163.05 & 221.17 & RW2 \\ 
  Niger & TILLABERI & 2003 & 183.55 & 155.33 & 215.30 & RW2 \\ 
  Niger & TILLABERI & 2004 & 176.21 & 146.15 & 211.29 & RW2 \\ 
  Niger & TILLABERI & 2005 & 168.21 & 136.24 & 204.72 & RW2 \\ 
  Niger & TILLABERI & 2006 & 160.20 & 131.13 & 194.03 & RW2 \\ 
  Niger & TILLABERI & 2007 & 152.59 & 126.49 & 182.64 & RW2 \\ 
  Niger & TILLABERI & 2008 & 144.95 & 117.31 & 178.91 & RW2 \\ 
  Niger & TILLABERI & 2009 & 137.53 & 101.33 & 184.87 & RW2 \\ 
  Niger & TILLABERI & 2010 & 130.50 & 81.45 & 203.47 & RW2 \\ 
  Niger & TILLABERI & 2011 & 124.13 & 64.59 & 224.42 & RW2 \\ 
  Niger & TILLABERI & 2012 & 117.75 & 49.50 & 253.13 & RW2 \\ 
  Niger & TILLABERI & 2013 & 110.96 & 36.94 & 284.78 & RW2 \\ 
  Niger & TILLABERI & 2014 & 105.61 & 27.39 & 331.17 & RW2 \\ 
  Niger & TILLABERI & 2015 & 99.89 & 19.72 & 374.35 & RW2 \\ 
  Niger & TILLABERI & 2016 & 94.75 & 13.76 & 438.73 & RW2 \\ 
  Niger & TILLABERI & 2017 & 90.22 & 9.62 & 498.08 & RW2 \\ 
  Niger & TILLABERI & 2018 & 85.46 & 6.58 & 561.23 & RW2 \\ 
  Niger & TILLABERI & 2019 & 80.15 & 4.31 & 635.71 & RW2 \\ 
  Niger & ZINDA/DIFFA & 1980 & 362.09 & 287.49 & 444.52 & RW2 \\ 
  Niger & ZINDA/DIFFA & 1981 & 365.77 & 313.19 & 422.10 & RW2 \\ 
  Niger & ZINDA/DIFFA & 1982 & 369.18 & 321.53 & 419.27 & RW2 \\ 
  Niger & ZINDA/DIFFA & 1983 & 372.26 & 319.83 & 427.23 & RW2 \\ 
  Niger & ZINDA/DIFFA & 1984 & 374.95 & 319.55 & 432.44 & RW2 \\ 
  Niger & ZINDA/DIFFA & 1985 & 377.40 & 326.48 & 430.86 & RW2 \\ 
  Niger & ZINDA/DIFFA & 1986 & 378.37 & 332.72 & 425.63 & RW2 \\ 
  Niger & ZINDA/DIFFA & 1987 & 377.89 & 335.34 & 421.80 & RW2 \\ 
  Niger & ZINDA/DIFFA & 1988 & 375.65 & 330.51 & 422.19 & RW2 \\ 
  Niger & ZINDA/DIFFA & 1989 & 371.46 & 323.18 & 421.92 & RW2 \\ 
  Niger & ZINDA/DIFFA & 1990 & 365.06 & 317.57 & 414.82 & RW2 \\ 
  Niger & ZINDA/DIFFA & 1991 & 356.75 & 313.46 & 402.48 & RW2 \\ 
  Niger & ZINDA/DIFFA & 1992 & 346.29 & 305.27 & 389.68 & RW2 \\ 
  Niger & ZINDA/DIFFA & 1993 & 333.95 & 292.88 & 378.44 & RW2 \\ 
  Niger & ZINDA/DIFFA & 1994 & 320.37 & 276.91 & 368.61 & RW2 \\ 
  Niger & ZINDA/DIFFA & 1995 & 305.68 & 262.66 & 351.66 & RW2 \\ 
  Niger & ZINDA/DIFFA & 1996 & 291.88 & 252.90 & 333.77 & RW2 \\ 
  Niger & ZINDA/DIFFA & 1997 & 278.75 & 242.77 & 317.48 & RW2 \\ 
  Niger & ZINDA/DIFFA & 1998 & 266.42 & 230.21 & 306.45 & RW2 \\ 
  Niger & ZINDA/DIFFA & 1999 & 254.78 & 216.22 & 297.13 & RW2 \\ 
  Niger & ZINDA/DIFFA & 2000 & 244.10 & 206.50 & 286.07 & RW2 \\ 
  Niger & ZINDA/DIFFA & 2001 & 232.46 & 199.16 & 269.83 & RW2 \\ 
  Niger & ZINDA/DIFFA & 2002 & 220.27 & 189.42 & 254.58 & RW2 \\ 
  Niger & ZINDA/DIFFA & 2003 & 207.17 & 176.15 & 242.19 & RW2 \\ 
  Niger & ZINDA/DIFFA & 2004 & 193.66 & 160.91 & 231.21 & RW2 \\ 
  Niger & ZINDA/DIFFA & 2005 & 179.60 & 145.39 & 219.05 & RW2 \\ 
  Niger & ZINDA/DIFFA & 2006 & 166.04 & 135.64 & 201.30 & RW2 \\ 
  Niger & ZINDA/DIFFA & 2007 & 152.94 & 126.56 & 183.51 & RW2 \\ 
  Niger & ZINDA/DIFFA & 2008 & 140.63 & 113.55 & 173.19 & RW2 \\ 
  Niger & ZINDA/DIFFA & 2009 & 128.97 & 95.02 & 173.16 & RW2 \\ 
  Niger & ZINDA/DIFFA & 2010 & 118.52 & 73.26 & 185.61 & RW2 \\ 
  Niger & ZINDA/DIFFA & 2011 & 108.25 & 55.94 & 199.49 & RW2 \\ 
  Niger & ZINDA/DIFFA & 2012 & 99.25 & 41.23 & 218.51 & RW2 \\ 
  Niger & ZINDA/DIFFA & 2013 & 90.52 & 29.40 & 244.10 & RW2 \\ 
  Niger & ZINDA/DIFFA & 2014 & 82.70 & 20.60 & 270.61 & RW2 \\ 
  Niger & ZINDA/DIFFA & 2015 & 75.57 & 14.38 & 313.31 & RW2 \\ 
  Niger & ZINDA/DIFFA & 2016 & 68.85 & 9.84 & 355.84 & RW2 \\ 
  Niger & ZINDA/DIFFA & 2017 & 62.88 & 6.49 & 409.91 & RW2 \\ 
  Niger & ZINDA/DIFFA & 2018 & 56.91 & 4.08 & 463.28 & RW2 \\ 
  Niger & ZINDA/DIFFA & 2019 & 51.70 & 2.71 & 519.85 & RW2 \\ 
  Nigeria & ALL & 1980 & 223.97 & 216.34 & 232.08 & IHME \\ 
  Nigeria & ALL & 1980 & 210.04 & 163.07 & 265.74 & RW2 \\ 
  Nigeria & ALL & 1980 & 214.40 & 197.70 & 232.30 & UN \\ 
  Nigeria & ALL & 1981 & 220.19 & 212.93 & 227.88 & IHME \\ 
  Nigeria & ALL & 1981 & 210.31 & 176.19 & 248.53 & RW2 \\ 
  Nigeria & ALL & 1981 & 211.50 & 195.40 & 228.50 & UN \\ 
  Nigeria & ALL & 1982 & 216.92 & 210.18 & 224.73 & IHME \\ 
  Nigeria & ALL & 1982 & 210.62 & 179.41 & 245.61 & RW2 \\ 
  Nigeria & ALL & 1982 & 209.60 & 194.20 & 226.10 & UN \\ 
  Nigeria & ALL & 1983 & 214.92 & 208.45 & 222.21 & IHME \\ 
  Nigeria & ALL & 1983 & 210.69 & 176.26 & 249.84 & RW2 \\ 
  Nigeria & ALL & 1983 & 208.90 & 193.90 & 224.80 & UN \\ 
  Nigeria & ALL & 1984 & 213.61 & 206.97 & 220.44 & IHME \\ 
  Nigeria & ALL & 1984 & 211.12 & 173.24 & 254.27 & RW2 \\ 
  Nigeria & ALL & 1984 & 208.80 & 194.40 & 224.30 & UN \\ 
  Nigeria & ALL & 1985 & 211.91 & 205.57 & 218.77 & IHME \\ 
  Nigeria & ALL & 1985 & 211.08 & 176.51 & 250.44 & RW2 \\ 
  Nigeria & ALL & 1985 & 209.50 & 195.50 & 224.70 & UN \\ 
  Nigeria & ALL & 1986 & 211.24 & 204.85 & 217.91 & IHME \\ 
  Nigeria & ALL & 1986 & 211.33 & 179.34 & 247.22 & RW2 \\ 
  Nigeria & ALL & 1986 & 210.60 & 196.70 & 225.40 & UN \\ 
  Nigeria & ALL & 1987 & 211.06 & 204.72 & 217.38 & IHME \\ 
  Nigeria & ALL & 1987 & 211.66 & 181.26 & 246.16 & RW2 \\ 
  Nigeria & ALL & 1987 & 211.50 & 197.90 & 226.10 & UN \\ 
  Nigeria & ALL & 1988 & 210.59 & 204.01 & 216.99 & IHME \\ 
  Nigeria & ALL & 1988 & 211.84 & 179.81 & 247.77 & RW2 \\ 
  Nigeria & ALL & 1988 & 212.20 & 198.70 & 226.50 & UN \\ 
  Nigeria & ALL & 1989 & 209.54 & 203.15 & 215.96 & IHME \\ 
  Nigeria & ALL & 1989 & 211.98 & 178.07 & 250.39 & RW2 \\ 
  Nigeria & ALL & 1989 & 212.60 & 199.10 & 226.60 & UN \\ 
  Nigeria & ALL & 1990 & 207.88 & 201.67 & 214.16 & IHME \\ 
  Nigeria & ALL & 1990 & 212.10 & 178.50 & 250.64 & RW2 \\ 
  Nigeria & ALL & 1990 & 212.50 & 199.20 & 226.30 & UN \\ 
  Nigeria & ALL & 1991 & 206.28 & 200.16 & 212.66 & IHME \\ 
  Nigeria & ALL & 1991 & 211.92 & 180.21 & 247.23 & RW2 \\ 
  Nigeria & ALL & 1991 & 212.20 & 199.00 & 225.90 & UN \\ 
  Nigeria & ALL & 1992 & 204.52 & 198.59 & 210.83 & IHME \\ 
  Nigeria & ALL & 1992 & 211.44 & 180.53 & 245.62 & RW2 \\ 
  Nigeria & ALL & 1992 & 211.70 & 198.50 & 225.20 & UN \\ 
  Nigeria & ALL & 1993 & 202.48 & 196.10 & 208.97 & IHME \\ 
  Nigeria & ALL & 1993 & 210.60 & 178.95 & 245.91 & RW2 \\ 
  Nigeria & ALL & 1993 & 211.00 & 197.90 & 224.30 & UN \\ 
  Nigeria & ALL & 1994 & 200.35 & 193.94 & 206.59 & IHME \\ 
  Nigeria & ALL & 1994 & 209.16 & 175.72 & 247.33 & RW2 \\ 
  Nigeria & ALL & 1994 & 209.70 & 196.90 & 223.00 & UN \\ 
  Nigeria & ALL & 1995 & 198.20 & 192.16 & 204.09 & IHME \\ 
  Nigeria & ALL & 1995 & 207.40 & 174.68 & 244.51 & RW2 \\ 
  Nigeria & ALL & 1995 & 207.80 & 195.00 & 220.80 & UN \\ 
  Nigeria & ALL & 1996 & 195.53 & 189.49 & 201.47 & IHME \\ 
  Nigeria & ALL & 1996 & 204.66 & 174.08 & 239.63 & RW2 \\ 
  Nigeria & ALL & 1996 & 205.10 & 192.40 & 218.00 & UN \\ 
  Nigeria & ALL & 1997 & 192.56 & 186.47 & 198.49 & IHME \\ 
  Nigeria & ALL & 1997 & 201.21 & 172.04 & 234.00 & RW2 \\ 
  Nigeria & ALL & 1997 & 201.40 & 189.00 & 214.20 & UN \\ 
  Nigeria & ALL & 1998 & 189.51 & 183.47 & 195.32 & IHME \\ 
  Nigeria & ALL & 1998 & 197.11 & 167.47 & 231.41 & RW2 \\ 
  Nigeria & ALL & 1998 & 197.00 & 184.90 & 209.50 & UN \\ 
  Nigeria & ALL & 1999 & 186.40 & 180.16 & 192.25 & IHME \\ 
  Nigeria & ALL & 1999 & 192.31 & 161.38 & 227.32 & RW2 \\ 
  Nigeria & ALL & 1999 & 192.00 & 180.40 & 204.20 & UN \\ 
  Nigeria & ALL & 2000 & 183.06 & 177.12 & 189.01 & IHME \\ 
  Nigeria & ALL & 2000 & 186.80 & 156.65 & 220.71 & RW2 \\ 
  Nigeria & ALL & 2000 & 186.80 & 175.40 & 198.60 & UN \\ 
  Nigeria & ALL & 2001 & 179.61 & 174.04 & 185.50 & IHME \\ 
  Nigeria & ALL & 2001 & 181.19 & 153.23 & 212.67 & RW2 \\ 
  Nigeria & ALL & 2001 & 181.30 & 170.00 & 192.90 & UN \\ 
  Nigeria & ALL & 2002 & 176.05 & 170.57 & 182.25 & IHME \\ 
  Nigeria & ALL & 2002 & 175.37 & 149.25 & 205.07 & RW2 \\ 
  Nigeria & ALL & 2002 & 175.60 & 164.80 & 187.10 & UN \\ 
  Nigeria & ALL & 2003 & 171.91 & 166.25 & 178.06 & IHME \\ 
  Nigeria & ALL & 2003 & 169.57 & 143.69 & 199.25 & RW2 \\ 
  Nigeria & ALL & 2003 & 169.90 & 159.10 & 181.10 & UN \\ 
  Nigeria & ALL & 2004 & 166.90 & 161.18 & 173.15 & IHME \\ 
  Nigeria & ALL & 2004 & 163.58 & 136.40 & 194.77 & RW2 \\ 
  Nigeria & ALL & 2004 & 164.10 & 153.40 & 175.10 & UN \\ 
  Nigeria & ALL & 2005 & 161.50 & 155.75 & 167.84 & IHME \\ 
  Nigeria & ALL & 2005 & 157.91 & 131.30 & 188.88 & RW2 \\ 
  Nigeria & ALL & 2005 & 158.10 & 147.80 & 169.10 & UN \\ 
  Nigeria & ALL & 2006 & 155.99 & 150.04 & 161.88 & IHME \\ 
  Nigeria & ALL & 2006 & 152.14 & 127.85 & 179.98 & RW2 \\ 
  Nigeria & ALL & 2006 & 152.20 & 141.90 & 163.20 & UN \\ 
  Nigeria & ALL & 2007 & 149.98 & 144.00 & 156.00 & IHME \\ 
  Nigeria & ALL & 2007 & 146.55 & 124.05 & 172.24 & RW2 \\ 
  Nigeria & ALL & 2007 & 146.40 & 135.90 & 157.70 & UN \\ 
  Nigeria & ALL & 2008 & 144.30 & 137.62 & 150.82 & IHME \\ 
  Nigeria & ALL & 2008 & 141.18 & 118.59 & 167.37 & RW2 \\ 
  Nigeria & ALL & 2008 & 140.90 & 129.50 & 153.10 & UN \\ 
  Nigeria & ALL & 2009 & 138.52 & 131.57 & 145.69 & IHME \\ 
  Nigeria & ALL & 2009 & 135.74 & 111.96 & 164.02 & RW2 \\ 
  Nigeria & ALL & 2009 & 135.50 & 122.70 & 149.50 & UN \\ 
  Nigeria & ALL & 2010 & 132.58 & 125.30 & 140.40 & IHME \\ 
  Nigeria & ALL & 2010 & 130.52 & 105.91 & 160.55 & RW2 \\ 
  Nigeria & ALL & 2010 & 130.30 & 115.80 & 146.40 & UN \\ 
  Nigeria & ALL & 2011 & 126.12 & 117.90 & 135.07 & IHME \\ 
  Nigeria & ALL & 2011 & 125.54 & 102.87 & 152.36 & RW2 \\ 
  Nigeria & ALL & 2011 & 125.50 & 109.00 & 143.60 & UN \\ 
  Nigeria & ALL & 2012 & 120.64 & 111.35 & 130.46 & IHME \\ 
  Nigeria & ALL & 2012 & 120.64 & 100.42 & 144.15 & RW2 \\ 
  Nigeria & ALL & 2012 & 120.90 & 102.50 & 141.90 & UN \\ 
  Nigeria & ALL & 2013 & 115.12 & 105.37 & 126.09 & IHME \\ 
  Nigeria & ALL & 2013 & 115.98 & 95.02 & 140.62 & RW2 \\ 
  Nigeria & ALL & 2013 & 116.60 & 95.70 & 140.60 & UN \\ 
  Nigeria & ALL & 2014 & 110.23 & 99.44 & 122.15 & IHME \\ 
  Nigeria & ALL & 2014 & 111.38 & 83.75 & 146.49 & RW2 \\ 
  Nigeria & ALL & 2014 & 112.50 & 89.20 & 140.00 & UN \\ 
  Nigeria & ALL & 2015 & 103.84 & 92.56 & 116.28 & IHME \\ 
  Nigeria & ALL & 2015 & 106.84 & 68.28 & 163.44 & RW2 \\ 
  Nigeria & ALL & 2015 & 108.80 & 83.40 & 139.70 & UN \\ 
  Nigeria & ALL & 2016 & 102.78 & 55.61 & 183.21 & RW2 \\ 
  Nigeria & ALL & 2017 & 98.46 & 43.71 & 208.39 & RW2 \\ 
  Nigeria & ALL & 2018 & 94.46 & 33.98 & 241.16 & RW2 \\ 
  Nigeria & ALL & 2019 & 90.50 & 24.96 & 276.50 & RW2 \\ 
  Nigeria & ALL & 80-84 & 209.97 & 220.34 & 199.97 & HT-Direct \\ 
  Nigeria & ALL & 85-89 & 213.59 & 221.98 & 205.44 & HT-Direct \\ 
  Nigeria & ALL & 90-94 & 209.73 & 217.79 & 201.89 & HT-Direct \\ 
  Nigeria & ALL & 95-99 & 204.68 & 211.23 & 198.29 & HT-Direct \\ 
  Nigeria & ALL & 00-04 & 178.99 & 184.63 & 173.49 & HT-Direct \\ 
  Nigeria & ALL & 05-09 & 151.13 & 156.59 & 145.82 & HT-Direct \\ 
  Nigeria & ALL & 10-14 & 116.78 & 124.19 & 109.75 & HT-Direct \\ 
  Nigeria & ALL & 15-19 & 98.46 & 44.34 & 205.34 & RW2 \\ 
  Nigeria & NORTH CENTRAL & 1980 & 168.35 & 130.06 & 215.61 & RW2 \\ 
  Nigeria & NORTH CENTRAL & 1981 & 167.43 & 138.64 & 201.39 & RW2 \\ 
  Nigeria & NORTH CENTRAL & 1982 & 166.29 & 140.41 & 195.90 & RW2 \\ 
  Nigeria & NORTH CENTRAL & 1983 & 165.15 & 138.05 & 196.31 & RW2 \\ 
  Nigeria & NORTH CENTRAL & 1984 & 163.89 & 135.44 & 196.43 & RW2 \\ 
  Nigeria & NORTH CENTRAL & 1985 & 162.69 & 137.10 & 192.03 & RW2 \\ 
  Nigeria & NORTH CENTRAL & 1986 & 161.13 & 138.06 & 186.78 & RW2 \\ 
  Nigeria & NORTH CENTRAL & 1987 & 159.30 & 137.75 & 183.49 & RW2 \\ 
  Nigeria & NORTH CENTRAL & 1988 & 157.29 & 135.17 & 182.40 & RW2 \\ 
  Nigeria & NORTH CENTRAL & 1989 & 155.15 & 132.12 & 181.63 & RW2 \\ 
  Nigeria & NORTH CENTRAL & 1990 & 152.76 & 130.02 & 178.49 & RW2 \\ 
  Nigeria & NORTH CENTRAL & 1991 & 150.90 & 130.05 & 174.45 & RW2 \\ 
  Nigeria & NORTH CENTRAL & 1992 & 149.21 & 129.11 & 171.68 & RW2 \\ 
  Nigeria & NORTH CENTRAL & 1993 & 147.72 & 127.18 & 170.75 & RW2 \\ 
  Nigeria & NORTH CENTRAL & 1994 & 146.59 & 124.60 & 170.95 & RW2 \\ 
  Nigeria & NORTH CENTRAL & 1995 & 145.62 & 124.23 & 170.13 & RW2 \\ 
  Nigeria & NORTH CENTRAL & 1996 & 144.24 & 124.36 & 166.55 & RW2 \\ 
  Nigeria & NORTH CENTRAL & 1997 & 142.41 & 123.57 & 163.72 & RW2 \\ 
  Nigeria & NORTH CENTRAL & 1998 & 140.15 & 121.20 & 161.73 & RW2 \\ 
  Nigeria & NORTH CENTRAL & 1999 & 137.46 & 116.91 & 160.32 & RW2 \\ 
  Nigeria & NORTH CENTRAL & 2000 & 134.17 & 114.02 & 156.60 & RW2 \\ 
  Nigeria & NORTH CENTRAL & 2001 & 130.96 & 112.54 & 151.40 & RW2 \\ 
  Nigeria & NORTH CENTRAL & 2002 & 127.61 & 110.48 & 146.94 & RW2 \\ 
  Nigeria & NORTH CENTRAL & 2003 & 124.19 & 106.92 & 144.09 & RW2 \\ 
  Nigeria & NORTH CENTRAL & 2004 & 120.79 & 102.50 & 141.75 & RW2 \\ 
  Nigeria & NORTH CENTRAL & 2005 & 117.37 & 99.49 & 138.26 & RW2 \\ 
  Nigeria & NORTH CENTRAL & 2006 & 113.77 & 97.59 & 132.51 & RW2 \\ 
  Nigeria & NORTH CENTRAL & 2007 & 109.93 & 94.51 & 127.26 & RW2 \\ 
  Nigeria & NORTH CENTRAL & 2008 & 105.98 & 90.20 & 124.03 & RW2 \\ 
  Nigeria & NORTH CENTRAL & 2009 & 101.79 & 85.02 & 121.42 & RW2 \\ 
  Nigeria & NORTH CENTRAL & 2010 & 97.59 & 80.14 & 118.48 & RW2 \\ 
  Nigeria & NORTH CENTRAL & 2011 & 93.41 & 77.30 & 112.58 & RW2 \\ 
  Nigeria & NORTH CENTRAL & 2012 & 89.36 & 74.79 & 106.54 & RW2 \\ 
  Nigeria & NORTH CENTRAL & 2013 & 85.44 & 70.01 & 104.29 & RW2 \\ 
  Nigeria & NORTH CENTRAL & 2014 & 81.68 & 61.76 & 107.27 & RW2 \\ 
  Nigeria & NORTH CENTRAL & 2015 & 78.08 & 51.05 & 117.51 & RW2 \\ 
  Nigeria & NORTH CENTRAL & 2016 & 74.50 & 41.68 & 129.32 & RW2 \\ 
  Nigeria & NORTH CENTRAL & 2017 & 71.25 & 33.32 & 145.55 & RW2 \\ 
  Nigeria & NORTH CENTRAL & 2018 & 67.89 & 25.94 & 166.27 & RW2 \\ 
  Nigeria & NORTH CENTRAL & 2019 & 64.86 & 19.87 & 192.93 & RW2 \\ 
  Nigeria & NORTH EAST & 1980 & 249.17 & 198.63 & 308.20 & RW2 \\ 
  Nigeria & NORTH EAST & 1981 & 252.11 & 215.40 & 292.78 & RW2 \\ 
  Nigeria & NORTH EAST & 1982 & 255.09 & 221.59 & 290.94 & RW2 \\ 
  Nigeria & NORTH EAST & 1983 & 257.76 & 221.47 & 297.80 & RW2 \\ 
  Nigeria & NORTH EAST & 1984 & 260.45 & 221.62 & 302.96 & RW2 \\ 
  Nigeria & NORTH EAST & 1985 & 262.91 & 226.30 & 302.00 & RW2 \\ 
  Nigeria & NORTH EAST & 1986 & 264.55 & 231.60 & 300.51 & RW2 \\ 
  Nigeria & NORTH EAST & 1987 & 265.74 & 234.50 & 299.44 & RW2 \\ 
  Nigeria & NORTH EAST & 1988 & 266.19 & 233.63 & 301.67 & RW2 \\ 
  Nigeria & NORTH EAST & 1989 & 265.99 & 231.26 & 304.71 & RW2 \\ 
  Nigeria & NORTH EAST & 1990 & 265.09 & 230.33 & 302.98 & RW2 \\ 
  Nigeria & NORTH EAST & 1991 & 264.58 & 232.26 & 298.84 & RW2 \\ 
  Nigeria & NORTH EAST & 1992 & 263.75 & 232.97 & 296.79 & RW2 \\ 
  Nigeria & NORTH EAST & 1993 & 263.01 & 231.64 & 297.49 & RW2 \\ 
  Nigeria & NORTH EAST & 1994 & 262.17 & 228.15 & 299.14 & RW2 \\ 
  Nigeria & NORTH EAST & 1995 & 261.01 & 227.63 & 298.24 & RW2 \\ 
  Nigeria & NORTH EAST & 1996 & 258.56 & 227.92 & 291.53 & RW2 \\ 
  Nigeria & NORTH EAST & 1997 & 254.92 & 225.83 & 286.90 & RW2 \\ 
  Nigeria & NORTH EAST & 1998 & 250.05 & 220.20 & 282.78 & RW2 \\ 
  Nigeria & NORTH EAST & 1999 & 243.69 & 212.10 & 278.28 & RW2 \\ 
  Nigeria & NORTH EAST & 2000 & 236.02 & 205.19 & 270.03 & RW2 \\ 
  Nigeria & NORTH EAST & 2001 & 227.93 & 199.85 & 258.72 & RW2 \\ 
  Nigeria & NORTH EAST & 2002 & 219.24 & 193.26 & 248.03 & RW2 \\ 
  Nigeria & NORTH EAST & 2003 & 210.25 & 183.84 & 239.43 & RW2 \\ 
  Nigeria & NORTH EAST & 2004 & 201.18 & 173.31 & 231.57 & RW2 \\ 
  Nigeria & NORTH EAST & 2005 & 192.18 & 165.01 & 222.58 & RW2 \\ 
  Nigeria & NORTH EAST & 2006 & 183.07 & 159.03 & 210.29 & RW2 \\ 
  Nigeria & NORTH EAST & 2007 & 174.24 & 152.04 & 199.26 & RW2 \\ 
  Nigeria & NORTH EAST & 2008 & 165.57 & 142.79 & 191.20 & RW2 \\ 
  Nigeria & NORTH EAST & 2009 & 156.99 & 132.70 & 184.65 & RW2 \\ 
  Nigeria & NORTH EAST & 2010 & 148.75 & 123.12 & 178.35 & RW2 \\ 
  Nigeria & NORTH EAST & 2011 & 140.86 & 117.35 & 168.17 & RW2 \\ 
  Nigeria & NORTH EAST & 2012 & 133.28 & 111.64 & 158.30 & RW2 \\ 
  Nigeria & NORTH EAST & 2013 & 126.04 & 102.97 & 153.35 & RW2 \\ 
  Nigeria & NORTH EAST & 2014 & 119.02 & 89.58 & 155.95 & RW2 \\ 
  Nigeria & NORTH EAST & 2015 & 112.57 & 73.52 & 168.35 & RW2 \\ 
  Nigeria & NORTH EAST & 2016 & 106.20 & 59.20 & 182.35 & RW2 \\ 
  Nigeria & NORTH EAST & 2017 & 100.40 & 46.76 & 201.65 & RW2 \\ 
  Nigeria & NORTH EAST & 2018 & 94.50 & 36.14 & 224.72 & RW2 \\ 
  Nigeria & NORTH EAST & 2019 & 89.37 & 27.53 & 253.70 & RW2 \\ 
  Nigeria & NORTH WEST & 1980 & 278.41 & 223.97 & 339.63 & RW2 \\ 
  Nigeria & NORTH WEST & 1981 & 282.26 & 243.85 & 325.04 & RW2 \\ 
  Nigeria & NORTH WEST & 1982 & 286.10 & 250.78 & 324.13 & RW2 \\ 
  Nigeria & NORTH WEST & 1983 & 289.74 & 250.60 & 331.90 & RW2 \\ 
  Nigeria & NORTH WEST & 1984 & 293.16 & 251.42 & 338.15 & RW2 \\ 
  Nigeria & NORTH WEST & 1985 & 296.37 & 258.05 & 338.13 & RW2 \\ 
  Nigeria & NORTH WEST & 1986 & 298.70 & 263.57 & 336.14 & RW2 \\ 
  Nigeria & NORTH WEST & 1987 & 300.09 & 266.73 & 335.66 & RW2 \\ 
  Nigeria & NORTH WEST & 1988 & 300.47 & 265.71 & 338.28 & RW2 \\ 
  Nigeria & NORTH WEST & 1989 & 300.09 & 263.18 & 340.98 & RW2 \\ 
  Nigeria & NORTH WEST & 1990 & 298.17 & 261.36 & 338.31 & RW2 \\ 
  Nigeria & NORTH WEST & 1991 & 296.34 & 261.61 & 332.94 & RW2 \\ 
  Nigeria & NORTH WEST & 1992 & 293.80 & 260.78 & 329.00 & RW2 \\ 
  Nigeria & NORTH WEST & 1993 & 290.62 & 256.74 & 327.21 & RW2 \\ 
  Nigeria & NORTH WEST & 1994 & 286.98 & 250.94 & 326.03 & RW2 \\ 
  Nigeria & NORTH WEST & 1995 & 283.08 & 248.32 & 321.36 & RW2 \\ 
  Nigeria & NORTH WEST & 1996 & 277.52 & 245.18 & 312.30 & RW2 \\ 
  Nigeria & NORTH WEST & 1997 & 271.14 & 241.08 & 303.88 & RW2 \\ 
  Nigeria & NORTH WEST & 1998 & 263.65 & 232.94 & 297.42 & RW2 \\ 
  Nigeria & NORTH WEST & 1999 & 255.38 & 222.56 & 290.90 & RW2 \\ 
  Nigeria & NORTH WEST & 2000 & 246.16 & 213.86 & 281.06 & RW2 \\ 
  Nigeria & NORTH WEST & 2001 & 237.07 & 208.19 & 268.67 & RW2 \\ 
  Nigeria & NORTH WEST & 2002 & 228.10 & 201.13 & 257.22 & RW2 \\ 
  Nigeria & NORTH WEST & 2003 & 219.25 & 192.04 & 249.31 & RW2 \\ 
  Nigeria & NORTH WEST & 2004 & 210.83 & 182.42 & 242.78 & RW2 \\ 
  Nigeria & NORTH WEST & 2005 & 202.85 & 174.97 & 234.24 & RW2 \\ 
  Nigeria & NORTH WEST & 2006 & 194.76 & 169.73 & 222.60 & RW2 \\ 
  Nigeria & NORTH WEST & 2007 & 186.76 & 163.65 & 212.35 & RW2 \\ 
  Nigeria & NORTH WEST & 2008 & 178.85 & 155.04 & 205.11 & RW2 \\ 
  Nigeria & NORTH WEST & 2009 & 171.12 & 145.76 & 199.76 & RW2 \\ 
  Nigeria & NORTH WEST & 2010 & 163.37 & 136.72 & 193.79 & RW2 \\ 
  Nigeria & NORTH WEST & 2011 & 155.90 & 131.56 & 183.66 & RW2 \\ 
  Nigeria & NORTH WEST & 2012 & 148.73 & 126.94 & 173.39 & RW2 \\ 
  Nigeria & NORTH WEST & 2013 & 141.74 & 119.04 & 168.62 & RW2 \\ 
  Nigeria & NORTH WEST & 2014 & 134.97 & 104.62 & 172.90 & RW2 \\ 
  Nigeria & NORTH WEST & 2015 & 128.55 & 86.19 & 188.41 & RW2 \\ 
  Nigeria & NORTH WEST & 2016 & 122.29 & 69.87 & 205.05 & RW2 \\ 
  Nigeria & NORTH WEST & 2017 & 116.47 & 55.77 & 227.39 & RW2 \\ 
  Nigeria & NORTH WEST & 2018 & 110.50 & 43.28 & 254.08 & RW2 \\ 
  Nigeria & NORTH WEST & 2019 & 105.34 & 32.93 & 289.94 & RW2 \\ 
  Nigeria & SOUTH EAST & 1980 & 131.93 & 99.38 & 173.99 & RW2 \\ 
  Nigeria & SOUTH EAST & 1981 & 132.89 & 107.81 & 163.48 & RW2 \\ 
  Nigeria & SOUTH EAST & 1982 & 133.92 & 111.14 & 160.58 & RW2 \\ 
  Nigeria & SOUTH EAST & 1983 & 134.74 & 111.03 & 162.46 & RW2 \\ 
  Nigeria & SOUTH EAST & 1984 & 135.57 & 110.93 & 164.85 & RW2 \\ 
  Nigeria & SOUTH EAST & 1985 & 136.44 & 113.59 & 162.90 & RW2 \\ 
  Nigeria & SOUTH EAST & 1986 & 136.99 & 116.34 & 160.69 & RW2 \\ 
  Nigeria & SOUTH EAST & 1987 & 137.37 & 117.52 & 159.86 & RW2 \\ 
  Nigeria & SOUTH EAST & 1988 & 137.64 & 117.32 & 160.89 & RW2 \\ 
  Nigeria & SOUTH EAST & 1989 & 137.83 & 116.30 & 162.66 & RW2 \\ 
  Nigeria & SOUTH EAST & 1990 & 137.87 & 116.33 & 161.93 & RW2 \\ 
  Nigeria & SOUTH EAST & 1991 & 138.36 & 118.14 & 161.12 & RW2 \\ 
  Nigeria & SOUTH EAST & 1992 & 139.11 & 119.82 & 161.32 & RW2 \\ 
  Nigeria & SOUTH EAST & 1993 & 140.32 & 120.31 & 163.38 & RW2 \\ 
  Nigeria & SOUTH EAST & 1994 & 141.65 & 119.58 & 166.11 & RW2 \\ 
  Nigeria & SOUTH EAST & 1995 & 143.59 & 121.80 & 168.55 & RW2 \\ 
  Nigeria & SOUTH EAST & 1996 & 145.10 & 124.52 & 168.74 & RW2 \\ 
  Nigeria & SOUTH EAST & 1997 & 146.21 & 126.35 & 168.66 & RW2 \\ 
  Nigeria & SOUTH EAST & 1998 & 147.08 & 126.24 & 170.49 & RW2 \\ 
  Nigeria & SOUTH EAST & 1999 & 147.27 & 125.03 & 172.24 & RW2 \\ 
  Nigeria & SOUTH EAST & 2000 & 146.83 & 124.89 & 171.51 & RW2 \\ 
  Nigeria & SOUTH EAST & 2001 & 146.15 & 125.72 & 169.38 & RW2 \\ 
  Nigeria & SOUTH EAST & 2002 & 145.24 & 125.72 & 167.33 & RW2 \\ 
  Nigeria & SOUTH EAST & 2003 & 143.78 & 123.44 & 166.55 & RW2 \\ 
  Nigeria & SOUTH EAST & 2004 & 142.17 & 120.56 & 166.83 & RW2 \\ 
  Nigeria & SOUTH EAST & 2005 & 140.27 & 118.81 & 165.20 & RW2 \\ 
  Nigeria & SOUTH EAST & 2006 & 137.81 & 117.81 & 160.69 & RW2 \\ 
  Nigeria & SOUTH EAST & 2007 & 135.01 & 115.84 & 156.43 & RW2 \\ 
  Nigeria & SOUTH EAST & 2008 & 132.03 & 112.03 & 154.60 & RW2 \\ 
  Nigeria & SOUTH EAST & 2009 & 128.64 & 106.70 & 154.21 & RW2 \\ 
  Nigeria & SOUTH EAST & 2010 & 125.06 & 101.78 & 152.92 & RW2 \\ 
  Nigeria & SOUTH EAST & 2011 & 121.31 & 98.46 & 148.45 & RW2 \\ 
  Nigeria & SOUTH EAST & 2012 & 117.70 & 95.40 & 144.08 & RW2 \\ 
  Nigeria & SOUTH EAST & 2013 & 114.16 & 89.73 & 143.83 & RW2 \\ 
  Nigeria & SOUTH EAST & 2014 & 110.71 & 80.42 & 149.81 & RW2 \\ 
  Nigeria & SOUTH EAST & 2015 & 107.26 & 68.06 & 165.15 & RW2 \\ 
  Nigeria & SOUTH EAST & 2016 & 103.98 & 56.30 & 182.32 & RW2 \\ 
  Nigeria & SOUTH EAST & 2017 & 100.67 & 45.70 & 205.80 & RW2 \\ 
  Nigeria & SOUTH EAST & 2018 & 97.52 & 36.52 & 235.95 & RW2 \\ 
  Nigeria & SOUTH EAST & 2019 & 94.57 & 27.97 & 270.16 & RW2 \\ 
  Nigeria & SOUTH SOUTH & 1980 & 143.82 & 109.65 & 186.99 & RW2 \\ 
  Nigeria & SOUTH SOUTH & 1981 & 142.78 & 117.26 & 173.40 & RW2 \\ 
  Nigeria & SOUTH SOUTH & 1982 & 141.64 & 118.73 & 168.54 & RW2 \\ 
  Nigeria & SOUTH SOUTH & 1983 & 140.80 & 116.55 & 168.73 & RW2 \\ 
  Nigeria & SOUTH SOUTH & 1984 & 139.62 & 114.34 & 168.69 & RW2 \\ 
  Nigeria & SOUTH SOUTH & 1985 & 138.71 & 115.51 & 165.15 & RW2 \\ 
  Nigeria & SOUTH SOUTH & 1986 & 137.50 & 116.48 & 161.03 & RW2 \\ 
  Nigeria & SOUTH SOUTH & 1987 & 136.29 & 116.42 & 158.72 & RW2 \\ 
  Nigeria & SOUTH SOUTH & 1988 & 135.04 & 114.98 & 157.66 & RW2 \\ 
  Nigeria & SOUTH SOUTH & 1989 & 133.93 & 113.03 & 158.44 & RW2 \\ 
  Nigeria & SOUTH SOUTH & 1990 & 132.78 & 112.10 & 156.48 & RW2 \\ 
  Nigeria & SOUTH SOUTH & 1991 & 132.07 & 112.93 & 153.94 & RW2 \\ 
  Nigeria & SOUTH SOUTH & 1992 & 131.74 & 113.35 & 152.43 & RW2 \\ 
  Nigeria & SOUTH SOUTH & 1993 & 131.83 & 112.85 & 153.26 & RW2 \\ 
  Nigeria & SOUTH SOUTH & 1994 & 132.08 & 111.71 & 155.22 & RW2 \\ 
  Nigeria & SOUTH SOUTH & 1995 & 132.82 & 112.67 & 156.03 & RW2 \\ 
  Nigeria & SOUTH SOUTH & 1996 & 133.01 & 114.11 & 154.66 & RW2 \\ 
  Nigeria & SOUTH SOUTH & 1997 & 132.84 & 114.54 & 153.58 & RW2 \\ 
  Nigeria & SOUTH SOUTH & 1998 & 132.18 & 113.58 & 153.49 & RW2 \\ 
  Nigeria & SOUTH SOUTH & 1999 & 130.94 & 110.98 & 153.82 & RW2 \\ 
  Nigeria & SOUTH SOUTH & 2000 & 129.18 & 109.62 & 151.51 & RW2 \\ 
  Nigeria & SOUTH SOUTH & 2001 & 127.09 & 109.05 & 147.79 & RW2 \\ 
  Nigeria & SOUTH SOUTH & 2002 & 124.66 & 107.66 & 144.34 & RW2 \\ 
  Nigeria & SOUTH SOUTH & 2003 & 121.87 & 104.40 & 141.91 & RW2 \\ 
  Nigeria & SOUTH SOUTH & 2004 & 118.79 & 100.65 & 140.00 & RW2 \\ 
  Nigeria & SOUTH SOUTH & 2005 & 115.57 & 97.57 & 136.37 & RW2 \\ 
  Nigeria & SOUTH SOUTH & 2006 & 111.84 & 95.43 & 130.85 & RW2 \\ 
  Nigeria & SOUTH SOUTH & 2007 & 108.01 & 92.60 & 125.42 & RW2 \\ 
  Nigeria & SOUTH SOUTH & 2008 & 103.83 & 88.16 & 122.24 & RW2 \\ 
  Nigeria & SOUTH SOUTH & 2009 & 99.43 & 82.59 & 119.38 & RW2 \\ 
  Nigeria & SOUTH SOUTH & 2010 & 94.92 & 77.20 & 116.10 & RW2 \\ 
  Nigeria & SOUTH SOUTH & 2011 & 90.63 & 74.05 & 110.10 & RW2 \\ 
  Nigeria & SOUTH SOUTH & 2012 & 86.36 & 70.81 & 104.48 & RW2 \\ 
  Nigeria & SOUTH SOUTH & 2013 & 82.11 & 65.57 & 101.72 & RW2 \\ 
  Nigeria & SOUTH SOUTH & 2014 & 78.21 & 57.72 & 104.81 & RW2 \\ 
  Nigeria & SOUTH SOUTH & 2015 & 74.38 & 47.59 & 113.71 & RW2 \\ 
  Nigeria & SOUTH SOUTH & 2016 & 70.77 & 38.59 & 126.02 & RW2 \\ 
  Nigeria & SOUTH SOUTH & 2017 & 67.45 & 30.67 & 140.94 & RW2 \\ 
  Nigeria & SOUTH SOUTH & 2018 & 64.16 & 23.83 & 159.90 & RW2 \\ 
  Nigeria & SOUTH SOUTH & 2019 & 60.75 & 17.99 & 185.80 & RW2 \\ 
  Nigeria & SOUTH WEST & 1980 & 160.25 & 119.91 & 211.96 & RW2 \\ 
  Nigeria & SOUTH WEST & 1981 & 156.09 & 125.76 & 192.87 & RW2 \\ 
  Nigeria & SOUTH WEST & 1982 & 151.89 & 125.20 & 183.38 & RW2 \\ 
  Nigeria & SOUTH WEST & 1983 & 147.67 & 121.05 & 179.09 & RW2 \\ 
  Nigeria & SOUTH WEST & 1984 & 143.52 & 117.22 & 174.39 & RW2 \\ 
  Nigeria & SOUTH WEST & 1985 & 139.47 & 115.67 & 166.95 & RW2 \\ 
  Nigeria & SOUTH WEST & 1986 & 135.35 & 114.27 & 159.25 & RW2 \\ 
  Nigeria & SOUTH WEST & 1987 & 131.19 & 112.01 & 152.94 & RW2 \\ 
  Nigeria & SOUTH WEST & 1988 & 127.07 & 107.76 & 149.15 & RW2 \\ 
  Nigeria & SOUTH WEST & 1989 & 123.12 & 103.36 & 146.10 & RW2 \\ 
  Nigeria & SOUTH WEST & 1990 & 119.24 & 100.16 & 141.29 & RW2 \\ 
  Nigeria & SOUTH WEST & 1991 & 116.14 & 98.65 & 136.32 & RW2 \\ 
  Nigeria & SOUTH WEST & 1992 & 113.58 & 96.86 & 132.57 & RW2 \\ 
  Nigeria & SOUTH WEST & 1993 & 111.51 & 94.76 & 130.55 & RW2 \\ 
  Nigeria & SOUTH WEST & 1994 & 109.90 & 92.25 & 130.11 & RW2 \\ 
  Nigeria & SOUTH WEST & 1995 & 108.84 & 91.66 & 128.76 & RW2 \\ 
  Nigeria & SOUTH WEST & 1996 & 107.64 & 91.59 & 125.93 & RW2 \\ 
  Nigeria & SOUTH WEST & 1997 & 106.35 & 91.14 & 123.48 & RW2 \\ 
  Nigeria & SOUTH WEST & 1998 & 104.85 & 89.52 & 122.69 & RW2 \\ 
  Nigeria & SOUTH WEST & 1999 & 103.15 & 86.97 & 122.14 & RW2 \\ 
  Nigeria & SOUTH WEST & 2000 & 101.14 & 84.97 & 119.31 & RW2 \\ 
  Nigeria & SOUTH WEST & 2001 & 99.23 & 84.67 & 115.84 & RW2 \\ 
  Nigeria & SOUTH WEST & 2002 & 97.33 & 83.54 & 113.13 & RW2 \\ 
  Nigeria & SOUTH WEST & 2003 & 95.35 & 81.37 & 111.61 & RW2 \\ 
  Nigeria & SOUTH WEST & 2004 & 93.44 & 78.68 & 110.60 & RW2 \\ 
  Nigeria & SOUTH WEST & 2005 & 91.57 & 76.70 & 109.15 & RW2 \\ 
  Nigeria & SOUTH WEST & 2006 & 89.46 & 75.66 & 105.63 & RW2 \\ 
  Nigeria & SOUTH WEST & 2007 & 87.16 & 74.00 & 102.39 & RW2 \\ 
  Nigeria & SOUTH WEST & 2008 & 84.75 & 71.15 & 100.69 & RW2 \\ 
  Nigeria & SOUTH WEST & 2009 & 82.10 & 67.50 & 99.71 & RW2 \\ 
  Nigeria & SOUTH WEST & 2010 & 79.43 & 63.53 & 98.72 & RW2 \\ 
  Nigeria & SOUTH WEST & 2011 & 76.62 & 61.40 & 95.49 & RW2 \\ 
  Nigeria & SOUTH WEST & 2012 & 73.97 & 59.05 & 92.51 & RW2 \\ 
  Nigeria & SOUTH WEST & 2013 & 71.28 & 55.09 & 92.17 & RW2 \\ 
  Nigeria & SOUTH WEST & 2014 & 68.74 & 49.02 & 95.51 & RW2 \\ 
  Nigeria & SOUTH WEST & 2015 & 66.31 & 41.15 & 105.83 & RW2 \\ 
  Nigeria & SOUTH WEST & 2016 & 63.91 & 34.06 & 117.70 & RW2 \\ 
  Nigeria & SOUTH WEST & 2017 & 61.65 & 27.34 & 134.35 & RW2 \\ 
  Nigeria & SOUTH WEST & 2018 & 59.26 & 21.24 & 154.54 & RW2 \\ 
  Nigeria & SOUTH WEST & 2019 & 57.05 & 16.59 & 179.41 & RW2 \\ 
  Rwanda & ALL & 1980 & 203.05 & 192.59 & 215.35 & IHME \\ 
  Rwanda & ALL & 1980 & 214.17 & 148.90 & 294.16 & RW2 \\ 
  Rwanda & ALL & 1980 & 218.20 & 203.60 & 233.60 & UN \\ 
  Rwanda & ALL & 1981 & 191.19 & 181.51 & 201.29 & IHME \\ 
  Rwanda & ALL & 1981 & 200.87 & 155.70 & 251.70 & RW2 \\ 
  Rwanda & ALL & 1981 & 201.60 & 188.50 & 215.60 & UN \\ 
  Rwanda & ALL & 1982 & 181.05 & 171.17 & 190.84 & IHME \\ 
  Rwanda & ALL & 1982 & 188.22 & 150.09 & 233.71 & RW2 \\ 
  Rwanda & ALL & 1982 & 186.50 & 174.30 & 199.20 & UN \\ 
  Rwanda & ALL & 1983 & 173.80 & 164.82 & 182.29 & IHME \\ 
  Rwanda & ALL & 1983 & 176.42 & 137.25 & 226.83 & RW2 \\ 
  Rwanda & ALL & 1983 & 174.40 & 162.70 & 186.40 & UN \\ 
  Rwanda & ALL & 1984 & 169.52 & 160.86 & 177.98 & IHME \\ 
  Rwanda & ALL & 1984 & 166.62 & 126.23 & 221.27 & RW2 \\ 
  Rwanda & ALL & 1984 & 166.00 & 155.30 & 177.20 & UN \\ 
  Rwanda & ALL & 1985 & 166.43 & 157.44 & 175.19 & IHME \\ 
  Rwanda & ALL & 1985 & 157.23 & 120.00 & 200.87 & RW2 \\ 
  Rwanda & ALL & 1985 & 160.10 & 150.10 & 171.10 & UN \\ 
  Rwanda & ALL & 1986 & 161.66 & 153.18 & 170.85 & IHME \\ 
  Rwanda & ALL & 1986 & 151.88 & 119.02 & 190.69 & RW2 \\ 
  Rwanda & ALL & 1986 & 155.30 & 145.80 & 165.70 & UN \\ 
  Rwanda & ALL & 1987 & 155.03 & 147.10 & 163.48 & IHME \\ 
  Rwanda & ALL & 1987 & 149.46 & 119.27 & 187.01 & RW2 \\ 
  Rwanda & ALL & 1987 & 151.60 & 142.50 & 161.30 & UN \\ 
  Rwanda & ALL & 1988 & 150.27 & 142.06 & 158.65 & IHME \\ 
  Rwanda & ALL & 1988 & 150.00 & 118.24 & 189.74 & RW2 \\ 
  Rwanda & ALL & 1988 & 149.40 & 140.30 & 159.00 & UN \\ 
  Rwanda & ALL & 1989 & 150.42 & 143.01 & 158.40 & IHME \\ 
  Rwanda & ALL & 1989 & 153.71 & 119.31 & 197.02 & RW2 \\ 
  Rwanda & ALL & 1989 & 149.30 & 140.40 & 158.80 & UN \\ 
  Rwanda & ALL & 1990 & 154.43 & 146.27 & 162.85 & IHME \\ 
  Rwanda & ALL & 1990 & 160.77 & 126.65 & 207.62 & RW2 \\ 
  Rwanda & ALL & 1990 & 151.80 & 142.50 & 161.30 & UN \\ 
  Rwanda & ALL & 1991 & 159.63 & 151.50 & 168.09 & IHME \\ 
  Rwanda & ALL & 1991 & 170.39 & 136.58 & 213.95 & RW2 \\ 
  Rwanda & ALL & 1991 & 157.30 & 147.60 & 167.30 & UN \\ 
  Rwanda & ALL & 1992 & 165.94 & 158.01 & 174.80 & IHME \\ 
  Rwanda & ALL & 1992 & 182.46 & 146.53 & 225.03 & RW2 \\ 
  Rwanda & ALL & 1992 & 165.90 & 154.80 & 177.80 & UN \\ 
  Rwanda & ALL & 1993 & 171.67 & 162.61 & 181.82 & IHME \\ 
  Rwanda & ALL & 1993 & 196.12 & 155.18 & 240.57 & RW2 \\ 
  Rwanda & ALL & 1993 & 185.40 & 170.40 & 202.20 & UN \\ 
  Rwanda & ALL & 1994 & 355.64 & 178.00 & 566.20 & IHME \\ 
  Rwanda & ALL & 1994 & 209.67 & 161.89 & 259.35 & RW2 \\ 
  Rwanda & ALL & 1994 & 299.60 & 270.60 & 332.90 & UN \\ 
  Rwanda & ALL & 1995 & 182.98 & 170.24 & 192.73 & IHME \\ 
  Rwanda & ALL & 1995 & 223.66 & 178.20 & 279.42 & RW2 \\ 
  Rwanda & ALL & 1995 & 268.30 & 238.60 & 303.00 & UN \\ 
  Rwanda & ALL & 1996 & 179.69 & 170.04 & 189.79 & IHME \\ 
  Rwanda & ALL & 1996 & 230.58 & 185.57 & 283.04 & RW2 \\ 
  Rwanda & ALL & 1996 & 203.30 & 179.90 & 231.30 & UN \\ 
  Rwanda & ALL & 1997 & 180.07 & 170.08 & 190.86 & IHME \\ 
  Rwanda & ALL & 1997 & 231.02 & 186.94 & 279.97 & RW2 \\ 
  Rwanda & ALL & 1997 & 223.00 & 198.00 & 252.60 & UN \\ 
  Rwanda & ALL & 1998 & 175.48 & 166.40 & 185.55 & IHME \\ 
  Rwanda & ALL & 1998 & 224.35 & 180.49 & 276.49 & RW2 \\ 
  Rwanda & ALL & 1998 & 234.00 & 209.90 & 262.10 & UN \\ 
  Rwanda & ALL & 1999 & 168.19 & 159.11 & 177.47 & IHME \\ 
  Rwanda & ALL & 1999 & 211.30 & 167.70 & 265.01 & RW2 \\ 
  Rwanda & ALL & 1999 & 200.60 & 183.10 & 220.50 & UN \\ 
  Rwanda & ALL & 2000 & 158.50 & 149.80 & 167.12 & IHME \\ 
  Rwanda & ALL & 2000 & 191.74 & 148.14 & 237.47 & RW2 \\ 
  Rwanda & ALL & 2000 & 183.80 & 170.30 & 198.70 & UN \\ 
  Rwanda & ALL & 2001 & 149.20 & 140.30 & 158.28 & IHME \\ 
  Rwanda & ALL & 2001 & 172.97 & 136.05 & 214.30 & RW2 \\ 
  Rwanda & ALL & 2001 & 170.20 & 158.80 & 182.20 & UN \\ 
  Rwanda & ALL & 2002 & 138.34 & 130.35 & 147.27 & IHME \\ 
  Rwanda & ALL & 2002 & 154.44 & 123.41 & 192.54 & RW2 \\ 
  Rwanda & ALL & 2002 & 154.70 & 144.70 & 165.50 & UN \\ 
  Rwanda & ALL & 2003 & 127.90 & 119.92 & 136.83 & IHME \\ 
  Rwanda & ALL & 2003 & 137.53 & 109.46 & 174.11 & RW2 \\ 
  Rwanda & ALL & 2003 & 139.40 & 130.10 & 149.10 & UN \\ 
  Rwanda & ALL & 2004 & 115.77 & 108.21 & 123.20 & IHME \\ 
  Rwanda & ALL & 2004 & 122.38 & 94.86 & 158.66 & RW2 \\ 
  Rwanda & ALL & 2004 & 124.40 & 115.70 & 133.50 & UN \\ 
  Rwanda & ALL & 2005 & 103.56 & 96.91 & 110.88 & IHME \\ 
  Rwanda & ALL & 2005 & 109.66 & 84.19 & 143.14 & RW2 \\ 
  Rwanda & ALL & 2005 & 110.70 & 102.40 & 119.40 & UN \\ 
  Rwanda & ALL & 2006 & 92.65 & 86.38 & 99.80 & IHME \\ 
  Rwanda & ALL & 2006 & 98.20 & 76.39 & 126.10 & RW2 \\ 
  Rwanda & ALL & 2006 & 98.60 & 90.90 & 106.90 & UN \\ 
  Rwanda & ALL & 2007 & 84.48 & 77.81 & 91.66 & IHME \\ 
  Rwanda & ALL & 2007 & 88.25 & 69.16 & 111.97 & RW2 \\ 
  Rwanda & ALL & 2007 & 87.80 & 80.70 & 95.60 & UN \\ 
  Rwanda & ALL & 2008 & 78.81 & 71.88 & 85.98 & IHME \\ 
  Rwanda & ALL & 2008 & 79.60 & 61.44 & 102.16 & RW2 \\ 
  Rwanda & ALL & 2008 & 78.30 & 71.60 & 85.80 & UN \\ 
  Rwanda & ALL & 2009 & 75.80 & 68.35 & 83.63 & IHME \\ 
  Rwanda & ALL & 2009 & 71.79 & 53.72 & 94.94 & RW2 \\ 
  Rwanda & ALL & 2009 & 70.60 & 64.00 & 78.00 & UN \\ 
  Rwanda & ALL & 2010 & 73.65 & 65.67 & 82.21 & IHME \\ 
  Rwanda & ALL & 2010 & 65.02 & 47.75 & 89.10 & RW2 \\ 
  Rwanda & ALL & 2010 & 64.10 & 57.10 & 71.90 & UN \\ 
  Rwanda & ALL & 2011 & 72.72 & 63.54 & 82.46 & IHME \\ 
  Rwanda & ALL & 2011 & 58.89 & 43.68 & 79.21 & RW2 \\ 
  Rwanda & ALL & 2011 & 57.70 & 50.20 & 66.30 & UN \\ 
  Rwanda & ALL & 2012 & 71.79 & 61.32 & 83.51 & IHME \\ 
  Rwanda & ALL & 2012 & 53.33 & 40.26 & 70.20 & RW2 \\ 
  Rwanda & ALL & 2012 & 52.10 & 43.60 & 62.30 & UN \\ 
  Rwanda & ALL & 2013 & 70.39 & 58.61 & 84.25 & IHME \\ 
  Rwanda & ALL & 2013 & 48.32 & 35.38 & 65.37 & RW2 \\ 
  Rwanda & ALL & 2013 & 47.80 & 38.00 & 60.30 & UN \\ 
  Rwanda & ALL & 2014 & 68.42 & 56.09 & 83.64 & IHME \\ 
  Rwanda & ALL & 2014 & 43.71 & 27.99 & 67.32 & RW2 \\ 
  Rwanda & ALL & 2014 & 44.30 & 33.50 & 58.90 & UN \\ 
  Rwanda & ALL & 2015 & 65.64 & 53.52 & 81.81 & IHME \\ 
  Rwanda & ALL & 2015 & 39.44 & 19.83 & 76.54 & RW2 \\ 
  Rwanda & ALL & 2015 & 41.70 & 29.90 & 58.40 & UN \\ 
  Rwanda & ALL & 2016 & 35.75 & 14.04 & 88.37 & RW2 \\ 
  Rwanda & ALL & 2017 & 32.20 & 9.46 & 104.65 & RW2 \\ 
  Rwanda & ALL & 2018 & 29.06 & 6.29 & 127.91 & RW2 \\ 
  Rwanda & ALL & 2019 & 26.18 & 3.86 & 154.73 & RW2 \\ 
  Rwanda & ALL & 80-84 & 184.82 & 197.83 & 172.49 & HT-Direct \\ 
  Rwanda & ALL & 85-89 & 153.79 & 162.00 & 145.92 & HT-Direct \\ 
  Rwanda & ALL & 90-94 & 212.65 & 220.05 & 205.44 & HT-Direct \\ 
  Rwanda & ALL & 95-99 & 217.09 & 224.24 & 210.11 & HT-Direct \\ 
  Rwanda & ALL & 00-04 & 152.26 & 158.15 & 146.55 & HT-Direct \\ 
  Rwanda & ALL & 05-09 & 83.45 & 87.87 & 79.23 & HT-Direct \\ 
  Rwanda & ALL & 10-14 & 49.34 & 55.42 & 43.89 & HT-Direct \\ 
  Rwanda & ALL & 15-19 & 32.19 & 9.67 & 102.31 & RW2 \\ 
  Rwanda & EAST & 1980 & 203.28 & 143.80 & 279.75 & RW2 \\ 
  Rwanda & EAST & 1981 & 193.05 & 151.76 & 242.68 & RW2 \\ 
  Rwanda & EAST & 1982 & 183.34 & 148.78 & 224.25 & RW2 \\ 
  Rwanda & EAST & 1983 & 174.27 & 139.38 & 217.21 & RW2 \\ 
  Rwanda & EAST & 1984 & 166.70 & 131.78 & 211.89 & RW2 \\ 
  Rwanda & EAST & 1985 & 160.20 & 128.21 & 197.04 & RW2 \\ 
  Rwanda & EAST & 1986 & 157.16 & 129.94 & 188.61 & RW2 \\ 
  Rwanda & EAST & 1987 & 157.10 & 131.65 & 186.65 & RW2 \\ 
  Rwanda & EAST & 1988 & 160.38 & 133.32 & 192.08 & RW2 \\ 
  Rwanda & EAST & 1989 & 166.96 & 136.54 & 202.43 & RW2 \\ 
  Rwanda & EAST & 1990 & 177.20 & 146.57 & 214.35 & RW2 \\ 
  Rwanda & EAST & 1991 & 190.00 & 160.52 & 225.18 & RW2 \\ 
  Rwanda & EAST & 1992 & 205.12 & 175.38 & 239.45 & RW2 \\ 
  Rwanda & EAST & 1993 & 221.79 & 187.52 & 258.96 & RW2 \\ 
  Rwanda & EAST & 1994 & 238.06 & 196.19 & 280.02 & RW2 \\ 
  Rwanda & EAST & 1995 & 254.16 & 213.21 & 300.56 & RW2 \\ 
  Rwanda & EAST & 1996 & 264.20 & 225.23 & 307.20 & RW2 \\ 
  Rwanda & EAST & 1997 & 267.76 & 230.56 & 307.68 & RW2 \\ 
  Rwanda & EAST & 1998 & 264.53 & 225.89 & 307.19 & RW2 \\ 
  Rwanda & EAST & 1999 & 254.32 & 213.71 & 300.86 & RW2 \\ 
  Rwanda & EAST & 2000 & 237.28 & 196.75 & 279.44 & RW2 \\ 
  Rwanda & EAST & 2001 & 219.01 & 184.63 & 256.72 & RW2 \\ 
  Rwanda & EAST & 2002 & 199.79 & 170.13 & 233.93 & RW2 \\ 
  Rwanda & EAST & 2003 & 180.47 & 151.45 & 214.42 & RW2 \\ 
  Rwanda & EAST & 2004 & 162.42 & 133.21 & 197.67 & RW2 \\ 
  Rwanda & EAST & 2005 & 145.97 & 118.97 & 179.09 & RW2 \\ 
  Rwanda & EAST & 2006 & 130.65 & 107.85 & 158.01 & RW2 \\ 
  Rwanda & EAST & 2007 & 116.90 & 97.03 & 139.74 & RW2 \\ 
  Rwanda & EAST & 2008 & 104.74 & 85.31 & 127.18 & RW2 \\ 
  Rwanda & EAST & 2009 & 93.75 & 73.59 & 118.09 & RW2 \\ 
  Rwanda & EAST & 2010 & 84.09 & 64.52 & 109.44 & RW2 \\ 
  Rwanda & EAST & 2011 & 75.22 & 57.62 & 97.61 & RW2 \\ 
  Rwanda & EAST & 2012 & 67.38 & 51.71 & 86.90 & RW2 \\ 
  Rwanda & EAST & 2013 & 60.32 & 44.09 & 81.30 & RW2 \\ 
  Rwanda & EAST & 2014 & 53.95 & 34.74 & 81.86 & RW2 \\ 
  Rwanda & EAST & 2015 & 48.12 & 25.22 & 89.37 & RW2 \\ 
  Rwanda & EAST & 2016 & 42.96 & 17.70 & 98.42 & RW2 \\ 
  Rwanda & EAST & 2017 & 38.27 & 12.13 & 112.30 & RW2 \\ 
  Rwanda & EAST & 2018 & 34.11 & 8.14 & 131.78 & RW2 \\ 
  Rwanda & EAST & 2019 & 30.43 & 5.15 & 155.01 & RW2 \\ 
  Rwanda & KIGALI & 1980 & 140.15 & 89.32 & 208.07 & RW2 \\ 
  Rwanda & KIGALI & 1981 & 133.34 & 94.85 & 180.26 & RW2 \\ 
  Rwanda & KIGALI & 1982 & 126.70 & 94.43 & 166.36 & RW2 \\ 
  Rwanda & KIGALI & 1983 & 120.39 & 90.60 & 159.05 & RW2 \\ 
  Rwanda & KIGALI & 1984 & 115.47 & 87.06 & 152.77 & RW2 \\ 
  Rwanda & KIGALI & 1985 & 111.00 & 85.33 & 142.55 & RW2 \\ 
  Rwanda & KIGALI & 1986 & 109.04 & 86.54 & 136.41 & RW2 \\ 
  Rwanda & KIGALI & 1987 & 109.16 & 88.07 & 135.05 & RW2 \\ 
  Rwanda & KIGALI & 1988 & 111.31 & 89.36 & 138.21 & RW2 \\ 
  Rwanda & KIGALI & 1989 & 115.65 & 91.84 & 145.17 & RW2 \\ 
  Rwanda & KIGALI & 1990 & 122.06 & 98.00 & 152.87 & RW2 \\ 
  Rwanda & KIGALI & 1991 & 129.65 & 106.72 & 158.66 & RW2 \\ 
  Rwanda & KIGALI & 1992 & 138.23 & 114.93 & 166.09 & RW2 \\ 
  Rwanda & KIGALI & 1993 & 146.80 & 120.26 & 176.89 & RW2 \\ 
  Rwanda & KIGALI & 1994 & 154.49 & 124.05 & 187.95 & RW2 \\ 
  Rwanda & KIGALI & 1995 & 160.96 & 131.17 & 196.80 & RW2 \\ 
  Rwanda & KIGALI & 1996 & 163.04 & 134.04 & 195.68 & RW2 \\ 
  Rwanda & KIGALI & 1997 & 160.78 & 132.91 & 191.76 & RW2 \\ 
  Rwanda & KIGALI & 1998 & 154.22 & 126.13 & 186.49 & RW2 \\ 
  Rwanda & KIGALI & 1999 & 143.83 & 115.36 & 178.22 & RW2 \\ 
  Rwanda & KIGALI & 2000 & 130.39 & 103.24 & 160.17 & RW2 \\ 
  Rwanda & KIGALI & 2001 & 117.36 & 94.73 & 143.09 & RW2 \\ 
  Rwanda & KIGALI & 2002 & 104.81 & 85.63 & 127.81 & RW2 \\ 
  Rwanda & KIGALI & 2003 & 93.17 & 75.02 & 115.35 & RW2 \\ 
  Rwanda & KIGALI & 2004 & 82.90 & 65.28 & 105.40 & RW2 \\ 
  Rwanda & KIGALI & 2005 & 73.96 & 57.84 & 94.71 & RW2 \\ 
  Rwanda & KIGALI & 2006 & 66.04 & 52.08 & 83.61 & RW2 \\ 
  Rwanda & KIGALI & 2007 & 59.10 & 46.83 & 74.35 & RW2 \\ 
  Rwanda & KIGALI & 2008 & 52.95 & 41.15 & 68.06 & RW2 \\ 
  Rwanda & KIGALI & 2009 & 47.58 & 35.54 & 63.23 & RW2 \\ 
  Rwanda & KIGALI & 2010 & 42.84 & 30.86 & 59.30 & RW2 \\ 
  Rwanda & KIGALI & 2011 & 38.59 & 27.66 & 53.98 & RW2 \\ 
  Rwanda & KIGALI & 2012 & 34.76 & 24.57 & 49.19 & RW2 \\ 
  Rwanda & KIGALI & 2013 & 31.30 & 20.95 & 46.83 & RW2 \\ 
  Rwanda & KIGALI & 2014 & 28.19 & 16.72 & 47.43 & RW2 \\ 
  Rwanda & KIGALI & 2015 & 25.40 & 12.26 & 51.85 & RW2 \\ 
  Rwanda & KIGALI & 2016 & 22.85 & 8.80 & 57.93 & RW2 \\ 
  Rwanda & KIGALI & 2017 & 20.53 & 6.14 & 66.44 & RW2 \\ 
  Rwanda & KIGALI & 2018 & 18.49 & 4.16 & 78.85 & RW2 \\ 
  Rwanda & KIGALI & 2019 & 16.64 & 2.70 & 95.21 & RW2 \\ 
  Rwanda & NORTH & 1980 & 237.30 & 168.61 & 324.08 & RW2 \\ 
  Rwanda & NORTH & 1981 & 218.75 & 172.33 & 276.22 & RW2 \\ 
  Rwanda & NORTH & 1982 & 201.42 & 163.02 & 247.73 & RW2 \\ 
  Rwanda & NORTH & 1983 & 185.63 & 147.93 & 232.26 & RW2 \\ 
  Rwanda & NORTH & 1984 & 172.03 & 135.79 & 218.05 & RW2 \\ 
  Rwanda & NORTH & 1985 & 160.24 & 128.16 & 197.76 & RW2 \\ 
  Rwanda & NORTH & 1986 & 152.69 & 125.38 & 184.01 & RW2 \\ 
  Rwanda & NORTH & 1987 & 148.60 & 123.85 & 177.20 & RW2 \\ 
  Rwanda & NORTH & 1988 & 148.06 & 122.19 & 178.57 & RW2 \\ 
  Rwanda & NORTH & 1989 & 151.15 & 122.99 & 184.73 & RW2 \\ 
  Rwanda & NORTH & 1990 & 157.64 & 129.82 & 192.79 & RW2 \\ 
  Rwanda & NORTH & 1991 & 167.00 & 140.00 & 198.93 & RW2 \\ 
  Rwanda & NORTH & 1992 & 178.57 & 151.30 & 209.44 & RW2 \\ 
  Rwanda & NORTH & 1993 & 191.34 & 160.25 & 225.13 & RW2 \\ 
  Rwanda & NORTH & 1994 & 203.96 & 167.41 & 242.84 & RW2 \\ 
  Rwanda & NORTH & 1995 & 215.66 & 179.83 & 257.85 & RW2 \\ 
  Rwanda & NORTH & 1996 & 220.83 & 186.38 & 259.33 & RW2 \\ 
  Rwanda & NORTH & 1997 & 219.63 & 187.47 & 255.77 & RW2 \\ 
  Rwanda & NORTH & 1998 & 211.33 & 178.80 & 249.28 & RW2 \\ 
  Rwanda & NORTH & 1999 & 197.17 & 163.43 & 237.34 & RW2 \\ 
  Rwanda & NORTH & 2000 & 177.72 & 144.61 & 213.38 & RW2 \\ 
  Rwanda & NORTH & 2001 & 158.86 & 131.79 & 189.23 & RW2 \\ 
  Rwanda & NORTH & 2002 & 140.90 & 117.96 & 167.25 & RW2 \\ 
  Rwanda & NORTH & 2003 & 124.69 & 103.19 & 150.78 & RW2 \\ 
  Rwanda & NORTH & 2004 & 110.72 & 89.64 & 137.20 & RW2 \\ 
  Rwanda & NORTH & 2005 & 98.93 & 79.43 & 123.34 & RW2 \\ 
  Rwanda & NORTH & 2006 & 88.34 & 71.93 & 108.51 & RW2 \\ 
  Rwanda & NORTH & 2007 & 78.97 & 64.67 & 96.28 & RW2 \\ 
  Rwanda & NORTH & 2008 & 70.62 & 56.51 & 87.64 & RW2 \\ 
  Rwanda & NORTH & 2009 & 63.17 & 48.76 & 81.18 & RW2 \\ 
  Rwanda & NORTH & 2010 & 56.51 & 42.31 & 75.26 & RW2 \\ 
  Rwanda & NORTH & 2011 & 50.46 & 37.55 & 67.54 & RW2 \\ 
  Rwanda & NORTH & 2012 & 45.08 & 33.20 & 60.67 & RW2 \\ 
  Rwanda & NORTH & 2013 & 40.23 & 28.27 & 56.85 & RW2 \\ 
  Rwanda & NORTH & 2014 & 35.83 & 22.24 & 56.92 & RW2 \\ 
  Rwanda & NORTH & 2015 & 31.91 & 16.10 & 62.01 & RW2 \\ 
  Rwanda & NORTH & 2016 & 28.38 & 11.31 & 68.43 & RW2 \\ 
  Rwanda & NORTH & 2017 & 25.30 & 7.79 & 78.18 & RW2 \\ 
  Rwanda & NORTH & 2018 & 22.41 & 5.16 & 90.88 & RW2 \\ 
  Rwanda & NORTH & 2019 & 20.02 & 3.33 & 110.31 & RW2 \\ 
  Rwanda & SOUTH & 1980 & 187.31 & 133.64 & 254.01 & RW2 \\ 
  Rwanda & SOUTH & 1981 & 180.15 & 143.31 & 222.74 & RW2 \\ 
  Rwanda & SOUTH & 1982 & 172.93 & 142.01 & 208.91 & RW2 \\ 
  Rwanda & SOUTH & 1983 & 166.47 & 134.17 & 206.10 & RW2 \\ 
  Rwanda & SOUTH & 1984 & 161.06 & 127.50 & 203.39 & RW2 \\ 
  Rwanda & SOUTH & 1985 & 156.43 & 125.64 & 191.91 & RW2 \\ 
  Rwanda & SOUTH & 1986 & 154.84 & 127.90 & 185.41 & RW2 \\ 
  Rwanda & SOUTH & 1987 & 155.75 & 130.67 & 185.16 & RW2 \\ 
  Rwanda & SOUTH & 1988 & 159.49 & 132.31 & 191.66 & RW2 \\ 
  Rwanda & SOUTH & 1989 & 165.97 & 135.71 & 201.66 & RW2 \\ 
  Rwanda & SOUTH & 1990 & 175.29 & 145.00 & 213.06 & RW2 \\ 
  Rwanda & SOUTH & 1991 & 186.38 & 157.66 & 221.13 & RW2 \\ 
  Rwanda & SOUTH & 1992 & 198.59 & 169.16 & 231.73 & RW2 \\ 
  Rwanda & SOUTH & 1993 & 210.88 & 177.61 & 246.19 & RW2 \\ 
  Rwanda & SOUTH & 1994 & 222.09 & 183.03 & 261.82 & RW2 \\ 
  Rwanda & SOUTH & 1995 & 231.50 & 193.66 & 275.00 & RW2 \\ 
  Rwanda & SOUTH & 1996 & 234.68 & 198.98 & 273.65 & RW2 \\ 
  Rwanda & SOUTH & 1997 & 231.59 & 197.96 & 268.24 & RW2 \\ 
  Rwanda & SOUTH & 1998 & 222.16 & 188.93 & 260.05 & RW2 \\ 
  Rwanda & SOUTH & 1999 & 207.42 & 172.39 & 248.09 & RW2 \\ 
  Rwanda & SOUTH & 2000 & 187.62 & 153.38 & 223.60 & RW2 \\ 
  Rwanda & SOUTH & 2001 & 168.33 & 140.15 & 199.03 & RW2 \\ 
  Rwanda & SOUTH & 2002 & 149.40 & 126.03 & 176.69 & RW2 \\ 
  Rwanda & SOUTH & 2003 & 131.96 & 110.12 & 158.80 & RW2 \\ 
  Rwanda & SOUTH & 2004 & 116.57 & 94.78 & 143.39 & RW2 \\ 
  Rwanda & SOUTH & 2005 & 103.39 & 83.48 & 128.23 & RW2 \\ 
  Rwanda & SOUTH & 2006 & 91.90 & 75.34 & 112.23 & RW2 \\ 
  Rwanda & SOUTH & 2007 & 82.01 & 67.38 & 99.18 & RW2 \\ 
  Rwanda & SOUTH & 2008 & 73.59 & 59.27 & 90.64 & RW2 \\ 
  Rwanda & SOUTH & 2009 & 66.24 & 51.54 & 84.38 & RW2 \\ 
  Rwanda & SOUTH & 2010 & 60.00 & 45.51 & 79.15 & RW2 \\ 
  Rwanda & SOUTH & 2011 & 54.36 & 41.24 & 71.74 & RW2 \\ 
  Rwanda & SOUTH & 2012 & 49.33 & 37.48 & 65.21 & RW2 \\ 
  Rwanda & SOUTH & 2013 & 44.80 & 32.49 & 62.44 & RW2 \\ 
  Rwanda & SOUTH & 2014 & 40.69 & 26.05 & 63.69 & RW2 \\ 
  Rwanda & SOUTH & 2015 & 36.92 & 19.19 & 70.28 & RW2 \\ 
  Rwanda & SOUTH & 2016 & 33.42 & 13.83 & 78.91 & RW2 \\ 
  Rwanda & SOUTH & 2017 & 30.37 & 9.68 & 91.72 & RW2 \\ 
  Rwanda & SOUTH & 2018 & 27.43 & 6.53 & 109.20 & RW2 \\ 
  Rwanda & SOUTH & 2019 & 24.88 & 4.31 & 133.40 & RW2 \\ 
  Rwanda & WEST & 1980 & 240.83 & 173.58 & 324.77 & RW2 \\ 
  Rwanda & WEST & 1981 & 219.83 & 175.33 & 271.58 & RW2 \\ 
  Rwanda & WEST & 1982 & 200.28 & 164.20 & 241.61 & RW2 \\ 
  Rwanda & WEST & 1983 & 182.41 & 146.96 & 226.25 & RW2 \\ 
  Rwanda & WEST & 1984 & 167.23 & 132.62 & 211.15 & RW2 \\ 
  Rwanda & WEST & 1985 & 154.00 & 122.70 & 189.43 & RW2 \\ 
  Rwanda & WEST & 1986 & 144.99 & 119.02 & 175.30 & RW2 \\ 
  Rwanda & WEST & 1987 & 139.71 & 116.30 & 166.97 & RW2 \\ 
  Rwanda & WEST & 1988 & 137.97 & 113.48 & 166.80 & RW2 \\ 
  Rwanda & WEST & 1989 & 139.70 & 112.94 & 171.78 & RW2 \\ 
  Rwanda & WEST & 1990 & 145.26 & 118.66 & 178.79 & RW2 \\ 
  Rwanda & WEST & 1991 & 153.92 & 128.43 & 184.13 & RW2 \\ 
  Rwanda & WEST & 1992 & 165.25 & 139.47 & 194.49 & RW2 \\ 
  Rwanda & WEST & 1993 & 178.70 & 149.54 & 210.89 & RW2 \\ 
  Rwanda & WEST & 1994 & 192.76 & 157.67 & 230.15 & RW2 \\ 
  Rwanda & WEST & 1995 & 206.18 & 170.96 & 248.17 & RW2 \\ 
  Rwanda & WEST & 1996 & 213.62 & 180.02 & 251.25 & RW2 \\ 
  Rwanda & WEST & 1997 & 214.22 & 182.10 & 250.78 & RW2 \\ 
  Rwanda & WEST & 1998 & 207.48 & 174.93 & 245.38 & RW2 \\ 
  Rwanda & WEST & 1999 & 194.05 & 160.82 & 233.99 & RW2 \\ 
  Rwanda & WEST & 2000 & 175.19 & 143.02 & 210.35 & RW2 \\ 
  Rwanda & WEST & 2001 & 156.89 & 130.21 & 186.63 & RW2 \\ 
  Rwanda & WEST & 2002 & 139.46 & 117.07 & 165.69 & RW2 \\ 
  Rwanda & WEST & 2003 & 124.06 & 102.72 & 149.73 & RW2 \\ 
  Rwanda & WEST & 2004 & 111.02 & 89.63 & 136.77 & RW2 \\ 
  Rwanda & WEST & 2005 & 100.23 & 80.40 & 124.68 & RW2 \\ 
  Rwanda & WEST & 2006 & 90.81 & 74.25 & 111.32 & RW2 \\ 
  Rwanda & WEST & 2007 & 82.76 & 68.21 & 100.45 & RW2 \\ 
  Rwanda & WEST & 2008 & 75.66 & 61.05 & 93.36 & RW2 \\ 
  Rwanda & WEST & 2009 & 69.24 & 53.87 & 88.34 & RW2 \\ 
  Rwanda & WEST & 2010 & 63.62 & 47.97 & 84.26 & RW2 \\ 
  Rwanda & WEST & 2011 & 58.40 & 44.00 & 77.63 & RW2 \\ 
  Rwanda & WEST & 2012 & 53.63 & 40.12 & 71.82 & RW2 \\ 
  Rwanda & WEST & 2013 & 49.24 & 34.95 & 69.53 & RW2 \\ 
  Rwanda & WEST & 2014 & 45.15 & 28.11 & 71.94 & RW2 \\ 
  Rwanda & WEST & 2015 & 41.48 & 20.97 & 80.60 & RW2 \\ 
  Rwanda & WEST & 2016 & 38.00 & 15.17 & 91.72 & RW2 \\ 
  Rwanda & WEST & 2017 & 34.94 & 10.70 & 107.92 & RW2 \\ 
  Rwanda & WEST & 2018 & 31.91 & 7.33 & 128.65 & RW2 \\ 
  Rwanda & WEST & 2019 & 29.37 & 4.93 & 156.79 & RW2 \\ 
  Senegal & ALL & 1980 & 195.96 & 190.93 & 201.05 & IHME \\ 
  Senegal & ALL & 1980 & 208.35 & 158.91 & 267.68 & RW2 \\ 
  Senegal & ALL & 1980 & 205.20 & 195.00 & 216.00 & UN \\ 
  Senegal & ALL & 1981 & 189.34 & 184.65 & 194.10 & IHME \\ 
  Senegal & ALL & 1981 & 200.03 & 165.46 & 239.38 & RW2 \\ 
  Senegal & ALL & 1981 & 198.60 & 188.80 & 208.80 & UN \\ 
  Senegal & ALL & 1982 & 182.51 & 178.09 & 187.06 & IHME \\ 
  Senegal & ALL & 1982 & 192.05 & 161.24 & 227.03 & RW2 \\ 
  Senegal & ALL & 1982 & 192.50 & 182.90 & 202.20 & UN \\ 
  Senegal & ALL & 1983 & 176.24 & 172.04 & 180.47 & IHME \\ 
  Senegal & ALL & 1983 & 184.12 & 151.09 & 222.29 & RW2 \\ 
  Senegal & ALL & 1983 & 186.50 & 177.30 & 195.70 & UN \\ 
  Senegal & ALL & 1984 & 170.66 & 166.49 & 174.73 & IHME \\ 
  Senegal & ALL & 1984 & 176.83 & 141.70 & 217.81 & RW2 \\ 
  Senegal & ALL & 1984 & 179.90 & 171.20 & 189.00 & UN \\ 
  Senegal & ALL & 1985 & 165.18 & 161.29 & 169.01 & IHME \\ 
  Senegal & ALL & 1985 & 169.53 & 138.30 & 205.48 & RW2 \\ 
  Senegal & ALL & 1985 & 172.60 & 164.20 & 181.40 & UN \\ 
  Senegal & ALL & 1986 & 160.20 & 156.55 & 163.88 & IHME \\ 
  Senegal & ALL & 1986 & 162.81 & 134.88 & 194.49 & RW2 \\ 
  Senegal & ALL & 1986 & 164.50 & 156.60 & 172.80 & UN \\ 
  Senegal & ALL & 1987 & 155.43 & 151.92 & 159.14 & IHME \\ 
  Senegal & ALL & 1987 & 156.61 & 131.16 & 186.39 & RW2 \\ 
  Senegal & ALL & 1987 & 156.60 & 148.90 & 164.70 & UN \\ 
  Senegal & ALL & 1988 & 151.14 & 147.53 & 154.76 & IHME \\ 
  Senegal & ALL & 1988 & 150.91 & 125.26 & 181.50 & RW2 \\ 
  Senegal & ALL & 1988 & 149.60 & 142.20 & 157.40 & UN \\ 
  Senegal & ALL & 1989 & 147.42 & 144.02 & 151.14 & IHME \\ 
  Senegal & ALL & 1989 & 146.09 & 119.93 & 178.32 & RW2 \\ 
  Senegal & ALL & 1989 & 144.10 & 136.80 & 151.60 & UN \\ 
  Senegal & ALL & 1990 & 144.23 & 140.95 & 147.83 & IHME \\ 
  Senegal & ALL & 1990 & 141.76 & 115.90 & 172.30 & RW2 \\ 
  Senegal & ALL & 1990 & 140.40 & 133.20 & 147.80 & UN \\ 
  Senegal & ALL & 1991 & 141.53 & 138.15 & 145.17 & IHME \\ 
  Senegal & ALL & 1991 & 139.35 & 115.48 & 167.08 & RW2 \\ 
  Senegal & ALL & 1991 & 138.70 & 131.60 & 145.90 & UN \\ 
  Senegal & ALL & 1992 & 139.46 & 136.00 & 143.16 & IHME \\ 
  Senegal & ALL & 1992 & 138.35 & 115.23 & 165.11 & RW2 \\ 
  Senegal & ALL & 1992 & 138.50 & 131.30 & 145.60 & UN \\ 
  Senegal & ALL & 1993 & 137.66 & 134.30 & 141.48 & IHME \\ 
  Senegal & ALL & 1993 & 138.66 & 114.66 & 166.19 & RW2 \\ 
  Senegal & ALL & 1993 & 139.30 & 132.20 & 146.50 & UN \\ 
  Senegal & ALL & 1994 & 135.84 & 132.44 & 139.54 & IHME \\ 
  Senegal & ALL & 1994 & 139.83 & 113.81 & 169.66 & RW2 \\ 
  Senegal & ALL & 1994 & 140.60 & 133.60 & 148.10 & UN \\ 
  Senegal & ALL & 1995 & 134.01 & 130.65 & 137.70 & IHME \\ 
  Senegal & ALL & 1995 & 142.43 & 117.40 & 173.74 & RW2 \\ 
  Senegal & ALL & 1995 & 142.10 & 134.80 & 149.90 & UN \\ 
  Senegal & ALL & 1996 & 131.63 & 128.24 & 135.17 & IHME \\ 
  Senegal & ALL & 1996 & 143.56 & 119.43 & 173.08 & RW2 \\ 
  Senegal & ALL & 1996 & 143.20 & 135.80 & 151.20 & UN \\ 
  Senegal & ALL & 1997 & 129.02 & 125.57 & 132.44 & IHME \\ 
  Senegal & ALL & 1997 & 143.68 & 119.92 & 171.15 & RW2 \\ 
  Senegal & ALL & 1997 & 143.40 & 135.70 & 151.80 & UN \\ 
  Senegal & ALL & 1998 & 125.79 & 122.43 & 129.16 & IHME \\ 
  Senegal & ALL & 1998 & 142.35 & 117.79 & 171.00 & RW2 \\ 
  Senegal & ALL & 1998 & 142.40 & 134.50 & 151.00 & UN \\ 
  Senegal & ALL & 1999 & 122.01 & 118.62 & 125.44 & IHME \\ 
  Senegal & ALL & 1999 & 139.26 & 113.63 & 168.49 & RW2 \\ 
  Senegal & ALL & 1999 & 139.70 & 131.50 & 148.40 & UN \\ 
  Senegal & ALL & 2000 & 117.63 & 114.35 & 121.06 & IHME \\ 
  Senegal & ALL & 2000 & 134.28 & 109.45 & 162.38 & RW2 \\ 
  Senegal & ALL & 2000 & 134.90 & 126.90 & 143.90 & UN \\ 
  Senegal & ALL & 2001 & 112.72 & 109.51 & 115.93 & IHME \\ 
  Senegal & ALL & 2001 & 128.17 & 105.52 & 153.93 & RW2 \\ 
  Senegal & ALL & 2001 & 128.40 & 120.60 & 137.20 & UN \\ 
  Senegal & ALL & 2002 & 107.97 & 104.91 & 111.02 & IHME \\ 
  Senegal & ALL & 2002 & 120.97 & 100.50 & 144.95 & RW2 \\ 
  Senegal & ALL & 2002 & 120.70 & 113.10 & 129.20 & UN \\ 
  Senegal & ALL & 2003 & 102.56 & 99.61 & 105.43 & IHME \\ 
  Senegal & ALL & 2003 & 113.17 & 93.77 & 136.75 & RW2 \\ 
  Senegal & ALL & 2003 & 112.50 & 105.10 & 120.50 & UN \\ 
  Senegal & ALL & 2004 & 97.34 & 94.45 & 100.37 & IHME \\ 
  Senegal & ALL & 2004 & 104.94 & 85.50 & 129.08 & RW2 \\ 
  Senegal & ALL & 2004 & 104.20 & 97.00 & 111.90 & UN \\ 
  Senegal & ALL & 2005 & 91.99 & 89.13 & 94.95 & IHME \\ 
  Senegal & ALL & 2005 & 96.64 & 77.91 & 118.80 & RW2 \\ 
  Senegal & ALL & 2005 & 96.20 & 89.10 & 103.80 & UN \\ 
  Senegal & ALL & 2006 & 87.01 & 84.08 & 89.94 & IHME \\ 
  Senegal & ALL & 2006 & 89.04 & 72.72 & 108.30 & RW2 \\ 
  Senegal & ALL & 2006 & 88.80 & 81.70 & 96.40 & UN \\ 
  Senegal & ALL & 2007 & 82.02 & 79.04 & 84.97 & IHME \\ 
  Senegal & ALL & 2007 & 82.09 & 67.66 & 99.27 & RW2 \\ 
  Senegal & ALL & 2007 & 82.00 & 74.80 & 89.70 & UN \\ 
  Senegal & ALL & 2008 & 77.44 & 74.35 & 80.41 & IHME \\ 
  Senegal & ALL & 2008 & 75.90 & 62.01 & 92.75 & RW2 \\ 
  Senegal & ALL & 2008 & 75.80 & 68.50 & 83.70 & UN \\ 
  Senegal & ALL & 2009 & 73.00 & 69.83 & 76.09 & IHME \\ 
  Senegal & ALL & 2009 & 70.24 & 56.19 & 87.69 & RW2 \\ 
  Senegal & ALL & 2009 & 70.10 & 62.50 & 78.70 & UN \\ 
  Senegal & ALL & 2010 & 68.90 & 65.73 & 72.31 & IHME \\ 
  Senegal & ALL & 2010 & 65.38 & 51.56 & 83.52 & RW2 \\ 
  Senegal & ALL & 2010 & 64.80 & 56.60 & 74.30 & UN \\ 
  Senegal & ALL & 2011 & 64.97 & 61.66 & 68.71 & IHME \\ 
  Senegal & ALL & 2011 & 60.83 & 48.52 & 76.33 & RW2 \\ 
  Senegal & ALL & 2011 & 60.00 & 50.70 & 71.00 & UN \\ 
  Senegal & ALL & 2012 & 61.43 & 57.87 & 65.39 & IHME \\ 
  Senegal & ALL & 2012 & 56.62 & 45.96 & 69.57 & RW2 \\ 
  Senegal & ALL & 2012 & 55.90 & 45.50 & 68.60 & UN \\ 
  Senegal & ALL & 2013 & 58.07 & 54.37 & 62.41 & IHME \\ 
  Senegal & ALL & 2013 & 52.75 & 42.01 & 65.75 & RW2 \\ 
  Senegal & ALL & 2013 & 52.50 & 41.20 & 66.90 & UN \\ 
  Senegal & ALL & 2014 & 55.02 & 51.10 & 59.58 & IHME \\ 
  Senegal & ALL & 2014 & 49.08 & 35.44 & 67.36 & RW2 \\ 
  Senegal & ALL & 2014 & 49.70 & 37.50 & 65.60 & UN \\ 
  Senegal & ALL & 2015 & 52.25 & 48.23 & 57.16 & IHME \\ 
  Senegal & ALL & 2015 & 45.57 & 27.38 & 74.82 & RW2 \\ 
  Senegal & ALL & 2015 & 47.20 & 34.30 & 64.70 & UN \\ 
  Senegal & ALL & 2016 & 42.46 & 21.19 & 83.86 & RW2 \\ 
  Senegal & ALL & 2017 & 39.38 & 15.79 & 95.93 & RW2 \\ 
  Senegal & ALL & 2018 & 36.58 & 11.64 & 112.51 & RW2 \\ 
  Senegal & ALL & 2019 & 33.93 & 8.08 & 131.05 & RW2 \\ 
  Senegal & ALL & 80-84 & 191.01 & 200.05 & 182.27 & HT-Direct \\ 
  Senegal & ALL & 85-89 & 151.16 & 157.61 & 144.93 & HT-Direct \\ 
  Senegal & ALL & 90-94 & 135.19 & 140.62 & 129.95 & HT-Direct \\ 
  Senegal & ALL & 95-99 & 139.17 & 144.62 & 133.89 & HT-Direct \\ 
  Senegal & ALL & 00-04 & 111.41 & 115.73 & 107.23 & HT-Direct \\ 
  Senegal & ALL & 05-09 & 76.27 & 79.84 & 72.84 & HT-Direct \\ 
  Senegal & ALL & 10-14 & 56.64 & 60.72 & 52.83 & HT-Direct \\ 
  Senegal & ALL & 15-19 & 39.37 & 16.06 & 94.27 & RW2 \\ 
  Senegal & DAKAR & 1980 & 142.65 & 104.39 & 191.96 & RW2 \\ 
  Senegal & DAKAR & 1981 & 137.72 & 110.14 & 170.92 & RW2 \\ 
  Senegal & DAKAR & 1982 & 132.82 & 108.76 & 161.00 & RW2 \\ 
  Senegal & DAKAR & 1983 & 127.96 & 103.42 & 157.01 & RW2 \\ 
  Senegal & DAKAR & 1984 & 123.22 & 98.65 & 152.31 & RW2 \\ 
  Senegal & DAKAR & 1985 & 118.74 & 96.43 & 144.68 & RW2 \\ 
  Senegal & DAKAR & 1986 & 114.12 & 94.38 & 136.61 & RW2 \\ 
  Senegal & DAKAR & 1987 & 109.51 & 91.65 & 130.11 & RW2 \\ 
  Senegal & DAKAR & 1988 & 105.08 & 87.24 & 126.30 & RW2 \\ 
  Senegal & DAKAR & 1989 & 101.06 & 82.96 & 123.39 & RW2 \\ 
  Senegal & DAKAR & 1990 & 97.11 & 79.45 & 118.03 & RW2 \\ 
  Senegal & DAKAR & 1991 & 94.79 & 78.64 & 114.15 & RW2 \\ 
  Senegal & DAKAR & 1992 & 93.53 & 77.97 & 112.07 & RW2 \\ 
  Senegal & DAKAR & 1993 & 93.28 & 77.29 & 112.19 & RW2 \\ 
  Senegal & DAKAR & 1994 & 93.85 & 76.47 & 114.31 & RW2 \\ 
  Senegal & DAKAR & 1995 & 95.54 & 78.30 & 117.06 & RW2 \\ 
  Senegal & DAKAR & 1996 & 96.54 & 79.75 & 116.73 & RW2 \\ 
  Senegal & DAKAR & 1997 & 96.98 & 80.41 & 116.12 & RW2 \\ 
  Senegal & DAKAR & 1998 & 96.49 & 79.41 & 116.40 & RW2 \\ 
  Senegal & DAKAR & 1999 & 94.88 & 76.86 & 115.92 & RW2 \\ 
  Senegal & DAKAR & 2000 & 92.13 & 74.38 & 112.38 & RW2 \\ 
  Senegal & DAKAR & 2001 & 88.33 & 72.43 & 106.62 & RW2 \\ 
  Senegal & DAKAR & 2002 & 83.71 & 69.12 & 100.70 & RW2 \\ 
  Senegal & DAKAR & 2003 & 78.40 & 64.50 & 95.31 & RW2 \\ 
  Senegal & DAKAR & 2004 & 72.91 & 59.26 & 89.99 & RW2 \\ 
  Senegal & DAKAR & 2005 & 67.08 & 53.81 & 82.96 & RW2 \\ 
  Senegal & DAKAR & 2006 & 62.01 & 50.34 & 76.19 & RW2 \\ 
  Senegal & DAKAR & 2007 & 57.43 & 46.85 & 70.39 & RW2 \\ 
  Senegal & DAKAR & 2008 & 53.47 & 43.04 & 66.40 & RW2 \\ 
  Senegal & DAKAR & 2009 & 49.95 & 39.26 & 63.50 & RW2 \\ 
  Senegal & DAKAR & 2010 & 47.19 & 36.12 & 61.77 & RW2 \\ 
  Senegal & DAKAR & 2011 & 44.40 & 33.95 & 58.24 & RW2 \\ 
  Senegal & DAKAR & 2012 & 41.95 & 31.76 & 55.20 & RW2 \\ 
  Senegal & DAKAR & 2013 & 39.60 & 28.73 & 54.13 & RW2 \\ 
  Senegal & DAKAR & 2014 & 37.41 & 24.69 & 55.65 & RW2 \\ 
  Senegal & DAKAR & 2015 & 35.30 & 19.88 & 61.71 & RW2 \\ 
  Senegal & DAKAR & 2016 & 33.30 & 15.78 & 69.11 & RW2 \\ 
  Senegal & DAKAR & 2017 & 31.45 & 12.10 & 79.95 & RW2 \\ 
  Senegal & DAKAR & 2018 & 29.57 & 8.94 & 93.48 & RW2 \\ 
  Senegal & DAKAR & 2019 & 27.86 & 6.65 & 110.73 & RW2 \\ 
  Senegal & DIOURBEL & 1980 & 246.27 & 190.34 & 312.39 & RW2 \\ 
  Senegal & DIOURBEL & 1981 & 236.60 & 197.20 & 281.57 & RW2 \\ 
  Senegal & DIOURBEL & 1982 & 227.33 & 192.96 & 266.38 & RW2 \\ 
  Senegal & DIOURBEL & 1983 & 218.37 & 182.41 & 259.21 & RW2 \\ 
  Senegal & DIOURBEL & 1984 & 209.65 & 172.12 & 251.30 & RW2 \\ 
  Senegal & DIOURBEL & 1985 & 201.34 & 168.35 & 238.78 & RW2 \\ 
  Senegal & DIOURBEL & 1986 & 193.00 & 163.84 & 225.20 & RW2 \\ 
  Senegal & DIOURBEL & 1987 & 184.97 & 158.42 & 214.61 & RW2 \\ 
  Senegal & DIOURBEL & 1988 & 177.44 & 150.65 & 208.03 & RW2 \\ 
  Senegal & DIOURBEL & 1989 & 170.64 & 143.61 & 203.54 & RW2 \\ 
  Senegal & DIOURBEL & 1990 & 164.54 & 137.74 & 194.28 & RW2 \\ 
  Senegal & DIOURBEL & 1991 & 161.02 & 136.54 & 188.67 & RW2 \\ 
  Senegal & DIOURBEL & 1992 & 159.55 & 136.61 & 185.47 & RW2 \\ 
  Senegal & DIOURBEL & 1993 & 159.89 & 135.67 & 187.35 & RW2 \\ 
  Senegal & DIOURBEL & 1994 & 161.81 & 135.43 & 191.20 & RW2 \\ 
  Senegal & DIOURBEL & 1995 & 165.60 & 139.66 & 196.93 & RW2 \\ 
  Senegal & DIOURBEL & 1996 & 168.45 & 144.17 & 197.66 & RW2 \\ 
  Senegal & DIOURBEL & 1997 & 170.41 & 145.90 & 197.66 & RW2 \\ 
  Senegal & DIOURBEL & 1998 & 170.56 & 144.96 & 198.90 & RW2 \\ 
  Senegal & DIOURBEL & 1999 & 168.78 & 141.13 & 198.70 & RW2 \\ 
  Senegal & DIOURBEL & 2000 & 164.77 & 137.77 & 194.88 & RW2 \\ 
  Senegal & DIOURBEL & 2001 & 158.50 & 134.23 & 185.67 & RW2 \\ 
  Senegal & DIOURBEL & 2002 & 150.48 & 128.39 & 175.85 & RW2 \\ 
  Senegal & DIOURBEL & 2003 & 141.10 & 119.53 & 166.09 & RW2 \\ 
  Senegal & DIOURBEL & 2004 & 130.95 & 109.28 & 157.23 & RW2 \\ 
  Senegal & DIOURBEL & 2005 & 120.30 & 99.40 & 144.25 & RW2 \\ 
  Senegal & DIOURBEL & 2006 & 110.94 & 92.75 & 131.78 & RW2 \\ 
  Senegal & DIOURBEL & 2007 & 102.71 & 86.62 & 121.81 & RW2 \\ 
  Senegal & DIOURBEL & 2008 & 95.60 & 79.75 & 114.71 & RW2 \\ 
  Senegal & DIOURBEL & 2009 & 89.70 & 72.76 & 109.96 & RW2 \\ 
  Senegal & DIOURBEL & 2010 & 85.00 & 68.01 & 106.72 & RW2 \\ 
  Senegal & DIOURBEL & 2011 & 80.60 & 64.94 & 99.89 & RW2 \\ 
  Senegal & DIOURBEL & 2012 & 76.59 & 62.50 & 94.00 & RW2 \\ 
  Senegal & DIOURBEL & 2013 & 72.84 & 57.68 & 91.16 & RW2 \\ 
  Senegal & DIOURBEL & 2014 & 69.24 & 50.24 & 94.76 & RW2 \\ 
  Senegal & DIOURBEL & 2015 & 65.82 & 40.22 & 105.63 & RW2 \\ 
  Senegal & DIOURBEL & 2016 & 62.40 & 32.35 & 117.84 & RW2 \\ 
  Senegal & DIOURBEL & 2017 & 59.22 & 24.77 & 135.19 & RW2 \\ 
  Senegal & DIOURBEL & 2018 & 56.30 & 18.85 & 156.21 & RW2 \\ 
  Senegal & DIOURBEL & 2019 & 53.48 & 13.91 & 182.16 & RW2 \\ 
  Senegal & FATICK & 1980 & 226.54 & 174.16 & 289.49 & RW2 \\ 
  Senegal & FATICK & 1981 & 218.47 & 181.40 & 261.24 & RW2 \\ 
  Senegal & FATICK & 1982 & 210.55 & 176.85 & 247.78 & RW2 \\ 
  Senegal & FATICK & 1983 & 202.92 & 168.05 & 242.33 & RW2 \\ 
  Senegal & FATICK & 1984 & 195.38 & 159.48 & 236.30 & RW2 \\ 
  Senegal & FATICK & 1985 & 188.31 & 156.14 & 224.83 & RW2 \\ 
  Senegal & FATICK & 1986 & 180.93 & 152.43 & 212.16 & RW2 \\ 
  Senegal & FATICK & 1987 & 173.68 & 147.80 & 202.75 & RW2 \\ 
  Senegal & FATICK & 1988 & 166.73 & 141.28 & 196.20 & RW2 \\ 
  Senegal & FATICK & 1989 & 160.52 & 134.44 & 191.56 & RW2 \\ 
  Senegal & FATICK & 1990 & 154.58 & 129.02 & 183.45 & RW2 \\ 
  Senegal & FATICK & 1991 & 151.05 & 128.07 & 177.92 & RW2 \\ 
  Senegal & FATICK & 1992 & 149.18 & 127.39 & 173.94 & RW2 \\ 
  Senegal & FATICK & 1993 & 148.93 & 126.24 & 174.63 & RW2 \\ 
  Senegal & FATICK & 1994 & 149.99 & 125.07 & 177.99 & RW2 \\ 
  Senegal & FATICK & 1995 & 152.72 & 128.53 & 182.34 & RW2 \\ 
  Senegal & FATICK & 1996 & 154.22 & 131.09 & 181.32 & RW2 \\ 
  Senegal & FATICK & 1997 & 154.85 & 132.24 & 180.16 & RW2 \\ 
  Senegal & FATICK & 1998 & 153.71 & 130.18 & 179.82 & RW2 \\ 
  Senegal & FATICK & 1999 & 150.66 & 125.35 & 179.23 & RW2 \\ 
  Senegal & FATICK & 2000 & 145.54 & 121.00 & 173.12 & RW2 \\ 
  Senegal & FATICK & 2001 & 138.40 & 116.51 & 162.64 & RW2 \\ 
  Senegal & FATICK & 2002 & 129.69 & 109.90 & 152.26 & RW2 \\ 
  Senegal & FATICK & 2003 & 119.84 & 101.02 & 142.11 & RW2 \\ 
  Senegal & FATICK & 2004 & 109.68 & 91.06 & 132.58 & RW2 \\ 
  Senegal & FATICK & 2005 & 99.07 & 81.03 & 119.56 & RW2 \\ 
  Senegal & FATICK & 2006 & 89.95 & 74.49 & 107.94 & RW2 \\ 
  Senegal & FATICK & 2007 & 81.78 & 67.96 & 98.13 & RW2 \\ 
  Senegal & FATICK & 2008 & 74.81 & 61.45 & 90.79 & RW2 \\ 
  Senegal & FATICK & 2009 & 69.03 & 55.29 & 85.78 & RW2 \\ 
  Senegal & FATICK & 2010 & 64.25 & 50.37 & 81.76 & RW2 \\ 
  Senegal & FATICK & 2011 & 59.82 & 47.30 & 75.68 & RW2 \\ 
  Senegal & FATICK & 2012 & 55.85 & 44.25 & 70.14 & RW2 \\ 
  Senegal & FATICK & 2013 & 52.19 & 40.08 & 66.91 & RW2 \\ 
  Senegal & FATICK & 2014 & 48.80 & 34.26 & 68.06 & RW2 \\ 
  Senegal & FATICK & 2015 & 45.47 & 27.41 & 74.47 & RW2 \\ 
  Senegal & FATICK & 2016 & 42.39 & 21.32 & 81.94 & RW2 \\ 
  Senegal & FATICK & 2017 & 39.62 & 16.28 & 92.63 & RW2 \\ 
  Senegal & FATICK & 2018 & 36.73 & 12.15 & 107.26 & RW2 \\ 
  Senegal & FATICK & 2019 & 34.24 & 8.82 & 124.85 & RW2 \\ 
  Senegal & KAOLACK & 1980 & 223.38 & 171.78 & 285.76 & RW2 \\ 
  Senegal & KAOLACK & 1981 & 216.24 & 179.67 & 258.40 & RW2 \\ 
  Senegal & KAOLACK & 1982 & 209.24 & 176.98 & 245.48 & RW2 \\ 
  Senegal & KAOLACK & 1983 & 202.33 & 168.52 & 241.01 & RW2 \\ 
  Senegal & KAOLACK & 1984 & 195.46 & 160.45 & 235.41 & RW2 \\ 
  Senegal & KAOLACK & 1985 & 189.03 & 157.60 & 224.62 & RW2 \\ 
  Senegal & KAOLACK & 1986 & 182.31 & 154.32 & 212.81 & RW2 \\ 
  Senegal & KAOLACK & 1987 & 175.48 & 149.95 & 203.57 & RW2 \\ 
  Senegal & KAOLACK & 1988 & 169.19 & 143.60 & 198.30 & RW2 \\ 
  Senegal & KAOLACK & 1989 & 163.44 & 137.26 & 194.69 & RW2 \\ 
  Senegal & KAOLACK & 1990 & 157.94 & 132.34 & 186.53 & RW2 \\ 
  Senegal & KAOLACK & 1991 & 154.82 & 131.74 & 181.31 & RW2 \\ 
  Senegal & KAOLACK & 1992 & 153.52 & 131.58 & 178.50 & RW2 \\ 
  Senegal & KAOLACK & 1993 & 153.97 & 131.20 & 179.72 & RW2 \\ 
  Senegal & KAOLACK & 1994 & 155.71 & 130.56 & 183.67 & RW2 \\ 
  Senegal & KAOLACK & 1995 & 159.21 & 134.49 & 189.05 & RW2 \\ 
  Senegal & KAOLACK & 1996 & 161.60 & 138.04 & 188.76 & RW2 \\ 
  Senegal & KAOLACK & 1997 & 163.02 & 139.73 & 189.06 & RW2 \\ 
  Senegal & KAOLACK & 1998 & 162.78 & 138.88 & 189.36 & RW2 \\ 
  Senegal & KAOLACK & 1999 & 160.45 & 134.56 & 189.10 & RW2 \\ 
  Senegal & KAOLACK & 2000 & 155.75 & 130.43 & 183.60 & RW2 \\ 
  Senegal & KAOLACK & 2001 & 148.90 & 126.46 & 173.91 & RW2 \\ 
  Senegal & KAOLACK & 2002 & 140.01 & 119.89 & 162.71 & RW2 \\ 
  Senegal & KAOLACK & 2003 & 129.96 & 110.60 & 153.32 & RW2 \\ 
  Senegal & KAOLACK & 2004 & 119.04 & 99.79 & 142.91 & RW2 \\ 
  Senegal & KAOLACK & 2005 & 107.64 & 89.29 & 128.42 & RW2 \\ 
  Senegal & KAOLACK & 2006 & 97.59 & 82.12 & 115.31 & RW2 \\ 
  Senegal & KAOLACK & 2007 & 88.64 & 75.14 & 104.29 & RW2 \\ 
  Senegal & KAOLACK & 2008 & 80.81 & 67.80 & 96.16 & RW2 \\ 
  Senegal & KAOLACK & 2009 & 74.09 & 60.67 & 90.13 & RW2 \\ 
  Senegal & KAOLACK & 2010 & 68.52 & 55.21 & 85.57 & RW2 \\ 
  Senegal & KAOLACK & 2011 & 63.42 & 51.29 & 77.94 & RW2 \\ 
  Senegal & KAOLACK & 2012 & 58.74 & 48.16 & 71.45 & RW2 \\ 
  Senegal & KAOLACK & 2013 & 54.49 & 43.45 & 67.77 & RW2 \\ 
  Senegal & KAOLACK & 2014 & 50.44 & 36.50 & 68.79 & RW2 \\ 
  Senegal & KAOLACK & 2015 & 46.72 & 28.58 & 75.29 & RW2 \\ 
  Senegal & KAOLACK & 2016 & 43.32 & 22.17 & 82.88 & RW2 \\ 
  Senegal & KAOLACK & 2017 & 39.99 & 16.67 & 92.51 & RW2 \\ 
  Senegal & KAOLACK & 2018 & 36.88 & 12.37 & 105.59 & RW2 \\ 
  Senegal & KAOLACK & 2019 & 34.17 & 8.93 & 124.43 & RW2 \\ 
  Senegal & KOLDA & 1980 & 261.52 & 202.11 & 331.10 & RW2 \\ 
  Senegal & KOLDA & 1981 & 256.86 & 214.63 & 304.49 & RW2 \\ 
  Senegal & KOLDA & 1982 & 252.21 & 214.43 & 293.99 & RW2 \\ 
  Senegal & KOLDA & 1983 & 247.33 & 207.68 & 291.38 & RW2 \\ 
  Senegal & KOLDA & 1984 & 242.35 & 201.02 & 288.64 & RW2 \\ 
  Senegal & KOLDA & 1985 & 237.55 & 199.78 & 278.47 & RW2 \\ 
  Senegal & KOLDA & 1986 & 231.55 & 198.53 & 267.24 & RW2 \\ 
  Senegal & KOLDA & 1987 & 225.12 & 194.50 & 258.60 & RW2 \\ 
  Senegal & KOLDA & 1988 & 218.56 & 187.63 & 253.69 & RW2 \\ 
  Senegal & KOLDA & 1989 & 212.27 & 179.93 & 249.87 & RW2 \\ 
  Senegal & KOLDA & 1990 & 205.84 & 174.27 & 241.49 & RW2 \\ 
  Senegal & KOLDA & 1991 & 202.42 & 173.03 & 235.34 & RW2 \\ 
  Senegal & KOLDA & 1992 & 201.04 & 173.22 & 232.22 & RW2 \\ 
  Senegal & KOLDA & 1993 & 202.00 & 173.09 & 234.31 & RW2 \\ 
  Senegal & KOLDA & 1994 & 204.73 & 172.91 & 239.40 & RW2 \\ 
  Senegal & KOLDA & 1995 & 209.76 & 179.26 & 246.33 & RW2 \\ 
  Senegal & KOLDA & 1996 & 213.58 & 184.43 & 247.12 & RW2 \\ 
  Senegal & KOLDA & 1997 & 216.38 & 187.74 & 247.90 & RW2 \\ 
  Senegal & KOLDA & 1998 & 217.14 & 186.33 & 249.78 & RW2 \\ 
  Senegal & KOLDA & 1999 & 215.12 & 182.23 & 250.07 & RW2 \\ 
  Senegal & KOLDA & 2000 & 210.35 & 178.51 & 244.94 & RW2 \\ 
  Senegal & KOLDA & 2001 & 202.49 & 173.34 & 234.09 & RW2 \\ 
  Senegal & KOLDA & 2002 & 192.13 & 165.85 & 221.66 & RW2 \\ 
  Senegal & KOLDA & 2003 & 179.59 & 153.59 & 209.51 & RW2 \\ 
  Senegal & KOLDA & 2004 & 166.12 & 140.25 & 197.38 & RW2 \\ 
  Senegal & KOLDA & 2005 & 151.71 & 126.26 & 180.07 & RW2 \\ 
  Senegal & KOLDA & 2006 & 139.19 & 117.39 & 164.01 & RW2 \\ 
  Senegal & KOLDA & 2007 & 128.11 & 108.60 & 150.61 & RW2 \\ 
  Senegal & KOLDA & 2008 & 118.71 & 99.51 & 140.97 & RW2 \\ 
  Senegal & KOLDA & 2009 & 110.84 & 90.80 & 134.34 & RW2 \\ 
  Senegal & KOLDA & 2010 & 104.57 & 84.47 & 129.74 & RW2 \\ 
  Senegal & KOLDA & 2011 & 98.85 & 80.51 & 121.28 & RW2 \\ 
  Senegal & KOLDA & 2012 & 93.61 & 77.32 & 113.16 & RW2 \\ 
  Senegal & KOLDA & 2013 & 88.78 & 71.38 & 109.99 & RW2 \\ 
  Senegal & KOLDA & 2014 & 84.13 & 61.72 & 113.55 & RW2 \\ 
  Senegal & KOLDA & 2015 & 79.70 & 49.41 & 125.84 & RW2 \\ 
  Senegal & KOLDA & 2016 & 75.38 & 39.00 & 140.61 & RW2 \\ 
  Senegal & KOLDA & 2017 & 71.42 & 30.08 & 162.81 & RW2 \\ 
  Senegal & KOLDA & 2018 & 67.63 & 22.66 & 186.80 & RW2 \\ 
  Senegal & KOLDA & 2019 & 64.11 & 16.97 & 216.09 & RW2 \\ 
  Senegal & LOUGA & 1980 & 212.52 & 162.34 & 273.52 & RW2 \\ 
  Senegal & LOUGA & 1981 & 203.33 & 167.39 & 246.16 & RW2 \\ 
  Senegal & LOUGA & 1982 & 194.80 & 162.34 & 231.81 & RW2 \\ 
  Senegal & LOUGA & 1983 & 186.28 & 153.18 & 224.95 & RW2 \\ 
  Senegal & LOUGA & 1984 & 177.87 & 144.13 & 216.69 & RW2 \\ 
  Senegal & LOUGA & 1985 & 169.93 & 139.97 & 204.12 & RW2 \\ 
  Senegal & LOUGA & 1986 & 161.64 & 135.79 & 191.00 & RW2 \\ 
  Senegal & LOUGA & 1987 & 153.51 & 129.95 & 179.91 & RW2 \\ 
  Senegal & LOUGA & 1988 & 145.68 & 122.91 & 172.51 & RW2 \\ 
  Senegal & LOUGA & 1989 & 138.40 & 115.20 & 166.65 & RW2 \\ 
  Senegal & LOUGA & 1990 & 131.33 & 108.95 & 156.88 & RW2 \\ 
  Senegal & LOUGA & 1991 & 126.58 & 106.77 & 149.61 & RW2 \\ 
  Senegal & LOUGA & 1992 & 123.36 & 104.40 & 145.01 & RW2 \\ 
  Senegal & LOUGA & 1993 & 121.71 & 102.48 & 143.90 & RW2 \\ 
  Senegal & LOUGA & 1994 & 121.15 & 100.05 & 144.87 & RW2 \\ 
  Senegal & LOUGA & 1995 & 122.30 & 101.70 & 147.45 & RW2 \\ 
  Senegal & LOUGA & 1996 & 122.55 & 103.05 & 145.61 & RW2 \\ 
  Senegal & LOUGA & 1997 & 122.32 & 102.92 & 144.09 & RW2 \\ 
  Senegal & LOUGA & 1998 & 121.01 & 100.94 & 143.60 & RW2 \\ 
  Senegal & LOUGA & 1999 & 118.38 & 97.40 & 142.03 & RW2 \\ 
  Senegal & LOUGA & 2000 & 114.44 & 94.02 & 137.72 & RW2 \\ 
  Senegal & LOUGA & 2001 & 109.15 & 90.74 & 129.94 & RW2 \\ 
  Senegal & LOUGA & 2002 & 102.71 & 86.27 & 121.82 & RW2 \\ 
  Senegal & LOUGA & 2003 & 95.63 & 79.50 & 114.85 & RW2 \\ 
  Senegal & LOUGA & 2004 & 88.33 & 72.77 & 107.77 & RW2 \\ 
  Senegal & LOUGA & 2005 & 80.69 & 65.63 & 98.24 & RW2 \\ 
  Senegal & LOUGA & 2006 & 74.34 & 61.37 & 89.74 & RW2 \\ 
  Senegal & LOUGA & 2007 & 68.81 & 57.17 & 82.78 & RW2 \\ 
  Senegal & LOUGA & 2008 & 64.16 & 52.62 & 78.06 & RW2 \\ 
  Senegal & LOUGA & 2009 & 60.37 & 48.31 & 75.16 & RW2 \\ 
  Senegal & LOUGA & 2010 & 57.41 & 45.23 & 73.35 & RW2 \\ 
  Senegal & LOUGA & 2011 & 54.71 & 43.38 & 69.02 & RW2 \\ 
  Senegal & LOUGA & 2012 & 52.24 & 41.83 & 65.32 & RW2 \\ 
  Senegal & LOUGA & 2013 & 50.00 & 39.02 & 63.94 & RW2 \\ 
  Senegal & LOUGA & 2014 & 47.82 & 34.21 & 66.72 & RW2 \\ 
  Senegal & LOUGA & 2015 & 45.65 & 27.71 & 74.42 & RW2 \\ 
  Senegal & LOUGA & 2016 & 43.63 & 22.23 & 83.85 & RW2 \\ 
  Senegal & LOUGA & 2017 & 41.64 & 17.45 & 96.95 & RW2 \\ 
  Senegal & LOUGA & 2018 & 39.76 & 13.20 & 114.69 & RW2 \\ 
  Senegal & LOUGA & 2019 & 38.05 & 9.87 & 137.21 & RW2 \\ 
  Senegal & MATAM & 1980 & 240.67 & 180.58 & 310.21 & RW2 \\ 
  Senegal & MATAM & 1981 & 234.32 & 189.86 & 282.92 & RW2 \\ 
  Senegal & MATAM & 1982 & 228.19 & 189.43 & 270.40 & RW2 \\ 
  Senegal & MATAM & 1983 & 221.62 & 182.93 & 265.10 & RW2 \\ 
  Senegal & MATAM & 1984 & 215.12 & 176.76 & 258.62 & RW2 \\ 
  Senegal & MATAM & 1985 & 208.60 & 174.17 & 247.19 & RW2 \\ 
  Senegal & MATAM & 1986 & 201.07 & 170.79 & 234.29 & RW2 \\ 
  Senegal & MATAM & 1987 & 193.05 & 165.90 & 224.16 & RW2 \\ 
  Senegal & MATAM & 1988 & 184.55 & 157.06 & 216.84 & RW2 \\ 
  Senegal & MATAM & 1989 & 176.14 & 147.83 & 210.12 & RW2 \\ 
  Senegal & MATAM & 1990 & 167.52 & 140.10 & 198.64 & RW2 \\ 
  Senegal & MATAM & 1991 & 161.06 & 136.41 & 189.82 & RW2 \\ 
  Senegal & MATAM & 1992 & 156.14 & 133.33 & 182.96 & RW2 \\ 
  Senegal & MATAM & 1993 & 152.84 & 128.99 & 180.32 & RW2 \\ 
  Senegal & MATAM & 1994 & 150.79 & 125.59 & 179.16 & RW2 \\ 
  Senegal & MATAM & 1995 & 150.36 & 126.05 & 180.05 & RW2 \\ 
  Senegal & MATAM & 1996 & 148.98 & 126.36 & 176.12 & RW2 \\ 
  Senegal & MATAM & 1997 & 146.81 & 124.83 & 171.98 & RW2 \\ 
  Senegal & MATAM & 1998 & 143.54 & 120.73 & 168.89 & RW2 \\ 
  Senegal & MATAM & 1999 & 138.75 & 114.92 & 165.56 & RW2 \\ 
  Senegal & MATAM & 2000 & 132.34 & 109.49 & 158.32 & RW2 \\ 
  Senegal & MATAM & 2001 & 124.56 & 104.05 & 147.64 & RW2 \\ 
  Senegal & MATAM & 2002 & 115.76 & 97.39 & 136.99 & RW2 \\ 
  Senegal & MATAM & 2003 & 106.23 & 88.55 & 126.88 & RW2 \\ 
  Senegal & MATAM & 2004 & 96.72 & 79.51 & 117.85 & RW2 \\ 
  Senegal & MATAM & 2005 & 87.13 & 70.77 & 106.11 & RW2 \\ 
  Senegal & MATAM & 2006 & 79.15 & 65.00 & 95.53 & RW2 \\ 
  Senegal & MATAM & 2007 & 72.15 & 59.72 & 87.05 & RW2 \\ 
  Senegal & MATAM & 2008 & 66.42 & 54.20 & 81.07 & RW2 \\ 
  Senegal & MATAM & 2009 & 61.60 & 49.03 & 77.14 & RW2 \\ 
  Senegal & MATAM & 2010 & 57.94 & 45.26 & 74.47 & RW2 \\ 
  Senegal & MATAM & 2011 & 54.44 & 42.77 & 69.29 & RW2 \\ 
  Senegal & MATAM & 2012 & 51.35 & 40.67 & 65.28 & RW2 \\ 
  Senegal & MATAM & 2013 & 48.52 & 37.26 & 63.29 & RW2 \\ 
  Senegal & MATAM & 2014 & 45.83 & 32.16 & 65.12 & RW2 \\ 
  Senegal & MATAM & 2015 & 43.25 & 25.89 & 71.50 & RW2 \\ 
  Senegal & MATAM & 2016 & 40.83 & 20.40 & 80.23 & RW2 \\ 
  Senegal & MATAM & 2017 & 38.50 & 15.81 & 91.98 & RW2 \\ 
  Senegal & MATAM & 2018 & 36.22 & 12.06 & 106.30 & RW2 \\ 
  Senegal & MATAM & 2019 & 34.23 & 8.85 & 126.07 & RW2 \\ 
  Senegal & SAINT-LOUIS & 1980 & 222.63 & 168.02 & 288.99 & RW2 \\ 
  Senegal & SAINT-LOUIS & 1981 & 214.28 & 174.99 & 259.66 & RW2 \\ 
  Senegal & SAINT-LOUIS & 1982 & 205.82 & 171.28 & 245.01 & RW2 \\ 
  Senegal & SAINT-LOUIS & 1983 & 197.55 & 162.18 & 237.97 & RW2 \\ 
  Senegal & SAINT-LOUIS & 1984 & 189.22 & 153.48 & 230.28 & RW2 \\ 
  Senegal & SAINT-LOUIS & 1985 & 181.07 & 149.33 & 217.82 & RW2 \\ 
  Senegal & SAINT-LOUIS & 1986 & 172.14 & 144.05 & 203.63 & RW2 \\ 
  Senegal & SAINT-LOUIS & 1987 & 162.80 & 137.50 & 192.21 & RW2 \\ 
  Senegal & SAINT-LOUIS & 1988 & 153.37 & 128.62 & 182.62 & RW2 \\ 
  Senegal & SAINT-LOUIS & 1989 & 144.28 & 120.00 & 174.19 & RW2 \\ 
  Senegal & SAINT-LOUIS & 1990 & 135.39 & 111.75 & 162.72 & RW2 \\ 
  Senegal & SAINT-LOUIS & 1991 & 128.59 & 107.60 & 153.01 & RW2 \\ 
  Senegal & SAINT-LOUIS & 1992 & 123.34 & 103.78 & 146.33 & RW2 \\ 
  Senegal & SAINT-LOUIS & 1993 & 119.65 & 99.97 & 142.61 & RW2 \\ 
  Senegal & SAINT-LOUIS & 1994 & 117.24 & 96.65 & 141.12 & RW2 \\ 
  Senegal & SAINT-LOUIS & 1995 & 116.51 & 96.10 & 141.20 & RW2 \\ 
  Senegal & SAINT-LOUIS & 1996 & 115.08 & 96.15 & 137.86 & RW2 \\ 
  Senegal & SAINT-LOUIS & 1997 & 113.45 & 94.76 & 134.49 & RW2 \\ 
  Senegal & SAINT-LOUIS & 1998 & 110.92 & 91.74 & 132.28 & RW2 \\ 
  Senegal & SAINT-LOUIS & 1999 & 107.54 & 87.92 & 130.09 & RW2 \\ 
  Senegal & SAINT-LOUIS & 2000 & 103.10 & 83.79 & 124.77 & RW2 \\ 
  Senegal & SAINT-LOUIS & 2001 & 97.67 & 80.49 & 117.26 & RW2 \\ 
  Senegal & SAINT-LOUIS & 2002 & 91.35 & 75.95 & 109.30 & RW2 \\ 
  Senegal & SAINT-LOUIS & 2003 & 84.55 & 69.93 & 102.24 & RW2 \\ 
  Senegal & SAINT-LOUIS & 2004 & 77.67 & 63.21 & 95.60 & RW2 \\ 
  Senegal & SAINT-LOUIS & 2005 & 70.62 & 56.91 & 86.86 & RW2 \\ 
  Senegal & SAINT-LOUIS & 2006 & 64.76 & 52.73 & 79.15 & RW2 \\ 
  Senegal & SAINT-LOUIS & 2007 & 59.70 & 48.80 & 73.09 & RW2 \\ 
  Senegal & SAINT-LOUIS & 2008 & 55.50 & 44.74 & 68.88 & RW2 \\ 
  Senegal & SAINT-LOUIS & 2009 & 52.04 & 40.75 & 66.44 & RW2 \\ 
  Senegal & SAINT-LOUIS & 2010 & 49.45 & 37.80 & 64.92 & RW2 \\ 
  Senegal & SAINT-LOUIS & 2011 & 47.01 & 35.90 & 61.82 & RW2 \\ 
  Senegal & SAINT-LOUIS & 2012 & 44.78 & 34.07 & 59.28 & RW2 \\ 
  Senegal & SAINT-LOUIS & 2013 & 42.71 & 31.52 & 58.41 & RW2 \\ 
  Senegal & SAINT-LOUIS & 2014 & 40.76 & 27.65 & 60.63 & RW2 \\ 
  Senegal & SAINT-LOUIS & 2015 & 38.89 & 22.40 & 67.22 & RW2 \\ 
  Senegal & SAINT-LOUIS & 2016 & 37.19 & 18.11 & 75.75 & RW2 \\ 
  Senegal & SAINT-LOUIS & 2017 & 35.36 & 14.10 & 88.32 & RW2 \\ 
  Senegal & SAINT-LOUIS & 2018 & 33.77 & 10.79 & 103.29 & RW2 \\ 
  Senegal & SAINT-LOUIS & 2019 & 32.25 & 8.09 & 125.14 & RW2 \\ 
  Senegal & TAMBACOUNDA & 1980 & 228.90 & 171.96 & 296.29 & RW2 \\ 
  Senegal & TAMBACOUNDA & 1981 & 226.29 & 183.60 & 274.83 & RW2 \\ 
  Senegal & TAMBACOUNDA & 1982 & 223.61 & 184.82 & 265.96 & RW2 \\ 
  Senegal & TAMBACOUNDA & 1983 & 220.68 & 180.96 & 265.00 & RW2 \\ 
  Senegal & TAMBACOUNDA & 1984 & 217.67 & 177.50 & 263.52 & RW2 \\ 
  Senegal & TAMBACOUNDA & 1985 & 214.58 & 177.90 & 255.71 & RW2 \\ 
  Senegal & TAMBACOUNDA & 1986 & 210.71 & 178.29 & 247.06 & RW2 \\ 
  Senegal & TAMBACOUNDA & 1987 & 206.39 & 176.51 & 240.19 & RW2 \\ 
  Senegal & TAMBACOUNDA & 1988 & 201.74 & 171.67 & 236.11 & RW2 \\ 
  Senegal & TAMBACOUNDA & 1989 & 197.26 & 166.55 & 234.26 & RW2 \\ 
  Senegal & TAMBACOUNDA & 1990 & 192.62 & 161.83 & 226.49 & RW2 \\ 
  Senegal & TAMBACOUNDA & 1991 & 190.52 & 162.62 & 222.12 & RW2 \\ 
  Senegal & TAMBACOUNDA & 1992 & 189.99 & 163.47 & 219.43 & RW2 \\ 
  Senegal & TAMBACOUNDA & 1993 & 191.45 & 163.71 & 222.57 & RW2 \\ 
  Senegal & TAMBACOUNDA & 1994 & 194.52 & 164.09 & 227.95 & RW2 \\ 
  Senegal & TAMBACOUNDA & 1995 & 199.37 & 169.24 & 235.30 & RW2 \\ 
  Senegal & TAMBACOUNDA & 1996 & 203.10 & 174.47 & 235.84 & RW2 \\ 
  Senegal & TAMBACOUNDA & 1997 & 205.42 & 177.02 & 236.77 & RW2 \\ 
  Senegal & TAMBACOUNDA & 1998 & 205.84 & 175.96 & 237.97 & RW2 \\ 
  Senegal & TAMBACOUNDA & 1999 & 204.03 & 172.37 & 238.75 & RW2 \\ 
  Senegal & TAMBACOUNDA & 2000 & 199.40 & 168.45 & 233.27 & RW2 \\ 
  Senegal & TAMBACOUNDA & 2001 & 192.07 & 164.62 & 222.39 & RW2 \\ 
  Senegal & TAMBACOUNDA & 2002 & 182.49 & 156.80 & 210.91 & RW2 \\ 
  Senegal & TAMBACOUNDA & 2003 & 171.12 & 146.21 & 200.41 & RW2 \\ 
  Senegal & TAMBACOUNDA & 2004 & 158.93 & 133.79 & 189.16 & RW2 \\ 
  Senegal & TAMBACOUNDA & 2005 & 145.79 & 121.28 & 173.26 & RW2 \\ 
  Senegal & TAMBACOUNDA & 2006 & 134.49 & 113.30 & 158.87 & RW2 \\ 
  Senegal & TAMBACOUNDA & 2007 & 124.29 & 105.32 & 145.80 & RW2 \\ 
  Senegal & TAMBACOUNDA & 2008 & 115.56 & 96.84 & 137.55 & RW2 \\ 
  Senegal & TAMBACOUNDA & 2009 & 108.32 & 88.65 & 131.60 & RW2 \\ 
  Senegal & TAMBACOUNDA & 2010 & 102.53 & 82.64 & 127.57 & RW2 \\ 
  Senegal & TAMBACOUNDA & 2011 & 97.10 & 78.95 & 119.22 & RW2 \\ 
  Senegal & TAMBACOUNDA & 2012 & 92.11 & 75.84 & 111.49 & RW2 \\ 
  Senegal & TAMBACOUNDA & 2013 & 87.53 & 70.38 & 107.92 & RW2 \\ 
  Senegal & TAMBACOUNDA & 2014 & 83.12 & 60.99 & 111.48 & RW2 \\ 
  Senegal & TAMBACOUNDA & 2015 & 78.79 & 49.14 & 123.64 & RW2 \\ 
  Senegal & TAMBACOUNDA & 2016 & 74.62 & 39.29 & 138.14 & RW2 \\ 
  Senegal & TAMBACOUNDA & 2017 & 70.87 & 30.25 & 157.41 & RW2 \\ 
  Senegal & TAMBACOUNDA & 2018 & 67.02 & 22.87 & 180.87 & RW2 \\ 
  Senegal & TAMBACOUNDA & 2019 & 63.43 & 17.14 & 211.80 & RW2 \\ 
  Senegal & THIES & 1980 & 174.11 & 130.74 & 228.36 & RW2 \\ 
  Senegal & THIES & 1981 & 166.40 & 134.50 & 204.16 & RW2 \\ 
  Senegal & THIES & 1982 & 159.00 & 131.21 & 191.69 & RW2 \\ 
  Senegal & THIES & 1983 & 151.81 & 123.29 & 185.01 & RW2 \\ 
  Senegal & THIES & 1984 & 144.73 & 115.99 & 178.67 & RW2 \\ 
  Senegal & THIES & 1985 & 138.28 & 112.67 & 168.18 & RW2 \\ 
  Senegal & THIES & 1986 & 131.75 & 109.41 & 157.08 & RW2 \\ 
  Senegal & THIES & 1987 & 125.45 & 105.03 & 148.45 & RW2 \\ 
  Senegal & THIES & 1988 & 119.49 & 99.71 & 142.66 & RW2 \\ 
  Senegal & THIES & 1989 & 114.15 & 94.32 & 138.58 & RW2 \\ 
  Senegal & THIES & 1990 & 109.10 & 89.98 & 131.37 & RW2 \\ 
  Senegal & THIES & 1991 & 105.94 & 88.74 & 125.99 & RW2 \\ 
  Senegal & THIES & 1992 & 104.19 & 87.81 & 123.12 & RW2 \\ 
  Senegal & THIES & 1993 & 103.54 & 86.60 & 123.40 & RW2 \\ 
  Senegal & THIES & 1994 & 104.00 & 85.91 & 125.02 & RW2 \\ 
  Senegal & THIES & 1995 & 105.56 & 87.85 & 128.06 & RW2 \\ 
  Senegal & THIES & 1996 & 106.41 & 89.29 & 127.10 & RW2 \\ 
  Senegal & THIES & 1997 & 106.54 & 89.74 & 125.86 & RW2 \\ 
  Senegal & THIES & 1998 & 105.53 & 87.94 & 125.59 & RW2 \\ 
  Senegal & THIES & 1999 & 103.20 & 85.00 & 124.15 & RW2 \\ 
  Senegal & THIES & 2000 & 99.53 & 81.58 & 120.05 & RW2 \\ 
  Senegal & THIES & 2001 & 94.51 & 78.70 & 112.87 & RW2 \\ 
  Senegal & THIES & 2002 & 88.51 & 74.17 & 105.31 & RW2 \\ 
  Senegal & THIES & 2003 & 81.78 & 68.01 & 98.54 & RW2 \\ 
  Senegal & THIES & 2004 & 74.88 & 61.21 & 92.18 & RW2 \\ 
  Senegal & THIES & 2005 & 67.73 & 54.67 & 83.23 & RW2 \\ 
  Senegal & THIES & 2006 & 61.57 & 50.37 & 75.23 & RW2 \\ 
  Senegal & THIES & 2007 & 56.10 & 46.04 & 68.20 & RW2 \\ 
  Senegal & THIES & 2008 & 51.42 & 41.62 & 63.57 & RW2 \\ 
  Senegal & THIES & 2009 & 47.42 & 37.42 & 59.97 & RW2 \\ 
  Senegal & THIES & 2010 & 44.20 & 34.14 & 57.61 & RW2 \\ 
  Senegal & THIES & 2011 & 41.12 & 31.78 & 53.23 & RW2 \\ 
  Senegal & THIES & 2012 & 38.40 & 29.64 & 49.43 & RW2 \\ 
  Senegal & THIES & 2013 & 35.89 & 26.80 & 47.48 & RW2 \\ 
  Senegal & THIES & 2014 & 33.53 & 22.97 & 48.29 & RW2 \\ 
  Senegal & THIES & 2015 & 31.21 & 18.18 & 52.36 & RW2 \\ 
  Senegal & THIES & 2016 & 29.12 & 14.33 & 58.00 & RW2 \\ 
  Senegal & THIES & 2017 & 27.20 & 11.00 & 65.88 & RW2 \\ 
  Senegal & THIES & 2018 & 25.37 & 8.29 & 76.66 & RW2 \\ 
  Senegal & THIES & 2019 & 23.64 & 5.92 & 90.98 & RW2 \\ 
  Senegal & ZUGUINCHOR & 1980 & 224.57 & 163.56 & 300.89 & RW2 \\ 
  Senegal & ZUGUINCHOR & 1981 & 217.42 & 170.82 & 272.82 & RW2 \\ 
  Senegal & ZUGUINCHOR & 1982 & 210.15 & 169.72 & 257.63 & RW2 \\ 
  Senegal & ZUGUINCHOR & 1983 & 203.02 & 163.05 & 249.22 & RW2 \\ 
  Senegal & ZUGUINCHOR & 1984 & 195.90 & 156.08 & 242.09 & RW2 \\ 
  Senegal & ZUGUINCHOR & 1985 & 189.04 & 153.31 & 230.19 & RW2 \\ 
  Senegal & ZUGUINCHOR & 1986 & 181.56 & 150.71 & 217.50 & RW2 \\ 
  Senegal & ZUGUINCHOR & 1987 & 173.96 & 145.51 & 206.75 & RW2 \\ 
  Senegal & ZUGUINCHOR & 1988 & 166.60 & 138.30 & 199.74 & RW2 \\ 
  Senegal & ZUGUINCHOR & 1989 & 159.70 & 131.77 & 193.72 & RW2 \\ 
  Senegal & ZUGUINCHOR & 1990 & 152.85 & 125.90 & 183.91 & RW2 \\ 
  Senegal & ZUGUINCHOR & 1991 & 148.33 & 123.80 & 176.52 & RW2 \\ 
  Senegal & ZUGUINCHOR & 1992 & 145.51 & 122.47 & 172.27 & RW2 \\ 
  Senegal & ZUGUINCHOR & 1993 & 144.52 & 120.70 & 171.70 & RW2 \\ 
  Senegal & ZUGUINCHOR & 1994 & 144.84 & 119.59 & 173.79 & RW2 \\ 
  Senegal & ZUGUINCHOR & 1995 & 146.56 & 121.71 & 176.77 & RW2 \\ 
  Senegal & ZUGUINCHOR & 1996 & 147.38 & 124.13 & 174.84 & RW2 \\ 
  Senegal & ZUGUINCHOR & 1997 & 147.06 & 124.15 & 173.55 & RW2 \\ 
  Senegal & ZUGUINCHOR & 1998 & 145.36 & 121.57 & 172.31 & RW2 \\ 
  Senegal & ZUGUINCHOR & 1999 & 141.70 & 116.82 & 170.17 & RW2 \\ 
  Senegal & ZUGUINCHOR & 2000 & 135.89 & 111.75 & 163.75 & RW2 \\ 
  Senegal & ZUGUINCHOR & 2001 & 128.26 & 106.59 & 153.37 & RW2 \\ 
  Senegal & ZUGUINCHOR & 2002 & 119.02 & 99.16 & 142.13 & RW2 \\ 
  Senegal & ZUGUINCHOR & 2003 & 108.76 & 89.73 & 131.52 & RW2 \\ 
  Senegal & ZUGUINCHOR & 2004 & 98.33 & 80.00 & 121.06 & RW2 \\ 
  Senegal & ZUGUINCHOR & 2005 & 87.64 & 70.06 & 108.74 & RW2 \\ 
  Senegal & ZUGUINCHOR & 2006 & 78.57 & 63.35 & 96.83 & RW2 \\ 
  Senegal & ZUGUINCHOR & 2007 & 70.69 & 57.06 & 87.30 & RW2 \\ 
  Senegal & ZUGUINCHOR & 2008 & 64.09 & 50.73 & 80.62 & RW2 \\ 
  Senegal & ZUGUINCHOR & 2009 & 58.59 & 45.19 & 75.65 & RW2 \\ 
  Senegal & ZUGUINCHOR & 2010 & 54.20 & 40.80 & 72.34 & RW2 \\ 
  Senegal & ZUGUINCHOR & 2011 & 50.23 & 37.41 & 67.36 & RW2 \\ 
  Senegal & ZUGUINCHOR & 2012 & 46.67 & 34.39 & 63.03 & RW2 \\ 
  Senegal & ZUGUINCHOR & 2013 & 43.39 & 30.67 & 60.87 & RW2 \\ 
  Senegal & ZUGUINCHOR & 2014 & 40.33 & 25.99 & 61.99 & RW2 \\ 
  Senegal & ZUGUINCHOR & 2015 & 37.43 & 20.65 & 67.19 & RW2 \\ 
  Senegal & ZUGUINCHOR & 2016 & 34.88 & 15.92 & 74.04 & RW2 \\ 
  Senegal & ZUGUINCHOR & 2017 & 32.24 & 12.14 & 83.90 & RW2 \\ 
  Senegal & ZUGUINCHOR & 2018 & 29.92 & 9.00 & 96.33 & RW2 \\ 
  Senegal & ZUGUINCHOR & 2019 & 27.90 & 6.45 & 112.92 & RW2 \\ 
  Sierra Leone & ALL & 1980 & 275.05 & 260.44 & 289.33 & IHME \\ 
  Sierra Leone & ALL & 1980 & 279.36 & 192.54 & 385.69 & RW2 \\ 
  Sierra Leone & ALL & 1980 & 287.60 & 258.00 & 319.40 & UN \\ 
  Sierra Leone & ALL & 1981 & 273.24 & 259.96 & 285.89 & IHME \\ 
  Sierra Leone & ALL & 1981 & 278.52 & 209.77 & 358.94 & RW2 \\ 
  Sierra Leone & ALL & 1981 & 284.00 & 254.90 & 315.50 & UN \\ 
  Sierra Leone & ALL & 1982 & 270.83 & 258.91 & 282.58 & IHME \\ 
  Sierra Leone & ALL & 1982 & 277.90 & 219.57 & 344.56 & RW2 \\ 
  Sierra Leone & ALL & 1982 & 280.90 & 252.10 & 311.40 & UN \\ 
  Sierra Leone & ALL & 1983 & 268.44 & 257.48 & 279.28 & IHME \\ 
  Sierra Leone & ALL & 1983 & 276.70 & 221.36 & 339.70 & RW2 \\ 
  Sierra Leone & ALL & 1983 & 277.90 & 249.90 & 307.30 & UN \\ 
  Sierra Leone & ALL & 1984 & 265.68 & 255.59 & 275.97 & IHME \\ 
  Sierra Leone & ALL & 1984 & 276.07 & 219.96 & 339.23 & RW2 \\ 
  Sierra Leone & ALL & 1984 & 275.20 & 248.20 & 303.50 & UN \\ 
  Sierra Leone & ALL & 1985 & 263.01 & 253.54 & 272.42 & IHME \\ 
  Sierra Leone & ALL & 1985 & 274.68 & 223.19 & 333.40 & RW2 \\ 
  Sierra Leone & ALL & 1985 & 272.80 & 246.50 & 299.50 & UN \\ 
  Sierra Leone & ALL & 1986 & 260.24 & 251.04 & 269.39 & IHME \\ 
  Sierra Leone & ALL & 1986 & 273.36 & 224.75 & 328.19 & RW2 \\ 
  Sierra Leone & ALL & 1986 & 270.50 & 245.20 & 296.40 & UN \\ 
  Sierra Leone & ALL & 1987 & 257.03 & 248.62 & 265.74 & IHME \\ 
  Sierra Leone & ALL & 1987 & 271.94 & 225.62 & 324.60 & RW2 \\ 
  Sierra Leone & ALL & 1987 & 268.20 & 244.00 & 293.30 & UN \\ 
  Sierra Leone & ALL & 1988 & 253.58 & 245.57 & 261.97 & IHME \\ 
  Sierra Leone & ALL & 1988 & 270.07 & 223.16 & 323.02 & RW2 \\ 
  Sierra Leone & ALL & 1988 & 266.70 & 243.70 & 290.50 & UN \\ 
  Sierra Leone & ALL & 1989 & 249.81 & 242.31 & 257.69 & IHME \\ 
  Sierra Leone & ALL & 1989 & 268.11 & 220.70 & 322.01 & RW2 \\ 
  Sierra Leone & ALL & 1989 & 265.20 & 243.30 & 288.10 & UN \\ 
  Sierra Leone & ALL & 1990 & 245.64 & 238.57 & 253.63 & IHME \\ 
  Sierra Leone & ALL & 1990 & 265.91 & 219.74 & 317.79 & RW2 \\ 
  Sierra Leone & ALL & 1990 & 264.30 & 243.20 & 286.30 & UN \\ 
  Sierra Leone & ALL & 1991 & 242.15 & 235.07 & 250.01 & IHME \\ 
  Sierra Leone & ALL & 1991 & 263.91 & 220.20 & 311.91 & RW2 \\ 
  Sierra Leone & ALL & 1991 & 263.40 & 243.10 & 284.60 & UN \\ 
  Sierra Leone & ALL & 1992 & 238.43 & 231.55 & 246.02 & IHME \\ 
  Sierra Leone & ALL & 1992 & 261.95 & 219.41 & 308.61 & RW2 \\ 
  Sierra Leone & ALL & 1992 & 262.60 & 242.70 & 283.20 & UN \\ 
  Sierra Leone & ALL & 1993 & 234.39 & 227.95 & 241.79 & IHME \\ 
  Sierra Leone & ALL & 1993 & 260.08 & 217.02 & 307.75 & RW2 \\ 
  Sierra Leone & ALL & 1993 & 261.50 & 242.20 & 282.00 & UN \\ 
  Sierra Leone & ALL & 1994 & 230.81 & 224.14 & 238.18 & IHME \\ 
  Sierra Leone & ALL & 1994 & 257.99 & 213.33 & 308.50 & RW2 \\ 
  Sierra Leone & ALL & 1994 & 259.90 & 240.80 & 280.40 & UN \\ 
  Sierra Leone & ALL & 1995 & 227.38 & 220.61 & 234.57 & IHME \\ 
  Sierra Leone & ALL & 1995 & 256.16 & 212.67 & 305.56 & RW2 \\ 
  Sierra Leone & ALL & 1995 & 257.50 & 238.80 & 277.90 & UN \\ 
  Sierra Leone & ALL & 1996 & 222.53 & 216.29 & 229.23 & IHME \\ 
  Sierra Leone & ALL & 1996 & 253.31 & 211.96 & 300.83 & RW2 \\ 
  Sierra Leone & ALL & 1996 & 254.40 & 235.70 & 274.40 & UN \\ 
  Sierra Leone & ALL & 1997 & 219.57 & 213.36 & 225.99 & IHME \\ 
  Sierra Leone & ALL & 1997 & 249.88 & 209.98 & 295.09 & RW2 \\ 
  Sierra Leone & ALL & 1997 & 250.50 & 232.00 & 270.20 & UN \\ 
  Sierra Leone & ALL & 1998 & 218.84 & 211.90 & 226.46 & IHME \\ 
  Sierra Leone & ALL & 1998 & 245.87 & 205.53 & 292.71 & RW2 \\ 
  Sierra Leone & ALL & 1998 & 246.00 & 227.80 & 265.10 & UN \\ 
  Sierra Leone & ALL & 1999 & 216.68 & 209.05 & 225.07 & IHME \\ 
  Sierra Leone & ALL & 1999 & 241.11 & 199.64 & 288.13 & RW2 \\ 
  Sierra Leone & ALL & 1999 & 241.10 & 223.30 & 259.60 & UN \\ 
  Sierra Leone & ALL & 2000 & 210.36 & 204.53 & 216.51 & IHME \\ 
  Sierra Leone & ALL & 2000 & 235.44 & 195.29 & 280.48 & RW2 \\ 
  Sierra Leone & ALL & 2000 & 235.80 & 218.50 & 254.00 & UN \\ 
  Sierra Leone & ALL & 2001 & 205.34 & 199.17 & 211.14 & IHME \\ 
  Sierra Leone & ALL & 2001 & 229.77 & 191.97 & 272.13 & RW2 \\ 
  Sierra Leone & ALL & 2001 & 229.90 & 213.30 & 247.70 & UN \\ 
  Sierra Leone & ALL & 2002 & 199.62 & 193.42 & 205.48 & IHME \\ 
  Sierra Leone & ALL & 2002 & 223.76 & 187.94 & 264.12 & RW2 \\ 
  Sierra Leone & ALL & 2002 & 223.90 & 207.60 & 241.20 & UN \\ 
  Sierra Leone & ALL & 2003 & 193.63 & 187.42 & 199.44 & IHME \\ 
  Sierra Leone & ALL & 2003 & 217.66 & 181.94 & 257.85 & RW2 \\ 
  Sierra Leone & ALL & 2003 & 217.40 & 201.50 & 234.70 & UN \\ 
  Sierra Leone & ALL & 2004 & 187.57 & 181.41 & 193.36 & IHME \\ 
  Sierra Leone & ALL & 2004 & 210.92 & 173.66 & 252.61 & RW2 \\ 
  Sierra Leone & ALL & 2004 & 210.80 & 195.20 & 227.80 & UN \\ 
  Sierra Leone & ALL & 2005 & 181.49 & 175.37 & 187.32 & IHME \\ 
  Sierra Leone & ALL & 2005 & 204.45 & 168.75 & 246.01 & RW2 \\ 
  Sierra Leone & ALL & 2005 & 203.70 & 188.30 & 220.90 & UN \\ 
  Sierra Leone & ALL & 2006 & 175.31 & 169.00 & 181.60 & IHME \\ 
  Sierra Leone & ALL & 2006 & 196.80 & 164.01 & 234.32 & RW2 \\ 
  Sierra Leone & ALL & 2006 & 196.20 & 180.90 & 213.20 & UN \\ 
  Sierra Leone & ALL & 2007 & 169.06 & 162.65 & 175.45 & IHME \\ 
  Sierra Leone & ALL & 2007 & 188.60 & 158.10 & 223.32 & RW2 \\ 
  Sierra Leone & ALL & 2007 & 188.00 & 173.10 & 204.50 & UN \\ 
  Sierra Leone & ALL & 2008 & 162.66 & 155.85 & 169.59 & IHME \\ 
  Sierra Leone & ALL & 2008 & 179.94 & 149.57 & 215.18 & RW2 \\ 
  Sierra Leone & ALL & 2008 & 179.10 & 164.60 & 195.50 & UN \\ 
  Sierra Leone & ALL & 2009 & 156.71 & 149.69 & 163.98 & IHME \\ 
  Sierra Leone & ALL & 2009 & 170.52 & 139.21 & 208.10 & RW2 \\ 
  Sierra Leone & ALL & 2009 & 169.90 & 155.50 & 185.80 & UN \\ 
  Sierra Leone & ALL & 2010 & 150.56 & 143.18 & 158.00 & IHME \\ 
  Sierra Leone & ALL & 2010 & 160.68 & 128.57 & 199.03 & RW2 \\ 
  Sierra Leone & ALL & 2010 & 160.20 & 145.40 & 176.70 & UN \\ 
  Sierra Leone & ALL & 2011 & 144.62 & 136.81 & 152.43 & IHME \\ 
  Sierra Leone & ALL & 2011 & 151.33 & 122.18 & 185.39 & RW2 \\ 
  Sierra Leone & ALL & 2011 & 150.60 & 134.80 & 168.30 & UN \\ 
  Sierra Leone & ALL & 2012 & 138.39 & 130.24 & 147.07 & IHME \\ 
  Sierra Leone & ALL & 2012 & 142.12 & 116.34 & 172.15 & RW2 \\ 
  Sierra Leone & ALL & 2012 & 141.60 & 124.50 & 160.40 & UN \\ 
  Sierra Leone & ALL & 2013 & 131.99 & 123.37 & 141.18 & IHME \\ 
  Sierra Leone & ALL & 2013 & 133.36 & 107.04 & 164.89 & RW2 \\ 
  Sierra Leone & ALL & 2013 & 133.40 & 114.70 & 153.80 & UN \\ 
  Sierra Leone & ALL & 2014 & 128.46 & 119.39 & 138.14 & IHME \\ 
  Sierra Leone & ALL & 2014 & 124.97 & 91.41 & 168.38 & RW2 \\ 
  Sierra Leone & ALL & 2014 & 126.40 & 105.80 & 148.50 & UN \\ 
  Sierra Leone & ALL & 2015 & 122.40 & 113.03 & 132.22 & IHME \\ 
  Sierra Leone & ALL & 2015 & 117.03 & 72.07 & 184.66 & RW2 \\ 
  Sierra Leone & ALL & 2015 & 120.40 & 97.80 & 145.30 & UN \\ 
  Sierra Leone & ALL & 2016 & 109.82 & 56.52 & 203.25 & RW2 \\ 
  Sierra Leone & ALL & 2017 & 102.55 & 42.65 & 227.27 & RW2 \\ 
  Sierra Leone & ALL & 2018 & 95.89 & 31.78 & 258.95 & RW2 \\ 
  Sierra Leone & ALL & 2019 & 89.50 & 22.27 & 292.52 & RW2 \\ 
  Sierra Leone & ALL & 80-84 & 279.46 & 333.49 & 231.15 & HT-Direct \\ 
  Sierra Leone & ALL & 85-89 & 288.50 & 323.94 & 255.47 & HT-Direct \\ 
  Sierra Leone & ALL & 90-94 & 249.46 & 269.94 & 230.05 & HT-Direct \\ 
  Sierra Leone & ALL & 95-99 & 228.30 & 245.06 & 212.36 & HT-Direct \\ 
  Sierra Leone & ALL & 00-04 & 217.99 & 232.00 & 204.61 & HT-Direct \\ 
  Sierra Leone & ALL & 05-09 & 192.05 & 202.60 & 181.93 & HT-Direct \\ 
  Sierra Leone & ALL & 10-14 & 130.86 & 141.31 & 121.08 & HT-Direct \\ 
  Sierra Leone & ALL & 15-19 & 102.58 & 43.29 & 223.53 & RW2 \\ 
  Sierra Leone & EASTERN & 1980 & 320.52 & 222.57 & 437.89 & RW2 \\ 
  Sierra Leone & EASTERN & 1981 & 321.35 & 240.40 & 414.25 & RW2 \\ 
  Sierra Leone & EASTERN & 1982 & 322.29 & 251.80 & 401.70 & RW2 \\ 
  Sierra Leone & EASTERN & 1983 & 323.04 & 256.69 & 397.26 & RW2 \\ 
  Sierra Leone & EASTERN & 1984 & 323.98 & 258.99 & 396.12 & RW2 \\ 
  Sierra Leone & EASTERN & 1985 & 324.29 & 262.83 & 392.26 & RW2 \\ 
  Sierra Leone & EASTERN & 1986 & 324.87 & 266.92 & 387.95 & RW2 \\ 
  Sierra Leone & EASTERN & 1987 & 324.58 & 269.34 & 385.67 & RW2 \\ 
  Sierra Leone & EASTERN & 1988 & 324.53 & 270.09 & 383.38 & RW2 \\ 
  Sierra Leone & EASTERN & 1989 & 323.99 & 271.14 & 381.95 & RW2 \\ 
  Sierra Leone & EASTERN & 1990 & 323.08 & 273.23 & 377.21 & RW2 \\ 
  Sierra Leone & EASTERN & 1991 & 321.88 & 275.66 & 372.30 & RW2 \\ 
  Sierra Leone & EASTERN & 1992 & 320.35 & 275.99 & 368.07 & RW2 \\ 
  Sierra Leone & EASTERN & 1993 & 318.50 & 274.09 & 366.64 & RW2 \\ 
  Sierra Leone & EASTERN & 1994 & 316.20 & 269.80 & 366.92 & RW2 \\ 
  Sierra Leone & EASTERN & 1995 & 313.06 & 267.49 & 362.76 & RW2 \\ 
  Sierra Leone & EASTERN & 1996 & 309.22 & 266.11 & 356.52 & RW2 \\ 
  Sierra Leone & EASTERN & 1997 & 304.27 & 262.82 & 349.58 & RW2 \\ 
  Sierra Leone & EASTERN & 1998 & 298.35 & 256.41 & 344.16 & RW2 \\ 
  Sierra Leone & EASTERN & 1999 & 291.50 & 248.65 & 339.19 & RW2 \\ 
  Sierra Leone & EASTERN & 2000 & 283.47 & 241.72 & 329.15 & RW2 \\ 
  Sierra Leone & EASTERN & 2001 & 275.26 & 237.39 & 317.36 & RW2 \\ 
  Sierra Leone & EASTERN & 2002 & 266.53 & 229.95 & 306.28 & RW2 \\ 
  Sierra Leone & EASTERN & 2003 & 257.12 & 220.20 & 296.95 & RW2 \\ 
  Sierra Leone & EASTERN & 2004 & 247.38 & 209.71 & 288.63 & RW2 \\ 
  Sierra Leone & EASTERN & 2005 & 237.25 & 201.04 & 277.45 & RW2 \\ 
  Sierra Leone & EASTERN & 2006 & 226.37 & 193.35 & 262.81 & RW2 \\ 
  Sierra Leone & EASTERN & 2007 & 214.86 & 184.47 & 249.72 & RW2 \\ 
  Sierra Leone & EASTERN & 2008 & 202.97 & 172.08 & 237.55 & RW2 \\ 
  Sierra Leone & EASTERN & 2009 & 190.64 & 158.75 & 227.60 & RW2 \\ 
  Sierra Leone & EASTERN & 2010 & 178.29 & 145.37 & 216.14 & RW2 \\ 
  Sierra Leone & EASTERN & 2011 & 166.20 & 136.04 & 201.27 & RW2 \\ 
  Sierra Leone & EASTERN & 2012 & 154.67 & 127.04 & 187.11 & RW2 \\ 
  Sierra Leone & EASTERN & 2013 & 143.66 & 113.73 & 178.83 & RW2 \\ 
  Sierra Leone & EASTERN & 2014 & 133.22 & 96.47 & 180.26 & RW2 \\ 
  Sierra Leone & EASTERN & 2015 & 123.63 & 76.93 & 191.92 & RW2 \\ 
  Sierra Leone & EASTERN & 2016 & 114.47 & 60.31 & 204.85 & RW2 \\ 
  Sierra Leone & EASTERN & 2017 & 105.83 & 45.92 & 223.49 & RW2 \\ 
  Sierra Leone & EASTERN & 2018 & 97.91 & 34.04 & 248.11 & RW2 \\ 
  Sierra Leone & EASTERN & 2019 & 90.31 & 24.84 & 277.99 & RW2 \\ 
  Sierra Leone & NORTHERN & 1980 & 196.73 & 124.89 & 290.97 & RW2 \\ 
  Sierra Leone & NORTHERN & 1981 & 201.04 & 140.05 & 275.72 & RW2 \\ 
  Sierra Leone & NORTHERN & 1982 & 205.24 & 151.13 & 269.63 & RW2 \\ 
  Sierra Leone & NORTHERN & 1983 & 208.85 & 158.93 & 268.49 & RW2 \\ 
  Sierra Leone & NORTHERN & 1984 & 213.05 & 164.93 & 269.54 & RW2 \\ 
  Sierra Leone & NORTHERN & 1985 & 216.83 & 171.69 & 269.95 & RW2 \\ 
  Sierra Leone & NORTHERN & 1986 & 220.27 & 177.68 & 269.75 & RW2 \\ 
  Sierra Leone & NORTHERN & 1987 & 223.45 & 182.36 & 270.72 & RW2 \\ 
  Sierra Leone & NORTHERN & 1988 & 225.90 & 185.48 & 271.89 & RW2 \\ 
  Sierra Leone & NORTHERN & 1989 & 227.99 & 187.87 & 274.04 & RW2 \\ 
  Sierra Leone & NORTHERN & 1990 & 229.29 & 190.41 & 273.10 & RW2 \\ 
  Sierra Leone & NORTHERN & 1991 & 230.29 & 194.08 & 271.43 & RW2 \\ 
  Sierra Leone & NORTHERN & 1992 & 231.01 & 195.85 & 270.60 & RW2 \\ 
  Sierra Leone & NORTHERN & 1993 & 231.12 & 195.00 & 271.97 & RW2 \\ 
  Sierra Leone & NORTHERN & 1994 & 230.91 & 193.56 & 273.20 & RW2 \\ 
  Sierra Leone & NORTHERN & 1995 & 229.99 & 193.74 & 271.37 & RW2 \\ 
  Sierra Leone & NORTHERN & 1996 & 228.53 & 193.88 & 267.29 & RW2 \\ 
  Sierra Leone & NORTHERN & 1997 & 226.26 & 193.03 & 263.76 & RW2 \\ 
  Sierra Leone & NORTHERN & 1998 & 223.37 & 189.68 & 261.32 & RW2 \\ 
  Sierra Leone & NORTHERN & 1999 & 219.51 & 184.47 & 258.93 & RW2 \\ 
  Sierra Leone & NORTHERN & 2000 & 215.42 & 181.54 & 252.82 & RW2 \\ 
  Sierra Leone & NORTHERN & 2001 & 210.87 & 179.53 & 245.57 & RW2 \\ 
  Sierra Leone & NORTHERN & 2002 & 206.20 & 176.59 & 239.23 & RW2 \\ 
  Sierra Leone & NORTHERN & 2003 & 201.10 & 170.92 & 234.21 & RW2 \\ 
  Sierra Leone & NORTHERN & 2004 & 195.83 & 164.91 & 230.61 & RW2 \\ 
  Sierra Leone & NORTHERN & 2005 & 190.18 & 160.74 & 223.82 & RW2 \\ 
  Sierra Leone & NORTHERN & 2006 & 183.96 & 157.01 & 214.11 & RW2 \\ 
  Sierra Leone & NORTHERN & 2007 & 177.19 & 152.34 & 204.84 & RW2 \\ 
  Sierra Leone & NORTHERN & 2008 & 169.77 & 144.82 & 198.45 & RW2 \\ 
  Sierra Leone & NORTHERN & 2009 & 162.07 & 135.40 & 192.77 & RW2 \\ 
  Sierra Leone & NORTHERN & 2010 & 153.91 & 125.70 & 186.43 & RW2 \\ 
  Sierra Leone & NORTHERN & 2011 & 146.00 & 120.40 & 175.98 & RW2 \\ 
  Sierra Leone & NORTHERN & 2012 & 138.22 & 114.64 & 165.45 & RW2 \\ 
  Sierra Leone & NORTHERN & 2013 & 130.68 & 105.20 & 161.02 & RW2 \\ 
  Sierra Leone & NORTHERN & 2014 & 123.46 & 91.10 & 164.91 & RW2 \\ 
  Sierra Leone & NORTHERN & 2015 & 116.74 & 73.78 & 179.13 & RW2 \\ 
  Sierra Leone & NORTHERN & 2016 & 110.18 & 58.99 & 195.45 & RW2 \\ 
  Sierra Leone & NORTHERN & 2017 & 103.86 & 46.16 & 216.26 & RW2 \\ 
  Sierra Leone & NORTHERN & 2018 & 98.03 & 35.30 & 243.70 & RW2 \\ 
  Sierra Leone & NORTHERN & 2019 & 92.48 & 26.14 & 276.52 & RW2 \\ 
  Sierra Leone & SOUTHERN & 1980 & 332.10 & 221.27 & 465.83 & RW2 \\ 
  Sierra Leone & SOUTHERN & 1981 & 330.58 & 237.47 & 439.49 & RW2 \\ 
  Sierra Leone & SOUTHERN & 1982 & 328.38 & 247.67 & 420.04 & RW2 \\ 
  Sierra Leone & SOUTHERN & 1983 & 326.32 & 252.64 & 409.07 & RW2 \\ 
  Sierra Leone & SOUTHERN & 1984 & 324.09 & 254.32 & 401.41 & RW2 \\ 
  Sierra Leone & SOUTHERN & 1985 & 322.09 & 258.95 & 392.12 & RW2 \\ 
  Sierra Leone & SOUTHERN & 1986 & 319.83 & 261.60 & 383.19 & RW2 \\ 
  Sierra Leone & SOUTHERN & 1987 & 317.37 & 263.13 & 376.72 & RW2 \\ 
  Sierra Leone & SOUTHERN & 1988 & 314.82 & 262.24 & 372.32 & RW2 \\ 
  Sierra Leone & SOUTHERN & 1989 & 312.10 & 260.87 & 368.12 & RW2 \\ 
  Sierra Leone & SOUTHERN & 1990 & 309.06 & 260.36 & 361.98 & RW2 \\ 
  Sierra Leone & SOUTHERN & 1991 & 306.26 & 261.11 & 355.05 & RW2 \\ 
  Sierra Leone & SOUTHERN & 1992 & 303.07 & 259.61 & 349.67 & RW2 \\ 
  Sierra Leone & SOUTHERN & 1993 & 299.54 & 256.30 & 346.31 & RW2 \\ 
  Sierra Leone & SOUTHERN & 1994 & 296.14 & 251.52 & 343.99 & RW2 \\ 
  Sierra Leone & SOUTHERN & 1995 & 292.17 & 248.99 & 339.32 & RW2 \\ 
  Sierra Leone & SOUTHERN & 1996 & 287.74 & 247.20 & 331.77 & RW2 \\ 
  Sierra Leone & SOUTHERN & 1997 & 282.54 & 243.94 & 325.17 & RW2 \\ 
  Sierra Leone & SOUTHERN & 1998 & 276.65 & 238.37 & 319.34 & RW2 \\ 
  Sierra Leone & SOUTHERN & 1999 & 269.99 & 229.58 & 313.94 & RW2 \\ 
  Sierra Leone & SOUTHERN & 2000 & 262.22 & 223.07 & 305.31 & RW2 \\ 
  Sierra Leone & SOUTHERN & 2001 & 253.93 & 217.62 & 293.92 & RW2 \\ 
  Sierra Leone & SOUTHERN & 2002 & 244.73 & 210.73 & 282.66 & RW2 \\ 
  Sierra Leone & SOUTHERN & 2003 & 234.72 & 201.13 & 272.67 & RW2 \\ 
  Sierra Leone & SOUTHERN & 2004 & 224.19 & 189.94 & 262.36 & RW2 \\ 
  Sierra Leone & SOUTHERN & 2005 & 213.23 & 181.14 & 249.72 & RW2 \\ 
  Sierra Leone & SOUTHERN & 2006 & 201.80 & 173.42 & 233.87 & RW2 \\ 
  Sierra Leone & SOUTHERN & 2007 & 190.06 & 163.54 & 219.24 & RW2 \\ 
  Sierra Leone & SOUTHERN & 2008 & 178.37 & 151.75 & 208.29 & RW2 \\ 
  Sierra Leone & SOUTHERN & 2009 & 166.64 & 138.95 & 198.57 & RW2 \\ 
  Sierra Leone & SOUTHERN & 2010 & 155.21 & 126.95 & 188.34 & RW2 \\ 
  Sierra Leone & SOUTHERN & 2011 & 144.29 & 118.40 & 174.65 & RW2 \\ 
  Sierra Leone & SOUTHERN & 2012 & 133.96 & 110.28 & 161.68 & RW2 \\ 
  Sierra Leone & SOUTHERN & 2013 & 124.25 & 99.11 & 154.91 & RW2 \\ 
  Sierra Leone & SOUTHERN & 2014 & 115.13 & 84.05 & 155.32 & RW2 \\ 
  Sierra Leone & SOUTHERN & 2015 & 106.62 & 66.84 & 165.37 & RW2 \\ 
  Sierra Leone & SOUTHERN & 2016 & 98.49 & 52.27 & 177.07 & RW2 \\ 
  Sierra Leone & SOUTHERN & 2017 & 91.16 & 39.91 & 193.99 & RW2 \\ 
  Sierra Leone & SOUTHERN & 2018 & 83.99 & 29.58 & 215.81 & RW2 \\ 
  Sierra Leone & SOUTHERN & 2019 & 77.59 & 21.53 & 244.02 & RW2 \\ 
  Sierra Leone & WESTERN & 1980 & 255.10 & 140.66 & 442.41 & RW2 \\ 
  Sierra Leone & WESTERN & 1981 & 249.59 & 149.67 & 405.39 & RW2 \\ 
  Sierra Leone & WESTERN & 1982 & 244.54 & 155.37 & 375.19 & RW2 \\ 
  Sierra Leone & WESTERN & 1983 & 238.82 & 158.47 & 353.71 & RW2 \\ 
  Sierra Leone & WESTERN & 1984 & 233.70 & 159.95 & 333.47 & RW2 \\ 
  Sierra Leone & WESTERN & 1985 & 228.32 & 161.06 & 314.76 & RW2 \\ 
  Sierra Leone & WESTERN & 1986 & 222.47 & 162.26 & 299.75 & RW2 \\ 
  Sierra Leone & WESTERN & 1987 & 217.04 & 162.03 & 285.59 & RW2 \\ 
  Sierra Leone & WESTERN & 1988 & 211.44 & 159.72 & 275.20 & RW2 \\ 
  Sierra Leone & WESTERN & 1989 & 205.78 & 156.23 & 267.10 & RW2 \\ 
  Sierra Leone & WESTERN & 1990 & 200.41 & 152.60 & 258.55 & RW2 \\ 
  Sierra Leone & WESTERN & 1991 & 195.79 & 149.44 & 249.74 & RW2 \\ 
  Sierra Leone & WESTERN & 1992 & 191.08 & 145.73 & 243.54 & RW2 \\ 
  Sierra Leone & WESTERN & 1993 & 187.05 & 141.95 & 239.49 & RW2 \\ 
  Sierra Leone & WESTERN & 1994 & 183.64 & 137.53 & 236.52 & RW2 \\ 
  Sierra Leone & WESTERN & 1995 & 180.34 & 134.68 & 232.59 & RW2 \\ 
  Sierra Leone & WESTERN & 1996 & 177.45 & 133.24 & 226.62 & RW2 \\ 
  Sierra Leone & WESTERN & 1997 & 174.87 & 132.02 & 223.27 & RW2 \\ 
  Sierra Leone & WESTERN & 1998 & 172.79 & 130.75 & 220.20 & RW2 \\ 
  Sierra Leone & WESTERN & 1999 & 170.59 & 129.44 & 217.35 & RW2 \\ 
  Sierra Leone & WESTERN & 2000 & 168.68 & 129.63 & 213.70 & RW2 \\ 
  Sierra Leone & WESTERN & 2001 & 167.24 & 130.43 & 209.10 & RW2 \\ 
  Sierra Leone & WESTERN & 2002 & 165.83 & 131.25 & 205.27 & RW2 \\ 
  Sierra Leone & WESTERN & 2003 & 164.50 & 130.85 & 202.96 & RW2 \\ 
  Sierra Leone & WESTERN & 2004 & 163.25 & 129.80 & 201.29 & RW2 \\ 
  Sierra Leone & WESTERN & 2005 & 161.98 & 129.89 & 199.39 & RW2 \\ 
  Sierra Leone & WESTERN & 2006 & 160.24 & 130.64 & 195.31 & RW2 \\ 
  Sierra Leone & WESTERN & 2007 & 158.40 & 129.90 & 192.24 & RW2 \\ 
  Sierra Leone & WESTERN & 2008 & 156.17 & 126.60 & 191.54 & RW2 \\ 
  Sierra Leone & WESTERN & 2009 & 153.42 & 121.79 & 191.85 & RW2 \\ 
  Sierra Leone & WESTERN & 2010 & 150.43 & 116.40 & 192.34 & RW2 \\ 
  Sierra Leone & WESTERN & 2011 & 147.35 & 113.21 & 190.60 & RW2 \\ 
  Sierra Leone & WESTERN & 2012 & 144.14 & 109.20 & 190.03 & RW2 \\ 
  Sierra Leone & WESTERN & 2013 & 140.91 & 102.71 & 194.03 & RW2 \\ 
  Sierra Leone & WESTERN & 2014 & 137.61 & 92.08 & 204.86 & RW2 \\ 
  Sierra Leone & WESTERN & 2015 & 134.87 & 78.49 & 227.11 & RW2 \\ 
  Sierra Leone & WESTERN & 2016 & 131.76 & 65.21 & 253.03 & RW2 \\ 
  Sierra Leone & WESTERN & 2017 & 129.12 & 53.05 & 287.12 & RW2 \\ 
  Sierra Leone & WESTERN & 2018 & 125.94 & 42.15 & 326.99 & RW2 \\ 
  Sierra Leone & WESTERN & 2019 & 123.54 & 32.97 & 375.29 & RW2 \\ 
  Tanzania & ALL & 1980 & 175.21 & 170.28 & 180.41 & IHME \\ 
  Tanzania & ALL & 1980 & 182.08 & 137.66 & 236.40 & RW2 \\ 
  Tanzania & ALL & 1980 & 181.10 & 170.20 & 193.10 & UN \\ 
  Tanzania & ALL & 1981 & 173.37 & 168.31 & 178.12 & IHME \\ 
  Tanzania & ALL & 1981 & 180.70 & 148.37 & 217.88 & RW2 \\ 
  Tanzania & ALL & 1981 & 179.70 & 168.70 & 191.30 & UN \\ 
  Tanzania & ALL & 1982 & 171.60 & 166.81 & 176.10 & IHME \\ 
  Tanzania & ALL & 1982 & 179.38 & 150.26 & 212.63 & RW2 \\ 
  Tanzania & ALL & 1982 & 178.70 & 168.20 & 190.10 & UN \\ 
  Tanzania & ALL & 1983 & 169.82 & 165.09 & 174.35 & IHME \\ 
  Tanzania & ALL & 1983 & 177.82 & 146.31 & 214.28 & RW2 \\ 
  Tanzania & ALL & 1983 & 178.30 & 167.70 & 189.20 & UN \\ 
  Tanzania & ALL & 1984 & 167.83 & 163.28 & 172.25 & IHME \\ 
  Tanzania & ALL & 1984 & 176.55 & 142.30 & 216.33 & RW2 \\ 
  Tanzania & ALL & 1984 & 177.50 & 167.30 & 188.40 & UN \\ 
  Tanzania & ALL & 1985 & 165.56 & 161.12 & 169.86 & IHME \\ 
  Tanzania & ALL & 1985 & 174.87 & 143.65 & 211.17 & RW2 \\ 
  Tanzania & ALL & 1985 & 176.30 & 166.20 & 187.20 & UN \\ 
  Tanzania & ALL & 1986 & 163.70 & 159.49 & 167.81 & IHME \\ 
  Tanzania & ALL & 1986 & 173.27 & 144.44 & 206.25 & RW2 \\ 
  Tanzania & ALL & 1986 & 174.50 & 164.50 & 185.30 & UN \\ 
  Tanzania & ALL & 1987 & 161.88 & 157.84 & 165.75 & IHME \\ 
  Tanzania & ALL & 1987 & 171.63 & 144.50 & 203.17 & RW2 \\ 
  Tanzania & ALL & 1987 & 172.10 & 162.40 & 182.40 & UN \\ 
  Tanzania & ALL & 1988 & 159.56 & 155.59 & 163.45 & IHME \\ 
  Tanzania & ALL & 1988 & 169.79 & 141.74 & 202.36 & RW2 \\ 
  Tanzania & ALL & 1988 & 169.40 & 159.80 & 179.50 & UN \\ 
  Tanzania & ALL & 1989 & 157.39 & 153.48 & 161.30 & IHME \\ 
  Tanzania & ALL & 1989 & 167.95 & 138.84 & 202.34 & RW2 \\ 
  Tanzania & ALL & 1989 & 167.10 & 157.40 & 177.00 & UN \\ 
  Tanzania & ALL & 1990 & 155.17 & 151.18 & 159.19 & IHME \\ 
  Tanzania & ALL & 1990 & 165.95 & 137.29 & 199.53 & RW2 \\ 
  Tanzania & ALL & 1990 & 165.20 & 155.60 & 175.20 & UN \\ 
  Tanzania & ALL & 1991 & 152.86 & 148.88 & 156.88 & IHME \\ 
  Tanzania & ALL & 1991 & 164.30 & 137.52 & 194.87 & RW2 \\ 
  Tanzania & ALL & 1991 & 163.90 & 154.40 & 173.60 & UN \\ 
  Tanzania & ALL & 1992 & 150.59 & 146.71 & 154.79 & IHME \\ 
  Tanzania & ALL & 1992 & 162.77 & 136.82 & 192.18 & RW2 \\ 
  Tanzania & ALL & 1992 & 162.60 & 153.40 & 172.30 & UN \\ 
  Tanzania & ALL & 1993 & 148.28 & 144.29 & 152.40 & IHME \\ 
  Tanzania & ALL & 1993 & 161.27 & 134.72 & 191.39 & RW2 \\ 
  Tanzania & ALL & 1993 & 161.50 & 152.30 & 171.10 & UN \\ 
  Tanzania & ALL & 1994 & 145.74 & 141.75 & 149.88 & IHME \\ 
  Tanzania & ALL & 1994 & 159.47 & 131.41 & 191.76 & RW2 \\ 
  Tanzania & ALL & 1994 & 159.90 & 150.70 & 169.60 & UN \\ 
  Tanzania & ALL & 1995 & 142.86 & 138.73 & 146.91 & IHME \\ 
  Tanzania & ALL & 1995 & 157.81 & 130.83 & 189.98 & RW2 \\ 
  Tanzania & ALL & 1995 & 157.60 & 148.40 & 167.50 & UN \\ 
  Tanzania & ALL & 1996 & 139.54 & 135.50 & 143.51 & IHME \\ 
  Tanzania & ALL & 1996 & 154.54 & 129.29 & 184.57 & RW2 \\ 
  Tanzania & ALL & 1996 & 154.30 & 145.00 & 164.20 & UN \\ 
  Tanzania & ALL & 1997 & 135.65 & 131.59 & 139.77 & IHME \\ 
  Tanzania & ALL & 1997 & 150.16 & 126.17 & 177.87 & RW2 \\ 
  Tanzania & ALL & 1997 & 149.90 & 140.60 & 160.00 & UN \\ 
  Tanzania & ALL & 1998 & 131.31 & 127.26 & 135.49 & IHME \\ 
  Tanzania & ALL & 1998 & 144.65 & 120.65 & 173.12 & RW2 \\ 
  Tanzania & ALL & 1998 & 144.50 & 135.30 & 154.50 & UN \\ 
  Tanzania & ALL & 1999 & 126.54 & 122.45 & 130.62 & IHME \\ 
  Tanzania & ALL & 1999 & 138.02 & 113.60 & 166.52 & RW2 \\ 
  Tanzania & ALL & 1999 & 137.90 & 129.00 & 147.50 & UN \\ 
  Tanzania & ALL & 2000 & 121.48 & 117.34 & 125.64 & IHME \\ 
  Tanzania & ALL & 2000 & 130.26 & 106.78 & 157.01 & RW2 \\ 
  Tanzania & ALL & 2000 & 130.60 & 122.00 & 139.80 & UN \\ 
  Tanzania & ALL & 2001 & 116.21 & 112.04 & 120.46 & IHME \\ 
  Tanzania & ALL & 2001 & 122.44 & 101.24 & 146.76 & RW2 \\ 
  Tanzania & ALL & 2001 & 122.60 & 114.30 & 131.60 & UN \\ 
  Tanzania & ALL & 2002 & 110.90 & 106.70 & 115.04 & IHME \\ 
  Tanzania & ALL & 2002 & 114.46 & 95.35 & 136.87 & RW2 \\ 
  Tanzania & ALL & 2002 & 114.70 & 106.40 & 123.50 & UN \\ 
  Tanzania & ALL & 2003 & 105.69 & 101.43 & 109.97 & IHME \\ 
  Tanzania & ALL & 2003 & 106.70 & 88.54 & 128.42 & RW2 \\ 
  Tanzania & ALL & 2003 & 106.90 & 98.50 & 115.60 & UN \\ 
  Tanzania & ALL & 2004 & 100.75 & 96.61 & 105.12 & IHME \\ 
  Tanzania & ALL & 2004 & 99.10 & 80.88 & 121.11 & RW2 \\ 
  Tanzania & ALL & 2004 & 99.40 & 90.90 & 108.20 & UN \\ 
  Tanzania & ALL & 2005 & 96.02 & 91.56 & 100.59 & IHME \\ 
  Tanzania & ALL & 2005 & 92.03 & 74.68 & 112.86 & RW2 \\ 
  Tanzania & ALL & 2005 & 92.40 & 83.90 & 101.30 & UN \\ 
  Tanzania & ALL & 2006 & 91.77 & 87.03 & 96.66 & IHME \\ 
  Tanzania & ALL & 2006 & 85.45 & 70.10 & 103.69 & RW2 \\ 
  Tanzania & ALL & 2006 & 85.80 & 77.10 & 95.30 & UN \\ 
  Tanzania & ALL & 2007 & 87.64 & 82.63 & 93.00 & IHME \\ 
  Tanzania & ALL & 2007 & 79.47 & 65.67 & 95.84 & RW2 \\ 
  Tanzania & ALL & 2007 & 79.50 & 70.20 & 89.80 & UN \\ 
  Tanzania & ALL & 2008 & 83.69 & 78.20 & 89.27 & IHME \\ 
  Tanzania & ALL & 2008 & 74.10 & 60.68 & 90.21 & RW2 \\ 
  Tanzania & ALL & 2008 & 73.70 & 63.80 & 85.00 & UN \\ 
  Tanzania & ALL & 2009 & 80.02 & 74.21 & 85.95 & IHME \\ 
  Tanzania & ALL & 2009 & 69.07 & 55.50 & 85.79 & RW2 \\ 
  Tanzania & ALL & 2009 & 68.60 & 57.90 & 80.80 & UN \\ 
  Tanzania & ALL & 2010 & 76.24 & 70.26 & 83.12 & IHME \\ 
  Tanzania & ALL & 2010 & 64.62 & 51.23 & 81.87 & RW2 \\ 
  Tanzania & ALL & 2010 & 63.40 & 52.10 & 76.80 & UN \\ 
  Tanzania & ALL & 2011 & 72.68 & 66.28 & 80.01 & IHME \\ 
  Tanzania & ALL & 2011 & 60.47 & 48.46 & 75.38 & RW2 \\ 
  Tanzania & ALL & 2011 & 58.90 & 46.90 & 73.70 & UN \\ 
  Tanzania & ALL & 2012 & 69.12 & 62.53 & 77.10 & IHME \\ 
  Tanzania & ALL & 2012 & 56.58 & 46.06 & 69.31 & RW2 \\ 
  Tanzania & ALL & 2012 & 55.70 & 43.10 & 72.00 & UN \\ 
  Tanzania & ALL & 2013 & 66.00 & 59.18 & 74.23 & IHME \\ 
  Tanzania & ALL & 2013 & 52.98 & 42.26 & 66.06 & RW2 \\ 
  Tanzania & ALL & 2013 & 53.30 & 39.80 & 71.20 & UN \\ 
  Tanzania & ALL & 2014 & 63.04 & 55.95 & 71.42 & IHME \\ 
  Tanzania & ALL & 2014 & 49.55 & 35.95 & 67.81 & RW2 \\ 
  Tanzania & ALL & 2014 & 50.50 & 36.50 & 70.20 & UN \\ 
  Tanzania & ALL & 2015 & 60.08 & 52.89 & 69.16 & IHME \\ 
  Tanzania & ALL & 2015 & 46.26 & 28.16 & 75.08 & RW2 \\ 
  Tanzania & ALL & 2015 & 48.70 & 33.80 & 70.30 & UN \\ 
  Tanzania & ALL & 2016 & 43.34 & 22.08 & 83.90 & RW2 \\ 
  Tanzania & ALL & 2017 & 40.41 & 16.69 & 95.59 & RW2 \\ 
  Tanzania & ALL & 2018 & 37.75 & 12.49 & 111.57 & RW2 \\ 
  Tanzania & ALL & 2019 & 35.21 & 8.82 & 129.37 & RW2 \\ 
  Tanzania & ALL & 80-84 & 170.57 & 182.00 & 159.72 & HT-Direct \\ 
  Tanzania & ALL & 85-89 & 163.24 & 172.25 & 154.60 & HT-Direct \\ 
  Tanzania & ALL & 90-94 & 162.13 & 169.55 & 154.97 & HT-Direct \\ 
  Tanzania & ALL & 95-99 & 148.63 & 155.48 & 142.03 & HT-Direct \\ 
  Tanzania & ALL & 00-04 & 113.41 & 119.51 & 107.58 & HT-Direct \\ 
  Tanzania & ALL & 05-09 & 87.26 & 93.37 & 81.52 & HT-Direct \\ 
  Tanzania & ALL & 10-14 & 70.14 & 77.35 & 63.55 & HT-Direct \\ 
  Tanzania & ALL & 15-19 & 40.41 & 16.95 & 93.98 & RW2 \\ 
  Tanzania & ARUSHA & 1980 & 111.44 & 76.91 & 158.98 & RW2 \\ 
  Tanzania & ARUSHA & 1981 & 110.69 & 81.93 & 147.83 & RW2 \\ 
  Tanzania & ARUSHA & 1982 & 109.76 & 83.65 & 142.99 & RW2 \\ 
  Tanzania & ARUSHA & 1983 & 108.85 & 83.12 & 141.05 & RW2 \\ 
  Tanzania & ARUSHA & 1984 & 107.89 & 82.35 & 139.87 & RW2 \\ 
  Tanzania & ARUSHA & 1985 & 106.90 & 83.27 & 136.03 & RW2 \\ 
  Tanzania & ARUSHA & 1986 & 105.69 & 84.49 & 131.95 & RW2 \\ 
  Tanzania & ARUSHA & 1987 & 104.34 & 84.33 & 128.67 & RW2 \\ 
  Tanzania & ARUSHA & 1988 & 102.98 & 82.98 & 127.19 & RW2 \\ 
  Tanzania & ARUSHA & 1989 & 101.53 & 81.72 & 125.95 & RW2 \\ 
  Tanzania & ARUSHA & 1990 & 99.87 & 80.77 & 122.98 & RW2 \\ 
  Tanzania & ARUSHA & 1991 & 98.41 & 80.55 & 119.68 & RW2 \\ 
  Tanzania & ARUSHA & 1992 & 97.02 & 80.03 & 117.25 & RW2 \\ 
  Tanzania & ARUSHA & 1993 & 95.79 & 78.54 & 116.04 & RW2 \\ 
  Tanzania & ARUSHA & 1994 & 94.47 & 76.77 & 115.51 & RW2 \\ 
  Tanzania & ARUSHA & 1995 & 93.03 & 75.67 & 113.93 & RW2 \\ 
  Tanzania & ARUSHA & 1996 & 90.93 & 74.91 & 109.94 & RW2 \\ 
  Tanzania & ARUSHA & 1997 & 88.06 & 72.80 & 106.25 & RW2 \\ 
  Tanzania & ARUSHA & 1998 & 84.65 & 69.55 & 102.63 & RW2 \\ 
  Tanzania & ARUSHA & 1999 & 80.61 & 65.50 & 98.90 & RW2 \\ 
  Tanzania & ARUSHA & 2000 & 75.87 & 61.41 & 93.21 & RW2 \\ 
  Tanzania & ARUSHA & 2001 & 71.06 & 58.08 & 86.68 & RW2 \\ 
  Tanzania & ARUSHA & 2002 & 66.07 & 54.13 & 80.41 & RW2 \\ 
  Tanzania & ARUSHA & 2003 & 61.09 & 49.64 & 75.00 & RW2 \\ 
  Tanzania & ARUSHA & 2004 & 56.33 & 45.28 & 70.05 & RW2 \\ 
  Tanzania & ARUSHA & 2005 & 51.69 & 41.11 & 64.91 & RW2 \\ 
  Tanzania & ARUSHA & 2006 & 47.51 & 38.04 & 59.20 & RW2 \\ 
  Tanzania & ARUSHA & 2007 & 43.68 & 34.98 & 54.50 & RW2 \\ 
  Tanzania & ARUSHA & 2008 & 40.24 & 31.65 & 51.02 & RW2 \\ 
  Tanzania & ARUSHA & 2009 & 37.14 & 28.57 & 48.18 & RW2 \\ 
  Tanzania & ARUSHA & 2010 & 34.38 & 25.86 & 45.86 & RW2 \\ 
  Tanzania & ARUSHA & 2011 & 31.86 & 23.75 & 42.66 & RW2 \\ 
  Tanzania & ARUSHA & 2012 & 29.54 & 21.84 & 39.75 & RW2 \\ 
  Tanzania & ARUSHA & 2013 & 27.37 & 19.55 & 38.01 & RW2 \\ 
  Tanzania & ARUSHA & 2014 & 25.36 & 16.72 & 38.12 & RW2 \\ 
  Tanzania & ARUSHA & 2015 & 23.48 & 13.48 & 40.63 & RW2 \\ 
  Tanzania & ARUSHA & 2016 & 21.82 & 10.55 & 43.97 & RW2 \\ 
  Tanzania & ARUSHA & 2017 & 20.13 & 8.18 & 48.92 & RW2 \\ 
  Tanzania & ARUSHA & 2018 & 18.63 & 6.17 & 55.19 & RW2 \\ 
  Tanzania & ARUSHA & 2019 & 17.33 & 4.51 & 63.61 & RW2 \\ 
  Tanzania & DAR ES SALAAM & 1980 & 205.93 & 144.49 & 284.42 & RW2 \\ 
  Tanzania & DAR ES SALAAM & 1981 & 202.81 & 154.73 & 260.87 & RW2 \\ 
  Tanzania & DAR ES SALAAM & 1982 & 199.67 & 157.97 & 249.19 & RW2 \\ 
  Tanzania & DAR ES SALAAM & 1983 & 196.03 & 155.95 & 244.32 & RW2 \\ 
  Tanzania & DAR ES SALAAM & 1984 & 192.42 & 152.52 & 240.73 & RW2 \\ 
  Tanzania & DAR ES SALAAM & 1985 & 187.70 & 150.67 & 232.16 & RW2 \\ 
  Tanzania & DAR ES SALAAM & 1986 & 182.38 & 148.27 & 223.11 & RW2 \\ 
  Tanzania & DAR ES SALAAM & 1987 & 176.01 & 144.02 & 214.67 & RW2 \\ 
  Tanzania & DAR ES SALAAM & 1988 & 168.82 & 137.84 & 206.22 & RW2 \\ 
  Tanzania & DAR ES SALAAM & 1989 & 161.25 & 130.09 & 198.42 & RW2 \\ 
  Tanzania & DAR ES SALAAM & 1990 & 153.62 & 124.08 & 188.76 & RW2 \\ 
  Tanzania & DAR ES SALAAM & 1991 & 146.63 & 119.25 & 178.72 & RW2 \\ 
  Tanzania & DAR ES SALAAM & 1992 & 140.82 & 114.94 & 170.83 & RW2 \\ 
  Tanzania & DAR ES SALAAM & 1993 & 136.12 & 110.33 & 165.90 & RW2 \\ 
  Tanzania & DAR ES SALAAM & 1994 & 132.46 & 105.98 & 163.04 & RW2 \\ 
  Tanzania & DAR ES SALAAM & 1995 & 129.82 & 104.02 & 160.77 & RW2 \\ 
  Tanzania & DAR ES SALAAM & 1996 & 126.99 & 102.29 & 155.82 & RW2 \\ 
  Tanzania & DAR ES SALAAM & 1997 & 124.16 & 100.51 & 152.09 & RW2 \\ 
  Tanzania & DAR ES SALAAM & 1998 & 121.18 & 96.92 & 149.25 & RW2 \\ 
  Tanzania & DAR ES SALAAM & 1999 & 117.55 & 93.06 & 146.16 & RW2 \\ 
  Tanzania & DAR ES SALAAM & 2000 & 113.44 & 89.69 & 142.13 & RW2 \\ 
  Tanzania & DAR ES SALAAM & 2001 & 109.62 & 86.94 & 136.46 & RW2 \\ 
  Tanzania & DAR ES SALAAM & 2002 & 105.42 & 83.86 & 131.37 & RW2 \\ 
  Tanzania & DAR ES SALAAM & 2003 & 101.27 & 80.28 & 127.13 & RW2 \\ 
  Tanzania & DAR ES SALAAM & 2004 & 97.22 & 76.37 & 123.67 & RW2 \\ 
  Tanzania & DAR ES SALAAM & 2005 & 93.36 & 73.07 & 118.56 & RW2 \\ 
  Tanzania & DAR ES SALAAM & 2006 & 89.76 & 70.89 & 113.53 & RW2 \\ 
  Tanzania & DAR ES SALAAM & 2007 & 86.67 & 68.98 & 109.00 & RW2 \\ 
  Tanzania & DAR ES SALAAM & 2008 & 83.77 & 66.45 & 105.46 & RW2 \\ 
  Tanzania & DAR ES SALAAM & 2009 & 81.20 & 63.77 & 103.00 & RW2 \\ 
  Tanzania & DAR ES SALAAM & 2010 & 79.06 & 61.79 & 101.18 & RW2 \\ 
  Tanzania & DAR ES SALAAM & 2011 & 76.90 & 60.77 & 97.24 & RW2 \\ 
  Tanzania & DAR ES SALAAM & 2012 & 74.89 & 59.24 & 94.10 & RW2 \\ 
  Tanzania & DAR ES SALAAM & 2013 & 72.85 & 55.65 & 94.78 & RW2 \\ 
  Tanzania & DAR ES SALAAM & 2014 & 70.90 & 49.32 & 101.09 & RW2 \\ 
  Tanzania & DAR ES SALAAM & 2015 & 68.97 & 40.98 & 114.32 & RW2 \\ 
  Tanzania & DAR ES SALAAM & 2016 & 66.95 & 33.17 & 130.48 & RW2 \\ 
  Tanzania & DAR ES SALAAM & 2017 & 65.19 & 26.52 & 152.30 & RW2 \\ 
  Tanzania & DAR ES SALAAM & 2018 & 63.53 & 20.24 & 180.76 & RW2 \\ 
  Tanzania & DAR ES SALAAM & 2019 & 61.49 & 15.48 & 217.00 & RW2 \\ 
  Tanzania & DODOMA & 1980 & 231.42 & 165.81 & 311.12 & RW2 \\ 
  Tanzania & DODOMA & 1981 & 230.98 & 176.74 & 293.80 & RW2 \\ 
  Tanzania & DODOMA & 1982 & 230.50 & 182.51 & 286.23 & RW2 \\ 
  Tanzania & DODOMA & 1983 & 229.92 & 183.14 & 283.76 & RW2 \\ 
  Tanzania & DODOMA & 1984 & 229.04 & 182.92 & 282.64 & RW2 \\ 
  Tanzania & DODOMA & 1985 & 228.30 & 185.88 & 276.92 & RW2 \\ 
  Tanzania & DODOMA & 1986 & 227.14 & 188.55 & 270.52 & RW2 \\ 
  Tanzania & DODOMA & 1987 & 225.55 & 188.96 & 266.26 & RW2 \\ 
  Tanzania & DODOMA & 1988 & 223.46 & 187.12 & 264.52 & RW2 \\ 
  Tanzania & DODOMA & 1989 & 220.95 & 183.80 & 264.00 & RW2 \\ 
  Tanzania & DODOMA & 1990 & 217.82 & 181.58 & 259.68 & RW2 \\ 
  Tanzania & DODOMA & 1991 & 214.85 & 180.85 & 253.96 & RW2 \\ 
  Tanzania & DODOMA & 1992 & 211.88 & 178.86 & 249.29 & RW2 \\ 
  Tanzania & DODOMA & 1993 & 208.32 & 174.90 & 246.72 & RW2 \\ 
  Tanzania & DODOMA & 1994 & 204.53 & 170.29 & 243.73 & RW2 \\ 
  Tanzania & DODOMA & 1995 & 200.28 & 167.09 & 239.47 & RW2 \\ 
  Tanzania & DODOMA & 1996 & 194.41 & 163.09 & 230.41 & RW2 \\ 
  Tanzania & DODOMA & 1997 & 187.07 & 157.41 & 220.54 & RW2 \\ 
  Tanzania & DODOMA & 1998 & 178.26 & 148.75 & 211.97 & RW2 \\ 
  Tanzania & DODOMA & 1999 & 168.34 & 139.09 & 202.29 & RW2 \\ 
  Tanzania & DODOMA & 2000 & 157.45 & 128.88 & 189.88 & RW2 \\ 
  Tanzania & DODOMA & 2001 & 146.37 & 120.93 & 175.85 & RW2 \\ 
  Tanzania & DODOMA & 2002 & 135.41 & 111.98 & 162.58 & RW2 \\ 
  Tanzania & DODOMA & 2003 & 124.66 & 102.15 & 151.31 & RW2 \\ 
  Tanzania & DODOMA & 2004 & 114.59 & 92.39 & 141.52 & RW2 \\ 
  Tanzania & DODOMA & 2005 & 105.09 & 84.02 & 130.48 & RW2 \\ 
  Tanzania & DODOMA & 2006 & 96.60 & 77.66 & 119.82 & RW2 \\ 
  Tanzania & DODOMA & 2007 & 88.92 & 71.27 & 110.22 & RW2 \\ 
  Tanzania & DODOMA & 2008 & 82.11 & 64.86 & 103.58 & RW2 \\ 
  Tanzania & DODOMA & 2009 & 76.02 & 58.69 & 98.00 & RW2 \\ 
  Tanzania & DODOMA & 2010 & 70.70 & 53.35 & 93.69 & RW2 \\ 
  Tanzania & DODOMA & 2011 & 65.65 & 49.17 & 87.40 & RW2 \\ 
  Tanzania & DODOMA & 2012 & 61.09 & 45.21 & 82.16 & RW2 \\ 
  Tanzania & DODOMA & 2013 & 56.84 & 40.52 & 79.30 & RW2 \\ 
  Tanzania & DODOMA & 2014 & 52.87 & 34.81 & 79.73 & RW2 \\ 
  Tanzania & DODOMA & 2015 & 49.02 & 27.96 & 84.38 & RW2 \\ 
  Tanzania & DODOMA & 2016 & 45.54 & 22.26 & 91.25 & RW2 \\ 
  Tanzania & DODOMA & 2017 & 42.35 & 17.26 & 101.03 & RW2 \\ 
  Tanzania & DODOMA & 2018 & 39.33 & 13.13 & 114.42 & RW2 \\ 
  Tanzania & DODOMA & 2019 & 36.48 & 9.49 & 131.94 & RW2 \\ 
  Tanzania & IRINGA & 1980 & 229.84 & 170.49 & 303.77 & RW2 \\ 
  Tanzania & IRINGA & 1981 & 223.54 & 177.32 & 279.55 & RW2 \\ 
  Tanzania & IRINGA & 1982 & 217.90 & 176.52 & 266.60 & RW2 \\ 
  Tanzania & IRINGA & 1983 & 212.12 & 171.24 & 259.85 & RW2 \\ 
  Tanzania & IRINGA & 1984 & 206.46 & 165.98 & 254.53 & RW2 \\ 
  Tanzania & IRINGA & 1985 & 201.22 & 163.46 & 244.31 & RW2 \\ 
  Tanzania & IRINGA & 1986 & 195.82 & 161.94 & 234.71 & RW2 \\ 
  Tanzania & IRINGA & 1987 & 191.00 & 158.95 & 226.97 & RW2 \\ 
  Tanzania & IRINGA & 1988 & 186.21 & 154.89 & 222.34 & RW2 \\ 
  Tanzania & IRINGA & 1989 & 181.97 & 150.42 & 219.03 & RW2 \\ 
  Tanzania & IRINGA & 1990 & 177.72 & 146.88 & 212.81 & RW2 \\ 
  Tanzania & IRINGA & 1991 & 174.00 & 145.68 & 206.39 & RW2 \\ 
  Tanzania & IRINGA & 1992 & 170.72 & 143.88 & 201.13 & RW2 \\ 
  Tanzania & IRINGA & 1993 & 167.58 & 140.45 & 198.88 & RW2 \\ 
  Tanzania & IRINGA & 1994 & 164.10 & 135.82 & 196.14 & RW2 \\ 
  Tanzania & IRINGA & 1995 & 160.95 & 133.13 & 193.11 & RW2 \\ 
  Tanzania & IRINGA & 1996 & 156.35 & 130.54 & 186.07 & RW2 \\ 
  Tanzania & IRINGA & 1997 & 150.91 & 126.58 & 178.54 & RW2 \\ 
  Tanzania & IRINGA & 1998 & 144.47 & 120.21 & 172.25 & RW2 \\ 
  Tanzania & IRINGA & 1999 & 137.03 & 112.84 & 165.08 & RW2 \\ 
  Tanzania & IRINGA & 2000 & 128.85 & 105.79 & 155.60 & RW2 \\ 
  Tanzania & IRINGA & 2001 & 120.34 & 99.57 & 144.26 & RW2 \\ 
  Tanzania & IRINGA & 2002 & 111.80 & 93.09 & 133.71 & RW2 \\ 
  Tanzania & IRINGA & 2003 & 103.48 & 85.83 & 124.74 & RW2 \\ 
  Tanzania & IRINGA & 2004 & 95.40 & 77.94 & 116.70 & RW2 \\ 
  Tanzania & IRINGA & 2005 & 87.68 & 71.31 & 107.45 & RW2 \\ 
  Tanzania & IRINGA & 2006 & 80.70 & 65.99 & 98.12 & RW2 \\ 
  Tanzania & IRINGA & 2007 & 74.21 & 60.92 & 90.28 & RW2 \\ 
  Tanzania & IRINGA & 2008 & 68.42 & 55.38 & 84.39 & RW2 \\ 
  Tanzania & IRINGA & 2009 & 63.11 & 49.81 & 79.69 & RW2 \\ 
  Tanzania & IRINGA & 2010 & 58.33 & 45.10 & 75.49 & RW2 \\ 
  Tanzania & IRINGA & 2011 & 53.84 & 41.32 & 70.15 & RW2 \\ 
  Tanzania & IRINGA & 2012 & 49.78 & 37.80 & 65.17 & RW2 \\ 
  Tanzania & IRINGA & 2013 & 46.02 & 33.71 & 62.43 & RW2 \\ 
  Tanzania & IRINGA & 2014 & 42.49 & 28.66 & 62.84 & RW2 \\ 
  Tanzania & IRINGA & 2015 & 39.26 & 22.80 & 66.52 & RW2 \\ 
  Tanzania & IRINGA & 2016 & 36.19 & 17.95 & 71.77 & RW2 \\ 
  Tanzania & IRINGA & 2017 & 33.47 & 13.64 & 79.33 & RW2 \\ 
  Tanzania & IRINGA & 2018 & 30.80 & 10.25 & 89.55 & RW2 \\ 
  Tanzania & IRINGA & 2019 & 28.43 & 7.46 & 102.96 & RW2 \\ 
  Tanzania & KAGERA & 1980 & 212.30 & 151.83 & 288.63 & RW2 \\ 
  Tanzania & KAGERA & 1981 & 209.75 & 161.59 & 268.51 & RW2 \\ 
  Tanzania & KAGERA & 1982 & 207.25 & 164.22 & 258.57 & RW2 \\ 
  Tanzania & KAGERA & 1983 & 204.56 & 162.65 & 254.07 & RW2 \\ 
  Tanzania & KAGERA & 1984 & 202.04 & 161.01 & 250.58 & RW2 \\ 
  Tanzania & KAGERA & 1985 & 199.48 & 161.43 & 243.55 & RW2 \\ 
  Tanzania & KAGERA & 1986 & 196.99 & 162.67 & 236.61 & RW2 \\ 
  Tanzania & KAGERA & 1987 & 194.22 & 162.24 & 230.68 & RW2 \\ 
  Tanzania & KAGERA & 1988 & 191.72 & 159.77 & 228.48 & RW2 \\ 
  Tanzania & KAGERA & 1989 & 189.18 & 156.50 & 226.82 & RW2 \\ 
  Tanzania & KAGERA & 1990 & 186.56 & 154.94 & 223.20 & RW2 \\ 
  Tanzania & KAGERA & 1991 & 184.54 & 155.13 & 218.24 & RW2 \\ 
  Tanzania & KAGERA & 1992 & 182.72 & 154.40 & 214.90 & RW2 \\ 
  Tanzania & KAGERA & 1993 & 180.97 & 152.24 & 213.51 & RW2 \\ 
  Tanzania & KAGERA & 1994 & 179.25 & 149.26 & 212.71 & RW2 \\ 
  Tanzania & KAGERA & 1995 & 177.53 & 148.62 & 211.37 & RW2 \\ 
  Tanzania & KAGERA & 1996 & 174.34 & 147.02 & 205.32 & RW2 \\ 
  Tanzania & KAGERA & 1997 & 170.31 & 144.24 & 200.01 & RW2 \\ 
  Tanzania & KAGERA & 1998 & 164.83 & 138.45 & 194.89 & RW2 \\ 
  Tanzania & KAGERA & 1999 & 158.01 & 130.97 & 189.18 & RW2 \\ 
  Tanzania & KAGERA & 2000 & 150.03 & 123.62 & 180.34 & RW2 \\ 
  Tanzania & KAGERA & 2001 & 141.41 & 117.36 & 169.44 & RW2 \\ 
  Tanzania & KAGERA & 2002 & 132.20 & 109.84 & 158.44 & RW2 \\ 
  Tanzania & KAGERA & 2003 & 122.84 & 101.10 & 148.88 & RW2 \\ 
  Tanzania & KAGERA & 2004 & 113.51 & 92.00 & 139.61 & RW2 \\ 
  Tanzania & KAGERA & 2005 & 104.40 & 83.99 & 129.05 & RW2 \\ 
  Tanzania & KAGERA & 2006 & 95.96 & 77.36 & 118.70 & RW2 \\ 
  Tanzania & KAGERA & 2007 & 88.31 & 70.71 & 109.47 & RW2 \\ 
  Tanzania & KAGERA & 2008 & 81.30 & 64.18 & 102.50 & RW2 \\ 
  Tanzania & KAGERA & 2009 & 75.03 & 57.56 & 97.22 & RW2 \\ 
  Tanzania & KAGERA & 2010 & 69.44 & 51.74 & 92.28 & RW2 \\ 
  Tanzania & KAGERA & 2011 & 64.28 & 47.42 & 86.25 & RW2 \\ 
  Tanzania & KAGERA & 2012 & 59.43 & 43.23 & 80.98 & RW2 \\ 
  Tanzania & KAGERA & 2013 & 55.02 & 38.44 & 77.74 & RW2 \\ 
  Tanzania & KAGERA & 2014 & 50.82 & 32.68 & 77.69 & RW2 \\ 
  Tanzania & KAGERA & 2015 & 46.93 & 26.30 & 82.45 & RW2 \\ 
  Tanzania & KAGERA & 2016 & 43.50 & 20.71 & 88.25 & RW2 \\ 
  Tanzania & KAGERA & 2017 & 40.21 & 15.77 & 96.72 & RW2 \\ 
  Tanzania & KAGERA & 2018 & 36.98 & 11.89 & 109.98 & RW2 \\ 
  Tanzania & KAGERA & 2019 & 34.17 & 8.60 & 124.54 & RW2 \\ 
  Tanzania & KIGOMA & 1980 & 213.07 & 147.32 & 296.69 & RW2 \\ 
  Tanzania & KIGOMA & 1981 & 210.26 & 155.77 & 276.48 & RW2 \\ 
  Tanzania & KIGOMA & 1982 & 207.12 & 159.07 & 265.02 & RW2 \\ 
  Tanzania & KIGOMA & 1983 & 204.21 & 158.46 & 258.46 & RW2 \\ 
  Tanzania & KIGOMA & 1984 & 201.13 & 157.51 & 253.75 & RW2 \\ 
  Tanzania & KIGOMA & 1985 & 198.65 & 158.63 & 245.11 & RW2 \\ 
  Tanzania & KIGOMA & 1986 & 195.55 & 159.74 & 236.99 & RW2 \\ 
  Tanzania & KIGOMA & 1987 & 192.71 & 159.60 & 230.51 & RW2 \\ 
  Tanzania & KIGOMA & 1988 & 189.94 & 157.45 & 227.57 & RW2 \\ 
  Tanzania & KIGOMA & 1989 & 187.09 & 154.33 & 225.66 & RW2 \\ 
  Tanzania & KIGOMA & 1990 & 183.94 & 152.18 & 220.56 & RW2 \\ 
  Tanzania & KIGOMA & 1991 & 181.31 & 151.52 & 215.71 & RW2 \\ 
  Tanzania & KIGOMA & 1992 & 178.62 & 150.03 & 210.88 & RW2 \\ 
  Tanzania & KIGOMA & 1993 & 175.91 & 147.05 & 209.07 & RW2 \\ 
  Tanzania & KIGOMA & 1994 & 172.67 & 142.57 & 207.37 & RW2 \\ 
  Tanzania & KIGOMA & 1995 & 169.73 & 140.31 & 204.55 & RW2 \\ 
  Tanzania & KIGOMA & 1996 & 165.08 & 137.28 & 197.35 & RW2 \\ 
  Tanzania & KIGOMA & 1997 & 159.45 & 133.19 & 190.06 & RW2 \\ 
  Tanzania & KIGOMA & 1998 & 152.68 & 126.86 & 183.37 & RW2 \\ 
  Tanzania & KIGOMA & 1999 & 144.81 & 119.02 & 175.65 & RW2 \\ 
  Tanzania & KIGOMA & 2000 & 136.01 & 111.11 & 164.81 & RW2 \\ 
  Tanzania & KIGOMA & 2001 & 126.89 & 104.65 & 153.10 & RW2 \\ 
  Tanzania & KIGOMA & 2002 & 117.60 & 97.11 & 141.85 & RW2 \\ 
  Tanzania & KIGOMA & 2003 & 108.37 & 88.77 & 131.99 & RW2 \\ 
  Tanzania & KIGOMA & 2004 & 99.36 & 80.12 & 122.98 & RW2 \\ 
  Tanzania & KIGOMA & 2005 & 90.85 & 72.53 & 113.23 & RW2 \\ 
  Tanzania & KIGOMA & 2006 & 83.19 & 66.34 & 103.60 & RW2 \\ 
  Tanzania & KIGOMA & 2007 & 76.07 & 60.33 & 95.44 & RW2 \\ 
  Tanzania & KIGOMA & 2008 & 69.88 & 54.33 & 89.25 & RW2 \\ 
  Tanzania & KIGOMA & 2009 & 64.28 & 48.64 & 84.39 & RW2 \\ 
  Tanzania & KIGOMA & 2010 & 59.30 & 43.64 & 80.39 & RW2 \\ 
  Tanzania & KIGOMA & 2011 & 54.79 & 39.56 & 75.13 & RW2 \\ 
  Tanzania & KIGOMA & 2012 & 50.62 & 35.88 & 70.33 & RW2 \\ 
  Tanzania & KIGOMA & 2013 & 46.73 & 31.77 & 67.48 & RW2 \\ 
  Tanzania & KIGOMA & 2014 & 43.23 & 27.05 & 67.76 & RW2 \\ 
  Tanzania & KIGOMA & 2015 & 39.81 & 21.69 & 71.36 & RW2 \\ 
  Tanzania & KIGOMA & 2016 & 36.71 & 17.20 & 76.55 & RW2 \\ 
  Tanzania & KIGOMA & 2017 & 33.85 & 13.14 & 85.36 & RW2 \\ 
  Tanzania & KIGOMA & 2018 & 31.19 & 9.84 & 95.08 & RW2 \\ 
  Tanzania & KIGOMA & 2019 & 28.82 & 7.18 & 107.87 & RW2 \\ 
  Tanzania & KILIMANJARO & 1980 & 115.58 & 77.37 & 171.57 & RW2 \\ 
  Tanzania & KILIMANJARO & 1981 & 111.42 & 80.72 & 153.19 & RW2 \\ 
  Tanzania & KILIMANJARO & 1982 & 107.16 & 80.35 & 142.31 & RW2 \\ 
  Tanzania & KILIMANJARO & 1983 & 103.14 & 77.93 & 135.59 & RW2 \\ 
  Tanzania & KILIMANJARO & 1984 & 99.02 & 75.30 & 129.38 & RW2 \\ 
  Tanzania & KILIMANJARO & 1985 & 95.19 & 73.93 & 121.99 & RW2 \\ 
  Tanzania & KILIMANJARO & 1986 & 91.55 & 72.31 & 115.33 & RW2 \\ 
  Tanzania & KILIMANJARO & 1987 & 87.96 & 70.25 & 109.65 & RW2 \\ 
  Tanzania & KILIMANJARO & 1988 & 84.46 & 67.15 & 105.68 & RW2 \\ 
  Tanzania & KILIMANJARO & 1989 & 81.34 & 64.06 & 102.77 & RW2 \\ 
  Tanzania & KILIMANJARO & 1990 & 78.38 & 61.74 & 98.94 & RW2 \\ 
  Tanzania & KILIMANJARO & 1991 & 75.96 & 60.06 & 95.30 & RW2 \\ 
  Tanzania & KILIMANJARO & 1992 & 73.94 & 58.43 & 92.10 & RW2 \\ 
  Tanzania & KILIMANJARO & 1993 & 72.21 & 56.62 & 90.88 & RW2 \\ 
  Tanzania & KILIMANJARO & 1994 & 70.85 & 54.98 & 90.07 & RW2 \\ 
  Tanzania & KILIMANJARO & 1995 & 69.83 & 53.89 & 89.26 & RW2 \\ 
  Tanzania & KILIMANJARO & 1996 & 68.63 & 53.34 & 87.22 & RW2 \\ 
  Tanzania & KILIMANJARO & 1997 & 67.17 & 52.23 & 84.96 & RW2 \\ 
  Tanzania & KILIMANJARO & 1998 & 65.55 & 50.85 & 83.62 & RW2 \\ 
  Tanzania & KILIMANJARO & 1999 & 63.57 & 48.88 & 81.97 & RW2 \\ 
  Tanzania & KILIMANJARO & 2000 & 61.28 & 47.00 & 79.19 & RW2 \\ 
  Tanzania & KILIMANJARO & 2001 & 58.80 & 45.23 & 75.98 & RW2 \\ 
  Tanzania & KILIMANJARO & 2002 & 56.18 & 43.18 & 72.71 & RW2 \\ 
  Tanzania & KILIMANJARO & 2003 & 53.48 & 40.53 & 70.57 & RW2 \\ 
  Tanzania & KILIMANJARO & 2004 & 50.72 & 37.81 & 68.28 & RW2 \\ 
  Tanzania & KILIMANJARO & 2005 & 48.01 & 35.19 & 65.65 & RW2 \\ 
  Tanzania & KILIMANJARO & 2006 & 45.53 & 32.91 & 62.97 & RW2 \\ 
  Tanzania & KILIMANJARO & 2007 & 43.25 & 30.82 & 60.88 & RW2 \\ 
  Tanzania & KILIMANJARO & 2008 & 41.23 & 28.50 & 59.50 & RW2 \\ 
  Tanzania & KILIMANJARO & 2009 & 39.37 & 26.33 & 58.83 & RW2 \\ 
  Tanzania & KILIMANJARO & 2010 & 37.84 & 24.33 & 59.08 & RW2 \\ 
  Tanzania & KILIMANJARO & 2011 & 36.26 & 22.62 & 58.34 & RW2 \\ 
  Tanzania & KILIMANJARO & 2012 & 34.77 & 21.11 & 57.93 & RW2 \\ 
  Tanzania & KILIMANJARO & 2013 & 33.43 & 19.38 & 58.75 & RW2 \\ 
  Tanzania & KILIMANJARO & 2014 & 32.23 & 17.07 & 61.89 & RW2 \\ 
  Tanzania & KILIMANJARO & 2015 & 30.94 & 14.40 & 67.17 & RW2 \\ 
  Tanzania & KILIMANJARO & 2016 & 29.75 & 11.84 & 75.12 & RW2 \\ 
  Tanzania & KILIMANJARO & 2017 & 28.55 & 9.55 & 85.11 & RW2 \\ 
  Tanzania & KILIMANJARO & 2018 & 27.52 & 7.50 & 99.29 & RW2 \\ 
  Tanzania & KILIMANJARO & 2019 & 26.55 & 5.74 & 116.72 & RW2 \\ 
  Tanzania & LINDI & 1980 & 243.30 & 180.36 & 318.98 & RW2 \\ 
  Tanzania & LINDI & 1981 & 245.34 & 194.90 & 303.62 & RW2 \\ 
  Tanzania & LINDI & 1982 & 247.49 & 201.70 & 299.74 & RW2 \\ 
  Tanzania & LINDI & 1983 & 249.68 & 203.23 & 301.97 & RW2 \\ 
  Tanzania & LINDI & 1984 & 251.49 & 204.80 & 304.66 & RW2 \\ 
  Tanzania & LINDI & 1985 & 253.37 & 209.43 & 303.24 & RW2 \\ 
  Tanzania & LINDI & 1986 & 254.64 & 213.24 & 300.17 & RW2 \\ 
  Tanzania & LINDI & 1987 & 255.57 & 216.25 & 298.95 & RW2 \\ 
  Tanzania & LINDI & 1988 & 255.73 & 216.08 & 300.34 & RW2 \\ 
  Tanzania & LINDI & 1989 & 255.75 & 214.52 & 302.29 & RW2 \\ 
  Tanzania & LINDI & 1990 & 254.66 & 214.18 & 299.73 & RW2 \\ 
  Tanzania & LINDI & 1991 & 253.41 & 215.48 & 295.37 & RW2 \\ 
  Tanzania & LINDI & 1992 & 251.71 & 215.32 & 292.36 & RW2 \\ 
  Tanzania & LINDI & 1993 & 249.70 & 212.61 & 290.09 & RW2 \\ 
  Tanzania & LINDI & 1994 & 246.44 & 208.37 & 289.04 & RW2 \\ 
  Tanzania & LINDI & 1995 & 242.18 & 204.78 & 285.07 & RW2 \\ 
  Tanzania & LINDI & 1996 & 235.56 & 200.78 & 274.78 & RW2 \\ 
  Tanzania & LINDI & 1997 & 226.39 & 192.94 & 263.95 & RW2 \\ 
  Tanzania & LINDI & 1998 & 214.94 & 181.95 & 252.11 & RW2 \\ 
  Tanzania & LINDI & 1999 & 201.54 & 168.29 & 239.99 & RW2 \\ 
  Tanzania & LINDI & 2000 & 186.41 & 154.13 & 223.39 & RW2 \\ 
  Tanzania & LINDI & 2001 & 170.84 & 141.83 & 204.32 & RW2 \\ 
  Tanzania & LINDI & 2002 & 155.18 & 128.61 & 186.58 & RW2 \\ 
  Tanzania & LINDI & 2003 & 140.14 & 114.68 & 170.38 & RW2 \\ 
  Tanzania & LINDI & 2004 & 125.79 & 101.28 & 155.66 & RW2 \\ 
  Tanzania & LINDI & 2005 & 112.20 & 89.26 & 139.89 & RW2 \\ 
  Tanzania & LINDI & 2006 & 100.29 & 79.53 & 125.69 & RW2 \\ 
  Tanzania & LINDI & 2007 & 89.68 & 70.49 & 113.15 & RW2 \\ 
  Tanzania & LINDI & 2008 & 80.17 & 61.82 & 103.38 & RW2 \\ 
  Tanzania & LINDI & 2009 & 71.91 & 53.79 & 95.53 & RW2 \\ 
  Tanzania & LINDI & 2010 & 64.65 & 46.79 & 88.18 & RW2 \\ 
  Tanzania & LINDI & 2011 & 58.04 & 41.21 & 80.71 & RW2 \\ 
  Tanzania & LINDI & 2012 & 52.20 & 36.13 & 74.16 & RW2 \\ 
  Tanzania & LINDI & 2013 & 46.83 & 30.85 & 69.22 & RW2 \\ 
  Tanzania & LINDI & 2014 & 42.03 & 25.40 & 67.34 & RW2 \\ 
  Tanzania & LINDI & 2015 & 37.72 & 19.78 & 69.12 & RW2 \\ 
  Tanzania & LINDI & 2016 & 33.82 & 15.20 & 72.52 & RW2 \\ 
  Tanzania & LINDI & 2017 & 30.16 & 11.24 & 77.86 & RW2 \\ 
  Tanzania & LINDI & 2018 & 27.05 & 8.24 & 84.93 & RW2 \\ 
  Tanzania & LINDI & 2019 & 24.20 & 5.90 & 94.21 & RW2 \\ 
  Tanzania & MARA & 1980 & 209.17 & 149.83 & 284.47 & RW2 \\ 
  Tanzania & MARA & 1981 & 207.98 & 159.53 & 266.69 & RW2 \\ 
  Tanzania & MARA & 1982 & 206.91 & 163.52 & 258.58 & RW2 \\ 
  Tanzania & MARA & 1983 & 206.13 & 163.93 & 256.04 & RW2 \\ 
  Tanzania & MARA & 1984 & 204.83 & 163.50 & 253.74 & RW2 \\ 
  Tanzania & MARA & 1985 & 203.72 & 165.77 & 248.36 & RW2 \\ 
  Tanzania & MARA & 1986 & 202.45 & 167.79 & 242.22 & RW2 \\ 
  Tanzania & MARA & 1987 & 200.91 & 168.89 & 237.43 & RW2 \\ 
  Tanzania & MARA & 1988 & 199.47 & 167.15 & 235.70 & RW2 \\ 
  Tanzania & MARA & 1989 & 197.92 & 165.40 & 235.53 & RW2 \\ 
  Tanzania & MARA & 1990 & 196.22 & 164.17 & 232.38 & RW2 \\ 
  Tanzania & MARA & 1991 & 195.14 & 165.82 & 228.23 & RW2 \\ 
  Tanzania & MARA & 1992 & 194.43 & 166.05 & 226.18 & RW2 \\ 
  Tanzania & MARA & 1993 & 194.07 & 164.97 & 226.03 & RW2 \\ 
  Tanzania & MARA & 1994 & 193.66 & 162.93 & 228.15 & RW2 \\ 
  Tanzania & MARA & 1995 & 193.76 & 162.88 & 228.76 & RW2 \\ 
  Tanzania & MARA & 1996 & 192.26 & 163.45 & 224.92 & RW2 \\ 
  Tanzania & MARA & 1997 & 189.85 & 162.10 & 221.04 & RW2 \\ 
  Tanzania & MARA & 1998 & 186.03 & 157.91 & 217.87 & RW2 \\ 
  Tanzania & MARA & 1999 & 180.49 & 151.44 & 213.91 & RW2 \\ 
  Tanzania & MARA & 2000 & 173.54 & 144.87 & 206.24 & RW2 \\ 
  Tanzania & MARA & 2001 & 165.52 & 139.35 & 195.67 & RW2 \\ 
  Tanzania & MARA & 2002 & 156.57 & 132.37 & 184.75 & RW2 \\ 
  Tanzania & MARA & 2003 & 146.87 & 122.99 & 174.95 & RW2 \\ 
  Tanzania & MARA & 2004 & 136.92 & 113.14 & 165.29 & RW2 \\ 
  Tanzania & MARA & 2005 & 126.92 & 104.12 & 153.80 & RW2 \\ 
  Tanzania & MARA & 2006 & 117.27 & 96.76 & 141.65 & RW2 \\ 
  Tanzania & MARA & 2007 & 108.27 & 89.44 & 130.38 & RW2 \\ 
  Tanzania & MARA & 2008 & 99.91 & 81.56 & 122.24 & RW2 \\ 
  Tanzania & MARA & 2009 & 92.24 & 73.29 & 115.42 & RW2 \\ 
  Tanzania & MARA & 2010 & 85.45 & 66.23 & 109.36 & RW2 \\ 
  Tanzania & MARA & 2011 & 78.91 & 60.86 & 101.66 & RW2 \\ 
  Tanzania & MARA & 2012 & 72.96 & 55.76 & 94.59 & RW2 \\ 
  Tanzania & MARA & 2013 & 67.41 & 49.46 & 90.48 & RW2 \\ 
  Tanzania & MARA & 2014 & 62.32 & 42.12 & 90.32 & RW2 \\ 
  Tanzania & MARA & 2015 & 57.44 & 33.53 & 95.65 & RW2 \\ 
  Tanzania & MARA & 2016 & 52.89 & 26.19 & 103.86 & RW2 \\ 
  Tanzania & MARA & 2017 & 48.75 & 19.83 & 113.23 & RW2 \\ 
  Tanzania & MARA & 2018 & 44.98 & 14.86 & 128.78 & RW2 \\ 
  Tanzania & MARA & 2019 & 41.49 & 11.05 & 144.55 & RW2 \\ 
  Tanzania & MBEYA & 1980 & 151.95 & 106.39 & 212.71 & RW2 \\ 
  Tanzania & MBEYA & 1981 & 151.40 & 114.08 & 198.48 & RW2 \\ 
  Tanzania & MBEYA & 1982 & 150.90 & 117.15 & 192.15 & RW2 \\ 
  Tanzania & MBEYA & 1983 & 150.40 & 116.98 & 191.34 & RW2 \\ 
  Tanzania & MBEYA & 1984 & 150.12 & 116.87 & 190.67 & RW2 \\ 
  Tanzania & MBEYA & 1985 & 149.74 & 119.07 & 186.76 & RW2 \\ 
  Tanzania & MBEYA & 1986 & 149.57 & 120.83 & 183.69 & RW2 \\ 
  Tanzania & MBEYA & 1987 & 149.47 & 122.12 & 181.66 & RW2 \\ 
  Tanzania & MBEYA & 1988 & 149.54 & 122.09 & 182.09 & RW2 \\ 
  Tanzania & MBEYA & 1989 & 149.90 & 121.64 & 183.30 & RW2 \\ 
  Tanzania & MBEYA & 1990 & 149.90 & 122.07 & 182.87 & RW2 \\ 
  Tanzania & MBEYA & 1991 & 150.53 & 124.36 & 181.27 & RW2 \\ 
  Tanzania & MBEYA & 1992 & 150.96 & 125.42 & 180.87 & RW2 \\ 
  Tanzania & MBEYA & 1993 & 151.12 & 124.88 & 181.52 & RW2 \\ 
  Tanzania & MBEYA & 1994 & 151.10 & 123.86 & 183.26 & RW2 \\ 
  Tanzania & MBEYA & 1995 & 150.93 & 123.90 & 183.29 & RW2 \\ 
  Tanzania & MBEYA & 1996 & 149.14 & 123.63 & 179.09 & RW2 \\ 
  Tanzania & MBEYA & 1997 & 146.17 & 121.63 & 175.05 & RW2 \\ 
  Tanzania & MBEYA & 1998 & 142.04 & 117.28 & 170.93 & RW2 \\ 
  Tanzania & MBEYA & 1999 & 136.85 & 111.84 & 166.32 & RW2 \\ 
  Tanzania & MBEYA & 2000 & 130.34 & 106.27 & 158.76 & RW2 \\ 
  Tanzania & MBEYA & 2001 & 123.66 & 101.26 & 149.95 & RW2 \\ 
  Tanzania & MBEYA & 2002 & 116.83 & 96.05 & 141.03 & RW2 \\ 
  Tanzania & MBEYA & 2003 & 109.78 & 89.60 & 134.06 & RW2 \\ 
  Tanzania & MBEYA & 2004 & 103.21 & 83.12 & 127.61 & RW2 \\ 
  Tanzania & MBEYA & 2005 & 96.76 & 77.18 & 120.59 & RW2 \\ 
  Tanzania & MBEYA & 2006 & 91.10 & 72.83 & 112.88 & RW2 \\ 
  Tanzania & MBEYA & 2007 & 85.99 & 68.50 & 107.05 & RW2 \\ 
  Tanzania & MBEYA & 2008 & 81.48 & 64.08 & 103.18 & RW2 \\ 
  Tanzania & MBEYA & 2009 & 77.40 & 59.52 & 99.88 & RW2 \\ 
  Tanzania & MBEYA & 2010 & 74.00 & 55.58 & 97.80 & RW2 \\ 
  Tanzania & MBEYA & 2011 & 70.71 & 52.86 & 93.86 & RW2 \\ 
  Tanzania & MBEYA & 2012 & 67.65 & 50.11 & 90.73 & RW2 \\ 
  Tanzania & MBEYA & 2013 & 64.75 & 46.30 & 90.00 & RW2 \\ 
  Tanzania & MBEYA & 2014 & 61.98 & 41.04 & 92.95 & RW2 \\ 
  Tanzania & MBEYA & 2015 & 59.12 & 34.11 & 101.21 & RW2 \\ 
  Tanzania & MBEYA & 2016 & 56.65 & 27.81 & 112.21 & RW2 \\ 
  Tanzania & MBEYA & 2017 & 54.08 & 21.95 & 126.56 & RW2 \\ 
  Tanzania & MBEYA & 2018 & 51.98 & 17.22 & 147.10 & RW2 \\ 
  Tanzania & MBEYA & 2019 & 49.66 & 13.01 & 171.99 & RW2 \\ 
  Tanzania & MOROGORO & 1980 & 222.66 & 163.60 & 296.05 & RW2 \\ 
  Tanzania & MOROGORO & 1981 & 221.14 & 173.60 & 278.12 & RW2 \\ 
  Tanzania & MOROGORO & 1982 & 219.91 & 177.06 & 269.77 & RW2 \\ 
  Tanzania & MOROGORO & 1983 & 218.35 & 175.45 & 267.63 & RW2 \\ 
  Tanzania & MOROGORO & 1984 & 216.58 & 174.42 & 265.26 & RW2 \\ 
  Tanzania & MOROGORO & 1985 & 214.95 & 176.45 & 259.48 & RW2 \\ 
  Tanzania & MOROGORO & 1986 & 213.02 & 177.27 & 253.26 & RW2 \\ 
  Tanzania & MOROGORO & 1987 & 210.91 & 177.94 & 248.06 & RW2 \\ 
  Tanzania & MOROGORO & 1988 & 208.58 & 175.53 & 245.89 & RW2 \\ 
  Tanzania & MOROGORO & 1989 & 206.10 & 172.08 & 244.93 & RW2 \\ 
  Tanzania & MOROGORO & 1990 & 203.46 & 170.47 & 241.13 & RW2 \\ 
  Tanzania & MOROGORO & 1991 & 201.16 & 170.15 & 235.69 & RW2 \\ 
  Tanzania & MOROGORO & 1992 & 199.03 & 169.83 & 231.65 & RW2 \\ 
  Tanzania & MOROGORO & 1993 & 196.96 & 166.93 & 230.71 & RW2 \\ 
  Tanzania & MOROGORO & 1994 & 194.50 & 162.90 & 229.72 & RW2 \\ 
  Tanzania & MOROGORO & 1995 & 191.88 & 161.24 & 227.25 & RW2 \\ 
  Tanzania & MOROGORO & 1996 & 187.36 & 158.82 & 220.24 & RW2 \\ 
  Tanzania & MOROGORO & 1997 & 181.36 & 153.86 & 212.41 & RW2 \\ 
  Tanzania & MOROGORO & 1998 & 173.73 & 145.95 & 205.45 & RW2 \\ 
  Tanzania & MOROGORO & 1999 & 164.34 & 136.42 & 197.35 & RW2 \\ 
  Tanzania & MOROGORO & 2000 & 153.68 & 126.29 & 185.29 & RW2 \\ 
  Tanzania & MOROGORO & 2001 & 142.64 & 117.70 & 171.65 & RW2 \\ 
  Tanzania & MOROGORO & 2002 & 131.36 & 108.20 & 158.67 & RW2 \\ 
  Tanzania & MOROGORO & 2003 & 120.23 & 97.82 & 146.93 & RW2 \\ 
  Tanzania & MOROGORO & 2004 & 109.54 & 87.71 & 135.97 & RW2 \\ 
  Tanzania & MOROGORO & 2005 & 99.36 & 78.50 & 124.60 & RW2 \\ 
  Tanzania & MOROGORO & 2006 & 90.33 & 71.45 & 113.80 & RW2 \\ 
  Tanzania & MOROGORO & 2007 & 82.00 & 64.46 & 103.77 & RW2 \\ 
  Tanzania & MOROGORO & 2008 & 74.64 & 57.60 & 96.15 & RW2 \\ 
  Tanzania & MOROGORO & 2009 & 68.08 & 51.14 & 89.89 & RW2 \\ 
  Tanzania & MOROGORO & 2010 & 62.34 & 45.65 & 84.29 & RW2 \\ 
  Tanzania & MOROGORO & 2011 & 56.91 & 41.21 & 77.87 & RW2 \\ 
  Tanzania & MOROGORO & 2012 & 52.12 & 37.25 & 72.33 & RW2 \\ 
  Tanzania & MOROGORO & 2013 & 47.63 & 32.61 & 68.18 & RW2 \\ 
  Tanzania & MOROGORO & 2014 & 43.49 & 27.65 & 67.35 & RW2 \\ 
  Tanzania & MOROGORO & 2015 & 39.71 & 21.90 & 69.86 & RW2 \\ 
  Tanzania & MOROGORO & 2016 & 36.26 & 17.19 & 74.38 & RW2 \\ 
  Tanzania & MOROGORO & 2017 & 33.06 & 13.03 & 81.21 & RW2 \\ 
  Tanzania & MOROGORO & 2018 & 30.15 & 9.61 & 89.25 & RW2 \\ 
  Tanzania & MOROGORO & 2019 & 27.48 & 7.02 & 100.80 & RW2 \\ 
  Tanzania & MTWARA & 1980 & 194.97 & 136.98 & 269.11 & RW2 \\ 
  Tanzania & MTWARA & 1981 & 197.61 & 149.72 & 255.35 & RW2 \\ 
  Tanzania & MTWARA & 1982 & 200.69 & 157.67 & 251.09 & RW2 \\ 
  Tanzania & MTWARA & 1983 & 203.32 & 160.86 & 252.71 & RW2 \\ 
  Tanzania & MTWARA & 1984 & 206.00 & 163.87 & 255.83 & RW2 \\ 
  Tanzania & MTWARA & 1985 & 208.98 & 168.65 & 255.05 & RW2 \\ 
  Tanzania & MTWARA & 1986 & 211.80 & 174.63 & 254.00 & RW2 \\ 
  Tanzania & MTWARA & 1987 & 214.38 & 178.65 & 254.30 & RW2 \\ 
  Tanzania & MTWARA & 1988 & 216.56 & 180.88 & 257.31 & RW2 \\ 
  Tanzania & MTWARA & 1989 & 218.70 & 181.69 & 261.51 & RW2 \\ 
  Tanzania & MTWARA & 1990 & 220.02 & 183.39 & 261.94 & RW2 \\ 
  Tanzania & MTWARA & 1991 & 221.08 & 186.35 & 260.72 & RW2 \\ 
  Tanzania & MTWARA & 1992 & 221.31 & 187.59 & 259.61 & RW2 \\ 
  Tanzania & MTWARA & 1993 & 220.47 & 185.50 & 260.20 & RW2 \\ 
  Tanzania & MTWARA & 1994 & 218.01 & 181.77 & 259.50 & RW2 \\ 
  Tanzania & MTWARA & 1995 & 214.26 & 178.67 & 256.04 & RW2 \\ 
  Tanzania & MTWARA & 1996 & 207.73 & 174.22 & 246.87 & RW2 \\ 
  Tanzania & MTWARA & 1997 & 198.77 & 166.80 & 235.25 & RW2 \\ 
  Tanzania & MTWARA & 1998 & 187.45 & 155.80 & 224.22 & RW2 \\ 
  Tanzania & MTWARA & 1999 & 174.36 & 142.90 & 211.44 & RW2 \\ 
  Tanzania & MTWARA & 2000 & 160.02 & 130.15 & 195.02 & RW2 \\ 
  Tanzania & MTWARA & 2001 & 145.70 & 118.50 & 177.70 & RW2 \\ 
  Tanzania & MTWARA & 2002 & 131.38 & 106.55 & 160.59 & RW2 \\ 
  Tanzania & MTWARA & 2003 & 117.99 & 94.19 & 146.35 & RW2 \\ 
  Tanzania & MTWARA & 2004 & 105.60 & 82.77 & 133.36 & RW2 \\ 
  Tanzania & MTWARA & 2005 & 94.14 & 72.48 & 120.55 & RW2 \\ 
  Tanzania & MTWARA & 2006 & 84.34 & 64.79 & 108.50 & RW2 \\ 
  Tanzania & MTWARA & 2007 & 75.50 & 57.60 & 98.29 & RW2 \\ 
  Tanzania & MTWARA & 2008 & 67.92 & 50.72 & 90.00 & RW2 \\ 
  Tanzania & MTWARA & 2009 & 61.22 & 44.63 & 83.51 & RW2 \\ 
  Tanzania & MTWARA & 2010 & 55.48 & 39.30 & 77.96 & RW2 \\ 
  Tanzania & MTWARA & 2011 & 50.31 & 34.90 & 71.98 & RW2 \\ 
  Tanzania & MTWARA & 2012 & 45.64 & 30.83 & 66.97 & RW2 \\ 
  Tanzania & MTWARA & 2013 & 41.35 & 26.68 & 63.73 & RW2 \\ 
  Tanzania & MTWARA & 2014 & 37.57 & 22.22 & 62.81 & RW2 \\ 
  Tanzania & MTWARA & 2015 & 33.92 & 17.39 & 65.03 & RW2 \\ 
  Tanzania & MTWARA & 2016 & 30.69 & 13.44 & 69.04 & RW2 \\ 
  Tanzania & MTWARA & 2017 & 27.81 & 10.03 & 75.23 & RW2 \\ 
  Tanzania & MTWARA & 2018 & 25.15 & 7.35 & 83.26 & RW2 \\ 
  Tanzania & MTWARA & 2019 & 22.80 & 5.27 & 94.57 & RW2 \\ 
  Tanzania & MWANZA & 1980 & 180.92 & 124.28 & 255.81 & RW2 \\ 
  Tanzania & MWANZA & 1981 & 180.05 & 133.47 & 238.05 & RW2 \\ 
  Tanzania & MWANZA & 1982 & 179.22 & 137.82 & 229.49 & RW2 \\ 
  Tanzania & MWANZA & 1983 & 178.08 & 138.37 & 225.58 & RW2 \\ 
  Tanzania & MWANZA & 1984 & 177.11 & 138.80 & 223.22 & RW2 \\ 
  Tanzania & MWANZA & 1985 & 175.96 & 140.98 & 217.52 & RW2 \\ 
  Tanzania & MWANZA & 1986 & 174.70 & 143.02 & 211.36 & RW2 \\ 
  Tanzania & MWANZA & 1987 & 173.11 & 143.68 & 207.37 & RW2 \\ 
  Tanzania & MWANZA & 1988 & 171.47 & 142.54 & 205.57 & RW2 \\ 
  Tanzania & MWANZA & 1989 & 169.62 & 140.08 & 204.22 & RW2 \\ 
  Tanzania & MWANZA & 1990 & 167.36 & 138.40 & 201.23 & RW2 \\ 
  Tanzania & MWANZA & 1991 & 165.53 & 138.85 & 196.47 & RW2 \\ 
  Tanzania & MWANZA & 1992 & 163.70 & 138.12 & 192.96 & RW2 \\ 
  Tanzania & MWANZA & 1993 & 161.99 & 135.88 & 191.67 & RW2 \\ 
  Tanzania & MWANZA & 1994 & 160.12 & 132.83 & 190.91 & RW2 \\ 
  Tanzania & MWANZA & 1995 & 158.50 & 132.00 & 188.84 & RW2 \\ 
  Tanzania & MWANZA & 1996 & 155.62 & 130.76 & 184.24 & RW2 \\ 
  Tanzania & MWANZA & 1997 & 152.09 & 128.51 & 178.74 & RW2 \\ 
  Tanzania & MWANZA & 1998 & 147.74 & 123.92 & 175.00 & RW2 \\ 
  Tanzania & MWANZA & 1999 & 142.26 & 117.92 & 170.78 & RW2 \\ 
  Tanzania & MWANZA & 2000 & 136.12 & 112.40 & 163.66 & RW2 \\ 
  Tanzania & MWANZA & 2001 & 129.14 & 107.53 & 154.47 & RW2 \\ 
  Tanzania & MWANZA & 2002 & 121.68 & 101.33 & 145.21 & RW2 \\ 
  Tanzania & MWANZA & 2003 & 114.06 & 94.69 & 137.69 & RW2 \\ 
  Tanzania & MWANZA & 2004 & 106.22 & 86.95 & 129.44 & RW2 \\ 
  Tanzania & MWANZA & 2005 & 98.47 & 79.87 & 121.00 & RW2 \\ 
  Tanzania & MWANZA & 2006 & 91.30 & 74.62 & 111.47 & RW2 \\ 
  Tanzania & MWANZA & 2007 & 84.64 & 69.17 & 103.28 & RW2 \\ 
  Tanzania & MWANZA & 2008 & 78.63 & 63.44 & 97.18 & RW2 \\ 
  Tanzania & MWANZA & 2009 & 73.17 & 57.58 & 92.71 & RW2 \\ 
  Tanzania & MWANZA & 2010 & 68.34 & 52.54 & 88.57 & RW2 \\ 
  Tanzania & MWANZA & 2011 & 63.79 & 48.65 & 83.11 & RW2 \\ 
  Tanzania & MWANZA & 2012 & 59.56 & 44.94 & 78.61 & RW2 \\ 
  Tanzania & MWANZA & 2013 & 55.62 & 40.26 & 76.12 & RW2 \\ 
  Tanzania & MWANZA & 2014 & 51.88 & 34.46 & 77.17 & RW2 \\ 
  Tanzania & MWANZA & 2015 & 48.33 & 27.78 & 82.51 & RW2 \\ 
  Tanzania & MWANZA & 2016 & 45.16 & 22.04 & 91.15 & RW2 \\ 
  Tanzania & MWANZA & 2017 & 41.92 & 16.98 & 100.91 & RW2 \\ 
  Tanzania & MWANZA & 2018 & 39.36 & 12.89 & 114.96 & RW2 \\ 
  Tanzania & MWANZA & 2019 & 36.64 & 9.36 & 132.40 & RW2 \\ 
  Tanzania & PWANI & 1980 & 239.08 & 169.20 & 325.17 & RW2 \\ 
  Tanzania & PWANI & 1981 & 237.95 & 180.90 & 305.53 & RW2 \\ 
  Tanzania & PWANI & 1982 & 235.93 & 185.50 & 295.14 & RW2 \\ 
  Tanzania & PWANI & 1983 & 234.35 & 185.41 & 291.05 & RW2 \\ 
  Tanzania & PWANI & 1984 & 232.19 & 184.80 & 287.90 & RW2 \\ 
  Tanzania & PWANI & 1985 & 229.68 & 186.61 & 280.05 & RW2 \\ 
  Tanzania & PWANI & 1986 & 226.11 & 187.55 & 270.61 & RW2 \\ 
  Tanzania & PWANI & 1987 & 221.73 & 185.72 & 262.94 & RW2 \\ 
  Tanzania & PWANI & 1988 & 216.20 & 180.85 & 256.61 & RW2 \\ 
  Tanzania & PWANI & 1989 & 210.34 & 174.71 & 251.60 & RW2 \\ 
  Tanzania & PWANI & 1990 & 203.74 & 169.05 & 243.02 & RW2 \\ 
  Tanzania & PWANI & 1991 & 197.56 & 165.80 & 233.35 & RW2 \\ 
  Tanzania & PWANI & 1992 & 191.84 & 162.04 & 225.83 & RW2 \\ 
  Tanzania & PWANI & 1993 & 186.69 & 156.30 & 221.24 & RW2 \\ 
  Tanzania & PWANI & 1994 & 181.89 & 150.52 & 217.67 & RW2 \\ 
  Tanzania & PWANI & 1995 & 177.32 & 146.79 & 212.27 & RW2 \\ 
  Tanzania & PWANI & 1996 & 171.61 & 142.93 & 204.07 & RW2 \\ 
  Tanzania & PWANI & 1997 & 164.98 & 137.95 & 195.69 & RW2 \\ 
  Tanzania & PWANI & 1998 & 157.48 & 130.67 & 187.95 & RW2 \\ 
  Tanzania & PWANI & 1999 & 148.86 & 122.15 & 180.04 & RW2 \\ 
  Tanzania & PWANI & 2000 & 139.52 & 114.02 & 169.22 & RW2 \\ 
  Tanzania & PWANI & 2001 & 130.08 & 106.67 & 157.30 & RW2 \\ 
  Tanzania & PWANI & 2002 & 120.52 & 99.06 & 145.75 & RW2 \\ 
  Tanzania & PWANI & 2003 & 111.47 & 90.71 & 136.94 & RW2 \\ 
  Tanzania & PWANI & 2004 & 102.72 & 82.13 & 128.13 & RW2 \\ 
  Tanzania & PWANI & 2005 & 94.49 & 74.77 & 118.85 & RW2 \\ 
  Tanzania & PWANI & 2006 & 87.13 & 68.81 & 110.01 & RW2 \\ 
  Tanzania & PWANI & 2007 & 80.56 & 63.08 & 102.69 & RW2 \\ 
  Tanzania & PWANI & 2008 & 74.67 & 57.55 & 96.77 & RW2 \\ 
  Tanzania & PWANI & 2009 & 69.44 & 51.99 & 92.43 & RW2 \\ 
  Tanzania & PWANI & 2010 & 64.78 & 47.32 & 88.45 & RW2 \\ 
  Tanzania & PWANI & 2011 & 60.52 & 43.62 & 83.71 & RW2 \\ 
  Tanzania & PWANI & 2012 & 56.58 & 40.07 & 79.60 & RW2 \\ 
  Tanzania & PWANI & 2013 & 52.78 & 35.98 & 77.27 & RW2 \\ 
  Tanzania & PWANI & 2014 & 49.28 & 31.08 & 78.04 & RW2 \\ 
  Tanzania & PWANI & 2015 & 45.96 & 25.06 & 82.81 & RW2 \\ 
  Tanzania & PWANI & 2016 & 42.88 & 20.17 & 89.51 & RW2 \\ 
  Tanzania & PWANI & 2017 & 39.92 & 15.57 & 98.86 & RW2 \\ 
  Tanzania & PWANI & 2018 & 37.41 & 11.74 & 112.91 & RW2 \\ 
  Tanzania & PWANI & 2019 & 34.73 & 8.79 & 131.48 & RW2 \\ 
  Tanzania & RUKWA & 1980 & 249.16 & 178.40 & 337.58 & RW2 \\ 
  Tanzania & RUKWA & 1981 & 242.17 & 186.51 & 309.86 & RW2 \\ 
  Tanzania & RUKWA & 1982 & 235.46 & 186.37 & 293.69 & RW2 \\ 
  Tanzania & RUKWA & 1983 & 228.68 & 182.48 & 283.54 & RW2 \\ 
  Tanzania & RUKWA & 1984 & 222.30 & 177.49 & 274.44 & RW2 \\ 
  Tanzania & RUKWA & 1985 & 216.04 & 176.15 & 262.46 & RW2 \\ 
  Tanzania & RUKWA & 1986 & 210.24 & 174.80 & 251.03 & RW2 \\ 
  Tanzania & RUKWA & 1987 & 204.51 & 171.76 & 241.46 & RW2 \\ 
  Tanzania & RUKWA & 1988 & 199.20 & 167.16 & 235.60 & RW2 \\ 
  Tanzania & RUKWA & 1989 & 194.08 & 161.82 & 231.12 & RW2 \\ 
  Tanzania & RUKWA & 1990 & 189.18 & 158.27 & 224.54 & RW2 \\ 
  Tanzania & RUKWA & 1991 & 184.86 & 156.70 & 217.34 & RW2 \\ 
  Tanzania & RUKWA & 1992 & 180.81 & 153.96 & 210.91 & RW2 \\ 
  Tanzania & RUKWA & 1993 & 176.78 & 149.30 & 208.03 & RW2 \\ 
  Tanzania & RUKWA & 1994 & 172.64 & 143.94 & 205.57 & RW2 \\ 
  Tanzania & RUKWA & 1995 & 168.49 & 140.35 & 201.46 & RW2 \\ 
  Tanzania & RUKWA & 1996 & 163.19 & 137.10 & 193.53 & RW2 \\ 
  Tanzania & RUKWA & 1997 & 156.88 & 131.82 & 185.03 & RW2 \\ 
  Tanzania & RUKWA & 1998 & 149.55 & 124.54 & 178.01 & RW2 \\ 
  Tanzania & RUKWA & 1999 & 141.45 & 116.78 & 170.16 & RW2 \\ 
  Tanzania & RUKWA & 2000 & 132.68 & 108.95 & 160.03 & RW2 \\ 
  Tanzania & RUKWA & 2001 & 124.08 & 102.86 & 148.32 & RW2 \\ 
  Tanzania & RUKWA & 2002 & 115.62 & 96.44 & 138.12 & RW2 \\ 
  Tanzania & RUKWA & 2003 & 107.62 & 89.14 & 129.23 & RW2 \\ 
  Tanzania & RUKWA & 2004 & 100.33 & 81.91 & 122.34 & RW2 \\ 
  Tanzania & RUKWA & 2005 & 93.58 & 75.98 & 114.22 & RW2 \\ 
  Tanzania & RUKWA & 2006 & 87.78 & 71.86 & 106.55 & RW2 \\ 
  Tanzania & RUKWA & 2007 & 82.71 & 67.98 & 100.10 & RW2 \\ 
  Tanzania & RUKWA & 2008 & 78.53 & 63.80 & 96.20 & RW2 \\ 
  Tanzania & RUKWA & 2009 & 74.85 & 59.69 & 93.32 & RW2 \\ 
  Tanzania & RUKWA & 2010 & 71.86 & 56.27 & 91.55 & RW2 \\ 
  Tanzania & RUKWA & 2011 & 69.09 & 54.10 & 88.18 & RW2 \\ 
  Tanzania & RUKWA & 2012 & 66.50 & 52.01 & 85.05 & RW2 \\ 
  Tanzania & RUKWA & 2013 & 64.11 & 48.49 & 85.03 & RW2 \\ 
  Tanzania & RUKWA & 2014 & 61.79 & 42.86 & 89.32 & RW2 \\ 
  Tanzania & RUKWA & 2015 & 59.55 & 35.40 & 98.72 & RW2 \\ 
  Tanzania & RUKWA & 2016 & 57.23 & 28.85 & 111.69 & RW2 \\ 
  Tanzania & RUKWA & 2017 & 55.10 & 23.03 & 127.72 & RW2 \\ 
  Tanzania & RUKWA & 2018 & 52.92 & 17.95 & 149.68 & RW2 \\ 
  Tanzania & RUKWA & 2019 & 51.22 & 13.66 & 174.66 & RW2 \\ 
  Tanzania & RUVUMA & 1980 & 143.80 & 98.63 & 205.28 & RW2 \\ 
  Tanzania & RUVUMA & 1981 & 145.52 & 107.07 & 194.20 & RW2 \\ 
  Tanzania & RUVUMA & 1982 & 147.43 & 112.38 & 190.37 & RW2 \\ 
  Tanzania & RUVUMA & 1983 & 149.27 & 114.74 & 191.85 & RW2 \\ 
  Tanzania & RUVUMA & 1984 & 151.01 & 116.72 & 193.18 & RW2 \\ 
  Tanzania & RUVUMA & 1985 & 153.14 & 120.84 & 191.90 & RW2 \\ 
  Tanzania & RUVUMA & 1986 & 154.96 & 124.97 & 190.62 & RW2 \\ 
  Tanzania & RUVUMA & 1987 & 157.05 & 128.25 & 190.71 & RW2 \\ 
  Tanzania & RUVUMA & 1988 & 159.08 & 130.03 & 193.26 & RW2 \\ 
  Tanzania & RUVUMA & 1989 & 161.05 & 131.06 & 196.50 & RW2 \\ 
  Tanzania & RUVUMA & 1990 & 162.89 & 133.13 & 197.95 & RW2 \\ 
  Tanzania & RUVUMA & 1991 & 164.95 & 136.44 & 198.05 & RW2 \\ 
  Tanzania & RUVUMA & 1992 & 166.73 & 138.63 & 198.96 & RW2 \\ 
  Tanzania & RUVUMA & 1993 & 168.02 & 139.24 & 201.23 & RW2 \\ 
  Tanzania & RUVUMA & 1994 & 168.58 & 138.55 & 203.32 & RW2 \\ 
  Tanzania & RUVUMA & 1995 & 168.47 & 139.13 & 204.00 & RW2 \\ 
  Tanzania & RUVUMA & 1996 & 166.47 & 138.54 & 199.64 & RW2 \\ 
  Tanzania & RUVUMA & 1997 & 162.34 & 135.54 & 193.73 & RW2 \\ 
  Tanzania & RUVUMA & 1998 & 156.57 & 129.89 & 188.25 & RW2 \\ 
  Tanzania & RUVUMA & 1999 & 149.04 & 121.90 & 180.71 & RW2 \\ 
  Tanzania & RUVUMA & 2000 & 139.88 & 114.27 & 169.90 & RW2 \\ 
  Tanzania & RUVUMA & 2001 & 130.30 & 107.00 & 157.79 & RW2 \\ 
  Tanzania & RUVUMA & 2002 & 120.22 & 98.87 & 145.64 & RW2 \\ 
  Tanzania & RUVUMA & 2003 & 110.26 & 89.74 & 135.04 & RW2 \\ 
  Tanzania & RUVUMA & 2004 & 100.76 & 80.86 & 125.29 & RW2 \\ 
  Tanzania & RUVUMA & 2005 & 91.65 & 72.48 & 114.96 & RW2 \\ 
  Tanzania & RUVUMA & 2006 & 83.51 & 65.86 & 105.11 & RW2 \\ 
  Tanzania & RUVUMA & 2007 & 76.18 & 59.79 & 96.59 & RW2 \\ 
  Tanzania & RUVUMA & 2008 & 69.64 & 53.57 & 90.22 & RW2 \\ 
  Tanzania & RUVUMA & 2009 & 63.81 & 47.63 & 84.82 & RW2 \\ 
  Tanzania & RUVUMA & 2010 & 58.77 & 42.64 & 80.40 & RW2 \\ 
  Tanzania & RUVUMA & 2011 & 54.05 & 38.43 & 75.14 & RW2 \\ 
  Tanzania & RUVUMA & 2012 & 49.75 & 34.62 & 70.56 & RW2 \\ 
  Tanzania & RUVUMA & 2013 & 45.82 & 30.47 & 67.98 & RW2 \\ 
  Tanzania & RUVUMA & 2014 & 42.15 & 25.74 & 67.61 & RW2 \\ 
  Tanzania & RUVUMA & 2015 & 38.86 & 20.73 & 71.53 & RW2 \\ 
  Tanzania & RUVUMA & 2016 & 35.75 & 16.15 & 76.24 & RW2 \\ 
  Tanzania & RUVUMA & 2017 & 32.77 & 12.45 & 83.56 & RW2 \\ 
  Tanzania & RUVUMA & 2018 & 30.12 & 9.21 & 94.94 & RW2 \\ 
  Tanzania & RUVUMA & 2019 & 27.74 & 6.75 & 105.99 & RW2 \\ 
  Tanzania & SHINYANGA & 1980 & 193.83 & 142.55 & 259.36 & RW2 \\ 
  Tanzania & SHINYANGA & 1981 & 190.85 & 150.53 & 239.73 & RW2 \\ 
  Tanzania & SHINYANGA & 1982 & 188.38 & 152.08 & 231.02 & RW2 \\ 
  Tanzania & SHINYANGA & 1983 & 185.73 & 149.85 & 228.28 & RW2 \\ 
  Tanzania & SHINYANGA & 1984 & 182.89 & 147.21 & 225.29 & RW2 \\ 
  Tanzania & SHINYANGA & 1985 & 180.59 & 147.57 & 218.90 & RW2 \\ 
  Tanzania & SHINYANGA & 1986 & 178.04 & 147.99 & 212.14 & RW2 \\ 
  Tanzania & SHINYANGA & 1987 & 175.41 & 147.85 & 207.38 & RW2 \\ 
  Tanzania & SHINYANGA & 1988 & 173.07 & 145.08 & 205.37 & RW2 \\ 
  Tanzania & SHINYANGA & 1989 & 170.70 & 142.25 & 204.03 & RW2 \\ 
  Tanzania & SHINYANGA & 1990 & 168.38 & 140.39 & 200.58 & RW2 \\ 
  Tanzania & SHINYANGA & 1991 & 166.55 & 140.83 & 195.90 & RW2 \\ 
  Tanzania & SHINYANGA & 1992 & 164.97 & 140.55 & 192.69 & RW2 \\ 
  Tanzania & SHINYANGA & 1993 & 163.68 & 138.77 & 191.84 & RW2 \\ 
  Tanzania & SHINYANGA & 1994 & 162.23 & 135.83 & 192.17 & RW2 \\ 
  Tanzania & SHINYANGA & 1995 & 160.81 & 135.02 & 190.80 & RW2 \\ 
  Tanzania & SHINYANGA & 1996 & 158.37 & 134.16 & 186.05 & RW2 \\ 
  Tanzania & SHINYANGA & 1997 & 155.11 & 132.09 & 180.90 & RW2 \\ 
  Tanzania & SHINYANGA & 1998 & 150.79 & 127.50 & 176.81 & RW2 \\ 
  Tanzania & SHINYANGA & 1999 & 145.32 & 121.53 & 172.62 & RW2 \\ 
  Tanzania & SHINYANGA & 2000 & 138.78 & 115.94 & 164.99 & RW2 \\ 
  Tanzania & SHINYANGA & 2001 & 132.03 & 110.87 & 155.48 & RW2 \\ 
  Tanzania & SHINYANGA & 2002 & 124.63 & 105.52 & 146.89 & RW2 \\ 
  Tanzania & SHINYANGA & 2003 & 117.21 & 98.46 & 139.09 & RW2 \\ 
  Tanzania & SHINYANGA & 2004 & 109.76 & 91.13 & 132.10 & RW2 \\ 
  Tanzania & SHINYANGA & 2005 & 102.53 & 84.54 & 123.57 & RW2 \\ 
  Tanzania & SHINYANGA & 2006 & 95.87 & 79.92 & 114.82 & RW2 \\ 
  Tanzania & SHINYANGA & 2007 & 89.83 & 75.06 & 107.21 & RW2 \\ 
  Tanzania & SHINYANGA & 2008 & 84.35 & 69.79 & 101.70 & RW2 \\ 
  Tanzania & SHINYANGA & 2009 & 79.46 & 64.54 & 97.92 & RW2 \\ 
  Tanzania & SHINYANGA & 2010 & 75.16 & 60.00 & 94.03 & RW2 \\ 
  Tanzania & SHINYANGA & 2011 & 71.09 & 57.26 & 88.27 & RW2 \\ 
  Tanzania & SHINYANGA & 2012 & 67.31 & 54.50 & 82.95 & RW2 \\ 
  Tanzania & SHINYANGA & 2013 & 63.73 & 50.20 & 80.50 & RW2 \\ 
  Tanzania & SHINYANGA & 2014 & 60.32 & 43.75 & 82.56 & RW2 \\ 
  Tanzania & SHINYANGA & 2015 & 57.10 & 35.43 & 90.48 & RW2 \\ 
  Tanzania & SHINYANGA & 2016 & 54.00 & 28.54 & 100.08 & RW2 \\ 
  Tanzania & SHINYANGA & 2017 & 51.00 & 22.26 & 113.15 & RW2 \\ 
  Tanzania & SHINYANGA & 2018 & 48.09 & 17.01 & 128.35 & RW2 \\ 
  Tanzania & SHINYANGA & 2019 & 45.53 & 12.85 & 149.81 & RW2 \\ 
  Tanzania & SINGIDA & 1980 & 160.05 & 110.47 & 226.23 & RW2 \\ 
  Tanzania & SINGIDA & 1981 & 157.09 & 116.13 & 209.36 & RW2 \\ 
  Tanzania & SINGIDA & 1982 & 154.03 & 117.04 & 200.73 & RW2 \\ 
  Tanzania & SINGIDA & 1983 & 151.13 & 115.27 & 194.76 & RW2 \\ 
  Tanzania & SINGIDA & 1984 & 148.22 & 113.79 & 190.84 & RW2 \\ 
  Tanzania & SINGIDA & 1985 & 145.64 & 113.95 & 184.07 & RW2 \\ 
  Tanzania & SINGIDA & 1986 & 142.90 & 114.11 & 177.27 & RW2 \\ 
  Tanzania & SINGIDA & 1987 & 140.44 & 113.86 & 171.96 & RW2 \\ 
  Tanzania & SINGIDA & 1988 & 138.20 & 112.10 & 169.25 & RW2 \\ 
  Tanzania & SINGIDA & 1989 & 136.06 & 110.24 & 167.33 & RW2 \\ 
  Tanzania & SINGIDA & 1990 & 133.96 & 109.08 & 163.45 & RW2 \\ 
  Tanzania & SINGIDA & 1991 & 132.12 & 109.02 & 159.12 & RW2 \\ 
  Tanzania & SINGIDA & 1992 & 130.46 & 108.70 & 156.07 & RW2 \\ 
  Tanzania & SINGIDA & 1993 & 128.76 & 106.64 & 154.37 & RW2 \\ 
  Tanzania & SINGIDA & 1994 & 126.78 & 103.90 & 153.58 & RW2 \\ 
  Tanzania & SINGIDA & 1995 & 124.72 & 102.17 & 151.72 & RW2 \\ 
  Tanzania & SINGIDA & 1996 & 121.46 & 100.45 & 146.70 & RW2 \\ 
  Tanzania & SINGIDA & 1997 & 117.30 & 97.10 & 141.11 & RW2 \\ 
  Tanzania & SINGIDA & 1998 & 112.11 & 92.11 & 135.91 & RW2 \\ 
  Tanzania & SINGIDA & 1999 & 106.16 & 86.05 & 130.65 & RW2 \\ 
  Tanzania & SINGIDA & 2000 & 99.26 & 79.99 & 122.49 & RW2 \\ 
  Tanzania & SINGIDA & 2001 & 92.23 & 74.77 & 113.40 & RW2 \\ 
  Tanzania & SINGIDA & 2002 & 85.09 & 68.92 & 104.64 & RW2 \\ 
  Tanzania & SINGIDA & 2003 & 78.14 & 62.60 & 97.17 & RW2 \\ 
  Tanzania & SINGIDA & 2004 & 71.39 & 56.47 & 90.52 & RW2 \\ 
  Tanzania & SINGIDA & 2005 & 65.04 & 50.73 & 83.04 & RW2 \\ 
  Tanzania & SINGIDA & 2006 & 59.30 & 46.19 & 75.70 & RW2 \\ 
  Tanzania & SINGIDA & 2007 & 54.20 & 41.98 & 69.54 & RW2 \\ 
  Tanzania & SINGIDA & 2008 & 49.58 & 37.76 & 64.81 & RW2 \\ 
  Tanzania & SINGIDA & 2009 & 45.44 & 33.60 & 61.00 & RW2 \\ 
  Tanzania & SINGIDA & 2010 & 41.88 & 30.21 & 57.78 & RW2 \\ 
  Tanzania & SINGIDA & 2011 & 38.56 & 27.47 & 53.77 & RW2 \\ 
  Tanzania & SINGIDA & 2012 & 35.48 & 24.89 & 50.19 & RW2 \\ 
  Tanzania & SINGIDA & 2013 & 32.72 & 22.05 & 47.83 & RW2 \\ 
  Tanzania & SINGIDA & 2014 & 30.19 & 18.78 & 47.48 & RW2 \\ 
  Tanzania & SINGIDA & 2015 & 27.68 & 15.13 & 49.91 & RW2 \\ 
  Tanzania & SINGIDA & 2016 & 25.54 & 11.94 & 53.41 & RW2 \\ 
  Tanzania & SINGIDA & 2017 & 23.55 & 9.26 & 58.71 & RW2 \\ 
  Tanzania & SINGIDA & 2018 & 21.55 & 6.80 & 65.69 & RW2 \\ 
  Tanzania & SINGIDA & 2019 & 19.87 & 5.05 & 76.31 & RW2 \\ 
  Tanzania & TABORA & 1980 & 178.54 & 124.70 & 248.35 & RW2 \\ 
  Tanzania & TABORA & 1981 & 176.32 & 131.79 & 231.09 & RW2 \\ 
  Tanzania & TABORA & 1982 & 174.06 & 133.95 & 223.10 & RW2 \\ 
  Tanzania & TABORA & 1983 & 171.98 & 133.28 & 218.98 & RW2 \\ 
  Tanzania & TABORA & 1984 & 170.01 & 131.67 & 216.00 & RW2 \\ 
  Tanzania & TABORA & 1985 & 168.32 & 132.90 & 209.94 & RW2 \\ 
  Tanzania & TABORA & 1986 & 166.63 & 134.24 & 203.91 & RW2 \\ 
  Tanzania & TABORA & 1987 & 165.06 & 134.87 & 200.08 & RW2 \\ 
  Tanzania & TABORA & 1988 & 163.64 & 134.15 & 197.75 & RW2 \\ 
  Tanzania & TABORA & 1989 & 162.52 & 132.63 & 197.82 & RW2 \\ 
  Tanzania & TABORA & 1990 & 161.11 & 132.32 & 194.81 & RW2 \\ 
  Tanzania & TABORA & 1991 & 160.42 & 133.86 & 191.23 & RW2 \\ 
  Tanzania & TABORA & 1992 & 159.69 & 134.21 & 188.89 & RW2 \\ 
  Tanzania & TABORA & 1993 & 158.79 & 133.16 & 188.36 & RW2 \\ 
  Tanzania & TABORA & 1994 & 157.48 & 130.87 & 187.73 & RW2 \\ 
  Tanzania & TABORA & 1995 & 156.07 & 130.19 & 186.07 & RW2 \\ 
  Tanzania & TABORA & 1996 & 152.87 & 129.04 & 180.67 & RW2 \\ 
  Tanzania & TABORA & 1997 & 148.61 & 125.95 & 174.21 & RW2 \\ 
  Tanzania & TABORA & 1998 & 142.97 & 120.58 & 168.95 & RW2 \\ 
  Tanzania & TABORA & 1999 & 136.14 & 113.35 & 162.94 & RW2 \\ 
  Tanzania & TABORA & 2000 & 128.24 & 106.24 & 153.52 & RW2 \\ 
  Tanzania & TABORA & 2001 & 119.92 & 100.20 & 142.70 & RW2 \\ 
  Tanzania & TABORA & 2002 & 111.45 & 93.62 & 132.50 & RW2 \\ 
  Tanzania & TABORA & 2003 & 103.08 & 85.85 & 123.68 & RW2 \\ 
  Tanzania & TABORA & 2004 & 95.10 & 78.01 & 116.15 & RW2 \\ 
  Tanzania & TABORA & 2005 & 87.53 & 71.24 & 106.91 & RW2 \\ 
  Tanzania & TABORA & 2006 & 80.72 & 66.05 & 98.12 & RW2 \\ 
  Tanzania & TABORA & 2007 & 74.64 & 60.97 & 90.89 & RW2 \\ 
  Tanzania & TABORA & 2008 & 69.17 & 55.79 & 85.17 & RW2 \\ 
  Tanzania & TABORA & 2009 & 64.39 & 50.81 & 81.17 & RW2 \\ 
  Tanzania & TABORA & 2010 & 60.18 & 46.42 & 77.87 & RW2 \\ 
  Tanzania & TABORA & 2011 & 56.28 & 43.27 & 72.82 & RW2 \\ 
  Tanzania & TABORA & 2012 & 52.73 & 40.21 & 68.62 & RW2 \\ 
  Tanzania & TABORA & 2013 & 49.34 & 36.46 & 66.59 & RW2 \\ 
  Tanzania & TABORA & 2014 & 46.26 & 31.44 & 67.27 & RW2 \\ 
  Tanzania & TABORA & 2015 & 43.20 & 25.50 & 72.62 & RW2 \\ 
  Tanzania & TABORA & 2016 & 40.48 & 20.27 & 79.29 & RW2 \\ 
  Tanzania & TABORA & 2017 & 37.76 & 15.62 & 88.47 & RW2 \\ 
  Tanzania & TABORA & 2018 & 35.42 & 11.86 & 101.04 & RW2 \\ 
  Tanzania & TABORA & 2019 & 33.10 & 8.70 & 117.06 & RW2 \\ 
  Tanzania & TANGA & 1980 & 177.53 & 124.21 & 244.39 & RW2 \\ 
  Tanzania & TANGA & 1981 & 179.03 & 134.99 & 231.43 & RW2 \\ 
  Tanzania & TANGA & 1982 & 180.27 & 140.69 & 225.96 & RW2 \\ 
  Tanzania & TANGA & 1983 & 181.49 & 143.13 & 226.83 & RW2 \\ 
  Tanzania & TANGA & 1984 & 182.54 & 144.54 & 227.59 & RW2 \\ 
  Tanzania & TANGA & 1985 & 183.04 & 147.85 & 223.65 & RW2 \\ 
  Tanzania & TANGA & 1986 & 183.14 & 151.22 & 220.23 & RW2 \\ 
  Tanzania & TANGA & 1987 & 182.44 & 152.50 & 217.34 & RW2 \\ 
  Tanzania & TANGA & 1988 & 181.12 & 151.05 & 215.83 & RW2 \\ 
  Tanzania & TANGA & 1989 & 179.23 & 148.96 & 215.28 & RW2 \\ 
  Tanzania & TANGA & 1990 & 176.48 & 147.14 & 210.61 & RW2 \\ 
  Tanzania & TANGA & 1991 & 173.96 & 146.75 & 205.27 & RW2 \\ 
  Tanzania & TANGA & 1992 & 171.19 & 145.38 & 200.96 & RW2 \\ 
  Tanzania & TANGA & 1993 & 168.41 & 141.79 & 198.85 & RW2 \\ 
  Tanzania & TANGA & 1994 & 165.32 & 137.46 & 196.79 & RW2 \\ 
  Tanzania & TANGA & 1995 & 162.32 & 135.08 & 194.08 & RW2 \\ 
  Tanzania & TANGA & 1996 & 157.88 & 132.36 & 187.75 & RW2 \\ 
  Tanzania & TANGA & 1997 & 152.45 & 127.51 & 180.43 & RW2 \\ 
  Tanzania & TANGA & 1998 & 145.91 & 121.64 & 174.20 & RW2 \\ 
  Tanzania & TANGA & 1999 & 138.39 & 113.76 & 167.31 & RW2 \\ 
  Tanzania & TANGA & 2000 & 129.90 & 106.31 & 157.81 & RW2 \\ 
  Tanzania & TANGA & 2001 & 121.10 & 99.60 & 146.65 & RW2 \\ 
  Tanzania & TANGA & 2002 & 112.02 & 92.40 & 135.42 & RW2 \\ 
  Tanzania & TANGA & 2003 & 102.92 & 83.92 & 126.10 & RW2 \\ 
  Tanzania & TANGA & 2004 & 94.26 & 75.53 & 117.07 & RW2 \\ 
  Tanzania & TANGA & 2005 & 85.86 & 67.95 & 107.69 & RW2 \\ 
  Tanzania & TANGA & 2006 & 78.29 & 61.98 & 98.19 & RW2 \\ 
  Tanzania & TANGA & 2007 & 71.42 & 56.22 & 90.37 & RW2 \\ 
  Tanzania & TANGA & 2008 & 65.20 & 50.27 & 84.20 & RW2 \\ 
  Tanzania & TANGA & 2009 & 59.72 & 44.75 & 79.15 & RW2 \\ 
  Tanzania & TANGA & 2010 & 54.78 & 39.73 & 74.90 & RW2 \\ 
  Tanzania & TANGA & 2011 & 50.29 & 35.91 & 69.57 & RW2 \\ 
  Tanzania & TANGA & 2012 & 46.20 & 32.24 & 65.30 & RW2 \\ 
  Tanzania & TANGA & 2013 & 42.38 & 28.24 & 62.41 & RW2 \\ 
  Tanzania & TANGA & 2014 & 38.91 & 23.86 & 62.11 & RW2 \\ 
  Tanzania & TANGA & 2015 & 35.71 & 19.11 & 64.77 & RW2 \\ 
  Tanzania & TANGA & 2016 & 32.79 & 15.02 & 69.84 & RW2 \\ 
  Tanzania & TANGA & 2017 & 29.95 & 11.41 & 76.40 & RW2 \\ 
  Tanzania & TANGA & 2018 & 27.49 & 8.48 & 85.84 & RW2 \\ 
  Tanzania & TANGA & 2019 & 25.24 & 6.19 & 97.46 & RW2 \\ 
  Togo & ALL & 1980 & 168.86 & 162.69 & 175.00 & IHME \\ 
  Togo & ALL & 1980 & 178.01 & 133.38 & 233.24 & RW2 \\ 
  Togo & ALL & 1980 & 177.50 & 164.20 & 191.70 & UN \\ 
  Togo & ALL & 1981 & 165.07 & 159.14 & 171.14 & IHME \\ 
  Togo & ALL & 1981 & 173.86 & 141.43 & 211.38 & RW2 \\ 
  Togo & ALL & 1981 & 173.40 & 160.40 & 187.20 & UN \\ 
  Togo & ALL & 1982 & 161.80 & 156.03 & 167.68 & IHME \\ 
  Togo & ALL & 1982 & 169.82 & 141.27 & 202.86 & RW2 \\ 
  Togo & ALL & 1982 & 169.50 & 156.90 & 182.90 & UN \\ 
  Togo & ALL & 1983 & 159.37 & 153.97 & 165.06 & IHME \\ 
  Togo & ALL & 1983 & 165.69 & 135.56 & 201.28 & RW2 \\ 
  Togo & ALL & 1983 & 165.70 & 153.50 & 178.90 & UN \\ 
  Togo & ALL & 1984 & 156.95 & 151.61 & 162.47 & IHME \\ 
  Togo & ALL & 1984 & 162.10 & 129.83 & 200.49 & RW2 \\ 
  Togo & ALL & 1984 & 162.30 & 150.40 & 175.10 & UN \\ 
  Togo & ALL & 1985 & 154.67 & 149.39 & 160.01 & IHME \\ 
  Togo & ALL & 1985 & 158.19 & 128.66 & 192.87 & RW2 \\ 
  Togo & ALL & 1985 & 159.00 & 147.40 & 171.50 & UN \\ 
  Togo & ALL & 1986 & 152.55 & 147.19 & 157.84 & IHME \\ 
  Togo & ALL & 1986 & 155.08 & 127.86 & 186.65 & RW2 \\ 
  Togo & ALL & 1986 & 156.10 & 144.60 & 168.00 & UN \\ 
  Togo & ALL & 1987 & 150.77 & 145.59 & 155.72 & IHME \\ 
  Togo & ALL & 1987 & 152.43 & 126.84 & 182.40 & RW2 \\ 
  Togo & ALL & 1987 & 153.30 & 142.20 & 164.90 & UN \\ 
  Togo & ALL & 1988 & 148.78 & 143.68 & 153.92 & IHME \\ 
  Togo & ALL & 1988 & 150.07 & 123.75 & 180.48 & RW2 \\ 
  Togo & ALL & 1988 & 150.80 & 139.90 & 162.00 & UN \\ 
  Togo & ALL & 1989 & 146.74 & 141.71 & 151.88 & IHME \\ 
  Togo & ALL & 1989 & 148.05 & 120.88 & 179.64 & RW2 \\ 
  Togo & ALL & 1989 & 148.50 & 138.00 & 159.30 & UN \\ 
  Togo & ALL & 1990 & 144.53 & 139.63 & 149.59 & IHME \\ 
  Togo & ALL & 1990 & 146.49 & 120.22 & 177.97 & RW2 \\ 
  Togo & ALL & 1990 & 146.20 & 136.20 & 156.90 & UN \\ 
  Togo & ALL & 1991 & 142.41 & 137.40 & 147.48 & IHME \\ 
  Togo & ALL & 1991 & 144.66 & 119.99 & 173.23 & RW2 \\ 
  Togo & ALL & 1991 & 144.10 & 134.40 & 154.60 & UN \\ 
  Togo & ALL & 1992 & 140.40 & 135.47 & 145.46 & IHME \\ 
  Togo & ALL & 1992 & 142.71 & 118.84 & 170.12 & RW2 \\ 
  Togo & ALL & 1992 & 142.10 & 132.60 & 152.50 & UN \\ 
  Togo & ALL & 1993 & 139.14 & 134.04 & 144.22 & IHME \\ 
  Togo & ALL & 1993 & 140.57 & 116.47 & 168.65 & RW2 \\ 
  Togo & ALL & 1993 & 140.00 & 130.70 & 150.30 & UN \\ 
  Togo & ALL & 1994 & 137.39 & 132.30 & 142.55 & IHME \\ 
  Togo & ALL & 1994 & 138.12 & 113.04 & 168.24 & RW2 \\ 
  Togo & ALL & 1994 & 137.70 & 128.60 & 147.90 & UN \\ 
  Togo & ALL & 1995 & 135.15 & 130.09 & 140.09 & IHME \\ 
  Togo & ALL & 1995 & 135.46 & 110.67 & 164.56 & RW2 \\ 
  Togo & ALL & 1995 & 135.40 & 126.20 & 145.40 & UN \\ 
  Togo & ALL & 1996 & 132.73 & 127.48 & 137.77 & IHME \\ 
  Togo & ALL & 1996 & 132.63 & 109.39 & 160.26 & RW2 \\ 
  Togo & ALL & 1996 & 132.70 & 123.80 & 142.70 & UN \\ 
  Togo & ALL & 1997 & 130.07 & 124.91 & 135.17 & IHME \\ 
  Togo & ALL & 1997 & 129.71 & 107.59 & 155.66 & RW2 \\ 
  Togo & ALL & 1997 & 129.90 & 121.20 & 139.70 & UN \\ 
  Togo & ALL & 1998 & 127.76 & 122.62 & 132.94 & IHME \\ 
  Togo & ALL & 1998 & 126.78 & 104.48 & 153.66 & RW2 \\ 
  Togo & ALL & 1998 & 127.00 & 118.40 & 136.50 & UN \\ 
  Togo & ALL & 1999 & 125.33 & 120.11 & 130.55 & IHME \\ 
  Togo & ALL & 1999 & 123.74 & 100.63 & 150.95 & RW2 \\ 
  Togo & ALL & 1999 & 123.90 & 115.40 & 133.10 & UN \\ 
  Togo & ALL & 2000 & 123.00 & 117.62 & 128.38 & IHME \\ 
  Togo & ALL & 2000 & 120.65 & 97.82 & 147.81 & RW2 \\ 
  Togo & ALL & 2000 & 120.80 & 112.40 & 129.80 & UN \\ 
  Togo & ALL & 2001 & 120.68 & 115.41 & 125.84 & IHME \\ 
  Togo & ALL & 2001 & 117.55 & 95.79 & 143.37 & RW2 \\ 
  Togo & ALL & 2001 & 117.60 & 109.30 & 126.50 & UN \\ 
  Togo & ALL & 2002 & 118.22 & 112.94 & 123.54 & IHME \\ 
  Togo & ALL & 2002 & 114.39 & 93.70 & 139.03 & RW2 \\ 
  Togo & ALL & 2002 & 114.40 & 106.20 & 123.20 & UN \\ 
  Togo & ALL & 2003 & 115.60 & 110.27 & 120.83 & IHME \\ 
  Togo & ALL & 2003 & 111.31 & 90.86 & 135.74 & RW2 \\ 
  Togo & ALL & 2003 & 111.20 & 103.10 & 120.00 & UN \\ 
  Togo & ALL & 2004 & 112.94 & 107.66 & 118.19 & IHME \\ 
  Togo & ALL & 2004 & 108.04 & 87.03 & 133.23 & RW2 \\ 
  Togo & ALL & 2004 & 108.10 & 99.90 & 116.90 & UN \\ 
  Togo & ALL & 2005 & 110.45 & 105.11 & 115.73 & IHME \\ 
  Togo & ALL & 2005 & 105.00 & 84.46 & 129.82 & RW2 \\ 
  Togo & ALL & 2005 & 105.10 & 96.80 & 113.90 & UN \\ 
  Togo & ALL & 2006 & 107.33 & 101.97 & 112.75 & IHME \\ 
  Togo & ALL & 2006 & 101.92 & 82.72 & 124.81 & RW2 \\ 
  Togo & ALL & 2006 & 102.10 & 93.70 & 111.10 & UN \\ 
  Togo & ALL & 2007 & 104.38 & 99.02 & 110.06 & IHME \\ 
  Togo & ALL & 2007 & 98.97 & 80.80 & 120.62 & RW2 \\ 
  Togo & ALL & 2007 & 99.20 & 90.70 & 108.40 & UN \\ 
  Togo & ALL & 2008 & 101.33 & 95.85 & 107.02 & IHME \\ 
  Togo & ALL & 2008 & 96.23 & 77.93 & 118.31 & RW2 \\ 
  Togo & ALL & 2008 & 96.40 & 87.70 & 105.80 & UN \\ 
  Togo & ALL & 2009 & 98.22 & 92.54 & 104.23 & IHME \\ 
  Togo & ALL & 2009 & 93.38 & 74.48 & 116.79 & RW2 \\ 
  Togo & ALL & 2009 & 93.60 & 84.50 & 103.40 & UN \\ 
  Togo & ALL & 2010 & 95.02 & 89.31 & 101.46 & IHME \\ 
  Togo & ALL & 2010 & 90.71 & 71.56 & 114.96 & RW2 \\ 
  Togo & ALL & 2010 & 90.90 & 81.40 & 101.20 & UN \\ 
  Togo & ALL & 2011 & 91.41 & 85.36 & 98.29 & IHME \\ 
  Togo & ALL & 2011 & 88.18 & 70.27 & 110.14 & RW2 \\ 
  Togo & ALL & 2011 & 88.30 & 78.30 & 99.10 & UN \\ 
  Togo & ALL & 2012 & 87.89 & 81.60 & 95.11 & IHME \\ 
  Togo & ALL & 2012 & 85.66 & 69.02 & 105.78 & RW2 \\ 
  Togo & ALL & 2012 & 85.80 & 75.20 & 97.20 & UN \\ 
  Togo & ALL & 2013 & 84.63 & 78.21 & 92.40 & IHME \\ 
  Togo & ALL & 2013 & 83.27 & 65.34 & 105.40 & RW2 \\ 
  Togo & ALL & 2013 & 83.30 & 71.80 & 95.40 & UN \\ 
  Togo & ALL & 2014 & 81.42 & 74.85 & 89.39 & IHME \\ 
  Togo & ALL & 2014 & 80.87 & 57.69 & 112.15 & RW2 \\ 
  Togo & ALL & 2014 & 80.80 & 68.60 & 94.30 & UN \\ 
  Togo & ALL & 2015 & 78.04 & 71.17 & 86.48 & IHME \\ 
  Togo & ALL & 2015 & 78.43 & 47.20 & 127.72 & RW2 \\ 
  Togo & ALL & 2015 & 78.40 & 65.10 & 93.90 & UN \\ 
  Togo & ALL & 2016 & 76.34 & 38.53 & 146.59 & RW2 \\ 
  Togo & ALL & 2017 & 73.96 & 30.34 & 170.90 & RW2 \\ 
  Togo & ALL & 2018 & 71.78 & 23.63 & 202.99 & RW2 \\ 
  Togo & ALL & 2019 & 69.58 & 17.37 & 238.60 & RW2 \\ 
  Togo & ALL & 80-84 & 177.38 & 192.44 & 163.26 & HT-Direct \\ 
  Togo & ALL & 85-89 & 153.65 & 165.35 & 142.64 & HT-Direct \\ 
  Togo & ALL & 90-94 & 150.87 & 160.27 & 141.92 & HT-Direct \\ 
  Togo & ALL & 95-99 & 133.40 & 143.17 & 124.21 & HT-Direct \\ 
  Togo & ALL & 00-04 & 114.83 & 126.25 & 104.32 & HT-Direct \\ 
  Togo & ALL & 05-09 & 97.69 & 107.39 & 88.78 & HT-Direct \\ 
  Togo & ALL & 10-14 & 86.51 & 96.89 & 77.15 & HT-Direct \\ 
  Togo & ALL & 15-19 & 73.96 & 30.80 & 168.21 & RW2 \\ 
  Togo & CENTRALE & 1980 & 193.68 & 145.29 & 254.56 & RW2 \\ 
  Togo & CENTRALE & 1981 & 187.27 & 150.27 & 231.88 & RW2 \\ 
  Togo & CENTRALE & 1982 & 180.74 & 148.40 & 219.03 & RW2 \\ 
  Togo & CENTRALE & 1983 & 174.62 & 142.49 & 212.98 & RW2 \\ 
  Togo & CENTRALE & 1984 & 168.92 & 136.46 & 207.62 & RW2 \\ 
  Togo & CENTRALE & 1985 & 163.51 & 134.09 & 197.81 & RW2 \\ 
  Togo & CENTRALE & 1986 & 159.33 & 132.81 & 189.57 & RW2 \\ 
  Togo & CENTRALE & 1987 & 156.01 & 131.35 & 184.24 & RW2 \\ 
  Togo & CENTRALE & 1988 & 153.56 & 128.62 & 181.90 & RW2 \\ 
  Togo & CENTRALE & 1989 & 151.74 & 126.09 & 180.75 & RW2 \\ 
  Togo & CENTRALE & 1990 & 150.78 & 126.06 & 179.87 & RW2 \\ 
  Togo & CENTRALE & 1991 & 149.32 & 126.23 & 176.01 & RW2 \\ 
  Togo & CENTRALE & 1992 & 147.43 & 124.95 & 172.89 & RW2 \\ 
  Togo & CENTRALE & 1993 & 145.03 & 122.28 & 171.11 & RW2 \\ 
  Togo & CENTRALE & 1994 & 142.50 & 118.56 & 169.99 & RW2 \\ 
  Togo & CENTRALE & 1995 & 139.19 & 115.16 & 166.23 & RW2 \\ 
  Togo & CENTRALE & 1996 & 136.66 & 113.97 & 162.03 & RW2 \\ 
  Togo & CENTRALE & 1997 & 134.42 & 112.71 & 159.14 & RW2 \\ 
  Togo & CENTRALE & 1998 & 132.66 & 110.88 & 157.69 & RW2 \\ 
  Togo & CENTRALE & 1999 & 131.29 & 107.88 & 157.64 & RW2 \\ 
  Togo & CENTRALE & 2000 & 130.39 & 107.00 & 157.81 & RW2 \\ 
  Togo & CENTRALE & 2001 & 129.23 & 106.60 & 155.56 & RW2 \\ 
  Togo & CENTRALE & 2002 & 127.78 & 105.82 & 153.48 & RW2 \\ 
  Togo & CENTRALE & 2003 & 125.91 & 103.88 & 152.22 & RW2 \\ 
  Togo & CENTRALE & 2004 & 123.79 & 101.07 & 150.84 & RW2 \\ 
  Togo & CENTRALE & 2005 & 121.16 & 98.82 & 147.71 & RW2 \\ 
  Togo & CENTRALE & 2006 & 118.65 & 97.60 & 143.59 & RW2 \\ 
  Togo & CENTRALE & 2007 & 116.02 & 95.32 & 140.18 & RW2 \\ 
  Togo & CENTRALE & 2008 & 113.52 & 92.27 & 138.75 & RW2 \\ 
  Togo & CENTRALE & 2009 & 110.96 & 88.78 & 137.78 & RW2 \\ 
  Togo & CENTRALE & 2010 & 108.69 & 85.97 & 136.84 & RW2 \\ 
  Togo & CENTRALE & 2011 & 106.33 & 84.56 & 133.31 & RW2 \\ 
  Togo & CENTRALE & 2012 & 104.10 & 83.13 & 130.09 & RW2 \\ 
  Togo & CENTRALE & 2013 & 101.94 & 79.19 & 131.24 & RW2 \\ 
  Togo & CENTRALE & 2014 & 99.84 & 71.52 & 138.14 & RW2 \\ 
  Togo & CENTRALE & 2015 & 97.73 & 60.63 & 154.24 & RW2 \\ 
  Togo & CENTRALE & 2016 & 95.51 & 50.61 & 173.52 & RW2 \\ 
  Togo & CENTRALE & 2017 & 93.60 & 41.31 & 199.46 & RW2 \\ 
  Togo & CENTRALE & 2018 & 91.35 & 32.76 & 232.16 & RW2 \\ 
  Togo & CENTRALE & 2019 & 89.44 & 25.55 & 273.47 & RW2 \\ 
  Togo & GRANDE AGGLOMÉRATION DE LOMÉ & 1980 & 131.08 & 89.57 & 186.95 & RW2 \\ 
  Togo & GRANDE AGGLOMÉRATION DE LOMÉ & 1981 & 126.88 & 93.13 & 169.14 & RW2 \\ 
  Togo & GRANDE AGGLOMÉRATION DE LOMÉ & 1982 & 122.84 & 93.10 & 158.92 & RW2 \\ 
  Togo & GRANDE AGGLOMÉRATION DE LOMÉ & 1983 & 118.79 & 90.93 & 153.79 & RW2 \\ 
  Togo & GRANDE AGGLOMÉRATION DE LOMÉ & 1984 & 115.29 & 88.56 & 148.74 & RW2 \\ 
  Togo & GRANDE AGGLOMÉRATION DE LOMÉ & 1985 & 111.83 & 87.11 & 141.43 & RW2 \\ 
  Togo & GRANDE AGGLOMÉRATION DE LOMÉ & 1986 & 109.17 & 87.06 & 136.12 & RW2 \\ 
  Togo & GRANDE AGGLOMÉRATION DE LOMÉ & 1987 & 107.29 & 86.74 & 131.81 & RW2 \\ 
  Togo & GRANDE AGGLOMÉRATION DE LOMÉ & 1988 & 105.90 & 85.60 & 129.99 & RW2 \\ 
  Togo & GRANDE AGGLOMÉRATION DE LOMÉ & 1989 & 104.85 & 84.32 & 129.46 & RW2 \\ 
  Togo & GRANDE AGGLOMÉRATION DE LOMÉ & 1990 & 104.47 & 84.62 & 128.83 & RW2 \\ 
  Togo & GRANDE AGGLOMÉRATION DE LOMÉ & 1991 & 103.66 & 84.89 & 125.80 & RW2 \\ 
  Togo & GRANDE AGGLOMÉRATION DE LOMÉ & 1992 & 102.33 & 84.44 & 123.61 & RW2 \\ 
  Togo & GRANDE AGGLOMÉRATION DE LOMÉ & 1993 & 100.68 & 82.93 & 122.45 & RW2 \\ 
  Togo & GRANDE AGGLOMÉRATION DE LOMÉ & 1994 & 98.71 & 80.29 & 121.47 & RW2 \\ 
  Togo & GRANDE AGGLOMÉRATION DE LOMÉ & 1995 & 95.97 & 77.64 & 118.26 & RW2 \\ 
  Togo & GRANDE AGGLOMÉRATION DE LOMÉ & 1996 & 93.56 & 76.48 & 114.06 & RW2 \\ 
  Togo & GRANDE AGGLOMÉRATION DE LOMÉ & 1997 & 91.22 & 74.99 & 111.29 & RW2 \\ 
  Togo & GRANDE AGGLOMÉRATION DE LOMÉ & 1998 & 89.08 & 72.80 & 109.06 & RW2 \\ 
  Togo & GRANDE AGGLOMÉRATION DE LOMÉ & 1999 & 86.91 & 70.19 & 107.25 & RW2 \\ 
  Togo & GRANDE AGGLOMÉRATION DE LOMÉ & 2000 & 85.04 & 68.62 & 105.67 & RW2 \\ 
  Togo & GRANDE AGGLOMÉRATION DE LOMÉ & 2001 & 82.88 & 67.10 & 102.33 & RW2 \\ 
  Togo & GRANDE AGGLOMÉRATION DE LOMÉ & 2002 & 80.51 & 65.34 & 99.10 & RW2 \\ 
  Togo & GRANDE AGGLOMÉRATION DE LOMÉ & 2003 & 77.95 & 62.78 & 96.44 & RW2 \\ 
  Togo & GRANDE AGGLOMÉRATION DE LOMÉ & 2004 & 75.29 & 59.90 & 93.85 & RW2 \\ 
  Togo & GRANDE AGGLOMÉRATION DE LOMÉ & 2005 & 72.41 & 57.37 & 90.62 & RW2 \\ 
  Togo & GRANDE AGGLOMÉRATION DE LOMÉ & 2006 & 69.65 & 55.66 & 86.82 & RW2 \\ 
  Togo & GRANDE AGGLOMÉRATION DE LOMÉ & 2007 & 67.10 & 53.64 & 83.63 & RW2 \\ 
  Togo & GRANDE AGGLOMÉRATION DE LOMÉ & 2008 & 64.68 & 51.04 & 81.53 & RW2 \\ 
  Togo & GRANDE AGGLOMÉRATION DE LOMÉ & 2009 & 62.32 & 48.35 & 79.84 & RW2 \\ 
  Togo & GRANDE AGGLOMÉRATION DE LOMÉ & 2010 & 60.24 & 45.98 & 78.45 & RW2 \\ 
  Togo & GRANDE AGGLOMÉRATION DE LOMÉ & 2011 & 58.20 & 44.54 & 75.76 & RW2 \\ 
  Togo & GRANDE AGGLOMÉRATION DE LOMÉ & 2012 & 56.26 & 42.86 & 73.49 & RW2 \\ 
  Togo & GRANDE AGGLOMÉRATION DE LOMÉ & 2013 & 54.39 & 40.08 & 73.40 & RW2 \\ 
  Togo & GRANDE AGGLOMÉRATION DE LOMÉ & 2014 & 52.51 & 35.51 & 76.77 & RW2 \\ 
  Togo & GRANDE AGGLOMÉRATION DE LOMÉ & 2015 & 50.82 & 29.75 & 85.54 & RW2 \\ 
  Togo & GRANDE AGGLOMÉRATION DE LOMÉ & 2016 & 49.05 & 24.35 & 96.22 & RW2 \\ 
  Togo & GRANDE AGGLOMÉRATION DE LOMÉ & 2017 & 47.49 & 19.54 & 111.08 & RW2 \\ 
  Togo & GRANDE AGGLOMÉRATION DE LOMÉ & 2018 & 45.76 & 15.34 & 129.52 & RW2 \\ 
  Togo & GRANDE AGGLOMÉRATION DE LOMÉ & 2019 & 44.35 & 11.87 & 153.57 & RW2 \\ 
  Togo & KARA & 1980 & 209.57 & 157.50 & 273.35 & RW2 \\ 
  Togo & KARA & 1981 & 203.30 & 164.33 & 249.83 & RW2 \\ 
  Togo & KARA & 1982 & 197.15 & 162.69 & 237.23 & RW2 \\ 
  Togo & KARA & 1983 & 191.23 & 156.68 & 231.85 & RW2 \\ 
  Togo & KARA & 1984 & 185.81 & 151.41 & 226.52 & RW2 \\ 
  Togo & KARA & 1985 & 180.62 & 149.15 & 217.14 & RW2 \\ 
  Togo & KARA & 1986 & 176.76 & 148.24 & 209.26 & RW2 \\ 
  Togo & KARA & 1987 & 173.80 & 147.19 & 203.91 & RW2 \\ 
  Togo & KARA & 1988 & 171.69 & 144.80 & 202.12 & RW2 \\ 
  Togo & KARA & 1989 & 170.38 & 142.60 & 201.86 & RW2 \\ 
  Togo & KARA & 1990 & 169.74 & 143.06 & 201.29 & RW2 \\ 
  Togo & KARA & 1991 & 168.61 & 143.21 & 197.17 & RW2 \\ 
  Togo & KARA & 1992 & 166.91 & 142.63 & 194.20 & RW2 \\ 
  Togo & KARA & 1993 & 164.53 & 139.93 & 192.82 & RW2 \\ 
  Togo & KARA & 1994 & 161.77 & 136.02 & 192.00 & RW2 \\ 
  Togo & KARA & 1995 & 158.28 & 132.44 & 187.37 & RW2 \\ 
  Togo & KARA & 1996 & 155.25 & 130.80 & 182.62 & RW2 \\ 
  Togo & KARA & 1997 & 152.79 & 129.67 & 179.22 & RW2 \\ 
  Togo & KARA & 1998 & 150.70 & 127.09 & 178.04 & RW2 \\ 
  Togo & KARA & 1999 & 149.05 & 123.84 & 177.81 & RW2 \\ 
  Togo & KARA & 2000 & 147.86 & 122.45 & 178.01 & RW2 \\ 
  Togo & KARA & 2001 & 146.19 & 121.88 & 175.12 & RW2 \\ 
  Togo & KARA & 2002 & 144.22 & 120.39 & 172.02 & RW2 \\ 
  Togo & KARA & 2003 & 141.65 & 117.60 & 170.27 & RW2 \\ 
  Togo & KARA & 2004 & 138.74 & 114.16 & 168.45 & RW2 \\ 
  Togo & KARA & 2005 & 135.26 & 110.91 & 164.05 & RW2 \\ 
  Togo & KARA & 2006 & 131.57 & 108.72 & 158.37 & RW2 \\ 
  Togo & KARA & 2007 & 127.72 & 105.84 & 153.46 & RW2 \\ 
  Togo & KARA & 2008 & 123.84 & 101.47 & 150.17 & RW2 \\ 
  Togo & KARA & 2009 & 120.09 & 96.92 & 147.73 & RW2 \\ 
  Togo & KARA & 2010 & 116.38 & 92.77 & 144.95 & RW2 \\ 
  Togo & KARA & 2011 & 112.67 & 90.37 & 139.60 & RW2 \\ 
  Togo & KARA & 2012 & 109.15 & 87.83 & 134.64 & RW2 \\ 
  Togo & KARA & 2013 & 105.66 & 82.98 & 133.86 & RW2 \\ 
  Togo & KARA & 2014 & 102.18 & 73.87 & 139.79 & RW2 \\ 
  Togo & KARA & 2015 & 98.84 & 61.63 & 155.45 & RW2 \\ 
  Togo & KARA & 2016 & 95.52 & 50.45 & 173.31 & RW2 \\ 
  Togo & KARA & 2017 & 92.47 & 40.66 & 197.28 & RW2 \\ 
  Togo & KARA & 2018 & 89.16 & 31.82 & 226.33 & RW2 \\ 
  Togo & KARA & 2019 & 86.44 & 24.41 & 265.63 & RW2 \\ 
  Togo & MARITIME (SANS AGGLOMÉRATION DE LOMÉ) & 1980 & 172.40 & 126.83 & 230.25 & RW2 \\ 
  Togo & MARITIME (SANS AGGLOMÉRATION DE LOMÉ) & 1981 & 167.04 & 131.68 & 209.46 & RW2 \\ 
  Togo & MARITIME (SANS AGGLOMÉRATION DE LOMÉ) & 1982 & 161.86 & 130.60 & 198.58 & RW2 \\ 
  Togo & MARITIME (SANS AGGLOMÉRATION DE LOMÉ) & 1983 & 156.72 & 125.86 & 193.55 & RW2 \\ 
  Togo & MARITIME (SANS AGGLOMÉRATION DE LOMÉ) & 1984 & 152.09 & 121.35 & 189.70 & RW2 \\ 
  Togo & MARITIME (SANS AGGLOMÉRATION DE LOMÉ) & 1985 & 147.63 & 119.15 & 181.04 & RW2 \\ 
  Togo & MARITIME (SANS AGGLOMÉRATION DE LOMÉ) & 1986 & 144.21 & 118.60 & 174.23 & RW2 \\ 
  Togo & MARITIME (SANS AGGLOMÉRATION DE LOMÉ) & 1987 & 141.55 & 117.38 & 169.54 & RW2 \\ 
  Togo & MARITIME (SANS AGGLOMÉRATION DE LOMÉ) & 1988 & 139.66 & 115.62 & 167.29 & RW2 \\ 
  Togo & MARITIME (SANS AGGLOMÉRATION DE LOMÉ) & 1989 & 138.23 & 113.57 & 166.40 & RW2 \\ 
  Togo & MARITIME (SANS AGGLOMÉRATION DE LOMÉ) & 1990 & 137.57 & 113.87 & 165.18 & RW2 \\ 
  Togo & MARITIME (SANS AGGLOMÉRATION DE LOMÉ) & 1991 & 136.24 & 114.09 & 161.95 & RW2 \\ 
  Togo & MARITIME (SANS AGGLOMÉRATION DE LOMÉ) & 1992 & 134.42 & 113.56 & 159.05 & RW2 \\ 
  Togo & MARITIME (SANS AGGLOMÉRATION DE LOMÉ) & 1993 & 132.24 & 111.24 & 157.27 & RW2 \\ 
  Togo & MARITIME (SANS AGGLOMÉRATION DE LOMÉ) & 1994 & 129.41 & 107.14 & 155.41 & RW2 \\ 
  Togo & MARITIME (SANS AGGLOMÉRATION DE LOMÉ) & 1995 & 126.05 & 103.92 & 151.50 & RW2 \\ 
  Togo & MARITIME (SANS AGGLOMÉRATION DE LOMÉ) & 1996 & 123.04 & 102.34 & 147.31 & RW2 \\ 
  Togo & MARITIME (SANS AGGLOMÉRATION DE LOMÉ) & 1997 & 120.03 & 100.33 & 143.21 & RW2 \\ 
  Togo & MARITIME (SANS AGGLOMÉRATION DE LOMÉ) & 1998 & 117.45 & 97.18 & 141.12 & RW2 \\ 
  Togo & MARITIME (SANS AGGLOMÉRATION DE LOMÉ) & 1999 & 114.83 & 93.58 & 139.55 & RW2 \\ 
  Togo & MARITIME (SANS AGGLOMÉRATION DE LOMÉ) & 2000 & 112.63 & 91.48 & 138.13 & RW2 \\ 
  Togo & MARITIME (SANS AGGLOMÉRATION DE LOMÉ) & 2001 & 110.00 & 89.40 & 135.01 & RW2 \\ 
  Togo & MARITIME (SANS AGGLOMÉRATION DE LOMÉ) & 2002 & 107.37 & 86.99 & 131.77 & RW2 \\ 
  Togo & MARITIME (SANS AGGLOMÉRATION DE LOMÉ) & 2003 & 104.27 & 83.46 & 129.09 & RW2 \\ 
  Togo & MARITIME (SANS AGGLOMÉRATION DE LOMÉ) & 2004 & 101.18 & 79.92 & 127.03 & RW2 \\ 
  Togo & MARITIME (SANS AGGLOMÉRATION DE LOMÉ) & 2005 & 97.79 & 76.78 & 123.45 & RW2 \\ 
  Togo & MARITIME (SANS AGGLOMÉRATION DE LOMÉ) & 2006 & 94.49 & 74.23 & 119.19 & RW2 \\ 
  Togo & MARITIME (SANS AGGLOMÉRATION DE LOMÉ) & 2007 & 91.32 & 71.49 & 115.39 & RW2 \\ 
  Togo & MARITIME (SANS AGGLOMÉRATION DE LOMÉ) & 2008 & 88.50 & 68.39 & 113.23 & RW2 \\ 
  Togo & MARITIME (SANS AGGLOMÉRATION DE LOMÉ) & 2009 & 85.71 & 65.00 & 111.97 & RW2 \\ 
  Togo & MARITIME (SANS AGGLOMÉRATION DE LOMÉ) & 2010 & 83.19 & 62.39 & 110.31 & RW2 \\ 
  Togo & MARITIME (SANS AGGLOMÉRATION DE LOMÉ) & 2011 & 80.59 & 60.24 & 107.09 & RW2 \\ 
  Togo & MARITIME (SANS AGGLOMÉRATION DE LOMÉ) & 2012 & 78.25 & 58.30 & 104.24 & RW2 \\ 
  Togo & MARITIME (SANS AGGLOMÉRATION DE LOMÉ) & 2013 & 75.98 & 54.94 & 104.32 & RW2 \\ 
  Togo & MARITIME (SANS AGGLOMÉRATION DE LOMÉ) & 2014 & 73.80 & 49.44 & 108.80 & RW2 \\ 
  Togo & MARITIME (SANS AGGLOMÉRATION DE LOMÉ) & 2015 & 71.58 & 41.95 & 120.43 & RW2 \\ 
  Togo & MARITIME (SANS AGGLOMÉRATION DE LOMÉ) & 2016 & 69.48 & 34.63 & 134.19 & RW2 \\ 
  Togo & MARITIME (SANS AGGLOMÉRATION DE LOMÉ) & 2017 & 67.36 & 28.01 & 153.65 & RW2 \\ 
  Togo & MARITIME (SANS AGGLOMÉRATION DE LOMÉ) & 2018 & 65.35 & 22.27 & 179.43 & RW2 \\ 
  Togo & MARITIME (SANS AGGLOMÉRATION DE LOMÉ) & 2019 & 63.47 & 16.92 & 209.50 & RW2 \\ 
  Togo & PLATEAUX & 1980 & 181.35 & 137.25 & 236.49 & RW2 \\ 
  Togo & PLATEAUX & 1981 & 175.14 & 142.76 & 213.56 & RW2 \\ 
  Togo & PLATEAUX & 1982 & 168.94 & 140.90 & 202.02 & RW2 \\ 
  Togo & PLATEAUX & 1983 & 163.42 & 134.74 & 197.03 & RW2 \\ 
  Togo & PLATEAUX & 1984 & 157.98 & 128.81 & 192.36 & RW2 \\ 
  Togo & PLATEAUX & 1985 & 152.97 & 126.27 & 183.37 & RW2 \\ 
  Togo & PLATEAUX & 1986 & 148.97 & 125.10 & 175.98 & RW2 \\ 
  Togo & PLATEAUX & 1987 & 145.89 & 123.48 & 171.40 & RW2 \\ 
  Togo & PLATEAUX & 1988 & 143.56 & 120.96 & 168.85 & RW2 \\ 
  Togo & PLATEAUX & 1989 & 141.89 & 118.31 & 168.91 & RW2 \\ 
  Togo & PLATEAUX & 1990 & 141.04 & 118.11 & 168.29 & RW2 \\ 
  Togo & PLATEAUX & 1991 & 139.50 & 117.97 & 164.64 & RW2 \\ 
  Togo & PLATEAUX & 1992 & 137.63 & 116.83 & 161.37 & RW2 \\ 
  Togo & PLATEAUX & 1993 & 135.41 & 114.27 & 159.82 & RW2 \\ 
  Togo & PLATEAUX & 1994 & 132.71 & 110.61 & 158.74 & RW2 \\ 
  Togo & PLATEAUX & 1995 & 129.54 & 107.37 & 154.62 & RW2 \\ 
  Togo & PLATEAUX & 1996 & 126.77 & 106.12 & 150.36 & RW2 \\ 
  Togo & PLATEAUX & 1997 & 124.30 & 104.54 & 146.93 & RW2 \\ 
  Togo & PLATEAUX & 1998 & 122.15 & 102.37 & 145.06 & RW2 \\ 
  Togo & PLATEAUX & 1999 & 120.26 & 99.29 & 144.46 & RW2 \\ 
  Togo & PLATEAUX & 2000 & 119.00 & 98.29 & 143.77 & RW2 \\ 
  Togo & PLATEAUX & 2001 & 117.32 & 97.40 & 141.07 & RW2 \\ 
  Togo & PLATEAUX & 2002 & 115.44 & 96.10 & 138.49 & RW2 \\ 
  Togo & PLATEAUX & 2003 & 113.25 & 93.54 & 136.54 & RW2 \\ 
  Togo & PLATEAUX & 2004 & 110.78 & 90.76 & 134.93 & RW2 \\ 
  Togo & PLATEAUX & 2005 & 108.06 & 88.08 & 131.48 & RW2 \\ 
  Togo & PLATEAUX & 2006 & 105.29 & 86.43 & 127.66 & RW2 \\ 
  Togo & PLATEAUX & 2007 & 102.81 & 84.43 & 124.29 & RW2 \\ 
  Togo & PLATEAUX & 2008 & 100.29 & 81.49 & 123.15 & RW2 \\ 
  Togo & PLATEAUX & 2009 & 97.83 & 77.98 & 122.20 & RW2 \\ 
  Togo & PLATEAUX & 2010 & 95.57 & 75.05 & 121.17 & RW2 \\ 
  Togo & PLATEAUX & 2011 & 93.51 & 73.76 & 117.89 & RW2 \\ 
  Togo & PLATEAUX & 2012 & 91.40 & 72.18 & 115.38 & RW2 \\ 
  Togo & PLATEAUX & 2013 & 89.20 & 68.45 & 115.84 & RW2 \\ 
  Togo & PLATEAUX & 2014 & 87.30 & 61.86 & 122.79 & RW2 \\ 
  Togo & PLATEAUX & 2015 & 85.28 & 52.34 & 136.50 & RW2 \\ 
  Togo & PLATEAUX & 2016 & 83.39 & 43.48 & 155.55 & RW2 \\ 
  Togo & PLATEAUX & 2017 & 81.68 & 35.38 & 178.61 & RW2 \\ 
  Togo & PLATEAUX & 2018 & 79.85 & 28.11 & 207.65 & RW2 \\ 
  Togo & PLATEAUX & 2019 & 77.71 & 21.66 & 246.58 & RW2 \\ 
  Togo & SAVANES & 1980 & 224.65 & 173.09 & 286.03 & RW2 \\ 
  Togo & SAVANES & 1981 & 217.30 & 180.51 & 258.64 & RW2 \\ 
  Togo & SAVANES & 1982 & 209.93 & 177.93 & 245.83 & RW2 \\ 
  Togo & SAVANES & 1983 & 202.84 & 169.73 & 240.95 & RW2 \\ 
  Togo & SAVANES & 1984 & 196.34 & 162.60 & 235.51 & RW2 \\ 
  Togo & SAVANES & 1985 & 190.12 & 158.46 & 225.41 & RW2 \\ 
  Togo & SAVANES & 1986 & 185.42 & 156.86 & 217.06 & RW2 \\ 
  Togo & SAVANES & 1987 & 181.77 & 155.29 & 211.09 & RW2 \\ 
  Togo & SAVANES & 1988 & 179.09 & 151.93 & 208.92 & RW2 \\ 
  Togo & SAVANES & 1989 & 177.13 & 148.90 & 208.22 & RW2 \\ 
  Togo & SAVANES & 1990 & 176.09 & 149.31 & 207.27 & RW2 \\ 
  Togo & SAVANES & 1991 & 174.39 & 149.52 & 202.96 & RW2 \\ 
  Togo & SAVANES & 1992 & 172.15 & 148.10 & 199.16 & RW2 \\ 
  Togo & SAVANES & 1993 & 169.18 & 144.97 & 196.66 & RW2 \\ 
  Togo & SAVANES & 1994 & 165.62 & 139.99 & 195.35 & RW2 \\ 
  Togo & SAVANES & 1995 & 161.23 & 135.26 & 190.14 & RW2 \\ 
  Togo & SAVANES & 1996 & 157.43 & 133.05 & 184.67 & RW2 \\ 
  Togo & SAVANES & 1997 & 153.88 & 130.77 & 179.90 & RW2 \\ 
  Togo & SAVANES & 1998 & 150.66 & 127.41 & 177.72 & RW2 \\ 
  Togo & SAVANES & 1999 & 147.75 & 123.20 & 176.38 & RW2 \\ 
  Togo & SAVANES & 2000 & 145.26 & 120.56 & 174.12 & RW2 \\ 
  Togo & SAVANES & 2001 & 142.25 & 119.13 & 169.60 & RW2 \\ 
  Togo & SAVANES & 2002 & 138.90 & 116.49 & 165.34 & RW2 \\ 
  Togo & SAVANES & 2003 & 134.91 & 112.50 & 161.62 & RW2 \\ 
  Togo & SAVANES & 2004 & 130.56 & 107.70 & 157.77 & RW2 \\ 
  Togo & SAVANES & 2005 & 125.67 & 102.94 & 152.56 & RW2 \\ 
  Togo & SAVANES & 2006 & 120.68 & 99.42 & 145.76 & RW2 \\ 
  Togo & SAVANES & 2007 & 115.57 & 95.24 & 139.45 & RW2 \\ 
  Togo & SAVANES & 2008 & 110.56 & 90.04 & 134.93 & RW2 \\ 
  Togo & SAVANES & 2009 & 105.51 & 84.41 & 131.07 & RW2 \\ 
  Togo & SAVANES & 2010 & 100.82 & 78.88 & 127.50 & RW2 \\ 
  Togo & SAVANES & 2011 & 96.02 & 75.25 & 121.48 & RW2 \\ 
  Togo & SAVANES & 2012 & 91.61 & 71.19 & 116.15 & RW2 \\ 
  Togo & SAVANES & 2013 & 87.24 & 65.19 & 114.46 & RW2 \\ 
  Togo & SAVANES & 2014 & 83.08 & 56.96 & 117.62 & RW2 \\ 
  Togo & SAVANES & 2015 & 79.10 & 46.97 & 129.28 & RW2 \\ 
  Togo & SAVANES & 2016 & 75.23 & 38.14 & 142.75 & RW2 \\ 
  Togo & SAVANES & 2017 & 71.62 & 29.98 & 161.81 & RW2 \\ 
  Togo & SAVANES & 2018 & 67.92 & 22.76 & 184.72 & RW2 \\ 
  Togo & SAVANES & 2019 & 64.51 & 17.37 & 212.72 & RW2 \\ 
  Uganda & ALL & 1980 & 196.61 & 190.11 & 204.38 & IHME \\ 
  Uganda & ALL & 1980 & 216.80 & 167.12 & 275.78 & RW2 \\ 
  Uganda & ALL & 1980 & 215.60 & 200.80 & 232.20 & UN \\ 
  Uganda & ALL & 1981 & 195.38 & 188.88 & 202.68 & IHME \\ 
  Uganda & ALL & 1981 & 212.95 & 177.37 & 252.82 & RW2 \\ 
  Uganda & ALL & 1981 & 214.40 & 200.00 & 230.40 & UN \\ 
  Uganda & ALL & 1982 & 193.12 & 186.63 & 200.28 & IHME \\ 
  Uganda & ALL & 1982 & 209.18 & 177.15 & 245.28 & RW2 \\ 
  Uganda & ALL & 1982 & 210.40 & 197.00 & 225.20 & UN \\ 
  Uganda & ALL & 1983 & 190.08 & 183.89 & 197.44 & IHME \\ 
  Uganda & ALL & 1983 & 205.26 & 170.44 & 245.24 & RW2 \\ 
  Uganda & ALL & 1983 & 205.00 & 192.40 & 218.70 & UN \\ 
  Uganda & ALL & 1984 & 187.00 & 180.91 & 193.98 & IHME \\ 
  Uganda & ALL & 1984 & 201.83 & 164.04 & 245.49 & RW2 \\ 
  Uganda & ALL & 1984 & 199.80 & 187.40 & 212.90 & UN \\ 
  Uganda & ALL & 1985 & 184.09 & 178.15 & 190.53 & IHME \\ 
  Uganda & ALL & 1985 & 198.01 & 163.71 & 237.40 & RW2 \\ 
  Uganda & ALL & 1985 & 195.80 & 183.60 & 208.20 & UN \\ 
  Uganda & ALL & 1986 & 181.33 & 175.60 & 187.38 & IHME \\ 
  Uganda & ALL & 1986 & 194.79 & 163.37 & 230.41 & RW2 \\ 
  Uganda & ALL & 1986 & 193.20 & 181.40 & 205.20 & UN \\ 
  Uganda & ALL & 1987 & 176.29 & 171.84 & 180.66 & IHME \\ 
  Uganda & ALL & 1987 & 191.88 & 162.41 & 225.78 & RW2 \\ 
  Uganda & ALL & 1987 & 191.80 & 180.60 & 203.50 & UN \\ 
  Uganda & ALL & 1988 & 173.34 & 169.27 & 177.62 & IHME \\ 
  Uganda & ALL & 1988 & 189.11 & 158.54 & 223.94 & RW2 \\ 
  Uganda & ALL & 1988 & 190.90 & 180.00 & 202.50 & UN \\ 
  Uganda & ALL & 1989 & 170.61 & 166.41 & 175.00 & IHME \\ 
  Uganda & ALL & 1989 & 186.59 & 154.70 & 223.33 & RW2 \\ 
  Uganda & ALL & 1989 & 189.50 & 178.90 & 201.30 & UN \\ 
  Uganda & ALL & 1990 & 167.93 & 163.67 & 172.44 & IHME \\ 
  Uganda & ALL & 1990 & 184.31 & 152.97 & 220.82 & RW2 \\ 
  Uganda & ALL & 1990 & 187.10 & 176.50 & 198.60 & UN \\ 
  Uganda & ALL & 1991 & 165.36 & 161.24 & 169.87 & IHME \\ 
  Uganda & ALL & 1991 & 182.01 & 152.64 & 215.24 & RW2 \\ 
  Uganda & ALL & 1991 & 183.70 & 173.00 & 194.80 & UN \\ 
  Uganda & ALL & 1992 & 162.56 & 158.57 & 167.23 & IHME \\ 
  Uganda & ALL & 1992 & 179.66 & 151.28 & 211.64 & RW2 \\ 
  Uganda & ALL & 1992 & 179.60 & 169.00 & 190.50 & UN \\ 
  Uganda & ALL & 1993 & 159.61 & 155.62 & 164.11 & IHME \\ 
  Uganda & ALL & 1993 & 177.19 & 148.45 & 210.01 & RW2 \\ 
  Uganda & ALL & 1993 & 175.60 & 165.10 & 186.50 & UN \\ 
  Uganda & ALL & 1994 & 156.04 & 151.87 & 160.52 & IHME \\ 
  Uganda & ALL & 1994 & 174.40 & 144.33 & 209.60 & RW2 \\ 
  Uganda & ALL & 1994 & 172.20 & 162.10 & 183.10 & UN \\ 
  Uganda & ALL & 1995 & 152.25 & 148.16 & 156.58 & IHME \\ 
  Uganda & ALL & 1995 & 171.44 & 142.18 & 205.68 & RW2 \\ 
  Uganda & ALL & 1995 & 169.50 & 159.70 & 180.20 & UN \\ 
  Uganda & ALL & 1996 & 148.42 & 144.32 & 152.71 & IHME \\ 
  Uganda & ALL & 1996 & 167.84 & 140.53 & 200.05 & RW2 \\ 
  Uganda & ALL & 1996 & 166.90 & 157.20 & 177.50 & UN \\ 
  Uganda & ALL & 1997 & 144.22 & 140.11 & 148.39 & IHME \\ 
  Uganda & ALL & 1997 & 163.73 & 137.71 & 193.67 & RW2 \\ 
  Uganda & ALL & 1997 & 163.90 & 154.20 & 174.20 & UN \\ 
  Uganda & ALL & 1998 & 139.84 & 135.81 & 144.00 & IHME \\ 
  Uganda & ALL & 1998 & 159.08 & 132.75 & 189.85 & RW2 \\ 
  Uganda & ALL & 1998 & 159.90 & 150.50 & 170.20 & UN \\ 
  Uganda & ALL & 1999 & 135.34 & 131.22 & 139.59 & IHME \\ 
  Uganda & ALL & 1999 & 153.65 & 126.45 & 184.49 & RW2 \\ 
  Uganda & ALL & 1999 & 154.80 & 145.50 & 165.10 & UN \\ 
  Uganda & ALL & 2000 & 130.97 & 126.88 & 135.15 & IHME \\ 
  Uganda & ALL & 2000 & 147.67 & 121.80 & 177.77 & RW2 \\ 
  Uganda & ALL & 2000 & 148.40 & 139.30 & 158.40 & UN \\ 
  Uganda & ALL & 2001 & 126.40 & 122.33 & 130.48 & IHME \\ 
  Uganda & ALL & 2001 & 140.67 & 116.88 & 168.15 & RW2 \\ 
  Uganda & ALL & 2001 & 141.00 & 132.00 & 150.80 & UN \\ 
  Uganda & ALL & 2002 & 122.05 & 118.05 & 125.99 & IHME \\ 
  Uganda & ALL & 2002 & 132.93 & 111.13 & 158.31 & RW2 \\ 
  Uganda & ALL & 2002 & 132.80 & 124.20 & 142.10 & UN \\ 
  Uganda & ALL & 2003 & 117.50 & 113.59 & 121.41 & IHME \\ 
  Uganda & ALL & 2003 & 124.74 & 103.87 & 149.54 & RW2 \\ 
  Uganda & ALL & 2003 & 124.20 & 116.10 & 133.20 & UN \\ 
  Uganda & ALL & 2004 & 113.17 & 109.18 & 117.19 & IHME \\ 
  Uganda & ALL & 2004 & 116.16 & 95.17 & 141.53 & RW2 \\ 
  Uganda & ALL & 2004 & 115.60 & 107.80 & 124.10 & UN \\ 
  Uganda & ALL & 2005 & 108.73 & 104.64 & 112.93 & IHME \\ 
  Uganda & ALL & 2005 & 107.48 & 87.12 & 131.14 & RW2 \\ 
  Uganda & ALL & 2005 & 107.40 & 99.70 & 115.40 & UN \\ 
  Uganda & ALL & 2006 & 104.56 & 100.19 & 108.77 & IHME \\ 
  Uganda & ALL & 2006 & 99.58 & 81.61 & 120.44 & RW2 \\ 
  Uganda & ALL & 2006 & 99.70 & 92.10 & 107.60 & UN \\ 
  Uganda & ALL & 2007 & 100.39 & 95.71 & 104.82 & IHME \\ 
  Uganda & ALL & 2007 & 92.36 & 76.33 & 111.23 & RW2 \\ 
  Uganda & ALL & 2007 & 92.50 & 85.00 & 100.30 & UN \\ 
  Uganda & ALL & 2008 & 96.12 & 91.36 & 100.94 & IHME \\ 
  Uganda & ALL & 2008 & 85.97 & 70.44 & 104.62 & RW2 \\ 
  Uganda & ALL & 2008 & 85.90 & 78.30 & 94.00 & UN \\ 
  Uganda & ALL & 2009 & 92.13 & 87.12 & 97.14 & IHME \\ 
  Uganda & ALL & 2009 & 80.10 & 64.31 & 99.59 & RW2 \\ 
  Uganda & ALL & 2009 & 80.10 & 72.20 & 88.90 & UN \\ 
  Uganda & ALL & 2010 & 88.77 & 83.45 & 94.47 & IHME \\ 
  Uganda & ALL & 2010 & 75.08 & 59.22 & 95.79 & RW2 \\ 
  Uganda & ALL & 2010 & 75.20 & 66.30 & 85.10 & UN \\ 
  Uganda & ALL & 2011 & 84.67 & 79.15 & 90.89 & IHME \\ 
  Uganda & ALL & 2011 & 70.37 & 55.52 & 89.24 & RW2 \\ 
  Uganda & ALL & 2011 & 69.80 & 59.80 & 81.30 & UN \\ 
  Uganda & ALL & 2012 & 79.97 & 74.11 & 86.63 & IHME \\ 
  Uganda & ALL & 2012 & 65.98 & 51.93 & 83.64 & RW2 \\ 
  Uganda & ALL & 2012 & 64.10 & 53.20 & 77.30 & UN \\ 
  Uganda & ALL & 2013 & 76.14 & 69.99 & 83.44 & IHME \\ 
  Uganda & ALL & 2013 & 61.93 & 46.89 & 81.31 & RW2 \\ 
  Uganda & ALL & 2013 & 60.30 & 48.40 & 75.80 & UN \\ 
  Uganda & ALL & 2014 & 72.64 & 66.31 & 80.34 & IHME \\ 
  Uganda & ALL & 2014 & 58.04 & 39.88 & 83.88 & RW2 \\ 
  Uganda & ALL & 2014 & 56.90 & 44.10 & 74.30 & UN \\ 
  Uganda & ALL & 2015 & 69.35 & 62.82 & 77.42 & IHME \\ 
  Uganda & ALL & 2015 & 54.28 & 31.54 & 92.38 & RW2 \\ 
  Uganda & ALL & 2015 & 54.60 & 40.60 & 74.20 & UN \\ 
  Uganda & ALL & 2016 & 50.97 & 24.85 & 103.19 & RW2 \\ 
  Uganda & ALL & 2017 & 47.61 & 18.87 & 117.47 & RW2 \\ 
  Uganda & ALL & 2018 & 44.56 & 14.17 & 136.90 & RW2 \\ 
  Uganda & ALL & 2019 & 41.63 & 10.04 & 158.39 & RW2 \\ 
  Uganda & ALL & 80-84 & 201.20 & 210.87 & 191.86 & HT-Direct \\ 
  Uganda & ALL & 85-89 & 172.06 & 179.68 & 164.70 & HT-Direct \\ 
  Uganda & ALL & 90-94 & 161.67 & 168.81 & 154.77 & HT-Direct \\ 
  Uganda & ALL & 95-99 & 155.74 & 163.26 & 148.50 & HT-Direct \\ 
  Uganda & ALL & 00-04 & 133.56 & 140.53 & 126.89 & HT-Direct \\ 
  Uganda & ALL & 05-09 & 102.60 & 110.35 & 95.34 & HT-Direct \\ 
  Uganda & ALL & 10-14 & 89.18 & 108.33 & 73.14 & HT-Direct \\ 
  Uganda & ALL & 15-19 & 47.60 & 19.15 & 115.65 & RW2 \\ 
  Uganda & CENTRAL & 1980 & 216.04 & 170.00 & 271.63 & RW2 \\ 
  Uganda & CENTRAL & 1981 & 210.42 & 177.77 & 247.63 & RW2 \\ 
  Uganda & CENTRAL & 1982 & 204.95 & 176.19 & 237.36 & RW2 \\ 
  Uganda & CENTRAL & 1983 & 199.53 & 168.83 & 234.51 & RW2 \\ 
  Uganda & CENTRAL & 1984 & 194.35 & 162.10 & 231.36 & RW2 \\ 
  Uganda & CENTRAL & 1985 & 189.07 & 159.59 & 222.43 & RW2 \\ 
  Uganda & CENTRAL & 1986 & 184.20 & 157.89 & 213.38 & RW2 \\ 
  Uganda & CENTRAL & 1987 & 179.27 & 154.58 & 207.13 & RW2 \\ 
  Uganda & CENTRAL & 1988 & 174.76 & 149.35 & 202.95 & RW2 \\ 
  Uganda & CENTRAL & 1989 & 170.43 & 144.31 & 200.27 & RW2 \\ 
  Uganda & CENTRAL & 1990 & 166.32 & 140.77 & 195.25 & RW2 \\ 
  Uganda & CENTRAL & 1991 & 162.60 & 139.02 & 189.39 & RW2 \\ 
  Uganda & CENTRAL & 1992 & 159.19 & 136.63 & 184.37 & RW2 \\ 
  Uganda & CENTRAL & 1993 & 156.09 & 133.31 & 181.76 & RW2 \\ 
  Uganda & CENTRAL & 1994 & 153.18 & 129.04 & 180.73 & RW2 \\ 
  Uganda & CENTRAL & 1995 & 150.30 & 126.75 & 177.37 & RW2 \\ 
  Uganda & CENTRAL & 1996 & 147.26 & 125.39 & 172.31 & RW2 \\ 
  Uganda & CENTRAL & 1997 & 143.82 & 123.07 & 167.26 & RW2 \\ 
  Uganda & CENTRAL & 1998 & 139.91 & 118.88 & 163.68 & RW2 \\ 
  Uganda & CENTRAL & 1999 & 135.42 & 113.78 & 160.54 & RW2 \\ 
  Uganda & CENTRAL & 2000 & 130.25 & 109.31 & 154.53 & RW2 \\ 
  Uganda & CENTRAL & 2001 & 124.36 & 105.78 & 146.25 & RW2 \\ 
  Uganda & CENTRAL & 2002 & 117.73 & 100.25 & 138.02 & RW2 \\ 
  Uganda & CENTRAL & 2003 & 110.43 & 93.20 & 130.59 & RW2 \\ 
  Uganda & CENTRAL & 2004 & 102.88 & 85.62 & 123.64 & RW2 \\ 
  Uganda & CENTRAL & 2005 & 94.96 & 78.24 & 114.50 & RW2 \\ 
  Uganda & CENTRAL & 2006 & 87.63 & 72.68 & 105.12 & RW2 \\ 
  Uganda & CENTRAL & 2007 & 80.71 & 67.20 & 97.18 & RW2 \\ 
  Uganda & CENTRAL & 2008 & 74.38 & 60.89 & 90.41 & RW2 \\ 
  Uganda & CENTRAL & 2009 & 68.52 & 54.73 & 85.41 & RW2 \\ 
  Uganda & CENTRAL & 2010 & 63.42 & 49.14 & 81.24 & RW2 \\ 
  Uganda & CENTRAL & 2011 & 58.54 & 44.58 & 75.92 & RW2 \\ 
  Uganda & CENTRAL & 2012 & 54.12 & 40.18 & 71.31 & RW2 \\ 
  Uganda & CENTRAL & 2013 & 50.01 & 34.90 & 68.74 & RW2 \\ 
  Uganda & CENTRAL & 2014 & 46.12 & 29.22 & 69.59 & RW2 \\ 
  Uganda & CENTRAL & 2015 & 42.55 & 23.26 & 74.08 & RW2 \\ 
  Uganda & CENTRAL & 2016 & 39.20 & 18.24 & 79.56 & RW2 \\ 
  Uganda & CENTRAL & 2017 & 36.07 & 13.92 & 87.75 & RW2 \\ 
  Uganda & CENTRAL & 2018 & 33.24 & 10.36 & 98.99 & RW2 \\ 
  Uganda & CENTRAL & 2019 & 30.54 & 7.61 & 113.25 & RW2 \\ 
  Uganda & EASTERN & 1980 & 225.68 & 175.61 & 283.32 & RW2 \\ 
  Uganda & EASTERN & 1981 & 222.61 & 187.02 & 261.60 & RW2 \\ 
  Uganda & EASTERN & 1982 & 219.43 & 188.12 & 253.95 & RW2 \\ 
  Uganda & EASTERN & 1983 & 215.95 & 183.10 & 253.19 & RW2 \\ 
  Uganda & EASTERN & 1984 & 212.75 & 178.27 & 251.70 & RW2 \\ 
  Uganda & EASTERN & 1985 & 209.19 & 177.40 & 245.25 & RW2 \\ 
  Uganda & EASTERN & 1986 & 205.39 & 176.59 & 237.80 & RW2 \\ 
  Uganda & EASTERN & 1987 & 201.33 & 173.92 & 232.24 & RW2 \\ 
  Uganda & EASTERN & 1988 & 196.80 & 168.90 & 228.20 & RW2 \\ 
  Uganda & EASTERN & 1989 & 192.03 & 163.16 & 225.26 & RW2 \\ 
  Uganda & EASTERN & 1990 & 186.72 & 158.32 & 218.91 & RW2 \\ 
  Uganda & EASTERN & 1991 & 181.33 & 155.69 & 210.81 & RW2 \\ 
  Uganda & EASTERN & 1992 & 175.87 & 151.70 & 203.41 & RW2 \\ 
  Uganda & EASTERN & 1993 & 170.16 & 145.59 & 198.19 & RW2 \\ 
  Uganda & EASTERN & 1994 & 164.41 & 139.12 & 193.25 & RW2 \\ 
  Uganda & EASTERN & 1995 & 158.53 & 134.62 & 186.32 & RW2 \\ 
  Uganda & EASTERN & 1996 & 152.47 & 130.42 & 177.36 & RW2 \\ 
  Uganda & EASTERN & 1997 & 146.22 & 125.57 & 169.51 & RW2 \\ 
  Uganda & EASTERN & 1998 & 139.85 & 118.97 & 163.27 & RW2 \\ 
  Uganda & EASTERN & 1999 & 133.12 & 111.35 & 157.56 & RW2 \\ 
  Uganda & EASTERN & 2000 & 126.58 & 105.88 & 149.79 & RW2 \\ 
  Uganda & EASTERN & 2001 & 119.66 & 101.08 & 140.48 & RW2 \\ 
  Uganda & EASTERN & 2002 & 112.74 & 95.84 & 131.91 & RW2 \\ 
  Uganda & EASTERN & 2003 & 105.72 & 89.13 & 124.57 & RW2 \\ 
  Uganda & EASTERN & 2004 & 98.96 & 82.45 & 118.57 & RW2 \\ 
  Uganda & EASTERN & 2005 & 92.17 & 76.34 & 110.51 & RW2 \\ 
  Uganda & EASTERN & 2006 & 86.26 & 72.01 & 102.67 & RW2 \\ 
  Uganda & EASTERN & 2007 & 80.94 & 67.92 & 96.11 & RW2 \\ 
  Uganda & EASTERN & 2008 & 76.16 & 63.13 & 91.95 & RW2 \\ 
  Uganda & EASTERN & 2009 & 71.98 & 58.02 & 89.10 & RW2 \\ 
  Uganda & EASTERN & 2010 & 68.36 & 53.47 & 87.59 & RW2 \\ 
  Uganda & EASTERN & 2011 & 64.95 & 50.35 & 84.54 & RW2 \\ 
  Uganda & EASTERN & 2012 & 61.75 & 46.94 & 82.06 & RW2 \\ 
  Uganda & EASTERN & 2013 & 58.72 & 42.50 & 82.22 & RW2 \\ 
  Uganda & EASTERN & 2014 & 55.85 & 36.88 & 85.65 & RW2 \\ 
  Uganda & EASTERN & 2015 & 53.16 & 30.19 & 93.87 & RW2 \\ 
  Uganda & EASTERN & 2016 & 50.52 & 24.40 & 104.15 & RW2 \\ 
  Uganda & EASTERN & 2017 & 47.97 & 19.32 & 117.51 & RW2 \\ 
  Uganda & EASTERN & 2018 & 45.63 & 14.97 & 135.67 & RW2 \\ 
  Uganda & EASTERN & 2019 & 43.39 & 11.24 & 158.25 & RW2 \\ 
  Uganda & NORTHERN & 1980 & 257.95 & 202.02 & 323.27 & RW2 \\ 
  Uganda & NORTHERN & 1981 & 253.98 & 212.78 & 300.41 & RW2 \\ 
  Uganda & NORTHERN & 1982 & 249.73 & 213.39 & 289.85 & RW2 \\ 
  Uganda & NORTHERN & 1983 & 245.58 & 207.68 & 287.87 & RW2 \\ 
  Uganda & NORTHERN & 1984 & 241.41 & 201.62 & 285.80 & RW2 \\ 
  Uganda & NORTHERN & 1985 & 237.39 & 201.25 & 277.75 & RW2 \\ 
  Uganda & NORTHERN & 1986 & 233.36 & 200.64 & 269.07 & RW2 \\ 
  Uganda & NORTHERN & 1987 & 229.33 & 198.72 & 263.06 & RW2 \\ 
  Uganda & NORTHERN & 1988 & 225.38 & 194.10 & 260.08 & RW2 \\ 
  Uganda & NORTHERN & 1989 & 221.50 & 189.14 & 257.70 & RW2 \\ 
  Uganda & NORTHERN & 1990 & 217.61 & 185.89 & 252.88 & RW2 \\ 
  Uganda & NORTHERN & 1991 & 214.21 & 185.11 & 246.45 & RW2 \\ 
  Uganda & NORTHERN & 1992 & 210.86 & 182.91 & 241.52 & RW2 \\ 
  Uganda & NORTHERN & 1993 & 207.56 & 179.23 & 238.84 & RW2 \\ 
  Uganda & NORTHERN & 1994 & 204.49 & 174.45 & 237.39 & RW2 \\ 
  Uganda & NORTHERN & 1995 & 201.22 & 172.06 & 234.20 & RW2 \\ 
  Uganda & NORTHERN & 1996 & 197.42 & 170.41 & 227.54 & RW2 \\ 
  Uganda & NORTHERN & 1997 & 192.88 & 167.37 & 221.52 & RW2 \\ 
  Uganda & NORTHERN & 1998 & 187.52 & 162.04 & 216.16 & RW2 \\ 
  Uganda & NORTHERN & 1999 & 181.22 & 153.94 & 211.00 & RW2 \\ 
  Uganda & NORTHERN & 2000 & 173.96 & 147.80 & 203.21 & RW2 \\ 
  Uganda & NORTHERN & 2001 & 165.64 & 141.98 & 191.90 & RW2 \\ 
  Uganda & NORTHERN & 2002 & 156.37 & 134.78 & 180.69 & RW2 \\ 
  Uganda & NORTHERN & 2003 & 146.41 & 125.38 & 170.88 & RW2 \\ 
  Uganda & NORTHERN & 2004 & 136.29 & 114.88 & 161.36 & RW2 \\ 
  Uganda & NORTHERN & 2005 & 125.94 & 105.17 & 149.70 & RW2 \\ 
  Uganda & NORTHERN & 2006 & 116.60 & 98.14 & 137.88 & RW2 \\ 
  Uganda & NORTHERN & 2007 & 108.02 & 90.69 & 127.80 & RW2 \\ 
  Uganda & NORTHERN & 2008 & 100.41 & 82.93 & 120.83 & RW2 \\ 
  Uganda & NORTHERN & 2009 & 93.53 & 75.24 & 115.50 & RW2 \\ 
  Uganda & NORTHERN & 2010 & 87.71 & 68.74 & 111.69 & RW2 \\ 
  Uganda & NORTHERN & 2011 & 82.15 & 63.42 & 106.09 & RW2 \\ 
  Uganda & NORTHERN & 2012 & 77.07 & 58.24 & 101.44 & RW2 \\ 
  Uganda & NORTHERN & 2013 & 72.37 & 52.07 & 100.13 & RW2 \\ 
  Uganda & NORTHERN & 2014 & 67.92 & 44.62 & 102.15 & RW2 \\ 
  Uganda & NORTHERN & 2015 & 63.66 & 36.21 & 109.77 & RW2 \\ 
  Uganda & NORTHERN & 2016 & 59.57 & 28.89 & 119.43 & RW2 \\ 
  Uganda & NORTHERN & 2017 & 55.90 & 22.54 & 133.25 & RW2 \\ 
  Uganda & NORTHERN & 2018 & 52.21 & 17.09 & 151.27 & RW2 \\ 
  Uganda & NORTHERN & 2019 & 48.93 & 12.73 & 174.92 & RW2 \\ 
  Uganda & WESTERN & 1980 & 188.62 & 146.72 & 239.90 & RW2 \\ 
  Uganda & WESTERN & 1981 & 187.30 & 157.48 & 221.42 & RW2 \\ 
  Uganda & WESTERN & 1982 & 186.03 & 159.31 & 215.50 & RW2 \\ 
  Uganda & WESTERN & 1983 & 184.60 & 155.85 & 217.56 & RW2 \\ 
  Uganda & WESTERN & 1984 & 183.44 & 152.78 & 218.48 & RW2 \\ 
  Uganda & WESTERN & 1985 & 182.35 & 153.16 & 214.56 & RW2 \\ 
  Uganda & WESTERN & 1986 & 181.34 & 154.89 & 211.08 & RW2 \\ 
  Uganda & WESTERN & 1987 & 180.71 & 155.52 & 208.57 & RW2 \\ 
  Uganda & WESTERN & 1988 & 180.25 & 154.11 & 209.41 & RW2 \\ 
  Uganda & WESTERN & 1989 & 179.96 & 152.28 & 211.63 & RW2 \\ 
  Uganda & WESTERN & 1990 & 179.90 & 152.19 & 211.28 & RW2 \\ 
  Uganda & WESTERN & 1991 & 180.35 & 154.21 & 209.03 & RW2 \\ 
  Uganda & WESTERN & 1992 & 180.72 & 155.61 & 208.68 & RW2 \\ 
  Uganda & WESTERN & 1993 & 181.24 & 155.60 & 210.64 & RW2 \\ 
  Uganda & WESTERN & 1994 & 181.66 & 153.94 & 213.37 & RW2 \\ 
  Uganda & WESTERN & 1995 & 181.56 & 153.97 & 213.90 & RW2 \\ 
  Uganda & WESTERN & 1996 & 180.49 & 154.67 & 209.95 & RW2 \\ 
  Uganda & WESTERN & 1997 & 178.30 & 153.54 & 207.12 & RW2 \\ 
  Uganda & WESTERN & 1998 & 174.92 & 149.60 & 203.97 & RW2 \\ 
  Uganda & WESTERN & 1999 & 169.92 & 143.60 & 199.81 & RW2 \\ 
  Uganda & WESTERN & 2000 & 163.71 & 138.55 & 192.99 & RW2 \\ 
  Uganda & WESTERN & 2001 & 156.14 & 133.25 & 182.30 & RW2 \\ 
  Uganda & WESTERN & 2002 & 147.43 & 126.51 & 171.40 & RW2 \\ 
  Uganda & WESTERN & 2003 & 138.01 & 117.44 & 161.70 & RW2 \\ 
  Uganda & WESTERN & 2004 & 128.35 & 107.51 & 152.32 & RW2 \\ 
  Uganda & WESTERN & 2005 & 118.47 & 98.26 & 141.33 & RW2 \\ 
  Uganda & WESTERN & 2006 & 109.50 & 91.76 & 129.99 & RW2 \\ 
  Uganda & WESTERN & 2007 & 101.47 & 85.29 & 120.26 & RW2 \\ 
  Uganda & WESTERN & 2008 & 94.31 & 78.26 & 113.07 & RW2 \\ 
  Uganda & WESTERN & 2009 & 87.89 & 71.44 & 107.41 & RW2 \\ 
  Uganda & WESTERN & 2010 & 82.47 & 65.73 & 103.19 & RW2 \\ 
  Uganda & WESTERN & 2011 & 77.37 & 61.63 & 96.96 & RW2 \\ 
  Uganda & WESTERN & 2012 & 72.68 & 57.38 & 91.70 & RW2 \\ 
  Uganda & WESTERN & 2013 & 68.31 & 51.78 & 89.57 & RW2 \\ 
  Uganda & WESTERN & 2014 & 64.10 & 44.28 & 91.70 & RW2 \\ 
  Uganda & WESTERN & 2015 & 60.24 & 35.87 & 99.62 & RW2 \\ 
  Uganda & WESTERN & 2016 & 56.48 & 28.45 & 109.21 & RW2 \\ 
  Uganda & WESTERN & 2017 & 53.10 & 22.15 & 122.74 & RW2 \\ 
  Uganda & WESTERN & 2018 & 49.67 & 16.87 & 139.30 & RW2 \\ 
  Uganda & WESTERN & 2019 & 46.73 & 12.67 & 160.80 & RW2 \\ 
  Zambia & ALL & 1980 & 160.55 & 151.46 & 170.16 & IHME \\ 
  Zambia & ALL & 1980 & 152.41 & 114.83 & 199.40 & RW2 \\ 
  Zambia & ALL & 1980 & 156.80 & 146.40 & 167.80 & UN \\ 
  Zambia & ALL & 1981 & 159.61 & 149.64 & 169.37 & IHME \\ 
  Zambia & ALL & 1981 & 157.23 & 129.13 & 189.98 & RW2 \\ 
  Zambia & ALL & 1981 & 158.50 & 148.00 & 169.50 & UN \\ 
  Zambia & ALL & 1982 & 158.97 & 150.24 & 167.94 & IHME \\ 
  Zambia & ALL & 1982 & 162.21 & 135.57 & 192.81 & RW2 \\ 
  Zambia & ALL & 1982 & 160.90 & 150.40 & 171.90 & UN \\ 
  Zambia & ALL & 1983 & 160.43 & 151.75 & 169.54 & IHME \\ 
  Zambia & ALL & 1983 & 167.00 & 136.70 & 201.98 & RW2 \\ 
  Zambia & ALL & 1983 & 164.50 & 153.70 & 175.50 & UN \\ 
  Zambia & ALL & 1984 & 163.84 & 155.15 & 172.87 & IHME \\ 
  Zambia & ALL & 1984 & 171.99 & 137.80 & 211.36 & RW2 \\ 
  Zambia & ALL & 1984 & 169.20 & 158.30 & 180.20 & UN \\ 
  Zambia & ALL & 1985 & 168.10 & 159.09 & 177.17 & IHME \\ 
  Zambia & ALL & 1985 & 176.59 & 145.03 & 213.66 & RW2 \\ 
  Zambia & ALL & 1985 & 174.70 & 163.90 & 185.90 & UN \\ 
  Zambia & ALL & 1986 & 171.96 & 162.98 & 181.03 & IHME \\ 
  Zambia & ALL & 1986 & 180.70 & 150.88 & 214.91 & RW2 \\ 
  Zambia & ALL & 1986 & 180.10 & 169.30 & 191.70 & UN \\ 
  Zambia & ALL & 1987 & 172.76 & 163.76 & 181.39 & IHME \\ 
  Zambia & ALL & 1987 & 184.25 & 155.43 & 217.51 & RW2 \\ 
  Zambia & ALL & 1987 & 184.70 & 173.60 & 196.40 & UN \\ 
  Zambia & ALL & 1988 & 170.66 & 161.82 & 179.41 & IHME \\ 
  Zambia & ALL & 1988 & 186.91 & 156.21 & 221.93 & RW2 \\ 
  Zambia & ALL & 1988 & 188.00 & 176.80 & 199.70 & UN \\ 
  Zambia & ALL & 1989 & 169.06 & 160.69 & 177.70 & IHME \\ 
  Zambia & ALL & 1989 & 188.75 & 156.10 & 226.57 & RW2 \\ 
  Zambia & ALL & 1989 & 190.00 & 178.60 & 201.60 & UN \\ 
  Zambia & ALL & 1990 & 168.91 & 161.09 & 177.34 & IHME \\ 
  Zambia & ALL & 1990 & 189.65 & 156.85 & 227.29 & RW2 \\ 
  Zambia & ALL & 1990 & 190.60 & 179.30 & 202.40 & UN \\ 
  Zambia & ALL & 1991 & 169.28 & 160.91 & 177.88 & IHME \\ 
  Zambia & ALL & 1991 & 189.83 & 158.92 & 224.39 & RW2 \\ 
  Zambia & ALL & 1991 & 190.30 & 178.90 & 202.20 & UN \\ 
  Zambia & ALL & 1992 & 169.42 & 161.47 & 178.45 & IHME \\ 
  Zambia & ALL & 1992 & 189.20 & 159.28 & 222.83 & RW2 \\ 
  Zambia & ALL & 1992 & 189.40 & 178.10 & 201.00 & UN \\ 
  Zambia & ALL & 1993 & 168.37 & 159.58 & 177.56 & IHME \\ 
  Zambia & ALL & 1993 & 187.84 & 157.47 & 222.77 & RW2 \\ 
  Zambia & ALL & 1993 & 187.50 & 176.20 & 199.20 & UN \\ 
  Zambia & ALL & 1994 & 165.83 & 156.94 & 175.24 & IHME \\ 
  Zambia & ALL & 1994 & 185.70 & 153.71 & 223.58 & RW2 \\ 
  Zambia & ALL & 1994 & 184.50 & 173.50 & 196.00 & UN \\ 
  Zambia & ALL & 1995 & 161.63 & 153.43 & 170.35 & IHME \\ 
  Zambia & ALL & 1995 & 182.93 & 151.36 & 219.83 & RW2 \\ 
  Zambia & ALL & 1995 & 181.10 & 170.40 & 192.40 & UN \\ 
  Zambia & ALL & 1996 & 157.03 & 148.66 & 165.14 & IHME \\ 
  Zambia & ALL & 1996 & 179.50 & 149.97 & 214.37 & RW2 \\ 
  Zambia & ALL & 1996 & 178.10 & 167.30 & 189.30 & UN \\ 
  Zambia & ALL & 1997 & 153.48 & 144.29 & 162.00 & IHME \\ 
  Zambia & ALL & 1997 & 175.49 & 147.26 & 207.92 & RW2 \\ 
  Zambia & ALL & 1997 & 175.70 & 164.80 & 186.50 & UN \\ 
  Zambia & ALL & 1998 & 152.10 & 143.32 & 161.41 & IHME \\ 
  Zambia & ALL & 1998 & 170.88 & 142.16 & 204.18 & RW2 \\ 
  Zambia & ALL & 1998 & 172.90 & 162.30 & 184.00 & UN \\ 
  Zambia & ALL & 1999 & 150.83 & 142.39 & 160.27 & IHME \\ 
  Zambia & ALL & 1999 & 165.39 & 135.54 & 198.80 & RW2 \\ 
  Zambia & ALL & 1999 & 169.10 & 158.40 & 180.40 & UN \\ 
  Zambia & ALL & 2000 & 147.30 & 137.62 & 156.69 & IHME \\ 
  Zambia & ALL & 2000 & 159.47 & 131.17 & 192.81 & RW2 \\ 
  Zambia & ALL & 2000 & 163.10 & 152.40 & 174.90 & UN \\ 
  Zambia & ALL & 2001 & 140.44 & 131.66 & 150.11 & IHME \\ 
  Zambia & ALL & 2001 & 151.92 & 125.74 & 182.39 & RW2 \\ 
  Zambia & ALL & 2001 & 154.20 & 143.90 & 165.60 & UN \\ 
  Zambia & ALL & 2002 & 131.21 & 122.85 & 140.37 & IHME \\ 
  Zambia & ALL & 2002 & 143.29 & 119.27 & 171.39 & RW2 \\ 
  Zambia & ALL & 2002 & 142.90 & 133.60 & 153.10 & UN \\ 
  Zambia & ALL & 2003 & 122.29 & 114.02 & 130.78 & IHME \\ 
  Zambia & ALL & 2003 & 133.97 & 111.06 & 161.41 & RW2 \\ 
  Zambia & ALL & 2003 & 131.20 & 122.30 & 140.70 & UN \\ 
  Zambia & ALL & 2004 & 114.74 & 106.63 & 122.74 & IHME \\ 
  Zambia & ALL & 2004 & 124.15 & 101.22 & 152.23 & RW2 \\ 
  Zambia & ALL & 2004 & 120.60 & 111.70 & 129.40 & UN \\ 
  Zambia & ALL & 2005 & 108.61 & 100.92 & 117.05 & IHME \\ 
  Zambia & ALL & 2005 & 114.04 & 91.38 & 139.84 & RW2 \\ 
  Zambia & ALL & 2005 & 111.70 & 103.10 & 120.50 & UN \\ 
  Zambia & ALL & 2006 & 103.84 & 96.33 & 112.21 & IHME \\ 
  Zambia & ALL & 2006 & 105.37 & 85.21 & 128.48 & RW2 \\ 
  Zambia & ALL & 2006 & 104.50 & 96.10 & 113.00 & UN \\ 
  Zambia & ALL & 2007 & 98.97 & 91.64 & 106.93 & IHME \\ 
  Zambia & ALL & 2007 & 97.79 & 79.65 & 119.04 & RW2 \\ 
  Zambia & ALL & 2007 & 98.00 & 90.20 & 106.00 & UN \\ 
  Zambia & ALL & 2008 & 94.03 & 86.20 & 102.67 & IHME \\ 
  Zambia & ALL & 2008 & 91.49 & 73.85 & 112.49 & RW2 \\ 
  Zambia & ALL & 2008 & 93.10 & 85.50 & 101.20 & UN \\ 
  Zambia & ALL & 2009 & 89.64 & 81.63 & 98.89 & IHME \\ 
  Zambia & ALL & 2009 & 86.03 & 68.23 & 107.64 & RW2 \\ 
  Zambia & ALL & 2009 & 87.40 & 80.00 & 95.70 & UN \\ 
  Zambia & ALL & 2010 & 83.00 & 75.23 & 90.84 & IHME \\ 
  Zambia & ALL & 2010 & 81.87 & 64.54 & 104.57 & RW2 \\ 
  Zambia & ALL & 2010 & 82.10 & 74.60 & 90.40 & UN \\ 
  Zambia & ALL & 2011 & 77.32 & 69.63 & 85.57 & IHME \\ 
  Zambia & ALL & 2011 & 77.91 & 62.18 & 97.79 & RW2 \\ 
  Zambia & ALL & 2011 & 78.60 & 70.60 & 87.60 & UN \\ 
  Zambia & ALL & 2012 & 73.30 & 65.43 & 82.07 & IHME \\ 
  Zambia & ALL & 2012 & 74.30 & 60.14 & 91.62 & RW2 \\ 
  Zambia & ALL & 2012 & 74.40 & 65.40 & 84.80 & UN \\ 
  Zambia & ALL & 2013 & 69.63 & 60.72 & 79.07 & IHME \\ 
  Zambia & ALL & 2013 & 71.00 & 56.04 & 89.27 & RW2 \\ 
  Zambia & ALL & 2013 & 70.20 & 59.40 & 82.40 & UN \\ 
  Zambia & ALL & 2014 & 67.12 & 57.74 & 78.19 & IHME \\ 
  Zambia & ALL & 2014 & 67.73 & 48.54 & 93.91 & RW2 \\ 
  Zambia & ALL & 2014 & 66.60 & 54.10 & 81.10 & UN \\ 
  Zambia & ALL & 2015 & 65.93 & 55.36 & 78.00 & IHME \\ 
  Zambia & ALL & 2015 & 64.44 & 38.67 & 106.15 & RW2 \\ 
  Zambia & ALL & 2015 & 64.00 & 49.40 & 81.30 & UN \\ 
  Zambia & ALL & 2016 & 61.57 & 30.89 & 121.34 & RW2 \\ 
  Zambia & ALL & 2017 & 58.55 & 23.77 & 141.15 & RW2 \\ 
  Zambia & ALL & 2018 & 55.78 & 18.10 & 167.73 & RW2 \\ 
  Zambia & ALL & 2019 & 53.06 & 12.99 & 197.39 & RW2 \\ 
  Zambia & ALL & 80-84 & 159.63 & 167.50 & 152.06 & HT-Direct \\ 
  Zambia & ALL & 85-89 & 178.94 & 185.94 & 172.16 & HT-Direct \\ 
  Zambia & ALL & 90-94 & 181.18 & 187.93 & 174.62 & HT-Direct \\ 
  Zambia & ALL & 95-99 & 157.06 & 164.07 & 150.29 & HT-Direct \\ 
  Zambia & ALL & 00-04 & 129.42 & 136.26 & 122.88 & HT-Direct \\ 
  Zambia & ALL & 05-09 & 83.31 & 89.82 & 77.24 & HT-Direct \\ 
  Zambia & ALL & 10-14 & 71.80 & 78.63 & 65.53 & HT-Direct \\ 
  Zambia & ALL & 15-19 & 58.53 & 24.18 & 138.98 & RW2 \\ 
  Zambia & CENTRAL & 1980 & 132.52 & 99.58 & 174.88 & RW2 \\ 
  Zambia & CENTRAL & 1981 & 136.96 & 110.93 & 168.46 & RW2 \\ 
  Zambia & CENTRAL & 1982 & 141.57 & 116.88 & 170.38 & RW2 \\ 
  Zambia & CENTRAL & 1983 & 146.04 & 119.33 & 177.29 & RW2 \\ 
  Zambia & CENTRAL & 1984 & 150.56 & 122.07 & 184.42 & RW2 \\ 
  Zambia & CENTRAL & 1985 & 155.09 & 128.19 & 186.36 & RW2 \\ 
  Zambia & CENTRAL & 1986 & 159.11 & 134.44 & 187.46 & RW2 \\ 
  Zambia & CENTRAL & 1987 & 162.69 & 138.68 & 189.81 & RW2 \\ 
  Zambia & CENTRAL & 1988 & 165.74 & 140.78 & 194.03 & RW2 \\ 
  Zambia & CENTRAL & 1989 & 168.12 & 141.34 & 198.61 & RW2 \\ 
  Zambia & CENTRAL & 1990 & 169.74 & 142.81 & 199.75 & RW2 \\ 
  Zambia & CENTRAL & 1991 & 170.76 & 145.45 & 199.11 & RW2 \\ 
  Zambia & CENTRAL & 1992 & 171.00 & 146.99 & 198.55 & RW2 \\ 
  Zambia & CENTRAL & 1993 & 170.66 & 145.99 & 199.21 & RW2 \\ 
  Zambia & CENTRAL & 1994 & 169.41 & 142.52 & 199.55 & RW2 \\ 
  Zambia & CENTRAL & 1995 & 167.68 & 141.20 & 197.98 & RW2 \\ 
  Zambia & CENTRAL & 1996 & 165.10 & 140.44 & 193.59 & RW2 \\ 
  Zambia & CENTRAL & 1997 & 161.59 & 138.21 & 188.30 & RW2 \\ 
  Zambia & CENTRAL & 1998 & 157.51 & 133.44 & 184.67 & RW2 \\ 
  Zambia & CENTRAL & 1999 & 152.34 & 127.23 & 180.55 & RW2 \\ 
  Zambia & CENTRAL & 2000 & 146.58 & 122.40 & 174.54 & RW2 \\ 
  Zambia & CENTRAL & 2001 & 139.73 & 117.44 & 165.71 & RW2 \\ 
  Zambia & CENTRAL & 2002 & 132.34 & 111.45 & 156.54 & RW2 \\ 
  Zambia & CENTRAL & 2003 & 124.22 & 103.44 & 148.37 & RW2 \\ 
  Zambia & CENTRAL & 2004 & 116.14 & 95.16 & 141.24 & RW2 \\ 
  Zambia & CENTRAL & 2005 & 107.82 & 87.21 & 132.11 & RW2 \\ 
  Zambia & CENTRAL & 2006 & 100.37 & 81.29 & 122.99 & RW2 \\ 
  Zambia & CENTRAL & 2007 & 93.70 & 75.69 & 115.02 & RW2 \\ 
  Zambia & CENTRAL & 2008 & 87.99 & 70.13 & 109.37 & RW2 \\ 
  Zambia & CENTRAL & 2009 & 82.91 & 64.76 & 105.19 & RW2 \\ 
  Zambia & CENTRAL & 2010 & 78.73 & 60.97 & 101.54 & RW2 \\ 
  Zambia & CENTRAL & 2011 & 74.61 & 57.86 & 95.90 & RW2 \\ 
  Zambia & CENTRAL & 2012 & 70.95 & 55.16 & 90.75 & RW2 \\ 
  Zambia & CENTRAL & 2013 & 67.52 & 51.04 & 88.64 & RW2 \\ 
  Zambia & CENTRAL & 2014 & 64.24 & 44.86 & 91.03 & RW2 \\ 
  Zambia & CENTRAL & 2015 & 60.99 & 36.95 & 99.68 & RW2 \\ 
  Zambia & CENTRAL & 2016 & 57.97 & 29.73 & 109.94 & RW2 \\ 
  Zambia & CENTRAL & 2017 & 55.02 & 23.45 & 124.70 & RW2 \\ 
  Zambia & CENTRAL & 2018 & 52.26 & 18.19 & 144.47 & RW2 \\ 
  Zambia & CENTRAL & 2019 & 49.68 & 13.50 & 167.56 & RW2 \\ 
  Zambia & COPPERBELT & 1980 & 116.40 & 86.35 & 154.66 & RW2 \\ 
  Zambia & COPPERBELT & 1981 & 121.63 & 97.94 & 149.75 & RW2 \\ 
  Zambia & COPPERBELT & 1982 & 126.89 & 104.81 & 152.72 & RW2 \\ 
  Zambia & COPPERBELT & 1983 & 132.57 & 108.40 & 160.56 & RW2 \\ 
  Zambia & COPPERBELT & 1984 & 137.98 & 111.88 & 168.14 & RW2 \\ 
  Zambia & COPPERBELT & 1985 & 143.59 & 118.95 & 171.88 & RW2 \\ 
  Zambia & COPPERBELT & 1986 & 148.61 & 125.71 & 174.40 & RW2 \\ 
  Zambia & COPPERBELT & 1987 & 153.20 & 130.91 & 178.43 & RW2 \\ 
  Zambia & COPPERBELT & 1988 & 157.11 & 133.76 & 183.41 & RW2 \\ 
  Zambia & COPPERBELT & 1989 & 160.34 & 135.13 & 189.76 & RW2 \\ 
  Zambia & COPPERBELT & 1990 & 162.65 & 137.12 & 192.08 & RW2 \\ 
  Zambia & COPPERBELT & 1991 & 164.09 & 140.05 & 191.69 & RW2 \\ 
  Zambia & COPPERBELT & 1992 & 164.66 & 141.30 & 191.13 & RW2 \\ 
  Zambia & COPPERBELT & 1993 & 164.44 & 140.20 & 192.17 & RW2 \\ 
  Zambia & COPPERBELT & 1994 & 163.22 & 137.20 & 193.31 & RW2 \\ 
  Zambia & COPPERBELT & 1995 & 161.36 & 135.31 & 191.44 & RW2 \\ 
  Zambia & COPPERBELT & 1996 & 158.44 & 133.99 & 186.74 & RW2 \\ 
  Zambia & COPPERBELT & 1997 & 154.71 & 131.13 & 181.77 & RW2 \\ 
  Zambia & COPPERBELT & 1998 & 150.14 & 126.46 & 177.47 & RW2 \\ 
  Zambia & COPPERBELT & 1999 & 144.70 & 119.83 & 173.30 & RW2 \\ 
  Zambia & COPPERBELT & 2000 & 138.85 & 114.85 & 167.01 & RW2 \\ 
  Zambia & COPPERBELT & 2001 & 131.92 & 109.70 & 157.97 & RW2 \\ 
  Zambia & COPPERBELT & 2002 & 124.36 & 103.61 & 148.80 & RW2 \\ 
  Zambia & COPPERBELT & 2003 & 116.33 & 95.86 & 140.50 & RW2 \\ 
  Zambia & COPPERBELT & 2004 & 108.21 & 87.97 & 132.84 & RW2 \\ 
  Zambia & COPPERBELT & 2005 & 100.08 & 80.18 & 123.39 & RW2 \\ 
  Zambia & COPPERBELT & 2006 & 92.79 & 74.49 & 114.72 & RW2 \\ 
  Zambia & COPPERBELT & 2007 & 86.53 & 69.20 & 107.14 & RW2 \\ 
  Zambia & COPPERBELT & 2008 & 80.95 & 63.85 & 102.12 & RW2 \\ 
  Zambia & COPPERBELT & 2009 & 76.06 & 58.68 & 97.85 & RW2 \\ 
  Zambia & COPPERBELT & 2010 & 72.06 & 54.68 & 94.57 & RW2 \\ 
  Zambia & COPPERBELT & 2011 & 68.41 & 51.78 & 89.60 & RW2 \\ 
  Zambia & COPPERBELT & 2012 & 64.98 & 48.83 & 85.53 & RW2 \\ 
  Zambia & COPPERBELT & 2013 & 61.63 & 44.81 & 83.63 & RW2 \\ 
  Zambia & COPPERBELT & 2014 & 58.64 & 39.40 & 86.38 & RW2 \\ 
  Zambia & COPPERBELT & 2015 & 55.61 & 32.43 & 93.58 & RW2 \\ 
  Zambia & COPPERBELT & 2016 & 52.82 & 26.16 & 104.68 & RW2 \\ 
  Zambia & COPPERBELT & 2017 & 50.26 & 20.65 & 118.39 & RW2 \\ 
  Zambia & COPPERBELT & 2018 & 47.72 & 15.90 & 136.17 & RW2 \\ 
  Zambia & COPPERBELT & 2019 & 45.08 & 11.86 & 161.04 & RW2 \\ 
  Zambia & EASTERN & 1980 & 217.78 & 169.17 & 276.27 & RW2 \\ 
  Zambia & EASTERN & 1981 & 221.37 & 185.81 & 262.09 & RW2 \\ 
  Zambia & EASTERN & 1982 & 224.84 & 192.10 & 261.50 & RW2 \\ 
  Zambia & EASTERN & 1983 & 228.16 & 192.37 & 268.28 & RW2 \\ 
  Zambia & EASTERN & 1984 & 231.30 & 193.54 & 273.40 & RW2 \\ 
  Zambia & EASTERN & 1985 & 234.27 & 199.02 & 273.17 & RW2 \\ 
  Zambia & EASTERN & 1986 & 236.60 & 204.64 & 271.31 & RW2 \\ 
  Zambia & EASTERN & 1987 & 238.22 & 208.27 & 270.72 & RW2 \\ 
  Zambia & EASTERN & 1988 & 239.02 & 207.42 & 273.40 & RW2 \\ 
  Zambia & EASTERN & 1989 & 238.98 & 205.22 & 276.24 & RW2 \\ 
  Zambia & EASTERN & 1990 & 237.96 & 204.61 & 274.87 & RW2 \\ 
  Zambia & EASTERN & 1991 & 236.45 & 205.54 & 270.64 & RW2 \\ 
  Zambia & EASTERN & 1992 & 234.19 & 204.27 & 266.86 & RW2 \\ 
  Zambia & EASTERN & 1993 & 231.17 & 200.65 & 264.71 & RW2 \\ 
  Zambia & EASTERN & 1994 & 227.51 & 194.69 & 264.01 & RW2 \\ 
  Zambia & EASTERN & 1995 & 223.33 & 190.60 & 259.69 & RW2 \\ 
  Zambia & EASTERN & 1996 & 218.55 & 187.79 & 252.36 & RW2 \\ 
  Zambia & EASTERN & 1997 & 213.14 & 183.92 & 244.93 & RW2 \\ 
  Zambia & EASTERN & 1998 & 207.03 & 177.63 & 239.91 & RW2 \\ 
  Zambia & EASTERN & 1999 & 200.22 & 169.40 & 234.85 & RW2 \\ 
  Zambia & EASTERN & 2000 & 192.88 & 162.52 & 226.59 & RW2 \\ 
  Zambia & EASTERN & 2001 & 184.52 & 157.23 & 215.20 & RW2 \\ 
  Zambia & EASTERN & 2002 & 175.47 & 150.04 & 204.20 & RW2 \\ 
  Zambia & EASTERN & 2003 & 165.72 & 140.83 & 194.57 & RW2 \\ 
  Zambia & EASTERN & 2004 & 155.95 & 130.67 & 185.55 & RW2 \\ 
  Zambia & EASTERN & 2005 & 145.94 & 120.56 & 174.91 & RW2 \\ 
  Zambia & EASTERN & 2006 & 137.02 & 113.74 & 163.87 & RW2 \\ 
  Zambia & EASTERN & 2007 & 128.95 & 107.11 & 154.29 & RW2 \\ 
  Zambia & EASTERN & 2008 & 121.95 & 100.15 & 147.58 & RW2 \\ 
  Zambia & EASTERN & 2009 & 115.71 & 93.43 & 142.42 & RW2 \\ 
  Zambia & EASTERN & 2010 & 110.72 & 87.88 & 138.91 & RW2 \\ 
  Zambia & EASTERN & 2011 & 105.69 & 84.55 & 132.22 & RW2 \\ 
  Zambia & EASTERN & 2012 & 101.23 & 81.09 & 126.38 & RW2 \\ 
  Zambia & EASTERN & 2013 & 96.90 & 75.22 & 124.75 & RW2 \\ 
  Zambia & EASTERN & 2014 & 92.81 & 66.27 & 128.84 & RW2 \\ 
  Zambia & EASTERN & 2015 & 88.80 & 54.81 & 142.34 & RW2 \\ 
  Zambia & EASTERN & 2016 & 84.93 & 44.77 & 158.12 & RW2 \\ 
  Zambia & EASTERN & 2017 & 81.30 & 35.41 & 180.19 & RW2 \\ 
  Zambia & EASTERN & 2018 & 77.54 & 27.06 & 206.58 & RW2 \\ 
  Zambia & EASTERN & 2019 & 74.05 & 20.78 & 238.61 & RW2 \\ 
  Zambia & LUAPULA & 1980 & 204.71 & 157.07 & 261.12 & RW2 \\ 
  Zambia & LUAPULA & 1981 & 211.01 & 175.13 & 251.21 & RW2 \\ 
  Zambia & LUAPULA & 1982 & 217.43 & 184.88 & 254.15 & RW2 \\ 
  Zambia & LUAPULA & 1983 & 223.95 & 188.50 & 264.12 & RW2 \\ 
  Zambia & LUAPULA & 1984 & 230.32 & 191.46 & 273.21 & RW2 \\ 
  Zambia & LUAPULA & 1985 & 236.34 & 200.38 & 277.10 & RW2 \\ 
  Zambia & LUAPULA & 1986 & 241.53 & 208.06 & 278.60 & RW2 \\ 
  Zambia & LUAPULA & 1987 & 245.78 & 213.64 & 281.35 & RW2 \\ 
  Zambia & LUAPULA & 1988 & 248.89 & 214.71 & 286.37 & RW2 \\ 
  Zambia & LUAPULA & 1989 & 250.55 & 214.54 & 291.52 & RW2 \\ 
  Zambia & LUAPULA & 1990 & 251.20 & 215.10 & 290.77 & RW2 \\ 
  Zambia & LUAPULA & 1991 & 250.68 & 216.68 & 288.08 & RW2 \\ 
  Zambia & LUAPULA & 1992 & 249.16 & 216.84 & 284.49 & RW2 \\ 
  Zambia & LUAPULA & 1993 & 246.38 & 212.72 & 283.75 & RW2 \\ 
  Zambia & LUAPULA & 1994 & 242.55 & 206.92 & 282.21 & RW2 \\ 
  Zambia & LUAPULA & 1995 & 237.62 & 202.03 & 277.32 & RW2 \\ 
  Zambia & LUAPULA & 1996 & 231.57 & 198.94 & 268.82 & RW2 \\ 
  Zambia & LUAPULA & 1997 & 224.31 & 192.51 & 259.24 & RW2 \\ 
  Zambia & LUAPULA & 1998 & 215.59 & 183.65 & 251.26 & RW2 \\ 
  Zambia & LUAPULA & 1999 & 205.93 & 172.56 & 242.58 & RW2 \\ 
  Zambia & LUAPULA & 2000 & 195.48 & 163.08 & 232.28 & RW2 \\ 
  Zambia & LUAPULA & 2001 & 183.86 & 154.13 & 217.68 & RW2 \\ 
  Zambia & LUAPULA & 2002 & 171.80 & 143.99 & 203.65 & RW2 \\ 
  Zambia & LUAPULA & 2003 & 159.46 & 132.09 & 190.60 & RW2 \\ 
  Zambia & LUAPULA & 2004 & 147.16 & 119.74 & 179.39 & RW2 \\ 
  Zambia & LUAPULA & 2005 & 135.14 & 108.52 & 166.38 & RW2 \\ 
  Zambia & LUAPULA & 2006 & 124.40 & 99.66 & 153.51 & RW2 \\ 
  Zambia & LUAPULA & 2007 & 114.88 & 91.65 & 143.06 & RW2 \\ 
  Zambia & LUAPULA & 2008 & 106.38 & 83.52 & 134.79 & RW2 \\ 
  Zambia & LUAPULA & 2009 & 99.10 & 75.65 & 128.35 & RW2 \\ 
  Zambia & LUAPULA & 2010 & 92.85 & 69.70 & 123.08 & RW2 \\ 
  Zambia & LUAPULA & 2011 & 87.04 & 64.75 & 115.91 & RW2 \\ 
  Zambia & LUAPULA & 2012 & 81.65 & 60.18 & 110.49 & RW2 \\ 
  Zambia & LUAPULA & 2013 & 76.60 & 54.08 & 107.25 & RW2 \\ 
  Zambia & LUAPULA & 2014 & 71.85 & 46.85 & 109.30 & RW2 \\ 
  Zambia & LUAPULA & 2015 & 67.42 & 37.81 & 117.85 & RW2 \\ 
  Zambia & LUAPULA & 2016 & 63.03 & 30.45 & 127.86 & RW2 \\ 
  Zambia & LUAPULA & 2017 & 59.00 & 23.33 & 142.88 & RW2 \\ 
  Zambia & LUAPULA & 2018 & 55.34 & 17.74 & 161.07 & RW2 \\ 
  Zambia & LUAPULA & 2019 & 51.84 & 13.08 & 183.53 & RW2 \\ 
  Zambia & LUSAKA & 1980 & 113.80 & 85.12 & 150.49 & RW2 \\ 
  Zambia & LUSAKA & 1981 & 118.33 & 96.09 & 144.87 & RW2 \\ 
  Zambia & LUSAKA & 1982 & 122.93 & 101.61 & 147.33 & RW2 \\ 
  Zambia & LUSAKA & 1983 & 127.71 & 104.70 & 154.63 & RW2 \\ 
  Zambia & LUSAKA & 1984 & 132.48 & 107.57 & 161.76 & RW2 \\ 
  Zambia & LUSAKA & 1985 & 137.33 & 113.75 & 164.93 & RW2 \\ 
  Zambia & LUSAKA & 1986 & 141.78 & 119.68 & 166.69 & RW2 \\ 
  Zambia & LUSAKA & 1987 & 145.87 & 124.50 & 170.17 & RW2 \\ 
  Zambia & LUSAKA & 1988 & 149.46 & 127.00 & 175.06 & RW2 \\ 
  Zambia & LUSAKA & 1989 & 152.50 & 128.08 & 180.49 & RW2 \\ 
  Zambia & LUSAKA & 1990 & 154.85 & 130.23 & 183.06 & RW2 \\ 
  Zambia & LUSAKA & 1991 & 156.60 & 133.41 & 183.62 & RW2 \\ 
  Zambia & LUSAKA & 1992 & 157.54 & 134.97 & 183.01 & RW2 \\ 
  Zambia & LUSAKA & 1993 & 157.71 & 134.23 & 184.46 & RW2 \\ 
  Zambia & LUSAKA & 1994 & 157.13 & 131.82 & 186.32 & RW2 \\ 
  Zambia & LUSAKA & 1995 & 155.93 & 130.66 & 185.25 & RW2 \\ 
  Zambia & LUSAKA & 1996 & 153.87 & 130.00 & 181.17 & RW2 \\ 
  Zambia & LUSAKA & 1997 & 151.24 & 128.35 & 177.08 & RW2 \\ 
  Zambia & LUSAKA & 1998 & 147.70 & 124.51 & 174.16 & RW2 \\ 
  Zambia & LUSAKA & 1999 & 143.52 & 119.05 & 171.97 & RW2 \\ 
  Zambia & LUSAKA & 2000 & 138.72 & 114.86 & 166.52 & RW2 \\ 
  Zambia & LUSAKA & 2001 & 132.87 & 110.95 & 158.01 & RW2 \\ 
  Zambia & LUSAKA & 2002 & 126.34 & 105.79 & 150.20 & RW2 \\ 
  Zambia & LUSAKA & 2003 & 119.22 & 99.09 & 142.92 & RW2 \\ 
  Zambia & LUSAKA & 2004 & 112.01 & 91.56 & 136.55 & RW2 \\ 
  Zambia & LUSAKA & 2005 & 104.44 & 84.02 & 128.21 & RW2 \\ 
  Zambia & LUSAKA & 2006 & 97.84 & 78.89 & 120.25 & RW2 \\ 
  Zambia & LUSAKA & 2007 & 91.73 & 73.68 & 113.40 & RW2 \\ 
  Zambia & LUSAKA & 2008 & 86.44 & 68.65 & 108.03 & RW2 \\ 
  Zambia & LUSAKA & 2009 & 82.04 & 63.93 & 104.36 & RW2 \\ 
  Zambia & LUSAKA & 2010 & 78.26 & 60.05 & 101.07 & RW2 \\ 
  Zambia & LUSAKA & 2011 & 74.70 & 57.78 & 96.23 & RW2 \\ 
  Zambia & LUSAKA & 2012 & 71.43 & 55.14 & 92.05 & RW2 \\ 
  Zambia & LUSAKA & 2013 & 68.38 & 51.14 & 90.22 & RW2 \\ 
  Zambia & LUSAKA & 2014 & 65.50 & 45.18 & 93.29 & RW2 \\ 
  Zambia & LUSAKA & 2015 & 62.56 & 37.65 & 102.86 & RW2 \\ 
  Zambia & LUSAKA & 2016 & 59.76 & 30.44 & 114.20 & RW2 \\ 
  Zambia & LUSAKA & 2017 & 57.24 & 24.15 & 129.99 & RW2 \\ 
  Zambia & LUSAKA & 2018 & 54.39 & 18.75 & 151.12 & RW2 \\ 
  Zambia & LUSAKA & 2019 & 51.98 & 14.16 & 176.26 & RW2 \\ 
  Zambia & NORTH-WESTERN & 1980 & 144.22 & 109.76 & 187.54 & RW2 \\ 
  Zambia & NORTH-WESTERN & 1981 & 147.22 & 121.39 & 177.83 & RW2 \\ 
  Zambia & NORTH-WESTERN & 1982 & 150.23 & 126.00 & 178.28 & RW2 \\ 
  Zambia & NORTH-WESTERN & 1983 & 153.10 & 126.87 & 183.65 & RW2 \\ 
  Zambia & NORTH-WESTERN & 1984 & 155.93 & 127.79 & 188.91 & RW2 \\ 
  Zambia & NORTH-WESTERN & 1985 & 158.80 & 132.18 & 189.04 & RW2 \\ 
  Zambia & NORTH-WESTERN & 1986 & 160.94 & 136.77 & 188.21 & RW2 \\ 
  Zambia & NORTH-WESTERN & 1987 & 162.66 & 139.38 & 188.73 & RW2 \\ 
  Zambia & NORTH-WESTERN & 1988 & 163.87 & 139.57 & 191.32 & RW2 \\ 
  Zambia & NORTH-WESTERN & 1989 & 164.40 & 138.38 & 193.88 & RW2 \\ 
  Zambia & NORTH-WESTERN & 1990 & 164.14 & 138.88 & 193.46 & RW2 \\ 
  Zambia & NORTH-WESTERN & 1991 & 163.44 & 139.48 & 190.54 & RW2 \\ 
  Zambia & NORTH-WESTERN & 1992 & 161.98 & 139.22 & 187.56 & RW2 \\ 
  Zambia & NORTH-WESTERN & 1993 & 159.95 & 136.73 & 186.42 & RW2 \\ 
  Zambia & NORTH-WESTERN & 1994 & 157.23 & 132.46 & 185.51 & RW2 \\ 
  Zambia & NORTH-WESTERN & 1995 & 153.95 & 129.71 & 182.07 & RW2 \\ 
  Zambia & NORTH-WESTERN & 1996 & 150.04 & 127.35 & 176.03 & RW2 \\ 
  Zambia & NORTH-WESTERN & 1997 & 145.59 & 124.02 & 170.13 & RW2 \\ 
  Zambia & NORTH-WESTERN & 1998 & 140.52 & 118.35 & 165.42 & RW2 \\ 
  Zambia & NORTH-WESTERN & 1999 & 134.71 & 111.94 & 160.54 & RW2 \\ 
  Zambia & NORTH-WESTERN & 2000 & 128.57 & 106.76 & 153.97 & RW2 \\ 
  Zambia & NORTH-WESTERN & 2001 & 121.55 & 101.36 & 144.97 & RW2 \\ 
  Zambia & NORTH-WESTERN & 2002 & 114.14 & 95.51 & 136.02 & RW2 \\ 
  Zambia & NORTH-WESTERN & 2003 & 106.27 & 87.83 & 128.03 & RW2 \\ 
  Zambia & NORTH-WESTERN & 2004 & 98.48 & 80.16 & 120.76 & RW2 \\ 
  Zambia & NORTH-WESTERN & 2005 & 90.59 & 72.58 & 112.04 & RW2 \\ 
  Zambia & NORTH-WESTERN & 2006 & 83.74 & 67.26 & 103.64 & RW2 \\ 
  Zambia & NORTH-WESTERN & 2007 & 77.61 & 62.09 & 96.52 & RW2 \\ 
  Zambia & NORTH-WESTERN & 2008 & 72.29 & 57.09 & 90.97 & RW2 \\ 
  Zambia & NORTH-WESTERN & 2009 & 67.69 & 52.47 & 86.64 & RW2 \\ 
  Zambia & NORTH-WESTERN & 2010 & 63.78 & 48.93 & 83.04 & RW2 \\ 
  Zambia & NORTH-WESTERN & 2011 & 60.22 & 46.33 & 78.11 & RW2 \\ 
  Zambia & NORTH-WESTERN & 2012 & 56.88 & 43.96 & 73.64 & RW2 \\ 
  Zambia & NORTH-WESTERN & 2013 & 53.79 & 40.43 & 71.73 & RW2 \\ 
  Zambia & NORTH-WESTERN & 2014 & 50.85 & 35.33 & 73.06 & RW2 \\ 
  Zambia & NORTH-WESTERN & 2015 & 48.08 & 28.78 & 79.72 & RW2 \\ 
  Zambia & NORTH-WESTERN & 2016 & 45.37 & 23.01 & 88.21 & RW2 \\ 
  Zambia & NORTH-WESTERN & 2017 & 42.90 & 17.97 & 101.52 & RW2 \\ 
  Zambia & NORTH-WESTERN & 2018 & 40.54 & 13.72 & 116.05 & RW2 \\ 
  Zambia & NORTH-WESTERN & 2019 & 38.36 & 10.42 & 134.18 & RW2 \\ 
  Zambia & NORTHERN & 1980 & 179.30 & 138.13 & 229.46 & RW2 \\ 
  Zambia & NORTHERN & 1981 & 185.02 & 154.04 & 220.34 & RW2 \\ 
  Zambia & NORTHERN & 1982 & 190.76 & 162.05 & 222.89 & RW2 \\ 
  Zambia & NORTHERN & 1983 & 196.50 & 164.95 & 232.61 & RW2 \\ 
  Zambia & NORTHERN & 1984 & 202.10 & 167.78 & 241.16 & RW2 \\ 
  Zambia & NORTHERN & 1985 & 207.69 & 175.43 & 244.27 & RW2 \\ 
  Zambia & NORTHERN & 1986 & 212.62 & 182.53 & 245.52 & RW2 \\ 
  Zambia & NORTHERN & 1987 & 216.70 & 187.81 & 248.09 & RW2 \\ 
  Zambia & NORTHERN & 1988 & 220.25 & 189.70 & 253.52 & RW2 \\ 
  Zambia & NORTHERN & 1989 & 222.78 & 190.06 & 259.23 & RW2 \\ 
  Zambia & NORTHERN & 1990 & 224.33 & 191.96 & 260.03 & RW2 \\ 
  Zambia & NORTHERN & 1991 & 225.07 & 194.95 & 258.66 & RW2 \\ 
  Zambia & NORTHERN & 1992 & 225.01 & 195.99 & 256.95 & RW2 \\ 
  Zambia & NORTHERN & 1993 & 224.09 & 194.30 & 257.18 & RW2 \\ 
  Zambia & NORTHERN & 1994 & 222.16 & 190.05 & 257.97 & RW2 \\ 
  Zambia & NORTHERN & 1995 & 219.35 & 187.15 & 255.27 & RW2 \\ 
  Zambia & NORTHERN & 1996 & 215.39 & 185.46 & 248.42 & RW2 \\ 
  Zambia & NORTHERN & 1997 & 210.29 & 181.73 & 242.11 & RW2 \\ 
  Zambia & NORTHERN & 1998 & 204.01 & 175.60 & 236.12 & RW2 \\ 
  Zambia & NORTHERN & 1999 & 196.56 & 166.50 & 230.40 & RW2 \\ 
  Zambia & NORTHERN & 2000 & 188.02 & 158.72 & 221.13 & RW2 \\ 
  Zambia & NORTHERN & 2001 & 178.26 & 151.84 & 208.44 & RW2 \\ 
  Zambia & NORTHERN & 2002 & 167.44 & 143.19 & 195.06 & RW2 \\ 
  Zambia & NORTHERN & 2003 & 156.32 & 132.56 & 184.31 & RW2 \\ 
  Zambia & NORTHERN & 2004 & 144.79 & 120.84 & 173.27 & RW2 \\ 
  Zambia & NORTHERN & 2005 & 133.23 & 109.98 & 159.76 & RW2 \\ 
  Zambia & NORTHERN & 2006 & 122.87 & 101.86 & 146.98 & RW2 \\ 
  Zambia & NORTHERN & 2007 & 113.57 & 94.02 & 136.07 & RW2 \\ 
  Zambia & NORTHERN & 2008 & 105.22 & 86.05 & 127.68 & RW2 \\ 
  Zambia & NORTHERN & 2009 & 97.89 & 78.45 & 121.04 & RW2 \\ 
  Zambia & NORTHERN & 2010 & 91.56 & 72.38 & 115.49 & RW2 \\ 
  Zambia & NORTHERN & 2011 & 85.76 & 67.60 & 107.32 & RW2 \\ 
  Zambia & NORTHERN & 2012 & 80.31 & 63.39 & 100.84 & RW2 \\ 
  Zambia & NORTHERN & 2013 & 75.30 & 57.50 & 97.44 & RW2 \\ 
  Zambia & NORTHERN & 2014 & 70.45 & 49.31 & 99.12 & RW2 \\ 
  Zambia & NORTHERN & 2015 & 65.98 & 39.80 & 107.66 & RW2 \\ 
  Zambia & NORTHERN & 2016 & 61.84 & 31.65 & 117.91 & RW2 \\ 
  Zambia & NORTHERN & 2017 & 57.70 & 24.42 & 130.84 & RW2 \\ 
  Zambia & NORTHERN & 2018 & 53.79 & 18.57 & 148.17 & RW2 \\ 
  Zambia & NORTHERN & 2019 & 50.38 & 13.76 & 172.63 & RW2 \\ 
  Zambia & SOUTHERN & 1980 & 125.95 & 95.05 & 164.97 & RW2 \\ 
  Zambia & SOUTHERN & 1981 & 129.20 & 105.71 & 157.55 & RW2 \\ 
  Zambia & SOUTHERN & 1982 & 132.73 & 110.52 & 158.35 & RW2 \\ 
  Zambia & SOUTHERN & 1983 & 136.14 & 112.27 & 164.42 & RW2 \\ 
  Zambia & SOUTHERN & 1984 & 139.45 & 113.67 & 169.49 & RW2 \\ 
  Zambia & SOUTHERN & 1985 & 142.82 & 118.64 & 170.83 & RW2 \\ 
  Zambia & SOUTHERN & 1986 & 145.74 & 123.57 & 171.18 & RW2 \\ 
  Zambia & SOUTHERN & 1987 & 148.34 & 126.71 & 172.43 & RW2 \\ 
  Zambia & SOUTHERN & 1988 & 150.48 & 128.02 & 176.11 & RW2 \\ 
  Zambia & SOUTHERN & 1989 & 152.02 & 127.61 & 180.16 & RW2 \\ 
  Zambia & SOUTHERN & 1990 & 152.80 & 128.50 & 180.50 & RW2 \\ 
  Zambia & SOUTHERN & 1991 & 153.21 & 130.68 & 178.96 & RW2 \\ 
  Zambia & SOUTHERN & 1992 & 152.92 & 130.77 & 177.65 & RW2 \\ 
  Zambia & SOUTHERN & 1993 & 152.03 & 129.55 & 177.77 & RW2 \\ 
  Zambia & SOUTHERN & 1994 & 150.34 & 126.06 & 178.01 & RW2 \\ 
  Zambia & SOUTHERN & 1995 & 148.29 & 124.10 & 176.07 & RW2 \\ 
  Zambia & SOUTHERN & 1996 & 145.36 & 122.96 & 171.06 & RW2 \\ 
  Zambia & SOUTHERN & 1997 & 141.89 & 120.29 & 166.23 & RW2 \\ 
  Zambia & SOUTHERN & 1998 & 137.73 & 116.01 & 162.65 & RW2 \\ 
  Zambia & SOUTHERN & 1999 & 132.90 & 110.60 & 158.55 & RW2 \\ 
  Zambia & SOUTHERN & 2000 & 127.64 & 105.98 & 153.00 & RW2 \\ 
  Zambia & SOUTHERN & 2001 & 121.50 & 101.73 & 144.45 & RW2 \\ 
  Zambia & SOUTHERN & 2002 & 114.71 & 96.65 & 136.02 & RW2 \\ 
  Zambia & SOUTHERN & 2003 & 107.64 & 89.53 & 129.06 & RW2 \\ 
  Zambia & SOUTHERN & 2004 & 100.50 & 82.59 & 122.30 & RW2 \\ 
  Zambia & SOUTHERN & 2005 & 93.16 & 75.37 & 114.17 & RW2 \\ 
  Zambia & SOUTHERN & 2006 & 86.83 & 70.55 & 106.33 & RW2 \\ 
  Zambia & SOUTHERN & 2007 & 81.15 & 65.82 & 99.66 & RW2 \\ 
  Zambia & SOUTHERN & 2008 & 76.17 & 60.92 & 94.66 & RW2 \\ 
  Zambia & SOUTHERN & 2009 & 71.95 & 56.43 & 91.06 & RW2 \\ 
  Zambia & SOUTHERN & 2010 & 68.35 & 53.03 & 88.13 & RW2 \\ 
  Zambia & SOUTHERN & 2011 & 65.06 & 50.69 & 83.18 & RW2 \\ 
  Zambia & SOUTHERN & 2012 & 61.99 & 48.42 & 79.19 & RW2 \\ 
  Zambia & SOUTHERN & 2013 & 59.17 & 44.85 & 77.66 & RW2 \\ 
  Zambia & SOUTHERN & 2014 & 56.41 & 39.44 & 80.42 & RW2 \\ 
  Zambia & SOUTHERN & 2015 & 53.69 & 32.34 & 88.36 & RW2 \\ 
  Zambia & SOUTHERN & 2016 & 51.15 & 26.16 & 98.21 & RW2 \\ 
  Zambia & SOUTHERN & 2017 & 48.67 & 20.71 & 111.90 & RW2 \\ 
  Zambia & SOUTHERN & 2018 & 46.32 & 15.82 & 130.30 & RW2 \\ 
  Zambia & SOUTHERN & 2019 & 44.19 & 11.94 & 153.32 & RW2 \\ 
  Zambia & WESTERN & 1980 & 203.48 & 157.46 & 258.95 & RW2 \\ 
  Zambia & WESTERN & 1981 & 205.98 & 171.73 & 245.11 & RW2 \\ 
  Zambia & WESTERN & 1982 & 208.66 & 176.79 & 244.44 & RW2 \\ 
  Zambia & WESTERN & 1983 & 210.93 & 176.43 & 250.18 & RW2 \\ 
  Zambia & WESTERN & 1984 & 213.22 & 176.97 & 254.53 & RW2 \\ 
  Zambia & WESTERN & 1985 & 215.42 & 181.30 & 254.02 & RW2 \\ 
  Zambia & WESTERN & 1986 & 216.94 & 185.47 & 251.73 & RW2 \\ 
  Zambia & WESTERN & 1987 & 218.04 & 188.40 & 251.44 & RW2 \\ 
  Zambia & WESTERN & 1988 & 218.35 & 187.14 & 253.32 & RW2 \\ 
  Zambia & WESTERN & 1989 & 217.95 & 184.70 & 255.15 & RW2 \\ 
  Zambia & WESTERN & 1990 & 216.76 & 184.27 & 252.87 & RW2 \\ 
  Zambia & WESTERN & 1991 & 214.93 & 184.69 & 248.78 & RW2 \\ 
  Zambia & WESTERN & 1992 & 212.31 & 183.76 & 244.30 & RW2 \\ 
  Zambia & WESTERN & 1993 & 208.99 & 179.21 & 242.31 & RW2 \\ 
  Zambia & WESTERN & 1994 & 204.91 & 173.85 & 239.74 & RW2 \\ 
  Zambia & WESTERN & 1995 & 200.08 & 169.33 & 235.00 & RW2 \\ 
  Zambia & WESTERN & 1996 & 194.46 & 166.22 & 226.97 & RW2 \\ 
  Zambia & WESTERN & 1997 & 187.87 & 161.04 & 218.55 & RW2 \\ 
  Zambia & WESTERN & 1998 & 180.56 & 153.28 & 211.44 & RW2 \\ 
  Zambia & WESTERN & 1999 & 172.33 & 144.12 & 204.69 & RW2 \\ 
  Zambia & WESTERN & 2000 & 163.30 & 136.01 & 195.20 & RW2 \\ 
  Zambia & WESTERN & 2001 & 153.31 & 128.18 & 182.52 & RW2 \\ 
  Zambia & WESTERN & 2002 & 142.81 & 119.49 & 170.26 & RW2 \\ 
  Zambia & WESTERN & 2003 & 131.84 & 108.84 & 158.75 & RW2 \\ 
  Zambia & WESTERN & 2004 & 121.07 & 98.19 & 148.66 & RW2 \\ 
  Zambia & WESTERN & 2005 & 110.30 & 88.03 & 136.88 & RW2 \\ 
  Zambia & WESTERN & 2006 & 100.97 & 80.33 & 125.54 & RW2 \\ 
  Zambia & WESTERN & 2007 & 92.42 & 73.22 & 116.01 & RW2 \\ 
  Zambia & WESTERN & 2008 & 85.10 & 66.16 & 108.39 & RW2 \\ 
  Zambia & WESTERN & 2009 & 78.57 & 59.71 & 102.43 & RW2 \\ 
  Zambia & WESTERN & 2010 & 73.20 & 54.64 & 97.40 & RW2 \\ 
  Zambia & WESTERN & 2011 & 68.04 & 50.68 & 90.34 & RW2 \\ 
  Zambia & WESTERN & 2012 & 63.39 & 47.05 & 84.92 & RW2 \\ 
  Zambia & WESTERN & 2013 & 59.13 & 42.33 & 81.44 & RW2 \\ 
  Zambia & WESTERN & 2014 & 55.10 & 36.42 & 81.82 & RW2 \\ 
  Zambia & WESTERN & 2015 & 51.28 & 29.57 & 87.01 & RW2 \\ 
  Zambia & WESTERN & 2016 & 47.73 & 23.37 & 94.82 & RW2 \\ 
  Zambia & WESTERN & 2017 & 44.38 & 18.13 & 105.61 & RW2 \\ 
  Zambia & WESTERN & 2018 & 41.15 & 13.85 & 118.66 & RW2 \\ 
  Zambia & WESTERN & 2019 & 38.34 & 10.18 & 136.80 & RW2 \\ 
  Zimbabwe & ALL & 1980 & 89.58 & 85.95 & 93.29 & IHME \\ 
  Zimbabwe & ALL & 1980 & 106.33 & 72.43 & 152.64 & RW2 \\ 
  Zimbabwe & ALL & 1980 & 103.40 & 95.30 & 111.90 & UN \\ 
  Zimbabwe & ALL & 1981 & 86.73 & 83.27 & 90.15 & IHME \\ 
  Zimbabwe & ALL & 1981 & 100.01 & 76.20 & 129.05 & RW2 \\ 
  Zimbabwe & ALL & 1981 & 99.40 & 91.90 & 107.60 & UN \\ 
  Zimbabwe & ALL & 1982 & 83.62 & 80.33 & 86.93 & IHME \\ 
  Zimbabwe & ALL & 1982 & 94.02 & 73.89 & 119.25 & RW2 \\ 
  Zimbabwe & ALL & 1982 & 94.70 & 87.70 & 102.50 & UN \\ 
  Zimbabwe & ALL & 1983 & 80.23 & 77.17 & 83.42 & IHME \\ 
  Zimbabwe & ALL & 1983 & 88.41 & 67.79 & 115.74 & RW2 \\ 
  Zimbabwe & ALL & 1983 & 89.80 & 83.10 & 96.90 & UN \\ 
  Zimbabwe & ALL & 1984 & 76.83 & 74.07 & 79.81 & IHME \\ 
  Zimbabwe & ALL & 1984 & 83.73 & 62.37 & 113.02 & RW2 \\ 
  Zimbabwe & ALL & 1984 & 84.90 & 78.70 & 91.50 & UN \\ 
  Zimbabwe & ALL & 1985 & 73.69 & 71.07 & 76.42 & IHME \\ 
  Zimbabwe & ALL & 1985 & 79.15 & 59.60 & 103.44 & RW2 \\ 
  Zimbabwe & ALL & 1985 & 80.60 & 74.70 & 86.80 & UN \\ 
  Zimbabwe & ALL & 1986 & 70.91 & 68.35 & 73.55 & IHME \\ 
  Zimbabwe & ALL & 1986 & 76.31 & 58.81 & 97.92 & RW2 \\ 
  Zimbabwe & ALL & 1986 & 77.10 & 71.20 & 83.10 & UN \\ 
  Zimbabwe & ALL & 1987 & 68.65 & 66.27 & 71.00 & IHME \\ 
  Zimbabwe & ALL & 1987 & 74.69 & 58.49 & 95.16 & RW2 \\ 
  Zimbabwe & ALL & 1987 & 74.60 & 68.90 & 80.60 & UN \\ 
  Zimbabwe & ALL & 1988 & 66.88 & 64.55 & 69.23 & IHME \\ 
  Zimbabwe & ALL & 1988 & 74.23 & 57.32 & 95.33 & RW2 \\ 
  Zimbabwe & ALL & 1988 & 73.50 & 67.70 & 79.50 & UN \\ 
  Zimbabwe & ALL & 1989 & 65.70 & 63.38 & 68.06 & IHME \\ 
  Zimbabwe & ALL & 1989 & 74.94 & 56.93 & 97.51 & RW2 \\ 
  Zimbabwe & ALL & 1989 & 73.90 & 67.90 & 80.10 & UN \\ 
  Zimbabwe & ALL & 1990 & 65.62 & 63.23 & 67.96 & IHME \\ 
  Zimbabwe & ALL & 1990 & 77.01 & 58.95 & 101.20 & RW2 \\ 
  Zimbabwe & ALL & 1990 & 75.80 & 69.70 & 82.00 & UN \\ 
  Zimbabwe & ALL & 1991 & 65.97 & 63.46 & 68.40 & IHME \\ 
  Zimbabwe & ALL & 1991 & 79.62 & 61.78 & 102.36 & RW2 \\ 
  Zimbabwe & ALL & 1991 & 78.80 & 72.50 & 85.20 & UN \\ 
  Zimbabwe & ALL & 1992 & 66.77 & 64.29 & 69.29 & IHME \\ 
  Zimbabwe & ALL & 1992 & 82.90 & 64.61 & 105.54 & RW2 \\ 
  Zimbabwe & ALL & 1992 & 82.70 & 76.30 & 89.30 & UN \\ 
  Zimbabwe & ALL & 1993 & 67.86 & 65.26 & 70.36 & IHME \\ 
  Zimbabwe & ALL & 1993 & 86.67 & 67.01 & 110.87 & RW2 \\ 
  Zimbabwe & ALL & 1993 & 87.00 & 80.40 & 93.90 & UN \\ 
  Zimbabwe & ALL & 1994 & 68.90 & 66.20 & 71.38 & IHME \\ 
  Zimbabwe & ALL & 1994 & 90.60 & 68.86 & 117.94 & RW2 \\ 
  Zimbabwe & ALL & 1994 & 91.20 & 84.30 & 98.60 & UN \\ 
  Zimbabwe & ALL & 1995 & 71.32 & 68.64 & 73.88 & IHME \\ 
  Zimbabwe & ALL & 1995 & 94.82 & 72.52 & 123.98 & RW2 \\ 
  Zimbabwe & ALL & 1995 & 95.50 & 88.10 & 103.30 & UN \\ 
  Zimbabwe & ALL & 1996 & 73.31 & 70.59 & 76.01 & IHME \\ 
  Zimbabwe & ALL & 1996 & 98.38 & 76.02 & 127.53 & RW2 \\ 
  Zimbabwe & ALL & 1996 & 99.20 & 91.30 & 107.50 & UN \\ 
  Zimbabwe & ALL & 1997 & 74.49 & 71.70 & 77.40 & IHME \\ 
  Zimbabwe & ALL & 1997 & 101.36 & 78.87 & 129.89 & RW2 \\ 
  Zimbabwe & ALL & 1997 & 102.00 & 93.70 & 110.70 & UN \\ 
  Zimbabwe & ALL & 1998 & 75.41 & 72.57 & 78.38 & IHME \\ 
  Zimbabwe & ALL & 1998 & 103.64 & 80.11 & 134.33 & RW2 \\ 
  Zimbabwe & ALL & 1998 & 103.90 & 95.20 & 113.00 & UN \\ 
  Zimbabwe & ALL & 1999 & 76.30 & 73.40 & 79.37 & IHME \\ 
  Zimbabwe & ALL & 1999 & 105.02 & 80.08 & 136.88 & RW2 \\ 
  Zimbabwe & ALL & 1999 & 105.20 & 96.00 & 114.70 & UN \\ 
  Zimbabwe & ALL & 2000 & 77.87 & 75.03 & 80.97 & IHME \\ 
  Zimbabwe & ALL & 2000 & 105.30 & 79.90 & 136.72 & RW2 \\ 
  Zimbabwe & ALL & 2000 & 105.80 & 96.30 & 115.80 & UN \\ 
  Zimbabwe & ALL & 2001 & 79.33 & 76.36 & 82.49 & IHME \\ 
  Zimbabwe & ALL & 2001 & 105.25 & 80.64 & 135.58 & RW2 \\ 
  Zimbabwe & ALL & 2001 & 105.60 & 95.70 & 115.80 & UN \\ 
  Zimbabwe & ALL & 2002 & 81.06 & 78.03 & 84.25 & IHME \\ 
  Zimbabwe & ALL & 2002 & 104.63 & 80.91 & 134.16 & RW2 \\ 
  Zimbabwe & ALL & 2002 & 105.10 & 95.10 & 115.30 & UN \\ 
  Zimbabwe & ALL & 2003 & 82.53 & 79.30 & 85.96 & IHME \\ 
  Zimbabwe & ALL & 2003 & 103.74 & 79.72 & 134.00 & RW2 \\ 
  Zimbabwe & ALL & 2003 & 104.00 & 94.20 & 114.30 & UN \\ 
  Zimbabwe & ALL & 2004 & 83.99 & 80.44 & 87.52 & IHME \\ 
  Zimbabwe & ALL & 2004 & 102.34 & 76.88 & 134.82 & RW2 \\ 
  Zimbabwe & ALL & 2004 & 103.00 & 93.40 & 113.40 & UN \\ 
  Zimbabwe & ALL & 2005 & 83.02 & 79.55 & 86.50 & IHME \\ 
  Zimbabwe & ALL & 2005 & 100.96 & 74.72 & 134.68 & RW2 \\ 
  Zimbabwe & ALL & 2005 & 101.90 & 92.50 & 112.00 & UN \\ 
  Zimbabwe & ALL & 2006 & 81.63 & 78.07 & 85.29 & IHME \\ 
  Zimbabwe & ALL & 2006 & 99.35 & 75.34 & 129.51 & RW2 \\ 
  Zimbabwe & ALL & 2006 & 100.00 & 91.00 & 110.20 & UN \\ 
  Zimbabwe & ALL & 2007 & 81.70 & 77.77 & 85.61 & IHME \\ 
  Zimbabwe & ALL & 2007 & 97.75 & 76.49 & 124.03 & RW2 \\ 
  Zimbabwe & ALL & 2007 & 98.00 & 89.00 & 108.40 & UN \\ 
  Zimbabwe & ALL & 2008 & 81.99 & 77.64 & 86.24 & IHME \\ 
  Zimbabwe & ALL & 2008 & 96.26 & 73.64 & 125.15 & RW2 \\ 
  Zimbabwe & ALL & 2008 & 95.40 & 86.20 & 106.00 & UN \\ 
  Zimbabwe & ALL & 2009 & 81.43 & 76.62 & 86.37 & IHME \\ 
  Zimbabwe & ALL & 2009 & 94.51 & 64.22 & 137.95 & RW2 \\ 
  Zimbabwe & ALL & 2009 & 92.90 & 83.10 & 104.10 & UN \\ 
  Zimbabwe & ALL & 2010 & 76.95 & 71.81 & 82.35 & IHME \\ 
  Zimbabwe & ALL & 2010 & 92.87 & 50.90 & 166.23 & RW2 \\ 
  Zimbabwe & ALL & 2010 & 89.50 & 79.00 & 102.20 & UN \\ 
  Zimbabwe & ALL & 2011 & 71.87 & 66.38 & 77.87 & IHME \\ 
  Zimbabwe & ALL & 2011 & 91.39 & 39.36 & 199.25 & RW2 \\ 
  Zimbabwe & ALL & 2011 & 85.60 & 73.80 & 100.70 & UN \\ 
  Zimbabwe & ALL & 2012 & 66.34 & 60.58 & 72.98 & IHME \\ 
  Zimbabwe & ALL & 2012 & 89.71 & 29.32 & 244.17 & RW2 \\ 
  Zimbabwe & ALL & 2012 & 78.50 & 65.20 & 96.20 & UN \\ 
  Zimbabwe & ALL & 2013 & 63.99 & 57.90 & 71.53 & IHME \\ 
  Zimbabwe & ALL & 2013 & 88.46 & 21.57 & 299.18 & RW2 \\ 
  Zimbabwe & ALL & 2013 & 74.50 & 59.10 & 95.10 & UN \\ 
  Zimbabwe & ALL & 2014 & 62.75 & 55.94 & 71.15 & IHME \\ 
  Zimbabwe & ALL & 2014 & 86.83 & 15.38 & 367.35 & RW2 \\ 
  Zimbabwe & ALL & 2014 & 72.30 & 54.60 & 96.30 & UN \\ 
  Zimbabwe & ALL & 2015 & 54.81 & 48.09 & 63.31 & IHME \\ 
  Zimbabwe & ALL & 2015 & 84.97 & 10.92 & 441.84 & RW2 \\ 
  Zimbabwe & ALL & 2015 & 70.70 & 50.90 & 97.60 & UN \\ 
  Zimbabwe & ALL & 2016 & 84.17 & 7.64 & 533.30 & RW2 \\ 
  Zimbabwe & ALL & 2017 & 82.18 & 5.05 & 622.85 & RW2 \\ 
  Zimbabwe & ALL & 2018 & 80.73 & 3.42 & 712.95 & RW2 \\ 
  Zimbabwe & ALL & 2019 & 79.05 & 2.03 & 779.77 & RW2 \\ 
  Zimbabwe & ALL & 80-84 & 91.40 & 99.43 & 83.96 & HT-Direct \\ 
  Zimbabwe & ALL & 85-89 & 69.33 & 75.28 & 63.82 & HT-Direct \\ 
  Zimbabwe & ALL & 90-94 & 70.25 & 76.09 & 64.83 & HT-Direct \\ 
  Zimbabwe & ALL & 95-99 & 74.96 & 81.04 & 69.29 & HT-Direct \\ 
  Zimbabwe & ALL & 00-04 & 74.80 & 80.79 & 69.22 & HT-Direct \\ 
  Zimbabwe & ALL & 05-09 & 84.10 & 90.87 & 77.79 & HT-Direct \\ 
  Zimbabwe & ALL & 15-19 & 82.18 & 5.10 & 617.28 & RW2 \\ 
  Zimbabwe & BULAWAYO & 1980 & 64.36 & 39.04 & 103.68 & RW2 \\ 
  Zimbabwe & BULAWAYO & 1981 & 60.62 & 40.84 & 89.08 & RW2 \\ 
  Zimbabwe & BULAWAYO & 1982 & 57.09 & 39.62 & 81.16 & RW2 \\ 
  Zimbabwe & BULAWAYO & 1983 & 53.82 & 37.30 & 77.17 & RW2 \\ 
  Zimbabwe & BULAWAYO & 1984 & 51.02 & 35.35 & 73.80 & RW2 \\ 
  Zimbabwe & BULAWAYO & 1985 & 48.45 & 34.01 & 68.19 & RW2 \\ 
  Zimbabwe & BULAWAYO & 1986 & 46.86 & 33.84 & 64.59 & RW2 \\ 
  Zimbabwe & BULAWAYO & 1987 & 46.05 & 33.80 & 62.64 & RW2 \\ 
  Zimbabwe & BULAWAYO & 1988 & 46.03 & 33.64 & 62.62 & RW2 \\ 
  Zimbabwe & BULAWAYO & 1989 & 46.88 & 33.97 & 64.56 & RW2 \\ 
  Zimbabwe & BULAWAYO & 1990 & 48.67 & 35.25 & 66.69 & RW2 \\ 
  Zimbabwe & BULAWAYO & 1991 & 51.44 & 37.83 & 69.73 & RW2 \\ 
  Zimbabwe & BULAWAYO & 1992 & 54.99 & 40.81 & 73.47 & RW2 \\ 
  Zimbabwe & BULAWAYO & 1993 & 59.48 & 43.90 & 79.94 & RW2 \\ 
  Zimbabwe & BULAWAYO & 1994 & 64.63 & 47.03 & 87.70 & RW2 \\ 
  Zimbabwe & BULAWAYO & 1995 & 70.42 & 51.38 & 96.95 & RW2 \\ 
  Zimbabwe & BULAWAYO & 1996 & 75.41 & 55.43 & 102.53 & RW2 \\ 
  Zimbabwe & BULAWAYO & 1997 & 79.09 & 58.25 & 107.31 & RW2 \\ 
  Zimbabwe & BULAWAYO & 1998 & 81.29 & 59.54 & 110.88 & RW2 \\ 
  Zimbabwe & BULAWAYO & 1999 & 82.09 & 59.48 & 113.52 & RW2 \\ 
  Zimbabwe & BULAWAYO & 2000 & 80.74 & 57.69 & 111.25 & RW2 \\ 
  Zimbabwe & BULAWAYO & 2001 & 78.97 & 56.90 & 108.45 & RW2 \\ 
  Zimbabwe & BULAWAYO & 2002 & 76.72 & 54.83 & 105.85 & RW2 \\ 
  Zimbabwe & BULAWAYO & 2003 & 74.14 & 52.32 & 104.02 & RW2 \\ 
  Zimbabwe & BULAWAYO & 2004 & 71.60 & 49.47 & 101.96 & RW2 \\ 
  Zimbabwe & BULAWAYO & 2005 & 69.01 & 47.24 & 99.42 & RW2 \\ 
  Zimbabwe & BULAWAYO & 2006 & 66.62 & 46.13 & 95.29 & RW2 \\ 
  Zimbabwe & BULAWAYO & 2007 & 64.08 & 44.71 & 90.62 & RW2 \\ 
  Zimbabwe & BULAWAYO & 2008 & 61.67 & 41.59 & 90.64 & RW2 \\ 
  Zimbabwe & BULAWAYO & 2009 & 59.49 & 36.09 & 96.79 & RW2 \\ 
  Zimbabwe & BULAWAYO & 2010 & 57.28 & 28.74 & 112.43 & RW2 \\ 
  Zimbabwe & BULAWAYO & 2011 & 55.20 & 22.20 & 132.49 & RW2 \\ 
  Zimbabwe & BULAWAYO & 2012 & 52.98 & 16.54 & 161.48 & RW2 \\ 
  Zimbabwe & BULAWAYO & 2013 & 51.02 & 12.05 & 198.43 & RW2 \\ 
  Zimbabwe & BULAWAYO & 2014 & 49.13 & 8.39 & 245.13 & RW2 \\ 
  Zimbabwe & BULAWAYO & 2015 & 47.21 & 5.81 & 309.88 & RW2 \\ 
  Zimbabwe & BULAWAYO & 2016 & 45.15 & 3.95 & 384.02 & RW2 \\ 
  Zimbabwe & BULAWAYO & 2017 & 43.69 & 2.54 & 468.19 & RW2 \\ 
  Zimbabwe & BULAWAYO & 2018 & 41.72 & 1.63 & 552.68 & RW2 \\ 
  Zimbabwe & BULAWAYO & 2019 & 39.99 & 1.07 & 646.04 & RW2 \\ 
  Zimbabwe & HARARE & 1980 & 75.64 & 46.93 & 119.54 & RW2 \\ 
  Zimbabwe & HARARE & 1981 & 71.82 & 49.47 & 102.87 & RW2 \\ 
  Zimbabwe & HARARE & 1982 & 68.02 & 48.34 & 94.86 & RW2 \\ 
  Zimbabwe & HARARE & 1983 & 64.59 & 45.68 & 90.96 & RW2 \\ 
  Zimbabwe & HARARE & 1984 & 61.62 & 43.41 & 87.40 & RW2 \\ 
  Zimbabwe & HARARE & 1985 & 58.99 & 42.31 & 81.48 & RW2 \\ 
  Zimbabwe & HARARE & 1986 & 57.39 & 42.21 & 77.05 & RW2 \\ 
  Zimbabwe & HARARE & 1987 & 56.62 & 42.46 & 75.22 & RW2 \\ 
  Zimbabwe & HARARE & 1988 & 56.80 & 42.44 & 75.53 & RW2 \\ 
  Zimbabwe & HARARE & 1989 & 58.01 & 43.07 & 77.84 & RW2 \\ 
  Zimbabwe & HARARE & 1990 & 60.40 & 44.74 & 81.13 & RW2 \\ 
  Zimbabwe & HARARE & 1991 & 63.84 & 47.99 & 84.53 & RW2 \\ 
  Zimbabwe & HARARE & 1992 & 68.38 & 51.75 & 89.95 & RW2 \\ 
  Zimbabwe & HARARE & 1993 & 73.93 & 55.64 & 97.51 & RW2 \\ 
  Zimbabwe & HARARE & 1994 & 80.21 & 59.63 & 106.98 & RW2 \\ 
  Zimbabwe & HARARE & 1995 & 87.54 & 64.88 & 118.17 & RW2 \\ 
  Zimbabwe & HARARE & 1996 & 93.51 & 70.07 & 125.47 & RW2 \\ 
  Zimbabwe & HARARE & 1997 & 98.38 & 73.55 & 130.84 & RW2 \\ 
  Zimbabwe & HARARE & 1998 & 101.20 & 75.13 & 135.55 & RW2 \\ 
  Zimbabwe & HARARE & 1999 & 102.27 & 75.22 & 139.00 & RW2 \\ 
  Zimbabwe & HARARE & 2000 & 100.91 & 72.63 & 136.77 & RW2 \\ 
  Zimbabwe & HARARE & 2001 & 99.15 & 71.81 & 134.55 & RW2 \\ 
  Zimbabwe & HARARE & 2002 & 96.61 & 70.00 & 131.69 & RW2 \\ 
  Zimbabwe & HARARE & 2003 & 93.82 & 67.06 & 129.68 & RW2 \\ 
  Zimbabwe & HARARE & 2004 & 91.00 & 63.47 & 128.16 & RW2 \\ 
  Zimbabwe & HARARE & 2005 & 88.23 & 60.94 & 125.75 & RW2 \\ 
  Zimbabwe & HARARE & 2006 & 85.53 & 59.65 & 120.92 & RW2 \\ 
  Zimbabwe & HARARE & 2007 & 82.82 & 58.21 & 116.72 & RW2 \\ 
  Zimbabwe & HARARE & 2008 & 80.21 & 54.63 & 116.35 & RW2 \\ 
  Zimbabwe & HARARE & 2009 & 77.54 & 47.58 & 124.76 & RW2 \\ 
  Zimbabwe & HARARE & 2010 & 75.12 & 38.01 & 144.38 & RW2 \\ 
  Zimbabwe & HARARE & 2011 & 72.71 & 29.68 & 170.35 & RW2 \\ 
  Zimbabwe & HARARE & 2012 & 70.19 & 22.10 & 207.09 & RW2 \\ 
  Zimbabwe & HARARE & 2013 & 67.69 & 16.24 & 250.93 & RW2 \\ 
  Zimbabwe & HARARE & 2014 & 65.31 & 11.62 & 306.00 & RW2 \\ 
  Zimbabwe & HARARE & 2015 & 63.22 & 7.84 & 375.78 & RW2 \\ 
  Zimbabwe & HARARE & 2016 & 61.46 & 5.46 & 455.17 & RW2 \\ 
  Zimbabwe & HARARE & 2017 & 58.79 & 3.59 & 550.50 & RW2 \\ 
  Zimbabwe & HARARE & 2018 & 56.99 & 2.36 & 632.26 & RW2 \\ 
  Zimbabwe & HARARE & 2019 & 55.21 & 1.52 & 725.31 & RW2 \\ 
  Zimbabwe & MANICALAND & 1980 & 124.51 & 84.07 & 180.40 & RW2 \\ 
  Zimbabwe & MANICALAND & 1981 & 117.89 & 88.82 & 154.99 & RW2 \\ 
  Zimbabwe & MANICALAND & 1982 & 111.47 & 86.49 & 143.14 & RW2 \\ 
  Zimbabwe & MANICALAND & 1983 & 105.93 & 80.53 & 138.58 & RW2 \\ 
  Zimbabwe & MANICALAND & 1984 & 100.81 & 75.39 & 134.21 & RW2 \\ 
  Zimbabwe & MANICALAND & 1985 & 96.44 & 72.97 & 125.31 & RW2 \\ 
  Zimbabwe & MANICALAND & 1986 & 93.66 & 72.79 & 119.01 & RW2 \\ 
  Zimbabwe & MANICALAND & 1987 & 92.40 & 72.84 & 116.46 & RW2 \\ 
  Zimbabwe & MANICALAND & 1988 & 92.72 & 72.70 & 117.24 & RW2 \\ 
  Zimbabwe & MANICALAND & 1989 & 94.82 & 73.25 & 122.04 & RW2 \\ 
  Zimbabwe & MANICALAND & 1990 & 98.70 & 76.28 & 127.01 & RW2 \\ 
  Zimbabwe & MANICALAND & 1991 & 104.29 & 81.89 & 132.11 & RW2 \\ 
  Zimbabwe & MANICALAND & 1992 & 111.70 & 88.37 & 139.81 & RW2 \\ 
  Zimbabwe & MANICALAND & 1993 & 120.84 & 94.71 & 152.04 & RW2 \\ 
  Zimbabwe & MANICALAND & 1994 & 131.00 & 100.72 & 167.50 & RW2 \\ 
  Zimbabwe & MANICALAND & 1995 & 142.76 & 110.22 & 185.05 & RW2 \\ 
  Zimbabwe & MANICALAND & 1996 & 152.56 & 118.43 & 196.30 & RW2 \\ 
  Zimbabwe & MANICALAND & 1997 & 160.44 & 124.69 & 205.41 & RW2 \\ 
  Zimbabwe & MANICALAND & 1998 & 165.52 & 128.25 & 212.93 & RW2 \\ 
  Zimbabwe & MANICALAND & 1999 & 167.59 & 127.92 & 218.22 & RW2 \\ 
  Zimbabwe & MANICALAND & 2000 & 166.45 & 125.82 & 215.92 & RW2 \\ 
  Zimbabwe & MANICALAND & 2001 & 164.36 & 124.84 & 212.74 & RW2 \\ 
  Zimbabwe & MANICALAND & 2002 & 161.21 & 122.54 & 209.20 & RW2 \\ 
  Zimbabwe & MANICALAND & 2003 & 157.45 & 117.87 & 206.66 & RW2 \\ 
  Zimbabwe & MANICALAND & 2004 & 153.50 & 113.08 & 204.91 & RW2 \\ 
  Zimbabwe & MANICALAND & 2005 & 149.86 & 109.03 & 201.34 & RW2 \\ 
  Zimbabwe & MANICALAND & 2006 & 145.73 & 107.66 & 194.47 & RW2 \\ 
  Zimbabwe & MANICALAND & 2007 & 142.18 & 106.13 & 187.47 & RW2 \\ 
  Zimbabwe & MANICALAND & 2008 & 138.39 & 100.01 & 189.38 & RW2 \\ 
  Zimbabwe & MANICALAND & 2009 & 134.42 & 87.28 & 203.05 & RW2 \\ 
  Zimbabwe & MANICALAND & 2010 & 130.55 & 70.07 & 234.05 & RW2 \\ 
  Zimbabwe & MANICALAND & 2011 & 127.26 & 55.00 & 271.51 & RW2 \\ 
  Zimbabwe & MANICALAND & 2012 & 123.64 & 41.37 & 321.87 & RW2 \\ 
  Zimbabwe & MANICALAND & 2013 & 119.14 & 30.12 & 377.50 & RW2 \\ 
  Zimbabwe & MANICALAND & 2014 & 116.33 & 21.75 & 454.53 & RW2 \\ 
  Zimbabwe & MANICALAND & 2015 & 112.70 & 15.17 & 524.33 & RW2 \\ 
  Zimbabwe & MANICALAND & 2016 & 109.63 & 10.18 & 617.07 & RW2 \\ 
  Zimbabwe & MANICALAND & 2017 & 107.15 & 6.85 & 694.01 & RW2 \\ 
  Zimbabwe & MANICALAND & 2018 & 104.10 & 4.48 & 764.86 & RW2 \\ 
  Zimbabwe & MANICALAND & 2019 & 99.93 & 2.79 & 833.77 & RW2 \\ 
  Zimbabwe & MASHONALAND CENTRAL & 1980 & 165.98 & 115.57 & 231.30 & RW2 \\ 
  Zimbabwe & MASHONALAND CENTRAL & 1981 & 154.92 & 120.66 & 195.86 & RW2 \\ 
  Zimbabwe & MASHONALAND CENTRAL & 1982 & 144.33 & 115.15 & 179.33 & RW2 \\ 
  Zimbabwe & MASHONALAND CENTRAL & 1983 & 134.56 & 104.77 & 172.14 & RW2 \\ 
  Zimbabwe & MASHONALAND CENTRAL & 1984 & 125.96 & 96.51 & 164.10 & RW2 \\ 
  Zimbabwe & MASHONALAND CENTRAL & 1985 & 118.21 & 90.90 & 151.01 & RW2 \\ 
  Zimbabwe & MASHONALAND CENTRAL & 1986 & 112.70 & 88.76 & 141.24 & RW2 \\ 
  Zimbabwe & MASHONALAND CENTRAL & 1987 & 108.94 & 87.10 & 135.23 & RW2 \\ 
  Zimbabwe & MASHONALAND CENTRAL & 1988 & 107.02 & 84.45 & 134.46 & RW2 \\ 
  Zimbabwe & MASHONALAND CENTRAL & 1989 & 106.95 & 82.89 & 136.55 & RW2 \\ 
  Zimbabwe & MASHONALAND CENTRAL & 1990 & 108.70 & 84.25 & 139.36 & RW2 \\ 
  Zimbabwe & MASHONALAND CENTRAL & 1991 & 112.32 & 88.24 & 142.21 & RW2 \\ 
  Zimbabwe & MASHONALAND CENTRAL & 1992 & 117.62 & 92.79 & 147.53 & RW2 \\ 
  Zimbabwe & MASHONALAND CENTRAL & 1993 & 124.25 & 97.36 & 156.22 & RW2 \\ 
  Zimbabwe & MASHONALAND CENTRAL & 1994 & 131.69 & 101.26 & 168.23 & RW2 \\ 
  Zimbabwe & MASHONALAND CENTRAL & 1995 & 140.20 & 108.39 & 181.41 & RW2 \\ 
  Zimbabwe & MASHONALAND CENTRAL & 1996 & 146.72 & 113.93 & 188.16 & RW2 \\ 
  Zimbabwe & MASHONALAND CENTRAL & 1997 & 151.09 & 117.82 & 192.35 & RW2 \\ 
  Zimbabwe & MASHONALAND CENTRAL & 1998 & 152.68 & 118.64 & 196.25 & RW2 \\ 
  Zimbabwe & MASHONALAND CENTRAL & 1999 & 151.52 & 116.12 & 197.35 & RW2 \\ 
  Zimbabwe & MASHONALAND CENTRAL & 2000 & 147.10 & 110.82 & 190.51 & RW2 \\ 
  Zimbabwe & MASHONALAND CENTRAL & 2001 & 142.19 & 108.33 & 183.36 & RW2 \\ 
  Zimbabwe & MASHONALAND CENTRAL & 2002 & 136.55 & 104.06 & 176.54 & RW2 \\ 
  Zimbabwe & MASHONALAND CENTRAL & 2003 & 130.43 & 98.27 & 170.89 & RW2 \\ 
  Zimbabwe & MASHONALAND CENTRAL & 2004 & 124.57 & 92.07 & 165.66 & RW2 \\ 
  Zimbabwe & MASHONALAND CENTRAL & 2005 & 119.04 & 86.81 & 160.41 & RW2 \\ 
  Zimbabwe & MASHONALAND CENTRAL & 2006 & 113.44 & 84.33 & 150.76 & RW2 \\ 
  Zimbabwe & MASHONALAND CENTRAL & 2007 & 108.03 & 81.81 & 141.30 & RW2 \\ 
  Zimbabwe & MASHONALAND CENTRAL & 2008 & 102.95 & 75.63 & 138.76 & RW2 \\ 
  Zimbabwe & MASHONALAND CENTRAL & 2009 & 97.82 & 64.56 & 146.75 & RW2 \\ 
  Zimbabwe & MASHONALAND CENTRAL & 2010 & 93.26 & 50.00 & 169.09 & RW2 \\ 
  Zimbabwe & MASHONALAND CENTRAL & 2011 & 88.30 & 38.20 & 196.15 & RW2 \\ 
  Zimbabwe & MASHONALAND CENTRAL & 2012 & 84.09 & 27.99 & 232.04 & RW2 \\ 
  Zimbabwe & MASHONALAND CENTRAL & 2013 & 79.62 & 19.72 & 279.48 & RW2 \\ 
  Zimbabwe & MASHONALAND CENTRAL & 2014 & 75.62 & 13.63 & 330.65 & RW2 \\ 
  Zimbabwe & MASHONALAND CENTRAL & 2015 & 71.88 & 9.38 & 406.93 & RW2 \\ 
  Zimbabwe & MASHONALAND CENTRAL & 2016 & 68.13 & 6.31 & 482.66 & RW2 \\ 
  Zimbabwe & MASHONALAND CENTRAL & 2017 & 64.80 & 4.06 & 570.97 & RW2 \\ 
  Zimbabwe & MASHONALAND CENTRAL & 2018 & 60.98 & 2.47 & 651.23 & RW2 \\ 
  Zimbabwe & MASHONALAND CENTRAL & 2019 & 57.69 & 1.60 & 725.97 & RW2 \\ 
  Zimbabwe & MASHONALAND EAST & 1980 & 95.73 & 61.59 & 144.28 & RW2 \\ 
  Zimbabwe & MASHONALAND EAST & 1981 & 90.98 & 64.89 & 125.09 & RW2 \\ 
  Zimbabwe & MASHONALAND EAST & 1982 & 86.48 & 63.93 & 116.29 & RW2 \\ 
  Zimbabwe & MASHONALAND EAST & 1983 & 82.44 & 60.49 & 112.28 & RW2 \\ 
  Zimbabwe & MASHONALAND EAST & 1984 & 78.97 & 57.24 & 108.38 & RW2 \\ 
  Zimbabwe & MASHONALAND EAST & 1985 & 75.79 & 56.04 & 101.55 & RW2 \\ 
  Zimbabwe & MASHONALAND EAST & 1986 & 73.98 & 56.21 & 96.57 & RW2 \\ 
  Zimbabwe & MASHONALAND EAST & 1987 & 73.30 & 56.76 & 94.29 & RW2 \\ 
  Zimbabwe & MASHONALAND EAST & 1988 & 73.84 & 56.73 & 95.44 & RW2 \\ 
  Zimbabwe & MASHONALAND EAST & 1989 & 75.59 & 57.54 & 99.13 & RW2 \\ 
  Zimbabwe & MASHONALAND EAST & 1990 & 78.93 & 60.11 & 102.88 & RW2 \\ 
  Zimbabwe & MASHONALAND EAST & 1991 & 83.62 & 64.63 & 107.67 & RW2 \\ 
  Zimbabwe & MASHONALAND EAST & 1992 & 89.86 & 70.33 & 113.95 & RW2 \\ 
  Zimbabwe & MASHONALAND EAST & 1993 & 97.32 & 75.47 & 124.19 & RW2 \\ 
  Zimbabwe & MASHONALAND EAST & 1994 & 105.76 & 80.83 & 136.30 & RW2 \\ 
  Zimbabwe & MASHONALAND EAST & 1995 & 115.34 & 88.78 & 150.46 & RW2 \\ 
  Zimbabwe & MASHONALAND EAST & 1996 & 123.57 & 96.36 & 159.57 & RW2 \\ 
  Zimbabwe & MASHONALAND EAST & 1997 & 130.21 & 101.43 & 166.36 & RW2 \\ 
  Zimbabwe & MASHONALAND EAST & 1998 & 134.25 & 104.17 & 172.88 & RW2 \\ 
  Zimbabwe & MASHONALAND EAST & 1999 & 136.03 & 103.90 & 176.83 & RW2 \\ 
  Zimbabwe & MASHONALAND EAST & 2000 & 134.76 & 101.42 & 175.59 & RW2 \\ 
  Zimbabwe & MASHONALAND EAST & 2001 & 132.76 & 100.63 & 172.37 & RW2 \\ 
  Zimbabwe & MASHONALAND EAST & 2002 & 130.03 & 98.63 & 169.27 & RW2 \\ 
  Zimbabwe & MASHONALAND EAST & 2003 & 126.88 & 94.85 & 166.44 & RW2 \\ 
  Zimbabwe & MASHONALAND EAST & 2004 & 123.46 & 90.45 & 165.10 & RW2 \\ 
  Zimbabwe & MASHONALAND EAST & 2005 & 120.40 & 87.77 & 162.36 & RW2 \\ 
  Zimbabwe & MASHONALAND EAST & 2006 & 117.00 & 86.93 & 155.04 & RW2 \\ 
  Zimbabwe & MASHONALAND EAST & 2007 & 113.79 & 86.49 & 148.64 & RW2 \\ 
  Zimbabwe & MASHONALAND EAST & 2008 & 110.50 & 81.74 & 148.44 & RW2 \\ 
  Zimbabwe & MASHONALAND EAST & 2009 & 107.45 & 70.81 & 160.32 & RW2 \\ 
  Zimbabwe & MASHONALAND EAST & 2010 & 104.30 & 56.78 & 187.03 & RW2 \\ 
  Zimbabwe & MASHONALAND EAST & 2011 & 101.38 & 44.16 & 218.98 & RW2 \\ 
  Zimbabwe & MASHONALAND EAST & 2012 & 98.40 & 33.74 & 265.51 & RW2 \\ 
  Zimbabwe & MASHONALAND EAST & 2013 & 95.38 & 24.25 & 315.63 & RW2 \\ 
  Zimbabwe & MASHONALAND EAST & 2014 & 92.45 & 17.50 & 385.32 & RW2 \\ 
  Zimbabwe & MASHONALAND EAST & 2015 & 90.12 & 11.77 & 468.26 & RW2 \\ 
  Zimbabwe & MASHONALAND EAST & 2016 & 86.84 & 8.28 & 544.62 & RW2 \\ 
  Zimbabwe & MASHONALAND EAST & 2017 & 84.10 & 5.25 & 631.29 & RW2 \\ 
  Zimbabwe & MASHONALAND EAST & 2018 & 81.87 & 3.48 & 707.08 & RW2 \\ 
  Zimbabwe & MASHONALAND EAST & 2019 & 79.53 & 2.19 & 774.95 & RW2 \\ 
  Zimbabwe & MASHONALAND WEST & 1980 & 121.11 & 80.87 & 176.94 & RW2 \\ 
  Zimbabwe & MASHONALAND WEST & 1981 & 114.89 & 85.64 & 152.11 & RW2 \\ 
  Zimbabwe & MASHONALAND WEST & 1982 & 108.89 & 83.35 & 140.48 & RW2 \\ 
  Zimbabwe & MASHONALAND WEST & 1983 & 103.48 & 78.53 & 135.91 & RW2 \\ 
  Zimbabwe & MASHONALAND WEST & 1984 & 98.72 & 74.05 & 131.74 & RW2 \\ 
  Zimbabwe & MASHONALAND WEST & 1985 & 94.50 & 71.93 & 122.67 & RW2 \\ 
  Zimbabwe & MASHONALAND WEST & 1986 & 91.84 & 71.87 & 115.92 & RW2 \\ 
  Zimbabwe & MASHONALAND WEST & 1987 & 90.50 & 72.15 & 113.05 & RW2 \\ 
  Zimbabwe & MASHONALAND WEST & 1988 & 90.63 & 71.81 & 113.86 & RW2 \\ 
  Zimbabwe & MASHONALAND WEST & 1989 & 92.35 & 71.86 & 117.65 & RW2 \\ 
  Zimbabwe & MASHONALAND WEST & 1990 & 95.75 & 74.55 & 122.33 & RW2 \\ 
  Zimbabwe & MASHONALAND WEST & 1991 & 100.81 & 79.81 & 127.32 & RW2 \\ 
  Zimbabwe & MASHONALAND WEST & 1992 & 107.49 & 85.73 & 133.54 & RW2 \\ 
  Zimbabwe & MASHONALAND WEST & 1993 & 115.61 & 91.30 & 144.41 & RW2 \\ 
  Zimbabwe & MASHONALAND WEST & 1994 & 124.75 & 96.57 & 158.49 & RW2 \\ 
  Zimbabwe & MASHONALAND WEST & 1995 & 135.22 & 105.44 & 174.24 & RW2 \\ 
  Zimbabwe & MASHONALAND WEST & 1996 & 143.85 & 112.80 & 183.21 & RW2 \\ 
  Zimbabwe & MASHONALAND WEST & 1997 & 150.79 & 118.72 & 190.35 & RW2 \\ 
  Zimbabwe & MASHONALAND WEST & 1998 & 154.84 & 121.38 & 196.84 & RW2 \\ 
  Zimbabwe & MASHONALAND WEST & 1999 & 156.31 & 120.67 & 202.28 & RW2 \\ 
  Zimbabwe & MASHONALAND WEST & 2000 & 154.45 & 117.73 & 198.51 & RW2 \\ 
  Zimbabwe & MASHONALAND WEST & 2001 & 151.87 & 116.73 & 193.61 & RW2 \\ 
  Zimbabwe & MASHONALAND WEST & 2002 & 148.42 & 114.19 & 189.96 & RW2 \\ 
  Zimbabwe & MASHONALAND WEST & 2003 & 144.45 & 109.82 & 187.11 & RW2 \\ 
  Zimbabwe & MASHONALAND WEST & 2004 & 140.65 & 104.59 & 185.73 & RW2 \\ 
  Zimbabwe & MASHONALAND WEST & 2005 & 136.82 & 100.23 & 182.72 & RW2 \\ 
  Zimbabwe & MASHONALAND WEST & 2006 & 133.05 & 99.01 & 176.11 & RW2 \\ 
  Zimbabwe & MASHONALAND WEST & 2007 & 129.02 & 97.04 & 169.69 & RW2 \\ 
  Zimbabwe & MASHONALAND WEST & 2008 & 125.19 & 91.24 & 169.70 & RW2 \\ 
  Zimbabwe & MASHONALAND WEST & 2009 & 121.78 & 79.67 & 182.93 & RW2 \\ 
  Zimbabwe & MASHONALAND WEST & 2010 & 118.03 & 63.29 & 210.63 & RW2 \\ 
  Zimbabwe & MASHONALAND WEST & 2011 & 114.40 & 50.09 & 246.70 & RW2 \\ 
  Zimbabwe & MASHONALAND WEST & 2012 & 110.87 & 37.39 & 295.36 & RW2 \\ 
  Zimbabwe & MASHONALAND WEST & 2013 & 107.53 & 27.02 & 349.20 & RW2 \\ 
  Zimbabwe & MASHONALAND WEST & 2014 & 104.74 & 19.12 & 417.77 & RW2 \\ 
  Zimbabwe & MASHONALAND WEST & 2015 & 101.27 & 13.78 & 500.25 & RW2 \\ 
  Zimbabwe & MASHONALAND WEST & 2016 & 97.90 & 9.05 & 577.22 & RW2 \\ 
  Zimbabwe & MASHONALAND WEST & 2017 & 95.43 & 6.06 & 658.77 & RW2 \\ 
  Zimbabwe & MASHONALAND WEST & 2018 & 91.02 & 3.98 & 739.36 & RW2 \\ 
  Zimbabwe & MASHONALAND WEST & 2019 & 88.24 & 2.51 & 803.95 & RW2 \\ 
  Zimbabwe & MASVINGO & 1980 & 111.37 & 74.52 & 162.54 & RW2 \\ 
  Zimbabwe & MASVINGO & 1981 & 103.84 & 77.50 & 137.29 & RW2 \\ 
  Zimbabwe & MASVINGO & 1982 & 96.91 & 74.41 & 125.33 & RW2 \\ 
  Zimbabwe & MASVINGO & 1983 & 90.37 & 68.39 & 119.44 & RW2 \\ 
  Zimbabwe & MASVINGO & 1984 & 84.75 & 63.59 & 113.45 & RW2 \\ 
  Zimbabwe & MASVINGO & 1985 & 79.74 & 60.41 & 104.11 & RW2 \\ 
  Zimbabwe & MASVINGO & 1986 & 76.23 & 59.31 & 97.08 & RW2 \\ 
  Zimbabwe & MASVINGO & 1987 & 74.03 & 58.78 & 93.35 & RW2 \\ 
  Zimbabwe & MASVINGO & 1988 & 73.09 & 57.37 & 92.99 & RW2 \\ 
  Zimbabwe & MASVINGO & 1989 & 73.48 & 56.62 & 94.72 & RW2 \\ 
  Zimbabwe & MASVINGO & 1990 & 75.24 & 58.21 & 96.83 & RW2 \\ 
  Zimbabwe & MASVINGO & 1991 & 78.28 & 61.47 & 99.53 & RW2 \\ 
  Zimbabwe & MASVINGO & 1992 & 82.54 & 65.47 & 103.72 & RW2 \\ 
  Zimbabwe & MASVINGO & 1993 & 87.91 & 68.63 & 111.36 & RW2 \\ 
  Zimbabwe & MASVINGO & 1994 & 93.94 & 72.13 & 120.52 & RW2 \\ 
  Zimbabwe & MASVINGO & 1995 & 100.81 & 77.66 & 131.80 & RW2 \\ 
  Zimbabwe & MASVINGO & 1996 & 106.29 & 82.49 & 137.90 & RW2 \\ 
  Zimbabwe & MASVINGO & 1997 & 110.08 & 85.56 & 142.00 & RW2 \\ 
  Zimbabwe & MASVINGO & 1998 & 111.95 & 86.23 & 145.31 & RW2 \\ 
  Zimbabwe & MASVINGO & 1999 & 111.62 & 84.80 & 147.10 & RW2 \\ 
  Zimbabwe & MASVINGO & 2000 & 108.61 & 81.40 & 142.81 & RW2 \\ 
  Zimbabwe & MASVINGO & 2001 & 105.20 & 79.25 & 137.64 & RW2 \\ 
  Zimbabwe & MASVINGO & 2002 & 101.24 & 76.49 & 132.71 & RW2 \\ 
  Zimbabwe & MASVINGO & 2003 & 96.89 & 72.11 & 128.15 & RW2 \\ 
  Zimbabwe & MASVINGO & 2004 & 92.69 & 67.78 & 124.83 & RW2 \\ 
  Zimbabwe & MASVINGO & 2005 & 88.62 & 64.36 & 120.33 & RW2 \\ 
  Zimbabwe & MASVINGO & 2006 & 84.76 & 62.88 & 112.75 & RW2 \\ 
  Zimbabwe & MASVINGO & 2007 & 80.81 & 61.48 & 105.75 & RW2 \\ 
  Zimbabwe & MASVINGO & 2008 & 77.21 & 56.90 & 103.86 & RW2 \\ 
  Zimbabwe & MASVINGO & 2009 & 73.53 & 48.32 & 111.28 & RW2 \\ 
  Zimbabwe & MASVINGO & 2010 & 70.25 & 37.58 & 129.79 & RW2 \\ 
  Zimbabwe & MASVINGO & 2011 & 66.77 & 28.60 & 150.46 & RW2 \\ 
  Zimbabwe & MASVINGO & 2012 & 63.53 & 21.32 & 183.72 & RW2 \\ 
  Zimbabwe & MASVINGO & 2013 & 60.58 & 15.07 & 222.25 & RW2 \\ 
  Zimbabwe & MASVINGO & 2014 & 57.67 & 10.35 & 272.07 & RW2 \\ 
  Zimbabwe & MASVINGO & 2015 & 54.93 & 7.17 & 331.02 & RW2 \\ 
  Zimbabwe & MASVINGO & 2016 & 52.32 & 4.65 & 408.64 & RW2 \\ 
  Zimbabwe & MASVINGO & 2017 & 49.69 & 3.09 & 492.43 & RW2 \\ 
  Zimbabwe & MASVINGO & 2018 & 46.91 & 2.06 & 571.89 & RW2 \\ 
  Zimbabwe & MASVINGO & 2019 & 44.87 & 1.27 & 661.26 & RW2 \\ 
  Zimbabwe & MATABELELAND NORTH & 1980 & 110.08 & 72.67 & 163.14 & RW2 \\ 
  Zimbabwe & MATABELELAND NORTH & 1981 & 102.65 & 75.28 & 138.18 & RW2 \\ 
  Zimbabwe & MATABELELAND NORTH & 1982 & 95.63 & 72.40 & 125.40 & RW2 \\ 
  Zimbabwe & MATABELELAND NORTH & 1983 & 89.23 & 66.78 & 119.29 & RW2 \\ 
  Zimbabwe & MATABELELAND NORTH & 1984 & 83.59 & 61.78 & 113.32 & RW2 \\ 
  Zimbabwe & MATABELELAND NORTH & 1985 & 78.59 & 58.80 & 103.89 & RW2 \\ 
  Zimbabwe & MATABELELAND NORTH & 1986 & 75.12 & 57.55 & 96.86 & RW2 \\ 
  Zimbabwe & MATABELELAND NORTH & 1987 & 72.76 & 56.63 & 92.77 & RW2 \\ 
  Zimbabwe & MATABELELAND NORTH & 1988 & 71.88 & 55.38 & 92.44 & RW2 \\ 
  Zimbabwe & MATABELELAND NORTH & 1989 & 72.23 & 54.75 & 94.44 & RW2 \\ 
  Zimbabwe & MATABELELAND NORTH & 1990 & 73.94 & 56.07 & 96.70 & RW2 \\ 
  Zimbabwe & MATABELELAND NORTH & 1991 & 76.90 & 59.07 & 99.59 & RW2 \\ 
  Zimbabwe & MATABELELAND NORTH & 1992 & 81.17 & 62.64 & 104.03 & RW2 \\ 
  Zimbabwe & MATABELELAND NORTH & 1993 & 86.56 & 66.24 & 111.40 & RW2 \\ 
  Zimbabwe & MATABELELAND NORTH & 1994 & 92.58 & 69.47 & 121.02 & RW2 \\ 
  Zimbabwe & MATABELELAND NORTH & 1995 & 99.55 & 74.90 & 132.09 & RW2 \\ 
  Zimbabwe & MATABELELAND NORTH & 1996 & 105.11 & 79.47 & 138.02 & RW2 \\ 
  Zimbabwe & MATABELELAND NORTH & 1997 & 109.17 & 82.65 & 143.15 & RW2 \\ 
  Zimbabwe & MATABELELAND NORTH & 1998 & 111.22 & 84.09 & 146.71 & RW2 \\ 
  Zimbabwe & MATABELELAND NORTH & 1999 & 111.23 & 82.72 & 148.91 & RW2 \\ 
  Zimbabwe & MATABELELAND NORTH & 2000 & 108.57 & 79.49 & 145.01 & RW2 \\ 
  Zimbabwe & MATABELELAND NORTH & 2001 & 105.64 & 77.87 & 140.94 & RW2 \\ 
  Zimbabwe & MATABELELAND NORTH & 2002 & 101.85 & 75.15 & 136.13 & RW2 \\ 
  Zimbabwe & MATABELELAND NORTH & 2003 & 98.20 & 71.39 & 133.59 & RW2 \\ 
  Zimbabwe & MATABELELAND NORTH & 2004 & 94.20 & 67.08 & 130.40 & RW2 \\ 
  Zimbabwe & MATABELELAND NORTH & 2005 & 90.53 & 63.99 & 126.00 & RW2 \\ 
  Zimbabwe & MATABELELAND NORTH & 2006 & 86.83 & 62.35 & 119.38 & RW2 \\ 
  Zimbabwe & MATABELELAND NORTH & 2007 & 83.38 & 60.63 & 113.30 & RW2 \\ 
  Zimbabwe & MATABELELAND NORTH & 2008 & 79.92 & 56.46 & 111.95 & RW2 \\ 
  Zimbabwe & MATABELELAND NORTH & 2009 & 76.59 & 48.70 & 119.05 & RW2 \\ 
  Zimbabwe & MATABELELAND NORTH & 2010 & 73.20 & 38.47 & 137.48 & RW2 \\ 
  Zimbabwe & MATABELELAND NORTH & 2011 & 70.29 & 29.10 & 159.44 & RW2 \\ 
  Zimbabwe & MATABELELAND NORTH & 2012 & 67.06 & 21.77 & 194.89 & RW2 \\ 
  Zimbabwe & MATABELELAND NORTH & 2013 & 64.24 & 15.78 & 237.86 & RW2 \\ 
  Zimbabwe & MATABELELAND NORTH & 2014 & 61.20 & 10.82 & 290.21 & RW2 \\ 
  Zimbabwe & MATABELELAND NORTH & 2015 & 58.94 & 7.50 & 361.39 & RW2 \\ 
  Zimbabwe & MATABELELAND NORTH & 2016 & 56.76 & 5.04 & 437.27 & RW2 \\ 
  Zimbabwe & MATABELELAND NORTH & 2017 & 53.90 & 3.28 & 512.61 & RW2 \\ 
  Zimbabwe & MATABELELAND NORTH & 2018 & 51.15 & 2.16 & 597.11 & RW2 \\ 
  Zimbabwe & MATABELELAND NORTH & 2019 & 49.21 & 1.36 & 692.89 & RW2 \\ 
  Zimbabwe & MATABELELAND SOUTH & 1980 & 76.92 & 50.60 & 114.68 & RW2 \\ 
  Zimbabwe & MATABELELAND SOUTH & 1981 & 72.37 & 53.18 & 97.51 & RW2 \\ 
  Zimbabwe & MATABELELAND SOUTH & 1982 & 68.08 & 51.31 & 89.86 & RW2 \\ 
  Zimbabwe & MATABELELAND SOUTH & 1983 & 64.09 & 47.62 & 86.28 & RW2 \\ 
  Zimbabwe & MATABELELAND SOUTH & 1984 & 60.68 & 44.45 & 83.23 & RW2 \\ 
  Zimbabwe & MATABELELAND SOUTH & 1985 & 57.70 & 42.58 & 76.94 & RW2 \\ 
  Zimbabwe & MATABELELAND SOUTH & 1986 & 55.68 & 42.31 & 72.63 & RW2 \\ 
  Zimbabwe & MATABELELAND SOUTH & 1987 & 54.59 & 42.04 & 70.52 & RW2 \\ 
  Zimbabwe & MATABELELAND SOUTH & 1988 & 54.51 & 41.66 & 71.00 & RW2 \\ 
  Zimbabwe & MATABELELAND SOUTH & 1989 & 55.42 & 41.64 & 73.09 & RW2 \\ 
  Zimbabwe & MATABELELAND SOUTH & 1990 & 57.35 & 43.38 & 75.93 & RW2 \\ 
  Zimbabwe & MATABELELAND SOUTH & 1991 & 60.40 & 46.11 & 78.85 & RW2 \\ 
  Zimbabwe & MATABELELAND SOUTH & 1992 & 64.43 & 49.63 & 83.15 & RW2 \\ 
  Zimbabwe & MATABELELAND SOUTH & 1993 & 69.46 & 53.07 & 90.10 & RW2 \\ 
  Zimbabwe & MATABELELAND SOUTH & 1994 & 75.08 & 56.22 & 98.85 & RW2 \\ 
  Zimbabwe & MATABELELAND SOUTH & 1995 & 81.53 & 61.48 & 108.98 & RW2 \\ 
  Zimbabwe & MATABELELAND SOUTH & 1996 & 86.86 & 65.65 & 115.23 & RW2 \\ 
  Zimbabwe & MATABELELAND SOUTH & 1997 & 91.04 & 68.86 & 120.23 & RW2 \\ 
  Zimbabwe & MATABELELAND SOUTH & 1998 & 93.51 & 69.85 & 124.53 & RW2 \\ 
  Zimbabwe & MATABELELAND SOUTH & 1999 & 94.04 & 69.37 & 126.88 & RW2 \\ 
  Zimbabwe & MATABELELAND SOUTH & 2000 & 92.52 & 67.45 & 124.74 & RW2 \\ 
  Zimbabwe & MATABELELAND SOUTH & 2001 & 90.51 & 65.97 & 122.17 & RW2 \\ 
  Zimbabwe & MATABELELAND SOUTH & 2002 & 88.05 & 64.21 & 119.35 & RW2 \\ 
  Zimbabwe & MATABELELAND SOUTH & 2003 & 85.13 & 60.96 & 117.08 & RW2 \\ 
  Zimbabwe & MATABELELAND SOUTH & 2004 & 82.30 & 57.90 & 115.35 & RW2 \\ 
  Zimbabwe & MATABELELAND SOUTH & 2005 & 79.50 & 55.21 & 112.72 & RW2 \\ 
  Zimbabwe & MATABELELAND SOUTH & 2006 & 76.78 & 54.14 & 107.75 & RW2 \\ 
  Zimbabwe & MATABELELAND SOUTH & 2007 & 74.10 & 52.72 & 103.22 & RW2 \\ 
  Zimbabwe & MATABELELAND SOUTH & 2008 & 71.55 & 49.29 & 102.65 & RW2 \\ 
  Zimbabwe & MATABELELAND SOUTH & 2009 & 69.03 & 42.86 & 109.66 & RW2 \\ 
  Zimbabwe & MATABELELAND SOUTH & 2010 & 66.43 & 34.18 & 127.12 & RW2 \\ 
  Zimbabwe & MATABELELAND SOUTH & 2011 & 64.18 & 26.39 & 150.71 & RW2 \\ 
  Zimbabwe & MATABELELAND SOUTH & 2012 & 61.72 & 19.89 & 182.04 & RW2 \\ 
  Zimbabwe & MATABELELAND SOUTH & 2013 & 59.50 & 14.48 & 225.90 & RW2 \\ 
  Zimbabwe & MATABELELAND SOUTH & 2014 & 57.27 & 10.30 & 273.48 & RW2 \\ 
  Zimbabwe & MATABELELAND SOUTH & 2015 & 55.43 & 7.06 & 341.17 & RW2 \\ 
  Zimbabwe & MATABELELAND SOUTH & 2016 & 53.19 & 4.67 & 418.04 & RW2 \\ 
  Zimbabwe & MATABELELAND SOUTH & 2017 & 51.42 & 3.08 & 520.39 & RW2 \\ 
  Zimbabwe & MATABELELAND SOUTH & 2018 & 49.64 & 2.00 & 601.00 & RW2 \\ 
  Zimbabwe & MATABELELAND SOUTH & 2019 & 48.07 & 1.33 & 679.60 & RW2 \\ 
  Zimbabwe & MIDLANDS & 1980 & 113.04 & 76.31 & 163.12 & RW2 \\ 
  Zimbabwe & MIDLANDS & 1981 & 105.84 & 79.68 & 139.18 & RW2 \\ 
  Zimbabwe & MIDLANDS & 1982 & 99.29 & 76.52 & 127.62 & RW2 \\ 
  Zimbabwe & MIDLANDS & 1983 & 93.17 & 70.78 & 122.84 & RW2 \\ 
  Zimbabwe & MIDLANDS & 1984 & 87.80 & 65.55 & 117.55 & RW2 \\ 
  Zimbabwe & MIDLANDS & 1985 & 83.05 & 62.68 & 108.50 & RW2 \\ 
  Zimbabwe & MIDLANDS & 1986 & 79.79 & 61.99 & 101.97 & RW2 \\ 
  Zimbabwe & MIDLANDS & 1987 & 77.86 & 61.28 & 98.13 & RW2 \\ 
  Zimbabwe & MIDLANDS & 1988 & 77.29 & 60.57 & 98.24 & RW2 \\ 
  Zimbabwe & MIDLANDS & 1989 & 78.07 & 60.04 & 100.78 & RW2 \\ 
  Zimbabwe & MIDLANDS & 1990 & 80.19 & 61.89 & 103.30 & RW2 \\ 
  Zimbabwe & MIDLANDS & 1991 & 83.82 & 65.99 & 106.16 & RW2 \\ 
  Zimbabwe & MIDLANDS & 1992 & 88.76 & 70.21 & 111.11 & RW2 \\ 
  Zimbabwe & MIDLANDS & 1993 & 94.94 & 74.73 & 119.33 & RW2 \\ 
  Zimbabwe & MIDLANDS & 1994 & 101.73 & 78.29 & 129.82 & RW2 \\ 
  Zimbabwe & MIDLANDS & 1995 & 109.70 & 84.95 & 142.33 & RW2 \\ 
  Zimbabwe & MIDLANDS & 1996 & 115.94 & 90.57 & 148.78 & RW2 \\ 
  Zimbabwe & MIDLANDS & 1997 & 120.66 & 94.29 & 153.83 & RW2 \\ 
  Zimbabwe & MIDLANDS & 1998 & 123.09 & 95.69 & 158.44 & RW2 \\ 
  Zimbabwe & MIDLANDS & 1999 & 123.23 & 94.70 & 160.33 & RW2 \\ 
  Zimbabwe & MIDLANDS & 2000 & 120.75 & 91.38 & 156.83 & RW2 \\ 
  Zimbabwe & MIDLANDS & 2001 & 117.69 & 89.70 & 152.00 & RW2 \\ 
  Zimbabwe & MIDLANDS & 2002 & 113.79 & 87.18 & 147.19 & RW2 \\ 
  Zimbabwe & MIDLANDS & 2003 & 109.74 & 82.44 & 144.13 & RW2 \\ 
  Zimbabwe & MIDLANDS & 2004 & 105.70 & 78.26 & 141.01 & RW2 \\ 
  Zimbabwe & MIDLANDS & 2005 & 101.70 & 74.41 & 137.00 & RW2 \\ 
  Zimbabwe & MIDLANDS & 2006 & 97.88 & 73.21 & 129.80 & RW2 \\ 
  Zimbabwe & MIDLANDS & 2007 & 94.12 & 71.73 & 122.81 & RW2 \\ 
  Zimbabwe & MIDLANDS & 2008 & 90.42 & 66.66 & 121.62 & RW2 \\ 
  Zimbabwe & MIDLANDS & 2009 & 86.92 & 57.12 & 130.88 & RW2 \\ 
  Zimbabwe & MIDLANDS & 2010 & 83.26 & 44.85 & 152.66 & RW2 \\ 
  Zimbabwe & MIDLANDS & 2011 & 79.99 & 34.33 & 178.07 & RW2 \\ 
  Zimbabwe & MIDLANDS & 2012 & 76.66 & 25.35 & 214.83 & RW2 \\ 
  Zimbabwe & MIDLANDS & 2013 & 73.78 & 18.23 & 260.70 & RW2 \\ 
  Zimbabwe & MIDLANDS & 2014 & 70.70 & 12.85 & 320.65 & RW2 \\ 
  Zimbabwe & MIDLANDS & 2015 & 67.60 & 8.76 & 388.42 & RW2 \\ 
  Zimbabwe & MIDLANDS & 2016 & 64.79 & 5.86 & 462.76 & RW2 \\ 
  Zimbabwe & MIDLANDS & 2017 & 61.92 & 3.92 & 547.15 & RW2 \\ 
  Zimbabwe & MIDLANDS & 2018 & 59.23 & 2.45 & 637.67 & RW2 \\ 
  Zimbabwe & MIDLANDS & 2019 & 57.03 & 1.55 & 720.86 & RW2 \\ 
  \hline
\caption{Complete results.} 
\label{fulltable}
\end{longtable}

}


\end{document}


